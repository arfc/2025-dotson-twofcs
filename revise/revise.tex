%        File: revise.tex
%     Created: Wed Oct 27 02:00 PM 2018 P
% Last Change: Wed Oct 27 02:00 PM 2018 P
%

%
% Copyright 2007, 2008, 2009 Elsevier Ltd
%
% This file is part of the 'Elsarticle Bundle'.
% ---------------------------------------------
%
% It may be distributed under the conditions of the LaTeX Project Public
% License, either version 1.2 of this license or (at your option) any
% later version.  The latest version of this license is in
%    http://www.latex-project.org/lppl.txt
% and version 1.2 or later is part of all distributions of LaTeX
% version 1999/12/01 or later.
%
% The list of all files belonging to the 'Elsarticle Bundle' is
% given in the file `manifest.txt'.
%

% Template article for Elsevier's document class `elsarticle'
% with numbered style bibliographic references
% SP 2008/03/01
%
%
%
% $Id: elsarticle-template-num.tex 4 2009-10-24 08:22:58Z rishi $
%
%
%\documentclass[preprint,12pt]{elsarticle}
\documentclass[answers,11pt]{exam}

% \documentclass[preprint,review,12pt]{elsarticle}

% Use the options 1p,twocolumn; 3p; 3p,twocolumn; 5p; or 5p,twocolumn
% for a journal layout:
% \documentclass[final,1p,times]{elsarticle}
% \documentclass[final,1p,times,twocolumn]{elsarticle}
% \documentclass[final,3p,times]{elsarticle}
% \documentclass[final,3p,times,twocolumn]{elsarticle}
% \documentclass[final,5p,times]{elsarticle}
% \documentclass[final,5p,times,twocolumn]{elsarticle}

% if you use PostScript figures in your article
% use the graphics package for simple commands
% \usepackage{graphics}
% or use the graphicx package for more complicated commands
\usepackage{graphicx}
% or use the epsfig package if you prefer to use the old commands
% \usepackage{epsfig}

% The amssymb package provides various useful mathematical symbols
\usepackage{amssymb}
% The amsthm package provides extended theorem environments
% \usepackage{amsthm}
\usepackage{amsmath}

% The lineno packages adds line numbers. Start line numbering with
% \begin{linenumbers}, end it with \end{linenumbers}. Or switch it on
% for the whole article with \linenumbers after \end{frontmatter}.
\usepackage{lineno}

% I like to be in control
\usepackage{placeins}

% natbib.sty is loaded by default. However, natbib options can be
% provided with \biboptions{...} command. Following options are
% valid:

%   round  -  round parentheses are used (default)
%   square -  square brackets are used   [option]
%   curly  -  curly braces are used      {option}
%   angle  -  angle brackets are used    <option>
%   semicolon  -  multiple citations separated by semi-colon
%   colon  - same as semicolon, an earlier confusion
%   comma  -  separated by comma
%   numbers-  selects numerical citations
%   super  -  numerical citations as superscripts
%   sort   -  sorts multiple citations according to order in ref. list
%   sort&compress   -  like sort, but also compresses numerical citations
%   compress - compresses without sorting
%
% \biboptions{comma,round}

% \biboptions{}


% Katy Huff addtions
\usepackage{xspace}
\usepackage{color}
\usepackage[hyphens]{url}

%\journal{Annals of Nuclear Energy}

\begin{document}

%\begin{frontmatter}

% Title, authors and addresses

% use the tnoteref command within \title for footnotes;
% use the tnotetext command for the associated footnote;
% use the fnref command within \author or \address for footnotes;
% use the fntext command for the associated footnote;
% use the corref command within \author for corresponding author footnotes;
% use the cortext command for the associated footnote;
% use the ead command for the email address,
% and the form \ead[url] for the home page:
%
% \title{Title\tnoteref{label1}}
% \tnotetext[label1]{}
% \author{Name\corref{cor1}\fnref{label2}}
% \ead{email address}
% \ead[url]{home page}
% \fntext[label2]{}
% \cortext[cor1]{}
% \address{Address\fnref{label3}}
% \fntext[label3]{}

\title{Synergistic Spent Nuclear Fuel Dynamics Within the European Union\\
        \large Response to Review Comments}
\author{Jin Whan Bae, Clifford E. Singer, Kathryn D. Huff}

% use optional labels to link authors explicitly to addresses:
% \author[label1,label2]{<author name>}
% \address[label1]{<address>}
% \address[label2]{<address>}


%\author[uiuc]{Kathryn Huff}
%        \ead{kdhuff@illinois.edu}
%  \address[uiuc]{Department of Nuclear, Plasma, and Radiological Engineering,
%        118 Talbot Laboratory, MC 234, Universicy of Illinois at
%        Urbana-Champaign, Urbana, IL 61801}
%
% \end{frontmatter}
\maketitle
\section*{Review General Response}
We would like to thank the reviewers for their detailed assessment of
this paper.


\section*{Reviewer 1}
\begin{questions}

        %---------------------------------------------------------------------
        \question Abstract. The abstract states that "These simulations 
        demonstrate that France can avoid deployment of additional LWRs by 
        accepting UNF from other EU nations, that lifetime extensions delay 
        time-to-transition, and improved breeding ratios are not particularly 
        impactful." These conclusions aren't highlighted or discussed properly 
        in the section of the report which presents the results and neither are 
        they highlighted in the conclusions.  

        \begin{solution}
                \textit{\begin{itemize}
                	\item Breeding ratio - Write a sentence on percent difference of LWR UNF reprocessed
                	\item Percent reduction of LWR UNF reprocessed:
                	\begin{itemize}
                		\item 1.11 br - 4.95\% reduction
                		\item 1.15 br - 7.80\% reduction
                		\item 1.18 br - 9.71\% reduction
                	\end{itemize}
                	\item Lifetime Extension - Write a sentence on percent difference of tot reprocessing (are under curve)
                \end{itemize}}
        \end{solution}

        %---------------------------------------------------------------------
        \question In figures 2 and 3 the colour scheme makes it difficult to 
        distinguish the different data.
        \begin{solution}
        	\textit{Not a fan of viridis? What should we do}
        \end{solution}

        %---------------------------------------------------------------------
        \question 1st sentence of Section 3 has been corrupted, with the ending 
        moved to the beginning.

        \begin{solution}
                Thank you for your input. This error is fixed.
        \end{solution}

        %---------------------------------------------------------------------
        \question Table 4 and 5 both give imply U-235 compositions for PWR UOX 
        fuel with a low U-235 content of just 3.1 w/o. This is inconsistent 
        with the mean discharge burnups in the scenario, which would demand an 
        initial enrichment of about 4.5 w/o (this also reflects current 
        practice). If this interpretation is correct, it will affect the 
        simulation and could impact the results. I suspect that the same may 
        also apply to the BWR UOX fuel. 
        \begin{solution}
                \textit{Should I increase it to 4.5\% enrichment and rerun?
                This should only increase the natural uranium used
                and tails inventory.}
        \end{solution}

        %---------------------------------------------------------------------
        \question Table 5 isn't referred to in the text.

        \begin{solution}
                There was a table with the same label in Latex. The error
                is fixed. Thank you.
        \end{solution}

        %---------------------------------------------------------------------
        \question Paragraph immediately after Table 6 the text should read nine 
        tons of tails not nine times of tails.
        \begin{solution}
                Thank you for your input. This error is fixed.
        \end{solution}

        %---------------------------------------------------------------------
        \question Fig 10 caption: the correct spelling is usage

        \begin{solution}
                Thank you for your input. This error is fixed.
        \end{solution}

        %---------------------------------------------------------------------
        \question Line 183 - corresponding is spelled incorrectly
        \begin{solution}
                Thank you for your input. This error is fixed.
        \end{solution}

        %---------------------------------------------------------------------
        \question Sections 6.1 and 6.2 should discuss the sensitivities to 
        breeding ratio and lifetime extension more completely and bring out the 
        conclusions noted in the abstract.

        \begin{solution}
                <++>
        \end{solution}

        %---------------------------------------------------------------------
        \question Line 239 should read: "The period during which ASTRID UNF.....
        \begin{solution}
                Thank you for your input. This error is fixed.
        \end{solution}

        %---------------------------------------------------------------------
        \question The conclusion should highlight the sensitivities to breeding 
        ratio and lifetime extension, as they are the main results from the 
        study.  

        \begin{solution}
                <++>
        \end{solution}


        \section*{Reviewer 2}


        %---------------------------------------------------------------------
        \question Overall, this paper does an excellent job of laying out its 
        modeling assumptions used to support their conclusions as to the 
        impacts of a spent fuel policy which would have France relying upon 
        international imports of UNF from EU member states to augment its 
        plutonium source material flows in order to "bootstrap" a fast reactor 
        transition policy. The use of data visualizations here is done quite 
        well, helping to visually guide the reader through the reasoning 
        process of the authors. In general, this is a thorough and 
        well-supported analysis which is worthwhile of publication.


        \begin{solution}
                Thank you for this thoughtful assessment.
        \end{solution}

        %---------------------------------------------------------------------
        \question First, in the introduction, it is stated, "However, the 
        current inventory of French Used Nuclear Fuel (UNF) is insufficient to 
        fuel that transition without building new Light Water Reactors (LWRs)." 
        Is this conclusively established in the literature? (i.e., this seems 
        like it may be covered in refs. 4 and/or 5, but a little bit more 
        detail into how this conclusion has been reached / how limiting the 
        present policy is would be useful).
        \begin{solution}
                \textit{Run a simulation with just France and show results?}
                
                \includegraphics{../images/french-transition/france_only_compare.png}
        \end{solution}

        %---------------------------------------------------------------------
        \question Table 3 indicates a minimum UNF cooling time of 36 months; 
        yet on page 12 (line 120) the authors state a minimum residency time of 
        72 months. Which one was used?

        \begin{solution}
                72 months of UOX UNF, and 36 months of SFR UNF
        \end{solution}

        %---------------------------------------------------------------------
        \question In section 4.1 (Material Depletion), I appreciate the fact 
        that the authors have attempted to convey a realistic depletion-based 
        recipe for their materials. However, the means by which they reached 
        these values are unclear. Which version of Origen was used for this 
        depletion analysis? What assumptions were made with respect to the fuel 
        depletion? (i.e., what reactor libraries did the authors use to 
        calculate depleted inventories?) This latter question is important 
        particularly for establishing the isotopic vector of the MOX / ASTRID 
        fuel; i.e., how was the flux spectrum shape treated to calculate these 
        depletion values? This does not require an exhaustive recounting of the 
        methods here, but a simple discussion of how the authors arrived at 
        these results would aid in the reproducibility of this work. Similarly, 
        I'm assuming a uniform irradiation power for each batch (reasonable for 
        this type of project), but it may be useful to clarify. 
        \begin{solution}
                \textit{Clearly describe how the VISION recipes are calculated}
        \end{solution}

        %---------------------------------------------------------------------
        \question For figures used to indicate sensitivities to modeling 
        assumptions (e.g., Figures 14, 15, 17, and 18). Here, I think 
        displaying these values as either a ratio to the baseline value or as a 
        net difference from the baseline would be more useful for conveying the 
        net effect; here, our interest in the effect at the margin, I would 
        assume, especially given that some of these differences are quite small 
        save for a short time period (e.g., the time region where the 
        decommissioning occurs). In as much, displaying these to represent the 
        differential effect (either through a net change or a ratio from the 
        baseline) would aid in highlighting the specific effect of these 
        assumptions which otherwise are somewhat obscured. If the authors feel 
        it necessary, a 2-axis design or a standalone articulation of these 
        throughputs for the baseline could easily support these figures if 
        showing the total throughput is required.)

        \begin{solution}
                <++>
        \end{solution}

        %---------------------------------------------------------------------
        \question Line 58: Megawatt is one word
        \begin{solution}
                Thank you for your input. This error is fixed.
        \end{solution}

        %---------------------------------------------------------------------
        \question Line 133: This section begins with a sentence fragment.

        \begin{solution}
                Thank you for your input. We changed the sentence
                "The scenario specifications defining the simulations
                presented in this work are listed in table 3."
                to
                "Table 3 lists the scenario specifications defining the
                simulations presented in this work."
        \end{solution}


        \section*{Reviewer 3}
        %---------------------------------------------------------------------
        \question Very nice approach of extracting realistic reactor data from 
        PRIS database and thorough literature review of other fuel cycle data
        \begin{solution}
                Thank you for this thoughtful assessment.
        \end{solution}


        %---------------------------------------------------------------------
        \question There's no explanation for why ASTRID reactors would last 80 
        years

        \begin{solution}
                <++>
        \end{solution}

        %---------------------------------------------------------------------
        \question Varying the lifetime extension via random sampling is very 
        innovative
        \begin{solution}
                Thank you !
        \end{solution}

        %---------------------------------------------------------------------
        \question How was the 36 month minimum cooling time decided? LWR UNF 
        typically requires 5 years before decay heat sufficiently low for 
        transport and for PUREX process

        \begin{solution}
                72 months (6 years) for LWR UNF, and 36 months (3 years) for ASTRID UNF
        \end{solution}

        %---------------------------------------------------------------------
        \question page 16, line 17, duplicate "recipe"
        \begin{solution}
                Thank you for your input. This error is fixed.
        \end{solution}

        %---------------------------------------------------------------------
        \question Are the first SFR cores loaded entirely with fresh 
        equilibrium fuel? At 22\% Pu?  If so, this will overpredict first core 
        fissile requirements. If not, then there should be emphasis on how the 
        initial core has different fuel zones with various Pu contents 
        similating burned fuel (at 13\%, 16\%, 19\%, 22\%, for example)

        \begin{solution}
                <++>
        \end{solution}


        %---------------------------------------------------------------------
        \question The estimated 200 SFRs that can be started up with the 985 
        tPu does not take into account the reload fuel required to sustain the 
        first SFRs until they can recycle their own fuel (which may take a few 
        years). Perhaps this is why Cyclus calculated 180 SFRs started up 
        instead of 200. The startup reload fuel requirements should also be 
        mentioned in addition to the startup first core requirements.
        \begin{solution}
                <++>
        \end{solution}

        %---------------------------------------------------------------------
        \question Several typos spotted. Run spell check and proof read.

        \begin{solution}
                <++>
        \end{solution}

        %---------------------------------------------------------------------
        \question Did you just artificially change the breeding ratio of the 
        SFR without design considerations?  i.e., changes in core size, 
        spectrum, burnup, Pu content, etc. If so, these results may not be 
        valid.  If not, please include descriptions of the SFR core 
        modifications made to enable the different breeding ratios. If the 
        excess reactivity decreases from BOC to EOC, then an increase in the 
        breeding ratio may result in larger initial core Pu mass requirements.
        \begin{solution}
                <++>
        \end{solution}

        %---------------------------------------------------------------------
        \question Can you show a figure of surplus separated Pu or surplus LWR 
        UNF for the scenario to show how much margin there is? It is stated 
        that there is enough Pu from the LWR UNF to start the SFR fleet, but by 
        how much?

        \begin{solution}
                <++>
        \end{solution}



        %---------------------------------------------------------------------
\end{questions}
%\bibliographystyle{unsrt}
%\bibliography{revise}
\end{document}

  %
  % End of file `elsarticle-template-num.tex'.
