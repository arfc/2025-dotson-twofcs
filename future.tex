\section{Future Nuclear Projections}

In summary, the current, and the future of nuclear energy in Europe is organized
in the table by the World Nuclear Association \cite{world_nuclear_association_nuclear_2017}:
 
\begin{table}[h]
	\centering
	\caption {Power Reactors under construction and planned \cite{world_nuclear_association_nuclear_2017}}
	\scalebox{0.70}{
	\begin{tabular}{|c|c|c|c|c|}
		\hline
		Exp. Operational & Country & Reactor & Type & Gross MWe\\
		\hline
		2018 & Slovakia  & Mochovce 3 & PWR & 440\\
		2018 & Slovakia & Mochovce 4 & PWR & 440 \\
		2018 & France & Flamanville 3 & PWR & 1600 \\
		2018 & Finland & Olkilouto 3 & PWR & 1720 \\
		2019 & Romania & Cernavoda 3 & PHWR & 720 \\
		2020 & Romania & Cernavoda 4 & PHWR & 720 \\
		2024 & Finland & Hanhikivi & VVER1200 & 1200 \\
		2024 & Hungary & Paks 5 & VVER1200 & 1200 \\
		2025 & Hungary & Paks 6 & VVER1200 & 1200 \\
		2025 & Bulgaria & Kozloduy 7 & AP1000? & 950 \\
		2026 & UK & Hinkley Point C1 & EPR & 1670 \\
		2027 & UK & Hinkley Point C2 & EPR & 1670 \\
		2029 & Poland & Choczewo? & N/A & 3000 \\
		2035 & Poland & East? & N/A & 3000 \\
		2035 & Czech Rep & Dukovany 5 & ? & 1200 \\
		2035 & Czech Rep & Temelin 3 & AP1000? & 1200 \\
		2040 & Czech Rep & Temelin 4 & AP1000? & 1200 \\
		\hline
	\end{tabular}
	}
\end{table}

The next table lists different nations in the EU and their nuclear program plans.
The attitude is divided into five, from very negative to very positive, where 
very negative means a complete shutdown of all nuclear facilities and 
very positive means a rigorous expansion of nuclear power in the country.

\begin{table}[h]
	\centering
	\caption {Future Nuclear Programs of EU Nations \cite{world_nuclear_association_nuclear_2017}}
	\scalebox{0.70}{
		\begin{tabular}{|c|c|c|}
			\hline
			Nation & Attitude & Specific Plan \\
			\hline
			France & neutral &Shutdown nuclear plants if they reach end of lifetime. No new construction.\\
			Germany & negative & Close all nuclear reactors by 2022.\\
			Czech Republic & positive & Additional 2,400 MWe (AP1000s) by 2035.\\
			UK & very positive & 13 units (17,900 MWe) by 2030.\\
			Belgium & very negative & All shutdown 2025.\\
			Sweden & very negative & All shutdown 2050.\\
			Finland & positive & Additional EPR in 2018, VVER in 2024.\\
			Bulgaria & positive & Additional AP1000 (1,000 MWe) construction in 2035. \\
			Poland & very positive & Additional 6,000 MWe by 2035.\\
			Romania & neutral & Additional 1,440 MWe by 2020. \\
			Hungary & positive & Additional 2,400 MWe (VVER-1200) by 2025. \\ 
			Spain & neutral & No plans to expand or early shutdown. \\
			Italy & neutral & No plans to expand or early shutdown. \\
			\hline
		\end{tabular}
	}
\end{table}


Though nuclear reactors are notorious for not being built on time,
for this paper it is assumed that all the reactors listed above are built
on time, and operating on the January of their years. Also,all the newly
built reactors are assumed to have a lifetime of 60 years.
