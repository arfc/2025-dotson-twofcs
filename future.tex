\section{Future Nuclear Projections}
This section lists the future projections and plans of
different EU countries. Only the major contributors have been listed.

\subsection{France}
The Energy Transition for Green Growth Bill passed by the National Assembly for 50\% of
nuclear contribution for electricity by 2025, from 63.2GWe. The 2015 amendment tried to remove
the cap, but this was not accepted by the lower house. Currently there is a
1600MWe reactor unit under construction at Flamanville, for operation from 2018.
However, the general atmosphere is to close nuclear power plants when they reach the end
of their lifetime, while not building new ones. This is due to the decreasing
public support for nuclear, along with the \gls{EDF}'s troubling financial situation \cite{noauthor_nuclear_2017}.

\subsection{Germany}
Germany, despite its past success with nuclear energy back in the 1970s,
has a dwindling nuclear program. The two incidents, Chernobyl and Fukushima,
caused the public and the government to stall and reduce its nuclear fleet.
In March 2011, the government declared a moratorium,which planned to immediately
shut down nuclear power plants that began prior to 1980. Furthermore, the government
passed a phase-out plan to close all reactors by 2022 \cite{noauthor_nuclear_2017-1}.

\subsection{Czech Republic}
The Czech Republic has six nuclear reactors ($3,904$ MWe), and the government is committed to the
future of nuclear energy, which is reflected in its 2015 energy policy to build 
more nuclear power plants. The country plans to build two more AP1000s ($2,400$ MWe) by 2035.
The public support of nuclear energy is also very strong \cite{noauthor_nuclear_2017-2}.

\subsection{United Kingdom}
United Kingdom not only has a substantial amount of nuclear power (total of 15 units
and $8,883$MWe), but also has fuel cycle facilities like a reprocessing plant. The government
claimed that nearly 25GWe of nuclear should be generated by 2025, but there hasn't been
a fixed amount. After the government became more favorable towards nuclear in 2006, utilities
began planning for new plants. Also, many different foreign utilities, from France (EDF),
Russia (Rosatom), and China (China General Nuclear Group). This led up to a total proposed
number of 13 units ($17,900$ MWe) by 2030, with a mixture of EPR, ABWR, and AP1000s \cite{noauthor_nuclear_2017-3}.

\subsection{Belgium}
Belgium currently has seven nuclear reactors ($5,943$ MWe), but with little government support
for nuclear energy, all of them are expected to shutdown in 2025. The government
said that its stance on nuclear phase-out "is now final". There are no known plans
to build new reactors \cite{noauthor_nuclear_2017-4}.

\subsection{Sweden}
Sweden has nine reactors that generate 40\% of the country's electricity. 
Sweden had four reactors that would close by 2020 due to decreasing profits,
with the nuclear tax of .75 Euro cents/kWh, which is one-third of the operating
cost. The country plans to shut down all reactors by 2050 \cite{noauthor_nuclear_2017-5}

\subsection{Finland}
Finland currently has four reactors (2700 MWe), two BWR and two VVER,
with a fifth and sixth one (EPR of 1600 MWe, VVER of 1200 MWe, respectively) planned.
The EPR is to start operation in 2018 and the VVER in 2024.
Finland, the country closest to an operating permanent geological repository, has 
a very optimistic view of nuclear energy \cite{noauthor_nuclear_2017-6}.

\subsection{Bulgaria}
Bulgaria currently has two nuclear reactors that provide 1926 MWe capacity. The
government has been historically in favor of nuclear power, and uses Russian
technology for its reactors, namely the VVER. The government attempted to build
new plants in 1980, but it was aborted in 1991 due to lack of funds. Yet,
the government planned to add another unit to the currently existing site.
There are bids in progress, and the most probable type and vendor seems to be
Westinghouse's AP1000 with a capacity of 950 MWe \cite{noauthor_nuclear_2016}.


\subsection{Poland}
Poland has traditionally been using coal for most of its electricity generation,
but due to recent concerns of CO2 emissions and air quality, its government decided
to invest on nuclear, and plans to construct 6,000 MWe capacity up till 2035. The government
is yet to decide on which type from which vendor, but estimates the operation date to be
2029 for the first 3,000 MWe, and 2035 for the other 3,000 MWe. \cite{noauthor_nuclear_2017-7}

\subsection{Summary}
With all the other countries only taking up a minute portion of the entire
EU nuclear fleet, it is assumed that they will not have additional constructions,
but will cease nuclear generation when their current operating reactors shut down.

In summary, the current, and the future of nuclear energy in Europe is organized
in the table by the World Nuclear Association \cite{world_nuclear_association_nuclear_2017}:

\begin{table}[h]
	\centering
	\caption {Power Reactors under construction and planned \cite{world_nuclear_association_nuclear_2017}}
	\scalebox{0.70}{
	\begin{tabular}{|c|c|c|c|c|}
		\hline
		Exp. Operational & Country & Reactor & Type & Gross MWe\\
		\hline
		2018 & Slovakia  & Mochovce 3 & PWR & 440\\
		2018 & Slovakia & Mochovce 4 & PWR & 440 \\
		2018 & France & Flamanville 3 & PWR & 1600 \\
		2018 & Finland & Olkilouto 3 & PWR & 1720 \\
		2019 & Romania & Cernavoda 3 & PHWR & 720 \\
		2020 & Romania & Cernavoda 4 & PHWR & 720 \\
		2024 & Finland & Hanhikivi & VVER1200 & 1200 \\
		2024 & Hungary & Paks 5 & VVER1200 & 1200 \\
		2025 & Hungary & Paks 6 & VVER1200 & 1200 \\
		2025 & Bulgaria & Kozloduy 7 & AP1000? & 950 \\
		2026 & UK & Hinkley Point C1 & EPR & 1670 \\
		2027 & UK & Hinkley Point C2 & EPR & 1670 \\
		2029 & Poland & Choczewo? & N/A & 3000 \\
		2035 & Poland & East? & N/A & 3000 \\
		2035 & Czech Rep & Dukovany 5 & ? & 1200 \\
		2035 & Czech Rep & Temelin 3 & AP1000? & 1200 \\
		2040 & Czech Rep & Temelin 4 & AP1000? & 1200 \\
		\hline
	\end{tabular}
	}
\end{table}

Though nuclear reactors are notorious for not being built on time,
for this paper it is assumed that all the reactors listed above are built
on time, and operating on the January of their years. Also,all the newly
built reactors are assumed to have a lifetime of 60 years.
