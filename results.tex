\section{Results}

\subsection{Historical Operation of \gls{EU} Reactors}


\begin{table}[h]
	\centering
	\scalebox{0.86}{
		\begin{tabular}{|c|c|c|c|}
			\hline
			Category & Unit & Value & Specifics\\ \hline
			Total UOX Usage & MTHM & 178,865 &  \\ \hline
			Total MOX Usage & MTHM & 8,909 & \\ \hline
			Total Used UOX Stored & MTHM & 157,472 & \gls{UNF} that is not reprocessed\\ \hline
			Total Used  MOX Stored & MTHM & 679 & \gls{UNF} that is not reprocessed \\ \hline
			Total Tailings & MTHM & 1,063,909 & \\ \hline
			Total Natural U Used & MTHM & 1,251,658 & \\ \hline
		\end{tabular}}
		\caption{Simulation Results for Historical Nuclear Operation of \gls{EU} Nations}
		\label{tab:sim_result}
		\end {table}

\Cref{tab:sim_result} lists the important metrics
obtained from the first simulation. The following
values are the \gls{EU} inventory and history at year 2050.

Figures \ref{fig:eu_tail} and \ref{fig:eu_snf} show the 
timeseries of mass of tailings and used fuel accumulation in \gls{EU}.
Figure \ref{fig:eu_fuel} shows the amount of fuel used in \gls{EU}.


\begin{figure}[htbp!]
	\begin{center}
		\includegraphics[scale=0.7]{./images/eu_future/tailings.png}
	\end{center}
	\caption{Timeseries of Tailings Mass in the \gls{EU}.}
	\label{fig:eu_tail}
\end{figure}

\begin{figure}[htbp!]
	\begin{center}
		\includegraphics[scale=0.7]{./images/eu_future/total_fuel.png}
	\end{center}
	\caption{Timeseries of Total Fuel Usage in \gls{EU}.}
	\label{fig:eu_fuel}
\end{figure}


\begin{figure}[htbp!]
	\begin{center}
			\includegraphics[scale=0.7]{./images/eu_future/snf.png}
	\end{center}
	\caption{Timeseries of Used Nuclear Fuel in \gls{EU}.}
	\label{fig:eu_snf}
\end{figure}
\FloatBarrier


\begin{table}[h]
	\centering
	\begin{tabular}{|c|c|c|}
		\hline
		Isotope & Mass Fraction in Used Fuel [\%] & Quantity [t] \\ \hline
		Total & 0.9358 & 1,473 \\ \hline
		Pu238 & 0.0111 & 17.47 \\ \hline
		Pu239 & 0.518 & 815.7 \\ \hline
		Pu240 & 0.232 & 365.33 \\ \hline
		Pu241 & 0.126 & 198.41 \\ \hline
		Pu242 & 0.0487 & 76.68 \\ \hline
	\end{tabular}
	\caption{Plutonium From Used Fuel}
	\label{tab:pu}
\end{table}


\gls{MOX} for an ASTRID reactor facility, is 11\% Pu and 89\% depleted uranium.
Thus $1,473$ tons of plutonium yields $13,390$ tons of
\gls{MOX}. \Cref{tab:pu} lists the isotope, mass fraction,
and quantity of plutonium that can be obtained from the 2050 \gls{UNF} inventory.


\subsection{French \gls{SFR} Transition Scenario}

From Varaine et al. \cite{marsaultmarie-sophie_pre-conceptual_2012}, a French
ASTRID-type \gls{SFR} of capacity 600 MWe needs $1.225$ tons of
plutonium a year, with an initial plutonium loading of $4.9$ tons. 
Thus, the number of \glspl{SFR} that can be loaded with the reprocessed
plutonium from \gls{UNF} can be estimated to $\frac{1,473}{4.9} \approx 300$ \glspl{SFR},
assuming infinite reprocessing and fabrication capacity as well as
abundant depleted uranium supply. 

Also, assuming that \gls{MOX} can be recycled indefinitely,
used \gls{MOX} from an ASTRID reactor
contains enough plutonium to produce a \gls{MOX} fuel with
the same mass, if mixed with depleted uranium. For example,
used \gls{MOX} from an ASTRID reactor is assumed to be 12.6\% plutonium
in this simulation (see \cref{tab:comp}), whereas a fresh \gls{MOX} is 11\% plutonium.
Separating plutonium from used \gls{MOX} from
an ASTRID reactor can create \gls{MOX} of the mass of used \gls{MOX}.
The plutonium breeding ratio in this simulation is thus assumed to be
$\approx 1.145$.

The second scenario, with the tailings and used \gls{UOX}
inventory, evaluates if the French can transition into \gls{SFR}
without constructing additional \gls{LWR}s. This simulation
assumed infinite reprocessing and fabrication capacity.

\Cref{fig:fuel} shows the timeseries mass of \gls{MOX} used in the 
\gls{SFR}s separated by their origin.
Note that the plot shows \gls{MOX}
accumulation prior to \gls{SFR} deployment from 2020.

\begin{figure}[htbp!]
	\begin{center}
		\includegraphics[scale=0.7]{./images/french-transition/where_fuel.png}
	\end{center}
	\caption{Timeseries of fuel used in the \gls{SFR}s [tons]}
	\label{fig:fuel}
\end{figure}

\Cref{fig:reprocess_waste} shows the amount of reprocessing waste
(minor actinides, fission products) over time. Note that 
reprocessing waste from \gls{UOX} reprocessing is substantially
greater than waste from \gls{MOX} reprocessing due to its lower
plutonium and uranium content.

\Cref{fig:pu_isotopics} shows the isotopics of the plutonium that are
reprocessed from the used fuel inventory.

\begin{figure}[htbp!]
	\begin{center}
		\includegraphics[scale=0.7]{./images/french-transition/rep_pu.png}
	\end{center}
	\caption{Plutonium timeseries separated by isotope}
	\label{fig:pu_isotopics}
\end{figure}

\begin{figure}[htbp!]
	\begin{center}
		\includegraphics[scale=0.7]{./images/french-transition/reprocess_waste.png}
	\end{center}
	\caption{Reprocessing Waste for French Transition Scenario.}
	\label{fig:reprocess_waste}
\end{figure}


\begin{table}[h]
	\centering
	\scalebox{0.86}{
		\begin{tabular}{|c|c|c|}
			\hline
			Category & Unit & Value  \\ \hline
			Total MOX used & MTHM & 116,115  \\ \hline
			Total \glspl{SFR} Deployed & & 200 \\ \hline
			Total Plutonium Reprocessed & MTHM & 14,414 \\ \hline
			Total MOX from UOX Waste & MTHM & 9,729  \\ \hline
			Total MOX from MOX Waste & MTHM  & 150,426  \\ \hline
			Total Tailings used & MTHM & 105,664 \\ \hline
			Total legacy UNF reprocessed & MTHM & 97,298 \\ \hline
			Total Reprocessed Uranium Stockpile & MTHM & 251,100 \\ \hline
			Total Reprocess Waste & MTHM & 14,414 \\ \hline
		\end{tabular}}
		\caption {\gls{SFR} Simulation Results}
		\label{tab:sfr_sim_result}
\end {table}
