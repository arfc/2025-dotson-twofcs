\section{Conclusion}
This work demonstrated that France can transition into
a fully \gls{SFR} fleet with installed capacity of 66,000 MWe by 2076,
if France receives \gls{UNF} from other \gls{EU} nations.
Supporting the \gls{SFR} fleet would require a reprocessing capacity of 250 MTHM per month,
and a fabrication capacity of 300 MTHM per month.

Since most \gls{EU} nations do not have an operating \gls{UNF}
repository or a management plan, they have a strong incentive
to send all their \gls{UNF} to France. The nations
with aggressive nuclear reduction will be able phase out nuclear
without constructing a permanent repository. France has an
incentive to take this fuel, since reuse of used fuel from
other nations will allow France to meet their MOX demand
without new construction of \glspl{LWR}.

\begin{table}[h]
    \centering
                \begin{tabularx}{\textwidth}{lbb}
                    \hline 
                    
                    \textbf{Nation} & \textbf{Growth Trajectory} & \small{\textbf{UNF in 2050 [MTHM] }}\\
                    \hline
                    UK & Aggressive Growth & 15,412\\
                    \hline
                    Finland & Modest Growth & 6,159\\
                    \hline
                    \textbf{Bulgaria} & \textbf{Modest Growth} & \textbf{3,736}\\
                    \hline
                    \textbf{Czech Rep.} & \textbf{Modest Growth} & \textbf{5,116}\\
                    \hline
                    \textbf{France} & \textbf{Modest Reduction} & \textbf{37,916}\\
                    \hline
                    \textbf{Spain} & \textbf{Modest Reduction} &  \textbf{10,998}\\
                    \hline
                    \textbf{Italy} & \textbf{Modest Reduction} & \textbf{612}\\
                    \hline
                    \textbf{Belgium} & \textbf{Aggressive Reduction} & \textbf{7,797}\\
                    \hline
                    Sweden & Aggressive Reduction & 16,579\\
                    \hline
                    \textbf{Germany} & \textbf{Aggressive Reduction} & \textbf{26,466}\\
                    \hline
                    
                \end{tabularx}
    \caption {Growth Trajectory and UNF Inventory of \gls{EU} Nations}
    \label{tab:which_count}
\end{table}

Though complex political and economic factors are overlooked,
 and various assumptions present for this scenario,
this option may hold value for the \gls{EU} as a nuclear community,
and for France to advance into a closed fuel cycle.
