\section{Conclusion}

This work demonstrates that France can transition into
a fully \gls{SFR} fleet with installed capacity of 66,000 \gls{MWe} without
building additional \glspl{LWR}
if France receives \gls{UNF} from other \gls{EU} nations.
Supporting the \gls{SFR} fleet requires an average 
reprocessing capacity of 73.27 \gls{MTHM} per month,
and an average fabrication capacity of 45.29 \gls{MTHM} per month.


Since most \gls{EU} nations do not have an operating \gls{UNF}
repository or a management plan, they have a strong incentive
to send their \gls{UNF} to France. In particular, the nations
planning aggressive nuclear reduction will be able phase out nuclear
without constructing a permanent repository. France has an
incentive to take this fuel, since recycling used fuel from
other nations will allow France to meet their MOX demand
without new construction of \glspl{LWR}.

\Cref{tab:which_send} lists \gls{EU} nations and their \gls{UNF} inventory
in 2050. We propose a strategy in which 
the nations reducing their nuclear fleet send their \gls{UNF} to France.
The sum of \gls{UNF} from Italy, Slovenia, Belgium, Spain and Germany
provides enough \gls{UNF} for the simulated transition ($\approx 54,000$ MTHM). Sweden is not considered because of its concrete waste management plan.


\begin{table}[h]
    \centering
    \caption {\gls{EU} nations and their respective \gls{UNF} inventory. The bolded countries'
              \gls{UNF} inventory adds up to the required \gls{UNF} amount for French \gls{SFR} transition. }
                \begin{tabularx}{\textwidth}{llr}
                    \hline 
                    \textbf{Nation} & \textbf{Growth Trajectory} & \small{\textbf{UNF in 2050 [MTHM] }}\\
                    \hline
                    Poland & Aggressive Growth & 1,807\\
                    Hungary & Aggressive Growth & 3,119 \\ 
                    UK & Aggressive Growth & 13,268\\
                    Slovakia & Modest Growth & 2,746\\
                    Bulgaria & Modest Growth & 3,237 \\
                    Czech Rep. & Modest Growth & 4,413\\
                    Finland & Modest Growth &  5,713\\
                    Netherlands & Modest Reduction & 539\\
                    \textbf{Italy} & \textbf{Modest Reduction} & \textbf{583}\\
                    \textbf{Slovenia} & \textbf{Modest Reduction} & \textbf{765}\\
                    \textbf{Lithuania} & \textbf{Modest Reduction} & \textbf{2,644} \\
                    \textbf{Belgium} & \textbf{Aggressive Reduction} & \textbf{6,644}\\
                    \textbf{Spain} & \textbf{Modest Reduction} &  \textbf{9,771} \\
                    \textbf{France} & \textbf{Modest Reduction} & \textbf{9,979} \\
                    Sweden & Aggressive Reduction & 16,035\\
                    \textbf{Germany} & \textbf{Aggressive Reduction} & \textbf{23,868}\\
                    \hline
                \end{tabularx}
    
    \label{tab:which_send}

\end{table}

In these simulations, some complex political and economic factors were not 
incorporated and various assumptions were present in this scenario (For
example, Germany's current policy is to not reprocess its \gls{LWR} fuel
\cite{topfer_germany_2011}, which creates a shortage in the supply of \gls{LWR}
\gls{UNF} for \gls{ASTRID} \gls{MOX} production).
However,
the collaborative option explored here may hold value for the \gls{EU} nuclear community,
and may enable France to advance more rapidly into a closed fuel cycle.

\FloatBarrier
