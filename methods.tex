\section{Methods}
This work is based off of  two \Cyclus simulations. 
The first simulation calculated
the mass and composition of used fuel and tails accumulated by \gls{EU} nations from 1970 to 2050.
All EU nations with the exception of France adopt a once-through fuel cycle.
France can reprocess used \gls{UOX} and \gls{MOX} to
produce \gls{MOX} from reprocessed plutonium and depleted uranium (tails).
\Cref{diag:fc} shows the material flow.

After obtaining the \gls{UNF} inventory of all the \gls{EU} nations in 2050, the second
simulation runs. It it, the \gls{UNF} inventory is reprocessed and fabricated
as fuel for the newly deployed \gls{SFR} reactors.
\glspl{SFR} are modeled after the ASTRID breeder reactor \cite{varaine_pre-conceptual_2012}.
The ASTRID-type \glspl{SFR} make up for the decommissioned capacity
of \glspl{LWR} in France, to maintain a constant installed capacity of $66,000$ MWe up to 2160.
Eventually, the  \gls{MOX} created from recycled \gls{MOX}
fuels the entire fleet of 110 \glspl{SFR}.

All scripts and data used
in this paper are available in \cite{bae_arfc/transition-scenarios:_2017}.


\subsection{\Cyclus}
\Cyclus is an agent-based fuel cycle simulation framework, meaning that
each reactor, reprocessing plant, and fuel fabrication plant is modeled as an agent.
At each timestep (one month),
agents put out their bids for materials (supply and/or demand) and exchange
with one another. A market-like mechanism called the dynamic resource exchange
\cite{gidden_agent-based_2015} governs the exchanges.
Each material item has a quantity, composition, name, and a unique identifier
for output analysis.
A \Cyclus input file contains prototypes, which are fuel cycle facilities with
pre-defined parameters, that are deployed in the simulation as \texttt{facility} agents.
Encapsulating the \texttt{facility} agents are the \texttt{Institution} and \texttt{Region}.
A \texttt{Region} agent holds a set of \texttt{Institution}s.
An \texttt{Institution} agent can deploy or decommission \texttt{facility} agents.
The \texttt{Institution} agent is part of a \texttt{Region} agent,
which can contain multiple \texttt{Institution} agents.


For example, `France' would be a \texttt{Region} agent,
that may contain two \texttt{Institution} agents \glspl{LWR}
and \glspl{SFR}. The \texttt{Institution} agents would then deploy
\glspl{LWR} and \glspl{SFR} agents, respectively, according to a pre-defined deployment
scheme.

\subsection{Simulated Nuclear Deployment in the European Union}

The \gls{IAEA} \gls{PRIS} database \cite{iaea_nuclear_2017} contains reactor
operation history.
The database is imported as a csv file, to populate the simulation
with deployment information, listing the country, reactor unit, type, net capacity (MWe), status,
operator, construction date, first criticality date, first grid date, commercial date, shutdown
date (if applicable), and unit capacity factor for 2013. Then only the \gls{EU} countries are extracted
from the csv file. We developed a python script to generate a \Cyclus input file from the csv file,
which lists the individual reactor units as agents. 

\begin{figure}
        \centering

\begin{tikzpicture}[node distance=1.5cm]
\node (database) [object] {Database (\texttt{.csv})};
\node (script) [process, below of=database] {Input Generation Script (\texttt{write\_deployinst\_input.py})};
\node (input) [object, below of=script] {\Cyclus Input File (\texttt{.xml})};
\node (cyclus) [process, below of=input]{\Cyclus};
\node (output) [object, below of=cyclus]{\texttt{Output File (\texttt{.Sqlite})}};
\node (script2) [process, below of=output]{Analysis Script (\texttt{analysis.py})};

\draw [arrow] (database) -- (script); 
\draw [arrow] (script) -- (input);
\draw [arrow] (input) -- (cyclus);
\draw [arrow] (cyclus) -- (output);
\draw [arrow] (output) -- (script2);
\end{tikzpicture}

\caption{Computational Workflow for this work. The green circles represent files, and the green
         boxes represent codes that process the files.}
\label{diag:comp}
\end{figure}


Projections of future reactor deployment in this simulation are based on
assessment of analyses from references such as \gls{PRIS} for reactors planned
for construction \cite{iaea_nuclear_2017}, the World Nuclear Association
\cite{world_nuclear_association_nuclear_2017}, and works on the future of
nuclear power in a global \cite{joskow_future_2012} and European context
\cite{hatch_politics_2015}.  The projections extend to 2050 at the latest. This
allows the simulation to take place from 1970 to 2050. Later sections explain,
in detail, the specific plans for each \gls{EU} nation.

\Cref{tab:eu_deployment} lists the reactors that are currently  planned or
under construction.  All  planned constructions are completed without delay or
failure and are assumed to reach a lifetime of 60 years.  


\begin{table}[h]
    \centering

    \label{tab:eu_deployment}
    \scalebox{0.9}{
    \begin{tabular}{ccccc}
        \hline
        \textbf{Exp. Operational }&\textbf{ Country} &\textbf{ Reactor} & \textbf{Type} & \textbf{Gross MWe}\\
        \hline
        2018 & Slovakia  & Mochovce 3 & PWR & 440\\
        2018 & Slovakia & Mochovce 4 & PWR & 440 \\
        2018 & France & Flamanville 3 & PWR & 1600 \\
        2018 & Finland & Olkilouto 3 & PWR & 1720 \\
        2019 & Romania & Cernavoda 3 & PHWR & 720 \\
        2020 & Romania & Cernavoda 4 & PHWR & 720 \\
        2024 & Finland & Hanhikivi & VVER1200 & 1200 \\
        2024 & Hungary & Paks 5 & VVER1200 & 1200 \\
        2025 & Hungary & Paks 6 & VVER1200 & 1200 \\
        2025 & Bulgaria & Kozloduy 7 & AP1000? & 950 \\
        2026 & UK & Hinkley Point C1 & EPR & 1670 \\
        2027 & UK & Hinkley Point C2 & EPR & 1670 \\
        2029 & Poland & Choczewo & N/A & 3000 \\
        2035 & Poland & N/A & N/A & 3000 \\
        2035 & Czech Rep & Dukovany 5 & N/A & 1200 \\
        2035 & Czech Rep & Temelin 3 & AP1000 & 1200 \\
        2040 & Czech Rep & Temelin 4 & AP1000 & 1200 \\
        \hline
    \end{tabular}
    }
    \caption {Power Reactors under construction and planned.
        Replicated from \cite{world_nuclear_association_nuclear_2017}.}
\end{table}
\FloatBarrier

For each \gls{EU} nation, the growth trajectory is categorized from
``Aggressive Growth'' to ``Aggressive Shutdown''. Aggressive growth is
characterized by a rigorous expansion of nuclear power while
Aggressive Shutdown is characterized as a transition to rapidly
de-nuclearize the nation's electric grid. A nation's growth trajectory is
categorized into five degrees depending on G, the growth trajectory metric.

 \[
 G = \left\{\begin{array}{ll}
 \text{Aggressive Growth}, & \text{for } G \geq 2\\
 \text{Modest Growth}, & \text{for } 1.2 \leq G < 2\\
 \text{Maintanence}, & \text{for } 0.8 \leq G < 1.2 \\
 \text{Modest Reduction}, & \text{for } 0.5 \leq G< 0.8\\
 \text{Aggressive Reduction}, & \text{for } G \leq 0.5
 \end{array}\right\} = \frac{C_{2040}}{C_{2017}}\\\\
 \]
 \[
  G = \text{Growth Trajectory  } [-]
 \]
 \[
 C_{i} = \text{Nuclear Capacity in Year i  } [MWe]
 \]

The growth trajectory and specific plan of each nation in the \gls{EU}
is listed in Table \ref{tab:eu_growth}.

\begin{table}[h]
    \centering
        \begin{tabularx}{\textwidth}{lmb}
            \hline

            \textbf{Nation} & \textbf{Growth Trajectory} & \textbf{Specific Plan }\\
            \hline
            UK & Aggressive Growth & {\small  13 units (17,900 MWe) by 2030.}\\
            \hline
            Poland & Aggressive Growth &  {\small Additional 6,000 MWe by 2035.}\\
            \hline
            Hungary & Aggressive Growth &  {\small Additional 2,400 MWe by 2025.} \\
            \hline
            Finland & Modest Growth &  {\small Additional 2,920 MWe by 2024.}\\
            \hline
            Bulgaria & Modest Growth &  {\small Additional 1,000 MWe by 2035.} \\
            \hline
            Romania & Modest Growth &  {\small Additional 1,440 MWe by 2020.} \\
            \hline
            Czech Rep. & Modest Growth & {\small  Additional 2,400 MWe by 2035.}\\
            \hline
            France & Modest Reduction & {\small No expansion or early shutdown.}\\
            \hline
            Spain & Modest Reduction &  {\small No expansion or early shutdown.} \\
            \hline
            Italy & Modest Reduction & {\small No expansion or early shutdown. }\\
            \hline
            Belgium & Aggressive Reduction & All shut down 2025.\\
            \hline
            Sweden & Aggressive Reduction & All shut down 2050.\\
            \hline
            Germany & Aggressive Reduction & All shut down by 2022.\\
            \hline

        \end{tabularx}

    \caption {Future Nuclear Programs of \gls{EU} Nations \cite{world_nuclear_association_nuclear_2017}}
  \label{tab:eu_growth}
\end{table}


Figure \ref{fig:eu_pow} displays the
timeseries of installed capacity in \gls{EU} nations.

\begin{figure}[htbp!]
	\begin{center}
		\includegraphics[scale=0.7]{./images/eu_future/power_plot.png}
	\end{center}
	\caption{The timeseries of installed nuclear capacity in the EU are separated by \texttt{Region}s in \Cyclus.
			 The sudden drops in capacity are caused by nuclear phaseout plans by nations like Germany and Belgium.
			 }
	\label{fig:eu_pow}
\end{figure}


Once \glspl{SFR} become available in 2040,
600-MWe \glspl{SFR} are deployed to make up for the 
decommissioned \gls{LWR} capacities. 
This results in an installed \gls{SFR} capacity of 66,000 MWe
 by 2076, when the last \gls{LWR} decommissions. The specific deployment 
 schedule is determined by the logic within the Cycamore GrowthRegion archetype. 



\subsection{Material Flow}
The simulated fuel cycle is illustrated in \cref{diag:fc}.
It it a source provides natural uranium. The uranium is enriched by an enrichment
facility to produce \gls{UOX}, while disposing enrichment waste (tails)
to the sink facility. The enriched \gls{UOX} fuels
the \gls{LWR}s and \gls{UOX} waste is produced. The used fuel
is sent to a pool to cool for 3 years \cite{carre_overview_2009}.
The cooled fuel is then reprocessed to separate plutonium and uranium,
or sent to a repository.
The plutonium mixed with depleted uranium (tails) makes \gls{MOX}.
Reprocessed uranium is unused and stockpiled. Uranium is reprocessed
in order to separate the raffinate (Minor actinides and fission products)
from `usable' material. Though neglected in this paper, reprocessed
uranium may substitute depleted uranium for \gls{MOX} production. In the
simulations, sufficient depleted uranium existed that using reprocessed
uranium was overlooked. However, further in the future where the depleted
uranium inventory drains, reprocessed uranium (or, natural uranium) will need to be utilized. 


% Define block styles
\tikzstyle{decision} = [diamond, draw, fill=blue!20, 
text width=4.5em, text badly centered, node distance=3cm, inner sep=0pt]
\tikzstyle{block} = [rectangle, draw, fill=blue!20, 
text width=5em, text centered, rounded corners, minimum height=4em]
\tikzstyle{line} = [draw, -latex']
\tikzstyle{cloud} = [draw, ellipse,fill=red!20, node distance=3cm,
minimum height=2em]


\begin{figure}
        \centering
        \scalebox{0.7}{
                \begin{tikzpicture}[align=center, node distance = 3cm and 3cm, auto]
                % Place nodes
                \node [block] (sr) {Mine (\texttt{SOURCE})};
                \node [cloud, below of=sr] (nu) {Nat U};
                \node [block, below of=nu] (enr) {Enrichment ({\footnotesize \texttt{ENRICHMENT}})};
                \node [cloud, below of=enr] (uox) {\gls{UOX}};
                \node [block, below of=uox] (lwr) {\gls{LWR} (\texttt{REACTOR})};
                \node [cloud, right of=lwr] (snf) {\gls{UNF}};
                \node [cloud, right of=uox] (cunf) {Cooled \gls{UNF}};
                \node [block, right of=snf] (pool) {Pool (\texttt{Storage})};
                \node [cloud, left of=lwr] (tl2) {Dep U};
                \node [cloud, right of=enr] (tl) {Dep U};
                \node [block, right of=tl] (sk) {Repository (\texttt{SINK})};
                \node [cloud, below of=pool] (cunf2) {Cooled \gls{UNF}};
                \node [block, below of=snf] (rep) {{\small Reprocessing ({\footnotesize \texttt{SEPARATIONS}})}};
                \node [cloud, below of=rep] (u) {Sep. U} ;
                \node [cloud, left of=rep] (pu) {Sep. Pu};
                \node [block, left of=pu] (mix) {Fabrication (\texttt{MIXER})};
                \node [cloud, below of=mix] (mox) {\gls{MOX}};
                \node [block, below of=mox] (mxr) {\gls{MOX} Reactors};
                \node [cloud, right of= mxr] (snmox) {Spent \gls{MOX}};
                
                \draw[->, thick] (sr) -- (nu);
                \draw[->, thick] (nu) -- (enr);
                \draw[->, thick] (enr) -- (tl);
                \draw[->, thick] (enr) -- (tl2);
                \draw[->, thick] (tl) -- (sk);
                \draw[->, thick] (tl2) -- (mix);
                \draw[->, thick] (enr) -- (uox);
                \draw[->, thick] (uox) -- (lwr);
                \draw[->, thick] (lwr) -- (snf);
                
                \draw[->, thick] (lwr) -- (snf);
                \draw[->, thick] (snf) -- (pool);
                \draw[->, thick] (pool) -- (cunf);
                \draw[->, thick] (pool) -- (cunf2);
                \draw[->, thick] (cunf) -- (sk);
                \draw[->, thick] (cunf2) -- (rep);
                
                \draw[->, thick] (rep) -- (u);
                \draw[->, thick] (rep) -- (pu);
                \draw[->, thick] (pu) -- (mix);
                \draw[->, thick] (mix) -- (mox);
                \draw[->, thick] (mox) -- (mxr);
                \draw[->, thick] (mxr) -- (snmox);
                \draw[->, thick] (snmox) -- (rep);
                
                \end{tikzpicture}
        
                }
                \caption{The blue boxes represent fuel cycle facilities, and the red ovals
	                	 represent materials. The facility names in parenthesis are archetype names
	                	 used in \Cyclus. \gls{MOX} Reactors include both \gls{MOX} \glspl{LWR} and
	                	 \glspl{SFR}.}
                \label{diag:fc}
\end{figure}

\subsection{Scenario Specifications}
%% EARLY???

The scenario specifications for both simulations are
listed in tables \ref{tab:sim_eu} and \ref{tab:sim_france}.
The reprocessing and \gls{MOX} fabrication capacity for the 
first simulation are modeled after the 
French La Hague and MELOX site \cite{schneider_spent_2008, hugelmann_melox_1999}.

\begin{table}[h]
	\centering
	\begin{tabularx}{\textwidth}{bb}
		\hline
		\textbf{Specification} &\textbf{ Value} \\
		\hline
		Simulation Time & 1970-2050 \\ 
		Reprocessing Capacity & 91.6 MTHM of \gls{UNF} per month \cite{schneider_spent_2008} \\
		Reprocessing Efficiency & 99.8\% \\
		Reprocessing Streams & plutonium and uranium \\
		\gls{MOX} Fabrication & \small{9\% Reprocessed Pu + 91\% Depleted U} \\
		\gls{MOX} Fabrication Throughput & 16.25 MTHM of \gls{MOX} per month  \cite{hugelmann_melox_1999} \\
		\gls{MOX} Fuel Reprocessing Stage &  Used \gls{MOX} gets reprocessed infinitely. \\  
		Reprocessed Uranium Usage &  None. Stockpile reprocessed U \\
		\hline
	\end{tabularx}
	\caption {Specification for Historical Operation of \gls{EU} Case}
	\label{tab:sim_eu}
\end{table}

\begin{table}[h]
	\centering
	\begin{tabularx}{\textwidth}{bb}
		\hline
		\textbf{Specification }& \textbf{Value} \\
		\hline
		Simulation Time & 1970-2160 \\
		\gls{SFR} Available Year & 2040 \\
		\gls{UOX} Reprocessing Capacity & 20 tons per timestep \\
		\gls{MOX} Reprocessing Capacity & $\infty$ \\
		Reprocessing and Fabrication Begins & 2020 \\
		Separation Efficiency & 99.8 \% \\
		Reprocessing Streams & plutonium and uranium \\
		\small{Used \gls{UOX} and Depleted U Inventory} & 125,453 MTHM {\small (From first simulation)} \\
		\small{Additional Used \gls{UOX} or Depleted U} & None  \\
		\gls{MOX} Fabrication &  \small{11\% Reprocessed Pu + 89\% Depleted U}  \\
		\gls{MOX} Fabrication Throughput & infinite \\
		\gls{MOX} Fuel Reprocessing Stage &  Used \gls{MOX} gets reprocessed infinitely. \\
		Reprocessed Uranium Usage &  None. Stockpile reprocessed U. \\
		\hline
	\end{tabularx}
	\caption {Specification for French Transition to \gls{SFR} Case}
	\label{tab:sim_france}
\end{table}


\subsection{Reactor Specifications}
Three major reactors are used in the simulation, \gls{PWR}, \gls{BWR}, and ASTRID-type \gls{SFR}reactors.

For \glspl{LWR}, we used a linear core size model to capture
varying reactor capacity. For example, a 
1,200 MWe PWR has $193*\frac{1,200}{1,000} = 232$ \gls{UOX} assemblies, each
weighing 523.4 kg.
After each 18 month cycle, one-third of the 
core (77 assemblies) discharges. Refueling
is assumed to take 2 months to complete, during which the reactor
is shut down. This value is acquired by averaging the 
historical refueling outage. The specifications are defined in 
\Cref{tab:reactor-specs} details the reactor specifications in this simulation.

\begin{table}[h]
	\centering
	\begin{tabularx}{\textwidth}{blll}
		\hline
		\textbf{Specification} & \textbf{\gls{PWR}} & \textbf{\gls{BWR}} & \textbf{\gls{SFR}}\\
		\hline
                Lifetime [y] & 60 & 60 & 80 \\
                Cycle Time [mos.]& 18 & 18 & 12\\ 
                Refueling Outage [mos.]& 2 & 2  & 2\\
                Rated Power [MWe] & 1000 & 1000 & 600\\
                Assembly mass [kg] & 523.4 & 180 & -- \\
                Batch mass [kg] & -- & -- & 5,568\\
                Discharge Burnup [GWd/tHM] & 51 & 51 & 105 \\
                Assemblies per core & 193  & 764 & -- \\
                Batches per core & 3 & 3 & 4\\
                Fuel & \gls{UOX} or \gls{MOX} & \gls{UOX} & \gls{MOX} \\
		\hline
	\end{tabularx}
        \caption {\gls{LWR} Specifications are derived from 
                \cite{duderstadt_nuclear_1976} while \gls{ASTRID} \gls{SFR} 
                specifications were sourced from 
        \cite{varaine_pre-conceptual_2012}. The simulated reactor lifetime reaches the licensed lifetime unless 
        the reactor is shut down prematurely.  The number of assemblies and corresponding \gls{LWR} core 
        masses are reported for a 1000MWe core. Reactors with different core  
        powers are modeled with a linear mass assumption.}
	\label{tab:reactor-specs}
	\end{table}

	
\FloatBarrier


\subsection{Material Definitions}
Depletion calculations of the nuclear fuel are recipe-based, such that a fresh 
and used fuel recipe is defined for each reactor type.
For the compositions of the fuel, a reference depletion calculation
from ORIGEN is used (see \cref{tab:comp}). The recipe has also been used for
\cite{wilson_adoption_2009}.
