\section{Scenario Specifications}
%% EARLY???
This paper uses results from two separate simulations.
The first simulation is a historical operation of \gls{EU} reactors, with a
realistic reprocessing and \gls{MOX} fabrication capacity, 
modeled after the French La Hague and MELOX site \cite{schneider_spent_2008, hugelmann_melox_1999}.
The second simulation is an ideal French Transition scenario to \gls{SFR},
where an ASTRID-type \gls{SFR} replaces the decommissioned
capacity of \glspl{LWR} in France. The specifications of the simulations
are listed in tables \ref{tab:sim_eu} and \ref{tab:sim_france}.

\begin{table}[h]
	\centering
	\begin{tabularx}{\textwidth}{bb}
		\hline
		Specification & Value \\
		\hline
		Simulation Time & 1970-2050 \\ 
		Reprocessing Capacity & 91.6 MTHM of \gls{UNF} per month \cite{schneider_spent_2008} \\
		Reprocessing Efficiency & 99.8\% \\
		Reprocessing Streams & Plutonium and Uranium \\
		\gls{MOX} Fabrication & \small{9\% Reprocessed Pu + 91\% Depleted U} \\
		\gls{MOX} Fabrication Throughput & 16.25 MTHM of \gls{MOX} per month  \cite{hugelmann_melox_1999} \\
		\gls{MOX} Fuel Reprocessing Stage &  Used \gls{MOX} gets reprocessed infinitely. \\  
		Reprocessed Uranium Usage &  None. Stockpile reprocessed U \\
		\hline
	\end{tabularx}
	\caption {Specification for Historical Operation of \gls{EU} Case}
	\label{tab:sim_eu}
\end{table}

\begin{table}[h]
	\centering
	\begin{tabularx}{\textwidth}{bb}
		\hline
		Specification & Value \\
		\hline
		Simulation Time & 1970-2160 \\
		\gls{SFR} Available Year & 2040 \\
		Reprocessing Capacity & $\infty$ \\
		Reprocessing and Fabrication Begins & 2020 \\
		Separation Efficiency & 99.8 \% \\
		Reprocessing Streams & plutonium and uranium \\
		\small{Used \gls{UOX} and Depleted U Inventory} & 157,472 MTHM (From first simulation) \\
		\small{Additional Used \gls{UOX} or Depleted U} & None  \\
		\gls{MOX} Fabrication &  \small{11\% Reprocessed Pu + 89\% Depleted U}  \\
		\gls{MOX} Fabrication Throughput & infinite \\
		\gls{MOX} Fuel Reprocessing Stage &  Used \gls{MOX} gets reprocessed infinitely. \\
		Reprocessed Uranium Usage &  None. Stockpile reprocessed U. \\
		\hline
	\end{tabularx}
	\caption {Specification for French Transition to \gls{SFR} case }
	\label{tab:sim_france}
\end{table}


\section{Reactor Specifications}
Two major reactors are used in the simulation, \gls{PWR} and ASTRID - type reactors.
For simplicity, the few \glspl{BWR} in the \gls{EU} fleet are assumed to be \glspl{PWR}.

For \glspl{PWR}, a linear core size model was assumed to capture
varying reactor capacity. For example, a 
1,330 MWe PWR has 257 \gls{UOX} assemblies, each
weighing 523.4 kg.
After each 18 month cycle, one-third of the 
core (85 assemblies) discharge. Refueling
is assumed to take 2 months to complete, during which the reactor
is shut down. The specifications are defined in \cref{tab:pwr}.

For the \gls{SFR}, a model design is adopted from
Marsault-Marie-Sophie et al. \cite{marsaultmarie-sophie_pre-conceptual_2012}.
The specifications are defined in \cref{tab:sfr}.

\begin{table}[h]
	\centering
	\begin{tabularx}{\textwidth}{bb}
		\hline
		Specification & Value \\
		\hline
		PWR Cycle Time & 18 months \\ 
		PWR Refueling Outage & 2 months \\
		Fuel Mass per Assembly & 523.4 kg \\
		Burnup & 51 GWd/tons \\
		Num. of Aseem. per Core & 257 for 1,330 MWe, linearly adjusted\\
		Num. of Assem. per Batch & 1/3 of the core \\
		Fuel & \small{French \glspl{PWR} prefer \gls{MOX} but also accept \gls{UOX}}\\
		\hline
	\end{tabularx}
	\caption {\gls{PWR} Specifications}
	\label{tab:pwr}
	\end{table}
	
\begin{table}[h]
	\centering
	\begin{tabularx}{\textwidth}{bb}
		\hline
		Specification & Value \\
		\hline
		SFR Cycle Time & 12 months \\ 
		SFR Refueling Outage & 2 months \\
		Fuel Mass per Batch & 11,136 kg \\
		Batch per Core & 4 \\
		Power Output & 600 MWe \\
		lifetime & 80 years \\
		Fuel & {\small \gls{MOX} (89\% Tailings, 11\% Separated Pu)}\\
		\hline
	\end{tabularx}
	\caption {\gls{SFR} ASTRID Specifications \cite{marsaultmarie-sophie_pre-conceptual_2012}}
	\label{tab:sfr}
\end{table}
\FloatBarrier

