\section{Scenario Specifications}

The scenario specifications  are
listed in tables \ref{tab:gen}, \ref{tab:sim_eu}, and \ref{tab:sim_france}.
The reprocessing and \gls{MOX} fabrication capacity in France
prior to 2020 is modeled after the 
French La Hague and MELOX site \cite{schneider_spent_2008, hugelmann_melox_1999}.


\begin{table}[h]
    \centering
    \begin{tabularx}{\textwidth}{bb}
        \hline
        \textbf{Specification} &\textbf{ Value} \\
        \hline
        Simulation Time & 1970-2160 \\ 
        Reprocessed Uranium Usage &  None. Stockpile reprocessed U \\
        Storage Residence Time & 36 months \\
        \gls{SFR} available year & 2040 \\
        Production of \gls{ASTRID} fuel begins & 2020 \\
        \hline
    \end{tabularx}
    \caption {General Simulation Specifications}
    \label{tab:gen}
\end{table}

\begin{table}[h]
	\centering
	\begin{tabularx}{\textwidth}{bb}
		\hline
		\textbf{Specification} &\textbf{ Value} \\
		\hline
		Reprocessing Capacity & 91.6 MTHM of \gls{UNF} per month \cite{schneider_spent_2008} \\
		Reprocessing Efficiency & 99.8\% \\
		Reprocessing Streams & plutonium and uranium \\
		\gls{PWR} \gls{MOX} Fabrication & \small{9\% Reprocessed Pu + 91\% Depleted U} \\
		\gls{MOX} Fabrication Throughput & 16.25 MTHM of \gls{MOX} per month  \cite{hugelmann_melox_1999} \\
		\gls{MOX} Fuel Reprocessing Stage &  Used \gls{MOX} is not reprocessed. \\  
		\hline
	\end{tabularx}
	\caption {Specification for Historical Operation of \gls{EU}}
	\label{tab:sim_eu}
\end{table}

\begin{table}[h]
	\centering
	\begin{tabularx}{\textwidth}{bb}
		\hline
		\textbf{Specification }& \textbf{Value} \\
		\hline
		Separation Efficiency & 99.8 \% \\
		Reprocessing Streams & plutonium and uranium \\
		\gls{ASTRID} Fuel Fabrication &  \small{22\% Reprocessed Pu + 78\% Depleted U}  \\
		\gls{ASTRID} Fuel Reprocessing Stage &  Used \gls{MOX} gets reprocessed infinitely. \\
		\hline
	\end{tabularx}
	\caption {Specification for French Transition to \glspl{ASTRID} }
	\label{tab:sim_france}
\end{table}


\section{Reactor Specifications}
Three major reactors are used in the simulation, \gls{PWR}, \gls{BWR}, and ASTRID-type \gls{SFR}reactors.

For \glspl{LWR}, we used a linear core size model to capture
varying reactor capacity. For example, a 
1,200 MWe PWR has $193*\frac{1,200}{1,000} = 232$ \gls{UOX} assemblies, each
weighing 523.4 kg.
After each 18 month cycle, one-third of the 
core (77 assemblies) discharges. Refueling
is assumed to take 2 months to complete, during which the reactor
is shut down. This value is acquired by averaging the 
historical refueling outage. The specifications are defined in 
\Cref{tab:reactor-specs} details the reactor specifications in this simulation.

\begin{table}[h]
	\centering
	\begin{tabularx}{\textwidth}{blll}
		\hline
		\textbf{Specification} & \textbf{\gls{PWR} \cite{sutharshan_ap1000tm_2011}} & \textbf{\gls{BWR} \cite{hinds_next-generation_2006}} & \textbf{\gls{SFR}} \cite{varaine_pre-conceptual_2012}\\
		\hline
                Lifetime [y] & 60 & 60 & 80 \\
                Cycle Time [mos.]& 18 & 18 & 12\\ 
                Refueling Outage [mos.]& 2 & 2  & 2\\
                Rated Power [MWe] & 1000 & 1000 & 600\\
                Assembly mass [kg] & 523.4 & 180 & -- \\
                Batch mass [kg] & -- & -- & 5,568\\
                Discharge Burnup [GWd/tHM] & 51 & 51 & 105 \\
                Assemblies per core & 193  & 764 & -- \\
                Batches per core & 3 & 3 & 4\\
                Initial Fissile Loading & 3.1t U235 & 4.2t U235 & 4.9t Pu \\
                Fuel & \gls{UOX} or \gls{MOX} & \gls{UOX} & \gls{MOX} \\
		\hline
	\end{tabularx}
        \caption {Model \gls{LWR} and \gls{ASTRID} specifications used for the simulations are listed, and \glspl{LWR} are modified
        linearly for varying power capacity. }
	\label{tab:reactor-specs}
	\end{table}

        \footnotetext{The simulated reactor lifetime reaches the licensed lifetime unless 
        the reactor is shut down prematurely.} 
        \footnotetext{Number of assemblies and corresponding \gls{LWR} core 
        masses are reported for a 1000MWe core. Reactors with different core  
        powers are modeled with a linear mass assumption.}
	
\FloatBarrier

