\section{Discussion}
From the simulation results, given infinite
reprocessing and \gls{MOX} fabrication capacities,
France can transition into a fully \gls{SFR} fleet
with installed capacity of 60,000 MWe by 2076.
The initial fuel demand will be filled by \gls{MOX}
created from the old \gls{SNF}, which later on
will be met by \gls{MOX} created from recycled \gls{MOX}.

Both France and other EU nations have the incentive
to send all its \gls{SNF} to France, since most EU nations do not have an 
operating \gls{SNF} repository or a management plan.
Other EU countries will benefit by getting rid of their \gls{SNF},
and France will benefit by financial incentives and no additional 
construction of \gls{LWR}s.

Though there are obvious political and economical factors to consider,
and various assumptions were made for this scenario, this option
may hold value for the EU as a nuclear community,
and for France to advance into a closed fuel cycle.