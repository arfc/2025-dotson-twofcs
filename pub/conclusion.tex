\section{Discussion}
% This is the conclusion section. Depending on the preferences of the journal, it
% may be called either ``Conclusions'' or ``Discussion''. Your results section
% should have made clear what was accomplished in the in silico experiments. Here,
% you want to reflect on what you promised in the introcution and answer the
% question ``So what?'' or ``Why is this interesting''.

% \begin{itemize}
%         % \item With only one or two sentences, summarize the main conclusions of
%         %         the paper (not a lengthy rehashing of the results section, but a
%         %         clarification on what those results mean for the field). 
%         \item 
%         % \item if there are any limitations in the scope of this paper, take a
%         %         sentence or two to state the future work that could be done to
%         %         extend it. 
%         \item Include a forward looking statement, perhaps one that predicts
%                 future uses of Osier or future multi-objective models inspired
%                 by this demonstrated capability, and the effect that will have
%                 on energy planning.  
% \end{itemize}

% point 1: summary
This paper introduces and demonstrates \gls{osier}, a novel \gls{esom} that
enables multi-objective optimization. First, we verified that \gls{osier} agrees
well with a mature \gls{esom} for a single objective problem. We also showed
that conventional \gls{mga} does not lead to measurably better results on
another objective by using \gls{osier} to optimize cost and carbon emissions
simultaneously. However, we also acknowledged the persistence of structural
uncertainty and demonstrated a novel algorithm that extends \gls{mga} into
N-dimensions. Second, we used \gls{osier} to reevaluate the results from a fuel
cycle options study through the lens of Pareto optimality --- corresponding to a
17-objective optimization problem. These results showed disagreement between
conclusions drawn from the \gls{set} and conclusions based on the Pareto front. 

% point 2: gaps addressed
\emph{In the Introduction, you motivated the effort with a description
                of a gap in the field. Now, make clear whether your results
                effectively filled that gap, whether they contradicted earlier
                work or complement that work. }

\textcolor{red}{Have I satisfied this point or does it still need clarification?}
\textcolor{blue}{I think you've done enough -- let's see what the reviewers 
say.}

% point 3: limitations
Although we successfully used Pareto optimality to reanalyze the results from 
a fuel cycle options study, our conclusions from this experiment are somewhat 
limited. This is because we assumed that the \glspl{eg} were mutually exclusive
and therefore we knew the design space \textit{a priori}. Further, our design space
was limited to units of \textit{entire fuel cycles} since we only used the raw 
performance data for each fuel cycle from the \gls{set}. Future work should 
extend this study by enabling \gls{osier} to conduct a fuel cycle from cradle to 
grave to develop a more sophisticated decision space. This may be accomplished
by adding more physics to \gls{osier} or by coupling \gls{osier} with a known 
fuel cycle simulator, such as \texttt{Cyclus}\cite{huff_fundamental_2016}.

% point 4: forward
Despite the tradition of optimizing energy systems based on a singular cost
objective, real-world decision-making is always a multi-objective problem.
\gls{osier} allows modelers and decision-makers to analyze tradeoffs for 
quantifiable objectives and, through N-dimensional \gls{mga}, consider
alternative solutions for objectives that are not quantifiable. By opening
the decision-making black box through explicit optimization on multiple objectives,
members of the public and other stakeholders may participate more effectively 
in the process of decision-making. Thereby democratizing our energy systems and improving
the justice and equity outcomes related to the ways we produce energy.
