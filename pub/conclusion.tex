\section{Discussion}
This is the conclusion section. Depending on the preferences of the journal, it 
may be called either ``Conclusions'' or ``Discussion''. 
Your results section should have made clear what was accomplished in the in 
silico experiments. Here, you want to reflect on what you promised in the 
introcution and answer the question ``So what?'' or ``Why is this 
interesting''.

\begin{itemize}
        \item With only one or two sentences, summarize the main conclusions of 
                the paper (not a lengthy rehashing of the results section, but 
                a clarification on what those results mean for the field). 
        \item In the Introduction, you motivated the effort with a description 
                of a gap in the field. Now, make clear whether your results 
                effectively filled that gap, whether they contradicted earlier 
                work or complement that work. 
        \item if there are any limitations in the scope of this paper, take a 
                sentence or two to state the future work that could be done to 
                extend it. 
        \item Include a forward looking statement, perhaps one that predicts 
                future uses of Osier or future multi-objective models inspired 
                by this demonstrated capability, and the effect that will have on energy 
                planning.  
\end{itemize}
