%        File: revise.tex
%     Created: Wed Oct 27 02:00 PM 2018 P
% Last Change: Wed Oct 27 02:00 PM 2018 P
%

%
% Copyright 2007, 2008, 2009 Elsevier Ltd
%
% This file is part of the 'Elsarticle Bundle'.
% ---------------------------------------------
%
% It may be distributed under the conditions of the LaTeX Project Public
% License, either version 1.2 of this license or (at your option) any later
% version.  The latest version of this license is in
% http://www.latex-project.org/lppl.txt and version 1.2 or later is part of all
% distributions of LaTeX version 1999/12/01 or later.
%
% The list of all files belonging to the 'Elsarticle Bundle' is given in the
% file `manifest.txt'.
%

% Template article for Elsevier's document class `elsarticle' with numbered
% style bibliographic references SP 2008/03/01
%
%
%
% $Id: elsarticle-template-num.tex 4 2009-10-24 08:22:58Z rishi $
%
%
%\documentclass[preprint,12pt]{elsarticle}
\documentclass[answers,11pt]{exam}

% \documentclass[preprint,review,12pt]{elsarticle}

% Use the options 1p,twocolumn; 3p; 3p,twocolumn; 5p; or 5p,twocolumn for a
% journal layout: \documentclass[final,1p,times]{elsarticle}
% \documentclass[final,1p,times,twocolumn]{elsarticle}
% \documentclass[final,3p,times]{elsarticle}
% \documentclass[final,3p,times,twocolumn]{elsarticle}
% \documentclass[final,5p,times]{elsarticle}
% \documentclass[final,5p,times,twocolumn]{elsarticle}

% if you use PostScript figures in your article use the graphics package for
% simple commands \usepackage{graphics} or use the graphicx package for more
% complicated commands
\usepackage{graphicx}
% or use the epsfig package if you prefer to use the old commands
% \usepackage{epsfig}

% The amssymb package provides various useful mathematical symbols
\usepackage{amssymb}
% The amsthm package provides extended theorem environments \usepackage{amsthm}
\usepackage{amsmath}

% The lineno packages adds line numbers. Start line numbering with
% \begin{linenumbers}, end it with \end{linenumbers}. Or switch it on for the
% whole article with \linenumbers after \end{frontmatter}.
\usepackage{lineno}

% I like to be in control
\usepackage{placeins}

% natbib.sty is loaded by default. However, natbib options can be provided with
% \biboptions{...} command. Following options are valid:

%   round  -  round parentheses are used (default) square -  square brackets are
%   used   [option] curly  -  curly braces are used      {option} angle  -
%   angle brackets are used    <option> semicolon  -  multiple citations
%   separated by semi-colon colon  - same as semicolon, an earlier confusion
%   comma  -  separated by comma numbers-  selects numerical citations super  -
%   numerical citations as superscripts sort   -  sorts multiple citations
%   according to order in ref. list sort&compress   -  like sort, but also
%   compresses numerical citations compress - compresses without sorting
%
% \biboptions{comma,round}

% \biboptions{}


\usepackage{xspace}
\usepackage{color}

\usepackage{multirow}
\usepackage[hyphens]{url}


\usepackage[acronym,toc]{glossaries}
\include{acros}

\makeglossaries

%\journal{Annals of Nuclear Energy}

\begin{document}

%\begin{frontmatter}

% Title, authors and addresses

% use the tnoteref command within \title for footnotes; use the tnotetext
% command for the associated footnote; use the fnref command within \author or
% \address for footnotes; use the fntext command for the associated footnote;
% use the corref command within \author for corresponding author footnotes; use
% the cortext command for the associated footnote; use the ead command for the
% email address, and the form \ead[url] for the home page:
%
% \title{Title\tnoteref{label1}} \tnotetext[label1]{}
% \author{Name\corref{cor1}\fnref{label2}} \ead{email address} \ead[url]{home
% page} \fntext[label2]{} \cortext[cor1]{} \address{Address\fnref{label3}}
% \fntext[label3]{}

\title{Demonstrating the Osier framework for energy system and nuclear fuel
cycle optimization\\
        \large Response to Review Comments}
\author{Samuel G. Dotson, Madicken Munk, Kathryn D. Huff}

% use optional labels to link authors explicitly to addresses:
% \author[label1,label2]{<author name>} \address[label1]{<address>}
% \address[label2]{<address>}


%\author[uiuc]{Kathryn Huff} \ead{kdhuff@illinois.edu} \address[uiuc]{Department
%        of Nuclear, Plasma, and Radiological Engineering, 118 Talbot
%        Laboratory, MC 234, Universicy of Illinois at Urbana-Champaign, Urbana,
%        IL 61801}
%
% \end{frontmatter}
\maketitle
\section*{Review General Response}
We would like to thank the reviewers for their detailed assessment of this
paper. Your comments have resulted in changes which certainly improved the
paper.


\section*{Reviewer 1}
\begin{questions}

        \question Page 2, line 7: Beginning of sentence is not capitalized.
        \begin{solution}
                Thank you for identifying this error. I have corrected the
                mistake by invoking the \texttt{\textbackslash Glspl} command
                rather than the \texttt{\textbackslash glspl}. The new sentence
                begins on page 2, line 8.
        \end{solution}
        \question Page 3, line 22: Should you have the citation with Osier the
        first time you mention it? And somewhere around this paragraph may be a
        good place to put the graphical abstract.
        \begin{solution}
                Yes, the citation for Osier belongs in the body of the text
                rather than just in the graphical abstract! I added the citation
                on page 3, line 24.
        \end{solution}
        \question Page 3, line 35: Are you doing two experiments, or two sets of
        experiments? You seem to say both in this line.
        \begin{solution}
                You are correct. I harmonized the language by referring only to
                a single experiment on page 3, line 35.
        \end{solution}
        \question Page 3, line 36: Do you have a citation for Temoa?
        \begin{solution}
                Absolutely, thank you for catching this omission. I added the
                citation on page 3, line 36.
        \end{solution}
        \question Page 3, lines 37-39: Do you mean to provide some results in
        your introduction?
        \begin{solution}
                No, the line has been removed.
        \end{solution}
        \question Page 4, lines 48-51: It's not clear to me the benefit in the
        sub-optimal sampling.
        \begin{solution}
                Thank you for identifying a possible point of confusion. I added
                a sentence on page 4, lines 49-53 that elaborates on the
                usefulness of sampling a model's sub-optimal space.
        \end{solution}
        \question Page 4, line 61: The Evaluation and Screening Study is more
        commonly referred to as the E\&S. I recommend making this change to be
        more consistent with existing literature and make things more obvious to
        your readers. I understand that the SET Tool is correctly named here,
        referring to the excel sheet. But when referring to the study, calling
        it the E\&S will be commonly recognized.
        \begin{solution}
                Thank you for the suggestion. I consolidated references to the
                Study and the Fuel Cycle Options campaign to ``E\&S'' for
                consistency with existing literature. Starting on page 5, line
                68.
        \end{solution}
        \question Page 7: CAP: is this thermal or electric capacity?
        \begin{solution}
                In theory, Osier could accommodate either. However, in this case
                I am referring to the \textit{electric} capacity. I modified the
                text to clarify.
        \end{solution}
        \question Page 7: fom/vom in the variable definitions: Are these
        fixed/variable operation \& maintenance costs? You only say operation --
        the m may be confusing if people aren't thinking while they read.
        \begin{solution}
                The omission was intentional but confusing. I replaced
                ``operation'' with ``O\&M'' to keep the definition short.
        \end{solution}
        \question Page 7, line 81: I feel like you're missing a word, "We
        generated 15 alternative solutions [with] Temoa's MGA"?
        \begin{solution}
                Thank you for catching this! I added the missing word on page 7,
                line 85.
        \end{solution}
        \question Page 7, lines 86-87: I'm not sure the expected reader for this
        article (someone interested in fuel cycle modeling) would be familiar
        with what you mean by "structural uncertainty". I recommend adding more
        about this, since it's clearly important.
        \begin{solution}
                I agree that this may be confusing since this version of the
                paper had not really defined ``structural uncertainty at this
                point.'' I believe I resolved this by defining the term as part
                of my response to item 6 (page 4, lines 49-53).
        \end{solution}
        \question Page 8, line 106: What do you mean by the Fuel Cycle Options
        report? Are you referring to one of the appendices in the E\&S? Or the
        Fuel Cycle Options Catalog?
        \begin{solution}
                Yes, please see my response to item 7.
        \end{solution}
        \question Page 8, line 107: Can you talk more about what you mean by
        mutually exclusive? And why that means you can skip the GA part of
        Osier?
        \begin{solution}
                I agree that this deserves further clarification. I added
                another sentence on page 8, line 111 elaborating on ``mutual
                exclusivity.''
        \end{solution}
        \question Page 9, line 114: When you mention the Nuclear Fuel Cycle
        Options Study, is this still the E\&S? You have the same citation as the
        E\&S, so I'm not sure why you're using a different name.
        \begin{solution}
               Yes, please see my response to item 7.
        \end{solution}
        \question Page 9, line 124: What exactly defines the sub-optimal space
        from each code? I would think sub-optimal would be anything above and to
        the right of your Pareto front in Figure 2, but you've only highlighted
        a section of that space of the graph. Also, you have this space labeled
        as "near-optimal" in Figure 2, so I recommend reconciling your language.
        \begin{solution}
                Thank you, I agree that the language could be more precise! You
                are correct about the sub-optimal space. The near-optimal space
                is decided when a user defines the slack variable (which is
                necessarily a subset of the sub-optimal/feasible space.) I
                changed relevant instances of ``sub-optimal'' to
                ``near-optimal'' on page 9, lines 129-130.
        \end{solution}
        \question Page 10, Figure 2: It's a little hard to read the x- and
        y-scales of the zoomed in portion because of the data points. Is there a
        way to make that more readable?
        \begin{solution}
                I changed the color of the tested points from black to gray in
                figure 2. I also changed the random seed (which selects a subset
                of all tested points for illustration purposes) which reduced
                the number of points overlapping with the tick labels.
        \end{solution}
        \question Page 14, line 164: You're minimizing Eq 3, which is minimizing
        the tradeoff between two solutions, but it's not clear what the
        tradeoffs are that you're minimizing. Are you minimizing the tradeoff
        between two of the metrics in the x-axis of Figure 6, if so, which ones?
        \begin{solution}
                The compromise solution is/are the solution(s) which minimize(s)
                the tradeoff (ratio between improvement and deterioration). I
                added a new equation and equation 3 is now equation 4, which
                clarifies that this is done over the set of nondominated
                solutions. I also added some text on page 9, lines 117-119.
        \end{solution}
        \question Page 16, Table 2: You discuss the different categories of the
        E\&S conclusions, and for the Osier results you just focus on if an EG
        is on the Pareto front or not. Can you identify if any of the
        "potentially promising" scenarios in the E\&S are in the "near-optimal"
        area of the Osier results? I feel like "potentially promising" and
        "near-optimal" results for the same EG would suggest some agreement
        between the two tools.
        \begin{solution}
                This is a valid question. All of the other EGs not listed in
                Table 2 were found to be Pareto optimal. Only the non-optimal
                solutions could be in the near-optimal space. Osier does not
                distinguish ``degrees of promising'' because these
                recommendations were partially based on qualitative expert
                judgement.
        \end{solution}
        \question Page 16: lines 184-188: I appreciate the discussion of the
        differences in binning and how they may lead to the differences in the
        Osier and E\&S results.
        \begin{solution}
                Thank you!
        \end{solution}
        \question Page 13, Section 3.2: Some of the main (and well-known)
        conclusions of the E\&S is the preferred performance of EGS 23, 29, 24,
        and 30. Since you didn't include them in Table 2, I'm guessing Osier
        identified them as on the Pareto front, and therefore optimal. But that
        would be a useful result to explicitly call out.
        \begin{solution}
                Thank you, I added a sentence at the end of the section, page 16 lines 186-188.
                indicating that Osier and the E&S demonstrate some agreement.
        \end{solution}
        \question Page 19, Citation 8: I recommend adding a URL for the E\&S. If
        you pulled data/information from different appendices of this report, I
        recommend pulling that out in a separate citation, since the report is
        very long and providing a more specific reference to your data is always
        helpful.
        \begin{solution}
                All of the ES citations now have a URL. Appendices C \& D have their own 
                citations.
        \end{solution}
\end{questions}

\section*{Reviewer 2}
\begin{questions}

        %---------------------------------------------------------------------
        \question The color scheme in figures 4 and 7 make it difficult to
        distinguish each individual components in the figures.

        \begin{solution}
        	Thank you for your valuable input. We changed the color scheme
        	to ensure that the colors representing each component are more
        	distinct and easier to tell apart.

        \end{solution}

        %---------------------------------------------------------------------
        \question Review comment 2
        \begin{solution}
        	Changes for review comment 2

        \end{solution}


        %---------------------------------------------------------------------
\end{questions}

\bibliographystyle{unsrt}
\bibliography{revise}
\end{document}

  %
  % End of file `elsarticle-template-num.tex'.
