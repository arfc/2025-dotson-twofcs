\begin{abstract}
Energy system optimization models are a class of
tools designed to optimize energy planning and are used by energy planners and
decision-makers to generate insights that inform energy policy. However, existing
tools are challenged by real-world scenarios which require optimization across multiple objectives..
In this paper, the multi-objective energy system optimization framework, \texttt{Osier}, is
demonstrated. \texttt{Osier} leverages a novel genetic algorithm to calculate a
subset of maximally different solutions within the
sub-optimal space to address this uncertainty related to unmodeled objectives.
By producing multiple solutions, \texttt{Osier} gives modelers and
decision-makers the tools to meaningfully engage with public stakeholders and
learn their preferences, thereby attending to issues of procedural and
recognition justice.  This work verifies \texttt{Osier}'s suitability for energy modeling problems with two \textit{in silico} experiments. The first set of experiments
compare \texttt{Osier} to a more mature energy system optimization model, \texttt{Temoa}, to verify
that \texttt{Osier} produces results consistent with known methods. The results
for a least-cost optimization with \texttt{Osier} and \texttt{Temoa} show
strong agreement, with 0.5\% of each other. In the second, \texttt{Osier} 
reanalyzes a set of nuclear fuel cycle options through the lens of Pareto optimality. 
\end{abstract}
