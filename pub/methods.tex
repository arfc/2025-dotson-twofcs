\section{Methodology}



% Insert information from the methodology chapter of your thesis that introduces
% the algorithmic choices you made in Osier and, why you chose those methods,
% and why you didn't choose other methods.

\gls{osier} applies \gls{moo} to capacity expansion problems by coupling
\glspl{ga} with an energy dispatch model, formulated as a linear program. We
chose genetic algorithms over scalarization methods, such as \gls{ec} or
\gls{ws}, for two reasons. First, scalarization methods are poorly behaved
around convex regions of a Pareto front because the points are either poorly
spaced or entirely missed in a convex region \cite{emmerich_tutorial_2018}.
Second, scalarization methods can only sample points along the Pareto front,
rather than sampling points in the sub-optimal space. Earlier energy modeling
research dismissed \gls{moo} for this reason \cite{decarolis_using_2011}.
\glspl{ga} do not suffer from either of these limitations.

\gls{osier} supports genetic algorithms implemented by the \texttt{Pymoo}
\cite{blank_pymoo_2020} and \texttt{DEAP} \cite{fortin_deap_2012} Python
libraries. These two libraries offer a broad suite of \glspl{ga}, however, the
present study uses the \gls{unsga3} for its superior performance on mono-,
multi-, and many-objective problems over the similar \gls{nsga2} and \gls{nsga3}
\cite{seada_unified_2016}. 

This paper demonstrates \gls{osier} by comparing it to two existing modeling
tools, \gls{temoa} \cite{decarolis_temoa_2010} and \gls{set}
\cite{wigeland_nuclear_2014-2}. \gls{temoa} is a mature capacity expansion model
and one of the first \glspl{esom} to implement a method to address structural
uncertainty, \gls{mga} \cite{decarolis_using_2011,decarolis_modelling_2016}. The
\gls{set} is an Excel-based application containing metric data from the Nuclear
Fuel Cycle Evaluation and Screening Study \cite{wigeland_nuclear_2014-2} and an
objective weighting calculator to help decision makers identify fuel cycles of
interest based on their priorities and preferences. The \gls{set} evaluates
representative fuel cycle options, \glspl{eg}, based on a set of nine aggregated
metrics \cite{wigeland_nuclear_2014}. Table \ref{tab:evaluation-metrics} lists
these metrics and their sub-criteria.

\begin{table}[htbp!]
    \centering
    \caption{Evaluation metrics and evaluation criteria
    \cite{wigeland_nuclear_2014-2}.}
    \label{tab:evaluation-metrics}
    \resizebox{\columnwidth}{!}{\begin{tabular}{ll}
    \toprule
    Metric & Criteria \\
    \midrule
    % Waste management
     & Mass of \gls{snf} + \gls{hlw}\\
     & Activity of \gls{snf} + \gls{hlw} (at 100 years)\\
    Nuclear Waste Management & Activity of \gls{snf} + \gls{hlw} (at 100,000 years)\\
    (per unit energy) & Mass of \gls{du} + \gls{ru}\\
     & \hspace{12em}+ \gls{rth}\\
     & Volume \gls{llw}\\
     \midrule
    % Proliferation
     Proliferation Risk & Material attractiveness -- normal operating conditions\\
     \midrule
    % Security
     Nuclear Material Security & Material attractiveness -- normal operating conditions\\
    Risk & Activity of \gls{snf} + \gls{hlw} (at 10 years) per unit energy\\
    \midrule
    % Safety
    Safety & Challenges of addressing safety hazards\\
    & Safety of deployed system\\
    \midrule
    % Environment
    & Land use\\
    Environmental Impact  & Water use\\
    (per unit energy) & Radiological exposure -- total estimated worker dose\\
     & \gls{co2} emissions\\
     \midrule
    % Resource efficiency
     Resource Utilization & Natural uranium required\\
    (per unit energy) & Natural thorium required\\
    \midrule
    % Deployment risk
    & Development time\\
    Development and & Developemnt cost\\
    Deployment Risk & Cost from prototype to \gls{foak}\\
    & Compatibility with existing infrastructure\\
    &Familiarity with licensing regulations\\
    & Existence of market dis/incenctives\\
    \midrule
    % institutional issues
      & Compatibility with existing infrastructure\\
    Institutional Issues & Familiarity with licensing regulations\\
     & Existence of market dis/incentives\\
     \midrule
    Financial Risks & \gls{lcoe} at equilibrium\\
    and Economics &\\
    \bottomrule
\end{tabular}}
\end{table}
% \FloatBarrier

\subsection{Description of the \gls{osier}-\gls{temoa} benchmark}
\textcolor{red}{I don't love the names of these two subsections...}

For the first experiment, we developed a test model for both \gls{osier} and
\gls{temoa} to directly compare their least-cost solutions. \gls{osier} and
\gls{temoa} optimized the capacity for a single future year with an hourly time
resolution. We used historical timeseries data from 2019 for the \gls{uiuc},
which includes electricity demand, solar panel generation, and wind turbine
power. To facilitate a fair comparison, we recalculated \gls{temoa}'s objective
function using \gls{osier}'s dispatch model to account for differences in the
ways the two models calculate costs. \gls{osier}'s total cost objective function
is given by

\begin{align}
    \label{eq:dispatch-objective}
    \mathcal{C} &= \left(\sum_g^G \textbf{CAP}_g \left(C^{cap}_g + C^{fom}_g\right)\right) 
    + \left(\sum_t^T\sum_g^G \left[C_{g,t}^{fuel} + C_{g,t}^{vom}\right]x_{g,t}
    \right)
    \intertext{where,}
    G &= \text{ the set of all generating technologies},\nonumber\\
    T &= \text{ the set of all time steps in the model},\nonumber\\
    \textbf{CAP}_g &= \text{ the capacity of the \textit{g$^{th}$} 
    technology}\quad \left[MW\right],\nonumber\\
    x_{g,t} &= \text{ the energy produced by generator, \textit{g}, 
    at time, \textit{t}}\quad \left[MWh\right]\nonumber,\\
    C^{cap}_{g} &= \text{the capital cost of the \textit{g$^{th}$} 
    technology} \quad \left[\frac{\$}{MW}\right]\nonumber,\\
    C^{fom}_{g} &= \text{the fixed operating costs of the \textit{g$^{th}$} 
    technology} \quad \left[\frac{\$}{MW}\right]\nonumber,\\
    C^{[fuel, vom]}_{g,t} &= \text{the variable operating costs of the 
    \textit{g$^{th}$} technology at time $t$} \quad \left[\frac{\$}{MWh}\right].\nonumber
\end{align}

We also compared the performance of these tools along a second objective,
lifecycle \gls{co2} emissions. We generated 25 alternative solutions
\gls{temoa}'s \gls{mga} algorithm and then passed the resulting decision
variables to \gls{osier}'s dispatch model to calculate the corresponding
\gls{co2} emissions. \gls{osier} simply optimized cost and emissions
simultaneously. \gls{osier}'s emissions objective function is given by

\begin{align}
    \mathcal{E} &= \sum_g^G \xi_g \sum_t^T x_{g,t},
    \intertext{where}
    \xi_g &= \text{the \gls{co2} intensity of the \textit{g$^{th}$} technology}\quad
    \left[\frac{\text{tons CO$_2$}}{MWh}\right].\nonumber
\end{align}

Since \gls{mga} is such an important tool for addressing structural uncertainty
\cite{yue_review_2018} and since structural uncertainty persists regardless of
the number of modeled objectives \cite{decarolis_using_2011}, \gls{osier}
extends the \gls{mga} concept into N-dimensional objective space by implementing
a novel algorithm that takes advantage of the many solutions that \glspl{ga}
test before converging. As with traditional \gls{mga}, the first step identifies
an acceptable slack value that bounds the sub-optimal space (e.g., within 10\%
of the Pareto front). Since there are guaranteed to be points within the
sub-optimal region, \gls{osier} skips the step to repeatedly solve the model
with additional constraints found in traditional \gls{mga} and instead 
determines which points reside within the designated sub-optimal region. Finally, since
\gls{mga} seeks maximally different solutions in decision space, \gls{osier}
uses a corresponding \textit{farthest-first-traversal} method to identify
maximally different solutions \cite{hochbaum_best_1985}. In addition to comparing
results between \gls{osier} and \gls{temoa}, we also demonstrate this novel approach
to \gls{mga}.

\subsection{Description of the \gls{set} reanalysis} The second experiment
compares \gls{osier} to the \gls{set}. Since the \gls{set} is primarily a
decision-support tool, we reexamine its results using Pareto optimality and
demonstrate a method --- available in \gls{osier} via \texttt{Pymoo} and
borrowed from \gls{mcda} --- to calculate high-tradeoff points. First, we
extracted the raw data for each evaluation criteria from the Fuel Cycle Options
report. We assumed that the nuclear fuel cycles represented in the \gls{set} are
mutually exclusive. Therefore, we skipped the genetic algorithm component of
\gls{osier} since the resulant decision space reduces to the identity matrix.
Then, we determined the Pareto optimality of each \gls{eg} and calculated the
high-tradeoff point which minimizes
\cite{rachmawati_multiobjective_2009}

\begin{align}
    \label{eqn:tradeoff}
    T\left(X_i, X_j\right) &= \frac{\sum_{1}^{M}\text{max}\left[0, 
    \left(f_m\left(X_j\right) - f_m\left(X_i\right)\right]\right)}
    {\sum_{1}^{M}\text{max}\left[0, -\left(f_m\left(X_j\right) - 
    f_m\left(X_i\right)\right)\right]},
    \intertext{where}
    T &= \text{the tradeoff between the i-th and j-th solutions},\nonumber\\
    X_i &= \text{the i-th solution vector}, \nonumber\\
    f_m &= \text{the m-th objective function}, \nonumber\\
    M &= \text{the total number of objectives}. \nonumber
\end{align}

Since the design space for this problem is known \textit{a priori}, the results
from this experiment are simply the non-dominated set of \glspl{eg} which we
compare to the recommendations from the Nuclear Fuel Cycle Options Study
\cite{wigeland_nuclear_2014-2}.

% Include also brief introductions to both Temo and the SET tool, as well as an
% explanation of how you set up the verification of Osier's suitability. 

% You do not have to go into incredible detail here. Generally it's ok to just
% provide enough information for folks to replicate the work and get the same
% conclusions you did. 

% \subsection{Description of Test Model}




% Recommend including key features of the tools and models, input data, and
% comparison data that are part of the work.
