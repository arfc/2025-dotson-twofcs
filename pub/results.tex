\section{Results}

Here we discuss the results from the two experiments described previously.
% Start with the critical figures from Chapter 5 of your thesis and build up
% just enough detail around them to support the conclusions that you draw. While
% you want to keep this level of detail fairly minimal, it must inlude at least
% a paragraph for each experiment of the following form (the below is just an
% example):

\subsection{Comparing \gls{osier} and \gls{temoa}}

Figure \ref{fig:osier-temoa-benchmark} shows that \gls{osier} achieved solutions
within 0.5\%  of those achieved by \gls{temoa}. This level of agreement is well
within the uncertainty expected in an energy system model and \gls{osier} can
therefore be considered in agreement with \gls{temoa}. This clearly indicates
that \gls{osier}'s multi-objective optimization can be reduced to
single-objective optimization at the level of detail common in existing
\glspl{esom}.

\begin{figure}[ht!]
    \begin{center}
        \resizebox{\columnwidth}{!}{%% Creator: Matplotlib, PGF backend
%%
%% To include the figure in your LaTeX document, write
%%   \input{<filename>.pgf}
%%
%% Make sure the required packages are loaded in your preamble
%%   \usepackage{pgf}
%%
%% Also ensure that all the required font packages are loaded; for instance,
%% the lmodern package is sometimes necessary when using math font.
%%   \usepackage{lmodern}
%%
%% Figures using additional raster images can only be included by \input if
%% they are in the same directory as the main LaTeX file. For loading figures
%% from other directories you can use the `import` package
%%   \usepackage{import}
%%
%% and then include the figures with
%%   \import{<path to file>}{<filename>.pgf}
%%
%% Matplotlib used the following preamble
%%   \def\mathdefault#1{#1}
%%   \everymath=\expandafter{\the\everymath\displaystyle}
%%   \IfFileExists{scrextend.sty}{
%%     \usepackage[fontsize=10.000000pt]{scrextend}
%%   }{
%%     \renewcommand{\normalsize}{\fontsize{10.000000}{12.000000}\selectfont}
%%     \normalsize
%%   }
%%   
%%   \makeatletter\@ifpackageloaded{underscore}{}{\usepackage[strings]{underscore}}\makeatother
%%
\begingroup%
\makeatletter%
\begin{pgfpicture}%
\pgfpathrectangle{\pgfpointorigin}{\pgfqpoint{7.900000in}{5.899894in}}%
\pgfusepath{use as bounding box, clip}%
\begin{pgfscope}%
\pgfsetbuttcap%
\pgfsetmiterjoin%
\definecolor{currentfill}{rgb}{1.000000,1.000000,1.000000}%
\pgfsetfillcolor{currentfill}%
\pgfsetlinewidth{0.000000pt}%
\definecolor{currentstroke}{rgb}{0.000000,0.000000,0.000000}%
\pgfsetstrokecolor{currentstroke}%
\pgfsetdash{}{0pt}%
\pgfpathmoveto{\pgfqpoint{0.000000in}{0.000000in}}%
\pgfpathlineto{\pgfqpoint{7.900000in}{0.000000in}}%
\pgfpathlineto{\pgfqpoint{7.900000in}{5.899894in}}%
\pgfpathlineto{\pgfqpoint{0.000000in}{5.899894in}}%
\pgfpathlineto{\pgfqpoint{0.000000in}{0.000000in}}%
\pgfpathclose%
\pgfusepath{fill}%
\end{pgfscope}%
\begin{pgfscope}%
\pgfsetbuttcap%
\pgfsetmiterjoin%
\definecolor{currentfill}{rgb}{1.000000,1.000000,1.000000}%
\pgfsetfillcolor{currentfill}%
\pgfsetlinewidth{0.000000pt}%
\definecolor{currentstroke}{rgb}{0.000000,0.000000,0.000000}%
\pgfsetstrokecolor{currentstroke}%
\pgfsetstrokeopacity{0.000000}%
\pgfsetdash{}{0pt}%
\pgfpathmoveto{\pgfqpoint{0.688192in}{0.670138in}}%
\pgfpathlineto{\pgfqpoint{7.800000in}{0.670138in}}%
\pgfpathlineto{\pgfqpoint{7.800000in}{5.731668in}}%
\pgfpathlineto{\pgfqpoint{0.688192in}{5.731668in}}%
\pgfpathlineto{\pgfqpoint{0.688192in}{0.670138in}}%
\pgfpathclose%
\pgfusepath{fill}%
\end{pgfscope}%
\begin{pgfscope}%
\pgfpathrectangle{\pgfqpoint{0.688192in}{0.670138in}}{\pgfqpoint{7.111808in}{5.061530in}}%
\pgfusepath{clip}%
\pgfsetbuttcap%
\pgfsetmiterjoin%
\definecolor{currentfill}{rgb}{0.501961,0.501961,0.501961}%
\pgfsetfillcolor{currentfill}%
\pgfsetfillopacity{0.500000}%
\pgfsetlinewidth{1.003750pt}%
\definecolor{currentstroke}{rgb}{0.501961,0.501961,0.501961}%
\pgfsetstrokecolor{currentstroke}%
\pgfsetstrokeopacity{0.500000}%
\pgfsetdash{}{0pt}%
\pgfpathmoveto{\pgfqpoint{0.749254in}{1.445149in}}%
\pgfpathlineto{\pgfqpoint{0.752983in}{1.252761in}}%
\pgfpathlineto{\pgfqpoint{0.754090in}{1.154399in}}%
\pgfpathlineto{\pgfqpoint{0.760839in}{1.136279in}}%
\pgfpathlineto{\pgfqpoint{0.764625in}{1.015211in}}%
\pgfpathlineto{\pgfqpoint{0.769073in}{0.982259in}}%
\pgfpathlineto{\pgfqpoint{0.771295in}{0.982065in}}%
\pgfpathlineto{\pgfqpoint{0.776750in}{0.941782in}}%
\pgfpathlineto{\pgfqpoint{0.777016in}{0.919143in}}%
\pgfpathlineto{\pgfqpoint{0.779964in}{0.894614in}}%
\pgfpathlineto{\pgfqpoint{0.781884in}{0.888672in}}%
\pgfpathlineto{\pgfqpoint{0.782833in}{0.883385in}}%
\pgfpathlineto{\pgfqpoint{0.782912in}{0.882838in}}%
\pgfpathlineto{\pgfqpoint{0.783279in}{0.849857in}}%
\pgfpathlineto{\pgfqpoint{0.783279in}{0.849857in}}%
\pgfpathlineto{\pgfqpoint{0.787475in}{0.843942in}}%
\pgfpathlineto{\pgfqpoint{0.791486in}{0.837055in}}%
\pgfpathlineto{\pgfqpoint{0.793302in}{0.809599in}}%
\pgfpathlineto{\pgfqpoint{0.793310in}{0.809534in}}%
\pgfpathlineto{\pgfqpoint{0.795978in}{0.801029in}}%
\pgfpathlineto{\pgfqpoint{0.796157in}{0.793846in}}%
\pgfpathlineto{\pgfqpoint{0.800976in}{0.790369in}}%
\pgfpathlineto{\pgfqpoint{0.807883in}{0.789563in}}%
\pgfpathlineto{\pgfqpoint{0.810327in}{0.766458in}}%
\pgfpathlineto{\pgfqpoint{0.816256in}{0.760843in}}%
\pgfpathlineto{\pgfqpoint{0.827191in}{0.759800in}}%
\pgfpathlineto{\pgfqpoint{0.830930in}{0.751031in}}%
\pgfpathlineto{\pgfqpoint{0.834764in}{0.749497in}}%
\pgfpathlineto{\pgfqpoint{0.837359in}{0.748345in}}%
\pgfpathlineto{\pgfqpoint{0.842310in}{0.745304in}}%
\pgfpathlineto{\pgfqpoint{0.842696in}{0.745162in}}%
\pgfpathlineto{\pgfqpoint{0.842811in}{0.745008in}}%
\pgfpathlineto{\pgfqpoint{0.849436in}{0.740184in}}%
\pgfpathlineto{\pgfqpoint{0.849953in}{0.739357in}}%
\pgfpathlineto{\pgfqpoint{0.850270in}{0.738398in}}%
\pgfpathlineto{\pgfqpoint{0.854347in}{0.736661in}}%
\pgfpathlineto{\pgfqpoint{0.872029in}{0.731700in}}%
\pgfpathlineto{\pgfqpoint{0.890835in}{0.724786in}}%
\pgfpathlineto{\pgfqpoint{0.916357in}{0.719853in}}%
\pgfpathlineto{\pgfqpoint{0.939729in}{0.718197in}}%
\pgfpathlineto{\pgfqpoint{0.940682in}{0.716009in}}%
\pgfpathlineto{\pgfqpoint{0.942062in}{0.712194in}}%
\pgfpathlineto{\pgfqpoint{0.945316in}{0.711471in}}%
\pgfpathlineto{\pgfqpoint{0.945763in}{0.710834in}}%
\pgfpathlineto{\pgfqpoint{0.946217in}{0.710820in}}%
\pgfpathlineto{\pgfqpoint{0.954596in}{0.710219in}}%
\pgfpathlineto{\pgfqpoint{0.958858in}{0.709014in}}%
\pgfpathlineto{\pgfqpoint{0.960596in}{0.708592in}}%
\pgfpathlineto{\pgfqpoint{0.987244in}{0.706065in}}%
\pgfpathlineto{\pgfqpoint{0.990735in}{0.704724in}}%
\pgfpathlineto{\pgfqpoint{0.990837in}{0.703282in}}%
\pgfpathlineto{\pgfqpoint{0.996565in}{0.703044in}}%
\pgfpathlineto{\pgfqpoint{1.010100in}{0.700622in}}%
\pgfpathlineto{\pgfqpoint{1.011096in}{0.700386in}}%
\pgfpathlineto{\pgfqpoint{1.016505in}{0.700376in}}%
\pgfpathlineto{\pgfqpoint{1.017496in}{0.699473in}}%
\pgfpathlineto{\pgfqpoint{1.070825in}{0.697151in}}%
\pgfpathlineto{\pgfqpoint{1.075000in}{0.692618in}}%
\pgfpathlineto{\pgfqpoint{1.077361in}{0.692269in}}%
\pgfpathlineto{\pgfqpoint{1.077372in}{0.692269in}}%
\pgfpathlineto{\pgfqpoint{1.084164in}{0.691950in}}%
\pgfpathlineto{\pgfqpoint{1.085816in}{0.691662in}}%
\pgfpathlineto{\pgfqpoint{1.102529in}{0.689352in}}%
\pgfpathlineto{\pgfqpoint{1.103303in}{0.689151in}}%
\pgfpathlineto{\pgfqpoint{1.108579in}{0.688860in}}%
\pgfpathlineto{\pgfqpoint{1.108617in}{0.688763in}}%
\pgfpathlineto{\pgfqpoint{1.109998in}{0.688225in}}%
\pgfpathlineto{\pgfqpoint{1.114241in}{0.688028in}}%
\pgfpathlineto{\pgfqpoint{1.116764in}{0.688006in}}%
\pgfpathlineto{\pgfqpoint{1.126908in}{0.687169in}}%
\pgfpathlineto{\pgfqpoint{1.128607in}{0.686506in}}%
\pgfpathlineto{\pgfqpoint{1.128867in}{0.686499in}}%
\pgfpathlineto{\pgfqpoint{1.147700in}{0.686452in}}%
\pgfpathlineto{\pgfqpoint{1.148995in}{0.685892in}}%
\pgfpathlineto{\pgfqpoint{1.151625in}{0.685778in}}%
\pgfpathlineto{\pgfqpoint{1.159390in}{0.685625in}}%
\pgfpathlineto{\pgfqpoint{1.164071in}{0.685586in}}%
\pgfpathlineto{\pgfqpoint{1.178009in}{0.685447in}}%
\pgfpathlineto{\pgfqpoint{1.197079in}{0.685403in}}%
\pgfpathlineto{\pgfqpoint{1.198965in}{0.685402in}}%
\pgfpathlineto{\pgfqpoint{1.200157in}{0.685395in}}%
\pgfpathlineto{\pgfqpoint{1.220931in}{0.685389in}}%
\pgfpathlineto{\pgfqpoint{1.315614in}{0.684844in}}%
\pgfpathlineto{\pgfqpoint{1.334128in}{0.684517in}}%
\pgfpathlineto{\pgfqpoint{1.363285in}{0.684419in}}%
\pgfpathlineto{\pgfqpoint{1.364139in}{0.684404in}}%
\pgfpathlineto{\pgfqpoint{1.376266in}{0.684291in}}%
\pgfpathlineto{\pgfqpoint{1.396594in}{0.684212in}}%
\pgfpathlineto{\pgfqpoint{1.399923in}{0.684150in}}%
\pgfpathlineto{\pgfqpoint{1.407033in}{0.684070in}}%
\pgfpathlineto{\pgfqpoint{1.444241in}{0.684035in}}%
\pgfpathlineto{\pgfqpoint{1.455660in}{0.683986in}}%
\pgfpathlineto{\pgfqpoint{1.469414in}{0.683962in}}%
\pgfpathlineto{\pgfqpoint{1.472622in}{0.683959in}}%
\pgfpathlineto{\pgfqpoint{1.473551in}{0.683948in}}%
\pgfpathlineto{\pgfqpoint{1.473926in}{0.683941in}}%
\pgfpathlineto{\pgfqpoint{1.474245in}{0.683935in}}%
\pgfpathlineto{\pgfqpoint{1.475910in}{0.683912in}}%
\pgfpathlineto{\pgfqpoint{1.477331in}{0.683875in}}%
\pgfpathlineto{\pgfqpoint{1.495917in}{0.683547in}}%
\pgfpathlineto{\pgfqpoint{1.503177in}{0.683449in}}%
\pgfpathlineto{\pgfqpoint{1.582513in}{0.683280in}}%
\pgfpathlineto{\pgfqpoint{1.658826in}{0.683277in}}%
\pgfpathlineto{\pgfqpoint{1.659921in}{0.683127in}}%
\pgfpathlineto{\pgfqpoint{1.663205in}{0.683065in}}%
\pgfpathlineto{\pgfqpoint{1.675058in}{0.682998in}}%
\pgfpathlineto{\pgfqpoint{1.678208in}{0.682876in}}%
\pgfpathlineto{\pgfqpoint{1.681839in}{0.682826in}}%
\pgfpathlineto{\pgfqpoint{1.684802in}{0.682783in}}%
\pgfpathlineto{\pgfqpoint{1.719213in}{0.682724in}}%
\pgfpathlineto{\pgfqpoint{1.731063in}{0.682549in}}%
\pgfpathlineto{\pgfqpoint{1.731531in}{0.682540in}}%
\pgfpathlineto{\pgfqpoint{1.731531in}{0.682540in}}%
\pgfpathlineto{\pgfqpoint{1.731710in}{0.682537in}}%
\pgfpathlineto{\pgfqpoint{1.732115in}{0.682529in}}%
\pgfpathlineto{\pgfqpoint{1.738010in}{0.682319in}}%
\pgfpathlineto{\pgfqpoint{1.738010in}{0.682319in}}%
\pgfpathlineto{\pgfqpoint{1.755173in}{0.682260in}}%
\pgfpathlineto{\pgfqpoint{1.759976in}{0.682015in}}%
\pgfpathlineto{\pgfqpoint{1.761288in}{0.681989in}}%
\pgfpathlineto{\pgfqpoint{1.765345in}{0.681943in}}%
\pgfpathlineto{\pgfqpoint{1.811432in}{0.681927in}}%
\pgfpathlineto{\pgfqpoint{1.811432in}{0.681927in}}%
\pgfpathlineto{\pgfqpoint{1.815579in}{0.681901in}}%
\pgfpathlineto{\pgfqpoint{1.826952in}{0.681747in}}%
\pgfpathlineto{\pgfqpoint{1.852021in}{0.681611in}}%
\pgfpathlineto{\pgfqpoint{1.892079in}{0.681466in}}%
\pgfpathlineto{\pgfqpoint{1.903233in}{0.681282in}}%
\pgfpathlineto{\pgfqpoint{1.928513in}{0.681173in}}%
\pgfpathlineto{\pgfqpoint{1.967310in}{0.681083in}}%
\pgfpathlineto{\pgfqpoint{1.970905in}{0.680890in}}%
\pgfpathlineto{\pgfqpoint{2.063345in}{0.680516in}}%
\pgfpathlineto{\pgfqpoint{2.225009in}{0.680020in}}%
\pgfpathlineto{\pgfqpoint{2.235916in}{0.679656in}}%
\pgfpathlineto{\pgfqpoint{2.237139in}{0.679645in}}%
\pgfpathlineto{\pgfqpoint{2.243733in}{0.679632in}}%
\pgfpathlineto{\pgfqpoint{2.244939in}{0.679615in}}%
\pgfpathlineto{\pgfqpoint{2.265105in}{0.679558in}}%
\pgfpathlineto{\pgfqpoint{2.378269in}{0.679052in}}%
\pgfpathlineto{\pgfqpoint{2.392882in}{0.678851in}}%
\pgfpathlineto{\pgfqpoint{2.429320in}{0.678525in}}%
\pgfpathlineto{\pgfqpoint{2.452500in}{0.678477in}}%
\pgfpathlineto{\pgfqpoint{2.489265in}{0.678468in}}%
\pgfpathlineto{\pgfqpoint{2.494145in}{0.678397in}}%
\pgfpathlineto{\pgfqpoint{2.509817in}{0.678139in}}%
\pgfpathlineto{\pgfqpoint{2.510252in}{0.678129in}}%
\pgfpathlineto{\pgfqpoint{2.542698in}{0.677943in}}%
\pgfpathlineto{\pgfqpoint{2.608500in}{0.677847in}}%
\pgfpathlineto{\pgfqpoint{2.672675in}{0.677398in}}%
\pgfpathlineto{\pgfqpoint{2.677313in}{0.677350in}}%
\pgfpathlineto{\pgfqpoint{2.911373in}{0.676955in}}%
\pgfpathlineto{\pgfqpoint{2.932466in}{0.676438in}}%
\pgfpathlineto{\pgfqpoint{2.976355in}{0.676171in}}%
\pgfpathlineto{\pgfqpoint{3.024552in}{0.675964in}}%
\pgfpathlineto{\pgfqpoint{3.280946in}{0.675128in}}%
\pgfpathlineto{\pgfqpoint{3.291859in}{0.674921in}}%
\pgfpathlineto{\pgfqpoint{3.560021in}{0.673854in}}%
\pgfpathlineto{\pgfqpoint{3.790533in}{0.673411in}}%
\pgfpathlineto{\pgfqpoint{3.825636in}{0.673260in}}%
\pgfpathlineto{\pgfqpoint{3.863667in}{0.673125in}}%
\pgfpathlineto{\pgfqpoint{3.986933in}{0.672748in}}%
\pgfpathlineto{\pgfqpoint{3.995987in}{0.672712in}}%
\pgfpathlineto{\pgfqpoint{4.155194in}{0.672354in}}%
\pgfpathlineto{\pgfqpoint{4.207141in}{0.672282in}}%
\pgfpathlineto{\pgfqpoint{4.213238in}{0.672266in}}%
\pgfpathlineto{\pgfqpoint{4.420707in}{0.672256in}}%
\pgfpathlineto{\pgfqpoint{4.421026in}{0.672255in}}%
\pgfpathlineto{\pgfqpoint{4.491507in}{0.672222in}}%
\pgfpathlineto{\pgfqpoint{4.576929in}{0.672144in}}%
\pgfpathlineto{\pgfqpoint{4.613721in}{0.672131in}}%
\pgfpathlineto{\pgfqpoint{4.631124in}{0.671946in}}%
\pgfpathlineto{\pgfqpoint{4.661935in}{0.671930in}}%
\pgfpathlineto{\pgfqpoint{4.688548in}{0.671723in}}%
\pgfpathlineto{\pgfqpoint{4.726657in}{0.671660in}}%
\pgfpathlineto{\pgfqpoint{4.735203in}{0.671563in}}%
\pgfpathlineto{\pgfqpoint{4.750941in}{0.671524in}}%
\pgfpathlineto{\pgfqpoint{4.764231in}{0.671484in}}%
\pgfpathlineto{\pgfqpoint{4.914508in}{0.671444in}}%
\pgfpathlineto{\pgfqpoint{4.924693in}{0.671413in}}%
\pgfpathlineto{\pgfqpoint{4.939656in}{0.671407in}}%
\pgfpathlineto{\pgfqpoint{4.965126in}{0.671366in}}%
\pgfpathlineto{\pgfqpoint{4.979341in}{0.671320in}}%
\pgfpathlineto{\pgfqpoint{4.994709in}{0.671264in}}%
\pgfpathlineto{\pgfqpoint{5.085461in}{0.671155in}}%
\pgfpathlineto{\pgfqpoint{5.135185in}{0.671116in}}%
\pgfpathlineto{\pgfqpoint{5.140992in}{0.670993in}}%
\pgfpathlineto{\pgfqpoint{5.159789in}{0.670855in}}%
\pgfpathlineto{\pgfqpoint{5.274608in}{0.670832in}}%
\pgfpathlineto{\pgfqpoint{5.274608in}{0.670832in}}%
\pgfpathlineto{\pgfqpoint{5.283066in}{0.670786in}}%
\pgfpathlineto{\pgfqpoint{5.359877in}{0.670710in}}%
\pgfpathlineto{\pgfqpoint{5.370884in}{0.670626in}}%
\pgfpathlineto{\pgfqpoint{5.527297in}{0.670601in}}%
\pgfpathlineto{\pgfqpoint{5.557238in}{0.670526in}}%
\pgfpathlineto{\pgfqpoint{5.614292in}{0.670505in}}%
\pgfpathlineto{\pgfqpoint{5.657223in}{0.670449in}}%
\pgfpathlineto{\pgfqpoint{5.714356in}{0.670428in}}%
\pgfpathlineto{\pgfqpoint{5.840166in}{0.670247in}}%
\pgfpathlineto{\pgfqpoint{6.004393in}{0.670138in}}%
\pgfpathlineto{\pgfqpoint{7.800000in}{0.684746in}}%
\pgfpathlineto{\pgfqpoint{7.619350in}{0.684865in}}%
\pgfpathlineto{\pgfqpoint{7.480959in}{0.685064in}}%
\pgfpathlineto{\pgfqpoint{7.418113in}{0.685088in}}%
\pgfpathlineto{\pgfqpoint{7.370888in}{0.685150in}}%
\pgfpathlineto{\pgfqpoint{7.308129in}{0.685172in}}%
\pgfpathlineto{\pgfqpoint{7.275194in}{0.685255in}}%
\pgfpathlineto{\pgfqpoint{7.103140in}{0.685283in}}%
\pgfpathlineto{\pgfqpoint{7.091032in}{0.685375in}}%
\pgfpathlineto{\pgfqpoint{7.006540in}{0.685458in}}%
\pgfpathlineto{\pgfqpoint{6.997236in}{0.685509in}}%
\pgfpathlineto{\pgfqpoint{6.997236in}{0.685509in}}%
\pgfpathlineto{\pgfqpoint{6.870936in}{0.685535in}}%
\pgfpathlineto{\pgfqpoint{6.850259in}{0.685685in}}%
\pgfpathlineto{\pgfqpoint{6.843871in}{0.685821in}}%
\pgfpathlineto{\pgfqpoint{6.789175in}{0.685864in}}%
\pgfpathlineto{\pgfqpoint{6.689348in}{0.685984in}}%
\pgfpathlineto{\pgfqpoint{6.672443in}{0.686045in}}%
\pgfpathlineto{\pgfqpoint{6.656806in}{0.686097in}}%
\pgfpathlineto{\pgfqpoint{6.628789in}{0.686141in}}%
\pgfpathlineto{\pgfqpoint{6.612329in}{0.686148in}}%
\pgfpathlineto{\pgfqpoint{6.601126in}{0.686182in}}%
\pgfpathlineto{\pgfqpoint{6.435822in}{0.686226in}}%
\pgfpathlineto{\pgfqpoint{6.421203in}{0.686270in}}%
\pgfpathlineto{\pgfqpoint{6.403891in}{0.686313in}}%
\pgfpathlineto{\pgfqpoint{6.394490in}{0.686420in}}%
\pgfpathlineto{\pgfqpoint{6.352570in}{0.686489in}}%
\pgfpathlineto{\pgfqpoint{6.323295in}{0.686716in}}%
\pgfpathlineto{\pgfqpoint{6.289404in}{0.686734in}}%
\pgfpathlineto{\pgfqpoint{6.270261in}{0.686938in}}%
\pgfpathlineto{\pgfqpoint{6.229790in}{0.686952in}}%
\pgfpathlineto{\pgfqpoint{6.135825in}{0.687037in}}%
\pgfpathlineto{\pgfqpoint{6.058296in}{0.687074in}}%
\pgfpathlineto{\pgfqpoint{6.057945in}{0.687076in}}%
\pgfpathlineto{\pgfqpoint{5.829729in}{0.687086in}}%
\pgfpathlineto{\pgfqpoint{5.823022in}{0.687104in}}%
\pgfpathlineto{\pgfqpoint{5.765880in}{0.687183in}}%
\pgfpathlineto{\pgfqpoint{5.590753in}{0.687577in}}%
\pgfpathlineto{\pgfqpoint{5.580794in}{0.687616in}}%
\pgfpathlineto{\pgfqpoint{5.445201in}{0.688032in}}%
\pgfpathlineto{\pgfqpoint{5.403367in}{0.688180in}}%
\pgfpathlineto{\pgfqpoint{5.364754in}{0.688346in}}%
\pgfpathlineto{\pgfqpoint{5.111190in}{0.688834in}}%
\pgfpathlineto{\pgfqpoint{4.816213in}{0.690007in}}%
\pgfpathlineto{\pgfqpoint{4.804209in}{0.690235in}}%
\pgfpathlineto{\pgfqpoint{4.522174in}{0.691154in}}%
\pgfpathlineto{\pgfqpoint{4.469159in}{0.691382in}}%
\pgfpathlineto{\pgfqpoint{4.420880in}{0.691675in}}%
\pgfpathlineto{\pgfqpoint{4.397678in}{0.692245in}}%
\pgfpathlineto{\pgfqpoint{4.140212in}{0.692679in}}%
\pgfpathlineto{\pgfqpoint{4.135110in}{0.692731in}}%
\pgfpathlineto{\pgfqpoint{4.064518in}{0.693225in}}%
\pgfpathlineto{\pgfqpoint{3.992135in}{0.693331in}}%
\pgfpathlineto{\pgfqpoint{3.956444in}{0.693536in}}%
\pgfpathlineto{\pgfqpoint{3.955967in}{0.693546in}}%
\pgfpathlineto{\pgfqpoint{3.938727in}{0.693830in}}%
\pgfpathlineto{\pgfqpoint{3.933359in}{0.693909in}}%
\pgfpathlineto{\pgfqpoint{3.892918in}{0.693918in}}%
\pgfpathlineto{\pgfqpoint{3.867419in}{0.693971in}}%
\pgfpathlineto{\pgfqpoint{3.827337in}{0.694330in}}%
\pgfpathlineto{\pgfqpoint{3.811263in}{0.694551in}}%
\pgfpathlineto{\pgfqpoint{3.686783in}{0.695107in}}%
\pgfpathlineto{\pgfqpoint{3.664601in}{0.695170in}}%
\pgfpathlineto{\pgfqpoint{3.663274in}{0.695189in}}%
\pgfpathlineto{\pgfqpoint{3.656021in}{0.695204in}}%
\pgfpathlineto{\pgfqpoint{3.654675in}{0.695216in}}%
\pgfpathlineto{\pgfqpoint{3.642678in}{0.695616in}}%
\pgfpathlineto{\pgfqpoint{3.464847in}{0.696161in}}%
\pgfpathlineto{\pgfqpoint{3.363163in}{0.696573in}}%
\pgfpathlineto{\pgfqpoint{3.359209in}{0.696784in}}%
\pgfpathlineto{\pgfqpoint{3.316532in}{0.696883in}}%
\pgfpathlineto{\pgfqpoint{3.288724in}{0.697004in}}%
\pgfpathlineto{\pgfqpoint{3.276454in}{0.697206in}}%
\pgfpathlineto{\pgfqpoint{3.232391in}{0.697366in}}%
\pgfpathlineto{\pgfqpoint{3.204814in}{0.697516in}}%
\pgfpathlineto{\pgfqpoint{3.192305in}{0.697685in}}%
\pgfpathlineto{\pgfqpoint{3.187743in}{0.697713in}}%
\pgfpathlineto{\pgfqpoint{3.187743in}{0.697713in}}%
\pgfpathlineto{\pgfqpoint{3.137047in}{0.697731in}}%
\pgfpathlineto{\pgfqpoint{3.132584in}{0.697781in}}%
\pgfpathlineto{\pgfqpoint{3.131141in}{0.697810in}}%
\pgfpathlineto{\pgfqpoint{3.125858in}{0.698080in}}%
\pgfpathlineto{\pgfqpoint{3.106979in}{0.698145in}}%
\pgfpathlineto{\pgfqpoint{3.106979in}{0.698145in}}%
\pgfpathlineto{\pgfqpoint{3.100494in}{0.698376in}}%
\pgfpathlineto{\pgfqpoint{3.100048in}{0.698385in}}%
\pgfpathlineto{\pgfqpoint{3.099852in}{0.698388in}}%
\pgfpathlineto{\pgfqpoint{3.099852in}{0.698388in}}%
\pgfpathlineto{\pgfqpoint{3.099336in}{0.698397in}}%
\pgfpathlineto{\pgfqpoint{3.086302in}{0.698590in}}%
\pgfpathlineto{\pgfqpoint{3.048450in}{0.698655in}}%
\pgfpathlineto{\pgfqpoint{3.045190in}{0.698703in}}%
\pgfpathlineto{\pgfqpoint{3.041196in}{0.698758in}}%
\pgfpathlineto{\pgfqpoint{3.037732in}{0.698892in}}%
\pgfpathlineto{\pgfqpoint{3.024693in}{0.698965in}}%
\pgfpathlineto{\pgfqpoint{3.021080in}{0.699034in}}%
\pgfpathlineto{\pgfqpoint{3.019876in}{0.699198in}}%
\pgfpathlineto{\pgfqpoint{2.935932in}{0.699202in}}%
\pgfpathlineto{\pgfqpoint{2.848662in}{0.699388in}}%
\pgfpathlineto{\pgfqpoint{2.840676in}{0.699496in}}%
\pgfpathlineto{\pgfqpoint{2.820232in}{0.699856in}}%
\pgfpathlineto{\pgfqpoint{2.818668in}{0.699897in}}%
\pgfpathlineto{\pgfqpoint{2.816837in}{0.699922in}}%
\pgfpathlineto{\pgfqpoint{2.816486in}{0.699929in}}%
\pgfpathlineto{\pgfqpoint{2.816074in}{0.699936in}}%
\pgfpathlineto{\pgfqpoint{2.815052in}{0.699949in}}%
\pgfpathlineto{\pgfqpoint{2.811523in}{0.699951in}}%
\pgfpathlineto{\pgfqpoint{2.796394in}{0.699978in}}%
\pgfpathlineto{\pgfqpoint{2.783832in}{0.700032in}}%
\pgfpathlineto{\pgfqpoint{2.742903in}{0.700071in}}%
\pgfpathlineto{\pgfqpoint{2.735082in}{0.700159in}}%
\pgfpathlineto{\pgfqpoint{2.731421in}{0.700227in}}%
\pgfpathlineto{\pgfqpoint{2.709060in}{0.700313in}}%
\pgfpathlineto{\pgfqpoint{2.695721in}{0.700438in}}%
\pgfpathlineto{\pgfqpoint{2.694781in}{0.700454in}}%
\pgfpathlineto{\pgfqpoint{2.662708in}{0.700563in}}%
\pgfpathlineto{\pgfqpoint{2.642343in}{0.700922in}}%
\pgfpathlineto{\pgfqpoint{2.538192in}{0.701522in}}%
\pgfpathlineto{\pgfqpoint{2.515340in}{0.701528in}}%
\pgfpathlineto{\pgfqpoint{2.514029in}{0.701536in}}%
\pgfpathlineto{\pgfqpoint{2.511954in}{0.701536in}}%
\pgfpathlineto{\pgfqpoint{2.490977in}{0.701585in}}%
\pgfpathlineto{\pgfqpoint{2.475646in}{0.701738in}}%
\pgfpathlineto{\pgfqpoint{2.470497in}{0.701782in}}%
\pgfpathlineto{\pgfqpoint{2.461955in}{0.701949in}}%
\pgfpathlineto{\pgfqpoint{2.459062in}{0.702074in}}%
\pgfpathlineto{\pgfqpoint{2.457637in}{0.702691in}}%
\pgfpathlineto{\pgfqpoint{2.436921in}{0.702743in}}%
\pgfpathlineto{\pgfqpoint{2.436635in}{0.702750in}}%
\pgfpathlineto{\pgfqpoint{2.434766in}{0.703479in}}%
\pgfpathlineto{\pgfqpoint{2.423608in}{0.704401in}}%
\pgfpathlineto{\pgfqpoint{2.420833in}{0.704424in}}%
\pgfpathlineto{\pgfqpoint{2.416166in}{0.704641in}}%
\pgfpathlineto{\pgfqpoint{2.414647in}{0.705233in}}%
\pgfpathlineto{\pgfqpoint{2.414604in}{0.705339in}}%
\pgfpathlineto{\pgfqpoint{2.408800in}{0.705660in}}%
\pgfpathlineto{\pgfqpoint{2.407949in}{0.705880in}}%
\pgfpathlineto{\pgfqpoint{2.389566in}{0.708422in}}%
\pgfpathlineto{\pgfqpoint{2.387748in}{0.708739in}}%
\pgfpathlineto{\pgfqpoint{2.380277in}{0.709090in}}%
\pgfpathlineto{\pgfqpoint{2.380264in}{0.709090in}}%
\pgfpathlineto{\pgfqpoint{2.377668in}{0.709474in}}%
\pgfpathlineto{\pgfqpoint{2.373075in}{0.714460in}}%
\pgfpathlineto{\pgfqpoint{2.314413in}{0.717013in}}%
\pgfpathlineto{\pgfqpoint{2.313323in}{0.718007in}}%
\pgfpathlineto{\pgfqpoint{2.307373in}{0.718018in}}%
\pgfpathlineto{\pgfqpoint{2.306278in}{0.718278in}}%
\pgfpathlineto{\pgfqpoint{2.291389in}{0.720942in}}%
\pgfpathlineto{\pgfqpoint{2.285088in}{0.721204in}}%
\pgfpathlineto{\pgfqpoint{2.284976in}{0.722790in}}%
\pgfpathlineto{\pgfqpoint{2.281136in}{0.724265in}}%
\pgfpathlineto{\pgfqpoint{2.251823in}{0.727045in}}%
\pgfpathlineto{\pgfqpoint{2.249911in}{0.727509in}}%
\pgfpathlineto{\pgfqpoint{2.245223in}{0.728835in}}%
\pgfpathlineto{\pgfqpoint{2.236007in}{0.729495in}}%
\pgfpathlineto{\pgfqpoint{2.235507in}{0.729511in}}%
\pgfpathlineto{\pgfqpoint{2.235016in}{0.730212in}}%
\pgfpathlineto{\pgfqpoint{2.231436in}{0.731007in}}%
\pgfpathlineto{\pgfqpoint{2.229917in}{0.735204in}}%
\pgfpathlineto{\pgfqpoint{2.228869in}{0.737610in}}%
\pgfpathlineto{\pgfqpoint{2.203161in}{0.739432in}}%
\pgfpathlineto{\pgfqpoint{2.175086in}{0.744858in}}%
\pgfpathlineto{\pgfqpoint{2.154400in}{0.752463in}}%
\pgfpathlineto{\pgfqpoint{2.134949in}{0.757921in}}%
\pgfpathlineto{\pgfqpoint{2.130465in}{0.759831in}}%
\pgfpathlineto{\pgfqpoint{2.130116in}{0.760887in}}%
\pgfpathlineto{\pgfqpoint{2.129547in}{0.761796in}}%
\pgfpathlineto{\pgfqpoint{2.122260in}{0.767103in}}%
\pgfpathlineto{\pgfqpoint{2.122133in}{0.767272in}}%
\pgfpathlineto{\pgfqpoint{2.121708in}{0.767428in}}%
\pgfpathlineto{\pgfqpoint{2.116263in}{0.770773in}}%
\pgfpathlineto{\pgfqpoint{2.113408in}{0.772041in}}%
\pgfpathlineto{\pgfqpoint{2.109190in}{0.773728in}}%
\pgfpathlineto{\pgfqpoint{2.105078in}{0.783373in}}%
\pgfpathlineto{\pgfqpoint{2.093049in}{0.784521in}}%
\pgfpathlineto{\pgfqpoint{2.086528in}{0.790698in}}%
\pgfpathlineto{\pgfqpoint{2.083839in}{0.816113in}}%
\pgfpathlineto{\pgfqpoint{2.076241in}{0.816999in}}%
\pgfpathlineto{\pgfqpoint{2.070940in}{0.820824in}}%
\pgfpathlineto{\pgfqpoint{2.070744in}{0.828726in}}%
\pgfpathlineto{\pgfqpoint{2.067808in}{0.838081in}}%
\pgfpathlineto{\pgfqpoint{2.067800in}{0.838153in}}%
\pgfpathlineto{\pgfqpoint{2.065802in}{0.868354in}}%
\pgfpathlineto{\pgfqpoint{2.061390in}{0.875930in}}%
\pgfpathlineto{\pgfqpoint{2.056775in}{0.882436in}}%
\pgfpathlineto{\pgfqpoint{2.056775in}{0.882436in}}%
\pgfpathlineto{\pgfqpoint{2.056370in}{0.918715in}}%
\pgfpathlineto{\pgfqpoint{2.056284in}{0.919317in}}%
\pgfpathlineto{\pgfqpoint{2.055240in}{0.925133in}}%
\pgfpathlineto{\pgfqpoint{2.053128in}{0.931670in}}%
\pgfpathlineto{\pgfqpoint{2.049885in}{0.958651in}}%
\pgfpathlineto{\pgfqpoint{2.049592in}{0.983554in}}%
\pgfpathlineto{\pgfqpoint{2.043592in}{1.027865in}}%
\pgfpathlineto{\pgfqpoint{2.041148in}{1.028078in}}%
\pgfpathlineto{\pgfqpoint{2.036255in}{1.064326in}}%
\pgfpathlineto{\pgfqpoint{2.032091in}{1.197500in}}%
\pgfpathlineto{\pgfqpoint{2.024666in}{1.217432in}}%
\pgfpathlineto{\pgfqpoint{2.023448in}{1.325631in}}%
\pgfpathlineto{\pgfqpoint{2.019347in}{1.537258in}}%
\pgfpathlineto{\pgfqpoint{0.749254in}{1.445149in}}%
\pgfpathclose%
\pgfusepath{stroke,fill}%
\end{pgfscope}%
\begin{pgfscope}%
\pgfpathrectangle{\pgfqpoint{0.688192in}{0.670138in}}{\pgfqpoint{7.111808in}{5.061530in}}%
\pgfusepath{clip}%
\pgfsetbuttcap%
\pgfsetroundjoin%
\pgfsetlinewidth{1.003750pt}%
\definecolor{currentstroke}{rgb}{1.000000,0.000000,0.000000}%
\pgfsetstrokecolor{currentstroke}%
\pgfsetdash{}{0pt}%
\pgfpathmoveto{\pgfqpoint{1.482100in}{5.217166in}}%
\pgfpathcurveto{\pgfqpoint{1.492995in}{5.217166in}}{\pgfqpoint{1.503446in}{5.221495in}}{\pgfqpoint{1.511150in}{5.229199in}}%
\pgfpathcurveto{\pgfqpoint{1.518855in}{5.236903in}}{\pgfqpoint{1.523183in}{5.247354in}}{\pgfqpoint{1.523183in}{5.258250in}}%
\pgfpathcurveto{\pgfqpoint{1.523183in}{5.269145in}}{\pgfqpoint{1.518855in}{5.279596in}}{\pgfqpoint{1.511150in}{5.287300in}}%
\pgfpathcurveto{\pgfqpoint{1.503446in}{5.295005in}}{\pgfqpoint{1.492995in}{5.299334in}}{\pgfqpoint{1.482100in}{5.299334in}}%
\pgfpathcurveto{\pgfqpoint{1.471204in}{5.299334in}}{\pgfqpoint{1.460753in}{5.295005in}}{\pgfqpoint{1.453049in}{5.287300in}}%
\pgfpathcurveto{\pgfqpoint{1.445344in}{5.279596in}}{\pgfqpoint{1.441016in}{5.269145in}}{\pgfqpoint{1.441016in}{5.258250in}}%
\pgfpathcurveto{\pgfqpoint{1.441016in}{5.247354in}}{\pgfqpoint{1.445344in}{5.236903in}}{\pgfqpoint{1.453049in}{5.229199in}}%
\pgfpathcurveto{\pgfqpoint{1.460753in}{5.221495in}}{\pgfqpoint{1.471204in}{5.217166in}}{\pgfqpoint{1.482100in}{5.217166in}}%
\pgfpathlineto{\pgfqpoint{1.482100in}{5.217166in}}%
\pgfpathclose%
\pgfusepath{stroke}%
\end{pgfscope}%
\begin{pgfscope}%
\pgfpathrectangle{\pgfqpoint{0.688192in}{0.670138in}}{\pgfqpoint{7.111808in}{5.061530in}}%
\pgfusepath{clip}%
\pgfsetbuttcap%
\pgfsetroundjoin%
\pgfsetlinewidth{1.003750pt}%
\definecolor{currentstroke}{rgb}{1.000000,0.000000,0.000000}%
\pgfsetstrokecolor{currentstroke}%
\pgfsetdash{}{0pt}%
\pgfpathmoveto{\pgfqpoint{1.164895in}{2.382394in}}%
\pgfpathcurveto{\pgfqpoint{1.175790in}{2.382394in}}{\pgfqpoint{1.186241in}{2.386722in}}{\pgfqpoint{1.193945in}{2.394427in}}%
\pgfpathcurveto{\pgfqpoint{1.201650in}{2.402131in}}{\pgfqpoint{1.205978in}{2.412582in}}{\pgfqpoint{1.205978in}{2.423477in}}%
\pgfpathcurveto{\pgfqpoint{1.205978in}{2.434373in}}{\pgfqpoint{1.201650in}{2.444824in}}{\pgfqpoint{1.193945in}{2.452528in}}%
\pgfpathcurveto{\pgfqpoint{1.186241in}{2.460232in}}{\pgfqpoint{1.175790in}{2.464561in}}{\pgfqpoint{1.164895in}{2.464561in}}%
\pgfpathcurveto{\pgfqpoint{1.153999in}{2.464561in}}{\pgfqpoint{1.143548in}{2.460232in}}{\pgfqpoint{1.135844in}{2.452528in}}%
\pgfpathcurveto{\pgfqpoint{1.128140in}{2.444824in}}{\pgfqpoint{1.123811in}{2.434373in}}{\pgfqpoint{1.123811in}{2.423477in}}%
\pgfpathcurveto{\pgfqpoint{1.123811in}{2.412582in}}{\pgfqpoint{1.128140in}{2.402131in}}{\pgfqpoint{1.135844in}{2.394427in}}%
\pgfpathcurveto{\pgfqpoint{1.143548in}{2.386722in}}{\pgfqpoint{1.153999in}{2.382394in}}{\pgfqpoint{1.164895in}{2.382394in}}%
\pgfpathlineto{\pgfqpoint{1.164895in}{2.382394in}}%
\pgfpathclose%
\pgfusepath{stroke}%
\end{pgfscope}%
\begin{pgfscope}%
\pgfpathrectangle{\pgfqpoint{0.688192in}{0.670138in}}{\pgfqpoint{7.111808in}{5.061530in}}%
\pgfusepath{clip}%
\pgfsetbuttcap%
\pgfsetroundjoin%
\pgfsetlinewidth{1.003750pt}%
\definecolor{currentstroke}{rgb}{1.000000,0.000000,0.000000}%
\pgfsetstrokecolor{currentstroke}%
\pgfsetdash{}{0pt}%
\pgfpathmoveto{\pgfqpoint{1.213667in}{2.459288in}}%
\pgfpathcurveto{\pgfqpoint{1.224562in}{2.459288in}}{\pgfqpoint{1.235013in}{2.463617in}}{\pgfqpoint{1.242717in}{2.471321in}}%
\pgfpathcurveto{\pgfqpoint{1.250422in}{2.479025in}}{\pgfqpoint{1.254750in}{2.489476in}}{\pgfqpoint{1.254750in}{2.500372in}}%
\pgfpathcurveto{\pgfqpoint{1.254750in}{2.511267in}}{\pgfqpoint{1.250422in}{2.521718in}}{\pgfqpoint{1.242717in}{2.529423in}}%
\pgfpathcurveto{\pgfqpoint{1.235013in}{2.537127in}}{\pgfqpoint{1.224562in}{2.541456in}}{\pgfqpoint{1.213667in}{2.541456in}}%
\pgfpathcurveto{\pgfqpoint{1.202771in}{2.541456in}}{\pgfqpoint{1.192320in}{2.537127in}}{\pgfqpoint{1.184616in}{2.529423in}}%
\pgfpathcurveto{\pgfqpoint{1.176912in}{2.521718in}}{\pgfqpoint{1.172583in}{2.511267in}}{\pgfqpoint{1.172583in}{2.500372in}}%
\pgfpathcurveto{\pgfqpoint{1.172583in}{2.489476in}}{\pgfqpoint{1.176912in}{2.479025in}}{\pgfqpoint{1.184616in}{2.471321in}}%
\pgfpathcurveto{\pgfqpoint{1.192320in}{2.463617in}}{\pgfqpoint{1.202771in}{2.459288in}}{\pgfqpoint{1.213667in}{2.459288in}}%
\pgfpathlineto{\pgfqpoint{1.213667in}{2.459288in}}%
\pgfpathclose%
\pgfusepath{stroke}%
\end{pgfscope}%
\begin{pgfscope}%
\pgfpathrectangle{\pgfqpoint{0.688192in}{0.670138in}}{\pgfqpoint{7.111808in}{5.061530in}}%
\pgfusepath{clip}%
\pgfsetbuttcap%
\pgfsetroundjoin%
\pgfsetlinewidth{1.003750pt}%
\definecolor{currentstroke}{rgb}{1.000000,0.000000,0.000000}%
\pgfsetstrokecolor{currentstroke}%
\pgfsetdash{}{0pt}%
\pgfpathmoveto{\pgfqpoint{1.371591in}{2.489555in}}%
\pgfpathcurveto{\pgfqpoint{1.382486in}{2.489555in}}{\pgfqpoint{1.392937in}{2.493884in}}{\pgfqpoint{1.400641in}{2.501588in}}%
\pgfpathcurveto{\pgfqpoint{1.408346in}{2.509293in}}{\pgfqpoint{1.412674in}{2.519743in}}{\pgfqpoint{1.412674in}{2.530639in}}%
\pgfpathcurveto{\pgfqpoint{1.412674in}{2.541534in}}{\pgfqpoint{1.408346in}{2.551985in}}{\pgfqpoint{1.400641in}{2.559690in}}%
\pgfpathcurveto{\pgfqpoint{1.392937in}{2.567394in}}{\pgfqpoint{1.382486in}{2.571723in}}{\pgfqpoint{1.371591in}{2.571723in}}%
\pgfpathcurveto{\pgfqpoint{1.360695in}{2.571723in}}{\pgfqpoint{1.350244in}{2.567394in}}{\pgfqpoint{1.342540in}{2.559690in}}%
\pgfpathcurveto{\pgfqpoint{1.334835in}{2.551985in}}{\pgfqpoint{1.330507in}{2.541534in}}{\pgfqpoint{1.330507in}{2.530639in}}%
\pgfpathcurveto{\pgfqpoint{1.330507in}{2.519743in}}{\pgfqpoint{1.334835in}{2.509293in}}{\pgfqpoint{1.342540in}{2.501588in}}%
\pgfpathcurveto{\pgfqpoint{1.350244in}{2.493884in}}{\pgfqpoint{1.360695in}{2.489555in}}{\pgfqpoint{1.371591in}{2.489555in}}%
\pgfpathlineto{\pgfqpoint{1.371591in}{2.489555in}}%
\pgfpathclose%
\pgfusepath{stroke}%
\end{pgfscope}%
\begin{pgfscope}%
\pgfpathrectangle{\pgfqpoint{0.688192in}{0.670138in}}{\pgfqpoint{7.111808in}{5.061530in}}%
\pgfusepath{clip}%
\pgfsetbuttcap%
\pgfsetroundjoin%
\pgfsetlinewidth{1.003750pt}%
\definecolor{currentstroke}{rgb}{1.000000,0.000000,0.000000}%
\pgfsetstrokecolor{currentstroke}%
\pgfsetdash{}{0pt}%
\pgfpathmoveto{\pgfqpoint{1.268438in}{2.623039in}}%
\pgfpathcurveto{\pgfqpoint{1.279334in}{2.623039in}}{\pgfqpoint{1.289784in}{2.627368in}}{\pgfqpoint{1.297489in}{2.635072in}}%
\pgfpathcurveto{\pgfqpoint{1.305193in}{2.642777in}}{\pgfqpoint{1.309522in}{2.653227in}}{\pgfqpoint{1.309522in}{2.664123in}}%
\pgfpathcurveto{\pgfqpoint{1.309522in}{2.675018in}}{\pgfqpoint{1.305193in}{2.685469in}}{\pgfqpoint{1.297489in}{2.693174in}}%
\pgfpathcurveto{\pgfqpoint{1.289784in}{2.700878in}}{\pgfqpoint{1.279334in}{2.705207in}}{\pgfqpoint{1.268438in}{2.705207in}}%
\pgfpathcurveto{\pgfqpoint{1.257542in}{2.705207in}}{\pgfqpoint{1.247092in}{2.700878in}}{\pgfqpoint{1.239387in}{2.693174in}}%
\pgfpathcurveto{\pgfqpoint{1.231683in}{2.685469in}}{\pgfqpoint{1.227354in}{2.675018in}}{\pgfqpoint{1.227354in}{2.664123in}}%
\pgfpathcurveto{\pgfqpoint{1.227354in}{2.653227in}}{\pgfqpoint{1.231683in}{2.642777in}}{\pgfqpoint{1.239387in}{2.635072in}}%
\pgfpathcurveto{\pgfqpoint{1.247092in}{2.627368in}}{\pgfqpoint{1.257542in}{2.623039in}}{\pgfqpoint{1.268438in}{2.623039in}}%
\pgfpathlineto{\pgfqpoint{1.268438in}{2.623039in}}%
\pgfpathclose%
\pgfusepath{stroke}%
\end{pgfscope}%
\begin{pgfscope}%
\pgfpathrectangle{\pgfqpoint{0.688192in}{0.670138in}}{\pgfqpoint{7.111808in}{5.061530in}}%
\pgfusepath{clip}%
\pgfsetbuttcap%
\pgfsetroundjoin%
\pgfsetlinewidth{1.003750pt}%
\definecolor{currentstroke}{rgb}{1.000000,0.000000,0.000000}%
\pgfsetstrokecolor{currentstroke}%
\pgfsetdash{}{0pt}%
\pgfpathmoveto{\pgfqpoint{1.190566in}{2.163432in}}%
\pgfpathcurveto{\pgfqpoint{1.201461in}{2.163432in}}{\pgfqpoint{1.211912in}{2.167761in}}{\pgfqpoint{1.219616in}{2.175465in}}%
\pgfpathcurveto{\pgfqpoint{1.227321in}{2.183169in}}{\pgfqpoint{1.231649in}{2.193620in}}{\pgfqpoint{1.231649in}{2.204516in}}%
\pgfpathcurveto{\pgfqpoint{1.231649in}{2.215411in}}{\pgfqpoint{1.227321in}{2.225862in}}{\pgfqpoint{1.219616in}{2.233566in}}%
\pgfpathcurveto{\pgfqpoint{1.211912in}{2.241271in}}{\pgfqpoint{1.201461in}{2.245600in}}{\pgfqpoint{1.190566in}{2.245600in}}%
\pgfpathcurveto{\pgfqpoint{1.179670in}{2.245600in}}{\pgfqpoint{1.169219in}{2.241271in}}{\pgfqpoint{1.161515in}{2.233566in}}%
\pgfpathcurveto{\pgfqpoint{1.153811in}{2.225862in}}{\pgfqpoint{1.149482in}{2.215411in}}{\pgfqpoint{1.149482in}{2.204516in}}%
\pgfpathcurveto{\pgfqpoint{1.149482in}{2.193620in}}{\pgfqpoint{1.153811in}{2.183169in}}{\pgfqpoint{1.161515in}{2.175465in}}%
\pgfpathcurveto{\pgfqpoint{1.169219in}{2.167761in}}{\pgfqpoint{1.179670in}{2.163432in}}{\pgfqpoint{1.190566in}{2.163432in}}%
\pgfpathlineto{\pgfqpoint{1.190566in}{2.163432in}}%
\pgfpathclose%
\pgfusepath{stroke}%
\end{pgfscope}%
\begin{pgfscope}%
\pgfpathrectangle{\pgfqpoint{0.688192in}{0.670138in}}{\pgfqpoint{7.111808in}{5.061530in}}%
\pgfusepath{clip}%
\pgfsetbuttcap%
\pgfsetroundjoin%
\pgfsetlinewidth{1.003750pt}%
\definecolor{currentstroke}{rgb}{1.000000,0.000000,0.000000}%
\pgfsetstrokecolor{currentstroke}%
\pgfsetdash{}{0pt}%
\pgfpathmoveto{\pgfqpoint{1.130491in}{1.831301in}}%
\pgfpathcurveto{\pgfqpoint{1.141386in}{1.831301in}}{\pgfqpoint{1.151837in}{1.835630in}}{\pgfqpoint{1.159542in}{1.843334in}}%
\pgfpathcurveto{\pgfqpoint{1.167246in}{1.851038in}}{\pgfqpoint{1.171575in}{1.861489in}}{\pgfqpoint{1.171575in}{1.872385in}}%
\pgfpathcurveto{\pgfqpoint{1.171575in}{1.883280in}}{\pgfqpoint{1.167246in}{1.893731in}}{\pgfqpoint{1.159542in}{1.901435in}}%
\pgfpathcurveto{\pgfqpoint{1.151837in}{1.909140in}}{\pgfqpoint{1.141386in}{1.913469in}}{\pgfqpoint{1.130491in}{1.913469in}}%
\pgfpathcurveto{\pgfqpoint{1.119595in}{1.913469in}}{\pgfqpoint{1.109145in}{1.909140in}}{\pgfqpoint{1.101440in}{1.901435in}}%
\pgfpathcurveto{\pgfqpoint{1.093736in}{1.893731in}}{\pgfqpoint{1.089407in}{1.883280in}}{\pgfqpoint{1.089407in}{1.872385in}}%
\pgfpathcurveto{\pgfqpoint{1.089407in}{1.861489in}}{\pgfqpoint{1.093736in}{1.851038in}}{\pgfqpoint{1.101440in}{1.843334in}}%
\pgfpathcurveto{\pgfqpoint{1.109145in}{1.835630in}}{\pgfqpoint{1.119595in}{1.831301in}}{\pgfqpoint{1.130491in}{1.831301in}}%
\pgfpathlineto{\pgfqpoint{1.130491in}{1.831301in}}%
\pgfpathclose%
\pgfusepath{stroke}%
\end{pgfscope}%
\begin{pgfscope}%
\pgfpathrectangle{\pgfqpoint{0.688192in}{0.670138in}}{\pgfqpoint{7.111808in}{5.061530in}}%
\pgfusepath{clip}%
\pgfsetbuttcap%
\pgfsetroundjoin%
\pgfsetlinewidth{1.003750pt}%
\definecolor{currentstroke}{rgb}{1.000000,0.000000,0.000000}%
\pgfsetstrokecolor{currentstroke}%
\pgfsetdash{}{0pt}%
\pgfpathmoveto{\pgfqpoint{1.127700in}{1.817179in}}%
\pgfpathcurveto{\pgfqpoint{1.138595in}{1.817179in}}{\pgfqpoint{1.149046in}{1.821508in}}{\pgfqpoint{1.156750in}{1.829212in}}%
\pgfpathcurveto{\pgfqpoint{1.164455in}{1.836917in}}{\pgfqpoint{1.168783in}{1.847367in}}{\pgfqpoint{1.168783in}{1.858263in}}%
\pgfpathcurveto{\pgfqpoint{1.168783in}{1.869158in}}{\pgfqpoint{1.164455in}{1.879609in}}{\pgfqpoint{1.156750in}{1.887314in}}%
\pgfpathcurveto{\pgfqpoint{1.149046in}{1.895018in}}{\pgfqpoint{1.138595in}{1.899347in}}{\pgfqpoint{1.127700in}{1.899347in}}%
\pgfpathcurveto{\pgfqpoint{1.116804in}{1.899347in}}{\pgfqpoint{1.106353in}{1.895018in}}{\pgfqpoint{1.098649in}{1.887314in}}%
\pgfpathcurveto{\pgfqpoint{1.090945in}{1.879609in}}{\pgfqpoint{1.086616in}{1.869158in}}{\pgfqpoint{1.086616in}{1.858263in}}%
\pgfpathcurveto{\pgfqpoint{1.086616in}{1.847367in}}{\pgfqpoint{1.090945in}{1.836917in}}{\pgfqpoint{1.098649in}{1.829212in}}%
\pgfpathcurveto{\pgfqpoint{1.106353in}{1.821508in}}{\pgfqpoint{1.116804in}{1.817179in}}{\pgfqpoint{1.127700in}{1.817179in}}%
\pgfpathlineto{\pgfqpoint{1.127700in}{1.817179in}}%
\pgfpathclose%
\pgfusepath{stroke}%
\end{pgfscope}%
\begin{pgfscope}%
\pgfpathrectangle{\pgfqpoint{0.688192in}{0.670138in}}{\pgfqpoint{7.111808in}{5.061530in}}%
\pgfusepath{clip}%
\pgfsetbuttcap%
\pgfsetroundjoin%
\pgfsetlinewidth{1.003750pt}%
\definecolor{currentstroke}{rgb}{1.000000,0.000000,0.000000}%
\pgfsetstrokecolor{currentstroke}%
\pgfsetdash{}{0pt}%
\pgfpathmoveto{\pgfqpoint{1.248471in}{2.325985in}}%
\pgfpathcurveto{\pgfqpoint{1.259367in}{2.325985in}}{\pgfqpoint{1.269818in}{2.330314in}}{\pgfqpoint{1.277522in}{2.338018in}}%
\pgfpathcurveto{\pgfqpoint{1.285226in}{2.345723in}}{\pgfqpoint{1.289555in}{2.356173in}}{\pgfqpoint{1.289555in}{2.367069in}}%
\pgfpathcurveto{\pgfqpoint{1.289555in}{2.377964in}}{\pgfqpoint{1.285226in}{2.388415in}}{\pgfqpoint{1.277522in}{2.396120in}}%
\pgfpathcurveto{\pgfqpoint{1.269818in}{2.403824in}}{\pgfqpoint{1.259367in}{2.408153in}}{\pgfqpoint{1.248471in}{2.408153in}}%
\pgfpathcurveto{\pgfqpoint{1.237576in}{2.408153in}}{\pgfqpoint{1.227125in}{2.403824in}}{\pgfqpoint{1.219421in}{2.396120in}}%
\pgfpathcurveto{\pgfqpoint{1.211716in}{2.388415in}}{\pgfqpoint{1.207387in}{2.377964in}}{\pgfqpoint{1.207387in}{2.367069in}}%
\pgfpathcurveto{\pgfqpoint{1.207387in}{2.356173in}}{\pgfqpoint{1.211716in}{2.345723in}}{\pgfqpoint{1.219421in}{2.338018in}}%
\pgfpathcurveto{\pgfqpoint{1.227125in}{2.330314in}}{\pgfqpoint{1.237576in}{2.325985in}}{\pgfqpoint{1.248471in}{2.325985in}}%
\pgfpathlineto{\pgfqpoint{1.248471in}{2.325985in}}%
\pgfpathclose%
\pgfusepath{stroke}%
\end{pgfscope}%
\begin{pgfscope}%
\pgfpathrectangle{\pgfqpoint{0.688192in}{0.670138in}}{\pgfqpoint{7.111808in}{5.061530in}}%
\pgfusepath{clip}%
\pgfsetbuttcap%
\pgfsetroundjoin%
\pgfsetlinewidth{1.003750pt}%
\definecolor{currentstroke}{rgb}{1.000000,0.000000,0.000000}%
\pgfsetstrokecolor{currentstroke}%
\pgfsetdash{}{0pt}%
\pgfpathmoveto{\pgfqpoint{1.229691in}{2.053609in}}%
\pgfpathcurveto{\pgfqpoint{1.240586in}{2.053609in}}{\pgfqpoint{1.251037in}{2.057938in}}{\pgfqpoint{1.258741in}{2.065643in}}%
\pgfpathcurveto{\pgfqpoint{1.266446in}{2.073347in}}{\pgfqpoint{1.270775in}{2.083798in}}{\pgfqpoint{1.270775in}{2.094693in}}%
\pgfpathcurveto{\pgfqpoint{1.270775in}{2.105589in}}{\pgfqpoint{1.266446in}{2.116040in}}{\pgfqpoint{1.258741in}{2.123744in}}%
\pgfpathcurveto{\pgfqpoint{1.251037in}{2.131448in}}{\pgfqpoint{1.240586in}{2.135777in}}{\pgfqpoint{1.229691in}{2.135777in}}%
\pgfpathcurveto{\pgfqpoint{1.218795in}{2.135777in}}{\pgfqpoint{1.208344in}{2.131448in}}{\pgfqpoint{1.200640in}{2.123744in}}%
\pgfpathcurveto{\pgfqpoint{1.192936in}{2.116040in}}{\pgfqpoint{1.188607in}{2.105589in}}{\pgfqpoint{1.188607in}{2.094693in}}%
\pgfpathcurveto{\pgfqpoint{1.188607in}{2.083798in}}{\pgfqpoint{1.192936in}{2.073347in}}{\pgfqpoint{1.200640in}{2.065643in}}%
\pgfpathcurveto{\pgfqpoint{1.208344in}{2.057938in}}{\pgfqpoint{1.218795in}{2.053609in}}{\pgfqpoint{1.229691in}{2.053609in}}%
\pgfpathlineto{\pgfqpoint{1.229691in}{2.053609in}}%
\pgfpathclose%
\pgfusepath{stroke}%
\end{pgfscope}%
\begin{pgfscope}%
\pgfpathrectangle{\pgfqpoint{0.688192in}{0.670138in}}{\pgfqpoint{7.111808in}{5.061530in}}%
\pgfusepath{clip}%
\pgfsetbuttcap%
\pgfsetroundjoin%
\pgfsetlinewidth{1.003750pt}%
\definecolor{currentstroke}{rgb}{1.000000,0.000000,0.000000}%
\pgfsetstrokecolor{currentstroke}%
\pgfsetdash{}{0pt}%
\pgfpathmoveto{\pgfqpoint{1.229448in}{2.049811in}}%
\pgfpathcurveto{\pgfqpoint{1.240343in}{2.049811in}}{\pgfqpoint{1.250794in}{2.054140in}}{\pgfqpoint{1.258498in}{2.061844in}}%
\pgfpathcurveto{\pgfqpoint{1.266203in}{2.069549in}}{\pgfqpoint{1.270531in}{2.079999in}}{\pgfqpoint{1.270531in}{2.090895in}}%
\pgfpathcurveto{\pgfqpoint{1.270531in}{2.101791in}}{\pgfqpoint{1.266203in}{2.112241in}}{\pgfqpoint{1.258498in}{2.119946in}}%
\pgfpathcurveto{\pgfqpoint{1.250794in}{2.127650in}}{\pgfqpoint{1.240343in}{2.131979in}}{\pgfqpoint{1.229448in}{2.131979in}}%
\pgfpathcurveto{\pgfqpoint{1.218552in}{2.131979in}}{\pgfqpoint{1.208101in}{2.127650in}}{\pgfqpoint{1.200397in}{2.119946in}}%
\pgfpathcurveto{\pgfqpoint{1.192692in}{2.112241in}}{\pgfqpoint{1.188364in}{2.101791in}}{\pgfqpoint{1.188364in}{2.090895in}}%
\pgfpathcurveto{\pgfqpoint{1.188364in}{2.079999in}}{\pgfqpoint{1.192692in}{2.069549in}}{\pgfqpoint{1.200397in}{2.061844in}}%
\pgfpathcurveto{\pgfqpoint{1.208101in}{2.054140in}}{\pgfqpoint{1.218552in}{2.049811in}}{\pgfqpoint{1.229448in}{2.049811in}}%
\pgfpathlineto{\pgfqpoint{1.229448in}{2.049811in}}%
\pgfpathclose%
\pgfusepath{stroke}%
\end{pgfscope}%
\begin{pgfscope}%
\pgfpathrectangle{\pgfqpoint{0.688192in}{0.670138in}}{\pgfqpoint{7.111808in}{5.061530in}}%
\pgfusepath{clip}%
\pgfsetbuttcap%
\pgfsetroundjoin%
\pgfsetlinewidth{1.003750pt}%
\definecolor{currentstroke}{rgb}{1.000000,0.000000,0.000000}%
\pgfsetstrokecolor{currentstroke}%
\pgfsetdash{}{0pt}%
\pgfpathmoveto{\pgfqpoint{1.227077in}{1.692979in}}%
\pgfpathcurveto{\pgfqpoint{1.237973in}{1.692979in}}{\pgfqpoint{1.248424in}{1.697308in}}{\pgfqpoint{1.256128in}{1.705012in}}%
\pgfpathcurveto{\pgfqpoint{1.263832in}{1.712717in}}{\pgfqpoint{1.268161in}{1.723168in}}{\pgfqpoint{1.268161in}{1.734063in}}%
\pgfpathcurveto{\pgfqpoint{1.268161in}{1.744959in}}{\pgfqpoint{1.263832in}{1.755410in}}{\pgfqpoint{1.256128in}{1.763114in}}%
\pgfpathcurveto{\pgfqpoint{1.248424in}{1.770818in}}{\pgfqpoint{1.237973in}{1.775147in}}{\pgfqpoint{1.227077in}{1.775147in}}%
\pgfpathcurveto{\pgfqpoint{1.216182in}{1.775147in}}{\pgfqpoint{1.205731in}{1.770818in}}{\pgfqpoint{1.198027in}{1.763114in}}%
\pgfpathcurveto{\pgfqpoint{1.190322in}{1.755410in}}{\pgfqpoint{1.185993in}{1.744959in}}{\pgfqpoint{1.185993in}{1.734063in}}%
\pgfpathcurveto{\pgfqpoint{1.185993in}{1.723168in}}{\pgfqpoint{1.190322in}{1.712717in}}{\pgfqpoint{1.198027in}{1.705012in}}%
\pgfpathcurveto{\pgfqpoint{1.205731in}{1.697308in}}{\pgfqpoint{1.216182in}{1.692979in}}{\pgfqpoint{1.227077in}{1.692979in}}%
\pgfpathlineto{\pgfqpoint{1.227077in}{1.692979in}}%
\pgfpathclose%
\pgfusepath{stroke}%
\end{pgfscope}%
\begin{pgfscope}%
\pgfpathrectangle{\pgfqpoint{0.688192in}{0.670138in}}{\pgfqpoint{7.111808in}{5.061530in}}%
\pgfusepath{clip}%
\pgfsetbuttcap%
\pgfsetroundjoin%
\pgfsetlinewidth{1.003750pt}%
\definecolor{currentstroke}{rgb}{1.000000,0.000000,0.000000}%
\pgfsetstrokecolor{currentstroke}%
\pgfsetdash{}{0pt}%
\pgfpathmoveto{\pgfqpoint{1.224567in}{1.695777in}}%
\pgfpathcurveto{\pgfqpoint{1.235463in}{1.695777in}}{\pgfqpoint{1.245914in}{1.700106in}}{\pgfqpoint{1.253618in}{1.707811in}}%
\pgfpathcurveto{\pgfqpoint{1.261322in}{1.715515in}}{\pgfqpoint{1.265651in}{1.725966in}}{\pgfqpoint{1.265651in}{1.736861in}}%
\pgfpathcurveto{\pgfqpoint{1.265651in}{1.747757in}}{\pgfqpoint{1.261322in}{1.758208in}}{\pgfqpoint{1.253618in}{1.765912in}}%
\pgfpathcurveto{\pgfqpoint{1.245914in}{1.773616in}}{\pgfqpoint{1.235463in}{1.777945in}}{\pgfqpoint{1.224567in}{1.777945in}}%
\pgfpathcurveto{\pgfqpoint{1.213672in}{1.777945in}}{\pgfqpoint{1.203221in}{1.773616in}}{\pgfqpoint{1.195517in}{1.765912in}}%
\pgfpathcurveto{\pgfqpoint{1.187812in}{1.758208in}}{\pgfqpoint{1.183483in}{1.747757in}}{\pgfqpoint{1.183483in}{1.736861in}}%
\pgfpathcurveto{\pgfqpoint{1.183483in}{1.725966in}}{\pgfqpoint{1.187812in}{1.715515in}}{\pgfqpoint{1.195517in}{1.707811in}}%
\pgfpathcurveto{\pgfqpoint{1.203221in}{1.700106in}}{\pgfqpoint{1.213672in}{1.695777in}}{\pgfqpoint{1.224567in}{1.695777in}}%
\pgfpathlineto{\pgfqpoint{1.224567in}{1.695777in}}%
\pgfpathclose%
\pgfusepath{stroke}%
\end{pgfscope}%
\begin{pgfscope}%
\pgfpathrectangle{\pgfqpoint{0.688192in}{0.670138in}}{\pgfqpoint{7.111808in}{5.061530in}}%
\pgfusepath{clip}%
\pgfsetbuttcap%
\pgfsetroundjoin%
\pgfsetlinewidth{1.003750pt}%
\definecolor{currentstroke}{rgb}{1.000000,0.000000,0.000000}%
\pgfsetstrokecolor{currentstroke}%
\pgfsetdash{}{0pt}%
\pgfpathmoveto{\pgfqpoint{1.384639in}{2.005194in}}%
\pgfpathcurveto{\pgfqpoint{1.395534in}{2.005194in}}{\pgfqpoint{1.405985in}{2.009522in}}{\pgfqpoint{1.413689in}{2.017227in}}%
\pgfpathcurveto{\pgfqpoint{1.421394in}{2.024931in}}{\pgfqpoint{1.425723in}{2.035382in}}{\pgfqpoint{1.425723in}{2.046277in}}%
\pgfpathcurveto{\pgfqpoint{1.425723in}{2.057173in}}{\pgfqpoint{1.421394in}{2.067624in}}{\pgfqpoint{1.413689in}{2.075328in}}%
\pgfpathcurveto{\pgfqpoint{1.405985in}{2.083032in}}{\pgfqpoint{1.395534in}{2.087361in}}{\pgfqpoint{1.384639in}{2.087361in}}%
\pgfpathcurveto{\pgfqpoint{1.373743in}{2.087361in}}{\pgfqpoint{1.363292in}{2.083032in}}{\pgfqpoint{1.355588in}{2.075328in}}%
\pgfpathcurveto{\pgfqpoint{1.347884in}{2.067624in}}{\pgfqpoint{1.343555in}{2.057173in}}{\pgfqpoint{1.343555in}{2.046277in}}%
\pgfpathcurveto{\pgfqpoint{1.343555in}{2.035382in}}{\pgfqpoint{1.347884in}{2.024931in}}{\pgfqpoint{1.355588in}{2.017227in}}%
\pgfpathcurveto{\pgfqpoint{1.363292in}{2.009522in}}{\pgfqpoint{1.373743in}{2.005194in}}{\pgfqpoint{1.384639in}{2.005194in}}%
\pgfpathlineto{\pgfqpoint{1.384639in}{2.005194in}}%
\pgfpathclose%
\pgfusepath{stroke}%
\end{pgfscope}%
\begin{pgfscope}%
\pgfpathrectangle{\pgfqpoint{0.688192in}{0.670138in}}{\pgfqpoint{7.111808in}{5.061530in}}%
\pgfusepath{clip}%
\pgfsetbuttcap%
\pgfsetroundjoin%
\pgfsetlinewidth{1.003750pt}%
\definecolor{currentstroke}{rgb}{1.000000,0.000000,0.000000}%
\pgfsetstrokecolor{currentstroke}%
\pgfsetdash{}{0pt}%
\pgfpathmoveto{\pgfqpoint{1.230001in}{1.676778in}}%
\pgfpathcurveto{\pgfqpoint{1.240897in}{1.676778in}}{\pgfqpoint{1.251347in}{1.681107in}}{\pgfqpoint{1.259052in}{1.688811in}}%
\pgfpathcurveto{\pgfqpoint{1.266756in}{1.696515in}}{\pgfqpoint{1.271085in}{1.706966in}}{\pgfqpoint{1.271085in}{1.717862in}}%
\pgfpathcurveto{\pgfqpoint{1.271085in}{1.728757in}}{\pgfqpoint{1.266756in}{1.739208in}}{\pgfqpoint{1.259052in}{1.746912in}}%
\pgfpathcurveto{\pgfqpoint{1.251347in}{1.754617in}}{\pgfqpoint{1.240897in}{1.758946in}}{\pgfqpoint{1.230001in}{1.758946in}}%
\pgfpathcurveto{\pgfqpoint{1.219106in}{1.758946in}}{\pgfqpoint{1.208655in}{1.754617in}}{\pgfqpoint{1.200950in}{1.746912in}}%
\pgfpathcurveto{\pgfqpoint{1.193246in}{1.739208in}}{\pgfqpoint{1.188917in}{1.728757in}}{\pgfqpoint{1.188917in}{1.717862in}}%
\pgfpathcurveto{\pgfqpoint{1.188917in}{1.706966in}}{\pgfqpoint{1.193246in}{1.696515in}}{\pgfqpoint{1.200950in}{1.688811in}}%
\pgfpathcurveto{\pgfqpoint{1.208655in}{1.681107in}}{\pgfqpoint{1.219106in}{1.676778in}}{\pgfqpoint{1.230001in}{1.676778in}}%
\pgfpathlineto{\pgfqpoint{1.230001in}{1.676778in}}%
\pgfpathclose%
\pgfusepath{stroke}%
\end{pgfscope}%
\begin{pgfscope}%
\pgfpathrectangle{\pgfqpoint{0.688192in}{0.670138in}}{\pgfqpoint{7.111808in}{5.061530in}}%
\pgfusepath{clip}%
\pgfsetbuttcap%
\pgfsetroundjoin%
\pgfsetlinewidth{1.003750pt}%
\definecolor{currentstroke}{rgb}{1.000000,0.000000,0.000000}%
\pgfsetstrokecolor{currentstroke}%
\pgfsetdash{}{0pt}%
\pgfpathmoveto{\pgfqpoint{1.350484in}{1.781832in}}%
\pgfpathcurveto{\pgfqpoint{1.361379in}{1.781832in}}{\pgfqpoint{1.371830in}{1.786161in}}{\pgfqpoint{1.379534in}{1.793865in}}%
\pgfpathcurveto{\pgfqpoint{1.387239in}{1.801569in}}{\pgfqpoint{1.391568in}{1.812020in}}{\pgfqpoint{1.391568in}{1.822916in}}%
\pgfpathcurveto{\pgfqpoint{1.391568in}{1.833811in}}{\pgfqpoint{1.387239in}{1.844262in}}{\pgfqpoint{1.379534in}{1.851966in}}%
\pgfpathcurveto{\pgfqpoint{1.371830in}{1.859671in}}{\pgfqpoint{1.361379in}{1.864000in}}{\pgfqpoint{1.350484in}{1.864000in}}%
\pgfpathcurveto{\pgfqpoint{1.339588in}{1.864000in}}{\pgfqpoint{1.329137in}{1.859671in}}{\pgfqpoint{1.321433in}{1.851966in}}%
\pgfpathcurveto{\pgfqpoint{1.313729in}{1.844262in}}{\pgfqpoint{1.309400in}{1.833811in}}{\pgfqpoint{1.309400in}{1.822916in}}%
\pgfpathcurveto{\pgfqpoint{1.309400in}{1.812020in}}{\pgfqpoint{1.313729in}{1.801569in}}{\pgfqpoint{1.321433in}{1.793865in}}%
\pgfpathcurveto{\pgfqpoint{1.329137in}{1.786161in}}{\pgfqpoint{1.339588in}{1.781832in}}{\pgfqpoint{1.350484in}{1.781832in}}%
\pgfpathlineto{\pgfqpoint{1.350484in}{1.781832in}}%
\pgfpathclose%
\pgfusepath{stroke}%
\end{pgfscope}%
\begin{pgfscope}%
\pgfpathrectangle{\pgfqpoint{0.688192in}{0.670138in}}{\pgfqpoint{7.111808in}{5.061530in}}%
\pgfusepath{clip}%
\pgfsetbuttcap%
\pgfsetroundjoin%
\pgfsetlinewidth{1.003750pt}%
\definecolor{currentstroke}{rgb}{1.000000,0.000000,0.000000}%
\pgfsetstrokecolor{currentstroke}%
\pgfsetdash{}{0pt}%
\pgfpathmoveto{\pgfqpoint{1.387370in}{1.722287in}}%
\pgfpathcurveto{\pgfqpoint{1.398265in}{1.722287in}}{\pgfqpoint{1.408716in}{1.726616in}}{\pgfqpoint{1.416420in}{1.734321in}}%
\pgfpathcurveto{\pgfqpoint{1.424125in}{1.742025in}}{\pgfqpoint{1.428453in}{1.752476in}}{\pgfqpoint{1.428453in}{1.763371in}}%
\pgfpathcurveto{\pgfqpoint{1.428453in}{1.774267in}}{\pgfqpoint{1.424125in}{1.784718in}}{\pgfqpoint{1.416420in}{1.792422in}}%
\pgfpathcurveto{\pgfqpoint{1.408716in}{1.800126in}}{\pgfqpoint{1.398265in}{1.804455in}}{\pgfqpoint{1.387370in}{1.804455in}}%
\pgfpathcurveto{\pgfqpoint{1.376474in}{1.804455in}}{\pgfqpoint{1.366023in}{1.800126in}}{\pgfqpoint{1.358319in}{1.792422in}}%
\pgfpathcurveto{\pgfqpoint{1.350615in}{1.784718in}}{\pgfqpoint{1.346286in}{1.774267in}}{\pgfqpoint{1.346286in}{1.763371in}}%
\pgfpathcurveto{\pgfqpoint{1.346286in}{1.752476in}}{\pgfqpoint{1.350615in}{1.742025in}}{\pgfqpoint{1.358319in}{1.734321in}}%
\pgfpathcurveto{\pgfqpoint{1.366023in}{1.726616in}}{\pgfqpoint{1.376474in}{1.722287in}}{\pgfqpoint{1.387370in}{1.722287in}}%
\pgfpathlineto{\pgfqpoint{1.387370in}{1.722287in}}%
\pgfpathclose%
\pgfusepath{stroke}%
\end{pgfscope}%
\begin{pgfscope}%
\pgfpathrectangle{\pgfqpoint{0.688192in}{0.670138in}}{\pgfqpoint{7.111808in}{5.061530in}}%
\pgfusepath{clip}%
\pgfsetbuttcap%
\pgfsetroundjoin%
\pgfsetlinewidth{1.003750pt}%
\definecolor{currentstroke}{rgb}{1.000000,0.000000,0.000000}%
\pgfsetstrokecolor{currentstroke}%
\pgfsetdash{}{0pt}%
\pgfpathmoveto{\pgfqpoint{1.214680in}{1.829222in}}%
\pgfpathcurveto{\pgfqpoint{1.225575in}{1.829222in}}{\pgfqpoint{1.236026in}{1.833551in}}{\pgfqpoint{1.243730in}{1.841255in}}%
\pgfpathcurveto{\pgfqpoint{1.251435in}{1.848960in}}{\pgfqpoint{1.255763in}{1.859410in}}{\pgfqpoint{1.255763in}{1.870306in}}%
\pgfpathcurveto{\pgfqpoint{1.255763in}{1.881202in}}{\pgfqpoint{1.251435in}{1.891652in}}{\pgfqpoint{1.243730in}{1.899357in}}%
\pgfpathcurveto{\pgfqpoint{1.236026in}{1.907061in}}{\pgfqpoint{1.225575in}{1.911390in}}{\pgfqpoint{1.214680in}{1.911390in}}%
\pgfpathcurveto{\pgfqpoint{1.203784in}{1.911390in}}{\pgfqpoint{1.193333in}{1.907061in}}{\pgfqpoint{1.185629in}{1.899357in}}%
\pgfpathcurveto{\pgfqpoint{1.177925in}{1.891652in}}{\pgfqpoint{1.173596in}{1.881202in}}{\pgfqpoint{1.173596in}{1.870306in}}%
\pgfpathcurveto{\pgfqpoint{1.173596in}{1.859410in}}{\pgfqpoint{1.177925in}{1.848960in}}{\pgfqpoint{1.185629in}{1.841255in}}%
\pgfpathcurveto{\pgfqpoint{1.193333in}{1.833551in}}{\pgfqpoint{1.203784in}{1.829222in}}{\pgfqpoint{1.214680in}{1.829222in}}%
\pgfpathlineto{\pgfqpoint{1.214680in}{1.829222in}}%
\pgfpathclose%
\pgfusepath{stroke}%
\end{pgfscope}%
\begin{pgfscope}%
\pgfpathrectangle{\pgfqpoint{0.688192in}{0.670138in}}{\pgfqpoint{7.111808in}{5.061530in}}%
\pgfusepath{clip}%
\pgfsetbuttcap%
\pgfsetroundjoin%
\pgfsetlinewidth{1.003750pt}%
\definecolor{currentstroke}{rgb}{1.000000,0.000000,0.000000}%
\pgfsetstrokecolor{currentstroke}%
\pgfsetdash{}{0pt}%
\pgfpathmoveto{\pgfqpoint{1.321264in}{1.626574in}}%
\pgfpathcurveto{\pgfqpoint{1.332159in}{1.626574in}}{\pgfqpoint{1.342610in}{1.630903in}}{\pgfqpoint{1.350315in}{1.638607in}}%
\pgfpathcurveto{\pgfqpoint{1.358019in}{1.646311in}}{\pgfqpoint{1.362348in}{1.656762in}}{\pgfqpoint{1.362348in}{1.667658in}}%
\pgfpathcurveto{\pgfqpoint{1.362348in}{1.678553in}}{\pgfqpoint{1.358019in}{1.689004in}}{\pgfqpoint{1.350315in}{1.696708in}}%
\pgfpathcurveto{\pgfqpoint{1.342610in}{1.704413in}}{\pgfqpoint{1.332159in}{1.708742in}}{\pgfqpoint{1.321264in}{1.708742in}}%
\pgfpathcurveto{\pgfqpoint{1.310368in}{1.708742in}}{\pgfqpoint{1.299917in}{1.704413in}}{\pgfqpoint{1.292213in}{1.696708in}}%
\pgfpathcurveto{\pgfqpoint{1.284509in}{1.689004in}}{\pgfqpoint{1.280180in}{1.678553in}}{\pgfqpoint{1.280180in}{1.667658in}}%
\pgfpathcurveto{\pgfqpoint{1.280180in}{1.656762in}}{\pgfqpoint{1.284509in}{1.646311in}}{\pgfqpoint{1.292213in}{1.638607in}}%
\pgfpathcurveto{\pgfqpoint{1.299917in}{1.630903in}}{\pgfqpoint{1.310368in}{1.626574in}}{\pgfqpoint{1.321264in}{1.626574in}}%
\pgfpathlineto{\pgfqpoint{1.321264in}{1.626574in}}%
\pgfpathclose%
\pgfusepath{stroke}%
\end{pgfscope}%
\begin{pgfscope}%
\pgfpathrectangle{\pgfqpoint{0.688192in}{0.670138in}}{\pgfqpoint{7.111808in}{5.061530in}}%
\pgfusepath{clip}%
\pgfsetbuttcap%
\pgfsetroundjoin%
\pgfsetlinewidth{1.003750pt}%
\definecolor{currentstroke}{rgb}{1.000000,0.000000,0.000000}%
\pgfsetstrokecolor{currentstroke}%
\pgfsetdash{}{0pt}%
\pgfpathmoveto{\pgfqpoint{1.361318in}{1.544167in}}%
\pgfpathcurveto{\pgfqpoint{1.372214in}{1.544167in}}{\pgfqpoint{1.382665in}{1.548496in}}{\pgfqpoint{1.390369in}{1.556200in}}%
\pgfpathcurveto{\pgfqpoint{1.398073in}{1.563905in}}{\pgfqpoint{1.402402in}{1.574356in}}{\pgfqpoint{1.402402in}{1.585251in}}%
\pgfpathcurveto{\pgfqpoint{1.402402in}{1.596147in}}{\pgfqpoint{1.398073in}{1.606597in}}{\pgfqpoint{1.390369in}{1.614302in}}%
\pgfpathcurveto{\pgfqpoint{1.382665in}{1.622006in}}{\pgfqpoint{1.372214in}{1.626335in}}{\pgfqpoint{1.361318in}{1.626335in}}%
\pgfpathcurveto{\pgfqpoint{1.350423in}{1.626335in}}{\pgfqpoint{1.339972in}{1.622006in}}{\pgfqpoint{1.332268in}{1.614302in}}%
\pgfpathcurveto{\pgfqpoint{1.324563in}{1.606597in}}{\pgfqpoint{1.320235in}{1.596147in}}{\pgfqpoint{1.320235in}{1.585251in}}%
\pgfpathcurveto{\pgfqpoint{1.320235in}{1.574356in}}{\pgfqpoint{1.324563in}{1.563905in}}{\pgfqpoint{1.332268in}{1.556200in}}%
\pgfpathcurveto{\pgfqpoint{1.339972in}{1.548496in}}{\pgfqpoint{1.350423in}{1.544167in}}{\pgfqpoint{1.361318in}{1.544167in}}%
\pgfpathlineto{\pgfqpoint{1.361318in}{1.544167in}}%
\pgfpathclose%
\pgfusepath{stroke}%
\end{pgfscope}%
\begin{pgfscope}%
\pgfpathrectangle{\pgfqpoint{0.688192in}{0.670138in}}{\pgfqpoint{7.111808in}{5.061530in}}%
\pgfusepath{clip}%
\pgfsetbuttcap%
\pgfsetroundjoin%
\pgfsetlinewidth{1.003750pt}%
\definecolor{currentstroke}{rgb}{1.000000,0.000000,0.000000}%
\pgfsetstrokecolor{currentstroke}%
\pgfsetdash{}{0pt}%
\pgfpathmoveto{\pgfqpoint{1.360232in}{1.579406in}}%
\pgfpathcurveto{\pgfqpoint{1.371128in}{1.579406in}}{\pgfqpoint{1.381579in}{1.583735in}}{\pgfqpoint{1.389283in}{1.591440in}}%
\pgfpathcurveto{\pgfqpoint{1.396987in}{1.599144in}}{\pgfqpoint{1.401316in}{1.609595in}}{\pgfqpoint{1.401316in}{1.620490in}}%
\pgfpathcurveto{\pgfqpoint{1.401316in}{1.631386in}}{\pgfqpoint{1.396987in}{1.641837in}}{\pgfqpoint{1.389283in}{1.649541in}}%
\pgfpathcurveto{\pgfqpoint{1.381579in}{1.657245in}}{\pgfqpoint{1.371128in}{1.661574in}}{\pgfqpoint{1.360232in}{1.661574in}}%
\pgfpathcurveto{\pgfqpoint{1.349337in}{1.661574in}}{\pgfqpoint{1.338886in}{1.657245in}}{\pgfqpoint{1.331182in}{1.649541in}}%
\pgfpathcurveto{\pgfqpoint{1.323477in}{1.641837in}}{\pgfqpoint{1.319148in}{1.631386in}}{\pgfqpoint{1.319148in}{1.620490in}}%
\pgfpathcurveto{\pgfqpoint{1.319148in}{1.609595in}}{\pgfqpoint{1.323477in}{1.599144in}}{\pgfqpoint{1.331182in}{1.591440in}}%
\pgfpathcurveto{\pgfqpoint{1.338886in}{1.583735in}}{\pgfqpoint{1.349337in}{1.579406in}}{\pgfqpoint{1.360232in}{1.579406in}}%
\pgfpathlineto{\pgfqpoint{1.360232in}{1.579406in}}%
\pgfpathclose%
\pgfusepath{stroke}%
\end{pgfscope}%
\begin{pgfscope}%
\pgfpathrectangle{\pgfqpoint{0.688192in}{0.670138in}}{\pgfqpoint{7.111808in}{5.061530in}}%
\pgfusepath{clip}%
\pgfsetbuttcap%
\pgfsetroundjoin%
\pgfsetlinewidth{1.003750pt}%
\definecolor{currentstroke}{rgb}{1.000000,0.000000,0.000000}%
\pgfsetstrokecolor{currentstroke}%
\pgfsetdash{}{0pt}%
\pgfpathmoveto{\pgfqpoint{1.215093in}{1.816528in}}%
\pgfpathcurveto{\pgfqpoint{1.225988in}{1.816528in}}{\pgfqpoint{1.236439in}{1.820857in}}{\pgfqpoint{1.244143in}{1.828562in}}%
\pgfpathcurveto{\pgfqpoint{1.251848in}{1.836266in}}{\pgfqpoint{1.256177in}{1.846717in}}{\pgfqpoint{1.256177in}{1.857612in}}%
\pgfpathcurveto{\pgfqpoint{1.256177in}{1.868508in}}{\pgfqpoint{1.251848in}{1.878959in}}{\pgfqpoint{1.244143in}{1.886663in}}%
\pgfpathcurveto{\pgfqpoint{1.236439in}{1.894367in}}{\pgfqpoint{1.225988in}{1.898696in}}{\pgfqpoint{1.215093in}{1.898696in}}%
\pgfpathcurveto{\pgfqpoint{1.204197in}{1.898696in}}{\pgfqpoint{1.193746in}{1.894367in}}{\pgfqpoint{1.186042in}{1.886663in}}%
\pgfpathcurveto{\pgfqpoint{1.178338in}{1.878959in}}{\pgfqpoint{1.174009in}{1.868508in}}{\pgfqpoint{1.174009in}{1.857612in}}%
\pgfpathcurveto{\pgfqpoint{1.174009in}{1.846717in}}{\pgfqpoint{1.178338in}{1.836266in}}{\pgfqpoint{1.186042in}{1.828562in}}%
\pgfpathcurveto{\pgfqpoint{1.193746in}{1.820857in}}{\pgfqpoint{1.204197in}{1.816528in}}{\pgfqpoint{1.215093in}{1.816528in}}%
\pgfpathlineto{\pgfqpoint{1.215093in}{1.816528in}}%
\pgfpathclose%
\pgfusepath{stroke}%
\end{pgfscope}%
\begin{pgfscope}%
\pgfpathrectangle{\pgfqpoint{0.688192in}{0.670138in}}{\pgfqpoint{7.111808in}{5.061530in}}%
\pgfusepath{clip}%
\pgfsetbuttcap%
\pgfsetroundjoin%
\pgfsetlinewidth{1.003750pt}%
\definecolor{currentstroke}{rgb}{1.000000,0.000000,0.000000}%
\pgfsetstrokecolor{currentstroke}%
\pgfsetdash{}{0pt}%
\pgfpathmoveto{\pgfqpoint{1.269489in}{2.005063in}}%
\pgfpathcurveto{\pgfqpoint{1.280384in}{2.005063in}}{\pgfqpoint{1.290835in}{2.009392in}}{\pgfqpoint{1.298539in}{2.017096in}}%
\pgfpathcurveto{\pgfqpoint{1.306244in}{2.024801in}}{\pgfqpoint{1.310573in}{2.035252in}}{\pgfqpoint{1.310573in}{2.046147in}}%
\pgfpathcurveto{\pgfqpoint{1.310573in}{2.057043in}}{\pgfqpoint{1.306244in}{2.067493in}}{\pgfqpoint{1.298539in}{2.075198in}}%
\pgfpathcurveto{\pgfqpoint{1.290835in}{2.082902in}}{\pgfqpoint{1.280384in}{2.087231in}}{\pgfqpoint{1.269489in}{2.087231in}}%
\pgfpathcurveto{\pgfqpoint{1.258593in}{2.087231in}}{\pgfqpoint{1.248142in}{2.082902in}}{\pgfqpoint{1.240438in}{2.075198in}}%
\pgfpathcurveto{\pgfqpoint{1.232734in}{2.067493in}}{\pgfqpoint{1.228405in}{2.057043in}}{\pgfqpoint{1.228405in}{2.046147in}}%
\pgfpathcurveto{\pgfqpoint{1.228405in}{2.035252in}}{\pgfqpoint{1.232734in}{2.024801in}}{\pgfqpoint{1.240438in}{2.017096in}}%
\pgfpathcurveto{\pgfqpoint{1.248142in}{2.009392in}}{\pgfqpoint{1.258593in}{2.005063in}}{\pgfqpoint{1.269489in}{2.005063in}}%
\pgfpathlineto{\pgfqpoint{1.269489in}{2.005063in}}%
\pgfpathclose%
\pgfusepath{stroke}%
\end{pgfscope}%
\begin{pgfscope}%
\pgfpathrectangle{\pgfqpoint{0.688192in}{0.670138in}}{\pgfqpoint{7.111808in}{5.061530in}}%
\pgfusepath{clip}%
\pgfsetbuttcap%
\pgfsetroundjoin%
\pgfsetlinewidth{1.003750pt}%
\definecolor{currentstroke}{rgb}{1.000000,0.000000,0.000000}%
\pgfsetstrokecolor{currentstroke}%
\pgfsetdash{}{0pt}%
\pgfpathmoveto{\pgfqpoint{1.234868in}{2.017024in}}%
\pgfpathcurveto{\pgfqpoint{1.245763in}{2.017024in}}{\pgfqpoint{1.256214in}{2.021353in}}{\pgfqpoint{1.263919in}{2.029057in}}%
\pgfpathcurveto{\pgfqpoint{1.271623in}{2.036762in}}{\pgfqpoint{1.275952in}{2.047212in}}{\pgfqpoint{1.275952in}{2.058108in}}%
\pgfpathcurveto{\pgfqpoint{1.275952in}{2.069004in}}{\pgfqpoint{1.271623in}{2.079454in}}{\pgfqpoint{1.263919in}{2.087159in}}%
\pgfpathcurveto{\pgfqpoint{1.256214in}{2.094863in}}{\pgfqpoint{1.245763in}{2.099192in}}{\pgfqpoint{1.234868in}{2.099192in}}%
\pgfpathcurveto{\pgfqpoint{1.223972in}{2.099192in}}{\pgfqpoint{1.213522in}{2.094863in}}{\pgfqpoint{1.205817in}{2.087159in}}%
\pgfpathcurveto{\pgfqpoint{1.198113in}{2.079454in}}{\pgfqpoint{1.193784in}{2.069004in}}{\pgfqpoint{1.193784in}{2.058108in}}%
\pgfpathcurveto{\pgfqpoint{1.193784in}{2.047212in}}{\pgfqpoint{1.198113in}{2.036762in}}{\pgfqpoint{1.205817in}{2.029057in}}%
\pgfpathcurveto{\pgfqpoint{1.213522in}{2.021353in}}{\pgfqpoint{1.223972in}{2.017024in}}{\pgfqpoint{1.234868in}{2.017024in}}%
\pgfpathlineto{\pgfqpoint{1.234868in}{2.017024in}}%
\pgfpathclose%
\pgfusepath{stroke}%
\end{pgfscope}%
\begin{pgfscope}%
\pgfpathrectangle{\pgfqpoint{0.688192in}{0.670138in}}{\pgfqpoint{7.111808in}{5.061530in}}%
\pgfusepath{clip}%
\pgfsetbuttcap%
\pgfsetroundjoin%
\pgfsetlinewidth{1.003750pt}%
\definecolor{currentstroke}{rgb}{1.000000,0.000000,0.000000}%
\pgfsetstrokecolor{currentstroke}%
\pgfsetdash{}{0pt}%
\pgfpathmoveto{\pgfqpoint{1.210918in}{1.826197in}}%
\pgfpathcurveto{\pgfqpoint{1.221814in}{1.826197in}}{\pgfqpoint{1.232265in}{1.830526in}}{\pgfqpoint{1.239969in}{1.838230in}}%
\pgfpathcurveto{\pgfqpoint{1.247673in}{1.845934in}}{\pgfqpoint{1.252002in}{1.856385in}}{\pgfqpoint{1.252002in}{1.867281in}}%
\pgfpathcurveto{\pgfqpoint{1.252002in}{1.878176in}}{\pgfqpoint{1.247673in}{1.888627in}}{\pgfqpoint{1.239969in}{1.896331in}}%
\pgfpathcurveto{\pgfqpoint{1.232265in}{1.904036in}}{\pgfqpoint{1.221814in}{1.908364in}}{\pgfqpoint{1.210918in}{1.908364in}}%
\pgfpathcurveto{\pgfqpoint{1.200023in}{1.908364in}}{\pgfqpoint{1.189572in}{1.904036in}}{\pgfqpoint{1.181867in}{1.896331in}}%
\pgfpathcurveto{\pgfqpoint{1.174163in}{1.888627in}}{\pgfqpoint{1.169834in}{1.878176in}}{\pgfqpoint{1.169834in}{1.867281in}}%
\pgfpathcurveto{\pgfqpoint{1.169834in}{1.856385in}}{\pgfqpoint{1.174163in}{1.845934in}}{\pgfqpoint{1.181867in}{1.838230in}}%
\pgfpathcurveto{\pgfqpoint{1.189572in}{1.830526in}}{\pgfqpoint{1.200023in}{1.826197in}}{\pgfqpoint{1.210918in}{1.826197in}}%
\pgfpathlineto{\pgfqpoint{1.210918in}{1.826197in}}%
\pgfpathclose%
\pgfusepath{stroke}%
\end{pgfscope}%
\begin{pgfscope}%
\pgfpathrectangle{\pgfqpoint{0.688192in}{0.670138in}}{\pgfqpoint{7.111808in}{5.061530in}}%
\pgfusepath{clip}%
\pgfsetbuttcap%
\pgfsetroundjoin%
\pgfsetlinewidth{1.003750pt}%
\definecolor{currentstroke}{rgb}{1.000000,0.000000,0.000000}%
\pgfsetstrokecolor{currentstroke}%
\pgfsetdash{}{0pt}%
\pgfpathmoveto{\pgfqpoint{0.688192in}{1.271741in}}%
\pgfpathcurveto{\pgfqpoint{0.699087in}{1.271741in}}{\pgfqpoint{0.709538in}{1.276070in}}{\pgfqpoint{0.717242in}{1.283775in}}%
\pgfpathcurveto{\pgfqpoint{0.724947in}{1.291479in}}{\pgfqpoint{0.729275in}{1.301930in}}{\pgfqpoint{0.729275in}{1.312825in}}%
\pgfpathcurveto{\pgfqpoint{0.729275in}{1.323721in}}{\pgfqpoint{0.724947in}{1.334172in}}{\pgfqpoint{0.717242in}{1.341876in}}%
\pgfpathcurveto{\pgfqpoint{0.709538in}{1.349580in}}{\pgfqpoint{0.699087in}{1.353909in}}{\pgfqpoint{0.688192in}{1.353909in}}%
\pgfpathcurveto{\pgfqpoint{0.677296in}{1.353909in}}{\pgfqpoint{0.666845in}{1.349580in}}{\pgfqpoint{0.659141in}{1.341876in}}%
\pgfpathcurveto{\pgfqpoint{0.651436in}{1.334172in}}{\pgfqpoint{0.647108in}{1.323721in}}{\pgfqpoint{0.647108in}{1.312825in}}%
\pgfpathcurveto{\pgfqpoint{0.647108in}{1.301930in}}{\pgfqpoint{0.651436in}{1.291479in}}{\pgfqpoint{0.659141in}{1.283775in}}%
\pgfpathcurveto{\pgfqpoint{0.666845in}{1.276070in}}{\pgfqpoint{0.677296in}{1.271741in}}{\pgfqpoint{0.688192in}{1.271741in}}%
\pgfpathlineto{\pgfqpoint{0.688192in}{1.271741in}}%
\pgfpathclose%
\pgfusepath{stroke}%
\end{pgfscope}%
\begin{pgfscope}%
\pgfpathrectangle{\pgfqpoint{0.688192in}{0.670138in}}{\pgfqpoint{7.111808in}{5.061530in}}%
\pgfusepath{clip}%
\pgfsetbuttcap%
\pgfsetroundjoin%
\pgfsetlinewidth{1.003750pt}%
\definecolor{currentstroke}{rgb}{0.000000,0.000000,0.000000}%
\pgfsetstrokecolor{currentstroke}%
\pgfsetdash{}{0pt}%
\pgfpathmoveto{\pgfqpoint{6.712361in}{2.552477in}}%
\pgfpathcurveto{\pgfqpoint{6.723257in}{2.552477in}}{\pgfqpoint{6.733708in}{2.556806in}}{\pgfqpoint{6.741412in}{2.564511in}}%
\pgfpathcurveto{\pgfqpoint{6.749116in}{2.572215in}}{\pgfqpoint{6.753445in}{2.582666in}}{\pgfqpoint{6.753445in}{2.593561in}}%
\pgfpathcurveto{\pgfqpoint{6.753445in}{2.604457in}}{\pgfqpoint{6.749116in}{2.614908in}}{\pgfqpoint{6.741412in}{2.622612in}}%
\pgfpathcurveto{\pgfqpoint{6.733708in}{2.630316in}}{\pgfqpoint{6.723257in}{2.634645in}}{\pgfqpoint{6.712361in}{2.634645in}}%
\pgfpathcurveto{\pgfqpoint{6.701466in}{2.634645in}}{\pgfqpoint{6.691015in}{2.630316in}}{\pgfqpoint{6.683311in}{2.622612in}}%
\pgfpathcurveto{\pgfqpoint{6.675606in}{2.614908in}}{\pgfqpoint{6.671277in}{2.604457in}}{\pgfqpoint{6.671277in}{2.593561in}}%
\pgfpathcurveto{\pgfqpoint{6.671277in}{2.582666in}}{\pgfqpoint{6.675606in}{2.572215in}}{\pgfqpoint{6.683311in}{2.564511in}}%
\pgfpathcurveto{\pgfqpoint{6.691015in}{2.556806in}}{\pgfqpoint{6.701466in}{2.552477in}}{\pgfqpoint{6.712361in}{2.552477in}}%
\pgfpathlineto{\pgfqpoint{6.712361in}{2.552477in}}%
\pgfpathclose%
\pgfusepath{stroke}%
\end{pgfscope}%
\begin{pgfscope}%
\pgfpathrectangle{\pgfqpoint{0.688192in}{0.670138in}}{\pgfqpoint{7.111808in}{5.061530in}}%
\pgfusepath{clip}%
\pgfsetbuttcap%
\pgfsetroundjoin%
\pgfsetlinewidth{1.003750pt}%
\definecolor{currentstroke}{rgb}{0.000000,0.000000,0.000000}%
\pgfsetstrokecolor{currentstroke}%
\pgfsetdash{}{0pt}%
\pgfpathmoveto{\pgfqpoint{2.489265in}{0.637384in}}%
\pgfpathcurveto{\pgfqpoint{2.500160in}{0.637384in}}{\pgfqpoint{2.510611in}{0.641713in}}{\pgfqpoint{2.518315in}{0.649417in}}%
\pgfpathcurveto{\pgfqpoint{2.526020in}{0.657122in}}{\pgfqpoint{2.530349in}{0.667572in}}{\pgfqpoint{2.530349in}{0.678468in}}%
\pgfpathcurveto{\pgfqpoint{2.530349in}{0.689364in}}{\pgfqpoint{2.526020in}{0.699814in}}{\pgfqpoint{2.518315in}{0.707519in}}%
\pgfpathcurveto{\pgfqpoint{2.510611in}{0.715223in}}{\pgfqpoint{2.500160in}{0.719552in}}{\pgfqpoint{2.489265in}{0.719552in}}%
\pgfpathcurveto{\pgfqpoint{2.478369in}{0.719552in}}{\pgfqpoint{2.467918in}{0.715223in}}{\pgfqpoint{2.460214in}{0.707519in}}%
\pgfpathcurveto{\pgfqpoint{2.452510in}{0.699814in}}{\pgfqpoint{2.448181in}{0.689364in}}{\pgfqpoint{2.448181in}{0.678468in}}%
\pgfpathcurveto{\pgfqpoint{2.448181in}{0.667572in}}{\pgfqpoint{2.452510in}{0.657122in}}{\pgfqpoint{2.460214in}{0.649417in}}%
\pgfpathcurveto{\pgfqpoint{2.467918in}{0.641713in}}{\pgfqpoint{2.478369in}{0.637384in}}{\pgfqpoint{2.489265in}{0.637384in}}%
\pgfusepath{stroke}%
\end{pgfscope}%
\begin{pgfscope}%
\pgfpathrectangle{\pgfqpoint{0.688192in}{0.670138in}}{\pgfqpoint{7.111808in}{5.061530in}}%
\pgfusepath{clip}%
\pgfsetbuttcap%
\pgfsetroundjoin%
\pgfsetlinewidth{1.003750pt}%
\definecolor{currentstroke}{rgb}{0.000000,0.000000,0.000000}%
\pgfsetstrokecolor{currentstroke}%
\pgfsetdash{}{0pt}%
\pgfpathmoveto{\pgfqpoint{0.990735in}{0.663640in}}%
\pgfpathcurveto{\pgfqpoint{1.001630in}{0.663640in}}{\pgfqpoint{1.012081in}{0.667969in}}{\pgfqpoint{1.019785in}{0.675673in}}%
\pgfpathcurveto{\pgfqpoint{1.027490in}{0.683378in}}{\pgfqpoint{1.031818in}{0.693828in}}{\pgfqpoint{1.031818in}{0.704724in}}%
\pgfpathcurveto{\pgfqpoint{1.031818in}{0.715620in}}{\pgfqpoint{1.027490in}{0.726070in}}{\pgfqpoint{1.019785in}{0.733775in}}%
\pgfpathcurveto{\pgfqpoint{1.012081in}{0.741479in}}{\pgfqpoint{1.001630in}{0.745808in}}{\pgfqpoint{0.990735in}{0.745808in}}%
\pgfpathcurveto{\pgfqpoint{0.979839in}{0.745808in}}{\pgfqpoint{0.969388in}{0.741479in}}{\pgfqpoint{0.961684in}{0.733775in}}%
\pgfpathcurveto{\pgfqpoint{0.953980in}{0.726070in}}{\pgfqpoint{0.949651in}{0.715620in}}{\pgfqpoint{0.949651in}{0.704724in}}%
\pgfpathcurveto{\pgfqpoint{0.949651in}{0.693828in}}{\pgfqpoint{0.953980in}{0.683378in}}{\pgfqpoint{0.961684in}{0.675673in}}%
\pgfpathcurveto{\pgfqpoint{0.969388in}{0.667969in}}{\pgfqpoint{0.979839in}{0.663640in}}{\pgfqpoint{0.990735in}{0.663640in}}%
\pgfusepath{stroke}%
\end{pgfscope}%
\begin{pgfscope}%
\pgfpathrectangle{\pgfqpoint{0.688192in}{0.670138in}}{\pgfqpoint{7.111808in}{5.061530in}}%
\pgfusepath{clip}%
\pgfsetbuttcap%
\pgfsetroundjoin%
\pgfsetlinewidth{1.003750pt}%
\definecolor{currentstroke}{rgb}{0.000000,0.000000,0.000000}%
\pgfsetstrokecolor{currentstroke}%
\pgfsetdash{}{0pt}%
\pgfpathmoveto{\pgfqpoint{3.224983in}{2.235929in}}%
\pgfpathcurveto{\pgfqpoint{3.235878in}{2.235929in}}{\pgfqpoint{3.246329in}{2.240258in}}{\pgfqpoint{3.254033in}{2.247962in}}%
\pgfpathcurveto{\pgfqpoint{3.261738in}{2.255666in}}{\pgfqpoint{3.266067in}{2.266117in}}{\pgfqpoint{3.266067in}{2.277013in}}%
\pgfpathcurveto{\pgfqpoint{3.266067in}{2.287908in}}{\pgfqpoint{3.261738in}{2.298359in}}{\pgfqpoint{3.254033in}{2.306063in}}%
\pgfpathcurveto{\pgfqpoint{3.246329in}{2.313768in}}{\pgfqpoint{3.235878in}{2.318096in}}{\pgfqpoint{3.224983in}{2.318096in}}%
\pgfpathcurveto{\pgfqpoint{3.214087in}{2.318096in}}{\pgfqpoint{3.203636in}{2.313768in}}{\pgfqpoint{3.195932in}{2.306063in}}%
\pgfpathcurveto{\pgfqpoint{3.188228in}{2.298359in}}{\pgfqpoint{3.183899in}{2.287908in}}{\pgfqpoint{3.183899in}{2.277013in}}%
\pgfpathcurveto{\pgfqpoint{3.183899in}{2.266117in}}{\pgfqpoint{3.188228in}{2.255666in}}{\pgfqpoint{3.195932in}{2.247962in}}%
\pgfpathcurveto{\pgfqpoint{3.203636in}{2.240258in}}{\pgfqpoint{3.214087in}{2.235929in}}{\pgfqpoint{3.224983in}{2.235929in}}%
\pgfpathlineto{\pgfqpoint{3.224983in}{2.235929in}}%
\pgfpathclose%
\pgfusepath{stroke}%
\end{pgfscope}%
\begin{pgfscope}%
\pgfpathrectangle{\pgfqpoint{0.688192in}{0.670138in}}{\pgfqpoint{7.111808in}{5.061530in}}%
\pgfusepath{clip}%
\pgfsetbuttcap%
\pgfsetroundjoin%
\pgfsetlinewidth{1.003750pt}%
\definecolor{currentstroke}{rgb}{0.000000,0.000000,0.000000}%
\pgfsetstrokecolor{currentstroke}%
\pgfsetdash{}{0pt}%
\pgfpathmoveto{\pgfqpoint{0.942062in}{0.671110in}}%
\pgfpathcurveto{\pgfqpoint{0.952958in}{0.671110in}}{\pgfqpoint{0.963408in}{0.675439in}}{\pgfqpoint{0.971113in}{0.683143in}}%
\pgfpathcurveto{\pgfqpoint{0.978817in}{0.690847in}}{\pgfqpoint{0.983146in}{0.701298in}}{\pgfqpoint{0.983146in}{0.712194in}}%
\pgfpathcurveto{\pgfqpoint{0.983146in}{0.723089in}}{\pgfqpoint{0.978817in}{0.733540in}}{\pgfqpoint{0.971113in}{0.741244in}}%
\pgfpathcurveto{\pgfqpoint{0.963408in}{0.748949in}}{\pgfqpoint{0.952958in}{0.753278in}}{\pgfqpoint{0.942062in}{0.753278in}}%
\pgfpathcurveto{\pgfqpoint{0.931166in}{0.753278in}}{\pgfqpoint{0.920716in}{0.748949in}}{\pgfqpoint{0.913011in}{0.741244in}}%
\pgfpathcurveto{\pgfqpoint{0.905307in}{0.733540in}}{\pgfqpoint{0.900978in}{0.723089in}}{\pgfqpoint{0.900978in}{0.712194in}}%
\pgfpathcurveto{\pgfqpoint{0.900978in}{0.701298in}}{\pgfqpoint{0.905307in}{0.690847in}}{\pgfqpoint{0.913011in}{0.683143in}}%
\pgfpathcurveto{\pgfqpoint{0.920716in}{0.675439in}}{\pgfqpoint{0.931166in}{0.671110in}}{\pgfqpoint{0.942062in}{0.671110in}}%
\pgfpathlineto{\pgfqpoint{0.942062in}{0.671110in}}%
\pgfpathclose%
\pgfusepath{stroke}%
\end{pgfscope}%
\begin{pgfscope}%
\pgfpathrectangle{\pgfqpoint{0.688192in}{0.670138in}}{\pgfqpoint{7.111808in}{5.061530in}}%
\pgfusepath{clip}%
\pgfsetbuttcap%
\pgfsetroundjoin%
\pgfsetlinewidth{1.003750pt}%
\definecolor{currentstroke}{rgb}{0.000000,0.000000,0.000000}%
\pgfsetstrokecolor{currentstroke}%
\pgfsetdash{}{0pt}%
\pgfpathmoveto{\pgfqpoint{2.436233in}{1.165480in}}%
\pgfpathcurveto{\pgfqpoint{2.447129in}{1.165480in}}{\pgfqpoint{2.457579in}{1.169809in}}{\pgfqpoint{2.465284in}{1.177514in}}%
\pgfpathcurveto{\pgfqpoint{2.472988in}{1.185218in}}{\pgfqpoint{2.477317in}{1.195669in}}{\pgfqpoint{2.477317in}{1.206564in}}%
\pgfpathcurveto{\pgfqpoint{2.477317in}{1.217460in}}{\pgfqpoint{2.472988in}{1.227911in}}{\pgfqpoint{2.465284in}{1.235615in}}%
\pgfpathcurveto{\pgfqpoint{2.457579in}{1.243319in}}{\pgfqpoint{2.447129in}{1.247648in}}{\pgfqpoint{2.436233in}{1.247648in}}%
\pgfpathcurveto{\pgfqpoint{2.425337in}{1.247648in}}{\pgfqpoint{2.414887in}{1.243319in}}{\pgfqpoint{2.407182in}{1.235615in}}%
\pgfpathcurveto{\pgfqpoint{2.399478in}{1.227911in}}{\pgfqpoint{2.395149in}{1.217460in}}{\pgfqpoint{2.395149in}{1.206564in}}%
\pgfpathcurveto{\pgfqpoint{2.395149in}{1.195669in}}{\pgfqpoint{2.399478in}{1.185218in}}{\pgfqpoint{2.407182in}{1.177514in}}%
\pgfpathcurveto{\pgfqpoint{2.414887in}{1.169809in}}{\pgfqpoint{2.425337in}{1.165480in}}{\pgfqpoint{2.436233in}{1.165480in}}%
\pgfpathlineto{\pgfqpoint{2.436233in}{1.165480in}}%
\pgfpathclose%
\pgfusepath{stroke}%
\end{pgfscope}%
\begin{pgfscope}%
\pgfpathrectangle{\pgfqpoint{0.688192in}{0.670138in}}{\pgfqpoint{7.111808in}{5.061530in}}%
\pgfusepath{clip}%
\pgfsetbuttcap%
\pgfsetroundjoin%
\pgfsetlinewidth{1.003750pt}%
\definecolor{currentstroke}{rgb}{0.000000,0.000000,0.000000}%
\pgfsetstrokecolor{currentstroke}%
\pgfsetdash{}{0pt}%
\pgfpathmoveto{\pgfqpoint{2.782315in}{1.300386in}}%
\pgfpathcurveto{\pgfqpoint{2.793210in}{1.300386in}}{\pgfqpoint{2.803661in}{1.304715in}}{\pgfqpoint{2.811365in}{1.312419in}}%
\pgfpathcurveto{\pgfqpoint{2.819070in}{1.320123in}}{\pgfqpoint{2.823399in}{1.330574in}}{\pgfqpoint{2.823399in}{1.341470in}}%
\pgfpathcurveto{\pgfqpoint{2.823399in}{1.352365in}}{\pgfqpoint{2.819070in}{1.362816in}}{\pgfqpoint{2.811365in}{1.370520in}}%
\pgfpathcurveto{\pgfqpoint{2.803661in}{1.378225in}}{\pgfqpoint{2.793210in}{1.382554in}}{\pgfqpoint{2.782315in}{1.382554in}}%
\pgfpathcurveto{\pgfqpoint{2.771419in}{1.382554in}}{\pgfqpoint{2.760968in}{1.378225in}}{\pgfqpoint{2.753264in}{1.370520in}}%
\pgfpathcurveto{\pgfqpoint{2.745560in}{1.362816in}}{\pgfqpoint{2.741231in}{1.352365in}}{\pgfqpoint{2.741231in}{1.341470in}}%
\pgfpathcurveto{\pgfqpoint{2.741231in}{1.330574in}}{\pgfqpoint{2.745560in}{1.320123in}}{\pgfqpoint{2.753264in}{1.312419in}}%
\pgfpathcurveto{\pgfqpoint{2.760968in}{1.304715in}}{\pgfqpoint{2.771419in}{1.300386in}}{\pgfqpoint{2.782315in}{1.300386in}}%
\pgfpathlineto{\pgfqpoint{2.782315in}{1.300386in}}%
\pgfpathclose%
\pgfusepath{stroke}%
\end{pgfscope}%
\begin{pgfscope}%
\pgfpathrectangle{\pgfqpoint{0.688192in}{0.670138in}}{\pgfqpoint{7.111808in}{5.061530in}}%
\pgfusepath{clip}%
\pgfsetbuttcap%
\pgfsetroundjoin%
\pgfsetlinewidth{1.003750pt}%
\definecolor{currentstroke}{rgb}{0.000000,0.000000,0.000000}%
\pgfsetstrokecolor{currentstroke}%
\pgfsetdash{}{0pt}%
\pgfpathmoveto{\pgfqpoint{3.213120in}{3.792894in}}%
\pgfpathcurveto{\pgfqpoint{3.224015in}{3.792894in}}{\pgfqpoint{3.234466in}{3.797222in}}{\pgfqpoint{3.242170in}{3.804927in}}%
\pgfpathcurveto{\pgfqpoint{3.249875in}{3.812631in}}{\pgfqpoint{3.254204in}{3.823082in}}{\pgfqpoint{3.254204in}{3.833977in}}%
\pgfpathcurveto{\pgfqpoint{3.254204in}{3.844873in}}{\pgfqpoint{3.249875in}{3.855324in}}{\pgfqpoint{3.242170in}{3.863028in}}%
\pgfpathcurveto{\pgfqpoint{3.234466in}{3.870733in}}{\pgfqpoint{3.224015in}{3.875061in}}{\pgfqpoint{3.213120in}{3.875061in}}%
\pgfpathcurveto{\pgfqpoint{3.202224in}{3.875061in}}{\pgfqpoint{3.191773in}{3.870733in}}{\pgfqpoint{3.184069in}{3.863028in}}%
\pgfpathcurveto{\pgfqpoint{3.176365in}{3.855324in}}{\pgfqpoint{3.172036in}{3.844873in}}{\pgfqpoint{3.172036in}{3.833977in}}%
\pgfpathcurveto{\pgfqpoint{3.172036in}{3.823082in}}{\pgfqpoint{3.176365in}{3.812631in}}{\pgfqpoint{3.184069in}{3.804927in}}%
\pgfpathcurveto{\pgfqpoint{3.191773in}{3.797222in}}{\pgfqpoint{3.202224in}{3.792894in}}{\pgfqpoint{3.213120in}{3.792894in}}%
\pgfpathlineto{\pgfqpoint{3.213120in}{3.792894in}}%
\pgfpathclose%
\pgfusepath{stroke}%
\end{pgfscope}%
\begin{pgfscope}%
\pgfpathrectangle{\pgfqpoint{0.688192in}{0.670138in}}{\pgfqpoint{7.111808in}{5.061530in}}%
\pgfusepath{clip}%
\pgfsetbuttcap%
\pgfsetroundjoin%
\pgfsetlinewidth{1.003750pt}%
\definecolor{currentstroke}{rgb}{0.000000,0.000000,0.000000}%
\pgfsetstrokecolor{currentstroke}%
\pgfsetdash{}{0pt}%
\pgfpathmoveto{\pgfqpoint{1.694175in}{2.535650in}}%
\pgfpathcurveto{\pgfqpoint{1.705070in}{2.535650in}}{\pgfqpoint{1.715521in}{2.539979in}}{\pgfqpoint{1.723225in}{2.547684in}}%
\pgfpathcurveto{\pgfqpoint{1.730930in}{2.555388in}}{\pgfqpoint{1.735258in}{2.565839in}}{\pgfqpoint{1.735258in}{2.576734in}}%
\pgfpathcurveto{\pgfqpoint{1.735258in}{2.587630in}}{\pgfqpoint{1.730930in}{2.598081in}}{\pgfqpoint{1.723225in}{2.605785in}}%
\pgfpathcurveto{\pgfqpoint{1.715521in}{2.613489in}}{\pgfqpoint{1.705070in}{2.617818in}}{\pgfqpoint{1.694175in}{2.617818in}}%
\pgfpathcurveto{\pgfqpoint{1.683279in}{2.617818in}}{\pgfqpoint{1.672828in}{2.613489in}}{\pgfqpoint{1.665124in}{2.605785in}}%
\pgfpathcurveto{\pgfqpoint{1.657419in}{2.598081in}}{\pgfqpoint{1.653091in}{2.587630in}}{\pgfqpoint{1.653091in}{2.576734in}}%
\pgfpathcurveto{\pgfqpoint{1.653091in}{2.565839in}}{\pgfqpoint{1.657419in}{2.555388in}}{\pgfqpoint{1.665124in}{2.547684in}}%
\pgfpathcurveto{\pgfqpoint{1.672828in}{2.539979in}}{\pgfqpoint{1.683279in}{2.535650in}}{\pgfqpoint{1.694175in}{2.535650in}}%
\pgfpathlineto{\pgfqpoint{1.694175in}{2.535650in}}%
\pgfpathclose%
\pgfusepath{stroke}%
\end{pgfscope}%
\begin{pgfscope}%
\pgfpathrectangle{\pgfqpoint{0.688192in}{0.670138in}}{\pgfqpoint{7.111808in}{5.061530in}}%
\pgfusepath{clip}%
\pgfsetbuttcap%
\pgfsetroundjoin%
\pgfsetlinewidth{1.003750pt}%
\definecolor{currentstroke}{rgb}{0.000000,0.000000,0.000000}%
\pgfsetstrokecolor{currentstroke}%
\pgfsetdash{}{0pt}%
\pgfpathmoveto{\pgfqpoint{5.607911in}{0.768226in}}%
\pgfpathcurveto{\pgfqpoint{5.618807in}{0.768226in}}{\pgfqpoint{5.629258in}{0.772555in}}{\pgfqpoint{5.636962in}{0.780259in}}%
\pgfpathcurveto{\pgfqpoint{5.644666in}{0.787964in}}{\pgfqpoint{5.648995in}{0.798415in}}{\pgfqpoint{5.648995in}{0.809310in}}%
\pgfpathcurveto{\pgfqpoint{5.648995in}{0.820206in}}{\pgfqpoint{5.644666in}{0.830656in}}{\pgfqpoint{5.636962in}{0.838361in}}%
\pgfpathcurveto{\pgfqpoint{5.629258in}{0.846065in}}{\pgfqpoint{5.618807in}{0.850394in}}{\pgfqpoint{5.607911in}{0.850394in}}%
\pgfpathcurveto{\pgfqpoint{5.597016in}{0.850394in}}{\pgfqpoint{5.586565in}{0.846065in}}{\pgfqpoint{5.578860in}{0.838361in}}%
\pgfpathcurveto{\pgfqpoint{5.571156in}{0.830656in}}{\pgfqpoint{5.566827in}{0.820206in}}{\pgfqpoint{5.566827in}{0.809310in}}%
\pgfpathcurveto{\pgfqpoint{5.566827in}{0.798415in}}{\pgfqpoint{5.571156in}{0.787964in}}{\pgfqpoint{5.578860in}{0.780259in}}%
\pgfpathcurveto{\pgfqpoint{5.586565in}{0.772555in}}{\pgfqpoint{5.597016in}{0.768226in}}{\pgfqpoint{5.607911in}{0.768226in}}%
\pgfpathlineto{\pgfqpoint{5.607911in}{0.768226in}}%
\pgfpathclose%
\pgfusepath{stroke}%
\end{pgfscope}%
\begin{pgfscope}%
\pgfpathrectangle{\pgfqpoint{0.688192in}{0.670138in}}{\pgfqpoint{7.111808in}{5.061530in}}%
\pgfusepath{clip}%
\pgfsetbuttcap%
\pgfsetroundjoin%
\pgfsetlinewidth{1.003750pt}%
\definecolor{currentstroke}{rgb}{0.000000,0.000000,0.000000}%
\pgfsetstrokecolor{currentstroke}%
\pgfsetdash{}{0pt}%
\pgfpathmoveto{\pgfqpoint{0.875052in}{1.726055in}}%
\pgfpathcurveto{\pgfqpoint{0.885947in}{1.726055in}}{\pgfqpoint{0.896398in}{1.730384in}}{\pgfqpoint{0.904102in}{1.738088in}}%
\pgfpathcurveto{\pgfqpoint{0.911807in}{1.745793in}}{\pgfqpoint{0.916136in}{1.756243in}}{\pgfqpoint{0.916136in}{1.767139in}}%
\pgfpathcurveto{\pgfqpoint{0.916136in}{1.778035in}}{\pgfqpoint{0.911807in}{1.788485in}}{\pgfqpoint{0.904102in}{1.796190in}}%
\pgfpathcurveto{\pgfqpoint{0.896398in}{1.803894in}}{\pgfqpoint{0.885947in}{1.808223in}}{\pgfqpoint{0.875052in}{1.808223in}}%
\pgfpathcurveto{\pgfqpoint{0.864156in}{1.808223in}}{\pgfqpoint{0.853705in}{1.803894in}}{\pgfqpoint{0.846001in}{1.796190in}}%
\pgfpathcurveto{\pgfqpoint{0.838297in}{1.788485in}}{\pgfqpoint{0.833968in}{1.778035in}}{\pgfqpoint{0.833968in}{1.767139in}}%
\pgfpathcurveto{\pgfqpoint{0.833968in}{1.756243in}}{\pgfqpoint{0.838297in}{1.745793in}}{\pgfqpoint{0.846001in}{1.738088in}}%
\pgfpathcurveto{\pgfqpoint{0.853705in}{1.730384in}}{\pgfqpoint{0.864156in}{1.726055in}}{\pgfqpoint{0.875052in}{1.726055in}}%
\pgfpathlineto{\pgfqpoint{0.875052in}{1.726055in}}%
\pgfpathclose%
\pgfusepath{stroke}%
\end{pgfscope}%
\begin{pgfscope}%
\pgfpathrectangle{\pgfqpoint{0.688192in}{0.670138in}}{\pgfqpoint{7.111808in}{5.061530in}}%
\pgfusepath{clip}%
\pgfsetbuttcap%
\pgfsetroundjoin%
\pgfsetlinewidth{1.003750pt}%
\definecolor{currentstroke}{rgb}{0.000000,0.000000,0.000000}%
\pgfsetstrokecolor{currentstroke}%
\pgfsetdash{}{0pt}%
\pgfpathmoveto{\pgfqpoint{1.761288in}{0.640905in}}%
\pgfpathcurveto{\pgfqpoint{1.772184in}{0.640905in}}{\pgfqpoint{1.782635in}{0.645234in}}{\pgfqpoint{1.790339in}{0.652938in}}%
\pgfpathcurveto{\pgfqpoint{1.798043in}{0.660642in}}{\pgfqpoint{1.802372in}{0.671093in}}{\pgfqpoint{1.802372in}{0.681989in}}%
\pgfpathcurveto{\pgfqpoint{1.802372in}{0.692884in}}{\pgfqpoint{1.798043in}{0.703335in}}{\pgfqpoint{1.790339in}{0.711039in}}%
\pgfpathcurveto{\pgfqpoint{1.782635in}{0.718744in}}{\pgfqpoint{1.772184in}{0.723073in}}{\pgfqpoint{1.761288in}{0.723073in}}%
\pgfpathcurveto{\pgfqpoint{1.750393in}{0.723073in}}{\pgfqpoint{1.739942in}{0.718744in}}{\pgfqpoint{1.732237in}{0.711039in}}%
\pgfpathcurveto{\pgfqpoint{1.724533in}{0.703335in}}{\pgfqpoint{1.720204in}{0.692884in}}{\pgfqpoint{1.720204in}{0.681989in}}%
\pgfpathcurveto{\pgfqpoint{1.720204in}{0.671093in}}{\pgfqpoint{1.724533in}{0.660642in}}{\pgfqpoint{1.732237in}{0.652938in}}%
\pgfpathcurveto{\pgfqpoint{1.739942in}{0.645234in}}{\pgfqpoint{1.750393in}{0.640905in}}{\pgfqpoint{1.761288in}{0.640905in}}%
\pgfusepath{stroke}%
\end{pgfscope}%
\begin{pgfscope}%
\pgfpathrectangle{\pgfqpoint{0.688192in}{0.670138in}}{\pgfqpoint{7.111808in}{5.061530in}}%
\pgfusepath{clip}%
\pgfsetbuttcap%
\pgfsetroundjoin%
\pgfsetlinewidth{1.003750pt}%
\definecolor{currentstroke}{rgb}{0.000000,0.000000,0.000000}%
\pgfsetstrokecolor{currentstroke}%
\pgfsetdash{}{0pt}%
\pgfpathmoveto{\pgfqpoint{4.576929in}{0.631060in}}%
\pgfpathcurveto{\pgfqpoint{4.587825in}{0.631060in}}{\pgfqpoint{4.598276in}{0.635389in}}{\pgfqpoint{4.605980in}{0.643094in}}%
\pgfpathcurveto{\pgfqpoint{4.613684in}{0.650798in}}{\pgfqpoint{4.618013in}{0.661249in}}{\pgfqpoint{4.618013in}{0.672144in}}%
\pgfpathcurveto{\pgfqpoint{4.618013in}{0.683040in}}{\pgfqpoint{4.613684in}{0.693491in}}{\pgfqpoint{4.605980in}{0.701195in}}%
\pgfpathcurveto{\pgfqpoint{4.598276in}{0.708899in}}{\pgfqpoint{4.587825in}{0.713228in}}{\pgfqpoint{4.576929in}{0.713228in}}%
\pgfpathcurveto{\pgfqpoint{4.566034in}{0.713228in}}{\pgfqpoint{4.555583in}{0.708899in}}{\pgfqpoint{4.547878in}{0.701195in}}%
\pgfpathcurveto{\pgfqpoint{4.540174in}{0.693491in}}{\pgfqpoint{4.535845in}{0.683040in}}{\pgfqpoint{4.535845in}{0.672144in}}%
\pgfpathcurveto{\pgfqpoint{4.535845in}{0.661249in}}{\pgfqpoint{4.540174in}{0.650798in}}{\pgfqpoint{4.547878in}{0.643094in}}%
\pgfpathcurveto{\pgfqpoint{4.555583in}{0.635389in}}{\pgfqpoint{4.566034in}{0.631060in}}{\pgfqpoint{4.576929in}{0.631060in}}%
\pgfusepath{stroke}%
\end{pgfscope}%
\begin{pgfscope}%
\pgfpathrectangle{\pgfqpoint{0.688192in}{0.670138in}}{\pgfqpoint{7.111808in}{5.061530in}}%
\pgfusepath{clip}%
\pgfsetbuttcap%
\pgfsetroundjoin%
\pgfsetlinewidth{1.003750pt}%
\definecolor{currentstroke}{rgb}{0.000000,0.000000,0.000000}%
\pgfsetstrokecolor{currentstroke}%
\pgfsetdash{}{0pt}%
\pgfpathmoveto{\pgfqpoint{1.128867in}{0.645415in}}%
\pgfpathcurveto{\pgfqpoint{1.139762in}{0.645415in}}{\pgfqpoint{1.150213in}{0.649744in}}{\pgfqpoint{1.157917in}{0.657449in}}%
\pgfpathcurveto{\pgfqpoint{1.165622in}{0.665153in}}{\pgfqpoint{1.169951in}{0.675604in}}{\pgfqpoint{1.169951in}{0.686499in}}%
\pgfpathcurveto{\pgfqpoint{1.169951in}{0.697395in}}{\pgfqpoint{1.165622in}{0.707846in}}{\pgfqpoint{1.157917in}{0.715550in}}%
\pgfpathcurveto{\pgfqpoint{1.150213in}{0.723254in}}{\pgfqpoint{1.139762in}{0.727583in}}{\pgfqpoint{1.128867in}{0.727583in}}%
\pgfpathcurveto{\pgfqpoint{1.117971in}{0.727583in}}{\pgfqpoint{1.107520in}{0.723254in}}{\pgfqpoint{1.099816in}{0.715550in}}%
\pgfpathcurveto{\pgfqpoint{1.092112in}{0.707846in}}{\pgfqpoint{1.087783in}{0.697395in}}{\pgfqpoint{1.087783in}{0.686499in}}%
\pgfpathcurveto{\pgfqpoint{1.087783in}{0.675604in}}{\pgfqpoint{1.092112in}{0.665153in}}{\pgfqpoint{1.099816in}{0.657449in}}%
\pgfpathcurveto{\pgfqpoint{1.107520in}{0.649744in}}{\pgfqpoint{1.117971in}{0.645415in}}{\pgfqpoint{1.128867in}{0.645415in}}%
\pgfusepath{stroke}%
\end{pgfscope}%
\begin{pgfscope}%
\pgfpathrectangle{\pgfqpoint{0.688192in}{0.670138in}}{\pgfqpoint{7.111808in}{5.061530in}}%
\pgfusepath{clip}%
\pgfsetbuttcap%
\pgfsetroundjoin%
\pgfsetlinewidth{1.003750pt}%
\definecolor{currentstroke}{rgb}{0.000000,0.000000,0.000000}%
\pgfsetstrokecolor{currentstroke}%
\pgfsetdash{}{0pt}%
\pgfpathmoveto{\pgfqpoint{4.965126in}{0.630283in}}%
\pgfpathcurveto{\pgfqpoint{4.976022in}{0.630283in}}{\pgfqpoint{4.986472in}{0.634611in}}{\pgfqpoint{4.994177in}{0.642316in}}%
\pgfpathcurveto{\pgfqpoint{5.001881in}{0.650020in}}{\pgfqpoint{5.006210in}{0.660471in}}{\pgfqpoint{5.006210in}{0.671366in}}%
\pgfpathcurveto{\pgfqpoint{5.006210in}{0.682262in}}{\pgfqpoint{5.001881in}{0.692713in}}{\pgfqpoint{4.994177in}{0.700417in}}%
\pgfpathcurveto{\pgfqpoint{4.986472in}{0.708122in}}{\pgfqpoint{4.976022in}{0.712450in}}{\pgfqpoint{4.965126in}{0.712450in}}%
\pgfpathcurveto{\pgfqpoint{4.954230in}{0.712450in}}{\pgfqpoint{4.943780in}{0.708122in}}{\pgfqpoint{4.936075in}{0.700417in}}%
\pgfpathcurveto{\pgfqpoint{4.928371in}{0.692713in}}{\pgfqpoint{4.924042in}{0.682262in}}{\pgfqpoint{4.924042in}{0.671366in}}%
\pgfpathcurveto{\pgfqpoint{4.924042in}{0.660471in}}{\pgfqpoint{4.928371in}{0.650020in}}{\pgfqpoint{4.936075in}{0.642316in}}%
\pgfpathcurveto{\pgfqpoint{4.943780in}{0.634611in}}{\pgfqpoint{4.954230in}{0.630283in}}{\pgfqpoint{4.965126in}{0.630283in}}%
\pgfusepath{stroke}%
\end{pgfscope}%
\begin{pgfscope}%
\pgfpathrectangle{\pgfqpoint{0.688192in}{0.670138in}}{\pgfqpoint{7.111808in}{5.061530in}}%
\pgfusepath{clip}%
\pgfsetbuttcap%
\pgfsetroundjoin%
\pgfsetlinewidth{1.003750pt}%
\definecolor{currentstroke}{rgb}{0.000000,0.000000,0.000000}%
\pgfsetstrokecolor{currentstroke}%
\pgfsetdash{}{0pt}%
\pgfpathmoveto{\pgfqpoint{2.056022in}{1.075198in}}%
\pgfpathcurveto{\pgfqpoint{2.066917in}{1.075198in}}{\pgfqpoint{2.077368in}{1.079527in}}{\pgfqpoint{2.085072in}{1.087231in}}%
\pgfpathcurveto{\pgfqpoint{2.092777in}{1.094936in}}{\pgfqpoint{2.097106in}{1.105386in}}{\pgfqpoint{2.097106in}{1.116282in}}%
\pgfpathcurveto{\pgfqpoint{2.097106in}{1.127178in}}{\pgfqpoint{2.092777in}{1.137628in}}{\pgfqpoint{2.085072in}{1.145333in}}%
\pgfpathcurveto{\pgfqpoint{2.077368in}{1.153037in}}{\pgfqpoint{2.066917in}{1.157366in}}{\pgfqpoint{2.056022in}{1.157366in}}%
\pgfpathcurveto{\pgfqpoint{2.045126in}{1.157366in}}{\pgfqpoint{2.034675in}{1.153037in}}{\pgfqpoint{2.026971in}{1.145333in}}%
\pgfpathcurveto{\pgfqpoint{2.019267in}{1.137628in}}{\pgfqpoint{2.014938in}{1.127178in}}{\pgfqpoint{2.014938in}{1.116282in}}%
\pgfpathcurveto{\pgfqpoint{2.014938in}{1.105386in}}{\pgfqpoint{2.019267in}{1.094936in}}{\pgfqpoint{2.026971in}{1.087231in}}%
\pgfpathcurveto{\pgfqpoint{2.034675in}{1.079527in}}{\pgfqpoint{2.045126in}{1.075198in}}{\pgfqpoint{2.056022in}{1.075198in}}%
\pgfpathlineto{\pgfqpoint{2.056022in}{1.075198in}}%
\pgfpathclose%
\pgfusepath{stroke}%
\end{pgfscope}%
\begin{pgfscope}%
\pgfpathrectangle{\pgfqpoint{0.688192in}{0.670138in}}{\pgfqpoint{7.111808in}{5.061530in}}%
\pgfusepath{clip}%
\pgfsetbuttcap%
\pgfsetroundjoin%
\pgfsetlinewidth{1.003750pt}%
\definecolor{currentstroke}{rgb}{0.000000,0.000000,0.000000}%
\pgfsetstrokecolor{currentstroke}%
\pgfsetdash{}{0pt}%
\pgfpathmoveto{\pgfqpoint{2.225009in}{0.638936in}}%
\pgfpathcurveto{\pgfqpoint{2.235905in}{0.638936in}}{\pgfqpoint{2.246356in}{0.643265in}}{\pgfqpoint{2.254060in}{0.650969in}}%
\pgfpathcurveto{\pgfqpoint{2.261764in}{0.658674in}}{\pgfqpoint{2.266093in}{0.669125in}}{\pgfqpoint{2.266093in}{0.680020in}}%
\pgfpathcurveto{\pgfqpoint{2.266093in}{0.690916in}}{\pgfqpoint{2.261764in}{0.701366in}}{\pgfqpoint{2.254060in}{0.709071in}}%
\pgfpathcurveto{\pgfqpoint{2.246356in}{0.716775in}}{\pgfqpoint{2.235905in}{0.721104in}}{\pgfqpoint{2.225009in}{0.721104in}}%
\pgfpathcurveto{\pgfqpoint{2.214114in}{0.721104in}}{\pgfqpoint{2.203663in}{0.716775in}}{\pgfqpoint{2.195959in}{0.709071in}}%
\pgfpathcurveto{\pgfqpoint{2.188254in}{0.701366in}}{\pgfqpoint{2.183926in}{0.690916in}}{\pgfqpoint{2.183926in}{0.680020in}}%
\pgfpathcurveto{\pgfqpoint{2.183926in}{0.669125in}}{\pgfqpoint{2.188254in}{0.658674in}}{\pgfqpoint{2.195959in}{0.650969in}}%
\pgfpathcurveto{\pgfqpoint{2.203663in}{0.643265in}}{\pgfqpoint{2.214114in}{0.638936in}}{\pgfqpoint{2.225009in}{0.638936in}}%
\pgfusepath{stroke}%
\end{pgfscope}%
\begin{pgfscope}%
\pgfpathrectangle{\pgfqpoint{0.688192in}{0.670138in}}{\pgfqpoint{7.111808in}{5.061530in}}%
\pgfusepath{clip}%
\pgfsetbuttcap%
\pgfsetroundjoin%
\pgfsetlinewidth{1.003750pt}%
\definecolor{currentstroke}{rgb}{0.000000,0.000000,0.000000}%
\pgfsetstrokecolor{currentstroke}%
\pgfsetdash{}{0pt}%
\pgfpathmoveto{\pgfqpoint{0.842310in}{0.704220in}}%
\pgfpathcurveto{\pgfqpoint{0.853205in}{0.704220in}}{\pgfqpoint{0.863656in}{0.708549in}}{\pgfqpoint{0.871361in}{0.716253in}}%
\pgfpathcurveto{\pgfqpoint{0.879065in}{0.723957in}}{\pgfqpoint{0.883394in}{0.734408in}}{\pgfqpoint{0.883394in}{0.745304in}}%
\pgfpathcurveto{\pgfqpoint{0.883394in}{0.756199in}}{\pgfqpoint{0.879065in}{0.766650in}}{\pgfqpoint{0.871361in}{0.774354in}}%
\pgfpathcurveto{\pgfqpoint{0.863656in}{0.782059in}}{\pgfqpoint{0.853205in}{0.786387in}}{\pgfqpoint{0.842310in}{0.786387in}}%
\pgfpathcurveto{\pgfqpoint{0.831414in}{0.786387in}}{\pgfqpoint{0.820964in}{0.782059in}}{\pgfqpoint{0.813259in}{0.774354in}}%
\pgfpathcurveto{\pgfqpoint{0.805555in}{0.766650in}}{\pgfqpoint{0.801226in}{0.756199in}}{\pgfqpoint{0.801226in}{0.745304in}}%
\pgfpathcurveto{\pgfqpoint{0.801226in}{0.734408in}}{\pgfqpoint{0.805555in}{0.723957in}}{\pgfqpoint{0.813259in}{0.716253in}}%
\pgfpathcurveto{\pgfqpoint{0.820964in}{0.708549in}}{\pgfqpoint{0.831414in}{0.704220in}}{\pgfqpoint{0.842310in}{0.704220in}}%
\pgfpathlineto{\pgfqpoint{0.842310in}{0.704220in}}%
\pgfpathclose%
\pgfusepath{stroke}%
\end{pgfscope}%
\begin{pgfscope}%
\pgfpathrectangle{\pgfqpoint{0.688192in}{0.670138in}}{\pgfqpoint{7.111808in}{5.061530in}}%
\pgfusepath{clip}%
\pgfsetbuttcap%
\pgfsetroundjoin%
\pgfsetlinewidth{1.003750pt}%
\definecolor{currentstroke}{rgb}{0.000000,0.000000,0.000000}%
\pgfsetstrokecolor{currentstroke}%
\pgfsetdash{}{0pt}%
\pgfpathmoveto{\pgfqpoint{0.854347in}{0.695577in}}%
\pgfpathcurveto{\pgfqpoint{0.865242in}{0.695577in}}{\pgfqpoint{0.875693in}{0.699906in}}{\pgfqpoint{0.883397in}{0.707611in}}%
\pgfpathcurveto{\pgfqpoint{0.891102in}{0.715315in}}{\pgfqpoint{0.895431in}{0.725766in}}{\pgfqpoint{0.895431in}{0.736661in}}%
\pgfpathcurveto{\pgfqpoint{0.895431in}{0.747557in}}{\pgfqpoint{0.891102in}{0.758008in}}{\pgfqpoint{0.883397in}{0.765712in}}%
\pgfpathcurveto{\pgfqpoint{0.875693in}{0.773416in}}{\pgfqpoint{0.865242in}{0.777745in}}{\pgfqpoint{0.854347in}{0.777745in}}%
\pgfpathcurveto{\pgfqpoint{0.843451in}{0.777745in}}{\pgfqpoint{0.833000in}{0.773416in}}{\pgfqpoint{0.825296in}{0.765712in}}%
\pgfpathcurveto{\pgfqpoint{0.817592in}{0.758008in}}{\pgfqpoint{0.813263in}{0.747557in}}{\pgfqpoint{0.813263in}{0.736661in}}%
\pgfpathcurveto{\pgfqpoint{0.813263in}{0.725766in}}{\pgfqpoint{0.817592in}{0.715315in}}{\pgfqpoint{0.825296in}{0.707611in}}%
\pgfpathcurveto{\pgfqpoint{0.833000in}{0.699906in}}{\pgfqpoint{0.843451in}{0.695577in}}{\pgfqpoint{0.854347in}{0.695577in}}%
\pgfpathlineto{\pgfqpoint{0.854347in}{0.695577in}}%
\pgfpathclose%
\pgfusepath{stroke}%
\end{pgfscope}%
\begin{pgfscope}%
\pgfpathrectangle{\pgfqpoint{0.688192in}{0.670138in}}{\pgfqpoint{7.111808in}{5.061530in}}%
\pgfusepath{clip}%
\pgfsetbuttcap%
\pgfsetroundjoin%
\pgfsetlinewidth{1.003750pt}%
\definecolor{currentstroke}{rgb}{0.000000,0.000000,0.000000}%
\pgfsetstrokecolor{currentstroke}%
\pgfsetdash{}{0pt}%
\pgfpathmoveto{\pgfqpoint{0.954596in}{0.669135in}}%
\pgfpathcurveto{\pgfqpoint{0.965492in}{0.669135in}}{\pgfqpoint{0.975942in}{0.673464in}}{\pgfqpoint{0.983647in}{0.681169in}}%
\pgfpathcurveto{\pgfqpoint{0.991351in}{0.688873in}}{\pgfqpoint{0.995680in}{0.699324in}}{\pgfqpoint{0.995680in}{0.710219in}}%
\pgfpathcurveto{\pgfqpoint{0.995680in}{0.721115in}}{\pgfqpoint{0.991351in}{0.731566in}}{\pgfqpoint{0.983647in}{0.739270in}}%
\pgfpathcurveto{\pgfqpoint{0.975942in}{0.746974in}}{\pgfqpoint{0.965492in}{0.751303in}}{\pgfqpoint{0.954596in}{0.751303in}}%
\pgfpathcurveto{\pgfqpoint{0.943700in}{0.751303in}}{\pgfqpoint{0.933250in}{0.746974in}}{\pgfqpoint{0.925545in}{0.739270in}}%
\pgfpathcurveto{\pgfqpoint{0.917841in}{0.731566in}}{\pgfqpoint{0.913512in}{0.721115in}}{\pgfqpoint{0.913512in}{0.710219in}}%
\pgfpathcurveto{\pgfqpoint{0.913512in}{0.699324in}}{\pgfqpoint{0.917841in}{0.688873in}}{\pgfqpoint{0.925545in}{0.681169in}}%
\pgfpathcurveto{\pgfqpoint{0.933250in}{0.673464in}}{\pgfqpoint{0.943700in}{0.669135in}}{\pgfqpoint{0.954596in}{0.669135in}}%
\pgfpathlineto{\pgfqpoint{0.954596in}{0.669135in}}%
\pgfpathclose%
\pgfusepath{stroke}%
\end{pgfscope}%
\begin{pgfscope}%
\pgfpathrectangle{\pgfqpoint{0.688192in}{0.670138in}}{\pgfqpoint{7.111808in}{5.061530in}}%
\pgfusepath{clip}%
\pgfsetbuttcap%
\pgfsetroundjoin%
\pgfsetlinewidth{1.003750pt}%
\definecolor{currentstroke}{rgb}{0.000000,0.000000,0.000000}%
\pgfsetstrokecolor{currentstroke}%
\pgfsetdash{}{0pt}%
\pgfpathmoveto{\pgfqpoint{6.702450in}{3.493359in}}%
\pgfpathcurveto{\pgfqpoint{6.713346in}{3.493359in}}{\pgfqpoint{6.723797in}{3.497688in}}{\pgfqpoint{6.731501in}{3.505392in}}%
\pgfpathcurveto{\pgfqpoint{6.739205in}{3.513096in}}{\pgfqpoint{6.743534in}{3.523547in}}{\pgfqpoint{6.743534in}{3.534443in}}%
\pgfpathcurveto{\pgfqpoint{6.743534in}{3.545338in}}{\pgfqpoint{6.739205in}{3.555789in}}{\pgfqpoint{6.731501in}{3.563493in}}%
\pgfpathcurveto{\pgfqpoint{6.723797in}{3.571198in}}{\pgfqpoint{6.713346in}{3.575527in}}{\pgfqpoint{6.702450in}{3.575527in}}%
\pgfpathcurveto{\pgfqpoint{6.691555in}{3.575527in}}{\pgfqpoint{6.681104in}{3.571198in}}{\pgfqpoint{6.673399in}{3.563493in}}%
\pgfpathcurveto{\pgfqpoint{6.665695in}{3.555789in}}{\pgfqpoint{6.661366in}{3.545338in}}{\pgfqpoint{6.661366in}{3.534443in}}%
\pgfpathcurveto{\pgfqpoint{6.661366in}{3.523547in}}{\pgfqpoint{6.665695in}{3.513096in}}{\pgfqpoint{6.673399in}{3.505392in}}%
\pgfpathcurveto{\pgfqpoint{6.681104in}{3.497688in}}{\pgfqpoint{6.691555in}{3.493359in}}{\pgfqpoint{6.702450in}{3.493359in}}%
\pgfpathlineto{\pgfqpoint{6.702450in}{3.493359in}}%
\pgfpathclose%
\pgfusepath{stroke}%
\end{pgfscope}%
\begin{pgfscope}%
\pgfpathrectangle{\pgfqpoint{0.688192in}{0.670138in}}{\pgfqpoint{7.111808in}{5.061530in}}%
\pgfusepath{clip}%
\pgfsetbuttcap%
\pgfsetroundjoin%
\pgfsetlinewidth{1.003750pt}%
\definecolor{currentstroke}{rgb}{0.000000,0.000000,0.000000}%
\pgfsetstrokecolor{currentstroke}%
\pgfsetdash{}{0pt}%
\pgfpathmoveto{\pgfqpoint{4.661935in}{0.630846in}}%
\pgfpathcurveto{\pgfqpoint{4.672830in}{0.630846in}}{\pgfqpoint{4.683281in}{0.635175in}}{\pgfqpoint{4.690985in}{0.642879in}}%
\pgfpathcurveto{\pgfqpoint{4.698690in}{0.650583in}}{\pgfqpoint{4.703018in}{0.661034in}}{\pgfqpoint{4.703018in}{0.671930in}}%
\pgfpathcurveto{\pgfqpoint{4.703018in}{0.682825in}}{\pgfqpoint{4.698690in}{0.693276in}}{\pgfqpoint{4.690985in}{0.700980in}}%
\pgfpathcurveto{\pgfqpoint{4.683281in}{0.708685in}}{\pgfqpoint{4.672830in}{0.713014in}}{\pgfqpoint{4.661935in}{0.713014in}}%
\pgfpathcurveto{\pgfqpoint{4.651039in}{0.713014in}}{\pgfqpoint{4.640588in}{0.708685in}}{\pgfqpoint{4.632884in}{0.700980in}}%
\pgfpathcurveto{\pgfqpoint{4.625180in}{0.693276in}}{\pgfqpoint{4.620851in}{0.682825in}}{\pgfqpoint{4.620851in}{0.671930in}}%
\pgfpathcurveto{\pgfqpoint{4.620851in}{0.661034in}}{\pgfqpoint{4.625180in}{0.650583in}}{\pgfqpoint{4.632884in}{0.642879in}}%
\pgfpathcurveto{\pgfqpoint{4.640588in}{0.635175in}}{\pgfqpoint{4.651039in}{0.630846in}}{\pgfqpoint{4.661935in}{0.630846in}}%
\pgfusepath{stroke}%
\end{pgfscope}%
\begin{pgfscope}%
\pgfpathrectangle{\pgfqpoint{0.688192in}{0.670138in}}{\pgfqpoint{7.111808in}{5.061530in}}%
\pgfusepath{clip}%
\pgfsetbuttcap%
\pgfsetroundjoin%
\pgfsetlinewidth{1.003750pt}%
\definecolor{currentstroke}{rgb}{0.000000,0.000000,0.000000}%
\pgfsetstrokecolor{currentstroke}%
\pgfsetdash{}{0pt}%
\pgfpathmoveto{\pgfqpoint{5.142949in}{0.659582in}}%
\pgfpathcurveto{\pgfqpoint{5.153844in}{0.659582in}}{\pgfqpoint{5.164295in}{0.663911in}}{\pgfqpoint{5.172000in}{0.671616in}}%
\pgfpathcurveto{\pgfqpoint{5.179704in}{0.679320in}}{\pgfqpoint{5.184033in}{0.689771in}}{\pgfqpoint{5.184033in}{0.700666in}}%
\pgfpathcurveto{\pgfqpoint{5.184033in}{0.711562in}}{\pgfqpoint{5.179704in}{0.722013in}}{\pgfqpoint{5.172000in}{0.729717in}}%
\pgfpathcurveto{\pgfqpoint{5.164295in}{0.737421in}}{\pgfqpoint{5.153844in}{0.741750in}}{\pgfqpoint{5.142949in}{0.741750in}}%
\pgfpathcurveto{\pgfqpoint{5.132053in}{0.741750in}}{\pgfqpoint{5.121603in}{0.737421in}}{\pgfqpoint{5.113898in}{0.729717in}}%
\pgfpathcurveto{\pgfqpoint{5.106194in}{0.722013in}}{\pgfqpoint{5.101865in}{0.711562in}}{\pgfqpoint{5.101865in}{0.700666in}}%
\pgfpathcurveto{\pgfqpoint{5.101865in}{0.689771in}}{\pgfqpoint{5.106194in}{0.679320in}}{\pgfqpoint{5.113898in}{0.671616in}}%
\pgfpathcurveto{\pgfqpoint{5.121603in}{0.663911in}}{\pgfqpoint{5.132053in}{0.659582in}}{\pgfqpoint{5.142949in}{0.659582in}}%
\pgfusepath{stroke}%
\end{pgfscope}%
\begin{pgfscope}%
\pgfpathrectangle{\pgfqpoint{0.688192in}{0.670138in}}{\pgfqpoint{7.111808in}{5.061530in}}%
\pgfusepath{clip}%
\pgfsetbuttcap%
\pgfsetroundjoin%
\pgfsetlinewidth{1.003750pt}%
\definecolor{currentstroke}{rgb}{0.000000,0.000000,0.000000}%
\pgfsetstrokecolor{currentstroke}%
\pgfsetdash{}{0pt}%
\pgfpathmoveto{\pgfqpoint{2.801640in}{0.701110in}}%
\pgfpathcurveto{\pgfqpoint{2.812536in}{0.701110in}}{\pgfqpoint{2.822986in}{0.705439in}}{\pgfqpoint{2.830691in}{0.713143in}}%
\pgfpathcurveto{\pgfqpoint{2.838395in}{0.720847in}}{\pgfqpoint{2.842724in}{0.731298in}}{\pgfqpoint{2.842724in}{0.742194in}}%
\pgfpathcurveto{\pgfqpoint{2.842724in}{0.753089in}}{\pgfqpoint{2.838395in}{0.763540in}}{\pgfqpoint{2.830691in}{0.771244in}}%
\pgfpathcurveto{\pgfqpoint{2.822986in}{0.778949in}}{\pgfqpoint{2.812536in}{0.783277in}}{\pgfqpoint{2.801640in}{0.783277in}}%
\pgfpathcurveto{\pgfqpoint{2.790745in}{0.783277in}}{\pgfqpoint{2.780294in}{0.778949in}}{\pgfqpoint{2.772589in}{0.771244in}}%
\pgfpathcurveto{\pgfqpoint{2.764885in}{0.763540in}}{\pgfqpoint{2.760556in}{0.753089in}}{\pgfqpoint{2.760556in}{0.742194in}}%
\pgfpathcurveto{\pgfqpoint{2.760556in}{0.731298in}}{\pgfqpoint{2.764885in}{0.720847in}}{\pgfqpoint{2.772589in}{0.713143in}}%
\pgfpathcurveto{\pgfqpoint{2.780294in}{0.705439in}}{\pgfqpoint{2.790745in}{0.701110in}}{\pgfqpoint{2.801640in}{0.701110in}}%
\pgfpathlineto{\pgfqpoint{2.801640in}{0.701110in}}%
\pgfpathclose%
\pgfusepath{stroke}%
\end{pgfscope}%
\begin{pgfscope}%
\pgfpathrectangle{\pgfqpoint{0.688192in}{0.670138in}}{\pgfqpoint{7.111808in}{5.061530in}}%
\pgfusepath{clip}%
\pgfsetbuttcap%
\pgfsetroundjoin%
\pgfsetlinewidth{1.003750pt}%
\definecolor{currentstroke}{rgb}{0.000000,0.000000,0.000000}%
\pgfsetstrokecolor{currentstroke}%
\pgfsetdash{}{0pt}%
\pgfpathmoveto{\pgfqpoint{3.949410in}{0.673831in}}%
\pgfpathcurveto{\pgfqpoint{3.960305in}{0.673831in}}{\pgfqpoint{3.970756in}{0.678159in}}{\pgfqpoint{3.978460in}{0.685864in}}%
\pgfpathcurveto{\pgfqpoint{3.986165in}{0.693568in}}{\pgfqpoint{3.990493in}{0.704019in}}{\pgfqpoint{3.990493in}{0.714915in}}%
\pgfpathcurveto{\pgfqpoint{3.990493in}{0.725810in}}{\pgfqpoint{3.986165in}{0.736261in}}{\pgfqpoint{3.978460in}{0.743965in}}%
\pgfpathcurveto{\pgfqpoint{3.970756in}{0.751670in}}{\pgfqpoint{3.960305in}{0.755998in}}{\pgfqpoint{3.949410in}{0.755998in}}%
\pgfpathcurveto{\pgfqpoint{3.938514in}{0.755998in}}{\pgfqpoint{3.928063in}{0.751670in}}{\pgfqpoint{3.920359in}{0.743965in}}%
\pgfpathcurveto{\pgfqpoint{3.912655in}{0.736261in}}{\pgfqpoint{3.908326in}{0.725810in}}{\pgfqpoint{3.908326in}{0.714915in}}%
\pgfpathcurveto{\pgfqpoint{3.908326in}{0.704019in}}{\pgfqpoint{3.912655in}{0.693568in}}{\pgfqpoint{3.920359in}{0.685864in}}%
\pgfpathcurveto{\pgfqpoint{3.928063in}{0.678159in}}{\pgfqpoint{3.938514in}{0.673831in}}{\pgfqpoint{3.949410in}{0.673831in}}%
\pgfpathlineto{\pgfqpoint{3.949410in}{0.673831in}}%
\pgfpathclose%
\pgfusepath{stroke}%
\end{pgfscope}%
\begin{pgfscope}%
\pgfpathrectangle{\pgfqpoint{0.688192in}{0.670138in}}{\pgfqpoint{7.111808in}{5.061530in}}%
\pgfusepath{clip}%
\pgfsetbuttcap%
\pgfsetroundjoin%
\pgfsetlinewidth{1.003750pt}%
\definecolor{currentstroke}{rgb}{0.000000,0.000000,0.000000}%
\pgfsetstrokecolor{currentstroke}%
\pgfsetdash{}{0pt}%
\pgfpathmoveto{\pgfqpoint{0.945763in}{0.669750in}}%
\pgfpathcurveto{\pgfqpoint{0.956659in}{0.669750in}}{\pgfqpoint{0.967110in}{0.674079in}}{\pgfqpoint{0.974814in}{0.681783in}}%
\pgfpathcurveto{\pgfqpoint{0.982518in}{0.689488in}}{\pgfqpoint{0.986847in}{0.699938in}}{\pgfqpoint{0.986847in}{0.710834in}}%
\pgfpathcurveto{\pgfqpoint{0.986847in}{0.721730in}}{\pgfqpoint{0.982518in}{0.732180in}}{\pgfqpoint{0.974814in}{0.739885in}}%
\pgfpathcurveto{\pgfqpoint{0.967110in}{0.747589in}}{\pgfqpoint{0.956659in}{0.751918in}}{\pgfqpoint{0.945763in}{0.751918in}}%
\pgfpathcurveto{\pgfqpoint{0.934868in}{0.751918in}}{\pgfqpoint{0.924417in}{0.747589in}}{\pgfqpoint{0.916713in}{0.739885in}}%
\pgfpathcurveto{\pgfqpoint{0.909008in}{0.732180in}}{\pgfqpoint{0.904679in}{0.721730in}}{\pgfqpoint{0.904679in}{0.710834in}}%
\pgfpathcurveto{\pgfqpoint{0.904679in}{0.699938in}}{\pgfqpoint{0.909008in}{0.689488in}}{\pgfqpoint{0.916713in}{0.681783in}}%
\pgfpathcurveto{\pgfqpoint{0.924417in}{0.674079in}}{\pgfqpoint{0.934868in}{0.669750in}}{\pgfqpoint{0.945763in}{0.669750in}}%
\pgfpathlineto{\pgfqpoint{0.945763in}{0.669750in}}%
\pgfpathclose%
\pgfusepath{stroke}%
\end{pgfscope}%
\begin{pgfscope}%
\pgfpathrectangle{\pgfqpoint{0.688192in}{0.670138in}}{\pgfqpoint{7.111808in}{5.061530in}}%
\pgfusepath{clip}%
\pgfsetbuttcap%
\pgfsetroundjoin%
\pgfsetlinewidth{1.003750pt}%
\definecolor{currentstroke}{rgb}{0.000000,0.000000,0.000000}%
\pgfsetstrokecolor{currentstroke}%
\pgfsetdash{}{0pt}%
\pgfpathmoveto{\pgfqpoint{4.213238in}{0.631182in}}%
\pgfpathcurveto{\pgfqpoint{4.224134in}{0.631182in}}{\pgfqpoint{4.234585in}{0.635511in}}{\pgfqpoint{4.242289in}{0.643215in}}%
\pgfpathcurveto{\pgfqpoint{4.249993in}{0.650919in}}{\pgfqpoint{4.254322in}{0.661370in}}{\pgfqpoint{4.254322in}{0.672266in}}%
\pgfpathcurveto{\pgfqpoint{4.254322in}{0.683161in}}{\pgfqpoint{4.249993in}{0.693612in}}{\pgfqpoint{4.242289in}{0.701317in}}%
\pgfpathcurveto{\pgfqpoint{4.234585in}{0.709021in}}{\pgfqpoint{4.224134in}{0.713350in}}{\pgfqpoint{4.213238in}{0.713350in}}%
\pgfpathcurveto{\pgfqpoint{4.202343in}{0.713350in}}{\pgfqpoint{4.191892in}{0.709021in}}{\pgfqpoint{4.184187in}{0.701317in}}%
\pgfpathcurveto{\pgfqpoint{4.176483in}{0.693612in}}{\pgfqpoint{4.172154in}{0.683161in}}{\pgfqpoint{4.172154in}{0.672266in}}%
\pgfpathcurveto{\pgfqpoint{4.172154in}{0.661370in}}{\pgfqpoint{4.176483in}{0.650919in}}{\pgfqpoint{4.184187in}{0.643215in}}%
\pgfpathcurveto{\pgfqpoint{4.191892in}{0.635511in}}{\pgfqpoint{4.202343in}{0.631182in}}{\pgfqpoint{4.213238in}{0.631182in}}%
\pgfusepath{stroke}%
\end{pgfscope}%
\begin{pgfscope}%
\pgfpathrectangle{\pgfqpoint{0.688192in}{0.670138in}}{\pgfqpoint{7.111808in}{5.061530in}}%
\pgfusepath{clip}%
\pgfsetbuttcap%
\pgfsetroundjoin%
\pgfsetlinewidth{1.003750pt}%
\definecolor{currentstroke}{rgb}{0.000000,0.000000,0.000000}%
\pgfsetstrokecolor{currentstroke}%
\pgfsetdash{}{0pt}%
\pgfpathmoveto{\pgfqpoint{1.898879in}{3.860723in}}%
\pgfpathcurveto{\pgfqpoint{1.909775in}{3.860723in}}{\pgfqpoint{1.920226in}{3.865052in}}{\pgfqpoint{1.927930in}{3.872756in}}%
\pgfpathcurveto{\pgfqpoint{1.935634in}{3.880461in}}{\pgfqpoint{1.939963in}{3.890911in}}{\pgfqpoint{1.939963in}{3.901807in}}%
\pgfpathcurveto{\pgfqpoint{1.939963in}{3.912703in}}{\pgfqpoint{1.935634in}{3.923153in}}{\pgfqpoint{1.927930in}{3.930858in}}%
\pgfpathcurveto{\pgfqpoint{1.920226in}{3.938562in}}{\pgfqpoint{1.909775in}{3.942891in}}{\pgfqpoint{1.898879in}{3.942891in}}%
\pgfpathcurveto{\pgfqpoint{1.887984in}{3.942891in}}{\pgfqpoint{1.877533in}{3.938562in}}{\pgfqpoint{1.869829in}{3.930858in}}%
\pgfpathcurveto{\pgfqpoint{1.862124in}{3.923153in}}{\pgfqpoint{1.857796in}{3.912703in}}{\pgfqpoint{1.857796in}{3.901807in}}%
\pgfpathcurveto{\pgfqpoint{1.857796in}{3.890911in}}{\pgfqpoint{1.862124in}{3.880461in}}{\pgfqpoint{1.869829in}{3.872756in}}%
\pgfpathcurveto{\pgfqpoint{1.877533in}{3.865052in}}{\pgfqpoint{1.887984in}{3.860723in}}{\pgfqpoint{1.898879in}{3.860723in}}%
\pgfpathlineto{\pgfqpoint{1.898879in}{3.860723in}}%
\pgfpathclose%
\pgfusepath{stroke}%
\end{pgfscope}%
\begin{pgfscope}%
\pgfpathrectangle{\pgfqpoint{0.688192in}{0.670138in}}{\pgfqpoint{7.111808in}{5.061530in}}%
\pgfusepath{clip}%
\pgfsetbuttcap%
\pgfsetroundjoin%
\pgfsetlinewidth{1.003750pt}%
\definecolor{currentstroke}{rgb}{0.000000,0.000000,0.000000}%
\pgfsetstrokecolor{currentstroke}%
\pgfsetdash{}{0pt}%
\pgfpathmoveto{\pgfqpoint{4.661935in}{0.630846in}}%
\pgfpathcurveto{\pgfqpoint{4.672830in}{0.630846in}}{\pgfqpoint{4.683281in}{0.635175in}}{\pgfqpoint{4.690985in}{0.642879in}}%
\pgfpathcurveto{\pgfqpoint{4.698690in}{0.650583in}}{\pgfqpoint{4.703018in}{0.661034in}}{\pgfqpoint{4.703018in}{0.671930in}}%
\pgfpathcurveto{\pgfqpoint{4.703018in}{0.682825in}}{\pgfqpoint{4.698690in}{0.693276in}}{\pgfqpoint{4.690985in}{0.700980in}}%
\pgfpathcurveto{\pgfqpoint{4.683281in}{0.708685in}}{\pgfqpoint{4.672830in}{0.713014in}}{\pgfqpoint{4.661935in}{0.713014in}}%
\pgfpathcurveto{\pgfqpoint{4.651039in}{0.713014in}}{\pgfqpoint{4.640588in}{0.708685in}}{\pgfqpoint{4.632884in}{0.700980in}}%
\pgfpathcurveto{\pgfqpoint{4.625180in}{0.693276in}}{\pgfqpoint{4.620851in}{0.682825in}}{\pgfqpoint{4.620851in}{0.671930in}}%
\pgfpathcurveto{\pgfqpoint{4.620851in}{0.661034in}}{\pgfqpoint{4.625180in}{0.650583in}}{\pgfqpoint{4.632884in}{0.642879in}}%
\pgfpathcurveto{\pgfqpoint{4.640588in}{0.635175in}}{\pgfqpoint{4.651039in}{0.630846in}}{\pgfqpoint{4.661935in}{0.630846in}}%
\pgfusepath{stroke}%
\end{pgfscope}%
\begin{pgfscope}%
\pgfpathrectangle{\pgfqpoint{0.688192in}{0.670138in}}{\pgfqpoint{7.111808in}{5.061530in}}%
\pgfusepath{clip}%
\pgfsetbuttcap%
\pgfsetroundjoin%
\pgfsetlinewidth{1.003750pt}%
\definecolor{currentstroke}{rgb}{0.000000,0.000000,0.000000}%
\pgfsetstrokecolor{currentstroke}%
\pgfsetdash{}{0pt}%
\pgfpathmoveto{\pgfqpoint{0.954596in}{0.669135in}}%
\pgfpathcurveto{\pgfqpoint{0.965492in}{0.669135in}}{\pgfqpoint{0.975942in}{0.673464in}}{\pgfqpoint{0.983647in}{0.681169in}}%
\pgfpathcurveto{\pgfqpoint{0.991351in}{0.688873in}}{\pgfqpoint{0.995680in}{0.699324in}}{\pgfqpoint{0.995680in}{0.710219in}}%
\pgfpathcurveto{\pgfqpoint{0.995680in}{0.721115in}}{\pgfqpoint{0.991351in}{0.731566in}}{\pgfqpoint{0.983647in}{0.739270in}}%
\pgfpathcurveto{\pgfqpoint{0.975942in}{0.746974in}}{\pgfqpoint{0.965492in}{0.751303in}}{\pgfqpoint{0.954596in}{0.751303in}}%
\pgfpathcurveto{\pgfqpoint{0.943700in}{0.751303in}}{\pgfqpoint{0.933250in}{0.746974in}}{\pgfqpoint{0.925545in}{0.739270in}}%
\pgfpathcurveto{\pgfqpoint{0.917841in}{0.731566in}}{\pgfqpoint{0.913512in}{0.721115in}}{\pgfqpoint{0.913512in}{0.710219in}}%
\pgfpathcurveto{\pgfqpoint{0.913512in}{0.699324in}}{\pgfqpoint{0.917841in}{0.688873in}}{\pgfqpoint{0.925545in}{0.681169in}}%
\pgfpathcurveto{\pgfqpoint{0.933250in}{0.673464in}}{\pgfqpoint{0.943700in}{0.669135in}}{\pgfqpoint{0.954596in}{0.669135in}}%
\pgfpathlineto{\pgfqpoint{0.954596in}{0.669135in}}%
\pgfpathclose%
\pgfusepath{stroke}%
\end{pgfscope}%
\begin{pgfscope}%
\pgfpathrectangle{\pgfqpoint{0.688192in}{0.670138in}}{\pgfqpoint{7.111808in}{5.061530in}}%
\pgfusepath{clip}%
\pgfsetbuttcap%
\pgfsetroundjoin%
\pgfsetlinewidth{1.003750pt}%
\definecolor{currentstroke}{rgb}{0.000000,0.000000,0.000000}%
\pgfsetstrokecolor{currentstroke}%
\pgfsetdash{}{0pt}%
\pgfpathmoveto{\pgfqpoint{5.812182in}{0.817903in}}%
\pgfpathcurveto{\pgfqpoint{5.823078in}{0.817903in}}{\pgfqpoint{5.833529in}{0.822232in}}{\pgfqpoint{5.841233in}{0.829936in}}%
\pgfpathcurveto{\pgfqpoint{5.848937in}{0.837640in}}{\pgfqpoint{5.853266in}{0.848091in}}{\pgfqpoint{5.853266in}{0.858987in}}%
\pgfpathcurveto{\pgfqpoint{5.853266in}{0.869882in}}{\pgfqpoint{5.848937in}{0.880333in}}{\pgfqpoint{5.841233in}{0.888037in}}%
\pgfpathcurveto{\pgfqpoint{5.833529in}{0.895742in}}{\pgfqpoint{5.823078in}{0.900070in}}{\pgfqpoint{5.812182in}{0.900070in}}%
\pgfpathcurveto{\pgfqpoint{5.801287in}{0.900070in}}{\pgfqpoint{5.790836in}{0.895742in}}{\pgfqpoint{5.783132in}{0.888037in}}%
\pgfpathcurveto{\pgfqpoint{5.775427in}{0.880333in}}{\pgfqpoint{5.771098in}{0.869882in}}{\pgfqpoint{5.771098in}{0.858987in}}%
\pgfpathcurveto{\pgfqpoint{5.771098in}{0.848091in}}{\pgfqpoint{5.775427in}{0.837640in}}{\pgfqpoint{5.783132in}{0.829936in}}%
\pgfpathcurveto{\pgfqpoint{5.790836in}{0.822232in}}{\pgfqpoint{5.801287in}{0.817903in}}{\pgfqpoint{5.812182in}{0.817903in}}%
\pgfpathlineto{\pgfqpoint{5.812182in}{0.817903in}}%
\pgfpathclose%
\pgfusepath{stroke}%
\end{pgfscope}%
\begin{pgfscope}%
\pgfpathrectangle{\pgfqpoint{0.688192in}{0.670138in}}{\pgfqpoint{7.111808in}{5.061530in}}%
\pgfusepath{clip}%
\pgfsetbuttcap%
\pgfsetroundjoin%
\pgfsetlinewidth{1.003750pt}%
\definecolor{currentstroke}{rgb}{0.000000,0.000000,0.000000}%
\pgfsetstrokecolor{currentstroke}%
\pgfsetdash{}{0pt}%
\pgfpathmoveto{\pgfqpoint{1.503177in}{0.642365in}}%
\pgfpathcurveto{\pgfqpoint{1.514073in}{0.642365in}}{\pgfqpoint{1.524523in}{0.646694in}}{\pgfqpoint{1.532228in}{0.654398in}}%
\pgfpathcurveto{\pgfqpoint{1.539932in}{0.662103in}}{\pgfqpoint{1.544261in}{0.672553in}}{\pgfqpoint{1.544261in}{0.683449in}}%
\pgfpathcurveto{\pgfqpoint{1.544261in}{0.694345in}}{\pgfqpoint{1.539932in}{0.704795in}}{\pgfqpoint{1.532228in}{0.712500in}}%
\pgfpathcurveto{\pgfqpoint{1.524523in}{0.720204in}}{\pgfqpoint{1.514073in}{0.724533in}}{\pgfqpoint{1.503177in}{0.724533in}}%
\pgfpathcurveto{\pgfqpoint{1.492281in}{0.724533in}}{\pgfqpoint{1.481831in}{0.720204in}}{\pgfqpoint{1.474126in}{0.712500in}}%
\pgfpathcurveto{\pgfqpoint{1.466422in}{0.704795in}}{\pgfqpoint{1.462093in}{0.694345in}}{\pgfqpoint{1.462093in}{0.683449in}}%
\pgfpathcurveto{\pgfqpoint{1.462093in}{0.672553in}}{\pgfqpoint{1.466422in}{0.662103in}}{\pgfqpoint{1.474126in}{0.654398in}}%
\pgfpathcurveto{\pgfqpoint{1.481831in}{0.646694in}}{\pgfqpoint{1.492281in}{0.642365in}}{\pgfqpoint{1.503177in}{0.642365in}}%
\pgfusepath{stroke}%
\end{pgfscope}%
\begin{pgfscope}%
\pgfpathrectangle{\pgfqpoint{0.688192in}{0.670138in}}{\pgfqpoint{7.111808in}{5.061530in}}%
\pgfusepath{clip}%
\pgfsetbuttcap%
\pgfsetroundjoin%
\pgfsetlinewidth{1.003750pt}%
\definecolor{currentstroke}{rgb}{0.000000,0.000000,0.000000}%
\pgfsetstrokecolor{currentstroke}%
\pgfsetdash{}{0pt}%
\pgfpathmoveto{\pgfqpoint{5.557238in}{0.629442in}}%
\pgfpathcurveto{\pgfqpoint{5.568133in}{0.629442in}}{\pgfqpoint{5.578584in}{0.633771in}}{\pgfqpoint{5.586288in}{0.641475in}}%
\pgfpathcurveto{\pgfqpoint{5.593993in}{0.649179in}}{\pgfqpoint{5.598322in}{0.659630in}}{\pgfqpoint{5.598322in}{0.670526in}}%
\pgfpathcurveto{\pgfqpoint{5.598322in}{0.681421in}}{\pgfqpoint{5.593993in}{0.691872in}}{\pgfqpoint{5.586288in}{0.699576in}}%
\pgfpathcurveto{\pgfqpoint{5.578584in}{0.707281in}}{\pgfqpoint{5.568133in}{0.711609in}}{\pgfqpoint{5.557238in}{0.711609in}}%
\pgfpathcurveto{\pgfqpoint{5.546342in}{0.711609in}}{\pgfqpoint{5.535891in}{0.707281in}}{\pgfqpoint{5.528187in}{0.699576in}}%
\pgfpathcurveto{\pgfqpoint{5.520483in}{0.691872in}}{\pgfqpoint{5.516154in}{0.681421in}}{\pgfqpoint{5.516154in}{0.670526in}}%
\pgfpathcurveto{\pgfqpoint{5.516154in}{0.659630in}}{\pgfqpoint{5.520483in}{0.649179in}}{\pgfqpoint{5.528187in}{0.641475in}}%
\pgfpathcurveto{\pgfqpoint{5.535891in}{0.633771in}}{\pgfqpoint{5.546342in}{0.629442in}}{\pgfqpoint{5.557238in}{0.629442in}}%
\pgfusepath{stroke}%
\end{pgfscope}%
\begin{pgfscope}%
\pgfpathrectangle{\pgfqpoint{0.688192in}{0.670138in}}{\pgfqpoint{7.111808in}{5.061530in}}%
\pgfusepath{clip}%
\pgfsetbuttcap%
\pgfsetroundjoin%
\pgfsetlinewidth{1.003750pt}%
\definecolor{currentstroke}{rgb}{0.000000,0.000000,0.000000}%
\pgfsetstrokecolor{currentstroke}%
\pgfsetdash{}{0pt}%
\pgfpathmoveto{\pgfqpoint{2.244939in}{0.638531in}}%
\pgfpathcurveto{\pgfqpoint{2.255835in}{0.638531in}}{\pgfqpoint{2.266286in}{0.642860in}}{\pgfqpoint{2.273990in}{0.650564in}}%
\pgfpathcurveto{\pgfqpoint{2.281695in}{0.658269in}}{\pgfqpoint{2.286023in}{0.668719in}}{\pgfqpoint{2.286023in}{0.679615in}}%
\pgfpathcurveto{\pgfqpoint{2.286023in}{0.690511in}}{\pgfqpoint{2.281695in}{0.700961in}}{\pgfqpoint{2.273990in}{0.708666in}}%
\pgfpathcurveto{\pgfqpoint{2.266286in}{0.716370in}}{\pgfqpoint{2.255835in}{0.720699in}}{\pgfqpoint{2.244939in}{0.720699in}}%
\pgfpathcurveto{\pgfqpoint{2.234044in}{0.720699in}}{\pgfqpoint{2.223593in}{0.716370in}}{\pgfqpoint{2.215889in}{0.708666in}}%
\pgfpathcurveto{\pgfqpoint{2.208184in}{0.700961in}}{\pgfqpoint{2.203856in}{0.690511in}}{\pgfqpoint{2.203856in}{0.679615in}}%
\pgfpathcurveto{\pgfqpoint{2.203856in}{0.668719in}}{\pgfqpoint{2.208184in}{0.658269in}}{\pgfqpoint{2.215889in}{0.650564in}}%
\pgfpathcurveto{\pgfqpoint{2.223593in}{0.642860in}}{\pgfqpoint{2.234044in}{0.638531in}}{\pgfqpoint{2.244939in}{0.638531in}}%
\pgfusepath{stroke}%
\end{pgfscope}%
\begin{pgfscope}%
\pgfpathrectangle{\pgfqpoint{0.688192in}{0.670138in}}{\pgfqpoint{7.111808in}{5.061530in}}%
\pgfusepath{clip}%
\pgfsetbuttcap%
\pgfsetroundjoin%
\pgfsetlinewidth{1.003750pt}%
\definecolor{currentstroke}{rgb}{0.000000,0.000000,0.000000}%
\pgfsetstrokecolor{currentstroke}%
\pgfsetdash{}{0pt}%
\pgfpathmoveto{\pgfqpoint{1.128607in}{0.645422in}}%
\pgfpathcurveto{\pgfqpoint{1.139503in}{0.645422in}}{\pgfqpoint{1.149954in}{0.649751in}}{\pgfqpoint{1.157658in}{0.657455in}}%
\pgfpathcurveto{\pgfqpoint{1.165362in}{0.665160in}}{\pgfqpoint{1.169691in}{0.675610in}}{\pgfqpoint{1.169691in}{0.686506in}}%
\pgfpathcurveto{\pgfqpoint{1.169691in}{0.697402in}}{\pgfqpoint{1.165362in}{0.707852in}}{\pgfqpoint{1.157658in}{0.715557in}}%
\pgfpathcurveto{\pgfqpoint{1.149954in}{0.723261in}}{\pgfqpoint{1.139503in}{0.727590in}}{\pgfqpoint{1.128607in}{0.727590in}}%
\pgfpathcurveto{\pgfqpoint{1.117712in}{0.727590in}}{\pgfqpoint{1.107261in}{0.723261in}}{\pgfqpoint{1.099556in}{0.715557in}}%
\pgfpathcurveto{\pgfqpoint{1.091852in}{0.707852in}}{\pgfqpoint{1.087523in}{0.697402in}}{\pgfqpoint{1.087523in}{0.686506in}}%
\pgfpathcurveto{\pgfqpoint{1.087523in}{0.675610in}}{\pgfqpoint{1.091852in}{0.665160in}}{\pgfqpoint{1.099556in}{0.657455in}}%
\pgfpathcurveto{\pgfqpoint{1.107261in}{0.649751in}}{\pgfqpoint{1.117712in}{0.645422in}}{\pgfqpoint{1.128607in}{0.645422in}}%
\pgfusepath{stroke}%
\end{pgfscope}%
\begin{pgfscope}%
\pgfpathrectangle{\pgfqpoint{0.688192in}{0.670138in}}{\pgfqpoint{7.111808in}{5.061530in}}%
\pgfusepath{clip}%
\pgfsetbuttcap%
\pgfsetroundjoin%
\pgfsetlinewidth{1.003750pt}%
\definecolor{currentstroke}{rgb}{0.000000,0.000000,0.000000}%
\pgfsetstrokecolor{currentstroke}%
\pgfsetdash{}{0pt}%
\pgfpathmoveto{\pgfqpoint{4.883398in}{0.731935in}}%
\pgfpathcurveto{\pgfqpoint{4.894294in}{0.731935in}}{\pgfqpoint{4.904745in}{0.736263in}}{\pgfqpoint{4.912449in}{0.743968in}}%
\pgfpathcurveto{\pgfqpoint{4.920153in}{0.751672in}}{\pgfqpoint{4.924482in}{0.762123in}}{\pgfqpoint{4.924482in}{0.773018in}}%
\pgfpathcurveto{\pgfqpoint{4.924482in}{0.783914in}}{\pgfqpoint{4.920153in}{0.794365in}}{\pgfqpoint{4.912449in}{0.802069in}}%
\pgfpathcurveto{\pgfqpoint{4.904745in}{0.809773in}}{\pgfqpoint{4.894294in}{0.814102in}}{\pgfqpoint{4.883398in}{0.814102in}}%
\pgfpathcurveto{\pgfqpoint{4.872503in}{0.814102in}}{\pgfqpoint{4.862052in}{0.809773in}}{\pgfqpoint{4.854348in}{0.802069in}}%
\pgfpathcurveto{\pgfqpoint{4.846643in}{0.794365in}}{\pgfqpoint{4.842314in}{0.783914in}}{\pgfqpoint{4.842314in}{0.773018in}}%
\pgfpathcurveto{\pgfqpoint{4.842314in}{0.762123in}}{\pgfqpoint{4.846643in}{0.751672in}}{\pgfqpoint{4.854348in}{0.743968in}}%
\pgfpathcurveto{\pgfqpoint{4.862052in}{0.736263in}}{\pgfqpoint{4.872503in}{0.731935in}}{\pgfqpoint{4.883398in}{0.731935in}}%
\pgfpathlineto{\pgfqpoint{4.883398in}{0.731935in}}%
\pgfpathclose%
\pgfusepath{stroke}%
\end{pgfscope}%
\begin{pgfscope}%
\pgfpathrectangle{\pgfqpoint{0.688192in}{0.670138in}}{\pgfqpoint{7.111808in}{5.061530in}}%
\pgfusepath{clip}%
\pgfsetbuttcap%
\pgfsetroundjoin%
\pgfsetlinewidth{1.003750pt}%
\definecolor{currentstroke}{rgb}{0.000000,0.000000,0.000000}%
\pgfsetstrokecolor{currentstroke}%
\pgfsetdash{}{0pt}%
\pgfpathmoveto{\pgfqpoint{1.495917in}{0.642463in}}%
\pgfpathcurveto{\pgfqpoint{1.506813in}{0.642463in}}{\pgfqpoint{1.517263in}{0.646792in}}{\pgfqpoint{1.524968in}{0.654497in}}%
\pgfpathcurveto{\pgfqpoint{1.532672in}{0.662201in}}{\pgfqpoint{1.537001in}{0.672652in}}{\pgfqpoint{1.537001in}{0.683547in}}%
\pgfpathcurveto{\pgfqpoint{1.537001in}{0.694443in}}{\pgfqpoint{1.532672in}{0.704894in}}{\pgfqpoint{1.524968in}{0.712598in}}%
\pgfpathcurveto{\pgfqpoint{1.517263in}{0.720302in}}{\pgfqpoint{1.506813in}{0.724631in}}{\pgfqpoint{1.495917in}{0.724631in}}%
\pgfpathcurveto{\pgfqpoint{1.485021in}{0.724631in}}{\pgfqpoint{1.474571in}{0.720302in}}{\pgfqpoint{1.466866in}{0.712598in}}%
\pgfpathcurveto{\pgfqpoint{1.459162in}{0.704894in}}{\pgfqpoint{1.454833in}{0.694443in}}{\pgfqpoint{1.454833in}{0.683547in}}%
\pgfpathcurveto{\pgfqpoint{1.454833in}{0.672652in}}{\pgfqpoint{1.459162in}{0.662201in}}{\pgfqpoint{1.466866in}{0.654497in}}%
\pgfpathcurveto{\pgfqpoint{1.474571in}{0.646792in}}{\pgfqpoint{1.485021in}{0.642463in}}{\pgfqpoint{1.495917in}{0.642463in}}%
\pgfusepath{stroke}%
\end{pgfscope}%
\begin{pgfscope}%
\pgfpathrectangle{\pgfqpoint{0.688192in}{0.670138in}}{\pgfqpoint{7.111808in}{5.061530in}}%
\pgfusepath{clip}%
\pgfsetbuttcap%
\pgfsetroundjoin%
\pgfsetlinewidth{1.003750pt}%
\definecolor{currentstroke}{rgb}{0.000000,0.000000,0.000000}%
\pgfsetstrokecolor{currentstroke}%
\pgfsetdash{}{0pt}%
\pgfpathmoveto{\pgfqpoint{1.077361in}{0.651185in}}%
\pgfpathcurveto{\pgfqpoint{1.088256in}{0.651185in}}{\pgfqpoint{1.098707in}{0.655514in}}{\pgfqpoint{1.106411in}{0.663218in}}%
\pgfpathcurveto{\pgfqpoint{1.114116in}{0.670923in}}{\pgfqpoint{1.118445in}{0.681373in}}{\pgfqpoint{1.118445in}{0.692269in}}%
\pgfpathcurveto{\pgfqpoint{1.118445in}{0.703165in}}{\pgfqpoint{1.114116in}{0.713615in}}{\pgfqpoint{1.106411in}{0.721320in}}%
\pgfpathcurveto{\pgfqpoint{1.098707in}{0.729024in}}{\pgfqpoint{1.088256in}{0.733353in}}{\pgfqpoint{1.077361in}{0.733353in}}%
\pgfpathcurveto{\pgfqpoint{1.066465in}{0.733353in}}{\pgfqpoint{1.056014in}{0.729024in}}{\pgfqpoint{1.048310in}{0.721320in}}%
\pgfpathcurveto{\pgfqpoint{1.040606in}{0.713615in}}{\pgfqpoint{1.036277in}{0.703165in}}{\pgfqpoint{1.036277in}{0.692269in}}%
\pgfpathcurveto{\pgfqpoint{1.036277in}{0.681373in}}{\pgfqpoint{1.040606in}{0.670923in}}{\pgfqpoint{1.048310in}{0.663218in}}%
\pgfpathcurveto{\pgfqpoint{1.056014in}{0.655514in}}{\pgfqpoint{1.066465in}{0.651185in}}{\pgfqpoint{1.077361in}{0.651185in}}%
\pgfusepath{stroke}%
\end{pgfscope}%
\begin{pgfscope}%
\pgfpathrectangle{\pgfqpoint{0.688192in}{0.670138in}}{\pgfqpoint{7.111808in}{5.061530in}}%
\pgfusepath{clip}%
\pgfsetbuttcap%
\pgfsetroundjoin%
\pgfsetlinewidth{1.003750pt}%
\definecolor{currentstroke}{rgb}{0.000000,0.000000,0.000000}%
\pgfsetstrokecolor{currentstroke}%
\pgfsetdash{}{0pt}%
\pgfpathmoveto{\pgfqpoint{4.959869in}{1.087095in}}%
\pgfpathcurveto{\pgfqpoint{4.970764in}{1.087095in}}{\pgfqpoint{4.981215in}{1.091424in}}{\pgfqpoint{4.988919in}{1.099129in}}%
\pgfpathcurveto{\pgfqpoint{4.996624in}{1.106833in}}{\pgfqpoint{5.000952in}{1.117284in}}{\pgfqpoint{5.000952in}{1.128179in}}%
\pgfpathcurveto{\pgfqpoint{5.000952in}{1.139075in}}{\pgfqpoint{4.996624in}{1.149526in}}{\pgfqpoint{4.988919in}{1.157230in}}%
\pgfpathcurveto{\pgfqpoint{4.981215in}{1.164934in}}{\pgfqpoint{4.970764in}{1.169263in}}{\pgfqpoint{4.959869in}{1.169263in}}%
\pgfpathcurveto{\pgfqpoint{4.948973in}{1.169263in}}{\pgfqpoint{4.938522in}{1.164934in}}{\pgfqpoint{4.930818in}{1.157230in}}%
\pgfpathcurveto{\pgfqpoint{4.923114in}{1.149526in}}{\pgfqpoint{4.918785in}{1.139075in}}{\pgfqpoint{4.918785in}{1.128179in}}%
\pgfpathcurveto{\pgfqpoint{4.918785in}{1.117284in}}{\pgfqpoint{4.923114in}{1.106833in}}{\pgfqpoint{4.930818in}{1.099129in}}%
\pgfpathcurveto{\pgfqpoint{4.938522in}{1.091424in}}{\pgfqpoint{4.948973in}{1.087095in}}{\pgfqpoint{4.959869in}{1.087095in}}%
\pgfpathlineto{\pgfqpoint{4.959869in}{1.087095in}}%
\pgfpathclose%
\pgfusepath{stroke}%
\end{pgfscope}%
\begin{pgfscope}%
\pgfpathrectangle{\pgfqpoint{0.688192in}{0.670138in}}{\pgfqpoint{7.111808in}{5.061530in}}%
\pgfusepath{clip}%
\pgfsetbuttcap%
\pgfsetroundjoin%
\pgfsetlinewidth{1.003750pt}%
\definecolor{currentstroke}{rgb}{0.000000,0.000000,0.000000}%
\pgfsetstrokecolor{currentstroke}%
\pgfsetdash{}{0pt}%
\pgfpathmoveto{\pgfqpoint{3.280946in}{0.634044in}}%
\pgfpathcurveto{\pgfqpoint{3.291842in}{0.634044in}}{\pgfqpoint{3.302293in}{0.638373in}}{\pgfqpoint{3.309997in}{0.646078in}}%
\pgfpathcurveto{\pgfqpoint{3.317701in}{0.653782in}}{\pgfqpoint{3.322030in}{0.664233in}}{\pgfqpoint{3.322030in}{0.675128in}}%
\pgfpathcurveto{\pgfqpoint{3.322030in}{0.686024in}}{\pgfqpoint{3.317701in}{0.696475in}}{\pgfqpoint{3.309997in}{0.704179in}}%
\pgfpathcurveto{\pgfqpoint{3.302293in}{0.711883in}}{\pgfqpoint{3.291842in}{0.716212in}}{\pgfqpoint{3.280946in}{0.716212in}}%
\pgfpathcurveto{\pgfqpoint{3.270051in}{0.716212in}}{\pgfqpoint{3.259600in}{0.711883in}}{\pgfqpoint{3.251896in}{0.704179in}}%
\pgfpathcurveto{\pgfqpoint{3.244191in}{0.696475in}}{\pgfqpoint{3.239862in}{0.686024in}}{\pgfqpoint{3.239862in}{0.675128in}}%
\pgfpathcurveto{\pgfqpoint{3.239862in}{0.664233in}}{\pgfqpoint{3.244191in}{0.653782in}}{\pgfqpoint{3.251896in}{0.646078in}}%
\pgfpathcurveto{\pgfqpoint{3.259600in}{0.638373in}}{\pgfqpoint{3.270051in}{0.634044in}}{\pgfqpoint{3.280946in}{0.634044in}}%
\pgfusepath{stroke}%
\end{pgfscope}%
\begin{pgfscope}%
\pgfpathrectangle{\pgfqpoint{0.688192in}{0.670138in}}{\pgfqpoint{7.111808in}{5.061530in}}%
\pgfusepath{clip}%
\pgfsetbuttcap%
\pgfsetroundjoin%
\pgfsetlinewidth{1.003750pt}%
\definecolor{currentstroke}{rgb}{0.000000,0.000000,0.000000}%
\pgfsetstrokecolor{currentstroke}%
\pgfsetdash{}{0pt}%
\pgfpathmoveto{\pgfqpoint{1.684802in}{0.641700in}}%
\pgfpathcurveto{\pgfqpoint{1.695698in}{0.641700in}}{\pgfqpoint{1.706148in}{0.646028in}}{\pgfqpoint{1.713853in}{0.653733in}}%
\pgfpathcurveto{\pgfqpoint{1.721557in}{0.661437in}}{\pgfqpoint{1.725886in}{0.671888in}}{\pgfqpoint{1.725886in}{0.682783in}}%
\pgfpathcurveto{\pgfqpoint{1.725886in}{0.693679in}}{\pgfqpoint{1.721557in}{0.704130in}}{\pgfqpoint{1.713853in}{0.711834in}}%
\pgfpathcurveto{\pgfqpoint{1.706148in}{0.719538in}}{\pgfqpoint{1.695698in}{0.723867in}}{\pgfqpoint{1.684802in}{0.723867in}}%
\pgfpathcurveto{\pgfqpoint{1.673907in}{0.723867in}}{\pgfqpoint{1.663456in}{0.719538in}}{\pgfqpoint{1.655751in}{0.711834in}}%
\pgfpathcurveto{\pgfqpoint{1.648047in}{0.704130in}}{\pgfqpoint{1.643718in}{0.693679in}}{\pgfqpoint{1.643718in}{0.682783in}}%
\pgfpathcurveto{\pgfqpoint{1.643718in}{0.671888in}}{\pgfqpoint{1.648047in}{0.661437in}}{\pgfqpoint{1.655751in}{0.653733in}}%
\pgfpathcurveto{\pgfqpoint{1.663456in}{0.646028in}}{\pgfqpoint{1.673907in}{0.641700in}}{\pgfqpoint{1.684802in}{0.641700in}}%
\pgfusepath{stroke}%
\end{pgfscope}%
\begin{pgfscope}%
\pgfpathrectangle{\pgfqpoint{0.688192in}{0.670138in}}{\pgfqpoint{7.111808in}{5.061530in}}%
\pgfusepath{clip}%
\pgfsetbuttcap%
\pgfsetroundjoin%
\pgfsetlinewidth{1.003750pt}%
\definecolor{currentstroke}{rgb}{0.000000,0.000000,0.000000}%
\pgfsetstrokecolor{currentstroke}%
\pgfsetdash{}{0pt}%
\pgfpathmoveto{\pgfqpoint{4.648652in}{0.735949in}}%
\pgfpathcurveto{\pgfqpoint{4.659547in}{0.735949in}}{\pgfqpoint{4.669998in}{0.740278in}}{\pgfqpoint{4.677703in}{0.747983in}}%
\pgfpathcurveto{\pgfqpoint{4.685407in}{0.755687in}}{\pgfqpoint{4.689736in}{0.766138in}}{\pgfqpoint{4.689736in}{0.777033in}}%
\pgfpathcurveto{\pgfqpoint{4.689736in}{0.787929in}}{\pgfqpoint{4.685407in}{0.798380in}}{\pgfqpoint{4.677703in}{0.806084in}}%
\pgfpathcurveto{\pgfqpoint{4.669998in}{0.813788in}}{\pgfqpoint{4.659547in}{0.818117in}}{\pgfqpoint{4.648652in}{0.818117in}}%
\pgfpathcurveto{\pgfqpoint{4.637756in}{0.818117in}}{\pgfqpoint{4.627305in}{0.813788in}}{\pgfqpoint{4.619601in}{0.806084in}}%
\pgfpathcurveto{\pgfqpoint{4.611897in}{0.798380in}}{\pgfqpoint{4.607568in}{0.787929in}}{\pgfqpoint{4.607568in}{0.777033in}}%
\pgfpathcurveto{\pgfqpoint{4.607568in}{0.766138in}}{\pgfqpoint{4.611897in}{0.755687in}}{\pgfqpoint{4.619601in}{0.747983in}}%
\pgfpathcurveto{\pgfqpoint{4.627305in}{0.740278in}}{\pgfqpoint{4.637756in}{0.735949in}}{\pgfqpoint{4.648652in}{0.735949in}}%
\pgfpathlineto{\pgfqpoint{4.648652in}{0.735949in}}%
\pgfpathclose%
\pgfusepath{stroke}%
\end{pgfscope}%
\begin{pgfscope}%
\pgfpathrectangle{\pgfqpoint{0.688192in}{0.670138in}}{\pgfqpoint{7.111808in}{5.061530in}}%
\pgfusepath{clip}%
\pgfsetbuttcap%
\pgfsetroundjoin%
\pgfsetlinewidth{1.003750pt}%
\definecolor{currentstroke}{rgb}{0.000000,0.000000,0.000000}%
\pgfsetstrokecolor{currentstroke}%
\pgfsetdash{}{0pt}%
\pgfpathmoveto{\pgfqpoint{2.801640in}{0.701110in}}%
\pgfpathcurveto{\pgfqpoint{2.812536in}{0.701110in}}{\pgfqpoint{2.822986in}{0.705439in}}{\pgfqpoint{2.830691in}{0.713143in}}%
\pgfpathcurveto{\pgfqpoint{2.838395in}{0.720847in}}{\pgfqpoint{2.842724in}{0.731298in}}{\pgfqpoint{2.842724in}{0.742194in}}%
\pgfpathcurveto{\pgfqpoint{2.842724in}{0.753089in}}{\pgfqpoint{2.838395in}{0.763540in}}{\pgfqpoint{2.830691in}{0.771244in}}%
\pgfpathcurveto{\pgfqpoint{2.822986in}{0.778949in}}{\pgfqpoint{2.812536in}{0.783277in}}{\pgfqpoint{2.801640in}{0.783277in}}%
\pgfpathcurveto{\pgfqpoint{2.790745in}{0.783277in}}{\pgfqpoint{2.780294in}{0.778949in}}{\pgfqpoint{2.772589in}{0.771244in}}%
\pgfpathcurveto{\pgfqpoint{2.764885in}{0.763540in}}{\pgfqpoint{2.760556in}{0.753089in}}{\pgfqpoint{2.760556in}{0.742194in}}%
\pgfpathcurveto{\pgfqpoint{2.760556in}{0.731298in}}{\pgfqpoint{2.764885in}{0.720847in}}{\pgfqpoint{2.772589in}{0.713143in}}%
\pgfpathcurveto{\pgfqpoint{2.780294in}{0.705439in}}{\pgfqpoint{2.790745in}{0.701110in}}{\pgfqpoint{2.801640in}{0.701110in}}%
\pgfpathlineto{\pgfqpoint{2.801640in}{0.701110in}}%
\pgfpathclose%
\pgfusepath{stroke}%
\end{pgfscope}%
\begin{pgfscope}%
\pgfpathrectangle{\pgfqpoint{0.688192in}{0.670138in}}{\pgfqpoint{7.111808in}{5.061530in}}%
\pgfusepath{clip}%
\pgfsetbuttcap%
\pgfsetroundjoin%
\pgfsetlinewidth{1.003750pt}%
\definecolor{currentstroke}{rgb}{0.000000,0.000000,0.000000}%
\pgfsetstrokecolor{currentstroke}%
\pgfsetdash{}{0pt}%
\pgfpathmoveto{\pgfqpoint{3.079529in}{1.758062in}}%
\pgfpathcurveto{\pgfqpoint{3.090425in}{1.758062in}}{\pgfqpoint{3.100875in}{1.762391in}}{\pgfqpoint{3.108580in}{1.770095in}}%
\pgfpathcurveto{\pgfqpoint{3.116284in}{1.777800in}}{\pgfqpoint{3.120613in}{1.788250in}}{\pgfqpoint{3.120613in}{1.799146in}}%
\pgfpathcurveto{\pgfqpoint{3.120613in}{1.810042in}}{\pgfqpoint{3.116284in}{1.820492in}}{\pgfqpoint{3.108580in}{1.828197in}}%
\pgfpathcurveto{\pgfqpoint{3.100875in}{1.835901in}}{\pgfqpoint{3.090425in}{1.840230in}}{\pgfqpoint{3.079529in}{1.840230in}}%
\pgfpathcurveto{\pgfqpoint{3.068634in}{1.840230in}}{\pgfqpoint{3.058183in}{1.835901in}}{\pgfqpoint{3.050478in}{1.828197in}}%
\pgfpathcurveto{\pgfqpoint{3.042774in}{1.820492in}}{\pgfqpoint{3.038445in}{1.810042in}}{\pgfqpoint{3.038445in}{1.799146in}}%
\pgfpathcurveto{\pgfqpoint{3.038445in}{1.788250in}}{\pgfqpoint{3.042774in}{1.777800in}}{\pgfqpoint{3.050478in}{1.770095in}}%
\pgfpathcurveto{\pgfqpoint{3.058183in}{1.762391in}}{\pgfqpoint{3.068634in}{1.758062in}}{\pgfqpoint{3.079529in}{1.758062in}}%
\pgfpathlineto{\pgfqpoint{3.079529in}{1.758062in}}%
\pgfpathclose%
\pgfusepath{stroke}%
\end{pgfscope}%
\begin{pgfscope}%
\pgfpathrectangle{\pgfqpoint{0.688192in}{0.670138in}}{\pgfqpoint{7.111808in}{5.061530in}}%
\pgfusepath{clip}%
\pgfsetbuttcap%
\pgfsetroundjoin%
\pgfsetlinewidth{1.003750pt}%
\definecolor{currentstroke}{rgb}{0.000000,0.000000,0.000000}%
\pgfsetstrokecolor{currentstroke}%
\pgfsetdash{}{0pt}%
\pgfpathmoveto{\pgfqpoint{6.736539in}{3.273526in}}%
\pgfpathcurveto{\pgfqpoint{6.747434in}{3.273526in}}{\pgfqpoint{6.757885in}{3.277854in}}{\pgfqpoint{6.765589in}{3.285559in}}%
\pgfpathcurveto{\pgfqpoint{6.773294in}{3.293263in}}{\pgfqpoint{6.777622in}{3.303714in}}{\pgfqpoint{6.777622in}{3.314609in}}%
\pgfpathcurveto{\pgfqpoint{6.777622in}{3.325505in}}{\pgfqpoint{6.773294in}{3.335956in}}{\pgfqpoint{6.765589in}{3.343660in}}%
\pgfpathcurveto{\pgfqpoint{6.757885in}{3.351364in}}{\pgfqpoint{6.747434in}{3.355693in}}{\pgfqpoint{6.736539in}{3.355693in}}%
\pgfpathcurveto{\pgfqpoint{6.725643in}{3.355693in}}{\pgfqpoint{6.715192in}{3.351364in}}{\pgfqpoint{6.707488in}{3.343660in}}%
\pgfpathcurveto{\pgfqpoint{6.699783in}{3.335956in}}{\pgfqpoint{6.695455in}{3.325505in}}{\pgfqpoint{6.695455in}{3.314609in}}%
\pgfpathcurveto{\pgfqpoint{6.695455in}{3.303714in}}{\pgfqpoint{6.699783in}{3.293263in}}{\pgfqpoint{6.707488in}{3.285559in}}%
\pgfpathcurveto{\pgfqpoint{6.715192in}{3.277854in}}{\pgfqpoint{6.725643in}{3.273526in}}{\pgfqpoint{6.736539in}{3.273526in}}%
\pgfpathlineto{\pgfqpoint{6.736539in}{3.273526in}}%
\pgfpathclose%
\pgfusepath{stroke}%
\end{pgfscope}%
\begin{pgfscope}%
\pgfpathrectangle{\pgfqpoint{0.688192in}{0.670138in}}{\pgfqpoint{7.111808in}{5.061530in}}%
\pgfusepath{clip}%
\pgfsetbuttcap%
\pgfsetroundjoin%
\pgfsetlinewidth{1.003750pt}%
\definecolor{currentstroke}{rgb}{0.000000,0.000000,0.000000}%
\pgfsetstrokecolor{currentstroke}%
\pgfsetdash{}{0pt}%
\pgfpathmoveto{\pgfqpoint{1.731710in}{0.641453in}}%
\pgfpathcurveto{\pgfqpoint{1.742605in}{0.641453in}}{\pgfqpoint{1.753056in}{0.645782in}}{\pgfqpoint{1.760760in}{0.653486in}}%
\pgfpathcurveto{\pgfqpoint{1.768465in}{0.661191in}}{\pgfqpoint{1.772794in}{0.671642in}}{\pgfqpoint{1.772794in}{0.682537in}}%
\pgfpathcurveto{\pgfqpoint{1.772794in}{0.693433in}}{\pgfqpoint{1.768465in}{0.703884in}}{\pgfqpoint{1.760760in}{0.711588in}}%
\pgfpathcurveto{\pgfqpoint{1.753056in}{0.719292in}}{\pgfqpoint{1.742605in}{0.723621in}}{\pgfqpoint{1.731710in}{0.723621in}}%
\pgfpathcurveto{\pgfqpoint{1.720814in}{0.723621in}}{\pgfqpoint{1.710363in}{0.719292in}}{\pgfqpoint{1.702659in}{0.711588in}}%
\pgfpathcurveto{\pgfqpoint{1.694955in}{0.703884in}}{\pgfqpoint{1.690626in}{0.693433in}}{\pgfqpoint{1.690626in}{0.682537in}}%
\pgfpathcurveto{\pgfqpoint{1.690626in}{0.671642in}}{\pgfqpoint{1.694955in}{0.661191in}}{\pgfqpoint{1.702659in}{0.653486in}}%
\pgfpathcurveto{\pgfqpoint{1.710363in}{0.645782in}}{\pgfqpoint{1.720814in}{0.641453in}}{\pgfqpoint{1.731710in}{0.641453in}}%
\pgfusepath{stroke}%
\end{pgfscope}%
\begin{pgfscope}%
\pgfpathrectangle{\pgfqpoint{0.688192in}{0.670138in}}{\pgfqpoint{7.111808in}{5.061530in}}%
\pgfusepath{clip}%
\pgfsetbuttcap%
\pgfsetroundjoin%
\pgfsetlinewidth{1.003750pt}%
\definecolor{currentstroke}{rgb}{0.000000,0.000000,0.000000}%
\pgfsetstrokecolor{currentstroke}%
\pgfsetdash{}{0pt}%
\pgfpathmoveto{\pgfqpoint{5.125130in}{3.920722in}}%
\pgfpathcurveto{\pgfqpoint{5.136026in}{3.920722in}}{\pgfqpoint{5.146477in}{3.925051in}}{\pgfqpoint{5.154181in}{3.932755in}}%
\pgfpathcurveto{\pgfqpoint{5.161885in}{3.940460in}}{\pgfqpoint{5.166214in}{3.950911in}}{\pgfqpoint{5.166214in}{3.961806in}}%
\pgfpathcurveto{\pgfqpoint{5.166214in}{3.972702in}}{\pgfqpoint{5.161885in}{3.983152in}}{\pgfqpoint{5.154181in}{3.990857in}}%
\pgfpathcurveto{\pgfqpoint{5.146477in}{3.998561in}}{\pgfqpoint{5.136026in}{4.002890in}}{\pgfqpoint{5.125130in}{4.002890in}}%
\pgfpathcurveto{\pgfqpoint{5.114235in}{4.002890in}}{\pgfqpoint{5.103784in}{3.998561in}}{\pgfqpoint{5.096080in}{3.990857in}}%
\pgfpathcurveto{\pgfqpoint{5.088375in}{3.983152in}}{\pgfqpoint{5.084047in}{3.972702in}}{\pgfqpoint{5.084047in}{3.961806in}}%
\pgfpathcurveto{\pgfqpoint{5.084047in}{3.950911in}}{\pgfqpoint{5.088375in}{3.940460in}}{\pgfqpoint{5.096080in}{3.932755in}}%
\pgfpathcurveto{\pgfqpoint{5.103784in}{3.925051in}}{\pgfqpoint{5.114235in}{3.920722in}}{\pgfqpoint{5.125130in}{3.920722in}}%
\pgfpathlineto{\pgfqpoint{5.125130in}{3.920722in}}%
\pgfpathclose%
\pgfusepath{stroke}%
\end{pgfscope}%
\begin{pgfscope}%
\pgfpathrectangle{\pgfqpoint{0.688192in}{0.670138in}}{\pgfqpoint{7.111808in}{5.061530in}}%
\pgfusepath{clip}%
\pgfsetbuttcap%
\pgfsetroundjoin%
\pgfsetlinewidth{1.003750pt}%
\definecolor{currentstroke}{rgb}{0.000000,0.000000,0.000000}%
\pgfsetstrokecolor{currentstroke}%
\pgfsetdash{}{0pt}%
\pgfpathmoveto{\pgfqpoint{0.939729in}{0.677113in}}%
\pgfpathcurveto{\pgfqpoint{0.950625in}{0.677113in}}{\pgfqpoint{0.961075in}{0.681442in}}{\pgfqpoint{0.968780in}{0.689146in}}%
\pgfpathcurveto{\pgfqpoint{0.976484in}{0.696850in}}{\pgfqpoint{0.980813in}{0.707301in}}{\pgfqpoint{0.980813in}{0.718197in}}%
\pgfpathcurveto{\pgfqpoint{0.980813in}{0.729092in}}{\pgfqpoint{0.976484in}{0.739543in}}{\pgfqpoint{0.968780in}{0.747247in}}%
\pgfpathcurveto{\pgfqpoint{0.961075in}{0.754952in}}{\pgfqpoint{0.950625in}{0.759281in}}{\pgfqpoint{0.939729in}{0.759281in}}%
\pgfpathcurveto{\pgfqpoint{0.928834in}{0.759281in}}{\pgfqpoint{0.918383in}{0.754952in}}{\pgfqpoint{0.910678in}{0.747247in}}%
\pgfpathcurveto{\pgfqpoint{0.902974in}{0.739543in}}{\pgfqpoint{0.898645in}{0.729092in}}{\pgfqpoint{0.898645in}{0.718197in}}%
\pgfpathcurveto{\pgfqpoint{0.898645in}{0.707301in}}{\pgfqpoint{0.902974in}{0.696850in}}{\pgfqpoint{0.910678in}{0.689146in}}%
\pgfpathcurveto{\pgfqpoint{0.918383in}{0.681442in}}{\pgfqpoint{0.928834in}{0.677113in}}{\pgfqpoint{0.939729in}{0.677113in}}%
\pgfpathlineto{\pgfqpoint{0.939729in}{0.677113in}}%
\pgfpathclose%
\pgfusepath{stroke}%
\end{pgfscope}%
\begin{pgfscope}%
\pgfpathrectangle{\pgfqpoint{0.688192in}{0.670138in}}{\pgfqpoint{7.111808in}{5.061530in}}%
\pgfusepath{clip}%
\pgfsetbuttcap%
\pgfsetroundjoin%
\pgfsetlinewidth{1.003750pt}%
\definecolor{currentstroke}{rgb}{0.000000,0.000000,0.000000}%
\pgfsetstrokecolor{currentstroke}%
\pgfsetdash{}{0pt}%
\pgfpathmoveto{\pgfqpoint{5.557238in}{0.629442in}}%
\pgfpathcurveto{\pgfqpoint{5.568133in}{0.629442in}}{\pgfqpoint{5.578584in}{0.633771in}}{\pgfqpoint{5.586288in}{0.641475in}}%
\pgfpathcurveto{\pgfqpoint{5.593993in}{0.649179in}}{\pgfqpoint{5.598322in}{0.659630in}}{\pgfqpoint{5.598322in}{0.670526in}}%
\pgfpathcurveto{\pgfqpoint{5.598322in}{0.681421in}}{\pgfqpoint{5.593993in}{0.691872in}}{\pgfqpoint{5.586288in}{0.699576in}}%
\pgfpathcurveto{\pgfqpoint{5.578584in}{0.707281in}}{\pgfqpoint{5.568133in}{0.711609in}}{\pgfqpoint{5.557238in}{0.711609in}}%
\pgfpathcurveto{\pgfqpoint{5.546342in}{0.711609in}}{\pgfqpoint{5.535891in}{0.707281in}}{\pgfqpoint{5.528187in}{0.699576in}}%
\pgfpathcurveto{\pgfqpoint{5.520483in}{0.691872in}}{\pgfqpoint{5.516154in}{0.681421in}}{\pgfqpoint{5.516154in}{0.670526in}}%
\pgfpathcurveto{\pgfqpoint{5.516154in}{0.659630in}}{\pgfqpoint{5.520483in}{0.649179in}}{\pgfqpoint{5.528187in}{0.641475in}}%
\pgfpathcurveto{\pgfqpoint{5.535891in}{0.633771in}}{\pgfqpoint{5.546342in}{0.629442in}}{\pgfqpoint{5.557238in}{0.629442in}}%
\pgfusepath{stroke}%
\end{pgfscope}%
\begin{pgfscope}%
\pgfpathrectangle{\pgfqpoint{0.688192in}{0.670138in}}{\pgfqpoint{7.111808in}{5.061530in}}%
\pgfusepath{clip}%
\pgfsetbuttcap%
\pgfsetroundjoin%
\pgfsetlinewidth{1.003750pt}%
\definecolor{currentstroke}{rgb}{0.000000,0.000000,0.000000}%
\pgfsetstrokecolor{currentstroke}%
\pgfsetdash{}{0pt}%
\pgfpathmoveto{\pgfqpoint{3.213120in}{3.792894in}}%
\pgfpathcurveto{\pgfqpoint{3.224015in}{3.792894in}}{\pgfqpoint{3.234466in}{3.797222in}}{\pgfqpoint{3.242170in}{3.804927in}}%
\pgfpathcurveto{\pgfqpoint{3.249875in}{3.812631in}}{\pgfqpoint{3.254204in}{3.823082in}}{\pgfqpoint{3.254204in}{3.833977in}}%
\pgfpathcurveto{\pgfqpoint{3.254204in}{3.844873in}}{\pgfqpoint{3.249875in}{3.855324in}}{\pgfqpoint{3.242170in}{3.863028in}}%
\pgfpathcurveto{\pgfqpoint{3.234466in}{3.870733in}}{\pgfqpoint{3.224015in}{3.875061in}}{\pgfqpoint{3.213120in}{3.875061in}}%
\pgfpathcurveto{\pgfqpoint{3.202224in}{3.875061in}}{\pgfqpoint{3.191773in}{3.870733in}}{\pgfqpoint{3.184069in}{3.863028in}}%
\pgfpathcurveto{\pgfqpoint{3.176365in}{3.855324in}}{\pgfqpoint{3.172036in}{3.844873in}}{\pgfqpoint{3.172036in}{3.833977in}}%
\pgfpathcurveto{\pgfqpoint{3.172036in}{3.823082in}}{\pgfqpoint{3.176365in}{3.812631in}}{\pgfqpoint{3.184069in}{3.804927in}}%
\pgfpathcurveto{\pgfqpoint{3.191773in}{3.797222in}}{\pgfqpoint{3.202224in}{3.792894in}}{\pgfqpoint{3.213120in}{3.792894in}}%
\pgfpathlineto{\pgfqpoint{3.213120in}{3.792894in}}%
\pgfpathclose%
\pgfusepath{stroke}%
\end{pgfscope}%
\begin{pgfscope}%
\pgfpathrectangle{\pgfqpoint{0.688192in}{0.670138in}}{\pgfqpoint{7.111808in}{5.061530in}}%
\pgfusepath{clip}%
\pgfsetbuttcap%
\pgfsetroundjoin%
\pgfsetlinewidth{1.003750pt}%
\definecolor{currentstroke}{rgb}{0.000000,0.000000,0.000000}%
\pgfsetstrokecolor{currentstroke}%
\pgfsetdash{}{0pt}%
\pgfpathmoveto{\pgfqpoint{4.491507in}{0.631138in}}%
\pgfpathcurveto{\pgfqpoint{4.502402in}{0.631138in}}{\pgfqpoint{4.512853in}{0.635467in}}{\pgfqpoint{4.520557in}{0.643171in}}%
\pgfpathcurveto{\pgfqpoint{4.528262in}{0.650875in}}{\pgfqpoint{4.532590in}{0.661326in}}{\pgfqpoint{4.532590in}{0.672222in}}%
\pgfpathcurveto{\pgfqpoint{4.532590in}{0.683117in}}{\pgfqpoint{4.528262in}{0.693568in}}{\pgfqpoint{4.520557in}{0.701272in}}%
\pgfpathcurveto{\pgfqpoint{4.512853in}{0.708977in}}{\pgfqpoint{4.502402in}{0.713305in}}{\pgfqpoint{4.491507in}{0.713305in}}%
\pgfpathcurveto{\pgfqpoint{4.480611in}{0.713305in}}{\pgfqpoint{4.470160in}{0.708977in}}{\pgfqpoint{4.462456in}{0.701272in}}%
\pgfpathcurveto{\pgfqpoint{4.454752in}{0.693568in}}{\pgfqpoint{4.450423in}{0.683117in}}{\pgfqpoint{4.450423in}{0.672222in}}%
\pgfpathcurveto{\pgfqpoint{4.450423in}{0.661326in}}{\pgfqpoint{4.454752in}{0.650875in}}{\pgfqpoint{4.462456in}{0.643171in}}%
\pgfpathcurveto{\pgfqpoint{4.470160in}{0.635467in}}{\pgfqpoint{4.480611in}{0.631138in}}{\pgfqpoint{4.491507in}{0.631138in}}%
\pgfusepath{stroke}%
\end{pgfscope}%
\begin{pgfscope}%
\pgfpathrectangle{\pgfqpoint{0.688192in}{0.670138in}}{\pgfqpoint{7.111808in}{5.061530in}}%
\pgfusepath{clip}%
\pgfsetbuttcap%
\pgfsetroundjoin%
\pgfsetlinewidth{1.003750pt}%
\definecolor{currentstroke}{rgb}{0.000000,0.000000,0.000000}%
\pgfsetstrokecolor{currentstroke}%
\pgfsetdash{}{0pt}%
\pgfpathmoveto{\pgfqpoint{5.539199in}{2.933666in}}%
\pgfpathcurveto{\pgfqpoint{5.550094in}{2.933666in}}{\pgfqpoint{5.560545in}{2.937995in}}{\pgfqpoint{5.568249in}{2.945699in}}%
\pgfpathcurveto{\pgfqpoint{5.575954in}{2.953404in}}{\pgfqpoint{5.580282in}{2.963855in}}{\pgfqpoint{5.580282in}{2.974750in}}%
\pgfpathcurveto{\pgfqpoint{5.580282in}{2.985646in}}{\pgfqpoint{5.575954in}{2.996097in}}{\pgfqpoint{5.568249in}{3.003801in}}%
\pgfpathcurveto{\pgfqpoint{5.560545in}{3.011505in}}{\pgfqpoint{5.550094in}{3.015834in}}{\pgfqpoint{5.539199in}{3.015834in}}%
\pgfpathcurveto{\pgfqpoint{5.528303in}{3.015834in}}{\pgfqpoint{5.517852in}{3.011505in}}{\pgfqpoint{5.510148in}{3.003801in}}%
\pgfpathcurveto{\pgfqpoint{5.502444in}{2.996097in}}{\pgfqpoint{5.498115in}{2.985646in}}{\pgfqpoint{5.498115in}{2.974750in}}%
\pgfpathcurveto{\pgfqpoint{5.498115in}{2.963855in}}{\pgfqpoint{5.502444in}{2.953404in}}{\pgfqpoint{5.510148in}{2.945699in}}%
\pgfpathcurveto{\pgfqpoint{5.517852in}{2.937995in}}{\pgfqpoint{5.528303in}{2.933666in}}{\pgfqpoint{5.539199in}{2.933666in}}%
\pgfpathlineto{\pgfqpoint{5.539199in}{2.933666in}}%
\pgfpathclose%
\pgfusepath{stroke}%
\end{pgfscope}%
\begin{pgfscope}%
\pgfpathrectangle{\pgfqpoint{0.688192in}{0.670138in}}{\pgfqpoint{7.111808in}{5.061530in}}%
\pgfusepath{clip}%
\pgfsetbuttcap%
\pgfsetroundjoin%
\pgfsetlinewidth{1.003750pt}%
\definecolor{currentstroke}{rgb}{0.000000,0.000000,0.000000}%
\pgfsetstrokecolor{currentstroke}%
\pgfsetdash{}{0pt}%
\pgfpathmoveto{\pgfqpoint{5.988781in}{0.754647in}}%
\pgfpathcurveto{\pgfqpoint{5.999677in}{0.754647in}}{\pgfqpoint{6.010128in}{0.758976in}}{\pgfqpoint{6.017832in}{0.766680in}}%
\pgfpathcurveto{\pgfqpoint{6.025536in}{0.774384in}}{\pgfqpoint{6.029865in}{0.784835in}}{\pgfqpoint{6.029865in}{0.795731in}}%
\pgfpathcurveto{\pgfqpoint{6.029865in}{0.806626in}}{\pgfqpoint{6.025536in}{0.817077in}}{\pgfqpoint{6.017832in}{0.824781in}}%
\pgfpathcurveto{\pgfqpoint{6.010128in}{0.832486in}}{\pgfqpoint{5.999677in}{0.836815in}}{\pgfqpoint{5.988781in}{0.836815in}}%
\pgfpathcurveto{\pgfqpoint{5.977886in}{0.836815in}}{\pgfqpoint{5.967435in}{0.832486in}}{\pgfqpoint{5.959731in}{0.824781in}}%
\pgfpathcurveto{\pgfqpoint{5.952026in}{0.817077in}}{\pgfqpoint{5.947698in}{0.806626in}}{\pgfqpoint{5.947698in}{0.795731in}}%
\pgfpathcurveto{\pgfqpoint{5.947698in}{0.784835in}}{\pgfqpoint{5.952026in}{0.774384in}}{\pgfqpoint{5.959731in}{0.766680in}}%
\pgfpathcurveto{\pgfqpoint{5.967435in}{0.758976in}}{\pgfqpoint{5.977886in}{0.754647in}}{\pgfqpoint{5.988781in}{0.754647in}}%
\pgfpathlineto{\pgfqpoint{5.988781in}{0.754647in}}%
\pgfpathclose%
\pgfusepath{stroke}%
\end{pgfscope}%
\begin{pgfscope}%
\pgfpathrectangle{\pgfqpoint{0.688192in}{0.670138in}}{\pgfqpoint{7.111808in}{5.061530in}}%
\pgfusepath{clip}%
\pgfsetbuttcap%
\pgfsetroundjoin%
\pgfsetlinewidth{1.003750pt}%
\definecolor{currentstroke}{rgb}{0.000000,0.000000,0.000000}%
\pgfsetstrokecolor{currentstroke}%
\pgfsetdash{}{0pt}%
\pgfpathmoveto{\pgfqpoint{1.363285in}{0.643335in}}%
\pgfpathcurveto{\pgfqpoint{1.374180in}{0.643335in}}{\pgfqpoint{1.384631in}{0.647663in}}{\pgfqpoint{1.392336in}{0.655368in}}%
\pgfpathcurveto{\pgfqpoint{1.400040in}{0.663072in}}{\pgfqpoint{1.404369in}{0.673523in}}{\pgfqpoint{1.404369in}{0.684419in}}%
\pgfpathcurveto{\pgfqpoint{1.404369in}{0.695314in}}{\pgfqpoint{1.400040in}{0.705765in}}{\pgfqpoint{1.392336in}{0.713469in}}%
\pgfpathcurveto{\pgfqpoint{1.384631in}{0.721174in}}{\pgfqpoint{1.374180in}{0.725502in}}{\pgfqpoint{1.363285in}{0.725502in}}%
\pgfpathcurveto{\pgfqpoint{1.352389in}{0.725502in}}{\pgfqpoint{1.341938in}{0.721174in}}{\pgfqpoint{1.334234in}{0.713469in}}%
\pgfpathcurveto{\pgfqpoint{1.326530in}{0.705765in}}{\pgfqpoint{1.322201in}{0.695314in}}{\pgfqpoint{1.322201in}{0.684419in}}%
\pgfpathcurveto{\pgfqpoint{1.322201in}{0.673523in}}{\pgfqpoint{1.326530in}{0.663072in}}{\pgfqpoint{1.334234in}{0.655368in}}%
\pgfpathcurveto{\pgfqpoint{1.341938in}{0.647663in}}{\pgfqpoint{1.352389in}{0.643335in}}{\pgfqpoint{1.363285in}{0.643335in}}%
\pgfusepath{stroke}%
\end{pgfscope}%
\begin{pgfscope}%
\pgfpathrectangle{\pgfqpoint{0.688192in}{0.670138in}}{\pgfqpoint{7.111808in}{5.061530in}}%
\pgfusepath{clip}%
\pgfsetbuttcap%
\pgfsetroundjoin%
\pgfsetlinewidth{1.003750pt}%
\definecolor{currentstroke}{rgb}{0.000000,0.000000,0.000000}%
\pgfsetstrokecolor{currentstroke}%
\pgfsetdash{}{0pt}%
\pgfpathmoveto{\pgfqpoint{5.505780in}{1.507933in}}%
\pgfpathcurveto{\pgfqpoint{5.516676in}{1.507933in}}{\pgfqpoint{5.527127in}{1.512262in}}{\pgfqpoint{5.534831in}{1.519966in}}%
\pgfpathcurveto{\pgfqpoint{5.542535in}{1.527671in}}{\pgfqpoint{5.546864in}{1.538122in}}{\pgfqpoint{5.546864in}{1.549017in}}%
\pgfpathcurveto{\pgfqpoint{5.546864in}{1.559913in}}{\pgfqpoint{5.542535in}{1.570364in}}{\pgfqpoint{5.534831in}{1.578068in}}%
\pgfpathcurveto{\pgfqpoint{5.527127in}{1.585772in}}{\pgfqpoint{5.516676in}{1.590101in}}{\pgfqpoint{5.505780in}{1.590101in}}%
\pgfpathcurveto{\pgfqpoint{5.494885in}{1.590101in}}{\pgfqpoint{5.484434in}{1.585772in}}{\pgfqpoint{5.476730in}{1.578068in}}%
\pgfpathcurveto{\pgfqpoint{5.469025in}{1.570364in}}{\pgfqpoint{5.464697in}{1.559913in}}{\pgfqpoint{5.464697in}{1.549017in}}%
\pgfpathcurveto{\pgfqpoint{5.464697in}{1.538122in}}{\pgfqpoint{5.469025in}{1.527671in}}{\pgfqpoint{5.476730in}{1.519966in}}%
\pgfpathcurveto{\pgfqpoint{5.484434in}{1.512262in}}{\pgfqpoint{5.494885in}{1.507933in}}{\pgfqpoint{5.505780in}{1.507933in}}%
\pgfpathlineto{\pgfqpoint{5.505780in}{1.507933in}}%
\pgfpathclose%
\pgfusepath{stroke}%
\end{pgfscope}%
\begin{pgfscope}%
\pgfpathrectangle{\pgfqpoint{0.688192in}{0.670138in}}{\pgfqpoint{7.111808in}{5.061530in}}%
\pgfusepath{clip}%
\pgfsetbuttcap%
\pgfsetroundjoin%
\pgfsetlinewidth{1.003750pt}%
\definecolor{currentstroke}{rgb}{0.000000,0.000000,0.000000}%
\pgfsetstrokecolor{currentstroke}%
\pgfsetdash{}{0pt}%
\pgfpathmoveto{\pgfqpoint{0.939729in}{0.677113in}}%
\pgfpathcurveto{\pgfqpoint{0.950625in}{0.677113in}}{\pgfqpoint{0.961075in}{0.681442in}}{\pgfqpoint{0.968780in}{0.689146in}}%
\pgfpathcurveto{\pgfqpoint{0.976484in}{0.696850in}}{\pgfqpoint{0.980813in}{0.707301in}}{\pgfqpoint{0.980813in}{0.718197in}}%
\pgfpathcurveto{\pgfqpoint{0.980813in}{0.729092in}}{\pgfqpoint{0.976484in}{0.739543in}}{\pgfqpoint{0.968780in}{0.747247in}}%
\pgfpathcurveto{\pgfqpoint{0.961075in}{0.754952in}}{\pgfqpoint{0.950625in}{0.759281in}}{\pgfqpoint{0.939729in}{0.759281in}}%
\pgfpathcurveto{\pgfqpoint{0.928834in}{0.759281in}}{\pgfqpoint{0.918383in}{0.754952in}}{\pgfqpoint{0.910678in}{0.747247in}}%
\pgfpathcurveto{\pgfqpoint{0.902974in}{0.739543in}}{\pgfqpoint{0.898645in}{0.729092in}}{\pgfqpoint{0.898645in}{0.718197in}}%
\pgfpathcurveto{\pgfqpoint{0.898645in}{0.707301in}}{\pgfqpoint{0.902974in}{0.696850in}}{\pgfqpoint{0.910678in}{0.689146in}}%
\pgfpathcurveto{\pgfqpoint{0.918383in}{0.681442in}}{\pgfqpoint{0.928834in}{0.677113in}}{\pgfqpoint{0.939729in}{0.677113in}}%
\pgfpathlineto{\pgfqpoint{0.939729in}{0.677113in}}%
\pgfpathclose%
\pgfusepath{stroke}%
\end{pgfscope}%
\begin{pgfscope}%
\pgfpathrectangle{\pgfqpoint{0.688192in}{0.670138in}}{\pgfqpoint{7.111808in}{5.061530in}}%
\pgfusepath{clip}%
\pgfsetbuttcap%
\pgfsetroundjoin%
\pgfsetlinewidth{1.003750pt}%
\definecolor{currentstroke}{rgb}{0.000000,0.000000,0.000000}%
\pgfsetstrokecolor{currentstroke}%
\pgfsetdash{}{0pt}%
\pgfpathmoveto{\pgfqpoint{5.673346in}{0.715770in}}%
\pgfpathcurveto{\pgfqpoint{5.684242in}{0.715770in}}{\pgfqpoint{5.694692in}{0.720099in}}{\pgfqpoint{5.702397in}{0.727803in}}%
\pgfpathcurveto{\pgfqpoint{5.710101in}{0.735508in}}{\pgfqpoint{5.714430in}{0.745958in}}{\pgfqpoint{5.714430in}{0.756854in}}%
\pgfpathcurveto{\pgfqpoint{5.714430in}{0.767749in}}{\pgfqpoint{5.710101in}{0.778200in}}{\pgfqpoint{5.702397in}{0.785905in}}%
\pgfpathcurveto{\pgfqpoint{5.694692in}{0.793609in}}{\pgfqpoint{5.684242in}{0.797938in}}{\pgfqpoint{5.673346in}{0.797938in}}%
\pgfpathcurveto{\pgfqpoint{5.662450in}{0.797938in}}{\pgfqpoint{5.652000in}{0.793609in}}{\pgfqpoint{5.644295in}{0.785905in}}%
\pgfpathcurveto{\pgfqpoint{5.636591in}{0.778200in}}{\pgfqpoint{5.632262in}{0.767749in}}{\pgfqpoint{5.632262in}{0.756854in}}%
\pgfpathcurveto{\pgfqpoint{5.632262in}{0.745958in}}{\pgfqpoint{5.636591in}{0.735508in}}{\pgfqpoint{5.644295in}{0.727803in}}%
\pgfpathcurveto{\pgfqpoint{5.652000in}{0.720099in}}{\pgfqpoint{5.662450in}{0.715770in}}{\pgfqpoint{5.673346in}{0.715770in}}%
\pgfpathlineto{\pgfqpoint{5.673346in}{0.715770in}}%
\pgfpathclose%
\pgfusepath{stroke}%
\end{pgfscope}%
\begin{pgfscope}%
\pgfpathrectangle{\pgfqpoint{0.688192in}{0.670138in}}{\pgfqpoint{7.111808in}{5.061530in}}%
\pgfusepath{clip}%
\pgfsetbuttcap%
\pgfsetroundjoin%
\pgfsetlinewidth{1.003750pt}%
\definecolor{currentstroke}{rgb}{0.000000,0.000000,0.000000}%
\pgfsetstrokecolor{currentstroke}%
\pgfsetdash{}{0pt}%
\pgfpathmoveto{\pgfqpoint{1.010100in}{0.659538in}}%
\pgfpathcurveto{\pgfqpoint{1.020996in}{0.659538in}}{\pgfqpoint{1.031447in}{0.663867in}}{\pgfqpoint{1.039151in}{0.671571in}}%
\pgfpathcurveto{\pgfqpoint{1.046856in}{0.679276in}}{\pgfqpoint{1.051184in}{0.689727in}}{\pgfqpoint{1.051184in}{0.700622in}}%
\pgfpathcurveto{\pgfqpoint{1.051184in}{0.711518in}}{\pgfqpoint{1.046856in}{0.721969in}}{\pgfqpoint{1.039151in}{0.729673in}}%
\pgfpathcurveto{\pgfqpoint{1.031447in}{0.737377in}}{\pgfqpoint{1.020996in}{0.741706in}}{\pgfqpoint{1.010100in}{0.741706in}}%
\pgfpathcurveto{\pgfqpoint{0.999205in}{0.741706in}}{\pgfqpoint{0.988754in}{0.737377in}}{\pgfqpoint{0.981050in}{0.729673in}}%
\pgfpathcurveto{\pgfqpoint{0.973345in}{0.721969in}}{\pgfqpoint{0.969017in}{0.711518in}}{\pgfqpoint{0.969017in}{0.700622in}}%
\pgfpathcurveto{\pgfqpoint{0.969017in}{0.689727in}}{\pgfqpoint{0.973345in}{0.679276in}}{\pgfqpoint{0.981050in}{0.671571in}}%
\pgfpathcurveto{\pgfqpoint{0.988754in}{0.663867in}}{\pgfqpoint{0.999205in}{0.659538in}}{\pgfqpoint{1.010100in}{0.659538in}}%
\pgfusepath{stroke}%
\end{pgfscope}%
\begin{pgfscope}%
\pgfpathrectangle{\pgfqpoint{0.688192in}{0.670138in}}{\pgfqpoint{7.111808in}{5.061530in}}%
\pgfusepath{clip}%
\pgfsetbuttcap%
\pgfsetroundjoin%
\pgfsetlinewidth{1.003750pt}%
\definecolor{currentstroke}{rgb}{0.000000,0.000000,0.000000}%
\pgfsetstrokecolor{currentstroke}%
\pgfsetdash{}{0pt}%
\pgfpathmoveto{\pgfqpoint{2.674298in}{3.704488in}}%
\pgfpathcurveto{\pgfqpoint{2.685194in}{3.704488in}}{\pgfqpoint{2.695645in}{3.708817in}}{\pgfqpoint{2.703349in}{3.716521in}}%
\pgfpathcurveto{\pgfqpoint{2.711053in}{3.724225in}}{\pgfqpoint{2.715382in}{3.734676in}}{\pgfqpoint{2.715382in}{3.745572in}}%
\pgfpathcurveto{\pgfqpoint{2.715382in}{3.756467in}}{\pgfqpoint{2.711053in}{3.766918in}}{\pgfqpoint{2.703349in}{3.774622in}}%
\pgfpathcurveto{\pgfqpoint{2.695645in}{3.782327in}}{\pgfqpoint{2.685194in}{3.786656in}}{\pgfqpoint{2.674298in}{3.786656in}}%
\pgfpathcurveto{\pgfqpoint{2.663403in}{3.786656in}}{\pgfqpoint{2.652952in}{3.782327in}}{\pgfqpoint{2.645248in}{3.774622in}}%
\pgfpathcurveto{\pgfqpoint{2.637543in}{3.766918in}}{\pgfqpoint{2.633214in}{3.756467in}}{\pgfqpoint{2.633214in}{3.745572in}}%
\pgfpathcurveto{\pgfqpoint{2.633214in}{3.734676in}}{\pgfqpoint{2.637543in}{3.724225in}}{\pgfqpoint{2.645248in}{3.716521in}}%
\pgfpathcurveto{\pgfqpoint{2.652952in}{3.708817in}}{\pgfqpoint{2.663403in}{3.704488in}}{\pgfqpoint{2.674298in}{3.704488in}}%
\pgfpathlineto{\pgfqpoint{2.674298in}{3.704488in}}%
\pgfpathclose%
\pgfusepath{stroke}%
\end{pgfscope}%
\begin{pgfscope}%
\pgfpathrectangle{\pgfqpoint{0.688192in}{0.670138in}}{\pgfqpoint{7.111808in}{5.061530in}}%
\pgfusepath{clip}%
\pgfsetbuttcap%
\pgfsetroundjoin%
\pgfsetlinewidth{1.003750pt}%
\definecolor{currentstroke}{rgb}{0.000000,0.000000,0.000000}%
\pgfsetstrokecolor{currentstroke}%
\pgfsetdash{}{0pt}%
\pgfpathmoveto{\pgfqpoint{1.178009in}{0.644363in}}%
\pgfpathcurveto{\pgfqpoint{1.188905in}{0.644363in}}{\pgfqpoint{1.199355in}{0.648692in}}{\pgfqpoint{1.207060in}{0.656396in}}%
\pgfpathcurveto{\pgfqpoint{1.214764in}{0.664100in}}{\pgfqpoint{1.219093in}{0.674551in}}{\pgfqpoint{1.219093in}{0.685447in}}%
\pgfpathcurveto{\pgfqpoint{1.219093in}{0.696342in}}{\pgfqpoint{1.214764in}{0.706793in}}{\pgfqpoint{1.207060in}{0.714497in}}%
\pgfpathcurveto{\pgfqpoint{1.199355in}{0.722202in}}{\pgfqpoint{1.188905in}{0.726531in}}{\pgfqpoint{1.178009in}{0.726531in}}%
\pgfpathcurveto{\pgfqpoint{1.167113in}{0.726531in}}{\pgfqpoint{1.156663in}{0.722202in}}{\pgfqpoint{1.148958in}{0.714497in}}%
\pgfpathcurveto{\pgfqpoint{1.141254in}{0.706793in}}{\pgfqpoint{1.136925in}{0.696342in}}{\pgfqpoint{1.136925in}{0.685447in}}%
\pgfpathcurveto{\pgfqpoint{1.136925in}{0.674551in}}{\pgfqpoint{1.141254in}{0.664100in}}{\pgfqpoint{1.148958in}{0.656396in}}%
\pgfpathcurveto{\pgfqpoint{1.156663in}{0.648692in}}{\pgfqpoint{1.167113in}{0.644363in}}{\pgfqpoint{1.178009in}{0.644363in}}%
\pgfusepath{stroke}%
\end{pgfscope}%
\begin{pgfscope}%
\pgfpathrectangle{\pgfqpoint{0.688192in}{0.670138in}}{\pgfqpoint{7.111808in}{5.061530in}}%
\pgfusepath{clip}%
\pgfsetbuttcap%
\pgfsetroundjoin%
\pgfsetlinewidth{1.003750pt}%
\definecolor{currentstroke}{rgb}{0.000000,0.000000,0.000000}%
\pgfsetstrokecolor{currentstroke}%
\pgfsetdash{}{0pt}%
\pgfpathmoveto{\pgfqpoint{1.407033in}{0.642986in}}%
\pgfpathcurveto{\pgfqpoint{1.417928in}{0.642986in}}{\pgfqpoint{1.428379in}{0.647315in}}{\pgfqpoint{1.436083in}{0.655020in}}%
\pgfpathcurveto{\pgfqpoint{1.443788in}{0.662724in}}{\pgfqpoint{1.448117in}{0.673175in}}{\pgfqpoint{1.448117in}{0.684070in}}%
\pgfpathcurveto{\pgfqpoint{1.448117in}{0.694966in}}{\pgfqpoint{1.443788in}{0.705417in}}{\pgfqpoint{1.436083in}{0.713121in}}%
\pgfpathcurveto{\pgfqpoint{1.428379in}{0.720825in}}{\pgfqpoint{1.417928in}{0.725154in}}{\pgfqpoint{1.407033in}{0.725154in}}%
\pgfpathcurveto{\pgfqpoint{1.396137in}{0.725154in}}{\pgfqpoint{1.385686in}{0.720825in}}{\pgfqpoint{1.377982in}{0.713121in}}%
\pgfpathcurveto{\pgfqpoint{1.370278in}{0.705417in}}{\pgfqpoint{1.365949in}{0.694966in}}{\pgfqpoint{1.365949in}{0.684070in}}%
\pgfpathcurveto{\pgfqpoint{1.365949in}{0.673175in}}{\pgfqpoint{1.370278in}{0.662724in}}{\pgfqpoint{1.377982in}{0.655020in}}%
\pgfpathcurveto{\pgfqpoint{1.385686in}{0.647315in}}{\pgfqpoint{1.396137in}{0.642986in}}{\pgfqpoint{1.407033in}{0.642986in}}%
\pgfusepath{stroke}%
\end{pgfscope}%
\begin{pgfscope}%
\pgfpathrectangle{\pgfqpoint{0.688192in}{0.670138in}}{\pgfqpoint{7.111808in}{5.061530in}}%
\pgfusepath{clip}%
\pgfsetbuttcap%
\pgfsetroundjoin%
\pgfsetlinewidth{1.003750pt}%
\definecolor{currentstroke}{rgb}{0.000000,0.000000,0.000000}%
\pgfsetstrokecolor{currentstroke}%
\pgfsetdash{}{0pt}%
\pgfpathmoveto{\pgfqpoint{1.108617in}{0.647679in}}%
\pgfpathcurveto{\pgfqpoint{1.119513in}{0.647679in}}{\pgfqpoint{1.129964in}{0.652008in}}{\pgfqpoint{1.137668in}{0.659712in}}%
\pgfpathcurveto{\pgfqpoint{1.145372in}{0.667417in}}{\pgfqpoint{1.149701in}{0.677868in}}{\pgfqpoint{1.149701in}{0.688763in}}%
\pgfpathcurveto{\pgfqpoint{1.149701in}{0.699659in}}{\pgfqpoint{1.145372in}{0.710109in}}{\pgfqpoint{1.137668in}{0.717814in}}%
\pgfpathcurveto{\pgfqpoint{1.129964in}{0.725518in}}{\pgfqpoint{1.119513in}{0.729847in}}{\pgfqpoint{1.108617in}{0.729847in}}%
\pgfpathcurveto{\pgfqpoint{1.097722in}{0.729847in}}{\pgfqpoint{1.087271in}{0.725518in}}{\pgfqpoint{1.079567in}{0.717814in}}%
\pgfpathcurveto{\pgfqpoint{1.071862in}{0.710109in}}{\pgfqpoint{1.067534in}{0.699659in}}{\pgfqpoint{1.067534in}{0.688763in}}%
\pgfpathcurveto{\pgfqpoint{1.067534in}{0.677868in}}{\pgfqpoint{1.071862in}{0.667417in}}{\pgfqpoint{1.079567in}{0.659712in}}%
\pgfpathcurveto{\pgfqpoint{1.087271in}{0.652008in}}{\pgfqpoint{1.097722in}{0.647679in}}{\pgfqpoint{1.108617in}{0.647679in}}%
\pgfusepath{stroke}%
\end{pgfscope}%
\begin{pgfscope}%
\pgfpathrectangle{\pgfqpoint{0.688192in}{0.670138in}}{\pgfqpoint{7.111808in}{5.061530in}}%
\pgfusepath{clip}%
\pgfsetbuttcap%
\pgfsetroundjoin%
\pgfsetlinewidth{1.003750pt}%
\definecolor{currentstroke}{rgb}{0.000000,0.000000,0.000000}%
\pgfsetstrokecolor{currentstroke}%
\pgfsetdash{}{0pt}%
\pgfpathmoveto{\pgfqpoint{1.198965in}{0.644318in}}%
\pgfpathcurveto{\pgfqpoint{1.209860in}{0.644318in}}{\pgfqpoint{1.220311in}{0.648647in}}{\pgfqpoint{1.228016in}{0.656352in}}%
\pgfpathcurveto{\pgfqpoint{1.235720in}{0.664056in}}{\pgfqpoint{1.240049in}{0.674507in}}{\pgfqpoint{1.240049in}{0.685402in}}%
\pgfpathcurveto{\pgfqpoint{1.240049in}{0.696298in}}{\pgfqpoint{1.235720in}{0.706749in}}{\pgfqpoint{1.228016in}{0.714453in}}%
\pgfpathcurveto{\pgfqpoint{1.220311in}{0.722157in}}{\pgfqpoint{1.209860in}{0.726486in}}{\pgfqpoint{1.198965in}{0.726486in}}%
\pgfpathcurveto{\pgfqpoint{1.188069in}{0.726486in}}{\pgfqpoint{1.177619in}{0.722157in}}{\pgfqpoint{1.169914in}{0.714453in}}%
\pgfpathcurveto{\pgfqpoint{1.162210in}{0.706749in}}{\pgfqpoint{1.157881in}{0.696298in}}{\pgfqpoint{1.157881in}{0.685402in}}%
\pgfpathcurveto{\pgfqpoint{1.157881in}{0.674507in}}{\pgfqpoint{1.162210in}{0.664056in}}{\pgfqpoint{1.169914in}{0.656352in}}%
\pgfpathcurveto{\pgfqpoint{1.177619in}{0.648647in}}{\pgfqpoint{1.188069in}{0.644318in}}{\pgfqpoint{1.198965in}{0.644318in}}%
\pgfusepath{stroke}%
\end{pgfscope}%
\begin{pgfscope}%
\pgfpathrectangle{\pgfqpoint{0.688192in}{0.670138in}}{\pgfqpoint{7.111808in}{5.061530in}}%
\pgfusepath{clip}%
\pgfsetbuttcap%
\pgfsetroundjoin%
\pgfsetlinewidth{1.003750pt}%
\definecolor{currentstroke}{rgb}{0.000000,0.000000,0.000000}%
\pgfsetstrokecolor{currentstroke}%
\pgfsetdash{}{0pt}%
\pgfpathmoveto{\pgfqpoint{0.939729in}{0.677113in}}%
\pgfpathcurveto{\pgfqpoint{0.950625in}{0.677113in}}{\pgfqpoint{0.961075in}{0.681442in}}{\pgfqpoint{0.968780in}{0.689146in}}%
\pgfpathcurveto{\pgfqpoint{0.976484in}{0.696850in}}{\pgfqpoint{0.980813in}{0.707301in}}{\pgfqpoint{0.980813in}{0.718197in}}%
\pgfpathcurveto{\pgfqpoint{0.980813in}{0.729092in}}{\pgfqpoint{0.976484in}{0.739543in}}{\pgfqpoint{0.968780in}{0.747247in}}%
\pgfpathcurveto{\pgfqpoint{0.961075in}{0.754952in}}{\pgfqpoint{0.950625in}{0.759281in}}{\pgfqpoint{0.939729in}{0.759281in}}%
\pgfpathcurveto{\pgfqpoint{0.928834in}{0.759281in}}{\pgfqpoint{0.918383in}{0.754952in}}{\pgfqpoint{0.910678in}{0.747247in}}%
\pgfpathcurveto{\pgfqpoint{0.902974in}{0.739543in}}{\pgfqpoint{0.898645in}{0.729092in}}{\pgfqpoint{0.898645in}{0.718197in}}%
\pgfpathcurveto{\pgfqpoint{0.898645in}{0.707301in}}{\pgfqpoint{0.902974in}{0.696850in}}{\pgfqpoint{0.910678in}{0.689146in}}%
\pgfpathcurveto{\pgfqpoint{0.918383in}{0.681442in}}{\pgfqpoint{0.928834in}{0.677113in}}{\pgfqpoint{0.939729in}{0.677113in}}%
\pgfpathlineto{\pgfqpoint{0.939729in}{0.677113in}}%
\pgfpathclose%
\pgfusepath{stroke}%
\end{pgfscope}%
\begin{pgfscope}%
\pgfpathrectangle{\pgfqpoint{0.688192in}{0.670138in}}{\pgfqpoint{7.111808in}{5.061530in}}%
\pgfusepath{clip}%
\pgfsetbuttcap%
\pgfsetroundjoin%
\pgfsetlinewidth{1.003750pt}%
\definecolor{currentstroke}{rgb}{0.000000,0.000000,0.000000}%
\pgfsetstrokecolor{currentstroke}%
\pgfsetdash{}{0pt}%
\pgfpathmoveto{\pgfqpoint{1.363285in}{0.643335in}}%
\pgfpathcurveto{\pgfqpoint{1.374180in}{0.643335in}}{\pgfqpoint{1.384631in}{0.647663in}}{\pgfqpoint{1.392336in}{0.655368in}}%
\pgfpathcurveto{\pgfqpoint{1.400040in}{0.663072in}}{\pgfqpoint{1.404369in}{0.673523in}}{\pgfqpoint{1.404369in}{0.684419in}}%
\pgfpathcurveto{\pgfqpoint{1.404369in}{0.695314in}}{\pgfqpoint{1.400040in}{0.705765in}}{\pgfqpoint{1.392336in}{0.713469in}}%
\pgfpathcurveto{\pgfqpoint{1.384631in}{0.721174in}}{\pgfqpoint{1.374180in}{0.725502in}}{\pgfqpoint{1.363285in}{0.725502in}}%
\pgfpathcurveto{\pgfqpoint{1.352389in}{0.725502in}}{\pgfqpoint{1.341938in}{0.721174in}}{\pgfqpoint{1.334234in}{0.713469in}}%
\pgfpathcurveto{\pgfqpoint{1.326530in}{0.705765in}}{\pgfqpoint{1.322201in}{0.695314in}}{\pgfqpoint{1.322201in}{0.684419in}}%
\pgfpathcurveto{\pgfqpoint{1.322201in}{0.673523in}}{\pgfqpoint{1.326530in}{0.663072in}}{\pgfqpoint{1.334234in}{0.655368in}}%
\pgfpathcurveto{\pgfqpoint{1.341938in}{0.647663in}}{\pgfqpoint{1.352389in}{0.643335in}}{\pgfqpoint{1.363285in}{0.643335in}}%
\pgfusepath{stroke}%
\end{pgfscope}%
\begin{pgfscope}%
\pgfpathrectangle{\pgfqpoint{0.688192in}{0.670138in}}{\pgfqpoint{7.111808in}{5.061530in}}%
\pgfusepath{clip}%
\pgfsetbuttcap%
\pgfsetroundjoin%
\pgfsetlinewidth{1.003750pt}%
\definecolor{currentstroke}{rgb}{0.000000,0.000000,0.000000}%
\pgfsetstrokecolor{currentstroke}%
\pgfsetdash{}{0pt}%
\pgfpathmoveto{\pgfqpoint{2.348683in}{2.153861in}}%
\pgfpathcurveto{\pgfqpoint{2.359578in}{2.153861in}}{\pgfqpoint{2.370029in}{2.158190in}}{\pgfqpoint{2.377733in}{2.165894in}}%
\pgfpathcurveto{\pgfqpoint{2.385438in}{2.173599in}}{\pgfqpoint{2.389767in}{2.184050in}}{\pgfqpoint{2.389767in}{2.194945in}}%
\pgfpathcurveto{\pgfqpoint{2.389767in}{2.205841in}}{\pgfqpoint{2.385438in}{2.216291in}}{\pgfqpoint{2.377733in}{2.223996in}}%
\pgfpathcurveto{\pgfqpoint{2.370029in}{2.231700in}}{\pgfqpoint{2.359578in}{2.236029in}}{\pgfqpoint{2.348683in}{2.236029in}}%
\pgfpathcurveto{\pgfqpoint{2.337787in}{2.236029in}}{\pgfqpoint{2.327336in}{2.231700in}}{\pgfqpoint{2.319632in}{2.223996in}}%
\pgfpathcurveto{\pgfqpoint{2.311928in}{2.216291in}}{\pgfqpoint{2.307599in}{2.205841in}}{\pgfqpoint{2.307599in}{2.194945in}}%
\pgfpathcurveto{\pgfqpoint{2.307599in}{2.184050in}}{\pgfqpoint{2.311928in}{2.173599in}}{\pgfqpoint{2.319632in}{2.165894in}}%
\pgfpathcurveto{\pgfqpoint{2.327336in}{2.158190in}}{\pgfqpoint{2.337787in}{2.153861in}}{\pgfqpoint{2.348683in}{2.153861in}}%
\pgfpathlineto{\pgfqpoint{2.348683in}{2.153861in}}%
\pgfpathclose%
\pgfusepath{stroke}%
\end{pgfscope}%
\begin{pgfscope}%
\pgfpathrectangle{\pgfqpoint{0.688192in}{0.670138in}}{\pgfqpoint{7.111808in}{5.061530in}}%
\pgfusepath{clip}%
\pgfsetbuttcap%
\pgfsetroundjoin%
\pgfsetlinewidth{1.003750pt}%
\definecolor{currentstroke}{rgb}{0.000000,0.000000,0.000000}%
\pgfsetstrokecolor{currentstroke}%
\pgfsetdash{}{0pt}%
\pgfpathmoveto{\pgfqpoint{2.324909in}{2.444468in}}%
\pgfpathcurveto{\pgfqpoint{2.335805in}{2.444468in}}{\pgfqpoint{2.346255in}{2.448797in}}{\pgfqpoint{2.353960in}{2.456501in}}%
\pgfpathcurveto{\pgfqpoint{2.361664in}{2.464205in}}{\pgfqpoint{2.365993in}{2.474656in}}{\pgfqpoint{2.365993in}{2.485552in}}%
\pgfpathcurveto{\pgfqpoint{2.365993in}{2.496447in}}{\pgfqpoint{2.361664in}{2.506898in}}{\pgfqpoint{2.353960in}{2.514602in}}%
\pgfpathcurveto{\pgfqpoint{2.346255in}{2.522307in}}{\pgfqpoint{2.335805in}{2.526636in}}{\pgfqpoint{2.324909in}{2.526636in}}%
\pgfpathcurveto{\pgfqpoint{2.314013in}{2.526636in}}{\pgfqpoint{2.303563in}{2.522307in}}{\pgfqpoint{2.295858in}{2.514602in}}%
\pgfpathcurveto{\pgfqpoint{2.288154in}{2.506898in}}{\pgfqpoint{2.283825in}{2.496447in}}{\pgfqpoint{2.283825in}{2.485552in}}%
\pgfpathcurveto{\pgfqpoint{2.283825in}{2.474656in}}{\pgfqpoint{2.288154in}{2.464205in}}{\pgfqpoint{2.295858in}{2.456501in}}%
\pgfpathcurveto{\pgfqpoint{2.303563in}{2.448797in}}{\pgfqpoint{2.314013in}{2.444468in}}{\pgfqpoint{2.324909in}{2.444468in}}%
\pgfpathlineto{\pgfqpoint{2.324909in}{2.444468in}}%
\pgfpathclose%
\pgfusepath{stroke}%
\end{pgfscope}%
\begin{pgfscope}%
\pgfpathrectangle{\pgfqpoint{0.688192in}{0.670138in}}{\pgfqpoint{7.111808in}{5.061530in}}%
\pgfusepath{clip}%
\pgfsetbuttcap%
\pgfsetroundjoin%
\pgfsetlinewidth{1.003750pt}%
\definecolor{currentstroke}{rgb}{0.000000,0.000000,0.000000}%
\pgfsetstrokecolor{currentstroke}%
\pgfsetdash{}{0pt}%
\pgfpathmoveto{\pgfqpoint{2.911373in}{0.635871in}}%
\pgfpathcurveto{\pgfqpoint{2.922269in}{0.635871in}}{\pgfqpoint{2.932719in}{0.640200in}}{\pgfqpoint{2.940424in}{0.647905in}}%
\pgfpathcurveto{\pgfqpoint{2.948128in}{0.655609in}}{\pgfqpoint{2.952457in}{0.666060in}}{\pgfqpoint{2.952457in}{0.676955in}}%
\pgfpathcurveto{\pgfqpoint{2.952457in}{0.687851in}}{\pgfqpoint{2.948128in}{0.698302in}}{\pgfqpoint{2.940424in}{0.706006in}}%
\pgfpathcurveto{\pgfqpoint{2.932719in}{0.713710in}}{\pgfqpoint{2.922269in}{0.718039in}}{\pgfqpoint{2.911373in}{0.718039in}}%
\pgfpathcurveto{\pgfqpoint{2.900477in}{0.718039in}}{\pgfqpoint{2.890027in}{0.713710in}}{\pgfqpoint{2.882322in}{0.706006in}}%
\pgfpathcurveto{\pgfqpoint{2.874618in}{0.698302in}}{\pgfqpoint{2.870289in}{0.687851in}}{\pgfqpoint{2.870289in}{0.676955in}}%
\pgfpathcurveto{\pgfqpoint{2.870289in}{0.666060in}}{\pgfqpoint{2.874618in}{0.655609in}}{\pgfqpoint{2.882322in}{0.647905in}}%
\pgfpathcurveto{\pgfqpoint{2.890027in}{0.640200in}}{\pgfqpoint{2.900477in}{0.635871in}}{\pgfqpoint{2.911373in}{0.635871in}}%
\pgfusepath{stroke}%
\end{pgfscope}%
\begin{pgfscope}%
\pgfpathrectangle{\pgfqpoint{0.688192in}{0.670138in}}{\pgfqpoint{7.111808in}{5.061530in}}%
\pgfusepath{clip}%
\pgfsetbuttcap%
\pgfsetroundjoin%
\pgfsetlinewidth{1.003750pt}%
\definecolor{currentstroke}{rgb}{0.000000,0.000000,0.000000}%
\pgfsetstrokecolor{currentstroke}%
\pgfsetdash{}{0pt}%
\pgfpathmoveto{\pgfqpoint{3.960835in}{1.502809in}}%
\pgfpathcurveto{\pgfqpoint{3.971731in}{1.502809in}}{\pgfqpoint{3.982182in}{1.507138in}}{\pgfqpoint{3.989886in}{1.514842in}}%
\pgfpathcurveto{\pgfqpoint{3.997590in}{1.522546in}}{\pgfqpoint{4.001919in}{1.532997in}}{\pgfqpoint{4.001919in}{1.543893in}}%
\pgfpathcurveto{\pgfqpoint{4.001919in}{1.554788in}}{\pgfqpoint{3.997590in}{1.565239in}}{\pgfqpoint{3.989886in}{1.572944in}}%
\pgfpathcurveto{\pgfqpoint{3.982182in}{1.580648in}}{\pgfqpoint{3.971731in}{1.584977in}}{\pgfqpoint{3.960835in}{1.584977in}}%
\pgfpathcurveto{\pgfqpoint{3.949940in}{1.584977in}}{\pgfqpoint{3.939489in}{1.580648in}}{\pgfqpoint{3.931785in}{1.572944in}}%
\pgfpathcurveto{\pgfqpoint{3.924080in}{1.565239in}}{\pgfqpoint{3.919751in}{1.554788in}}{\pgfqpoint{3.919751in}{1.543893in}}%
\pgfpathcurveto{\pgfqpoint{3.919751in}{1.532997in}}{\pgfqpoint{3.924080in}{1.522546in}}{\pgfqpoint{3.931785in}{1.514842in}}%
\pgfpathcurveto{\pgfqpoint{3.939489in}{1.507138in}}{\pgfqpoint{3.949940in}{1.502809in}}{\pgfqpoint{3.960835in}{1.502809in}}%
\pgfpathlineto{\pgfqpoint{3.960835in}{1.502809in}}%
\pgfpathclose%
\pgfusepath{stroke}%
\end{pgfscope}%
\begin{pgfscope}%
\pgfpathrectangle{\pgfqpoint{0.688192in}{0.670138in}}{\pgfqpoint{7.111808in}{5.061530in}}%
\pgfusepath{clip}%
\pgfsetbuttcap%
\pgfsetroundjoin%
\pgfsetlinewidth{1.003750pt}%
\definecolor{currentstroke}{rgb}{0.000000,0.000000,0.000000}%
\pgfsetstrokecolor{currentstroke}%
\pgfsetdash{}{0pt}%
\pgfpathmoveto{\pgfqpoint{1.970905in}{0.639806in}}%
\pgfpathcurveto{\pgfqpoint{1.981801in}{0.639806in}}{\pgfqpoint{1.992251in}{0.644135in}}{\pgfqpoint{1.999956in}{0.651839in}}%
\pgfpathcurveto{\pgfqpoint{2.007660in}{0.659543in}}{\pgfqpoint{2.011989in}{0.669994in}}{\pgfqpoint{2.011989in}{0.680890in}}%
\pgfpathcurveto{\pgfqpoint{2.011989in}{0.691785in}}{\pgfqpoint{2.007660in}{0.702236in}}{\pgfqpoint{1.999956in}{0.709941in}}%
\pgfpathcurveto{\pgfqpoint{1.992251in}{0.717645in}}{\pgfqpoint{1.981801in}{0.721974in}}{\pgfqpoint{1.970905in}{0.721974in}}%
\pgfpathcurveto{\pgfqpoint{1.960009in}{0.721974in}}{\pgfqpoint{1.949559in}{0.717645in}}{\pgfqpoint{1.941854in}{0.709941in}}%
\pgfpathcurveto{\pgfqpoint{1.934150in}{0.702236in}}{\pgfqpoint{1.929821in}{0.691785in}}{\pgfqpoint{1.929821in}{0.680890in}}%
\pgfpathcurveto{\pgfqpoint{1.929821in}{0.669994in}}{\pgfqpoint{1.934150in}{0.659543in}}{\pgfqpoint{1.941854in}{0.651839in}}%
\pgfpathcurveto{\pgfqpoint{1.949559in}{0.644135in}}{\pgfqpoint{1.960009in}{0.639806in}}{\pgfqpoint{1.970905in}{0.639806in}}%
\pgfusepath{stroke}%
\end{pgfscope}%
\begin{pgfscope}%
\pgfpathrectangle{\pgfqpoint{0.688192in}{0.670138in}}{\pgfqpoint{7.111808in}{5.061530in}}%
\pgfusepath{clip}%
\pgfsetbuttcap%
\pgfsetroundjoin%
\pgfsetlinewidth{1.003750pt}%
\definecolor{currentstroke}{rgb}{0.000000,0.000000,0.000000}%
\pgfsetstrokecolor{currentstroke}%
\pgfsetdash{}{0pt}%
\pgfpathmoveto{\pgfqpoint{1.663205in}{0.641981in}}%
\pgfpathcurveto{\pgfqpoint{1.674101in}{0.641981in}}{\pgfqpoint{1.684552in}{0.646310in}}{\pgfqpoint{1.692256in}{0.654014in}}%
\pgfpathcurveto{\pgfqpoint{1.699960in}{0.661719in}}{\pgfqpoint{1.704289in}{0.672169in}}{\pgfqpoint{1.704289in}{0.683065in}}%
\pgfpathcurveto{\pgfqpoint{1.704289in}{0.693961in}}{\pgfqpoint{1.699960in}{0.704411in}}{\pgfqpoint{1.692256in}{0.712116in}}%
\pgfpathcurveto{\pgfqpoint{1.684552in}{0.719820in}}{\pgfqpoint{1.674101in}{0.724149in}}{\pgfqpoint{1.663205in}{0.724149in}}%
\pgfpathcurveto{\pgfqpoint{1.652310in}{0.724149in}}{\pgfqpoint{1.641859in}{0.719820in}}{\pgfqpoint{1.634155in}{0.712116in}}%
\pgfpathcurveto{\pgfqpoint{1.626450in}{0.704411in}}{\pgfqpoint{1.622121in}{0.693961in}}{\pgfqpoint{1.622121in}{0.683065in}}%
\pgfpathcurveto{\pgfqpoint{1.622121in}{0.672169in}}{\pgfqpoint{1.626450in}{0.661719in}}{\pgfqpoint{1.634155in}{0.654014in}}%
\pgfpathcurveto{\pgfqpoint{1.641859in}{0.646310in}}{\pgfqpoint{1.652310in}{0.641981in}}{\pgfqpoint{1.663205in}{0.641981in}}%
\pgfusepath{stroke}%
\end{pgfscope}%
\begin{pgfscope}%
\pgfpathrectangle{\pgfqpoint{0.688192in}{0.670138in}}{\pgfqpoint{7.111808in}{5.061530in}}%
\pgfusepath{clip}%
\pgfsetbuttcap%
\pgfsetroundjoin%
\pgfsetlinewidth{1.003750pt}%
\definecolor{currentstroke}{rgb}{0.000000,0.000000,0.000000}%
\pgfsetstrokecolor{currentstroke}%
\pgfsetdash{}{0pt}%
\pgfpathmoveto{\pgfqpoint{5.602904in}{0.672237in}}%
\pgfpathcurveto{\pgfqpoint{5.613799in}{0.672237in}}{\pgfqpoint{5.624250in}{0.676566in}}{\pgfqpoint{5.631955in}{0.684270in}}%
\pgfpathcurveto{\pgfqpoint{5.639659in}{0.691974in}}{\pgfqpoint{5.643988in}{0.702425in}}{\pgfqpoint{5.643988in}{0.713321in}}%
\pgfpathcurveto{\pgfqpoint{5.643988in}{0.724216in}}{\pgfqpoint{5.639659in}{0.734667in}}{\pgfqpoint{5.631955in}{0.742371in}}%
\pgfpathcurveto{\pgfqpoint{5.624250in}{0.750076in}}{\pgfqpoint{5.613799in}{0.754404in}}{\pgfqpoint{5.602904in}{0.754404in}}%
\pgfpathcurveto{\pgfqpoint{5.592008in}{0.754404in}}{\pgfqpoint{5.581557in}{0.750076in}}{\pgfqpoint{5.573853in}{0.742371in}}%
\pgfpathcurveto{\pgfqpoint{5.566149in}{0.734667in}}{\pgfqpoint{5.561820in}{0.724216in}}{\pgfqpoint{5.561820in}{0.713321in}}%
\pgfpathcurveto{\pgfqpoint{5.561820in}{0.702425in}}{\pgfqpoint{5.566149in}{0.691974in}}{\pgfqpoint{5.573853in}{0.684270in}}%
\pgfpathcurveto{\pgfqpoint{5.581557in}{0.676566in}}{\pgfqpoint{5.592008in}{0.672237in}}{\pgfqpoint{5.602904in}{0.672237in}}%
\pgfpathlineto{\pgfqpoint{5.602904in}{0.672237in}}%
\pgfpathclose%
\pgfusepath{stroke}%
\end{pgfscope}%
\begin{pgfscope}%
\pgfpathrectangle{\pgfqpoint{0.688192in}{0.670138in}}{\pgfqpoint{7.111808in}{5.061530in}}%
\pgfusepath{clip}%
\pgfsetbuttcap%
\pgfsetroundjoin%
\pgfsetlinewidth{1.003750pt}%
\definecolor{currentstroke}{rgb}{0.000000,0.000000,0.000000}%
\pgfsetstrokecolor{currentstroke}%
\pgfsetdash{}{0pt}%
\pgfpathmoveto{\pgfqpoint{1.970905in}{0.639806in}}%
\pgfpathcurveto{\pgfqpoint{1.981801in}{0.639806in}}{\pgfqpoint{1.992251in}{0.644135in}}{\pgfqpoint{1.999956in}{0.651839in}}%
\pgfpathcurveto{\pgfqpoint{2.007660in}{0.659543in}}{\pgfqpoint{2.011989in}{0.669994in}}{\pgfqpoint{2.011989in}{0.680890in}}%
\pgfpathcurveto{\pgfqpoint{2.011989in}{0.691785in}}{\pgfqpoint{2.007660in}{0.702236in}}{\pgfqpoint{1.999956in}{0.709941in}}%
\pgfpathcurveto{\pgfqpoint{1.992251in}{0.717645in}}{\pgfqpoint{1.981801in}{0.721974in}}{\pgfqpoint{1.970905in}{0.721974in}}%
\pgfpathcurveto{\pgfqpoint{1.960009in}{0.721974in}}{\pgfqpoint{1.949559in}{0.717645in}}{\pgfqpoint{1.941854in}{0.709941in}}%
\pgfpathcurveto{\pgfqpoint{1.934150in}{0.702236in}}{\pgfqpoint{1.929821in}{0.691785in}}{\pgfqpoint{1.929821in}{0.680890in}}%
\pgfpathcurveto{\pgfqpoint{1.929821in}{0.669994in}}{\pgfqpoint{1.934150in}{0.659543in}}{\pgfqpoint{1.941854in}{0.651839in}}%
\pgfpathcurveto{\pgfqpoint{1.949559in}{0.644135in}}{\pgfqpoint{1.960009in}{0.639806in}}{\pgfqpoint{1.970905in}{0.639806in}}%
\pgfusepath{stroke}%
\end{pgfscope}%
\begin{pgfscope}%
\pgfpathrectangle{\pgfqpoint{0.688192in}{0.670138in}}{\pgfqpoint{7.111808in}{5.061530in}}%
\pgfusepath{clip}%
\pgfsetbuttcap%
\pgfsetroundjoin%
\pgfsetlinewidth{1.003750pt}%
\definecolor{currentstroke}{rgb}{0.000000,0.000000,0.000000}%
\pgfsetstrokecolor{currentstroke}%
\pgfsetdash{}{0pt}%
\pgfpathmoveto{\pgfqpoint{1.761288in}{0.640905in}}%
\pgfpathcurveto{\pgfqpoint{1.772184in}{0.640905in}}{\pgfqpoint{1.782635in}{0.645234in}}{\pgfqpoint{1.790339in}{0.652938in}}%
\pgfpathcurveto{\pgfqpoint{1.798043in}{0.660642in}}{\pgfqpoint{1.802372in}{0.671093in}}{\pgfqpoint{1.802372in}{0.681989in}}%
\pgfpathcurveto{\pgfqpoint{1.802372in}{0.692884in}}{\pgfqpoint{1.798043in}{0.703335in}}{\pgfqpoint{1.790339in}{0.711039in}}%
\pgfpathcurveto{\pgfqpoint{1.782635in}{0.718744in}}{\pgfqpoint{1.772184in}{0.723073in}}{\pgfqpoint{1.761288in}{0.723073in}}%
\pgfpathcurveto{\pgfqpoint{1.750393in}{0.723073in}}{\pgfqpoint{1.739942in}{0.718744in}}{\pgfqpoint{1.732237in}{0.711039in}}%
\pgfpathcurveto{\pgfqpoint{1.724533in}{0.703335in}}{\pgfqpoint{1.720204in}{0.692884in}}{\pgfqpoint{1.720204in}{0.681989in}}%
\pgfpathcurveto{\pgfqpoint{1.720204in}{0.671093in}}{\pgfqpoint{1.724533in}{0.660642in}}{\pgfqpoint{1.732237in}{0.652938in}}%
\pgfpathcurveto{\pgfqpoint{1.739942in}{0.645234in}}{\pgfqpoint{1.750393in}{0.640905in}}{\pgfqpoint{1.761288in}{0.640905in}}%
\pgfusepath{stroke}%
\end{pgfscope}%
\begin{pgfscope}%
\pgfpathrectangle{\pgfqpoint{0.688192in}{0.670138in}}{\pgfqpoint{7.111808in}{5.061530in}}%
\pgfusepath{clip}%
\pgfsetbuttcap%
\pgfsetroundjoin%
\pgfsetlinewidth{1.003750pt}%
\definecolor{currentstroke}{rgb}{0.000000,0.000000,0.000000}%
\pgfsetstrokecolor{currentstroke}%
\pgfsetdash{}{0pt}%
\pgfpathmoveto{\pgfqpoint{1.495917in}{0.642463in}}%
\pgfpathcurveto{\pgfqpoint{1.506813in}{0.642463in}}{\pgfqpoint{1.517263in}{0.646792in}}{\pgfqpoint{1.524968in}{0.654497in}}%
\pgfpathcurveto{\pgfqpoint{1.532672in}{0.662201in}}{\pgfqpoint{1.537001in}{0.672652in}}{\pgfqpoint{1.537001in}{0.683547in}}%
\pgfpathcurveto{\pgfqpoint{1.537001in}{0.694443in}}{\pgfqpoint{1.532672in}{0.704894in}}{\pgfqpoint{1.524968in}{0.712598in}}%
\pgfpathcurveto{\pgfqpoint{1.517263in}{0.720302in}}{\pgfqpoint{1.506813in}{0.724631in}}{\pgfqpoint{1.495917in}{0.724631in}}%
\pgfpathcurveto{\pgfqpoint{1.485021in}{0.724631in}}{\pgfqpoint{1.474571in}{0.720302in}}{\pgfqpoint{1.466866in}{0.712598in}}%
\pgfpathcurveto{\pgfqpoint{1.459162in}{0.704894in}}{\pgfqpoint{1.454833in}{0.694443in}}{\pgfqpoint{1.454833in}{0.683547in}}%
\pgfpathcurveto{\pgfqpoint{1.454833in}{0.672652in}}{\pgfqpoint{1.459162in}{0.662201in}}{\pgfqpoint{1.466866in}{0.654497in}}%
\pgfpathcurveto{\pgfqpoint{1.474571in}{0.646792in}}{\pgfqpoint{1.485021in}{0.642463in}}{\pgfqpoint{1.495917in}{0.642463in}}%
\pgfusepath{stroke}%
\end{pgfscope}%
\begin{pgfscope}%
\pgfpathrectangle{\pgfqpoint{0.688192in}{0.670138in}}{\pgfqpoint{7.111808in}{5.061530in}}%
\pgfusepath{clip}%
\pgfsetbuttcap%
\pgfsetroundjoin%
\pgfsetlinewidth{1.003750pt}%
\definecolor{currentstroke}{rgb}{0.000000,0.000000,0.000000}%
\pgfsetstrokecolor{currentstroke}%
\pgfsetdash{}{0pt}%
\pgfpathmoveto{\pgfqpoint{0.872029in}{0.690616in}}%
\pgfpathcurveto{\pgfqpoint{0.882925in}{0.690616in}}{\pgfqpoint{0.893376in}{0.694944in}}{\pgfqpoint{0.901080in}{0.702649in}}%
\pgfpathcurveto{\pgfqpoint{0.908784in}{0.710353in}}{\pgfqpoint{0.913113in}{0.720804in}}{\pgfqpoint{0.913113in}{0.731700in}}%
\pgfpathcurveto{\pgfqpoint{0.913113in}{0.742595in}}{\pgfqpoint{0.908784in}{0.753046in}}{\pgfqpoint{0.901080in}{0.760750in}}%
\pgfpathcurveto{\pgfqpoint{0.893376in}{0.768455in}}{\pgfqpoint{0.882925in}{0.772783in}}{\pgfqpoint{0.872029in}{0.772783in}}%
\pgfpathcurveto{\pgfqpoint{0.861134in}{0.772783in}}{\pgfqpoint{0.850683in}{0.768455in}}{\pgfqpoint{0.842979in}{0.760750in}}%
\pgfpathcurveto{\pgfqpoint{0.835274in}{0.753046in}}{\pgfqpoint{0.830945in}{0.742595in}}{\pgfqpoint{0.830945in}{0.731700in}}%
\pgfpathcurveto{\pgfqpoint{0.830945in}{0.720804in}}{\pgfqpoint{0.835274in}{0.710353in}}{\pgfqpoint{0.842979in}{0.702649in}}%
\pgfpathcurveto{\pgfqpoint{0.850683in}{0.694944in}}{\pgfqpoint{0.861134in}{0.690616in}}{\pgfqpoint{0.872029in}{0.690616in}}%
\pgfpathlineto{\pgfqpoint{0.872029in}{0.690616in}}%
\pgfpathclose%
\pgfusepath{stroke}%
\end{pgfscope}%
\begin{pgfscope}%
\pgfpathrectangle{\pgfqpoint{0.688192in}{0.670138in}}{\pgfqpoint{7.111808in}{5.061530in}}%
\pgfusepath{clip}%
\pgfsetbuttcap%
\pgfsetroundjoin%
\pgfsetlinewidth{1.003750pt}%
\definecolor{currentstroke}{rgb}{0.000000,0.000000,0.000000}%
\pgfsetstrokecolor{currentstroke}%
\pgfsetdash{}{0pt}%
\pgfpathmoveto{\pgfqpoint{2.244939in}{0.638531in}}%
\pgfpathcurveto{\pgfqpoint{2.255835in}{0.638531in}}{\pgfqpoint{2.266286in}{0.642860in}}{\pgfqpoint{2.273990in}{0.650564in}}%
\pgfpathcurveto{\pgfqpoint{2.281695in}{0.658269in}}{\pgfqpoint{2.286023in}{0.668719in}}{\pgfqpoint{2.286023in}{0.679615in}}%
\pgfpathcurveto{\pgfqpoint{2.286023in}{0.690511in}}{\pgfqpoint{2.281695in}{0.700961in}}{\pgfqpoint{2.273990in}{0.708666in}}%
\pgfpathcurveto{\pgfqpoint{2.266286in}{0.716370in}}{\pgfqpoint{2.255835in}{0.720699in}}{\pgfqpoint{2.244939in}{0.720699in}}%
\pgfpathcurveto{\pgfqpoint{2.234044in}{0.720699in}}{\pgfqpoint{2.223593in}{0.716370in}}{\pgfqpoint{2.215889in}{0.708666in}}%
\pgfpathcurveto{\pgfqpoint{2.208184in}{0.700961in}}{\pgfqpoint{2.203856in}{0.690511in}}{\pgfqpoint{2.203856in}{0.679615in}}%
\pgfpathcurveto{\pgfqpoint{2.203856in}{0.668719in}}{\pgfqpoint{2.208184in}{0.658269in}}{\pgfqpoint{2.215889in}{0.650564in}}%
\pgfpathcurveto{\pgfqpoint{2.223593in}{0.642860in}}{\pgfqpoint{2.234044in}{0.638531in}}{\pgfqpoint{2.244939in}{0.638531in}}%
\pgfusepath{stroke}%
\end{pgfscope}%
\begin{pgfscope}%
\pgfpathrectangle{\pgfqpoint{0.688192in}{0.670138in}}{\pgfqpoint{7.111808in}{5.061530in}}%
\pgfusepath{clip}%
\pgfsetbuttcap%
\pgfsetroundjoin%
\pgfsetlinewidth{1.003750pt}%
\definecolor{currentstroke}{rgb}{0.000000,0.000000,0.000000}%
\pgfsetstrokecolor{currentstroke}%
\pgfsetdash{}{0pt}%
\pgfpathmoveto{\pgfqpoint{2.458239in}{0.697123in}}%
\pgfpathcurveto{\pgfqpoint{2.469134in}{0.697123in}}{\pgfqpoint{2.479585in}{0.701452in}}{\pgfqpoint{2.487289in}{0.709157in}}%
\pgfpathcurveto{\pgfqpoint{2.494994in}{0.716861in}}{\pgfqpoint{2.499323in}{0.727312in}}{\pgfqpoint{2.499323in}{0.738207in}}%
\pgfpathcurveto{\pgfqpoint{2.499323in}{0.749103in}}{\pgfqpoint{2.494994in}{0.759554in}}{\pgfqpoint{2.487289in}{0.767258in}}%
\pgfpathcurveto{\pgfqpoint{2.479585in}{0.774962in}}{\pgfqpoint{2.469134in}{0.779291in}}{\pgfqpoint{2.458239in}{0.779291in}}%
\pgfpathcurveto{\pgfqpoint{2.447343in}{0.779291in}}{\pgfqpoint{2.436892in}{0.774962in}}{\pgfqpoint{2.429188in}{0.767258in}}%
\pgfpathcurveto{\pgfqpoint{2.421484in}{0.759554in}}{\pgfqpoint{2.417155in}{0.749103in}}{\pgfqpoint{2.417155in}{0.738207in}}%
\pgfpathcurveto{\pgfqpoint{2.417155in}{0.727312in}}{\pgfqpoint{2.421484in}{0.716861in}}{\pgfqpoint{2.429188in}{0.709157in}}%
\pgfpathcurveto{\pgfqpoint{2.436892in}{0.701452in}}{\pgfqpoint{2.447343in}{0.697123in}}{\pgfqpoint{2.458239in}{0.697123in}}%
\pgfpathlineto{\pgfqpoint{2.458239in}{0.697123in}}%
\pgfpathclose%
\pgfusepath{stroke}%
\end{pgfscope}%
\begin{pgfscope}%
\pgfpathrectangle{\pgfqpoint{0.688192in}{0.670138in}}{\pgfqpoint{7.111808in}{5.061530in}}%
\pgfusepath{clip}%
\pgfsetbuttcap%
\pgfsetroundjoin%
\pgfsetlinewidth{1.003750pt}%
\definecolor{currentstroke}{rgb}{0.000000,0.000000,0.000000}%
\pgfsetstrokecolor{currentstroke}%
\pgfsetdash{}{0pt}%
\pgfpathmoveto{\pgfqpoint{1.108617in}{0.647679in}}%
\pgfpathcurveto{\pgfqpoint{1.119513in}{0.647679in}}{\pgfqpoint{1.129964in}{0.652008in}}{\pgfqpoint{1.137668in}{0.659712in}}%
\pgfpathcurveto{\pgfqpoint{1.145372in}{0.667417in}}{\pgfqpoint{1.149701in}{0.677868in}}{\pgfqpoint{1.149701in}{0.688763in}}%
\pgfpathcurveto{\pgfqpoint{1.149701in}{0.699659in}}{\pgfqpoint{1.145372in}{0.710109in}}{\pgfqpoint{1.137668in}{0.717814in}}%
\pgfpathcurveto{\pgfqpoint{1.129964in}{0.725518in}}{\pgfqpoint{1.119513in}{0.729847in}}{\pgfqpoint{1.108617in}{0.729847in}}%
\pgfpathcurveto{\pgfqpoint{1.097722in}{0.729847in}}{\pgfqpoint{1.087271in}{0.725518in}}{\pgfqpoint{1.079567in}{0.717814in}}%
\pgfpathcurveto{\pgfqpoint{1.071862in}{0.710109in}}{\pgfqpoint{1.067534in}{0.699659in}}{\pgfqpoint{1.067534in}{0.688763in}}%
\pgfpathcurveto{\pgfqpoint{1.067534in}{0.677868in}}{\pgfqpoint{1.071862in}{0.667417in}}{\pgfqpoint{1.079567in}{0.659712in}}%
\pgfpathcurveto{\pgfqpoint{1.087271in}{0.652008in}}{\pgfqpoint{1.097722in}{0.647679in}}{\pgfqpoint{1.108617in}{0.647679in}}%
\pgfusepath{stroke}%
\end{pgfscope}%
\begin{pgfscope}%
\pgfpathrectangle{\pgfqpoint{0.688192in}{0.670138in}}{\pgfqpoint{7.111808in}{5.061530in}}%
\pgfusepath{clip}%
\pgfsetbuttcap%
\pgfsetroundjoin%
\pgfsetlinewidth{1.003750pt}%
\definecolor{currentstroke}{rgb}{0.000000,0.000000,0.000000}%
\pgfsetstrokecolor{currentstroke}%
\pgfsetdash{}{0pt}%
\pgfpathmoveto{\pgfqpoint{1.495917in}{0.642463in}}%
\pgfpathcurveto{\pgfqpoint{1.506813in}{0.642463in}}{\pgfqpoint{1.517263in}{0.646792in}}{\pgfqpoint{1.524968in}{0.654497in}}%
\pgfpathcurveto{\pgfqpoint{1.532672in}{0.662201in}}{\pgfqpoint{1.537001in}{0.672652in}}{\pgfqpoint{1.537001in}{0.683547in}}%
\pgfpathcurveto{\pgfqpoint{1.537001in}{0.694443in}}{\pgfqpoint{1.532672in}{0.704894in}}{\pgfqpoint{1.524968in}{0.712598in}}%
\pgfpathcurveto{\pgfqpoint{1.517263in}{0.720302in}}{\pgfqpoint{1.506813in}{0.724631in}}{\pgfqpoint{1.495917in}{0.724631in}}%
\pgfpathcurveto{\pgfqpoint{1.485021in}{0.724631in}}{\pgfqpoint{1.474571in}{0.720302in}}{\pgfqpoint{1.466866in}{0.712598in}}%
\pgfpathcurveto{\pgfqpoint{1.459162in}{0.704894in}}{\pgfqpoint{1.454833in}{0.694443in}}{\pgfqpoint{1.454833in}{0.683547in}}%
\pgfpathcurveto{\pgfqpoint{1.454833in}{0.672652in}}{\pgfqpoint{1.459162in}{0.662201in}}{\pgfqpoint{1.466866in}{0.654497in}}%
\pgfpathcurveto{\pgfqpoint{1.474571in}{0.646792in}}{\pgfqpoint{1.485021in}{0.642463in}}{\pgfqpoint{1.495917in}{0.642463in}}%
\pgfusepath{stroke}%
\end{pgfscope}%
\begin{pgfscope}%
\pgfpathrectangle{\pgfqpoint{0.688192in}{0.670138in}}{\pgfqpoint{7.111808in}{5.061530in}}%
\pgfusepath{clip}%
\pgfsetbuttcap%
\pgfsetroundjoin%
\pgfsetlinewidth{1.003750pt}%
\definecolor{currentstroke}{rgb}{0.000000,0.000000,0.000000}%
\pgfsetstrokecolor{currentstroke}%
\pgfsetdash{}{0pt}%
\pgfpathmoveto{\pgfqpoint{1.732115in}{0.641446in}}%
\pgfpathcurveto{\pgfqpoint{1.743010in}{0.641446in}}{\pgfqpoint{1.753461in}{0.645774in}}{\pgfqpoint{1.761165in}{0.653479in}}%
\pgfpathcurveto{\pgfqpoint{1.768870in}{0.661183in}}{\pgfqpoint{1.773199in}{0.671634in}}{\pgfqpoint{1.773199in}{0.682529in}}%
\pgfpathcurveto{\pgfqpoint{1.773199in}{0.693425in}}{\pgfqpoint{1.768870in}{0.703876in}}{\pgfqpoint{1.761165in}{0.711580in}}%
\pgfpathcurveto{\pgfqpoint{1.753461in}{0.719285in}}{\pgfqpoint{1.743010in}{0.723613in}}{\pgfqpoint{1.732115in}{0.723613in}}%
\pgfpathcurveto{\pgfqpoint{1.721219in}{0.723613in}}{\pgfqpoint{1.710768in}{0.719285in}}{\pgfqpoint{1.703064in}{0.711580in}}%
\pgfpathcurveto{\pgfqpoint{1.695360in}{0.703876in}}{\pgfqpoint{1.691031in}{0.693425in}}{\pgfqpoint{1.691031in}{0.682529in}}%
\pgfpathcurveto{\pgfqpoint{1.691031in}{0.671634in}}{\pgfqpoint{1.695360in}{0.661183in}}{\pgfqpoint{1.703064in}{0.653479in}}%
\pgfpathcurveto{\pgfqpoint{1.710768in}{0.645774in}}{\pgfqpoint{1.721219in}{0.641446in}}{\pgfqpoint{1.732115in}{0.641446in}}%
\pgfusepath{stroke}%
\end{pgfscope}%
\begin{pgfscope}%
\pgfpathrectangle{\pgfqpoint{0.688192in}{0.670138in}}{\pgfqpoint{7.111808in}{5.061530in}}%
\pgfusepath{clip}%
\pgfsetbuttcap%
\pgfsetroundjoin%
\pgfsetlinewidth{1.003750pt}%
\definecolor{currentstroke}{rgb}{0.000000,0.000000,0.000000}%
\pgfsetstrokecolor{currentstroke}%
\pgfsetdash{}{0pt}%
\pgfpathmoveto{\pgfqpoint{7.143599in}{1.867454in}}%
\pgfpathcurveto{\pgfqpoint{7.154494in}{1.867454in}}{\pgfqpoint{7.164945in}{1.871783in}}{\pgfqpoint{7.172649in}{1.879487in}}%
\pgfpathcurveto{\pgfqpoint{7.180354in}{1.887192in}}{\pgfqpoint{7.184683in}{1.897642in}}{\pgfqpoint{7.184683in}{1.908538in}}%
\pgfpathcurveto{\pgfqpoint{7.184683in}{1.919434in}}{\pgfqpoint{7.180354in}{1.929884in}}{\pgfqpoint{7.172649in}{1.937589in}}%
\pgfpathcurveto{\pgfqpoint{7.164945in}{1.945293in}}{\pgfqpoint{7.154494in}{1.949622in}}{\pgfqpoint{7.143599in}{1.949622in}}%
\pgfpathcurveto{\pgfqpoint{7.132703in}{1.949622in}}{\pgfqpoint{7.122252in}{1.945293in}}{\pgfqpoint{7.114548in}{1.937589in}}%
\pgfpathcurveto{\pgfqpoint{7.106844in}{1.929884in}}{\pgfqpoint{7.102515in}{1.919434in}}{\pgfqpoint{7.102515in}{1.908538in}}%
\pgfpathcurveto{\pgfqpoint{7.102515in}{1.897642in}}{\pgfqpoint{7.106844in}{1.887192in}}{\pgfqpoint{7.114548in}{1.879487in}}%
\pgfpathcurveto{\pgfqpoint{7.122252in}{1.871783in}}{\pgfqpoint{7.132703in}{1.867454in}}{\pgfqpoint{7.143599in}{1.867454in}}%
\pgfpathlineto{\pgfqpoint{7.143599in}{1.867454in}}%
\pgfpathclose%
\pgfusepath{stroke}%
\end{pgfscope}%
\begin{pgfscope}%
\pgfpathrectangle{\pgfqpoint{0.688192in}{0.670138in}}{\pgfqpoint{7.111808in}{5.061530in}}%
\pgfusepath{clip}%
\pgfsetbuttcap%
\pgfsetroundjoin%
\pgfsetlinewidth{1.003750pt}%
\definecolor{currentstroke}{rgb}{0.000000,0.000000,0.000000}%
\pgfsetstrokecolor{currentstroke}%
\pgfsetdash{}{0pt}%
\pgfpathmoveto{\pgfqpoint{4.420707in}{0.631172in}}%
\pgfpathcurveto{\pgfqpoint{4.431602in}{0.631172in}}{\pgfqpoint{4.442053in}{0.635501in}}{\pgfqpoint{4.449757in}{0.643206in}}%
\pgfpathcurveto{\pgfqpoint{4.457462in}{0.650910in}}{\pgfqpoint{4.461791in}{0.661361in}}{\pgfqpoint{4.461791in}{0.672256in}}%
\pgfpathcurveto{\pgfqpoint{4.461791in}{0.683152in}}{\pgfqpoint{4.457462in}{0.693603in}}{\pgfqpoint{4.449757in}{0.701307in}}%
\pgfpathcurveto{\pgfqpoint{4.442053in}{0.709011in}}{\pgfqpoint{4.431602in}{0.713340in}}{\pgfqpoint{4.420707in}{0.713340in}}%
\pgfpathcurveto{\pgfqpoint{4.409811in}{0.713340in}}{\pgfqpoint{4.399360in}{0.709011in}}{\pgfqpoint{4.391656in}{0.701307in}}%
\pgfpathcurveto{\pgfqpoint{4.383952in}{0.693603in}}{\pgfqpoint{4.379623in}{0.683152in}}{\pgfqpoint{4.379623in}{0.672256in}}%
\pgfpathcurveto{\pgfqpoint{4.379623in}{0.661361in}}{\pgfqpoint{4.383952in}{0.650910in}}{\pgfqpoint{4.391656in}{0.643206in}}%
\pgfpathcurveto{\pgfqpoint{4.399360in}{0.635501in}}{\pgfqpoint{4.409811in}{0.631172in}}{\pgfqpoint{4.420707in}{0.631172in}}%
\pgfusepath{stroke}%
\end{pgfscope}%
\begin{pgfscope}%
\pgfpathrectangle{\pgfqpoint{0.688192in}{0.670138in}}{\pgfqpoint{7.111808in}{5.061530in}}%
\pgfusepath{clip}%
\pgfsetbuttcap%
\pgfsetroundjoin%
\pgfsetlinewidth{1.003750pt}%
\definecolor{currentstroke}{rgb}{0.000000,0.000000,0.000000}%
\pgfsetstrokecolor{currentstroke}%
\pgfsetdash{}{0pt}%
\pgfpathmoveto{\pgfqpoint{1.731531in}{0.641456in}}%
\pgfpathcurveto{\pgfqpoint{1.742427in}{0.641456in}}{\pgfqpoint{1.752878in}{0.645785in}}{\pgfqpoint{1.760582in}{0.653489in}}%
\pgfpathcurveto{\pgfqpoint{1.768286in}{0.661194in}}{\pgfqpoint{1.772615in}{0.671645in}}{\pgfqpoint{1.772615in}{0.682540in}}%
\pgfpathcurveto{\pgfqpoint{1.772615in}{0.693436in}}{\pgfqpoint{1.768286in}{0.703886in}}{\pgfqpoint{1.760582in}{0.711591in}}%
\pgfpathcurveto{\pgfqpoint{1.752878in}{0.719295in}}{\pgfqpoint{1.742427in}{0.723624in}}{\pgfqpoint{1.731531in}{0.723624in}}%
\pgfpathcurveto{\pgfqpoint{1.720636in}{0.723624in}}{\pgfqpoint{1.710185in}{0.719295in}}{\pgfqpoint{1.702481in}{0.711591in}}%
\pgfpathcurveto{\pgfqpoint{1.694776in}{0.703886in}}{\pgfqpoint{1.690447in}{0.693436in}}{\pgfqpoint{1.690447in}{0.682540in}}%
\pgfpathcurveto{\pgfqpoint{1.690447in}{0.671645in}}{\pgfqpoint{1.694776in}{0.661194in}}{\pgfqpoint{1.702481in}{0.653489in}}%
\pgfpathcurveto{\pgfqpoint{1.710185in}{0.645785in}}{\pgfqpoint{1.720636in}{0.641456in}}{\pgfqpoint{1.731531in}{0.641456in}}%
\pgfusepath{stroke}%
\end{pgfscope}%
\begin{pgfscope}%
\pgfpathrectangle{\pgfqpoint{0.688192in}{0.670138in}}{\pgfqpoint{7.111808in}{5.061530in}}%
\pgfusepath{clip}%
\pgfsetbuttcap%
\pgfsetroundjoin%
\pgfsetlinewidth{1.003750pt}%
\definecolor{currentstroke}{rgb}{0.000000,0.000000,0.000000}%
\pgfsetstrokecolor{currentstroke}%
\pgfsetdash{}{0pt}%
\pgfpathmoveto{\pgfqpoint{1.469414in}{0.642878in}}%
\pgfpathcurveto{\pgfqpoint{1.480310in}{0.642878in}}{\pgfqpoint{1.490760in}{0.647207in}}{\pgfqpoint{1.498465in}{0.654911in}}%
\pgfpathcurveto{\pgfqpoint{1.506169in}{0.662615in}}{\pgfqpoint{1.510498in}{0.673066in}}{\pgfqpoint{1.510498in}{0.683962in}}%
\pgfpathcurveto{\pgfqpoint{1.510498in}{0.694857in}}{\pgfqpoint{1.506169in}{0.705308in}}{\pgfqpoint{1.498465in}{0.713012in}}%
\pgfpathcurveto{\pgfqpoint{1.490760in}{0.720717in}}{\pgfqpoint{1.480310in}{0.725046in}}{\pgfqpoint{1.469414in}{0.725046in}}%
\pgfpathcurveto{\pgfqpoint{1.458519in}{0.725046in}}{\pgfqpoint{1.448068in}{0.720717in}}{\pgfqpoint{1.440363in}{0.713012in}}%
\pgfpathcurveto{\pgfqpoint{1.432659in}{0.705308in}}{\pgfqpoint{1.428330in}{0.694857in}}{\pgfqpoint{1.428330in}{0.683962in}}%
\pgfpathcurveto{\pgfqpoint{1.428330in}{0.673066in}}{\pgfqpoint{1.432659in}{0.662615in}}{\pgfqpoint{1.440363in}{0.654911in}}%
\pgfpathcurveto{\pgfqpoint{1.448068in}{0.647207in}}{\pgfqpoint{1.458519in}{0.642878in}}{\pgfqpoint{1.469414in}{0.642878in}}%
\pgfusepath{stroke}%
\end{pgfscope}%
\begin{pgfscope}%
\pgfpathrectangle{\pgfqpoint{0.688192in}{0.670138in}}{\pgfqpoint{7.111808in}{5.061530in}}%
\pgfusepath{clip}%
\pgfsetbuttcap%
\pgfsetroundjoin%
\pgfsetlinewidth{1.003750pt}%
\definecolor{currentstroke}{rgb}{0.000000,0.000000,0.000000}%
\pgfsetstrokecolor{currentstroke}%
\pgfsetdash{}{0pt}%
\pgfpathmoveto{\pgfqpoint{2.952156in}{1.989543in}}%
\pgfpathcurveto{\pgfqpoint{2.963051in}{1.989543in}}{\pgfqpoint{2.973502in}{1.993872in}}{\pgfqpoint{2.981206in}{2.001576in}}%
\pgfpathcurveto{\pgfqpoint{2.988911in}{2.009281in}}{\pgfqpoint{2.993240in}{2.019731in}}{\pgfqpoint{2.993240in}{2.030627in}}%
\pgfpathcurveto{\pgfqpoint{2.993240in}{2.041522in}}{\pgfqpoint{2.988911in}{2.051973in}}{\pgfqpoint{2.981206in}{2.059678in}}%
\pgfpathcurveto{\pgfqpoint{2.973502in}{2.067382in}}{\pgfqpoint{2.963051in}{2.071711in}}{\pgfqpoint{2.952156in}{2.071711in}}%
\pgfpathcurveto{\pgfqpoint{2.941260in}{2.071711in}}{\pgfqpoint{2.930809in}{2.067382in}}{\pgfqpoint{2.923105in}{2.059678in}}%
\pgfpathcurveto{\pgfqpoint{2.915401in}{2.051973in}}{\pgfqpoint{2.911072in}{2.041522in}}{\pgfqpoint{2.911072in}{2.030627in}}%
\pgfpathcurveto{\pgfqpoint{2.911072in}{2.019731in}}{\pgfqpoint{2.915401in}{2.009281in}}{\pgfqpoint{2.923105in}{2.001576in}}%
\pgfpathcurveto{\pgfqpoint{2.930809in}{1.993872in}}{\pgfqpoint{2.941260in}{1.989543in}}{\pgfqpoint{2.952156in}{1.989543in}}%
\pgfpathlineto{\pgfqpoint{2.952156in}{1.989543in}}%
\pgfpathclose%
\pgfusepath{stroke}%
\end{pgfscope}%
\begin{pgfscope}%
\pgfpathrectangle{\pgfqpoint{0.688192in}{0.670138in}}{\pgfqpoint{7.111808in}{5.061530in}}%
\pgfusepath{clip}%
\pgfsetbuttcap%
\pgfsetroundjoin%
\pgfsetlinewidth{1.003750pt}%
\definecolor{currentstroke}{rgb}{0.000000,0.000000,0.000000}%
\pgfsetstrokecolor{currentstroke}%
\pgfsetdash{}{0pt}%
\pgfpathmoveto{\pgfqpoint{1.759976in}{0.640931in}}%
\pgfpathcurveto{\pgfqpoint{1.770871in}{0.640931in}}{\pgfqpoint{1.781322in}{0.645260in}}{\pgfqpoint{1.789027in}{0.652964in}}%
\pgfpathcurveto{\pgfqpoint{1.796731in}{0.660668in}}{\pgfqpoint{1.801060in}{0.671119in}}{\pgfqpoint{1.801060in}{0.682015in}}%
\pgfpathcurveto{\pgfqpoint{1.801060in}{0.692910in}}{\pgfqpoint{1.796731in}{0.703361in}}{\pgfqpoint{1.789027in}{0.711065in}}%
\pgfpathcurveto{\pgfqpoint{1.781322in}{0.718770in}}{\pgfqpoint{1.770871in}{0.723099in}}{\pgfqpoint{1.759976in}{0.723099in}}%
\pgfpathcurveto{\pgfqpoint{1.749080in}{0.723099in}}{\pgfqpoint{1.738629in}{0.718770in}}{\pgfqpoint{1.730925in}{0.711065in}}%
\pgfpathcurveto{\pgfqpoint{1.723221in}{0.703361in}}{\pgfqpoint{1.718892in}{0.692910in}}{\pgfqpoint{1.718892in}{0.682015in}}%
\pgfpathcurveto{\pgfqpoint{1.718892in}{0.671119in}}{\pgfqpoint{1.723221in}{0.660668in}}{\pgfqpoint{1.730925in}{0.652964in}}%
\pgfpathcurveto{\pgfqpoint{1.738629in}{0.645260in}}{\pgfqpoint{1.749080in}{0.640931in}}{\pgfqpoint{1.759976in}{0.640931in}}%
\pgfusepath{stroke}%
\end{pgfscope}%
\begin{pgfscope}%
\pgfpathrectangle{\pgfqpoint{0.688192in}{0.670138in}}{\pgfqpoint{7.111808in}{5.061530in}}%
\pgfusepath{clip}%
\pgfsetbuttcap%
\pgfsetroundjoin%
\pgfsetlinewidth{1.003750pt}%
\definecolor{currentstroke}{rgb}{0.000000,0.000000,0.000000}%
\pgfsetstrokecolor{currentstroke}%
\pgfsetdash{}{0pt}%
\pgfpathmoveto{\pgfqpoint{0.849953in}{0.698273in}}%
\pgfpathcurveto{\pgfqpoint{0.860849in}{0.698273in}}{\pgfqpoint{0.871300in}{0.702602in}}{\pgfqpoint{0.879004in}{0.710307in}}%
\pgfpathcurveto{\pgfqpoint{0.886709in}{0.718011in}}{\pgfqpoint{0.891037in}{0.728462in}}{\pgfqpoint{0.891037in}{0.739357in}}%
\pgfpathcurveto{\pgfqpoint{0.891037in}{0.750253in}}{\pgfqpoint{0.886709in}{0.760704in}}{\pgfqpoint{0.879004in}{0.768408in}}%
\pgfpathcurveto{\pgfqpoint{0.871300in}{0.776112in}}{\pgfqpoint{0.860849in}{0.780441in}}{\pgfqpoint{0.849953in}{0.780441in}}%
\pgfpathcurveto{\pgfqpoint{0.839058in}{0.780441in}}{\pgfqpoint{0.828607in}{0.776112in}}{\pgfqpoint{0.820903in}{0.768408in}}%
\pgfpathcurveto{\pgfqpoint{0.813198in}{0.760704in}}{\pgfqpoint{0.808870in}{0.750253in}}{\pgfqpoint{0.808870in}{0.739357in}}%
\pgfpathcurveto{\pgfqpoint{0.808870in}{0.728462in}}{\pgfqpoint{0.813198in}{0.718011in}}{\pgfqpoint{0.820903in}{0.710307in}}%
\pgfpathcurveto{\pgfqpoint{0.828607in}{0.702602in}}{\pgfqpoint{0.839058in}{0.698273in}}{\pgfqpoint{0.849953in}{0.698273in}}%
\pgfpathlineto{\pgfqpoint{0.849953in}{0.698273in}}%
\pgfpathclose%
\pgfusepath{stroke}%
\end{pgfscope}%
\begin{pgfscope}%
\pgfpathrectangle{\pgfqpoint{0.688192in}{0.670138in}}{\pgfqpoint{7.111808in}{5.061530in}}%
\pgfusepath{clip}%
\pgfsetbuttcap%
\pgfsetroundjoin%
\pgfsetlinewidth{1.003750pt}%
\definecolor{currentstroke}{rgb}{0.000000,0.000000,0.000000}%
\pgfsetstrokecolor{currentstroke}%
\pgfsetdash{}{0pt}%
\pgfpathmoveto{\pgfqpoint{1.077361in}{0.651185in}}%
\pgfpathcurveto{\pgfqpoint{1.088256in}{0.651185in}}{\pgfqpoint{1.098707in}{0.655514in}}{\pgfqpoint{1.106411in}{0.663218in}}%
\pgfpathcurveto{\pgfqpoint{1.114116in}{0.670923in}}{\pgfqpoint{1.118445in}{0.681373in}}{\pgfqpoint{1.118445in}{0.692269in}}%
\pgfpathcurveto{\pgfqpoint{1.118445in}{0.703165in}}{\pgfqpoint{1.114116in}{0.713615in}}{\pgfqpoint{1.106411in}{0.721320in}}%
\pgfpathcurveto{\pgfqpoint{1.098707in}{0.729024in}}{\pgfqpoint{1.088256in}{0.733353in}}{\pgfqpoint{1.077361in}{0.733353in}}%
\pgfpathcurveto{\pgfqpoint{1.066465in}{0.733353in}}{\pgfqpoint{1.056014in}{0.729024in}}{\pgfqpoint{1.048310in}{0.721320in}}%
\pgfpathcurveto{\pgfqpoint{1.040606in}{0.713615in}}{\pgfqpoint{1.036277in}{0.703165in}}{\pgfqpoint{1.036277in}{0.692269in}}%
\pgfpathcurveto{\pgfqpoint{1.036277in}{0.681373in}}{\pgfqpoint{1.040606in}{0.670923in}}{\pgfqpoint{1.048310in}{0.663218in}}%
\pgfpathcurveto{\pgfqpoint{1.056014in}{0.655514in}}{\pgfqpoint{1.066465in}{0.651185in}}{\pgfqpoint{1.077361in}{0.651185in}}%
\pgfusepath{stroke}%
\end{pgfscope}%
\begin{pgfscope}%
\pgfpathrectangle{\pgfqpoint{0.688192in}{0.670138in}}{\pgfqpoint{7.111808in}{5.061530in}}%
\pgfusepath{clip}%
\pgfsetbuttcap%
\pgfsetroundjoin%
\pgfsetlinewidth{1.003750pt}%
\definecolor{currentstroke}{rgb}{0.000000,0.000000,0.000000}%
\pgfsetstrokecolor{currentstroke}%
\pgfsetdash{}{0pt}%
\pgfpathmoveto{\pgfqpoint{2.911505in}{2.423179in}}%
\pgfpathcurveto{\pgfqpoint{2.922401in}{2.423179in}}{\pgfqpoint{2.932851in}{2.427508in}}{\pgfqpoint{2.940556in}{2.435212in}}%
\pgfpathcurveto{\pgfqpoint{2.948260in}{2.442916in}}{\pgfqpoint{2.952589in}{2.453367in}}{\pgfqpoint{2.952589in}{2.464263in}}%
\pgfpathcurveto{\pgfqpoint{2.952589in}{2.475158in}}{\pgfqpoint{2.948260in}{2.485609in}}{\pgfqpoint{2.940556in}{2.493313in}}%
\pgfpathcurveto{\pgfqpoint{2.932851in}{2.501018in}}{\pgfqpoint{2.922401in}{2.505346in}}{\pgfqpoint{2.911505in}{2.505346in}}%
\pgfpathcurveto{\pgfqpoint{2.900609in}{2.505346in}}{\pgfqpoint{2.890159in}{2.501018in}}{\pgfqpoint{2.882454in}{2.493313in}}%
\pgfpathcurveto{\pgfqpoint{2.874750in}{2.485609in}}{\pgfqpoint{2.870421in}{2.475158in}}{\pgfqpoint{2.870421in}{2.464263in}}%
\pgfpathcurveto{\pgfqpoint{2.870421in}{2.453367in}}{\pgfqpoint{2.874750in}{2.442916in}}{\pgfqpoint{2.882454in}{2.435212in}}%
\pgfpathcurveto{\pgfqpoint{2.890159in}{2.427508in}}{\pgfqpoint{2.900609in}{2.423179in}}{\pgfqpoint{2.911505in}{2.423179in}}%
\pgfpathlineto{\pgfqpoint{2.911505in}{2.423179in}}%
\pgfpathclose%
\pgfusepath{stroke}%
\end{pgfscope}%
\begin{pgfscope}%
\pgfpathrectangle{\pgfqpoint{0.688192in}{0.670138in}}{\pgfqpoint{7.111808in}{5.061530in}}%
\pgfusepath{clip}%
\pgfsetbuttcap%
\pgfsetroundjoin%
\pgfsetlinewidth{1.003750pt}%
\definecolor{currentstroke}{rgb}{0.000000,0.000000,0.000000}%
\pgfsetstrokecolor{currentstroke}%
\pgfsetdash{}{0pt}%
\pgfpathmoveto{\pgfqpoint{0.990735in}{0.663640in}}%
\pgfpathcurveto{\pgfqpoint{1.001630in}{0.663640in}}{\pgfqpoint{1.012081in}{0.667969in}}{\pgfqpoint{1.019785in}{0.675673in}}%
\pgfpathcurveto{\pgfqpoint{1.027490in}{0.683378in}}{\pgfqpoint{1.031818in}{0.693828in}}{\pgfqpoint{1.031818in}{0.704724in}}%
\pgfpathcurveto{\pgfqpoint{1.031818in}{0.715620in}}{\pgfqpoint{1.027490in}{0.726070in}}{\pgfqpoint{1.019785in}{0.733775in}}%
\pgfpathcurveto{\pgfqpoint{1.012081in}{0.741479in}}{\pgfqpoint{1.001630in}{0.745808in}}{\pgfqpoint{0.990735in}{0.745808in}}%
\pgfpathcurveto{\pgfqpoint{0.979839in}{0.745808in}}{\pgfqpoint{0.969388in}{0.741479in}}{\pgfqpoint{0.961684in}{0.733775in}}%
\pgfpathcurveto{\pgfqpoint{0.953980in}{0.726070in}}{\pgfqpoint{0.949651in}{0.715620in}}{\pgfqpoint{0.949651in}{0.704724in}}%
\pgfpathcurveto{\pgfqpoint{0.949651in}{0.693828in}}{\pgfqpoint{0.953980in}{0.683378in}}{\pgfqpoint{0.961684in}{0.675673in}}%
\pgfpathcurveto{\pgfqpoint{0.969388in}{0.667969in}}{\pgfqpoint{0.979839in}{0.663640in}}{\pgfqpoint{0.990735in}{0.663640in}}%
\pgfusepath{stroke}%
\end{pgfscope}%
\begin{pgfscope}%
\pgfpathrectangle{\pgfqpoint{0.688192in}{0.670138in}}{\pgfqpoint{7.111808in}{5.061530in}}%
\pgfusepath{clip}%
\pgfsetbuttcap%
\pgfsetroundjoin%
\pgfsetlinewidth{1.003750pt}%
\definecolor{currentstroke}{rgb}{0.000000,0.000000,0.000000}%
\pgfsetstrokecolor{currentstroke}%
\pgfsetdash{}{0pt}%
\pgfpathmoveto{\pgfqpoint{1.889926in}{2.920579in}}%
\pgfpathcurveto{\pgfqpoint{1.900822in}{2.920579in}}{\pgfqpoint{1.911273in}{2.924908in}}{\pgfqpoint{1.918977in}{2.932613in}}%
\pgfpathcurveto{\pgfqpoint{1.926681in}{2.940317in}}{\pgfqpoint{1.931010in}{2.950768in}}{\pgfqpoint{1.931010in}{2.961663in}}%
\pgfpathcurveto{\pgfqpoint{1.931010in}{2.972559in}}{\pgfqpoint{1.926681in}{2.983010in}}{\pgfqpoint{1.918977in}{2.990714in}}%
\pgfpathcurveto{\pgfqpoint{1.911273in}{2.998418in}}{\pgfqpoint{1.900822in}{3.002747in}}{\pgfqpoint{1.889926in}{3.002747in}}%
\pgfpathcurveto{\pgfqpoint{1.879031in}{3.002747in}}{\pgfqpoint{1.868580in}{2.998418in}}{\pgfqpoint{1.860876in}{2.990714in}}%
\pgfpathcurveto{\pgfqpoint{1.853171in}{2.983010in}}{\pgfqpoint{1.848843in}{2.972559in}}{\pgfqpoint{1.848843in}{2.961663in}}%
\pgfpathcurveto{\pgfqpoint{1.848843in}{2.950768in}}{\pgfqpoint{1.853171in}{2.940317in}}{\pgfqpoint{1.860876in}{2.932613in}}%
\pgfpathcurveto{\pgfqpoint{1.868580in}{2.924908in}}{\pgfqpoint{1.879031in}{2.920579in}}{\pgfqpoint{1.889926in}{2.920579in}}%
\pgfpathlineto{\pgfqpoint{1.889926in}{2.920579in}}%
\pgfpathclose%
\pgfusepath{stroke}%
\end{pgfscope}%
\begin{pgfscope}%
\pgfpathrectangle{\pgfqpoint{0.688192in}{0.670138in}}{\pgfqpoint{7.111808in}{5.061530in}}%
\pgfusepath{clip}%
\pgfsetbuttcap%
\pgfsetroundjoin%
\pgfsetlinewidth{1.003750pt}%
\definecolor{currentstroke}{rgb}{0.000000,0.000000,0.000000}%
\pgfsetstrokecolor{currentstroke}%
\pgfsetdash{}{0pt}%
\pgfpathmoveto{\pgfqpoint{5.370884in}{0.629542in}}%
\pgfpathcurveto{\pgfqpoint{5.381780in}{0.629542in}}{\pgfqpoint{5.392230in}{0.633871in}}{\pgfqpoint{5.399935in}{0.641576in}}%
\pgfpathcurveto{\pgfqpoint{5.407639in}{0.649280in}}{\pgfqpoint{5.411968in}{0.659731in}}{\pgfqpoint{5.411968in}{0.670626in}}%
\pgfpathcurveto{\pgfqpoint{5.411968in}{0.681522in}}{\pgfqpoint{5.407639in}{0.691973in}}{\pgfqpoint{5.399935in}{0.699677in}}%
\pgfpathcurveto{\pgfqpoint{5.392230in}{0.707381in}}{\pgfqpoint{5.381780in}{0.711710in}}{\pgfqpoint{5.370884in}{0.711710in}}%
\pgfpathcurveto{\pgfqpoint{5.359988in}{0.711710in}}{\pgfqpoint{5.349538in}{0.707381in}}{\pgfqpoint{5.341833in}{0.699677in}}%
\pgfpathcurveto{\pgfqpoint{5.334129in}{0.691973in}}{\pgfqpoint{5.329800in}{0.681522in}}{\pgfqpoint{5.329800in}{0.670626in}}%
\pgfpathcurveto{\pgfqpoint{5.329800in}{0.659731in}}{\pgfqpoint{5.334129in}{0.649280in}}{\pgfqpoint{5.341833in}{0.641576in}}%
\pgfpathcurveto{\pgfqpoint{5.349538in}{0.633871in}}{\pgfqpoint{5.359988in}{0.629542in}}{\pgfqpoint{5.370884in}{0.629542in}}%
\pgfusepath{stroke}%
\end{pgfscope}%
\begin{pgfscope}%
\pgfpathrectangle{\pgfqpoint{0.688192in}{0.670138in}}{\pgfqpoint{7.111808in}{5.061530in}}%
\pgfusepath{clip}%
\pgfsetbuttcap%
\pgfsetroundjoin%
\pgfsetlinewidth{1.003750pt}%
\definecolor{currentstroke}{rgb}{0.000000,0.000000,0.000000}%
\pgfsetstrokecolor{currentstroke}%
\pgfsetdash{}{0pt}%
\pgfpathmoveto{\pgfqpoint{5.946216in}{2.685863in}}%
\pgfpathcurveto{\pgfqpoint{5.957111in}{2.685863in}}{\pgfqpoint{5.967562in}{2.690192in}}{\pgfqpoint{5.975266in}{2.697896in}}%
\pgfpathcurveto{\pgfqpoint{5.982971in}{2.705601in}}{\pgfqpoint{5.987299in}{2.716051in}}{\pgfqpoint{5.987299in}{2.726947in}}%
\pgfpathcurveto{\pgfqpoint{5.987299in}{2.737843in}}{\pgfqpoint{5.982971in}{2.748293in}}{\pgfqpoint{5.975266in}{2.755998in}}%
\pgfpathcurveto{\pgfqpoint{5.967562in}{2.763702in}}{\pgfqpoint{5.957111in}{2.768031in}}{\pgfqpoint{5.946216in}{2.768031in}}%
\pgfpathcurveto{\pgfqpoint{5.935320in}{2.768031in}}{\pgfqpoint{5.924869in}{2.763702in}}{\pgfqpoint{5.917165in}{2.755998in}}%
\pgfpathcurveto{\pgfqpoint{5.909461in}{2.748293in}}{\pgfqpoint{5.905132in}{2.737843in}}{\pgfqpoint{5.905132in}{2.726947in}}%
\pgfpathcurveto{\pgfqpoint{5.905132in}{2.716051in}}{\pgfqpoint{5.909461in}{2.705601in}}{\pgfqpoint{5.917165in}{2.697896in}}%
\pgfpathcurveto{\pgfqpoint{5.924869in}{2.690192in}}{\pgfqpoint{5.935320in}{2.685863in}}{\pgfqpoint{5.946216in}{2.685863in}}%
\pgfpathlineto{\pgfqpoint{5.946216in}{2.685863in}}%
\pgfpathclose%
\pgfusepath{stroke}%
\end{pgfscope}%
\begin{pgfscope}%
\pgfpathrectangle{\pgfqpoint{0.688192in}{0.670138in}}{\pgfqpoint{7.111808in}{5.061530in}}%
\pgfusepath{clip}%
\pgfsetbuttcap%
\pgfsetroundjoin%
\pgfsetlinewidth{1.003750pt}%
\definecolor{currentstroke}{rgb}{0.000000,0.000000,0.000000}%
\pgfsetstrokecolor{currentstroke}%
\pgfsetdash{}{0pt}%
\pgfpathmoveto{\pgfqpoint{0.781884in}{0.847588in}}%
\pgfpathcurveto{\pgfqpoint{0.792780in}{0.847588in}}{\pgfqpoint{0.803230in}{0.851917in}}{\pgfqpoint{0.810935in}{0.859621in}}%
\pgfpathcurveto{\pgfqpoint{0.818639in}{0.867326in}}{\pgfqpoint{0.822968in}{0.877776in}}{\pgfqpoint{0.822968in}{0.888672in}}%
\pgfpathcurveto{\pgfqpoint{0.822968in}{0.899568in}}{\pgfqpoint{0.818639in}{0.910018in}}{\pgfqpoint{0.810935in}{0.917723in}}%
\pgfpathcurveto{\pgfqpoint{0.803230in}{0.925427in}}{\pgfqpoint{0.792780in}{0.929756in}}{\pgfqpoint{0.781884in}{0.929756in}}%
\pgfpathcurveto{\pgfqpoint{0.770989in}{0.929756in}}{\pgfqpoint{0.760538in}{0.925427in}}{\pgfqpoint{0.752833in}{0.917723in}}%
\pgfpathcurveto{\pgfqpoint{0.745129in}{0.910018in}}{\pgfqpoint{0.740800in}{0.899568in}}{\pgfqpoint{0.740800in}{0.888672in}}%
\pgfpathcurveto{\pgfqpoint{0.740800in}{0.877776in}}{\pgfqpoint{0.745129in}{0.867326in}}{\pgfqpoint{0.752833in}{0.859621in}}%
\pgfpathcurveto{\pgfqpoint{0.760538in}{0.851917in}}{\pgfqpoint{0.770989in}{0.847588in}}{\pgfqpoint{0.781884in}{0.847588in}}%
\pgfpathlineto{\pgfqpoint{0.781884in}{0.847588in}}%
\pgfpathclose%
\pgfusepath{stroke}%
\end{pgfscope}%
\begin{pgfscope}%
\pgfpathrectangle{\pgfqpoint{0.688192in}{0.670138in}}{\pgfqpoint{7.111808in}{5.061530in}}%
\pgfusepath{clip}%
\pgfsetbuttcap%
\pgfsetroundjoin%
\pgfsetlinewidth{1.003750pt}%
\definecolor{currentstroke}{rgb}{0.000000,0.000000,0.000000}%
\pgfsetstrokecolor{currentstroke}%
\pgfsetdash{}{0pt}%
\pgfpathmoveto{\pgfqpoint{4.773403in}{3.650217in}}%
\pgfpathcurveto{\pgfqpoint{4.784299in}{3.650217in}}{\pgfqpoint{4.794749in}{3.654546in}}{\pgfqpoint{4.802454in}{3.662251in}}%
\pgfpathcurveto{\pgfqpoint{4.810158in}{3.669955in}}{\pgfqpoint{4.814487in}{3.680406in}}{\pgfqpoint{4.814487in}{3.691301in}}%
\pgfpathcurveto{\pgfqpoint{4.814487in}{3.702197in}}{\pgfqpoint{4.810158in}{3.712648in}}{\pgfqpoint{4.802454in}{3.720352in}}%
\pgfpathcurveto{\pgfqpoint{4.794749in}{3.728056in}}{\pgfqpoint{4.784299in}{3.732385in}}{\pgfqpoint{4.773403in}{3.732385in}}%
\pgfpathcurveto{\pgfqpoint{4.762508in}{3.732385in}}{\pgfqpoint{4.752057in}{3.728056in}}{\pgfqpoint{4.744352in}{3.720352in}}%
\pgfpathcurveto{\pgfqpoint{4.736648in}{3.712648in}}{\pgfqpoint{4.732319in}{3.702197in}}{\pgfqpoint{4.732319in}{3.691301in}}%
\pgfpathcurveto{\pgfqpoint{4.732319in}{3.680406in}}{\pgfqpoint{4.736648in}{3.669955in}}{\pgfqpoint{4.744352in}{3.662251in}}%
\pgfpathcurveto{\pgfqpoint{4.752057in}{3.654546in}}{\pgfqpoint{4.762508in}{3.650217in}}{\pgfqpoint{4.773403in}{3.650217in}}%
\pgfpathlineto{\pgfqpoint{4.773403in}{3.650217in}}%
\pgfpathclose%
\pgfusepath{stroke}%
\end{pgfscope}%
\begin{pgfscope}%
\pgfpathrectangle{\pgfqpoint{0.688192in}{0.670138in}}{\pgfqpoint{7.111808in}{5.061530in}}%
\pgfusepath{clip}%
\pgfsetbuttcap%
\pgfsetroundjoin%
\pgfsetlinewidth{1.003750pt}%
\definecolor{currentstroke}{rgb}{0.000000,0.000000,0.000000}%
\pgfsetstrokecolor{currentstroke}%
\pgfsetdash{}{0pt}%
\pgfpathmoveto{\pgfqpoint{1.970905in}{0.639806in}}%
\pgfpathcurveto{\pgfqpoint{1.981801in}{0.639806in}}{\pgfqpoint{1.992251in}{0.644135in}}{\pgfqpoint{1.999956in}{0.651839in}}%
\pgfpathcurveto{\pgfqpoint{2.007660in}{0.659543in}}{\pgfqpoint{2.011989in}{0.669994in}}{\pgfqpoint{2.011989in}{0.680890in}}%
\pgfpathcurveto{\pgfqpoint{2.011989in}{0.691785in}}{\pgfqpoint{2.007660in}{0.702236in}}{\pgfqpoint{1.999956in}{0.709941in}}%
\pgfpathcurveto{\pgfqpoint{1.992251in}{0.717645in}}{\pgfqpoint{1.981801in}{0.721974in}}{\pgfqpoint{1.970905in}{0.721974in}}%
\pgfpathcurveto{\pgfqpoint{1.960009in}{0.721974in}}{\pgfqpoint{1.949559in}{0.717645in}}{\pgfqpoint{1.941854in}{0.709941in}}%
\pgfpathcurveto{\pgfqpoint{1.934150in}{0.702236in}}{\pgfqpoint{1.929821in}{0.691785in}}{\pgfqpoint{1.929821in}{0.680890in}}%
\pgfpathcurveto{\pgfqpoint{1.929821in}{0.669994in}}{\pgfqpoint{1.934150in}{0.659543in}}{\pgfqpoint{1.941854in}{0.651839in}}%
\pgfpathcurveto{\pgfqpoint{1.949559in}{0.644135in}}{\pgfqpoint{1.960009in}{0.639806in}}{\pgfqpoint{1.970905in}{0.639806in}}%
\pgfusepath{stroke}%
\end{pgfscope}%
\begin{pgfscope}%
\pgfpathrectangle{\pgfqpoint{0.688192in}{0.670138in}}{\pgfqpoint{7.111808in}{5.061530in}}%
\pgfusepath{clip}%
\pgfsetbuttcap%
\pgfsetroundjoin%
\pgfsetlinewidth{1.003750pt}%
\definecolor{currentstroke}{rgb}{0.000000,0.000000,0.000000}%
\pgfsetstrokecolor{currentstroke}%
\pgfsetdash{}{0pt}%
\pgfpathmoveto{\pgfqpoint{1.102529in}{0.648268in}}%
\pgfpathcurveto{\pgfqpoint{1.113424in}{0.648268in}}{\pgfqpoint{1.123875in}{0.652597in}}{\pgfqpoint{1.131579in}{0.660301in}}%
\pgfpathcurveto{\pgfqpoint{1.139284in}{0.668005in}}{\pgfqpoint{1.143613in}{0.678456in}}{\pgfqpoint{1.143613in}{0.689352in}}%
\pgfpathcurveto{\pgfqpoint{1.143613in}{0.700247in}}{\pgfqpoint{1.139284in}{0.710698in}}{\pgfqpoint{1.131579in}{0.718402in}}%
\pgfpathcurveto{\pgfqpoint{1.123875in}{0.726107in}}{\pgfqpoint{1.113424in}{0.730436in}}{\pgfqpoint{1.102529in}{0.730436in}}%
\pgfpathcurveto{\pgfqpoint{1.091633in}{0.730436in}}{\pgfqpoint{1.081182in}{0.726107in}}{\pgfqpoint{1.073478in}{0.718402in}}%
\pgfpathcurveto{\pgfqpoint{1.065774in}{0.710698in}}{\pgfqpoint{1.061445in}{0.700247in}}{\pgfqpoint{1.061445in}{0.689352in}}%
\pgfpathcurveto{\pgfqpoint{1.061445in}{0.678456in}}{\pgfqpoint{1.065774in}{0.668005in}}{\pgfqpoint{1.073478in}{0.660301in}}%
\pgfpathcurveto{\pgfqpoint{1.081182in}{0.652597in}}{\pgfqpoint{1.091633in}{0.648268in}}{\pgfqpoint{1.102529in}{0.648268in}}%
\pgfusepath{stroke}%
\end{pgfscope}%
\begin{pgfscope}%
\pgfpathrectangle{\pgfqpoint{0.688192in}{0.670138in}}{\pgfqpoint{7.111808in}{5.061530in}}%
\pgfusepath{clip}%
\pgfsetbuttcap%
\pgfsetroundjoin%
\pgfsetlinewidth{1.003750pt}%
\definecolor{currentstroke}{rgb}{0.000000,0.000000,0.000000}%
\pgfsetstrokecolor{currentstroke}%
\pgfsetdash{}{0pt}%
\pgfpathmoveto{\pgfqpoint{1.469414in}{0.642878in}}%
\pgfpathcurveto{\pgfqpoint{1.480310in}{0.642878in}}{\pgfqpoint{1.490760in}{0.647207in}}{\pgfqpoint{1.498465in}{0.654911in}}%
\pgfpathcurveto{\pgfqpoint{1.506169in}{0.662615in}}{\pgfqpoint{1.510498in}{0.673066in}}{\pgfqpoint{1.510498in}{0.683962in}}%
\pgfpathcurveto{\pgfqpoint{1.510498in}{0.694857in}}{\pgfqpoint{1.506169in}{0.705308in}}{\pgfqpoint{1.498465in}{0.713012in}}%
\pgfpathcurveto{\pgfqpoint{1.490760in}{0.720717in}}{\pgfqpoint{1.480310in}{0.725046in}}{\pgfqpoint{1.469414in}{0.725046in}}%
\pgfpathcurveto{\pgfqpoint{1.458519in}{0.725046in}}{\pgfqpoint{1.448068in}{0.720717in}}{\pgfqpoint{1.440363in}{0.713012in}}%
\pgfpathcurveto{\pgfqpoint{1.432659in}{0.705308in}}{\pgfqpoint{1.428330in}{0.694857in}}{\pgfqpoint{1.428330in}{0.683962in}}%
\pgfpathcurveto{\pgfqpoint{1.428330in}{0.673066in}}{\pgfqpoint{1.432659in}{0.662615in}}{\pgfqpoint{1.440363in}{0.654911in}}%
\pgfpathcurveto{\pgfqpoint{1.448068in}{0.647207in}}{\pgfqpoint{1.458519in}{0.642878in}}{\pgfqpoint{1.469414in}{0.642878in}}%
\pgfusepath{stroke}%
\end{pgfscope}%
\begin{pgfscope}%
\pgfpathrectangle{\pgfqpoint{0.688192in}{0.670138in}}{\pgfqpoint{7.111808in}{5.061530in}}%
\pgfusepath{clip}%
\pgfsetbuttcap%
\pgfsetroundjoin%
\pgfsetlinewidth{1.003750pt}%
\definecolor{currentstroke}{rgb}{0.000000,0.000000,0.000000}%
\pgfsetstrokecolor{currentstroke}%
\pgfsetdash{}{0pt}%
\pgfpathmoveto{\pgfqpoint{4.576929in}{0.631060in}}%
\pgfpathcurveto{\pgfqpoint{4.587825in}{0.631060in}}{\pgfqpoint{4.598276in}{0.635389in}}{\pgfqpoint{4.605980in}{0.643094in}}%
\pgfpathcurveto{\pgfqpoint{4.613684in}{0.650798in}}{\pgfqpoint{4.618013in}{0.661249in}}{\pgfqpoint{4.618013in}{0.672144in}}%
\pgfpathcurveto{\pgfqpoint{4.618013in}{0.683040in}}{\pgfqpoint{4.613684in}{0.693491in}}{\pgfqpoint{4.605980in}{0.701195in}}%
\pgfpathcurveto{\pgfqpoint{4.598276in}{0.708899in}}{\pgfqpoint{4.587825in}{0.713228in}}{\pgfqpoint{4.576929in}{0.713228in}}%
\pgfpathcurveto{\pgfqpoint{4.566034in}{0.713228in}}{\pgfqpoint{4.555583in}{0.708899in}}{\pgfqpoint{4.547878in}{0.701195in}}%
\pgfpathcurveto{\pgfqpoint{4.540174in}{0.693491in}}{\pgfqpoint{4.535845in}{0.683040in}}{\pgfqpoint{4.535845in}{0.672144in}}%
\pgfpathcurveto{\pgfqpoint{4.535845in}{0.661249in}}{\pgfqpoint{4.540174in}{0.650798in}}{\pgfqpoint{4.547878in}{0.643094in}}%
\pgfpathcurveto{\pgfqpoint{4.555583in}{0.635389in}}{\pgfqpoint{4.566034in}{0.631060in}}{\pgfqpoint{4.576929in}{0.631060in}}%
\pgfusepath{stroke}%
\end{pgfscope}%
\begin{pgfscope}%
\pgfpathrectangle{\pgfqpoint{0.688192in}{0.670138in}}{\pgfqpoint{7.111808in}{5.061530in}}%
\pgfusepath{clip}%
\pgfsetbuttcap%
\pgfsetroundjoin%
\pgfsetlinewidth{1.003750pt}%
\definecolor{currentstroke}{rgb}{0.000000,0.000000,0.000000}%
\pgfsetstrokecolor{currentstroke}%
\pgfsetdash{}{0pt}%
\pgfpathmoveto{\pgfqpoint{0.958858in}{0.667930in}}%
\pgfpathcurveto{\pgfqpoint{0.969753in}{0.667930in}}{\pgfqpoint{0.980204in}{0.672259in}}{\pgfqpoint{0.987908in}{0.679963in}}%
\pgfpathcurveto{\pgfqpoint{0.995613in}{0.687667in}}{\pgfqpoint{0.999941in}{0.698118in}}{\pgfqpoint{0.999941in}{0.709014in}}%
\pgfpathcurveto{\pgfqpoint{0.999941in}{0.719909in}}{\pgfqpoint{0.995613in}{0.730360in}}{\pgfqpoint{0.987908in}{0.738065in}}%
\pgfpathcurveto{\pgfqpoint{0.980204in}{0.745769in}}{\pgfqpoint{0.969753in}{0.750098in}}{\pgfqpoint{0.958858in}{0.750098in}}%
\pgfpathcurveto{\pgfqpoint{0.947962in}{0.750098in}}{\pgfqpoint{0.937511in}{0.745769in}}{\pgfqpoint{0.929807in}{0.738065in}}%
\pgfpathcurveto{\pgfqpoint{0.922102in}{0.730360in}}{\pgfqpoint{0.917774in}{0.719909in}}{\pgfqpoint{0.917774in}{0.709014in}}%
\pgfpathcurveto{\pgfqpoint{0.917774in}{0.698118in}}{\pgfqpoint{0.922102in}{0.687667in}}{\pgfqpoint{0.929807in}{0.679963in}}%
\pgfpathcurveto{\pgfqpoint{0.937511in}{0.672259in}}{\pgfqpoint{0.947962in}{0.667930in}}{\pgfqpoint{0.958858in}{0.667930in}}%
\pgfpathlineto{\pgfqpoint{0.958858in}{0.667930in}}%
\pgfpathclose%
\pgfusepath{stroke}%
\end{pgfscope}%
\begin{pgfscope}%
\pgfpathrectangle{\pgfqpoint{0.688192in}{0.670138in}}{\pgfqpoint{7.111808in}{5.061530in}}%
\pgfusepath{clip}%
\pgfsetbuttcap%
\pgfsetroundjoin%
\pgfsetlinewidth{1.003750pt}%
\definecolor{currentstroke}{rgb}{0.000000,0.000000,0.000000}%
\pgfsetstrokecolor{currentstroke}%
\pgfsetdash{}{0pt}%
\pgfpathmoveto{\pgfqpoint{6.176676in}{1.369910in}}%
\pgfpathcurveto{\pgfqpoint{6.187572in}{1.369910in}}{\pgfqpoint{6.198022in}{1.374239in}}{\pgfqpoint{6.205727in}{1.381943in}}%
\pgfpathcurveto{\pgfqpoint{6.213431in}{1.389648in}}{\pgfqpoint{6.217760in}{1.400099in}}{\pgfqpoint{6.217760in}{1.410994in}}%
\pgfpathcurveto{\pgfqpoint{6.217760in}{1.421890in}}{\pgfqpoint{6.213431in}{1.432340in}}{\pgfqpoint{6.205727in}{1.440045in}}%
\pgfpathcurveto{\pgfqpoint{6.198022in}{1.447749in}}{\pgfqpoint{6.187572in}{1.452078in}}{\pgfqpoint{6.176676in}{1.452078in}}%
\pgfpathcurveto{\pgfqpoint{6.165781in}{1.452078in}}{\pgfqpoint{6.155330in}{1.447749in}}{\pgfqpoint{6.147625in}{1.440045in}}%
\pgfpathcurveto{\pgfqpoint{6.139921in}{1.432340in}}{\pgfqpoint{6.135592in}{1.421890in}}{\pgfqpoint{6.135592in}{1.410994in}}%
\pgfpathcurveto{\pgfqpoint{6.135592in}{1.400099in}}{\pgfqpoint{6.139921in}{1.389648in}}{\pgfqpoint{6.147625in}{1.381943in}}%
\pgfpathcurveto{\pgfqpoint{6.155330in}{1.374239in}}{\pgfqpoint{6.165781in}{1.369910in}}{\pgfqpoint{6.176676in}{1.369910in}}%
\pgfpathlineto{\pgfqpoint{6.176676in}{1.369910in}}%
\pgfpathclose%
\pgfusepath{stroke}%
\end{pgfscope}%
\begin{pgfscope}%
\pgfpathrectangle{\pgfqpoint{0.688192in}{0.670138in}}{\pgfqpoint{7.111808in}{5.061530in}}%
\pgfusepath{clip}%
\pgfsetbuttcap%
\pgfsetroundjoin%
\pgfsetlinewidth{1.003750pt}%
\definecolor{currentstroke}{rgb}{0.000000,0.000000,0.000000}%
\pgfsetstrokecolor{currentstroke}%
\pgfsetdash{}{0pt}%
\pgfpathmoveto{\pgfqpoint{2.439145in}{4.149268in}}%
\pgfpathcurveto{\pgfqpoint{2.450040in}{4.149268in}}{\pgfqpoint{2.460491in}{4.153597in}}{\pgfqpoint{2.468195in}{4.161301in}}%
\pgfpathcurveto{\pgfqpoint{2.475900in}{4.169006in}}{\pgfqpoint{2.480229in}{4.179456in}}{\pgfqpoint{2.480229in}{4.190352in}}%
\pgfpathcurveto{\pgfqpoint{2.480229in}{4.201247in}}{\pgfqpoint{2.475900in}{4.211698in}}{\pgfqpoint{2.468195in}{4.219403in}}%
\pgfpathcurveto{\pgfqpoint{2.460491in}{4.227107in}}{\pgfqpoint{2.450040in}{4.231436in}}{\pgfqpoint{2.439145in}{4.231436in}}%
\pgfpathcurveto{\pgfqpoint{2.428249in}{4.231436in}}{\pgfqpoint{2.417798in}{4.227107in}}{\pgfqpoint{2.410094in}{4.219403in}}%
\pgfpathcurveto{\pgfqpoint{2.402390in}{4.211698in}}{\pgfqpoint{2.398061in}{4.201247in}}{\pgfqpoint{2.398061in}{4.190352in}}%
\pgfpathcurveto{\pgfqpoint{2.398061in}{4.179456in}}{\pgfqpoint{2.402390in}{4.169006in}}{\pgfqpoint{2.410094in}{4.161301in}}%
\pgfpathcurveto{\pgfqpoint{2.417798in}{4.153597in}}{\pgfqpoint{2.428249in}{4.149268in}}{\pgfqpoint{2.439145in}{4.149268in}}%
\pgfpathlineto{\pgfqpoint{2.439145in}{4.149268in}}%
\pgfpathclose%
\pgfusepath{stroke}%
\end{pgfscope}%
\begin{pgfscope}%
\pgfpathrectangle{\pgfqpoint{0.688192in}{0.670138in}}{\pgfqpoint{7.111808in}{5.061530in}}%
\pgfusepath{clip}%
\pgfsetbuttcap%
\pgfsetroundjoin%
\pgfsetlinewidth{1.003750pt}%
\definecolor{currentstroke}{rgb}{0.000000,0.000000,0.000000}%
\pgfsetstrokecolor{currentstroke}%
\pgfsetdash{}{0pt}%
\pgfpathmoveto{\pgfqpoint{2.387383in}{3.235956in}}%
\pgfpathcurveto{\pgfqpoint{2.398278in}{3.235956in}}{\pgfqpoint{2.408729in}{3.240285in}}{\pgfqpoint{2.416434in}{3.247989in}}%
\pgfpathcurveto{\pgfqpoint{2.424138in}{3.255693in}}{\pgfqpoint{2.428467in}{3.266144in}}{\pgfqpoint{2.428467in}{3.277040in}}%
\pgfpathcurveto{\pgfqpoint{2.428467in}{3.287935in}}{\pgfqpoint{2.424138in}{3.298386in}}{\pgfqpoint{2.416434in}{3.306090in}}%
\pgfpathcurveto{\pgfqpoint{2.408729in}{3.313795in}}{\pgfqpoint{2.398278in}{3.318123in}}{\pgfqpoint{2.387383in}{3.318123in}}%
\pgfpathcurveto{\pgfqpoint{2.376487in}{3.318123in}}{\pgfqpoint{2.366037in}{3.313795in}}{\pgfqpoint{2.358332in}{3.306090in}}%
\pgfpathcurveto{\pgfqpoint{2.350628in}{3.298386in}}{\pgfqpoint{2.346299in}{3.287935in}}{\pgfqpoint{2.346299in}{3.277040in}}%
\pgfpathcurveto{\pgfqpoint{2.346299in}{3.266144in}}{\pgfqpoint{2.350628in}{3.255693in}}{\pgfqpoint{2.358332in}{3.247989in}}%
\pgfpathcurveto{\pgfqpoint{2.366037in}{3.240285in}}{\pgfqpoint{2.376487in}{3.235956in}}{\pgfqpoint{2.387383in}{3.235956in}}%
\pgfpathlineto{\pgfqpoint{2.387383in}{3.235956in}}%
\pgfpathclose%
\pgfusepath{stroke}%
\end{pgfscope}%
\begin{pgfscope}%
\pgfpathrectangle{\pgfqpoint{0.688192in}{0.670138in}}{\pgfqpoint{7.111808in}{5.061530in}}%
\pgfusepath{clip}%
\pgfsetbuttcap%
\pgfsetroundjoin%
\pgfsetlinewidth{1.003750pt}%
\definecolor{currentstroke}{rgb}{0.000000,0.000000,0.000000}%
\pgfsetstrokecolor{currentstroke}%
\pgfsetdash{}{0pt}%
\pgfpathmoveto{\pgfqpoint{4.576929in}{0.631060in}}%
\pgfpathcurveto{\pgfqpoint{4.587825in}{0.631060in}}{\pgfqpoint{4.598276in}{0.635389in}}{\pgfqpoint{4.605980in}{0.643094in}}%
\pgfpathcurveto{\pgfqpoint{4.613684in}{0.650798in}}{\pgfqpoint{4.618013in}{0.661249in}}{\pgfqpoint{4.618013in}{0.672144in}}%
\pgfpathcurveto{\pgfqpoint{4.618013in}{0.683040in}}{\pgfqpoint{4.613684in}{0.693491in}}{\pgfqpoint{4.605980in}{0.701195in}}%
\pgfpathcurveto{\pgfqpoint{4.598276in}{0.708899in}}{\pgfqpoint{4.587825in}{0.713228in}}{\pgfqpoint{4.576929in}{0.713228in}}%
\pgfpathcurveto{\pgfqpoint{4.566034in}{0.713228in}}{\pgfqpoint{4.555583in}{0.708899in}}{\pgfqpoint{4.547878in}{0.701195in}}%
\pgfpathcurveto{\pgfqpoint{4.540174in}{0.693491in}}{\pgfqpoint{4.535845in}{0.683040in}}{\pgfqpoint{4.535845in}{0.672144in}}%
\pgfpathcurveto{\pgfqpoint{4.535845in}{0.661249in}}{\pgfqpoint{4.540174in}{0.650798in}}{\pgfqpoint{4.547878in}{0.643094in}}%
\pgfpathcurveto{\pgfqpoint{4.555583in}{0.635389in}}{\pgfqpoint{4.566034in}{0.631060in}}{\pgfqpoint{4.576929in}{0.631060in}}%
\pgfusepath{stroke}%
\end{pgfscope}%
\begin{pgfscope}%
\pgfpathrectangle{\pgfqpoint{0.688192in}{0.670138in}}{\pgfqpoint{7.111808in}{5.061530in}}%
\pgfusepath{clip}%
\pgfsetbuttcap%
\pgfsetroundjoin%
\pgfsetlinewidth{1.003750pt}%
\definecolor{currentstroke}{rgb}{0.000000,0.000000,0.000000}%
\pgfsetstrokecolor{currentstroke}%
\pgfsetdash{}{0pt}%
\pgfpathmoveto{\pgfqpoint{7.309293in}{2.583093in}}%
\pgfpathcurveto{\pgfqpoint{7.320189in}{2.583093in}}{\pgfqpoint{7.330639in}{2.587422in}}{\pgfqpoint{7.338344in}{2.595127in}}%
\pgfpathcurveto{\pgfqpoint{7.346048in}{2.602831in}}{\pgfqpoint{7.350377in}{2.613282in}}{\pgfqpoint{7.350377in}{2.624177in}}%
\pgfpathcurveto{\pgfqpoint{7.350377in}{2.635073in}}{\pgfqpoint{7.346048in}{2.645524in}}{\pgfqpoint{7.338344in}{2.653228in}}%
\pgfpathcurveto{\pgfqpoint{7.330639in}{2.660932in}}{\pgfqpoint{7.320189in}{2.665261in}}{\pgfqpoint{7.309293in}{2.665261in}}%
\pgfpathcurveto{\pgfqpoint{7.298397in}{2.665261in}}{\pgfqpoint{7.287947in}{2.660932in}}{\pgfqpoint{7.280242in}{2.653228in}}%
\pgfpathcurveto{\pgfqpoint{7.272538in}{2.645524in}}{\pgfqpoint{7.268209in}{2.635073in}}{\pgfqpoint{7.268209in}{2.624177in}}%
\pgfpathcurveto{\pgfqpoint{7.268209in}{2.613282in}}{\pgfqpoint{7.272538in}{2.602831in}}{\pgfqpoint{7.280242in}{2.595127in}}%
\pgfpathcurveto{\pgfqpoint{7.287947in}{2.587422in}}{\pgfqpoint{7.298397in}{2.583093in}}{\pgfqpoint{7.309293in}{2.583093in}}%
\pgfpathlineto{\pgfqpoint{7.309293in}{2.583093in}}%
\pgfpathclose%
\pgfusepath{stroke}%
\end{pgfscope}%
\begin{pgfscope}%
\pgfpathrectangle{\pgfqpoint{0.688192in}{0.670138in}}{\pgfqpoint{7.111808in}{5.061530in}}%
\pgfusepath{clip}%
\pgfsetbuttcap%
\pgfsetroundjoin%
\pgfsetlinewidth{1.003750pt}%
\definecolor{currentstroke}{rgb}{0.000000,0.000000,0.000000}%
\pgfsetstrokecolor{currentstroke}%
\pgfsetdash{}{0pt}%
\pgfpathmoveto{\pgfqpoint{0.940682in}{0.674925in}}%
\pgfpathcurveto{\pgfqpoint{0.951577in}{0.674925in}}{\pgfqpoint{0.962028in}{0.679254in}}{\pgfqpoint{0.969732in}{0.686959in}}%
\pgfpathcurveto{\pgfqpoint{0.977437in}{0.694663in}}{\pgfqpoint{0.981766in}{0.705114in}}{\pgfqpoint{0.981766in}{0.716009in}}%
\pgfpathcurveto{\pgfqpoint{0.981766in}{0.726905in}}{\pgfqpoint{0.977437in}{0.737356in}}{\pgfqpoint{0.969732in}{0.745060in}}%
\pgfpathcurveto{\pgfqpoint{0.962028in}{0.752764in}}{\pgfqpoint{0.951577in}{0.757093in}}{\pgfqpoint{0.940682in}{0.757093in}}%
\pgfpathcurveto{\pgfqpoint{0.929786in}{0.757093in}}{\pgfqpoint{0.919335in}{0.752764in}}{\pgfqpoint{0.911631in}{0.745060in}}%
\pgfpathcurveto{\pgfqpoint{0.903927in}{0.737356in}}{\pgfqpoint{0.899598in}{0.726905in}}{\pgfqpoint{0.899598in}{0.716009in}}%
\pgfpathcurveto{\pgfqpoint{0.899598in}{0.705114in}}{\pgfqpoint{0.903927in}{0.694663in}}{\pgfqpoint{0.911631in}{0.686959in}}%
\pgfpathcurveto{\pgfqpoint{0.919335in}{0.679254in}}{\pgfqpoint{0.929786in}{0.674925in}}{\pgfqpoint{0.940682in}{0.674925in}}%
\pgfpathlineto{\pgfqpoint{0.940682in}{0.674925in}}%
\pgfpathclose%
\pgfusepath{stroke}%
\end{pgfscope}%
\begin{pgfscope}%
\pgfpathrectangle{\pgfqpoint{0.688192in}{0.670138in}}{\pgfqpoint{7.111808in}{5.061530in}}%
\pgfusepath{clip}%
\pgfsetbuttcap%
\pgfsetroundjoin%
\pgfsetlinewidth{1.003750pt}%
\definecolor{currentstroke}{rgb}{0.000000,0.000000,0.000000}%
\pgfsetstrokecolor{currentstroke}%
\pgfsetdash{}{0pt}%
\pgfpathmoveto{\pgfqpoint{5.135185in}{0.630032in}}%
\pgfpathcurveto{\pgfqpoint{5.146081in}{0.630032in}}{\pgfqpoint{5.156532in}{0.634361in}}{\pgfqpoint{5.164236in}{0.642065in}}%
\pgfpathcurveto{\pgfqpoint{5.171940in}{0.649770in}}{\pgfqpoint{5.176269in}{0.660220in}}{\pgfqpoint{5.176269in}{0.671116in}}%
\pgfpathcurveto{\pgfqpoint{5.176269in}{0.682011in}}{\pgfqpoint{5.171940in}{0.692462in}}{\pgfqpoint{5.164236in}{0.700167in}}%
\pgfpathcurveto{\pgfqpoint{5.156532in}{0.707871in}}{\pgfqpoint{5.146081in}{0.712200in}}{\pgfqpoint{5.135185in}{0.712200in}}%
\pgfpathcurveto{\pgfqpoint{5.124290in}{0.712200in}}{\pgfqpoint{5.113839in}{0.707871in}}{\pgfqpoint{5.106135in}{0.700167in}}%
\pgfpathcurveto{\pgfqpoint{5.098430in}{0.692462in}}{\pgfqpoint{5.094101in}{0.682011in}}{\pgfqpoint{5.094101in}{0.671116in}}%
\pgfpathcurveto{\pgfqpoint{5.094101in}{0.660220in}}{\pgfqpoint{5.098430in}{0.649770in}}{\pgfqpoint{5.106135in}{0.642065in}}%
\pgfpathcurveto{\pgfqpoint{5.113839in}{0.634361in}}{\pgfqpoint{5.124290in}{0.630032in}}{\pgfqpoint{5.135185in}{0.630032in}}%
\pgfusepath{stroke}%
\end{pgfscope}%
\begin{pgfscope}%
\pgfpathrectangle{\pgfqpoint{0.688192in}{0.670138in}}{\pgfqpoint{7.111808in}{5.061530in}}%
\pgfusepath{clip}%
\pgfsetbuttcap%
\pgfsetroundjoin%
\pgfsetlinewidth{1.003750pt}%
\definecolor{currentstroke}{rgb}{0.000000,0.000000,0.000000}%
\pgfsetstrokecolor{currentstroke}%
\pgfsetdash{}{0pt}%
\pgfpathmoveto{\pgfqpoint{1.178009in}{0.644363in}}%
\pgfpathcurveto{\pgfqpoint{1.188905in}{0.644363in}}{\pgfqpoint{1.199355in}{0.648692in}}{\pgfqpoint{1.207060in}{0.656396in}}%
\pgfpathcurveto{\pgfqpoint{1.214764in}{0.664100in}}{\pgfqpoint{1.219093in}{0.674551in}}{\pgfqpoint{1.219093in}{0.685447in}}%
\pgfpathcurveto{\pgfqpoint{1.219093in}{0.696342in}}{\pgfqpoint{1.214764in}{0.706793in}}{\pgfqpoint{1.207060in}{0.714497in}}%
\pgfpathcurveto{\pgfqpoint{1.199355in}{0.722202in}}{\pgfqpoint{1.188905in}{0.726531in}}{\pgfqpoint{1.178009in}{0.726531in}}%
\pgfpathcurveto{\pgfqpoint{1.167113in}{0.726531in}}{\pgfqpoint{1.156663in}{0.722202in}}{\pgfqpoint{1.148958in}{0.714497in}}%
\pgfpathcurveto{\pgfqpoint{1.141254in}{0.706793in}}{\pgfqpoint{1.136925in}{0.696342in}}{\pgfqpoint{1.136925in}{0.685447in}}%
\pgfpathcurveto{\pgfqpoint{1.136925in}{0.674551in}}{\pgfqpoint{1.141254in}{0.664100in}}{\pgfqpoint{1.148958in}{0.656396in}}%
\pgfpathcurveto{\pgfqpoint{1.156663in}{0.648692in}}{\pgfqpoint{1.167113in}{0.644363in}}{\pgfqpoint{1.178009in}{0.644363in}}%
\pgfusepath{stroke}%
\end{pgfscope}%
\begin{pgfscope}%
\pgfpathrectangle{\pgfqpoint{0.688192in}{0.670138in}}{\pgfqpoint{7.111808in}{5.061530in}}%
\pgfusepath{clip}%
\pgfsetbuttcap%
\pgfsetroundjoin%
\pgfsetlinewidth{1.003750pt}%
\definecolor{currentstroke}{rgb}{0.000000,0.000000,0.000000}%
\pgfsetstrokecolor{currentstroke}%
\pgfsetdash{}{0pt}%
\pgfpathmoveto{\pgfqpoint{1.755173in}{0.641177in}}%
\pgfpathcurveto{\pgfqpoint{1.766069in}{0.641177in}}{\pgfqpoint{1.776519in}{0.645505in}}{\pgfqpoint{1.784224in}{0.653210in}}%
\pgfpathcurveto{\pgfqpoint{1.791928in}{0.660914in}}{\pgfqpoint{1.796257in}{0.671365in}}{\pgfqpoint{1.796257in}{0.682260in}}%
\pgfpathcurveto{\pgfqpoint{1.796257in}{0.693156in}}{\pgfqpoint{1.791928in}{0.703607in}}{\pgfqpoint{1.784224in}{0.711311in}}%
\pgfpathcurveto{\pgfqpoint{1.776519in}{0.719016in}}{\pgfqpoint{1.766069in}{0.723344in}}{\pgfqpoint{1.755173in}{0.723344in}}%
\pgfpathcurveto{\pgfqpoint{1.744277in}{0.723344in}}{\pgfqpoint{1.733827in}{0.719016in}}{\pgfqpoint{1.726122in}{0.711311in}}%
\pgfpathcurveto{\pgfqpoint{1.718418in}{0.703607in}}{\pgfqpoint{1.714089in}{0.693156in}}{\pgfqpoint{1.714089in}{0.682260in}}%
\pgfpathcurveto{\pgfqpoint{1.714089in}{0.671365in}}{\pgfqpoint{1.718418in}{0.660914in}}{\pgfqpoint{1.726122in}{0.653210in}}%
\pgfpathcurveto{\pgfqpoint{1.733827in}{0.645505in}}{\pgfqpoint{1.744277in}{0.641177in}}{\pgfqpoint{1.755173in}{0.641177in}}%
\pgfusepath{stroke}%
\end{pgfscope}%
\begin{pgfscope}%
\pgfpathrectangle{\pgfqpoint{0.688192in}{0.670138in}}{\pgfqpoint{7.111808in}{5.061530in}}%
\pgfusepath{clip}%
\pgfsetbuttcap%
\pgfsetroundjoin%
\pgfsetlinewidth{1.003750pt}%
\definecolor{currentstroke}{rgb}{0.000000,0.000000,0.000000}%
\pgfsetstrokecolor{currentstroke}%
\pgfsetdash{}{0pt}%
\pgfpathmoveto{\pgfqpoint{2.346757in}{1.590411in}}%
\pgfpathcurveto{\pgfqpoint{2.357653in}{1.590411in}}{\pgfqpoint{2.368104in}{1.594740in}}{\pgfqpoint{2.375808in}{1.602444in}}%
\pgfpathcurveto{\pgfqpoint{2.383512in}{1.610149in}}{\pgfqpoint{2.387841in}{1.620599in}}{\pgfqpoint{2.387841in}{1.631495in}}%
\pgfpathcurveto{\pgfqpoint{2.387841in}{1.642391in}}{\pgfqpoint{2.383512in}{1.652841in}}{\pgfqpoint{2.375808in}{1.660546in}}%
\pgfpathcurveto{\pgfqpoint{2.368104in}{1.668250in}}{\pgfqpoint{2.357653in}{1.672579in}}{\pgfqpoint{2.346757in}{1.672579in}}%
\pgfpathcurveto{\pgfqpoint{2.335862in}{1.672579in}}{\pgfqpoint{2.325411in}{1.668250in}}{\pgfqpoint{2.317707in}{1.660546in}}%
\pgfpathcurveto{\pgfqpoint{2.310002in}{1.652841in}}{\pgfqpoint{2.305674in}{1.642391in}}{\pgfqpoint{2.305674in}{1.631495in}}%
\pgfpathcurveto{\pgfqpoint{2.305674in}{1.620599in}}{\pgfqpoint{2.310002in}{1.610149in}}{\pgfqpoint{2.317707in}{1.602444in}}%
\pgfpathcurveto{\pgfqpoint{2.325411in}{1.594740in}}{\pgfqpoint{2.335862in}{1.590411in}}{\pgfqpoint{2.346757in}{1.590411in}}%
\pgfpathlineto{\pgfqpoint{2.346757in}{1.590411in}}%
\pgfpathclose%
\pgfusepath{stroke}%
\end{pgfscope}%
\begin{pgfscope}%
\pgfpathrectangle{\pgfqpoint{0.688192in}{0.670138in}}{\pgfqpoint{7.111808in}{5.061530in}}%
\pgfusepath{clip}%
\pgfsetbuttcap%
\pgfsetroundjoin%
\pgfsetlinewidth{1.003750pt}%
\definecolor{currentstroke}{rgb}{0.000000,0.000000,0.000000}%
\pgfsetstrokecolor{currentstroke}%
\pgfsetdash{}{0pt}%
\pgfpathmoveto{\pgfqpoint{5.370884in}{0.629542in}}%
\pgfpathcurveto{\pgfqpoint{5.381780in}{0.629542in}}{\pgfqpoint{5.392230in}{0.633871in}}{\pgfqpoint{5.399935in}{0.641576in}}%
\pgfpathcurveto{\pgfqpoint{5.407639in}{0.649280in}}{\pgfqpoint{5.411968in}{0.659731in}}{\pgfqpoint{5.411968in}{0.670626in}}%
\pgfpathcurveto{\pgfqpoint{5.411968in}{0.681522in}}{\pgfqpoint{5.407639in}{0.691973in}}{\pgfqpoint{5.399935in}{0.699677in}}%
\pgfpathcurveto{\pgfqpoint{5.392230in}{0.707381in}}{\pgfqpoint{5.381780in}{0.711710in}}{\pgfqpoint{5.370884in}{0.711710in}}%
\pgfpathcurveto{\pgfqpoint{5.359988in}{0.711710in}}{\pgfqpoint{5.349538in}{0.707381in}}{\pgfqpoint{5.341833in}{0.699677in}}%
\pgfpathcurveto{\pgfqpoint{5.334129in}{0.691973in}}{\pgfqpoint{5.329800in}{0.681522in}}{\pgfqpoint{5.329800in}{0.670626in}}%
\pgfpathcurveto{\pgfqpoint{5.329800in}{0.659731in}}{\pgfqpoint{5.334129in}{0.649280in}}{\pgfqpoint{5.341833in}{0.641576in}}%
\pgfpathcurveto{\pgfqpoint{5.349538in}{0.633871in}}{\pgfqpoint{5.359988in}{0.629542in}}{\pgfqpoint{5.370884in}{0.629542in}}%
\pgfusepath{stroke}%
\end{pgfscope}%
\begin{pgfscope}%
\pgfpathrectangle{\pgfqpoint{0.688192in}{0.670138in}}{\pgfqpoint{7.111808in}{5.061530in}}%
\pgfusepath{clip}%
\pgfsetbuttcap%
\pgfsetroundjoin%
\pgfsetlinewidth{1.003750pt}%
\definecolor{currentstroke}{rgb}{0.000000,0.000000,0.000000}%
\pgfsetstrokecolor{currentstroke}%
\pgfsetdash{}{0pt}%
\pgfpathmoveto{\pgfqpoint{2.429320in}{0.637441in}}%
\pgfpathcurveto{\pgfqpoint{2.440215in}{0.637441in}}{\pgfqpoint{2.450666in}{0.641770in}}{\pgfqpoint{2.458370in}{0.649474in}}%
\pgfpathcurveto{\pgfqpoint{2.466075in}{0.657178in}}{\pgfqpoint{2.470404in}{0.667629in}}{\pgfqpoint{2.470404in}{0.678525in}}%
\pgfpathcurveto{\pgfqpoint{2.470404in}{0.689420in}}{\pgfqpoint{2.466075in}{0.699871in}}{\pgfqpoint{2.458370in}{0.707575in}}%
\pgfpathcurveto{\pgfqpoint{2.450666in}{0.715280in}}{\pgfqpoint{2.440215in}{0.719609in}}{\pgfqpoint{2.429320in}{0.719609in}}%
\pgfpathcurveto{\pgfqpoint{2.418424in}{0.719609in}}{\pgfqpoint{2.407973in}{0.715280in}}{\pgfqpoint{2.400269in}{0.707575in}}%
\pgfpathcurveto{\pgfqpoint{2.392565in}{0.699871in}}{\pgfqpoint{2.388236in}{0.689420in}}{\pgfqpoint{2.388236in}{0.678525in}}%
\pgfpathcurveto{\pgfqpoint{2.388236in}{0.667629in}}{\pgfqpoint{2.392565in}{0.657178in}}{\pgfqpoint{2.400269in}{0.649474in}}%
\pgfpathcurveto{\pgfqpoint{2.407973in}{0.641770in}}{\pgfqpoint{2.418424in}{0.637441in}}{\pgfqpoint{2.429320in}{0.637441in}}%
\pgfusepath{stroke}%
\end{pgfscope}%
\begin{pgfscope}%
\pgfpathrectangle{\pgfqpoint{0.688192in}{0.670138in}}{\pgfqpoint{7.111808in}{5.061530in}}%
\pgfusepath{clip}%
\pgfsetbuttcap%
\pgfsetroundjoin%
\pgfsetlinewidth{1.003750pt}%
\definecolor{currentstroke}{rgb}{0.000000,0.000000,0.000000}%
\pgfsetstrokecolor{currentstroke}%
\pgfsetdash{}{0pt}%
\pgfpathmoveto{\pgfqpoint{1.495917in}{0.642463in}}%
\pgfpathcurveto{\pgfqpoint{1.506813in}{0.642463in}}{\pgfqpoint{1.517263in}{0.646792in}}{\pgfqpoint{1.524968in}{0.654497in}}%
\pgfpathcurveto{\pgfqpoint{1.532672in}{0.662201in}}{\pgfqpoint{1.537001in}{0.672652in}}{\pgfqpoint{1.537001in}{0.683547in}}%
\pgfpathcurveto{\pgfqpoint{1.537001in}{0.694443in}}{\pgfqpoint{1.532672in}{0.704894in}}{\pgfqpoint{1.524968in}{0.712598in}}%
\pgfpathcurveto{\pgfqpoint{1.517263in}{0.720302in}}{\pgfqpoint{1.506813in}{0.724631in}}{\pgfqpoint{1.495917in}{0.724631in}}%
\pgfpathcurveto{\pgfqpoint{1.485021in}{0.724631in}}{\pgfqpoint{1.474571in}{0.720302in}}{\pgfqpoint{1.466866in}{0.712598in}}%
\pgfpathcurveto{\pgfqpoint{1.459162in}{0.704894in}}{\pgfqpoint{1.454833in}{0.694443in}}{\pgfqpoint{1.454833in}{0.683547in}}%
\pgfpathcurveto{\pgfqpoint{1.454833in}{0.672652in}}{\pgfqpoint{1.459162in}{0.662201in}}{\pgfqpoint{1.466866in}{0.654497in}}%
\pgfpathcurveto{\pgfqpoint{1.474571in}{0.646792in}}{\pgfqpoint{1.485021in}{0.642463in}}{\pgfqpoint{1.495917in}{0.642463in}}%
\pgfusepath{stroke}%
\end{pgfscope}%
\begin{pgfscope}%
\pgfpathrectangle{\pgfqpoint{0.688192in}{0.670138in}}{\pgfqpoint{7.111808in}{5.061530in}}%
\pgfusepath{clip}%
\pgfsetbuttcap%
\pgfsetroundjoin%
\pgfsetlinewidth{1.003750pt}%
\definecolor{currentstroke}{rgb}{0.000000,0.000000,0.000000}%
\pgfsetstrokecolor{currentstroke}%
\pgfsetdash{}{0pt}%
\pgfpathmoveto{\pgfqpoint{6.005303in}{2.754155in}}%
\pgfpathcurveto{\pgfqpoint{6.016198in}{2.754155in}}{\pgfqpoint{6.026649in}{2.758484in}}{\pgfqpoint{6.034353in}{2.766188in}}%
\pgfpathcurveto{\pgfqpoint{6.042058in}{2.773893in}}{\pgfqpoint{6.046386in}{2.784344in}}{\pgfqpoint{6.046386in}{2.795239in}}%
\pgfpathcurveto{\pgfqpoint{6.046386in}{2.806135in}}{\pgfqpoint{6.042058in}{2.816586in}}{\pgfqpoint{6.034353in}{2.824290in}}%
\pgfpathcurveto{\pgfqpoint{6.026649in}{2.831994in}}{\pgfqpoint{6.016198in}{2.836323in}}{\pgfqpoint{6.005303in}{2.836323in}}%
\pgfpathcurveto{\pgfqpoint{5.994407in}{2.836323in}}{\pgfqpoint{5.983956in}{2.831994in}}{\pgfqpoint{5.976252in}{2.824290in}}%
\pgfpathcurveto{\pgfqpoint{5.968547in}{2.816586in}}{\pgfqpoint{5.964219in}{2.806135in}}{\pgfqpoint{5.964219in}{2.795239in}}%
\pgfpathcurveto{\pgfqpoint{5.964219in}{2.784344in}}{\pgfqpoint{5.968547in}{2.773893in}}{\pgfqpoint{5.976252in}{2.766188in}}%
\pgfpathcurveto{\pgfqpoint{5.983956in}{2.758484in}}{\pgfqpoint{5.994407in}{2.754155in}}{\pgfqpoint{6.005303in}{2.754155in}}%
\pgfpathlineto{\pgfqpoint{6.005303in}{2.754155in}}%
\pgfpathclose%
\pgfusepath{stroke}%
\end{pgfscope}%
\begin{pgfscope}%
\pgfpathrectangle{\pgfqpoint{0.688192in}{0.670138in}}{\pgfqpoint{7.111808in}{5.061530in}}%
\pgfusepath{clip}%
\pgfsetbuttcap%
\pgfsetroundjoin%
\pgfsetlinewidth{1.003750pt}%
\definecolor{currentstroke}{rgb}{0.000000,0.000000,0.000000}%
\pgfsetstrokecolor{currentstroke}%
\pgfsetdash{}{0pt}%
\pgfpathmoveto{\pgfqpoint{1.473926in}{0.642857in}}%
\pgfpathcurveto{\pgfqpoint{1.484821in}{0.642857in}}{\pgfqpoint{1.495272in}{0.647186in}}{\pgfqpoint{1.502976in}{0.654890in}}%
\pgfpathcurveto{\pgfqpoint{1.510681in}{0.662595in}}{\pgfqpoint{1.515009in}{0.673045in}}{\pgfqpoint{1.515009in}{0.683941in}}%
\pgfpathcurveto{\pgfqpoint{1.515009in}{0.694837in}}{\pgfqpoint{1.510681in}{0.705287in}}{\pgfqpoint{1.502976in}{0.712992in}}%
\pgfpathcurveto{\pgfqpoint{1.495272in}{0.720696in}}{\pgfqpoint{1.484821in}{0.725025in}}{\pgfqpoint{1.473926in}{0.725025in}}%
\pgfpathcurveto{\pgfqpoint{1.463030in}{0.725025in}}{\pgfqpoint{1.452579in}{0.720696in}}{\pgfqpoint{1.444875in}{0.712992in}}%
\pgfpathcurveto{\pgfqpoint{1.437171in}{0.705287in}}{\pgfqpoint{1.432842in}{0.694837in}}{\pgfqpoint{1.432842in}{0.683941in}}%
\pgfpathcurveto{\pgfqpoint{1.432842in}{0.673045in}}{\pgfqpoint{1.437171in}{0.662595in}}{\pgfqpoint{1.444875in}{0.654890in}}%
\pgfpathcurveto{\pgfqpoint{1.452579in}{0.647186in}}{\pgfqpoint{1.463030in}{0.642857in}}{\pgfqpoint{1.473926in}{0.642857in}}%
\pgfusepath{stroke}%
\end{pgfscope}%
\begin{pgfscope}%
\pgfpathrectangle{\pgfqpoint{0.688192in}{0.670138in}}{\pgfqpoint{7.111808in}{5.061530in}}%
\pgfusepath{clip}%
\pgfsetbuttcap%
\pgfsetroundjoin%
\pgfsetlinewidth{1.003750pt}%
\definecolor{currentstroke}{rgb}{0.000000,0.000000,0.000000}%
\pgfsetstrokecolor{currentstroke}%
\pgfsetdash{}{0pt}%
\pgfpathmoveto{\pgfqpoint{2.768315in}{3.990841in}}%
\pgfpathcurveto{\pgfqpoint{2.779211in}{3.990841in}}{\pgfqpoint{2.789661in}{3.995170in}}{\pgfqpoint{2.797366in}{4.002874in}}%
\pgfpathcurveto{\pgfqpoint{2.805070in}{4.010579in}}{\pgfqpoint{2.809399in}{4.021029in}}{\pgfqpoint{2.809399in}{4.031925in}}%
\pgfpathcurveto{\pgfqpoint{2.809399in}{4.042821in}}{\pgfqpoint{2.805070in}{4.053271in}}{\pgfqpoint{2.797366in}{4.060976in}}%
\pgfpathcurveto{\pgfqpoint{2.789661in}{4.068680in}}{\pgfqpoint{2.779211in}{4.073009in}}{\pgfqpoint{2.768315in}{4.073009in}}%
\pgfpathcurveto{\pgfqpoint{2.757419in}{4.073009in}}{\pgfqpoint{2.746969in}{4.068680in}}{\pgfqpoint{2.739264in}{4.060976in}}%
\pgfpathcurveto{\pgfqpoint{2.731560in}{4.053271in}}{\pgfqpoint{2.727231in}{4.042821in}}{\pgfqpoint{2.727231in}{4.031925in}}%
\pgfpathcurveto{\pgfqpoint{2.727231in}{4.021029in}}{\pgfqpoint{2.731560in}{4.010579in}}{\pgfqpoint{2.739264in}{4.002874in}}%
\pgfpathcurveto{\pgfqpoint{2.746969in}{3.995170in}}{\pgfqpoint{2.757419in}{3.990841in}}{\pgfqpoint{2.768315in}{3.990841in}}%
\pgfpathlineto{\pgfqpoint{2.768315in}{3.990841in}}%
\pgfpathclose%
\pgfusepath{stroke}%
\end{pgfscope}%
\begin{pgfscope}%
\pgfpathrectangle{\pgfqpoint{0.688192in}{0.670138in}}{\pgfqpoint{7.111808in}{5.061530in}}%
\pgfusepath{clip}%
\pgfsetbuttcap%
\pgfsetroundjoin%
\pgfsetlinewidth{1.003750pt}%
\definecolor{currentstroke}{rgb}{0.000000,0.000000,0.000000}%
\pgfsetstrokecolor{currentstroke}%
\pgfsetdash{}{0pt}%
\pgfpathmoveto{\pgfqpoint{2.510692in}{3.115711in}}%
\pgfpathcurveto{\pgfqpoint{2.521588in}{3.115711in}}{\pgfqpoint{2.532038in}{3.120040in}}{\pgfqpoint{2.539743in}{3.127744in}}%
\pgfpathcurveto{\pgfqpoint{2.547447in}{3.135448in}}{\pgfqpoint{2.551776in}{3.145899in}}{\pgfqpoint{2.551776in}{3.156795in}}%
\pgfpathcurveto{\pgfqpoint{2.551776in}{3.167690in}}{\pgfqpoint{2.547447in}{3.178141in}}{\pgfqpoint{2.539743in}{3.185845in}}%
\pgfpathcurveto{\pgfqpoint{2.532038in}{3.193550in}}{\pgfqpoint{2.521588in}{3.197878in}}{\pgfqpoint{2.510692in}{3.197878in}}%
\pgfpathcurveto{\pgfqpoint{2.499796in}{3.197878in}}{\pgfqpoint{2.489346in}{3.193550in}}{\pgfqpoint{2.481641in}{3.185845in}}%
\pgfpathcurveto{\pgfqpoint{2.473937in}{3.178141in}}{\pgfqpoint{2.469608in}{3.167690in}}{\pgfqpoint{2.469608in}{3.156795in}}%
\pgfpathcurveto{\pgfqpoint{2.469608in}{3.145899in}}{\pgfqpoint{2.473937in}{3.135448in}}{\pgfqpoint{2.481641in}{3.127744in}}%
\pgfpathcurveto{\pgfqpoint{2.489346in}{3.120040in}}{\pgfqpoint{2.499796in}{3.115711in}}{\pgfqpoint{2.510692in}{3.115711in}}%
\pgfpathlineto{\pgfqpoint{2.510692in}{3.115711in}}%
\pgfpathclose%
\pgfusepath{stroke}%
\end{pgfscope}%
\begin{pgfscope}%
\pgfpathrectangle{\pgfqpoint{0.688192in}{0.670138in}}{\pgfqpoint{7.111808in}{5.061530in}}%
\pgfusepath{clip}%
\pgfsetbuttcap%
\pgfsetroundjoin%
\pgfsetlinewidth{1.003750pt}%
\definecolor{currentstroke}{rgb}{0.000000,0.000000,0.000000}%
\pgfsetstrokecolor{currentstroke}%
\pgfsetdash{}{0pt}%
\pgfpathmoveto{\pgfqpoint{1.495917in}{0.642463in}}%
\pgfpathcurveto{\pgfqpoint{1.506813in}{0.642463in}}{\pgfqpoint{1.517263in}{0.646792in}}{\pgfqpoint{1.524968in}{0.654497in}}%
\pgfpathcurveto{\pgfqpoint{1.532672in}{0.662201in}}{\pgfqpoint{1.537001in}{0.672652in}}{\pgfqpoint{1.537001in}{0.683547in}}%
\pgfpathcurveto{\pgfqpoint{1.537001in}{0.694443in}}{\pgfqpoint{1.532672in}{0.704894in}}{\pgfqpoint{1.524968in}{0.712598in}}%
\pgfpathcurveto{\pgfqpoint{1.517263in}{0.720302in}}{\pgfqpoint{1.506813in}{0.724631in}}{\pgfqpoint{1.495917in}{0.724631in}}%
\pgfpathcurveto{\pgfqpoint{1.485021in}{0.724631in}}{\pgfqpoint{1.474571in}{0.720302in}}{\pgfqpoint{1.466866in}{0.712598in}}%
\pgfpathcurveto{\pgfqpoint{1.459162in}{0.704894in}}{\pgfqpoint{1.454833in}{0.694443in}}{\pgfqpoint{1.454833in}{0.683547in}}%
\pgfpathcurveto{\pgfqpoint{1.454833in}{0.672652in}}{\pgfqpoint{1.459162in}{0.662201in}}{\pgfqpoint{1.466866in}{0.654497in}}%
\pgfpathcurveto{\pgfqpoint{1.474571in}{0.646792in}}{\pgfqpoint{1.485021in}{0.642463in}}{\pgfqpoint{1.495917in}{0.642463in}}%
\pgfusepath{stroke}%
\end{pgfscope}%
\begin{pgfscope}%
\pgfpathrectangle{\pgfqpoint{0.688192in}{0.670138in}}{\pgfqpoint{7.111808in}{5.061530in}}%
\pgfusepath{clip}%
\pgfsetbuttcap%
\pgfsetroundjoin%
\pgfsetlinewidth{1.003750pt}%
\definecolor{currentstroke}{rgb}{0.000000,0.000000,0.000000}%
\pgfsetstrokecolor{currentstroke}%
\pgfsetdash{}{0pt}%
\pgfpathmoveto{\pgfqpoint{1.178009in}{0.644363in}}%
\pgfpathcurveto{\pgfqpoint{1.188905in}{0.644363in}}{\pgfqpoint{1.199355in}{0.648692in}}{\pgfqpoint{1.207060in}{0.656396in}}%
\pgfpathcurveto{\pgfqpoint{1.214764in}{0.664100in}}{\pgfqpoint{1.219093in}{0.674551in}}{\pgfqpoint{1.219093in}{0.685447in}}%
\pgfpathcurveto{\pgfqpoint{1.219093in}{0.696342in}}{\pgfqpoint{1.214764in}{0.706793in}}{\pgfqpoint{1.207060in}{0.714497in}}%
\pgfpathcurveto{\pgfqpoint{1.199355in}{0.722202in}}{\pgfqpoint{1.188905in}{0.726531in}}{\pgfqpoint{1.178009in}{0.726531in}}%
\pgfpathcurveto{\pgfqpoint{1.167113in}{0.726531in}}{\pgfqpoint{1.156663in}{0.722202in}}{\pgfqpoint{1.148958in}{0.714497in}}%
\pgfpathcurveto{\pgfqpoint{1.141254in}{0.706793in}}{\pgfqpoint{1.136925in}{0.696342in}}{\pgfqpoint{1.136925in}{0.685447in}}%
\pgfpathcurveto{\pgfqpoint{1.136925in}{0.674551in}}{\pgfqpoint{1.141254in}{0.664100in}}{\pgfqpoint{1.148958in}{0.656396in}}%
\pgfpathcurveto{\pgfqpoint{1.156663in}{0.648692in}}{\pgfqpoint{1.167113in}{0.644363in}}{\pgfqpoint{1.178009in}{0.644363in}}%
\pgfusepath{stroke}%
\end{pgfscope}%
\begin{pgfscope}%
\pgfpathrectangle{\pgfqpoint{0.688192in}{0.670138in}}{\pgfqpoint{7.111808in}{5.061530in}}%
\pgfusepath{clip}%
\pgfsetbuttcap%
\pgfsetroundjoin%
\pgfsetlinewidth{1.003750pt}%
\definecolor{currentstroke}{rgb}{0.000000,0.000000,0.000000}%
\pgfsetstrokecolor{currentstroke}%
\pgfsetdash{}{0pt}%
\pgfpathmoveto{\pgfqpoint{1.672396in}{2.153838in}}%
\pgfpathcurveto{\pgfqpoint{1.683291in}{2.153838in}}{\pgfqpoint{1.693742in}{2.158167in}}{\pgfqpoint{1.701446in}{2.165871in}}%
\pgfpathcurveto{\pgfqpoint{1.709151in}{2.173575in}}{\pgfqpoint{1.713480in}{2.184026in}}{\pgfqpoint{1.713480in}{2.194922in}}%
\pgfpathcurveto{\pgfqpoint{1.713480in}{2.205817in}}{\pgfqpoint{1.709151in}{2.216268in}}{\pgfqpoint{1.701446in}{2.223972in}}%
\pgfpathcurveto{\pgfqpoint{1.693742in}{2.231677in}}{\pgfqpoint{1.683291in}{2.236005in}}{\pgfqpoint{1.672396in}{2.236005in}}%
\pgfpathcurveto{\pgfqpoint{1.661500in}{2.236005in}}{\pgfqpoint{1.651049in}{2.231677in}}{\pgfqpoint{1.643345in}{2.223972in}}%
\pgfpathcurveto{\pgfqpoint{1.635641in}{2.216268in}}{\pgfqpoint{1.631312in}{2.205817in}}{\pgfqpoint{1.631312in}{2.194922in}}%
\pgfpathcurveto{\pgfqpoint{1.631312in}{2.184026in}}{\pgfqpoint{1.635641in}{2.173575in}}{\pgfqpoint{1.643345in}{2.165871in}}%
\pgfpathcurveto{\pgfqpoint{1.651049in}{2.158167in}}{\pgfqpoint{1.661500in}{2.153838in}}{\pgfqpoint{1.672396in}{2.153838in}}%
\pgfpathlineto{\pgfqpoint{1.672396in}{2.153838in}}%
\pgfpathclose%
\pgfusepath{stroke}%
\end{pgfscope}%
\begin{pgfscope}%
\pgfpathrectangle{\pgfqpoint{0.688192in}{0.670138in}}{\pgfqpoint{7.111808in}{5.061530in}}%
\pgfusepath{clip}%
\pgfsetbuttcap%
\pgfsetroundjoin%
\pgfsetlinewidth{1.003750pt}%
\definecolor{currentstroke}{rgb}{0.000000,0.000000,0.000000}%
\pgfsetstrokecolor{currentstroke}%
\pgfsetdash{}{0pt}%
\pgfpathmoveto{\pgfqpoint{4.688548in}{0.630639in}}%
\pgfpathcurveto{\pgfqpoint{4.699443in}{0.630639in}}{\pgfqpoint{4.709894in}{0.634968in}}{\pgfqpoint{4.717598in}{0.642672in}}%
\pgfpathcurveto{\pgfqpoint{4.725303in}{0.650377in}}{\pgfqpoint{4.729631in}{0.660827in}}{\pgfqpoint{4.729631in}{0.671723in}}%
\pgfpathcurveto{\pgfqpoint{4.729631in}{0.682619in}}{\pgfqpoint{4.725303in}{0.693069in}}{\pgfqpoint{4.717598in}{0.700774in}}%
\pgfpathcurveto{\pgfqpoint{4.709894in}{0.708478in}}{\pgfqpoint{4.699443in}{0.712807in}}{\pgfqpoint{4.688548in}{0.712807in}}%
\pgfpathcurveto{\pgfqpoint{4.677652in}{0.712807in}}{\pgfqpoint{4.667201in}{0.708478in}}{\pgfqpoint{4.659497in}{0.700774in}}%
\pgfpathcurveto{\pgfqpoint{4.651792in}{0.693069in}}{\pgfqpoint{4.647464in}{0.682619in}}{\pgfqpoint{4.647464in}{0.671723in}}%
\pgfpathcurveto{\pgfqpoint{4.647464in}{0.660827in}}{\pgfqpoint{4.651792in}{0.650377in}}{\pgfqpoint{4.659497in}{0.642672in}}%
\pgfpathcurveto{\pgfqpoint{4.667201in}{0.634968in}}{\pgfqpoint{4.677652in}{0.630639in}}{\pgfqpoint{4.688548in}{0.630639in}}%
\pgfusepath{stroke}%
\end{pgfscope}%
\begin{pgfscope}%
\pgfpathrectangle{\pgfqpoint{0.688192in}{0.670138in}}{\pgfqpoint{7.111808in}{5.061530in}}%
\pgfusepath{clip}%
\pgfsetbuttcap%
\pgfsetroundjoin%
\pgfsetlinewidth{1.003750pt}%
\definecolor{currentstroke}{rgb}{0.000000,0.000000,0.000000}%
\pgfsetstrokecolor{currentstroke}%
\pgfsetdash{}{0pt}%
\pgfpathmoveto{\pgfqpoint{3.291859in}{0.633837in}}%
\pgfpathcurveto{\pgfqpoint{3.302755in}{0.633837in}}{\pgfqpoint{3.313206in}{0.638166in}}{\pgfqpoint{3.320910in}{0.645871in}}%
\pgfpathcurveto{\pgfqpoint{3.328614in}{0.653575in}}{\pgfqpoint{3.332943in}{0.664026in}}{\pgfqpoint{3.332943in}{0.674921in}}%
\pgfpathcurveto{\pgfqpoint{3.332943in}{0.685817in}}{\pgfqpoint{3.328614in}{0.696268in}}{\pgfqpoint{3.320910in}{0.703972in}}%
\pgfpathcurveto{\pgfqpoint{3.313206in}{0.711676in}}{\pgfqpoint{3.302755in}{0.716005in}}{\pgfqpoint{3.291859in}{0.716005in}}%
\pgfpathcurveto{\pgfqpoint{3.280964in}{0.716005in}}{\pgfqpoint{3.270513in}{0.711676in}}{\pgfqpoint{3.262809in}{0.703972in}}%
\pgfpathcurveto{\pgfqpoint{3.255104in}{0.696268in}}{\pgfqpoint{3.250775in}{0.685817in}}{\pgfqpoint{3.250775in}{0.674921in}}%
\pgfpathcurveto{\pgfqpoint{3.250775in}{0.664026in}}{\pgfqpoint{3.255104in}{0.653575in}}{\pgfqpoint{3.262809in}{0.645871in}}%
\pgfpathcurveto{\pgfqpoint{3.270513in}{0.638166in}}{\pgfqpoint{3.280964in}{0.633837in}}{\pgfqpoint{3.291859in}{0.633837in}}%
\pgfusepath{stroke}%
\end{pgfscope}%
\begin{pgfscope}%
\pgfpathrectangle{\pgfqpoint{0.688192in}{0.670138in}}{\pgfqpoint{7.111808in}{5.061530in}}%
\pgfusepath{clip}%
\pgfsetbuttcap%
\pgfsetroundjoin%
\pgfsetlinewidth{1.003750pt}%
\definecolor{currentstroke}{rgb}{0.000000,0.000000,0.000000}%
\pgfsetstrokecolor{currentstroke}%
\pgfsetdash{}{0pt}%
\pgfpathmoveto{\pgfqpoint{1.659921in}{0.642043in}}%
\pgfpathcurveto{\pgfqpoint{1.670816in}{0.642043in}}{\pgfqpoint{1.681267in}{0.646372in}}{\pgfqpoint{1.688971in}{0.654076in}}%
\pgfpathcurveto{\pgfqpoint{1.696676in}{0.661781in}}{\pgfqpoint{1.701004in}{0.672232in}}{\pgfqpoint{1.701004in}{0.683127in}}%
\pgfpathcurveto{\pgfqpoint{1.701004in}{0.694023in}}{\pgfqpoint{1.696676in}{0.704473in}}{\pgfqpoint{1.688971in}{0.712178in}}%
\pgfpathcurveto{\pgfqpoint{1.681267in}{0.719882in}}{\pgfqpoint{1.670816in}{0.724211in}}{\pgfqpoint{1.659921in}{0.724211in}}%
\pgfpathcurveto{\pgfqpoint{1.649025in}{0.724211in}}{\pgfqpoint{1.638574in}{0.719882in}}{\pgfqpoint{1.630870in}{0.712178in}}%
\pgfpathcurveto{\pgfqpoint{1.623166in}{0.704473in}}{\pgfqpoint{1.618837in}{0.694023in}}{\pgfqpoint{1.618837in}{0.683127in}}%
\pgfpathcurveto{\pgfqpoint{1.618837in}{0.672232in}}{\pgfqpoint{1.623166in}{0.661781in}}{\pgfqpoint{1.630870in}{0.654076in}}%
\pgfpathcurveto{\pgfqpoint{1.638574in}{0.646372in}}{\pgfqpoint{1.649025in}{0.642043in}}{\pgfqpoint{1.659921in}{0.642043in}}%
\pgfusepath{stroke}%
\end{pgfscope}%
\begin{pgfscope}%
\pgfpathrectangle{\pgfqpoint{0.688192in}{0.670138in}}{\pgfqpoint{7.111808in}{5.061530in}}%
\pgfusepath{clip}%
\pgfsetbuttcap%
\pgfsetroundjoin%
\pgfsetlinewidth{1.003750pt}%
\definecolor{currentstroke}{rgb}{0.000000,0.000000,0.000000}%
\pgfsetstrokecolor{currentstroke}%
\pgfsetdash{}{0pt}%
\pgfpathmoveto{\pgfqpoint{5.283066in}{0.629702in}}%
\pgfpathcurveto{\pgfqpoint{5.293961in}{0.629702in}}{\pgfqpoint{5.304412in}{0.634031in}}{\pgfqpoint{5.312116in}{0.641735in}}%
\pgfpathcurveto{\pgfqpoint{5.319821in}{0.649439in}}{\pgfqpoint{5.324149in}{0.659890in}}{\pgfqpoint{5.324149in}{0.670786in}}%
\pgfpathcurveto{\pgfqpoint{5.324149in}{0.681681in}}{\pgfqpoint{5.319821in}{0.692132in}}{\pgfqpoint{5.312116in}{0.699837in}}%
\pgfpathcurveto{\pgfqpoint{5.304412in}{0.707541in}}{\pgfqpoint{5.293961in}{0.711870in}}{\pgfqpoint{5.283066in}{0.711870in}}%
\pgfpathcurveto{\pgfqpoint{5.272170in}{0.711870in}}{\pgfqpoint{5.261719in}{0.707541in}}{\pgfqpoint{5.254015in}{0.699837in}}%
\pgfpathcurveto{\pgfqpoint{5.246311in}{0.692132in}}{\pgfqpoint{5.241982in}{0.681681in}}{\pgfqpoint{5.241982in}{0.670786in}}%
\pgfpathcurveto{\pgfqpoint{5.241982in}{0.659890in}}{\pgfqpoint{5.246311in}{0.649439in}}{\pgfqpoint{5.254015in}{0.641735in}}%
\pgfpathcurveto{\pgfqpoint{5.261719in}{0.634031in}}{\pgfqpoint{5.272170in}{0.629702in}}{\pgfqpoint{5.283066in}{0.629702in}}%
\pgfusepath{stroke}%
\end{pgfscope}%
\begin{pgfscope}%
\pgfpathrectangle{\pgfqpoint{0.688192in}{0.670138in}}{\pgfqpoint{7.111808in}{5.061530in}}%
\pgfusepath{clip}%
\pgfsetbuttcap%
\pgfsetroundjoin%
\pgfsetlinewidth{1.003750pt}%
\definecolor{currentstroke}{rgb}{0.000000,0.000000,0.000000}%
\pgfsetstrokecolor{currentstroke}%
\pgfsetdash{}{0pt}%
\pgfpathmoveto{\pgfqpoint{1.363285in}{0.643335in}}%
\pgfpathcurveto{\pgfqpoint{1.374180in}{0.643335in}}{\pgfqpoint{1.384631in}{0.647663in}}{\pgfqpoint{1.392336in}{0.655368in}}%
\pgfpathcurveto{\pgfqpoint{1.400040in}{0.663072in}}{\pgfqpoint{1.404369in}{0.673523in}}{\pgfqpoint{1.404369in}{0.684419in}}%
\pgfpathcurveto{\pgfqpoint{1.404369in}{0.695314in}}{\pgfqpoint{1.400040in}{0.705765in}}{\pgfqpoint{1.392336in}{0.713469in}}%
\pgfpathcurveto{\pgfqpoint{1.384631in}{0.721174in}}{\pgfqpoint{1.374180in}{0.725502in}}{\pgfqpoint{1.363285in}{0.725502in}}%
\pgfpathcurveto{\pgfqpoint{1.352389in}{0.725502in}}{\pgfqpoint{1.341938in}{0.721174in}}{\pgfqpoint{1.334234in}{0.713469in}}%
\pgfpathcurveto{\pgfqpoint{1.326530in}{0.705765in}}{\pgfqpoint{1.322201in}{0.695314in}}{\pgfqpoint{1.322201in}{0.684419in}}%
\pgfpathcurveto{\pgfqpoint{1.322201in}{0.673523in}}{\pgfqpoint{1.326530in}{0.663072in}}{\pgfqpoint{1.334234in}{0.655368in}}%
\pgfpathcurveto{\pgfqpoint{1.341938in}{0.647663in}}{\pgfqpoint{1.352389in}{0.643335in}}{\pgfqpoint{1.363285in}{0.643335in}}%
\pgfusepath{stroke}%
\end{pgfscope}%
\begin{pgfscope}%
\pgfpathrectangle{\pgfqpoint{0.688192in}{0.670138in}}{\pgfqpoint{7.111808in}{5.061530in}}%
\pgfusepath{clip}%
\pgfsetbuttcap%
\pgfsetroundjoin%
\pgfsetlinewidth{1.003750pt}%
\definecolor{currentstroke}{rgb}{0.000000,0.000000,0.000000}%
\pgfsetstrokecolor{currentstroke}%
\pgfsetdash{}{0pt}%
\pgfpathmoveto{\pgfqpoint{0.958858in}{0.667930in}}%
\pgfpathcurveto{\pgfqpoint{0.969753in}{0.667930in}}{\pgfqpoint{0.980204in}{0.672259in}}{\pgfqpoint{0.987908in}{0.679963in}}%
\pgfpathcurveto{\pgfqpoint{0.995613in}{0.687667in}}{\pgfqpoint{0.999941in}{0.698118in}}{\pgfqpoint{0.999941in}{0.709014in}}%
\pgfpathcurveto{\pgfqpoint{0.999941in}{0.719909in}}{\pgfqpoint{0.995613in}{0.730360in}}{\pgfqpoint{0.987908in}{0.738065in}}%
\pgfpathcurveto{\pgfqpoint{0.980204in}{0.745769in}}{\pgfqpoint{0.969753in}{0.750098in}}{\pgfqpoint{0.958858in}{0.750098in}}%
\pgfpathcurveto{\pgfqpoint{0.947962in}{0.750098in}}{\pgfqpoint{0.937511in}{0.745769in}}{\pgfqpoint{0.929807in}{0.738065in}}%
\pgfpathcurveto{\pgfqpoint{0.922102in}{0.730360in}}{\pgfqpoint{0.917774in}{0.719909in}}{\pgfqpoint{0.917774in}{0.709014in}}%
\pgfpathcurveto{\pgfqpoint{0.917774in}{0.698118in}}{\pgfqpoint{0.922102in}{0.687667in}}{\pgfqpoint{0.929807in}{0.679963in}}%
\pgfpathcurveto{\pgfqpoint{0.937511in}{0.672259in}}{\pgfqpoint{0.947962in}{0.667930in}}{\pgfqpoint{0.958858in}{0.667930in}}%
\pgfpathlineto{\pgfqpoint{0.958858in}{0.667930in}}%
\pgfpathclose%
\pgfusepath{stroke}%
\end{pgfscope}%
\begin{pgfscope}%
\pgfpathrectangle{\pgfqpoint{0.688192in}{0.670138in}}{\pgfqpoint{7.111808in}{5.061530in}}%
\pgfusepath{clip}%
\pgfsetbuttcap%
\pgfsetroundjoin%
\pgfsetlinewidth{1.003750pt}%
\definecolor{currentstroke}{rgb}{0.000000,0.000000,0.000000}%
\pgfsetstrokecolor{currentstroke}%
\pgfsetdash{}{0pt}%
\pgfpathmoveto{\pgfqpoint{2.489265in}{0.637384in}}%
\pgfpathcurveto{\pgfqpoint{2.500160in}{0.637384in}}{\pgfqpoint{2.510611in}{0.641713in}}{\pgfqpoint{2.518315in}{0.649417in}}%
\pgfpathcurveto{\pgfqpoint{2.526020in}{0.657122in}}{\pgfqpoint{2.530349in}{0.667572in}}{\pgfqpoint{2.530349in}{0.678468in}}%
\pgfpathcurveto{\pgfqpoint{2.530349in}{0.689364in}}{\pgfqpoint{2.526020in}{0.699814in}}{\pgfqpoint{2.518315in}{0.707519in}}%
\pgfpathcurveto{\pgfqpoint{2.510611in}{0.715223in}}{\pgfqpoint{2.500160in}{0.719552in}}{\pgfqpoint{2.489265in}{0.719552in}}%
\pgfpathcurveto{\pgfqpoint{2.478369in}{0.719552in}}{\pgfqpoint{2.467918in}{0.715223in}}{\pgfqpoint{2.460214in}{0.707519in}}%
\pgfpathcurveto{\pgfqpoint{2.452510in}{0.699814in}}{\pgfqpoint{2.448181in}{0.689364in}}{\pgfqpoint{2.448181in}{0.678468in}}%
\pgfpathcurveto{\pgfqpoint{2.448181in}{0.667572in}}{\pgfqpoint{2.452510in}{0.657122in}}{\pgfqpoint{2.460214in}{0.649417in}}%
\pgfpathcurveto{\pgfqpoint{2.467918in}{0.641713in}}{\pgfqpoint{2.478369in}{0.637384in}}{\pgfqpoint{2.489265in}{0.637384in}}%
\pgfusepath{stroke}%
\end{pgfscope}%
\begin{pgfscope}%
\pgfpathrectangle{\pgfqpoint{0.688192in}{0.670138in}}{\pgfqpoint{7.111808in}{5.061530in}}%
\pgfusepath{clip}%
\pgfsetbuttcap%
\pgfsetroundjoin%
\pgfsetlinewidth{1.003750pt}%
\definecolor{currentstroke}{rgb}{0.000000,0.000000,0.000000}%
\pgfsetstrokecolor{currentstroke}%
\pgfsetdash{}{0pt}%
\pgfpathmoveto{\pgfqpoint{0.945763in}{0.669750in}}%
\pgfpathcurveto{\pgfqpoint{0.956659in}{0.669750in}}{\pgfqpoint{0.967110in}{0.674079in}}{\pgfqpoint{0.974814in}{0.681783in}}%
\pgfpathcurveto{\pgfqpoint{0.982518in}{0.689488in}}{\pgfqpoint{0.986847in}{0.699938in}}{\pgfqpoint{0.986847in}{0.710834in}}%
\pgfpathcurveto{\pgfqpoint{0.986847in}{0.721730in}}{\pgfqpoint{0.982518in}{0.732180in}}{\pgfqpoint{0.974814in}{0.739885in}}%
\pgfpathcurveto{\pgfqpoint{0.967110in}{0.747589in}}{\pgfqpoint{0.956659in}{0.751918in}}{\pgfqpoint{0.945763in}{0.751918in}}%
\pgfpathcurveto{\pgfqpoint{0.934868in}{0.751918in}}{\pgfqpoint{0.924417in}{0.747589in}}{\pgfqpoint{0.916713in}{0.739885in}}%
\pgfpathcurveto{\pgfqpoint{0.909008in}{0.732180in}}{\pgfqpoint{0.904679in}{0.721730in}}{\pgfqpoint{0.904679in}{0.710834in}}%
\pgfpathcurveto{\pgfqpoint{0.904679in}{0.699938in}}{\pgfqpoint{0.909008in}{0.689488in}}{\pgfqpoint{0.916713in}{0.681783in}}%
\pgfpathcurveto{\pgfqpoint{0.924417in}{0.674079in}}{\pgfqpoint{0.934868in}{0.669750in}}{\pgfqpoint{0.945763in}{0.669750in}}%
\pgfpathlineto{\pgfqpoint{0.945763in}{0.669750in}}%
\pgfpathclose%
\pgfusepath{stroke}%
\end{pgfscope}%
\begin{pgfscope}%
\pgfpathrectangle{\pgfqpoint{0.688192in}{0.670138in}}{\pgfqpoint{7.111808in}{5.061530in}}%
\pgfusepath{clip}%
\pgfsetbuttcap%
\pgfsetroundjoin%
\pgfsetlinewidth{1.003750pt}%
\definecolor{currentstroke}{rgb}{0.000000,0.000000,0.000000}%
\pgfsetstrokecolor{currentstroke}%
\pgfsetdash{}{0pt}%
\pgfpathmoveto{\pgfqpoint{5.359877in}{0.629627in}}%
\pgfpathcurveto{\pgfqpoint{5.370772in}{0.629627in}}{\pgfqpoint{5.381223in}{0.633955in}}{\pgfqpoint{5.388928in}{0.641660in}}%
\pgfpathcurveto{\pgfqpoint{5.396632in}{0.649364in}}{\pgfqpoint{5.400961in}{0.659815in}}{\pgfqpoint{5.400961in}{0.670710in}}%
\pgfpathcurveto{\pgfqpoint{5.400961in}{0.681606in}}{\pgfqpoint{5.396632in}{0.692057in}}{\pgfqpoint{5.388928in}{0.699761in}}%
\pgfpathcurveto{\pgfqpoint{5.381223in}{0.707465in}}{\pgfqpoint{5.370772in}{0.711794in}}{\pgfqpoint{5.359877in}{0.711794in}}%
\pgfpathcurveto{\pgfqpoint{5.348981in}{0.711794in}}{\pgfqpoint{5.338530in}{0.707465in}}{\pgfqpoint{5.330826in}{0.699761in}}%
\pgfpathcurveto{\pgfqpoint{5.323122in}{0.692057in}}{\pgfqpoint{5.318793in}{0.681606in}}{\pgfqpoint{5.318793in}{0.670710in}}%
\pgfpathcurveto{\pgfqpoint{5.318793in}{0.659815in}}{\pgfqpoint{5.323122in}{0.649364in}}{\pgfqpoint{5.330826in}{0.641660in}}%
\pgfpathcurveto{\pgfqpoint{5.338530in}{0.633955in}}{\pgfqpoint{5.348981in}{0.629627in}}{\pgfqpoint{5.359877in}{0.629627in}}%
\pgfusepath{stroke}%
\end{pgfscope}%
\begin{pgfscope}%
\pgfpathrectangle{\pgfqpoint{0.688192in}{0.670138in}}{\pgfqpoint{7.111808in}{5.061530in}}%
\pgfusepath{clip}%
\pgfsetbuttcap%
\pgfsetroundjoin%
\pgfsetlinewidth{1.003750pt}%
\definecolor{currentstroke}{rgb}{0.000000,0.000000,0.000000}%
\pgfsetstrokecolor{currentstroke}%
\pgfsetdash{}{0pt}%
\pgfpathmoveto{\pgfqpoint{1.732115in}{0.641446in}}%
\pgfpathcurveto{\pgfqpoint{1.743010in}{0.641446in}}{\pgfqpoint{1.753461in}{0.645774in}}{\pgfqpoint{1.761165in}{0.653479in}}%
\pgfpathcurveto{\pgfqpoint{1.768870in}{0.661183in}}{\pgfqpoint{1.773199in}{0.671634in}}{\pgfqpoint{1.773199in}{0.682529in}}%
\pgfpathcurveto{\pgfqpoint{1.773199in}{0.693425in}}{\pgfqpoint{1.768870in}{0.703876in}}{\pgfqpoint{1.761165in}{0.711580in}}%
\pgfpathcurveto{\pgfqpoint{1.753461in}{0.719285in}}{\pgfqpoint{1.743010in}{0.723613in}}{\pgfqpoint{1.732115in}{0.723613in}}%
\pgfpathcurveto{\pgfqpoint{1.721219in}{0.723613in}}{\pgfqpoint{1.710768in}{0.719285in}}{\pgfqpoint{1.703064in}{0.711580in}}%
\pgfpathcurveto{\pgfqpoint{1.695360in}{0.703876in}}{\pgfqpoint{1.691031in}{0.693425in}}{\pgfqpoint{1.691031in}{0.682529in}}%
\pgfpathcurveto{\pgfqpoint{1.691031in}{0.671634in}}{\pgfqpoint{1.695360in}{0.661183in}}{\pgfqpoint{1.703064in}{0.653479in}}%
\pgfpathcurveto{\pgfqpoint{1.710768in}{0.645774in}}{\pgfqpoint{1.721219in}{0.641446in}}{\pgfqpoint{1.732115in}{0.641446in}}%
\pgfusepath{stroke}%
\end{pgfscope}%
\begin{pgfscope}%
\pgfpathrectangle{\pgfqpoint{0.688192in}{0.670138in}}{\pgfqpoint{7.111808in}{5.061530in}}%
\pgfusepath{clip}%
\pgfsetbuttcap%
\pgfsetroundjoin%
\pgfsetlinewidth{1.003750pt}%
\definecolor{currentstroke}{rgb}{0.000000,0.000000,0.000000}%
\pgfsetstrokecolor{currentstroke}%
\pgfsetdash{}{0pt}%
\pgfpathmoveto{\pgfqpoint{5.983651in}{3.772302in}}%
\pgfpathcurveto{\pgfqpoint{5.994547in}{3.772302in}}{\pgfqpoint{6.004998in}{3.776630in}}{\pgfqpoint{6.012702in}{3.784335in}}%
\pgfpathcurveto{\pgfqpoint{6.020406in}{3.792039in}}{\pgfqpoint{6.024735in}{3.802490in}}{\pgfqpoint{6.024735in}{3.813385in}}%
\pgfpathcurveto{\pgfqpoint{6.024735in}{3.824281in}}{\pgfqpoint{6.020406in}{3.834732in}}{\pgfqpoint{6.012702in}{3.842436in}}%
\pgfpathcurveto{\pgfqpoint{6.004998in}{3.850140in}}{\pgfqpoint{5.994547in}{3.854469in}}{\pgfqpoint{5.983651in}{3.854469in}}%
\pgfpathcurveto{\pgfqpoint{5.972756in}{3.854469in}}{\pgfqpoint{5.962305in}{3.850140in}}{\pgfqpoint{5.954600in}{3.842436in}}%
\pgfpathcurveto{\pgfqpoint{5.946896in}{3.834732in}}{\pgfqpoint{5.942567in}{3.824281in}}{\pgfqpoint{5.942567in}{3.813385in}}%
\pgfpathcurveto{\pgfqpoint{5.942567in}{3.802490in}}{\pgfqpoint{5.946896in}{3.792039in}}{\pgfqpoint{5.954600in}{3.784335in}}%
\pgfpathcurveto{\pgfqpoint{5.962305in}{3.776630in}}{\pgfqpoint{5.972756in}{3.772302in}}{\pgfqpoint{5.983651in}{3.772302in}}%
\pgfpathlineto{\pgfqpoint{5.983651in}{3.772302in}}%
\pgfpathclose%
\pgfusepath{stroke}%
\end{pgfscope}%
\begin{pgfscope}%
\pgfpathrectangle{\pgfqpoint{0.688192in}{0.670138in}}{\pgfqpoint{7.111808in}{5.061530in}}%
\pgfusepath{clip}%
\pgfsetbuttcap%
\pgfsetroundjoin%
\pgfsetlinewidth{1.003750pt}%
\definecolor{currentstroke}{rgb}{0.000000,0.000000,0.000000}%
\pgfsetstrokecolor{currentstroke}%
\pgfsetdash{}{0pt}%
\pgfpathmoveto{\pgfqpoint{1.967310in}{0.639999in}}%
\pgfpathcurveto{\pgfqpoint{1.978206in}{0.639999in}}{\pgfqpoint{1.988657in}{0.644328in}}{\pgfqpoint{1.996361in}{0.652032in}}%
\pgfpathcurveto{\pgfqpoint{2.004065in}{0.659736in}}{\pgfqpoint{2.008394in}{0.670187in}}{\pgfqpoint{2.008394in}{0.681083in}}%
\pgfpathcurveto{\pgfqpoint{2.008394in}{0.691978in}}{\pgfqpoint{2.004065in}{0.702429in}}{\pgfqpoint{1.996361in}{0.710133in}}%
\pgfpathcurveto{\pgfqpoint{1.988657in}{0.717838in}}{\pgfqpoint{1.978206in}{0.722166in}}{\pgfqpoint{1.967310in}{0.722166in}}%
\pgfpathcurveto{\pgfqpoint{1.956415in}{0.722166in}}{\pgfqpoint{1.945964in}{0.717838in}}{\pgfqpoint{1.938259in}{0.710133in}}%
\pgfpathcurveto{\pgfqpoint{1.930555in}{0.702429in}}{\pgfqpoint{1.926226in}{0.691978in}}{\pgfqpoint{1.926226in}{0.681083in}}%
\pgfpathcurveto{\pgfqpoint{1.926226in}{0.670187in}}{\pgfqpoint{1.930555in}{0.659736in}}{\pgfqpoint{1.938259in}{0.652032in}}%
\pgfpathcurveto{\pgfqpoint{1.945964in}{0.644328in}}{\pgfqpoint{1.956415in}{0.639999in}}{\pgfqpoint{1.967310in}{0.639999in}}%
\pgfusepath{stroke}%
\end{pgfscope}%
\begin{pgfscope}%
\pgfpathrectangle{\pgfqpoint{0.688192in}{0.670138in}}{\pgfqpoint{7.111808in}{5.061530in}}%
\pgfusepath{clip}%
\pgfsetbuttcap%
\pgfsetroundjoin%
\pgfsetlinewidth{1.003750pt}%
\definecolor{currentstroke}{rgb}{0.000000,0.000000,0.000000}%
\pgfsetstrokecolor{currentstroke}%
\pgfsetdash{}{0pt}%
\pgfpathmoveto{\pgfqpoint{4.750941in}{0.630440in}}%
\pgfpathcurveto{\pgfqpoint{4.761837in}{0.630440in}}{\pgfqpoint{4.772288in}{0.634769in}}{\pgfqpoint{4.779992in}{0.642473in}}%
\pgfpathcurveto{\pgfqpoint{4.787696in}{0.650177in}}{\pgfqpoint{4.792025in}{0.660628in}}{\pgfqpoint{4.792025in}{0.671524in}}%
\pgfpathcurveto{\pgfqpoint{4.792025in}{0.682419in}}{\pgfqpoint{4.787696in}{0.692870in}}{\pgfqpoint{4.779992in}{0.700574in}}%
\pgfpathcurveto{\pgfqpoint{4.772288in}{0.708279in}}{\pgfqpoint{4.761837in}{0.712608in}}{\pgfqpoint{4.750941in}{0.712608in}}%
\pgfpathcurveto{\pgfqpoint{4.740046in}{0.712608in}}{\pgfqpoint{4.729595in}{0.708279in}}{\pgfqpoint{4.721891in}{0.700574in}}%
\pgfpathcurveto{\pgfqpoint{4.714186in}{0.692870in}}{\pgfqpoint{4.709858in}{0.682419in}}{\pgfqpoint{4.709858in}{0.671524in}}%
\pgfpathcurveto{\pgfqpoint{4.709858in}{0.660628in}}{\pgfqpoint{4.714186in}{0.650177in}}{\pgfqpoint{4.721891in}{0.642473in}}%
\pgfpathcurveto{\pgfqpoint{4.729595in}{0.634769in}}{\pgfqpoint{4.740046in}{0.630440in}}{\pgfqpoint{4.750941in}{0.630440in}}%
\pgfusepath{stroke}%
\end{pgfscope}%
\begin{pgfscope}%
\pgfpathrectangle{\pgfqpoint{0.688192in}{0.670138in}}{\pgfqpoint{7.111808in}{5.061530in}}%
\pgfusepath{clip}%
\pgfsetbuttcap%
\pgfsetroundjoin%
\pgfsetlinewidth{1.003750pt}%
\definecolor{currentstroke}{rgb}{0.000000,0.000000,0.000000}%
\pgfsetstrokecolor{currentstroke}%
\pgfsetdash{}{0pt}%
\pgfpathmoveto{\pgfqpoint{3.560021in}{0.632771in}}%
\pgfpathcurveto{\pgfqpoint{3.570916in}{0.632771in}}{\pgfqpoint{3.581367in}{0.637099in}}{\pgfqpoint{3.589071in}{0.644804in}}%
\pgfpathcurveto{\pgfqpoint{3.596776in}{0.652508in}}{\pgfqpoint{3.601105in}{0.662959in}}{\pgfqpoint{3.601105in}{0.673854in}}%
\pgfpathcurveto{\pgfqpoint{3.601105in}{0.684750in}}{\pgfqpoint{3.596776in}{0.695201in}}{\pgfqpoint{3.589071in}{0.702905in}}%
\pgfpathcurveto{\pgfqpoint{3.581367in}{0.710609in}}{\pgfqpoint{3.570916in}{0.714938in}}{\pgfqpoint{3.560021in}{0.714938in}}%
\pgfpathcurveto{\pgfqpoint{3.549125in}{0.714938in}}{\pgfqpoint{3.538674in}{0.710609in}}{\pgfqpoint{3.530970in}{0.702905in}}%
\pgfpathcurveto{\pgfqpoint{3.523266in}{0.695201in}}{\pgfqpoint{3.518937in}{0.684750in}}{\pgfqpoint{3.518937in}{0.673854in}}%
\pgfpathcurveto{\pgfqpoint{3.518937in}{0.662959in}}{\pgfqpoint{3.523266in}{0.652508in}}{\pgfqpoint{3.530970in}{0.644804in}}%
\pgfpathcurveto{\pgfqpoint{3.538674in}{0.637099in}}{\pgfqpoint{3.549125in}{0.632771in}}{\pgfqpoint{3.560021in}{0.632771in}}%
\pgfusepath{stroke}%
\end{pgfscope}%
\begin{pgfscope}%
\pgfpathrectangle{\pgfqpoint{0.688192in}{0.670138in}}{\pgfqpoint{7.111808in}{5.061530in}}%
\pgfusepath{clip}%
\pgfsetbuttcap%
\pgfsetroundjoin%
\pgfsetlinewidth{1.003750pt}%
\definecolor{currentstroke}{rgb}{0.000000,0.000000,0.000000}%
\pgfsetstrokecolor{currentstroke}%
\pgfsetdash{}{0pt}%
\pgfpathmoveto{\pgfqpoint{0.849436in}{0.699100in}}%
\pgfpathcurveto{\pgfqpoint{0.860331in}{0.699100in}}{\pgfqpoint{0.870782in}{0.703429in}}{\pgfqpoint{0.878486in}{0.711134in}}%
\pgfpathcurveto{\pgfqpoint{0.886191in}{0.718838in}}{\pgfqpoint{0.890519in}{0.729289in}}{\pgfqpoint{0.890519in}{0.740184in}}%
\pgfpathcurveto{\pgfqpoint{0.890519in}{0.751080in}}{\pgfqpoint{0.886191in}{0.761531in}}{\pgfqpoint{0.878486in}{0.769235in}}%
\pgfpathcurveto{\pgfqpoint{0.870782in}{0.776939in}}{\pgfqpoint{0.860331in}{0.781268in}}{\pgfqpoint{0.849436in}{0.781268in}}%
\pgfpathcurveto{\pgfqpoint{0.838540in}{0.781268in}}{\pgfqpoint{0.828089in}{0.776939in}}{\pgfqpoint{0.820385in}{0.769235in}}%
\pgfpathcurveto{\pgfqpoint{0.812681in}{0.761531in}}{\pgfqpoint{0.808352in}{0.751080in}}{\pgfqpoint{0.808352in}{0.740184in}}%
\pgfpathcurveto{\pgfqpoint{0.808352in}{0.729289in}}{\pgfqpoint{0.812681in}{0.718838in}}{\pgfqpoint{0.820385in}{0.711134in}}%
\pgfpathcurveto{\pgfqpoint{0.828089in}{0.703429in}}{\pgfqpoint{0.838540in}{0.699100in}}{\pgfqpoint{0.849436in}{0.699100in}}%
\pgfpathlineto{\pgfqpoint{0.849436in}{0.699100in}}%
\pgfpathclose%
\pgfusepath{stroke}%
\end{pgfscope}%
\begin{pgfscope}%
\pgfpathrectangle{\pgfqpoint{0.688192in}{0.670138in}}{\pgfqpoint{7.111808in}{5.061530in}}%
\pgfusepath{clip}%
\pgfsetbuttcap%
\pgfsetroundjoin%
\pgfsetlinewidth{1.003750pt}%
\definecolor{currentstroke}{rgb}{0.000000,0.000000,0.000000}%
\pgfsetstrokecolor{currentstroke}%
\pgfsetdash{}{0pt}%
\pgfpathmoveto{\pgfqpoint{0.960596in}{0.667508in}}%
\pgfpathcurveto{\pgfqpoint{0.971491in}{0.667508in}}{\pgfqpoint{0.981942in}{0.671837in}}{\pgfqpoint{0.989646in}{0.679542in}}%
\pgfpathcurveto{\pgfqpoint{0.997351in}{0.687246in}}{\pgfqpoint{1.001679in}{0.697697in}}{\pgfqpoint{1.001679in}{0.708592in}}%
\pgfpathcurveto{\pgfqpoint{1.001679in}{0.719488in}}{\pgfqpoint{0.997351in}{0.729939in}}{\pgfqpoint{0.989646in}{0.737643in}}%
\pgfpathcurveto{\pgfqpoint{0.981942in}{0.745347in}}{\pgfqpoint{0.971491in}{0.749676in}}{\pgfqpoint{0.960596in}{0.749676in}}%
\pgfpathcurveto{\pgfqpoint{0.949700in}{0.749676in}}{\pgfqpoint{0.939249in}{0.745347in}}{\pgfqpoint{0.931545in}{0.737643in}}%
\pgfpathcurveto{\pgfqpoint{0.923841in}{0.729939in}}{\pgfqpoint{0.919512in}{0.719488in}}{\pgfqpoint{0.919512in}{0.708592in}}%
\pgfpathcurveto{\pgfqpoint{0.919512in}{0.697697in}}{\pgfqpoint{0.923841in}{0.687246in}}{\pgfqpoint{0.931545in}{0.679542in}}%
\pgfpathcurveto{\pgfqpoint{0.939249in}{0.671837in}}{\pgfqpoint{0.949700in}{0.667508in}}{\pgfqpoint{0.960596in}{0.667508in}}%
\pgfusepath{stroke}%
\end{pgfscope}%
\begin{pgfscope}%
\pgfpathrectangle{\pgfqpoint{0.688192in}{0.670138in}}{\pgfqpoint{7.111808in}{5.061530in}}%
\pgfusepath{clip}%
\pgfsetbuttcap%
\pgfsetroundjoin%
\pgfsetlinewidth{1.003750pt}%
\definecolor{currentstroke}{rgb}{0.000000,0.000000,0.000000}%
\pgfsetstrokecolor{currentstroke}%
\pgfsetdash{}{0pt}%
\pgfpathmoveto{\pgfqpoint{1.108579in}{0.647776in}}%
\pgfpathcurveto{\pgfqpoint{1.119474in}{0.647776in}}{\pgfqpoint{1.129925in}{0.652105in}}{\pgfqpoint{1.137629in}{0.659809in}}%
\pgfpathcurveto{\pgfqpoint{1.145334in}{0.667513in}}{\pgfqpoint{1.149662in}{0.677964in}}{\pgfqpoint{1.149662in}{0.688860in}}%
\pgfpathcurveto{\pgfqpoint{1.149662in}{0.699755in}}{\pgfqpoint{1.145334in}{0.710206in}}{\pgfqpoint{1.137629in}{0.717910in}}%
\pgfpathcurveto{\pgfqpoint{1.129925in}{0.725615in}}{\pgfqpoint{1.119474in}{0.729944in}}{\pgfqpoint{1.108579in}{0.729944in}}%
\pgfpathcurveto{\pgfqpoint{1.097683in}{0.729944in}}{\pgfqpoint{1.087232in}{0.725615in}}{\pgfqpoint{1.079528in}{0.717910in}}%
\pgfpathcurveto{\pgfqpoint{1.071823in}{0.710206in}}{\pgfqpoint{1.067495in}{0.699755in}}{\pgfqpoint{1.067495in}{0.688860in}}%
\pgfpathcurveto{\pgfqpoint{1.067495in}{0.677964in}}{\pgfqpoint{1.071823in}{0.667513in}}{\pgfqpoint{1.079528in}{0.659809in}}%
\pgfpathcurveto{\pgfqpoint{1.087232in}{0.652105in}}{\pgfqpoint{1.097683in}{0.647776in}}{\pgfqpoint{1.108579in}{0.647776in}}%
\pgfusepath{stroke}%
\end{pgfscope}%
\begin{pgfscope}%
\pgfpathrectangle{\pgfqpoint{0.688192in}{0.670138in}}{\pgfqpoint{7.111808in}{5.061530in}}%
\pgfusepath{clip}%
\pgfsetbuttcap%
\pgfsetroundjoin%
\pgfsetlinewidth{1.003750pt}%
\definecolor{currentstroke}{rgb}{0.000000,0.000000,0.000000}%
\pgfsetstrokecolor{currentstroke}%
\pgfsetdash{}{0pt}%
\pgfpathmoveto{\pgfqpoint{1.495917in}{0.642463in}}%
\pgfpathcurveto{\pgfqpoint{1.506813in}{0.642463in}}{\pgfqpoint{1.517263in}{0.646792in}}{\pgfqpoint{1.524968in}{0.654497in}}%
\pgfpathcurveto{\pgfqpoint{1.532672in}{0.662201in}}{\pgfqpoint{1.537001in}{0.672652in}}{\pgfqpoint{1.537001in}{0.683547in}}%
\pgfpathcurveto{\pgfqpoint{1.537001in}{0.694443in}}{\pgfqpoint{1.532672in}{0.704894in}}{\pgfqpoint{1.524968in}{0.712598in}}%
\pgfpathcurveto{\pgfqpoint{1.517263in}{0.720302in}}{\pgfqpoint{1.506813in}{0.724631in}}{\pgfqpoint{1.495917in}{0.724631in}}%
\pgfpathcurveto{\pgfqpoint{1.485021in}{0.724631in}}{\pgfqpoint{1.474571in}{0.720302in}}{\pgfqpoint{1.466866in}{0.712598in}}%
\pgfpathcurveto{\pgfqpoint{1.459162in}{0.704894in}}{\pgfqpoint{1.454833in}{0.694443in}}{\pgfqpoint{1.454833in}{0.683547in}}%
\pgfpathcurveto{\pgfqpoint{1.454833in}{0.672652in}}{\pgfqpoint{1.459162in}{0.662201in}}{\pgfqpoint{1.466866in}{0.654497in}}%
\pgfpathcurveto{\pgfqpoint{1.474571in}{0.646792in}}{\pgfqpoint{1.485021in}{0.642463in}}{\pgfqpoint{1.495917in}{0.642463in}}%
\pgfusepath{stroke}%
\end{pgfscope}%
\begin{pgfscope}%
\pgfpathrectangle{\pgfqpoint{0.688192in}{0.670138in}}{\pgfqpoint{7.111808in}{5.061530in}}%
\pgfusepath{clip}%
\pgfsetbuttcap%
\pgfsetroundjoin%
\pgfsetlinewidth{1.003750pt}%
\definecolor{currentstroke}{rgb}{0.000000,0.000000,0.000000}%
\pgfsetstrokecolor{currentstroke}%
\pgfsetdash{}{0pt}%
\pgfpathmoveto{\pgfqpoint{1.444241in}{0.642951in}}%
\pgfpathcurveto{\pgfqpoint{1.455136in}{0.642951in}}{\pgfqpoint{1.465587in}{0.647280in}}{\pgfqpoint{1.473291in}{0.654984in}}%
\pgfpathcurveto{\pgfqpoint{1.480996in}{0.662688in}}{\pgfqpoint{1.485325in}{0.673139in}}{\pgfqpoint{1.485325in}{0.684035in}}%
\pgfpathcurveto{\pgfqpoint{1.485325in}{0.694930in}}{\pgfqpoint{1.480996in}{0.705381in}}{\pgfqpoint{1.473291in}{0.713085in}}%
\pgfpathcurveto{\pgfqpoint{1.465587in}{0.720790in}}{\pgfqpoint{1.455136in}{0.725119in}}{\pgfqpoint{1.444241in}{0.725119in}}%
\pgfpathcurveto{\pgfqpoint{1.433345in}{0.725119in}}{\pgfqpoint{1.422894in}{0.720790in}}{\pgfqpoint{1.415190in}{0.713085in}}%
\pgfpathcurveto{\pgfqpoint{1.407486in}{0.705381in}}{\pgfqpoint{1.403157in}{0.694930in}}{\pgfqpoint{1.403157in}{0.684035in}}%
\pgfpathcurveto{\pgfqpoint{1.403157in}{0.673139in}}{\pgfqpoint{1.407486in}{0.662688in}}{\pgfqpoint{1.415190in}{0.654984in}}%
\pgfpathcurveto{\pgfqpoint{1.422894in}{0.647280in}}{\pgfqpoint{1.433345in}{0.642951in}}{\pgfqpoint{1.444241in}{0.642951in}}%
\pgfusepath{stroke}%
\end{pgfscope}%
\begin{pgfscope}%
\pgfpathrectangle{\pgfqpoint{0.688192in}{0.670138in}}{\pgfqpoint{7.111808in}{5.061530in}}%
\pgfusepath{clip}%
\pgfsetbuttcap%
\pgfsetroundjoin%
\pgfsetlinewidth{1.003750pt}%
\definecolor{currentstroke}{rgb}{0.000000,0.000000,0.000000}%
\pgfsetstrokecolor{currentstroke}%
\pgfsetdash{}{0pt}%
\pgfpathmoveto{\pgfqpoint{1.761288in}{0.640905in}}%
\pgfpathcurveto{\pgfqpoint{1.772184in}{0.640905in}}{\pgfqpoint{1.782635in}{0.645234in}}{\pgfqpoint{1.790339in}{0.652938in}}%
\pgfpathcurveto{\pgfqpoint{1.798043in}{0.660642in}}{\pgfqpoint{1.802372in}{0.671093in}}{\pgfqpoint{1.802372in}{0.681989in}}%
\pgfpathcurveto{\pgfqpoint{1.802372in}{0.692884in}}{\pgfqpoint{1.798043in}{0.703335in}}{\pgfqpoint{1.790339in}{0.711039in}}%
\pgfpathcurveto{\pgfqpoint{1.782635in}{0.718744in}}{\pgfqpoint{1.772184in}{0.723073in}}{\pgfqpoint{1.761288in}{0.723073in}}%
\pgfpathcurveto{\pgfqpoint{1.750393in}{0.723073in}}{\pgfqpoint{1.739942in}{0.718744in}}{\pgfqpoint{1.732237in}{0.711039in}}%
\pgfpathcurveto{\pgfqpoint{1.724533in}{0.703335in}}{\pgfqpoint{1.720204in}{0.692884in}}{\pgfqpoint{1.720204in}{0.681989in}}%
\pgfpathcurveto{\pgfqpoint{1.720204in}{0.671093in}}{\pgfqpoint{1.724533in}{0.660642in}}{\pgfqpoint{1.732237in}{0.652938in}}%
\pgfpathcurveto{\pgfqpoint{1.739942in}{0.645234in}}{\pgfqpoint{1.750393in}{0.640905in}}{\pgfqpoint{1.761288in}{0.640905in}}%
\pgfusepath{stroke}%
\end{pgfscope}%
\begin{pgfscope}%
\pgfpathrectangle{\pgfqpoint{0.688192in}{0.670138in}}{\pgfqpoint{7.111808in}{5.061530in}}%
\pgfusepath{clip}%
\pgfsetbuttcap%
\pgfsetroundjoin%
\pgfsetlinewidth{1.003750pt}%
\definecolor{currentstroke}{rgb}{0.000000,0.000000,0.000000}%
\pgfsetstrokecolor{currentstroke}%
\pgfsetdash{}{0pt}%
\pgfpathmoveto{\pgfqpoint{5.274608in}{0.629748in}}%
\pgfpathcurveto{\pgfqpoint{5.285503in}{0.629748in}}{\pgfqpoint{5.295954in}{0.634077in}}{\pgfqpoint{5.303659in}{0.641781in}}%
\pgfpathcurveto{\pgfqpoint{5.311363in}{0.649486in}}{\pgfqpoint{5.315692in}{0.659937in}}{\pgfqpoint{5.315692in}{0.670832in}}%
\pgfpathcurveto{\pgfqpoint{5.315692in}{0.681728in}}{\pgfqpoint{5.311363in}{0.692179in}}{\pgfqpoint{5.303659in}{0.699883in}}%
\pgfpathcurveto{\pgfqpoint{5.295954in}{0.707587in}}{\pgfqpoint{5.285503in}{0.711916in}}{\pgfqpoint{5.274608in}{0.711916in}}%
\pgfpathcurveto{\pgfqpoint{5.263712in}{0.711916in}}{\pgfqpoint{5.253261in}{0.707587in}}{\pgfqpoint{5.245557in}{0.699883in}}%
\pgfpathcurveto{\pgfqpoint{5.237853in}{0.692179in}}{\pgfqpoint{5.233524in}{0.681728in}}{\pgfqpoint{5.233524in}{0.670832in}}%
\pgfpathcurveto{\pgfqpoint{5.233524in}{0.659937in}}{\pgfqpoint{5.237853in}{0.649486in}}{\pgfqpoint{5.245557in}{0.641781in}}%
\pgfpathcurveto{\pgfqpoint{5.253261in}{0.634077in}}{\pgfqpoint{5.263712in}{0.629748in}}{\pgfqpoint{5.274608in}{0.629748in}}%
\pgfusepath{stroke}%
\end{pgfscope}%
\begin{pgfscope}%
\pgfpathrectangle{\pgfqpoint{0.688192in}{0.670138in}}{\pgfqpoint{7.111808in}{5.061530in}}%
\pgfusepath{clip}%
\pgfsetbuttcap%
\pgfsetroundjoin%
\pgfsetlinewidth{1.003750pt}%
\definecolor{currentstroke}{rgb}{0.000000,0.000000,0.000000}%
\pgfsetstrokecolor{currentstroke}%
\pgfsetdash{}{0pt}%
\pgfpathmoveto{\pgfqpoint{2.348683in}{2.153861in}}%
\pgfpathcurveto{\pgfqpoint{2.359578in}{2.153861in}}{\pgfqpoint{2.370029in}{2.158190in}}{\pgfqpoint{2.377733in}{2.165894in}}%
\pgfpathcurveto{\pgfqpoint{2.385438in}{2.173599in}}{\pgfqpoint{2.389767in}{2.184050in}}{\pgfqpoint{2.389767in}{2.194945in}}%
\pgfpathcurveto{\pgfqpoint{2.389767in}{2.205841in}}{\pgfqpoint{2.385438in}{2.216291in}}{\pgfqpoint{2.377733in}{2.223996in}}%
\pgfpathcurveto{\pgfqpoint{2.370029in}{2.231700in}}{\pgfqpoint{2.359578in}{2.236029in}}{\pgfqpoint{2.348683in}{2.236029in}}%
\pgfpathcurveto{\pgfqpoint{2.337787in}{2.236029in}}{\pgfqpoint{2.327336in}{2.231700in}}{\pgfqpoint{2.319632in}{2.223996in}}%
\pgfpathcurveto{\pgfqpoint{2.311928in}{2.216291in}}{\pgfqpoint{2.307599in}{2.205841in}}{\pgfqpoint{2.307599in}{2.194945in}}%
\pgfpathcurveto{\pgfqpoint{2.307599in}{2.184050in}}{\pgfqpoint{2.311928in}{2.173599in}}{\pgfqpoint{2.319632in}{2.165894in}}%
\pgfpathcurveto{\pgfqpoint{2.327336in}{2.158190in}}{\pgfqpoint{2.337787in}{2.153861in}}{\pgfqpoint{2.348683in}{2.153861in}}%
\pgfpathlineto{\pgfqpoint{2.348683in}{2.153861in}}%
\pgfpathclose%
\pgfusepath{stroke}%
\end{pgfscope}%
\begin{pgfscope}%
\pgfpathrectangle{\pgfqpoint{0.688192in}{0.670138in}}{\pgfqpoint{7.111808in}{5.061530in}}%
\pgfusepath{clip}%
\pgfsetbuttcap%
\pgfsetroundjoin%
\pgfsetlinewidth{1.003750pt}%
\definecolor{currentstroke}{rgb}{0.000000,0.000000,0.000000}%
\pgfsetstrokecolor{currentstroke}%
\pgfsetdash{}{0pt}%
\pgfpathmoveto{\pgfqpoint{4.688548in}{0.630639in}}%
\pgfpathcurveto{\pgfqpoint{4.699443in}{0.630639in}}{\pgfqpoint{4.709894in}{0.634968in}}{\pgfqpoint{4.717598in}{0.642672in}}%
\pgfpathcurveto{\pgfqpoint{4.725303in}{0.650377in}}{\pgfqpoint{4.729631in}{0.660827in}}{\pgfqpoint{4.729631in}{0.671723in}}%
\pgfpathcurveto{\pgfqpoint{4.729631in}{0.682619in}}{\pgfqpoint{4.725303in}{0.693069in}}{\pgfqpoint{4.717598in}{0.700774in}}%
\pgfpathcurveto{\pgfqpoint{4.709894in}{0.708478in}}{\pgfqpoint{4.699443in}{0.712807in}}{\pgfqpoint{4.688548in}{0.712807in}}%
\pgfpathcurveto{\pgfqpoint{4.677652in}{0.712807in}}{\pgfqpoint{4.667201in}{0.708478in}}{\pgfqpoint{4.659497in}{0.700774in}}%
\pgfpathcurveto{\pgfqpoint{4.651792in}{0.693069in}}{\pgfqpoint{4.647464in}{0.682619in}}{\pgfqpoint{4.647464in}{0.671723in}}%
\pgfpathcurveto{\pgfqpoint{4.647464in}{0.660827in}}{\pgfqpoint{4.651792in}{0.650377in}}{\pgfqpoint{4.659497in}{0.642672in}}%
\pgfpathcurveto{\pgfqpoint{4.667201in}{0.634968in}}{\pgfqpoint{4.677652in}{0.630639in}}{\pgfqpoint{4.688548in}{0.630639in}}%
\pgfusepath{stroke}%
\end{pgfscope}%
\begin{pgfscope}%
\pgfpathrectangle{\pgfqpoint{0.688192in}{0.670138in}}{\pgfqpoint{7.111808in}{5.061530in}}%
\pgfusepath{clip}%
\pgfsetbuttcap%
\pgfsetroundjoin%
\pgfsetlinewidth{1.003750pt}%
\definecolor{currentstroke}{rgb}{0.000000,0.000000,0.000000}%
\pgfsetstrokecolor{currentstroke}%
\pgfsetdash{}{0pt}%
\pgfpathmoveto{\pgfqpoint{3.291859in}{0.633837in}}%
\pgfpathcurveto{\pgfqpoint{3.302755in}{0.633837in}}{\pgfqpoint{3.313206in}{0.638166in}}{\pgfqpoint{3.320910in}{0.645871in}}%
\pgfpathcurveto{\pgfqpoint{3.328614in}{0.653575in}}{\pgfqpoint{3.332943in}{0.664026in}}{\pgfqpoint{3.332943in}{0.674921in}}%
\pgfpathcurveto{\pgfqpoint{3.332943in}{0.685817in}}{\pgfqpoint{3.328614in}{0.696268in}}{\pgfqpoint{3.320910in}{0.703972in}}%
\pgfpathcurveto{\pgfqpoint{3.313206in}{0.711676in}}{\pgfqpoint{3.302755in}{0.716005in}}{\pgfqpoint{3.291859in}{0.716005in}}%
\pgfpathcurveto{\pgfqpoint{3.280964in}{0.716005in}}{\pgfqpoint{3.270513in}{0.711676in}}{\pgfqpoint{3.262809in}{0.703972in}}%
\pgfpathcurveto{\pgfqpoint{3.255104in}{0.696268in}}{\pgfqpoint{3.250775in}{0.685817in}}{\pgfqpoint{3.250775in}{0.674921in}}%
\pgfpathcurveto{\pgfqpoint{3.250775in}{0.664026in}}{\pgfqpoint{3.255104in}{0.653575in}}{\pgfqpoint{3.262809in}{0.645871in}}%
\pgfpathcurveto{\pgfqpoint{3.270513in}{0.638166in}}{\pgfqpoint{3.280964in}{0.633837in}}{\pgfqpoint{3.291859in}{0.633837in}}%
\pgfusepath{stroke}%
\end{pgfscope}%
\begin{pgfscope}%
\pgfpathrectangle{\pgfqpoint{0.688192in}{0.670138in}}{\pgfqpoint{7.111808in}{5.061530in}}%
\pgfusepath{clip}%
\pgfsetbuttcap%
\pgfsetroundjoin%
\pgfsetlinewidth{1.003750pt}%
\definecolor{currentstroke}{rgb}{0.000000,0.000000,0.000000}%
\pgfsetstrokecolor{currentstroke}%
\pgfsetdash{}{0pt}%
\pgfpathmoveto{\pgfqpoint{5.283066in}{0.629702in}}%
\pgfpathcurveto{\pgfqpoint{5.293961in}{0.629702in}}{\pgfqpoint{5.304412in}{0.634031in}}{\pgfqpoint{5.312116in}{0.641735in}}%
\pgfpathcurveto{\pgfqpoint{5.319821in}{0.649439in}}{\pgfqpoint{5.324149in}{0.659890in}}{\pgfqpoint{5.324149in}{0.670786in}}%
\pgfpathcurveto{\pgfqpoint{5.324149in}{0.681681in}}{\pgfqpoint{5.319821in}{0.692132in}}{\pgfqpoint{5.312116in}{0.699837in}}%
\pgfpathcurveto{\pgfqpoint{5.304412in}{0.707541in}}{\pgfqpoint{5.293961in}{0.711870in}}{\pgfqpoint{5.283066in}{0.711870in}}%
\pgfpathcurveto{\pgfqpoint{5.272170in}{0.711870in}}{\pgfqpoint{5.261719in}{0.707541in}}{\pgfqpoint{5.254015in}{0.699837in}}%
\pgfpathcurveto{\pgfqpoint{5.246311in}{0.692132in}}{\pgfqpoint{5.241982in}{0.681681in}}{\pgfqpoint{5.241982in}{0.670786in}}%
\pgfpathcurveto{\pgfqpoint{5.241982in}{0.659890in}}{\pgfqpoint{5.246311in}{0.649439in}}{\pgfqpoint{5.254015in}{0.641735in}}%
\pgfpathcurveto{\pgfqpoint{5.261719in}{0.634031in}}{\pgfqpoint{5.272170in}{0.629702in}}{\pgfqpoint{5.283066in}{0.629702in}}%
\pgfusepath{stroke}%
\end{pgfscope}%
\begin{pgfscope}%
\pgfpathrectangle{\pgfqpoint{0.688192in}{0.670138in}}{\pgfqpoint{7.111808in}{5.061530in}}%
\pgfusepath{clip}%
\pgfsetbuttcap%
\pgfsetroundjoin%
\pgfsetlinewidth{1.003750pt}%
\definecolor{currentstroke}{rgb}{0.000000,0.000000,0.000000}%
\pgfsetstrokecolor{currentstroke}%
\pgfsetdash{}{0pt}%
\pgfpathmoveto{\pgfqpoint{1.363285in}{0.643335in}}%
\pgfpathcurveto{\pgfqpoint{1.374180in}{0.643335in}}{\pgfqpoint{1.384631in}{0.647663in}}{\pgfqpoint{1.392336in}{0.655368in}}%
\pgfpathcurveto{\pgfqpoint{1.400040in}{0.663072in}}{\pgfqpoint{1.404369in}{0.673523in}}{\pgfqpoint{1.404369in}{0.684419in}}%
\pgfpathcurveto{\pgfqpoint{1.404369in}{0.695314in}}{\pgfqpoint{1.400040in}{0.705765in}}{\pgfqpoint{1.392336in}{0.713469in}}%
\pgfpathcurveto{\pgfqpoint{1.384631in}{0.721174in}}{\pgfqpoint{1.374180in}{0.725502in}}{\pgfqpoint{1.363285in}{0.725502in}}%
\pgfpathcurveto{\pgfqpoint{1.352389in}{0.725502in}}{\pgfqpoint{1.341938in}{0.721174in}}{\pgfqpoint{1.334234in}{0.713469in}}%
\pgfpathcurveto{\pgfqpoint{1.326530in}{0.705765in}}{\pgfqpoint{1.322201in}{0.695314in}}{\pgfqpoint{1.322201in}{0.684419in}}%
\pgfpathcurveto{\pgfqpoint{1.322201in}{0.673523in}}{\pgfqpoint{1.326530in}{0.663072in}}{\pgfqpoint{1.334234in}{0.655368in}}%
\pgfpathcurveto{\pgfqpoint{1.341938in}{0.647663in}}{\pgfqpoint{1.352389in}{0.643335in}}{\pgfqpoint{1.363285in}{0.643335in}}%
\pgfusepath{stroke}%
\end{pgfscope}%
\begin{pgfscope}%
\pgfpathrectangle{\pgfqpoint{0.688192in}{0.670138in}}{\pgfqpoint{7.111808in}{5.061530in}}%
\pgfusepath{clip}%
\pgfsetbuttcap%
\pgfsetroundjoin%
\pgfsetlinewidth{1.003750pt}%
\definecolor{currentstroke}{rgb}{0.000000,0.000000,0.000000}%
\pgfsetstrokecolor{currentstroke}%
\pgfsetdash{}{0pt}%
\pgfpathmoveto{\pgfqpoint{3.291859in}{0.633837in}}%
\pgfpathcurveto{\pgfqpoint{3.302755in}{0.633837in}}{\pgfqpoint{3.313206in}{0.638166in}}{\pgfqpoint{3.320910in}{0.645871in}}%
\pgfpathcurveto{\pgfqpoint{3.328614in}{0.653575in}}{\pgfqpoint{3.332943in}{0.664026in}}{\pgfqpoint{3.332943in}{0.674921in}}%
\pgfpathcurveto{\pgfqpoint{3.332943in}{0.685817in}}{\pgfqpoint{3.328614in}{0.696268in}}{\pgfqpoint{3.320910in}{0.703972in}}%
\pgfpathcurveto{\pgfqpoint{3.313206in}{0.711676in}}{\pgfqpoint{3.302755in}{0.716005in}}{\pgfqpoint{3.291859in}{0.716005in}}%
\pgfpathcurveto{\pgfqpoint{3.280964in}{0.716005in}}{\pgfqpoint{3.270513in}{0.711676in}}{\pgfqpoint{3.262809in}{0.703972in}}%
\pgfpathcurveto{\pgfqpoint{3.255104in}{0.696268in}}{\pgfqpoint{3.250775in}{0.685817in}}{\pgfqpoint{3.250775in}{0.674921in}}%
\pgfpathcurveto{\pgfqpoint{3.250775in}{0.664026in}}{\pgfqpoint{3.255104in}{0.653575in}}{\pgfqpoint{3.262809in}{0.645871in}}%
\pgfpathcurveto{\pgfqpoint{3.270513in}{0.638166in}}{\pgfqpoint{3.280964in}{0.633837in}}{\pgfqpoint{3.291859in}{0.633837in}}%
\pgfusepath{stroke}%
\end{pgfscope}%
\begin{pgfscope}%
\pgfpathrectangle{\pgfqpoint{0.688192in}{0.670138in}}{\pgfqpoint{7.111808in}{5.061530in}}%
\pgfusepath{clip}%
\pgfsetbuttcap%
\pgfsetroundjoin%
\pgfsetlinewidth{1.003750pt}%
\definecolor{currentstroke}{rgb}{0.000000,0.000000,0.000000}%
\pgfsetstrokecolor{currentstroke}%
\pgfsetdash{}{0pt}%
\pgfpathmoveto{\pgfqpoint{1.103303in}{0.648067in}}%
\pgfpathcurveto{\pgfqpoint{1.114198in}{0.648067in}}{\pgfqpoint{1.124649in}{0.652396in}}{\pgfqpoint{1.132353in}{0.660101in}}%
\pgfpathcurveto{\pgfqpoint{1.140058in}{0.667805in}}{\pgfqpoint{1.144386in}{0.678256in}}{\pgfqpoint{1.144386in}{0.689151in}}%
\pgfpathcurveto{\pgfqpoint{1.144386in}{0.700047in}}{\pgfqpoint{1.140058in}{0.710498in}}{\pgfqpoint{1.132353in}{0.718202in}}%
\pgfpathcurveto{\pgfqpoint{1.124649in}{0.725906in}}{\pgfqpoint{1.114198in}{0.730235in}}{\pgfqpoint{1.103303in}{0.730235in}}%
\pgfpathcurveto{\pgfqpoint{1.092407in}{0.730235in}}{\pgfqpoint{1.081956in}{0.725906in}}{\pgfqpoint{1.074252in}{0.718202in}}%
\pgfpathcurveto{\pgfqpoint{1.066548in}{0.710498in}}{\pgfqpoint{1.062219in}{0.700047in}}{\pgfqpoint{1.062219in}{0.689151in}}%
\pgfpathcurveto{\pgfqpoint{1.062219in}{0.678256in}}{\pgfqpoint{1.066548in}{0.667805in}}{\pgfqpoint{1.074252in}{0.660101in}}%
\pgfpathcurveto{\pgfqpoint{1.081956in}{0.652396in}}{\pgfqpoint{1.092407in}{0.648067in}}{\pgfqpoint{1.103303in}{0.648067in}}%
\pgfusepath{stroke}%
\end{pgfscope}%
\begin{pgfscope}%
\pgfpathrectangle{\pgfqpoint{0.688192in}{0.670138in}}{\pgfqpoint{7.111808in}{5.061530in}}%
\pgfusepath{clip}%
\pgfsetbuttcap%
\pgfsetroundjoin%
\pgfsetlinewidth{1.003750pt}%
\definecolor{currentstroke}{rgb}{0.000000,0.000000,0.000000}%
\pgfsetstrokecolor{currentstroke}%
\pgfsetdash{}{0pt}%
\pgfpathmoveto{\pgfqpoint{4.495422in}{4.650068in}}%
\pgfpathcurveto{\pgfqpoint{4.506318in}{4.650068in}}{\pgfqpoint{4.516769in}{4.654397in}}{\pgfqpoint{4.524473in}{4.662101in}}%
\pgfpathcurveto{\pgfqpoint{4.532177in}{4.669805in}}{\pgfqpoint{4.536506in}{4.680256in}}{\pgfqpoint{4.536506in}{4.691152in}}%
\pgfpathcurveto{\pgfqpoint{4.536506in}{4.702047in}}{\pgfqpoint{4.532177in}{4.712498in}}{\pgfqpoint{4.524473in}{4.720203in}}%
\pgfpathcurveto{\pgfqpoint{4.516769in}{4.727907in}}{\pgfqpoint{4.506318in}{4.732236in}}{\pgfqpoint{4.495422in}{4.732236in}}%
\pgfpathcurveto{\pgfqpoint{4.484527in}{4.732236in}}{\pgfqpoint{4.474076in}{4.727907in}}{\pgfqpoint{4.466371in}{4.720203in}}%
\pgfpathcurveto{\pgfqpoint{4.458667in}{4.712498in}}{\pgfqpoint{4.454338in}{4.702047in}}{\pgfqpoint{4.454338in}{4.691152in}}%
\pgfpathcurveto{\pgfqpoint{4.454338in}{4.680256in}}{\pgfqpoint{4.458667in}{4.669805in}}{\pgfqpoint{4.466371in}{4.662101in}}%
\pgfpathcurveto{\pgfqpoint{4.474076in}{4.654397in}}{\pgfqpoint{4.484527in}{4.650068in}}{\pgfqpoint{4.495422in}{4.650068in}}%
\pgfpathlineto{\pgfqpoint{4.495422in}{4.650068in}}%
\pgfpathclose%
\pgfusepath{stroke}%
\end{pgfscope}%
\begin{pgfscope}%
\pgfpathrectangle{\pgfqpoint{0.688192in}{0.670138in}}{\pgfqpoint{7.111808in}{5.061530in}}%
\pgfusepath{clip}%
\pgfsetbuttcap%
\pgfsetroundjoin%
\pgfsetlinewidth{1.003750pt}%
\definecolor{currentstroke}{rgb}{0.000000,0.000000,0.000000}%
\pgfsetstrokecolor{currentstroke}%
\pgfsetdash{}{0pt}%
\pgfpathmoveto{\pgfqpoint{1.128607in}{0.645422in}}%
\pgfpathcurveto{\pgfqpoint{1.139503in}{0.645422in}}{\pgfqpoint{1.149954in}{0.649751in}}{\pgfqpoint{1.157658in}{0.657455in}}%
\pgfpathcurveto{\pgfqpoint{1.165362in}{0.665160in}}{\pgfqpoint{1.169691in}{0.675610in}}{\pgfqpoint{1.169691in}{0.686506in}}%
\pgfpathcurveto{\pgfqpoint{1.169691in}{0.697402in}}{\pgfqpoint{1.165362in}{0.707852in}}{\pgfqpoint{1.157658in}{0.715557in}}%
\pgfpathcurveto{\pgfqpoint{1.149954in}{0.723261in}}{\pgfqpoint{1.139503in}{0.727590in}}{\pgfqpoint{1.128607in}{0.727590in}}%
\pgfpathcurveto{\pgfqpoint{1.117712in}{0.727590in}}{\pgfqpoint{1.107261in}{0.723261in}}{\pgfqpoint{1.099556in}{0.715557in}}%
\pgfpathcurveto{\pgfqpoint{1.091852in}{0.707852in}}{\pgfqpoint{1.087523in}{0.697402in}}{\pgfqpoint{1.087523in}{0.686506in}}%
\pgfpathcurveto{\pgfqpoint{1.087523in}{0.675610in}}{\pgfqpoint{1.091852in}{0.665160in}}{\pgfqpoint{1.099556in}{0.657455in}}%
\pgfpathcurveto{\pgfqpoint{1.107261in}{0.649751in}}{\pgfqpoint{1.117712in}{0.645422in}}{\pgfqpoint{1.128607in}{0.645422in}}%
\pgfusepath{stroke}%
\end{pgfscope}%
\begin{pgfscope}%
\pgfpathrectangle{\pgfqpoint{0.688192in}{0.670138in}}{\pgfqpoint{7.111808in}{5.061530in}}%
\pgfusepath{clip}%
\pgfsetbuttcap%
\pgfsetroundjoin%
\pgfsetlinewidth{1.003750pt}%
\definecolor{currentstroke}{rgb}{0.000000,0.000000,0.000000}%
\pgfsetstrokecolor{currentstroke}%
\pgfsetdash{}{0pt}%
\pgfpathmoveto{\pgfqpoint{2.429320in}{0.637441in}}%
\pgfpathcurveto{\pgfqpoint{2.440215in}{0.637441in}}{\pgfqpoint{2.450666in}{0.641770in}}{\pgfqpoint{2.458370in}{0.649474in}}%
\pgfpathcurveto{\pgfqpoint{2.466075in}{0.657178in}}{\pgfqpoint{2.470404in}{0.667629in}}{\pgfqpoint{2.470404in}{0.678525in}}%
\pgfpathcurveto{\pgfqpoint{2.470404in}{0.689420in}}{\pgfqpoint{2.466075in}{0.699871in}}{\pgfqpoint{2.458370in}{0.707575in}}%
\pgfpathcurveto{\pgfqpoint{2.450666in}{0.715280in}}{\pgfqpoint{2.440215in}{0.719609in}}{\pgfqpoint{2.429320in}{0.719609in}}%
\pgfpathcurveto{\pgfqpoint{2.418424in}{0.719609in}}{\pgfqpoint{2.407973in}{0.715280in}}{\pgfqpoint{2.400269in}{0.707575in}}%
\pgfpathcurveto{\pgfqpoint{2.392565in}{0.699871in}}{\pgfqpoint{2.388236in}{0.689420in}}{\pgfqpoint{2.388236in}{0.678525in}}%
\pgfpathcurveto{\pgfqpoint{2.388236in}{0.667629in}}{\pgfqpoint{2.392565in}{0.657178in}}{\pgfqpoint{2.400269in}{0.649474in}}%
\pgfpathcurveto{\pgfqpoint{2.407973in}{0.641770in}}{\pgfqpoint{2.418424in}{0.637441in}}{\pgfqpoint{2.429320in}{0.637441in}}%
\pgfusepath{stroke}%
\end{pgfscope}%
\begin{pgfscope}%
\pgfpathrectangle{\pgfqpoint{0.688192in}{0.670138in}}{\pgfqpoint{7.111808in}{5.061530in}}%
\pgfusepath{clip}%
\pgfsetbuttcap%
\pgfsetroundjoin%
\pgfsetlinewidth{1.003750pt}%
\definecolor{currentstroke}{rgb}{0.000000,0.000000,0.000000}%
\pgfsetstrokecolor{currentstroke}%
\pgfsetdash{}{0pt}%
\pgfpathmoveto{\pgfqpoint{1.495917in}{0.642463in}}%
\pgfpathcurveto{\pgfqpoint{1.506813in}{0.642463in}}{\pgfqpoint{1.517263in}{0.646792in}}{\pgfqpoint{1.524968in}{0.654497in}}%
\pgfpathcurveto{\pgfqpoint{1.532672in}{0.662201in}}{\pgfqpoint{1.537001in}{0.672652in}}{\pgfqpoint{1.537001in}{0.683547in}}%
\pgfpathcurveto{\pgfqpoint{1.537001in}{0.694443in}}{\pgfqpoint{1.532672in}{0.704894in}}{\pgfqpoint{1.524968in}{0.712598in}}%
\pgfpathcurveto{\pgfqpoint{1.517263in}{0.720302in}}{\pgfqpoint{1.506813in}{0.724631in}}{\pgfqpoint{1.495917in}{0.724631in}}%
\pgfpathcurveto{\pgfqpoint{1.485021in}{0.724631in}}{\pgfqpoint{1.474571in}{0.720302in}}{\pgfqpoint{1.466866in}{0.712598in}}%
\pgfpathcurveto{\pgfqpoint{1.459162in}{0.704894in}}{\pgfqpoint{1.454833in}{0.694443in}}{\pgfqpoint{1.454833in}{0.683547in}}%
\pgfpathcurveto{\pgfqpoint{1.454833in}{0.672652in}}{\pgfqpoint{1.459162in}{0.662201in}}{\pgfqpoint{1.466866in}{0.654497in}}%
\pgfpathcurveto{\pgfqpoint{1.474571in}{0.646792in}}{\pgfqpoint{1.485021in}{0.642463in}}{\pgfqpoint{1.495917in}{0.642463in}}%
\pgfusepath{stroke}%
\end{pgfscope}%
\begin{pgfscope}%
\pgfpathrectangle{\pgfqpoint{0.688192in}{0.670138in}}{\pgfqpoint{7.111808in}{5.061530in}}%
\pgfusepath{clip}%
\pgfsetbuttcap%
\pgfsetroundjoin%
\pgfsetlinewidth{1.003750pt}%
\definecolor{currentstroke}{rgb}{0.000000,0.000000,0.000000}%
\pgfsetstrokecolor{currentstroke}%
\pgfsetdash{}{0pt}%
\pgfpathmoveto{\pgfqpoint{3.863667in}{0.632041in}}%
\pgfpathcurveto{\pgfqpoint{3.874562in}{0.632041in}}{\pgfqpoint{3.885013in}{0.636370in}}{\pgfqpoint{3.892717in}{0.644075in}}%
\pgfpathcurveto{\pgfqpoint{3.900422in}{0.651779in}}{\pgfqpoint{3.904751in}{0.662230in}}{\pgfqpoint{3.904751in}{0.673125in}}%
\pgfpathcurveto{\pgfqpoint{3.904751in}{0.684021in}}{\pgfqpoint{3.900422in}{0.694472in}}{\pgfqpoint{3.892717in}{0.702176in}}%
\pgfpathcurveto{\pgfqpoint{3.885013in}{0.709880in}}{\pgfqpoint{3.874562in}{0.714209in}}{\pgfqpoint{3.863667in}{0.714209in}}%
\pgfpathcurveto{\pgfqpoint{3.852771in}{0.714209in}}{\pgfqpoint{3.842320in}{0.709880in}}{\pgfqpoint{3.834616in}{0.702176in}}%
\pgfpathcurveto{\pgfqpoint{3.826912in}{0.694472in}}{\pgfqpoint{3.822583in}{0.684021in}}{\pgfqpoint{3.822583in}{0.673125in}}%
\pgfpathcurveto{\pgfqpoint{3.822583in}{0.662230in}}{\pgfqpoint{3.826912in}{0.651779in}}{\pgfqpoint{3.834616in}{0.644075in}}%
\pgfpathcurveto{\pgfqpoint{3.842320in}{0.636370in}}{\pgfqpoint{3.852771in}{0.632041in}}{\pgfqpoint{3.863667in}{0.632041in}}%
\pgfusepath{stroke}%
\end{pgfscope}%
\begin{pgfscope}%
\pgfpathrectangle{\pgfqpoint{0.688192in}{0.670138in}}{\pgfqpoint{7.111808in}{5.061530in}}%
\pgfusepath{clip}%
\pgfsetbuttcap%
\pgfsetroundjoin%
\pgfsetlinewidth{1.003750pt}%
\definecolor{currentstroke}{rgb}{0.000000,0.000000,0.000000}%
\pgfsetstrokecolor{currentstroke}%
\pgfsetdash{}{0pt}%
\pgfpathmoveto{\pgfqpoint{4.885160in}{1.792974in}}%
\pgfpathcurveto{\pgfqpoint{4.896055in}{1.792974in}}{\pgfqpoint{4.906506in}{1.797303in}}{\pgfqpoint{4.914210in}{1.805007in}}%
\pgfpathcurveto{\pgfqpoint{4.921915in}{1.812711in}}{\pgfqpoint{4.926244in}{1.823162in}}{\pgfqpoint{4.926244in}{1.834058in}}%
\pgfpathcurveto{\pgfqpoint{4.926244in}{1.844953in}}{\pgfqpoint{4.921915in}{1.855404in}}{\pgfqpoint{4.914210in}{1.863108in}}%
\pgfpathcurveto{\pgfqpoint{4.906506in}{1.870813in}}{\pgfqpoint{4.896055in}{1.875142in}}{\pgfqpoint{4.885160in}{1.875142in}}%
\pgfpathcurveto{\pgfqpoint{4.874264in}{1.875142in}}{\pgfqpoint{4.863813in}{1.870813in}}{\pgfqpoint{4.856109in}{1.863108in}}%
\pgfpathcurveto{\pgfqpoint{4.848405in}{1.855404in}}{\pgfqpoint{4.844076in}{1.844953in}}{\pgfqpoint{4.844076in}{1.834058in}}%
\pgfpathcurveto{\pgfqpoint{4.844076in}{1.823162in}}{\pgfqpoint{4.848405in}{1.812711in}}{\pgfqpoint{4.856109in}{1.805007in}}%
\pgfpathcurveto{\pgfqpoint{4.863813in}{1.797303in}}{\pgfqpoint{4.874264in}{1.792974in}}{\pgfqpoint{4.885160in}{1.792974in}}%
\pgfpathlineto{\pgfqpoint{4.885160in}{1.792974in}}%
\pgfpathclose%
\pgfusepath{stroke}%
\end{pgfscope}%
\begin{pgfscope}%
\pgfpathrectangle{\pgfqpoint{0.688192in}{0.670138in}}{\pgfqpoint{7.111808in}{5.061530in}}%
\pgfusepath{clip}%
\pgfsetbuttcap%
\pgfsetroundjoin%
\pgfsetlinewidth{1.003750pt}%
\definecolor{currentstroke}{rgb}{0.000000,0.000000,0.000000}%
\pgfsetstrokecolor{currentstroke}%
\pgfsetdash{}{0pt}%
\pgfpathmoveto{\pgfqpoint{2.225009in}{0.638936in}}%
\pgfpathcurveto{\pgfqpoint{2.235905in}{0.638936in}}{\pgfqpoint{2.246356in}{0.643265in}}{\pgfqpoint{2.254060in}{0.650969in}}%
\pgfpathcurveto{\pgfqpoint{2.261764in}{0.658674in}}{\pgfqpoint{2.266093in}{0.669125in}}{\pgfqpoint{2.266093in}{0.680020in}}%
\pgfpathcurveto{\pgfqpoint{2.266093in}{0.690916in}}{\pgfqpoint{2.261764in}{0.701366in}}{\pgfqpoint{2.254060in}{0.709071in}}%
\pgfpathcurveto{\pgfqpoint{2.246356in}{0.716775in}}{\pgfqpoint{2.235905in}{0.721104in}}{\pgfqpoint{2.225009in}{0.721104in}}%
\pgfpathcurveto{\pgfqpoint{2.214114in}{0.721104in}}{\pgfqpoint{2.203663in}{0.716775in}}{\pgfqpoint{2.195959in}{0.709071in}}%
\pgfpathcurveto{\pgfqpoint{2.188254in}{0.701366in}}{\pgfqpoint{2.183926in}{0.690916in}}{\pgfqpoint{2.183926in}{0.680020in}}%
\pgfpathcurveto{\pgfqpoint{2.183926in}{0.669125in}}{\pgfqpoint{2.188254in}{0.658674in}}{\pgfqpoint{2.195959in}{0.650969in}}%
\pgfpathcurveto{\pgfqpoint{2.203663in}{0.643265in}}{\pgfqpoint{2.214114in}{0.638936in}}{\pgfqpoint{2.225009in}{0.638936in}}%
\pgfusepath{stroke}%
\end{pgfscope}%
\begin{pgfscope}%
\pgfpathrectangle{\pgfqpoint{0.688192in}{0.670138in}}{\pgfqpoint{7.111808in}{5.061530in}}%
\pgfusepath{clip}%
\pgfsetbuttcap%
\pgfsetroundjoin%
\pgfsetlinewidth{1.003750pt}%
\definecolor{currentstroke}{rgb}{0.000000,0.000000,0.000000}%
\pgfsetstrokecolor{currentstroke}%
\pgfsetdash{}{0pt}%
\pgfpathmoveto{\pgfqpoint{5.252788in}{2.395935in}}%
\pgfpathcurveto{\pgfqpoint{5.263684in}{2.395935in}}{\pgfqpoint{5.274134in}{2.400263in}}{\pgfqpoint{5.281839in}{2.407968in}}%
\pgfpathcurveto{\pgfqpoint{5.289543in}{2.415672in}}{\pgfqpoint{5.293872in}{2.426123in}}{\pgfqpoint{5.293872in}{2.437018in}}%
\pgfpathcurveto{\pgfqpoint{5.293872in}{2.447914in}}{\pgfqpoint{5.289543in}{2.458365in}}{\pgfqpoint{5.281839in}{2.466069in}}%
\pgfpathcurveto{\pgfqpoint{5.274134in}{2.473774in}}{\pgfqpoint{5.263684in}{2.478102in}}{\pgfqpoint{5.252788in}{2.478102in}}%
\pgfpathcurveto{\pgfqpoint{5.241893in}{2.478102in}}{\pgfqpoint{5.231442in}{2.473774in}}{\pgfqpoint{5.223737in}{2.466069in}}%
\pgfpathcurveto{\pgfqpoint{5.216033in}{2.458365in}}{\pgfqpoint{5.211704in}{2.447914in}}{\pgfqpoint{5.211704in}{2.437018in}}%
\pgfpathcurveto{\pgfqpoint{5.211704in}{2.426123in}}{\pgfqpoint{5.216033in}{2.415672in}}{\pgfqpoint{5.223737in}{2.407968in}}%
\pgfpathcurveto{\pgfqpoint{5.231442in}{2.400263in}}{\pgfqpoint{5.241893in}{2.395935in}}{\pgfqpoint{5.252788in}{2.395935in}}%
\pgfpathlineto{\pgfqpoint{5.252788in}{2.395935in}}%
\pgfpathclose%
\pgfusepath{stroke}%
\end{pgfscope}%
\begin{pgfscope}%
\pgfpathrectangle{\pgfqpoint{0.688192in}{0.670138in}}{\pgfqpoint{7.111808in}{5.061530in}}%
\pgfusepath{clip}%
\pgfsetbuttcap%
\pgfsetroundjoin%
\pgfsetlinewidth{1.003750pt}%
\definecolor{currentstroke}{rgb}{0.000000,0.000000,0.000000}%
\pgfsetstrokecolor{currentstroke}%
\pgfsetdash{}{0pt}%
\pgfpathmoveto{\pgfqpoint{1.363285in}{0.643335in}}%
\pgfpathcurveto{\pgfqpoint{1.374180in}{0.643335in}}{\pgfqpoint{1.384631in}{0.647663in}}{\pgfqpoint{1.392336in}{0.655368in}}%
\pgfpathcurveto{\pgfqpoint{1.400040in}{0.663072in}}{\pgfqpoint{1.404369in}{0.673523in}}{\pgfqpoint{1.404369in}{0.684419in}}%
\pgfpathcurveto{\pgfqpoint{1.404369in}{0.695314in}}{\pgfqpoint{1.400040in}{0.705765in}}{\pgfqpoint{1.392336in}{0.713469in}}%
\pgfpathcurveto{\pgfqpoint{1.384631in}{0.721174in}}{\pgfqpoint{1.374180in}{0.725502in}}{\pgfqpoint{1.363285in}{0.725502in}}%
\pgfpathcurveto{\pgfqpoint{1.352389in}{0.725502in}}{\pgfqpoint{1.341938in}{0.721174in}}{\pgfqpoint{1.334234in}{0.713469in}}%
\pgfpathcurveto{\pgfqpoint{1.326530in}{0.705765in}}{\pgfqpoint{1.322201in}{0.695314in}}{\pgfqpoint{1.322201in}{0.684419in}}%
\pgfpathcurveto{\pgfqpoint{1.322201in}{0.673523in}}{\pgfqpoint{1.326530in}{0.663072in}}{\pgfqpoint{1.334234in}{0.655368in}}%
\pgfpathcurveto{\pgfqpoint{1.341938in}{0.647663in}}{\pgfqpoint{1.352389in}{0.643335in}}{\pgfqpoint{1.363285in}{0.643335in}}%
\pgfusepath{stroke}%
\end{pgfscope}%
\begin{pgfscope}%
\pgfpathrectangle{\pgfqpoint{0.688192in}{0.670138in}}{\pgfqpoint{7.111808in}{5.061530in}}%
\pgfusepath{clip}%
\pgfsetbuttcap%
\pgfsetroundjoin%
\pgfsetlinewidth{1.003750pt}%
\definecolor{currentstroke}{rgb}{0.000000,0.000000,0.000000}%
\pgfsetstrokecolor{currentstroke}%
\pgfsetdash{}{0pt}%
\pgfpathmoveto{\pgfqpoint{2.745823in}{2.382906in}}%
\pgfpathcurveto{\pgfqpoint{2.756718in}{2.382906in}}{\pgfqpoint{2.767169in}{2.387235in}}{\pgfqpoint{2.774874in}{2.394939in}}%
\pgfpathcurveto{\pgfqpoint{2.782578in}{2.402644in}}{\pgfqpoint{2.786907in}{2.413095in}}{\pgfqpoint{2.786907in}{2.423990in}}%
\pgfpathcurveto{\pgfqpoint{2.786907in}{2.434886in}}{\pgfqpoint{2.782578in}{2.445337in}}{\pgfqpoint{2.774874in}{2.453041in}}%
\pgfpathcurveto{\pgfqpoint{2.767169in}{2.460745in}}{\pgfqpoint{2.756718in}{2.465074in}}{\pgfqpoint{2.745823in}{2.465074in}}%
\pgfpathcurveto{\pgfqpoint{2.734927in}{2.465074in}}{\pgfqpoint{2.724477in}{2.460745in}}{\pgfqpoint{2.716772in}{2.453041in}}%
\pgfpathcurveto{\pgfqpoint{2.709068in}{2.445337in}}{\pgfqpoint{2.704739in}{2.434886in}}{\pgfqpoint{2.704739in}{2.423990in}}%
\pgfpathcurveto{\pgfqpoint{2.704739in}{2.413095in}}{\pgfqpoint{2.709068in}{2.402644in}}{\pgfqpoint{2.716772in}{2.394939in}}%
\pgfpathcurveto{\pgfqpoint{2.724477in}{2.387235in}}{\pgfqpoint{2.734927in}{2.382906in}}{\pgfqpoint{2.745823in}{2.382906in}}%
\pgfpathlineto{\pgfqpoint{2.745823in}{2.382906in}}%
\pgfpathclose%
\pgfusepath{stroke}%
\end{pgfscope}%
\begin{pgfscope}%
\pgfpathrectangle{\pgfqpoint{0.688192in}{0.670138in}}{\pgfqpoint{7.111808in}{5.061530in}}%
\pgfusepath{clip}%
\pgfsetbuttcap%
\pgfsetroundjoin%
\pgfsetlinewidth{1.003750pt}%
\definecolor{currentstroke}{rgb}{0.000000,0.000000,0.000000}%
\pgfsetstrokecolor{currentstroke}%
\pgfsetdash{}{0pt}%
\pgfpathmoveto{\pgfqpoint{2.056022in}{1.075198in}}%
\pgfpathcurveto{\pgfqpoint{2.066917in}{1.075198in}}{\pgfqpoint{2.077368in}{1.079527in}}{\pgfqpoint{2.085072in}{1.087231in}}%
\pgfpathcurveto{\pgfqpoint{2.092777in}{1.094936in}}{\pgfqpoint{2.097106in}{1.105386in}}{\pgfqpoint{2.097106in}{1.116282in}}%
\pgfpathcurveto{\pgfqpoint{2.097106in}{1.127178in}}{\pgfqpoint{2.092777in}{1.137628in}}{\pgfqpoint{2.085072in}{1.145333in}}%
\pgfpathcurveto{\pgfqpoint{2.077368in}{1.153037in}}{\pgfqpoint{2.066917in}{1.157366in}}{\pgfqpoint{2.056022in}{1.157366in}}%
\pgfpathcurveto{\pgfqpoint{2.045126in}{1.157366in}}{\pgfqpoint{2.034675in}{1.153037in}}{\pgfqpoint{2.026971in}{1.145333in}}%
\pgfpathcurveto{\pgfqpoint{2.019267in}{1.137628in}}{\pgfqpoint{2.014938in}{1.127178in}}{\pgfqpoint{2.014938in}{1.116282in}}%
\pgfpathcurveto{\pgfqpoint{2.014938in}{1.105386in}}{\pgfqpoint{2.019267in}{1.094936in}}{\pgfqpoint{2.026971in}{1.087231in}}%
\pgfpathcurveto{\pgfqpoint{2.034675in}{1.079527in}}{\pgfqpoint{2.045126in}{1.075198in}}{\pgfqpoint{2.056022in}{1.075198in}}%
\pgfpathlineto{\pgfqpoint{2.056022in}{1.075198in}}%
\pgfpathclose%
\pgfusepath{stroke}%
\end{pgfscope}%
\begin{pgfscope}%
\pgfpathrectangle{\pgfqpoint{0.688192in}{0.670138in}}{\pgfqpoint{7.111808in}{5.061530in}}%
\pgfusepath{clip}%
\pgfsetbuttcap%
\pgfsetroundjoin%
\pgfsetlinewidth{1.003750pt}%
\definecolor{currentstroke}{rgb}{0.000000,0.000000,0.000000}%
\pgfsetstrokecolor{currentstroke}%
\pgfsetdash{}{0pt}%
\pgfpathmoveto{\pgfqpoint{1.495917in}{0.642463in}}%
\pgfpathcurveto{\pgfqpoint{1.506813in}{0.642463in}}{\pgfqpoint{1.517263in}{0.646792in}}{\pgfqpoint{1.524968in}{0.654497in}}%
\pgfpathcurveto{\pgfqpoint{1.532672in}{0.662201in}}{\pgfqpoint{1.537001in}{0.672652in}}{\pgfqpoint{1.537001in}{0.683547in}}%
\pgfpathcurveto{\pgfqpoint{1.537001in}{0.694443in}}{\pgfqpoint{1.532672in}{0.704894in}}{\pgfqpoint{1.524968in}{0.712598in}}%
\pgfpathcurveto{\pgfqpoint{1.517263in}{0.720302in}}{\pgfqpoint{1.506813in}{0.724631in}}{\pgfqpoint{1.495917in}{0.724631in}}%
\pgfpathcurveto{\pgfqpoint{1.485021in}{0.724631in}}{\pgfqpoint{1.474571in}{0.720302in}}{\pgfqpoint{1.466866in}{0.712598in}}%
\pgfpathcurveto{\pgfqpoint{1.459162in}{0.704894in}}{\pgfqpoint{1.454833in}{0.694443in}}{\pgfqpoint{1.454833in}{0.683547in}}%
\pgfpathcurveto{\pgfqpoint{1.454833in}{0.672652in}}{\pgfqpoint{1.459162in}{0.662201in}}{\pgfqpoint{1.466866in}{0.654497in}}%
\pgfpathcurveto{\pgfqpoint{1.474571in}{0.646792in}}{\pgfqpoint{1.485021in}{0.642463in}}{\pgfqpoint{1.495917in}{0.642463in}}%
\pgfusepath{stroke}%
\end{pgfscope}%
\begin{pgfscope}%
\pgfpathrectangle{\pgfqpoint{0.688192in}{0.670138in}}{\pgfqpoint{7.111808in}{5.061530in}}%
\pgfusepath{clip}%
\pgfsetbuttcap%
\pgfsetroundjoin%
\pgfsetlinewidth{1.003750pt}%
\definecolor{currentstroke}{rgb}{0.000000,0.000000,0.000000}%
\pgfsetstrokecolor{currentstroke}%
\pgfsetdash{}{0pt}%
\pgfpathmoveto{\pgfqpoint{5.992705in}{3.175063in}}%
\pgfpathcurveto{\pgfqpoint{6.003601in}{3.175063in}}{\pgfqpoint{6.014052in}{3.179392in}}{\pgfqpoint{6.021756in}{3.187096in}}%
\pgfpathcurveto{\pgfqpoint{6.029460in}{3.194800in}}{\pgfqpoint{6.033789in}{3.205251in}}{\pgfqpoint{6.033789in}{3.216147in}}%
\pgfpathcurveto{\pgfqpoint{6.033789in}{3.227042in}}{\pgfqpoint{6.029460in}{3.237493in}}{\pgfqpoint{6.021756in}{3.245197in}}%
\pgfpathcurveto{\pgfqpoint{6.014052in}{3.252902in}}{\pgfqpoint{6.003601in}{3.257230in}}{\pgfqpoint{5.992705in}{3.257230in}}%
\pgfpathcurveto{\pgfqpoint{5.981810in}{3.257230in}}{\pgfqpoint{5.971359in}{3.252902in}}{\pgfqpoint{5.963654in}{3.245197in}}%
\pgfpathcurveto{\pgfqpoint{5.955950in}{3.237493in}}{\pgfqpoint{5.951621in}{3.227042in}}{\pgfqpoint{5.951621in}{3.216147in}}%
\pgfpathcurveto{\pgfqpoint{5.951621in}{3.205251in}}{\pgfqpoint{5.955950in}{3.194800in}}{\pgfqpoint{5.963654in}{3.187096in}}%
\pgfpathcurveto{\pgfqpoint{5.971359in}{3.179392in}}{\pgfqpoint{5.981810in}{3.175063in}}{\pgfqpoint{5.992705in}{3.175063in}}%
\pgfpathlineto{\pgfqpoint{5.992705in}{3.175063in}}%
\pgfpathclose%
\pgfusepath{stroke}%
\end{pgfscope}%
\begin{pgfscope}%
\pgfpathrectangle{\pgfqpoint{0.688192in}{0.670138in}}{\pgfqpoint{7.111808in}{5.061530in}}%
\pgfusepath{clip}%
\pgfsetbuttcap%
\pgfsetroundjoin%
\pgfsetlinewidth{1.003750pt}%
\definecolor{currentstroke}{rgb}{0.000000,0.000000,0.000000}%
\pgfsetstrokecolor{currentstroke}%
\pgfsetdash{}{0pt}%
\pgfpathmoveto{\pgfqpoint{1.108579in}{0.647776in}}%
\pgfpathcurveto{\pgfqpoint{1.119474in}{0.647776in}}{\pgfqpoint{1.129925in}{0.652105in}}{\pgfqpoint{1.137629in}{0.659809in}}%
\pgfpathcurveto{\pgfqpoint{1.145334in}{0.667513in}}{\pgfqpoint{1.149662in}{0.677964in}}{\pgfqpoint{1.149662in}{0.688860in}}%
\pgfpathcurveto{\pgfqpoint{1.149662in}{0.699755in}}{\pgfqpoint{1.145334in}{0.710206in}}{\pgfqpoint{1.137629in}{0.717910in}}%
\pgfpathcurveto{\pgfqpoint{1.129925in}{0.725615in}}{\pgfqpoint{1.119474in}{0.729944in}}{\pgfqpoint{1.108579in}{0.729944in}}%
\pgfpathcurveto{\pgfqpoint{1.097683in}{0.729944in}}{\pgfqpoint{1.087232in}{0.725615in}}{\pgfqpoint{1.079528in}{0.717910in}}%
\pgfpathcurveto{\pgfqpoint{1.071823in}{0.710206in}}{\pgfqpoint{1.067495in}{0.699755in}}{\pgfqpoint{1.067495in}{0.688860in}}%
\pgfpathcurveto{\pgfqpoint{1.067495in}{0.677964in}}{\pgfqpoint{1.071823in}{0.667513in}}{\pgfqpoint{1.079528in}{0.659809in}}%
\pgfpathcurveto{\pgfqpoint{1.087232in}{0.652105in}}{\pgfqpoint{1.097683in}{0.647776in}}{\pgfqpoint{1.108579in}{0.647776in}}%
\pgfusepath{stroke}%
\end{pgfscope}%
\begin{pgfscope}%
\pgfpathrectangle{\pgfqpoint{0.688192in}{0.670138in}}{\pgfqpoint{7.111808in}{5.061530in}}%
\pgfusepath{clip}%
\pgfsetbuttcap%
\pgfsetroundjoin%
\pgfsetlinewidth{1.003750pt}%
\definecolor{currentstroke}{rgb}{0.000000,0.000000,0.000000}%
\pgfsetstrokecolor{currentstroke}%
\pgfsetdash{}{0pt}%
\pgfpathmoveto{\pgfqpoint{2.911373in}{0.635871in}}%
\pgfpathcurveto{\pgfqpoint{2.922269in}{0.635871in}}{\pgfqpoint{2.932719in}{0.640200in}}{\pgfqpoint{2.940424in}{0.647905in}}%
\pgfpathcurveto{\pgfqpoint{2.948128in}{0.655609in}}{\pgfqpoint{2.952457in}{0.666060in}}{\pgfqpoint{2.952457in}{0.676955in}}%
\pgfpathcurveto{\pgfqpoint{2.952457in}{0.687851in}}{\pgfqpoint{2.948128in}{0.698302in}}{\pgfqpoint{2.940424in}{0.706006in}}%
\pgfpathcurveto{\pgfqpoint{2.932719in}{0.713710in}}{\pgfqpoint{2.922269in}{0.718039in}}{\pgfqpoint{2.911373in}{0.718039in}}%
\pgfpathcurveto{\pgfqpoint{2.900477in}{0.718039in}}{\pgfqpoint{2.890027in}{0.713710in}}{\pgfqpoint{2.882322in}{0.706006in}}%
\pgfpathcurveto{\pgfqpoint{2.874618in}{0.698302in}}{\pgfqpoint{2.870289in}{0.687851in}}{\pgfqpoint{2.870289in}{0.676955in}}%
\pgfpathcurveto{\pgfqpoint{2.870289in}{0.666060in}}{\pgfqpoint{2.874618in}{0.655609in}}{\pgfqpoint{2.882322in}{0.647905in}}%
\pgfpathcurveto{\pgfqpoint{2.890027in}{0.640200in}}{\pgfqpoint{2.900477in}{0.635871in}}{\pgfqpoint{2.911373in}{0.635871in}}%
\pgfusepath{stroke}%
\end{pgfscope}%
\begin{pgfscope}%
\pgfpathrectangle{\pgfqpoint{0.688192in}{0.670138in}}{\pgfqpoint{7.111808in}{5.061530in}}%
\pgfusepath{clip}%
\pgfsetbuttcap%
\pgfsetroundjoin%
\pgfsetlinewidth{1.003750pt}%
\definecolor{currentstroke}{rgb}{0.000000,0.000000,0.000000}%
\pgfsetstrokecolor{currentstroke}%
\pgfsetdash{}{0pt}%
\pgfpathmoveto{\pgfqpoint{1.892079in}{0.640382in}}%
\pgfpathcurveto{\pgfqpoint{1.902974in}{0.640382in}}{\pgfqpoint{1.913425in}{0.644711in}}{\pgfqpoint{1.921129in}{0.652415in}}%
\pgfpathcurveto{\pgfqpoint{1.928834in}{0.660120in}}{\pgfqpoint{1.933163in}{0.670570in}}{\pgfqpoint{1.933163in}{0.681466in}}%
\pgfpathcurveto{\pgfqpoint{1.933163in}{0.692361in}}{\pgfqpoint{1.928834in}{0.702812in}}{\pgfqpoint{1.921129in}{0.710517in}}%
\pgfpathcurveto{\pgfqpoint{1.913425in}{0.718221in}}{\pgfqpoint{1.902974in}{0.722550in}}{\pgfqpoint{1.892079in}{0.722550in}}%
\pgfpathcurveto{\pgfqpoint{1.881183in}{0.722550in}}{\pgfqpoint{1.870732in}{0.718221in}}{\pgfqpoint{1.863028in}{0.710517in}}%
\pgfpathcurveto{\pgfqpoint{1.855324in}{0.702812in}}{\pgfqpoint{1.850995in}{0.692361in}}{\pgfqpoint{1.850995in}{0.681466in}}%
\pgfpathcurveto{\pgfqpoint{1.850995in}{0.670570in}}{\pgfqpoint{1.855324in}{0.660120in}}{\pgfqpoint{1.863028in}{0.652415in}}%
\pgfpathcurveto{\pgfqpoint{1.870732in}{0.644711in}}{\pgfqpoint{1.881183in}{0.640382in}}{\pgfqpoint{1.892079in}{0.640382in}}%
\pgfusepath{stroke}%
\end{pgfscope}%
\begin{pgfscope}%
\pgfpathrectangle{\pgfqpoint{0.688192in}{0.670138in}}{\pgfqpoint{7.111808in}{5.061530in}}%
\pgfusepath{clip}%
\pgfsetbuttcap%
\pgfsetroundjoin%
\pgfsetlinewidth{1.003750pt}%
\definecolor{currentstroke}{rgb}{0.000000,0.000000,0.000000}%
\pgfsetstrokecolor{currentstroke}%
\pgfsetdash{}{0pt}%
\pgfpathmoveto{\pgfqpoint{7.530702in}{0.740336in}}%
\pgfpathcurveto{\pgfqpoint{7.541598in}{0.740336in}}{\pgfqpoint{7.552049in}{0.744665in}}{\pgfqpoint{7.559753in}{0.752369in}}%
\pgfpathcurveto{\pgfqpoint{7.567457in}{0.760074in}}{\pgfqpoint{7.571786in}{0.770524in}}{\pgfqpoint{7.571786in}{0.781420in}}%
\pgfpathcurveto{\pgfqpoint{7.571786in}{0.792316in}}{\pgfqpoint{7.567457in}{0.802766in}}{\pgfqpoint{7.559753in}{0.810471in}}%
\pgfpathcurveto{\pgfqpoint{7.552049in}{0.818175in}}{\pgfqpoint{7.541598in}{0.822504in}}{\pgfqpoint{7.530702in}{0.822504in}}%
\pgfpathcurveto{\pgfqpoint{7.519807in}{0.822504in}}{\pgfqpoint{7.509356in}{0.818175in}}{\pgfqpoint{7.501652in}{0.810471in}}%
\pgfpathcurveto{\pgfqpoint{7.493947in}{0.802766in}}{\pgfqpoint{7.489618in}{0.792316in}}{\pgfqpoint{7.489618in}{0.781420in}}%
\pgfpathcurveto{\pgfqpoint{7.489618in}{0.770524in}}{\pgfqpoint{7.493947in}{0.760074in}}{\pgfqpoint{7.501652in}{0.752369in}}%
\pgfpathcurveto{\pgfqpoint{7.509356in}{0.744665in}}{\pgfqpoint{7.519807in}{0.740336in}}{\pgfqpoint{7.530702in}{0.740336in}}%
\pgfpathlineto{\pgfqpoint{7.530702in}{0.740336in}}%
\pgfpathclose%
\pgfusepath{stroke}%
\end{pgfscope}%
\begin{pgfscope}%
\pgfpathrectangle{\pgfqpoint{0.688192in}{0.670138in}}{\pgfqpoint{7.111808in}{5.061530in}}%
\pgfusepath{clip}%
\pgfsetbuttcap%
\pgfsetroundjoin%
\pgfsetlinewidth{1.003750pt}%
\definecolor{currentstroke}{rgb}{0.000000,0.000000,0.000000}%
\pgfsetstrokecolor{currentstroke}%
\pgfsetdash{}{0pt}%
\pgfpathmoveto{\pgfqpoint{5.260356in}{1.389874in}}%
\pgfpathcurveto{\pgfqpoint{5.271251in}{1.389874in}}{\pgfqpoint{5.281702in}{1.394203in}}{\pgfqpoint{5.289407in}{1.401907in}}%
\pgfpathcurveto{\pgfqpoint{5.297111in}{1.409611in}}{\pgfqpoint{5.301440in}{1.420062in}}{\pgfqpoint{5.301440in}{1.430958in}}%
\pgfpathcurveto{\pgfqpoint{5.301440in}{1.441853in}}{\pgfqpoint{5.297111in}{1.452304in}}{\pgfqpoint{5.289407in}{1.460008in}}%
\pgfpathcurveto{\pgfqpoint{5.281702in}{1.467713in}}{\pgfqpoint{5.271251in}{1.472042in}}{\pgfqpoint{5.260356in}{1.472042in}}%
\pgfpathcurveto{\pgfqpoint{5.249460in}{1.472042in}}{\pgfqpoint{5.239010in}{1.467713in}}{\pgfqpoint{5.231305in}{1.460008in}}%
\pgfpathcurveto{\pgfqpoint{5.223601in}{1.452304in}}{\pgfqpoint{5.219272in}{1.441853in}}{\pgfqpoint{5.219272in}{1.430958in}}%
\pgfpathcurveto{\pgfqpoint{5.219272in}{1.420062in}}{\pgfqpoint{5.223601in}{1.409611in}}{\pgfqpoint{5.231305in}{1.401907in}}%
\pgfpathcurveto{\pgfqpoint{5.239010in}{1.394203in}}{\pgfqpoint{5.249460in}{1.389874in}}{\pgfqpoint{5.260356in}{1.389874in}}%
\pgfpathlineto{\pgfqpoint{5.260356in}{1.389874in}}%
\pgfpathclose%
\pgfusepath{stroke}%
\end{pgfscope}%
\begin{pgfscope}%
\pgfpathrectangle{\pgfqpoint{0.688192in}{0.670138in}}{\pgfqpoint{7.111808in}{5.061530in}}%
\pgfusepath{clip}%
\pgfsetbuttcap%
\pgfsetroundjoin%
\pgfsetlinewidth{1.003750pt}%
\definecolor{currentstroke}{rgb}{0.000000,0.000000,0.000000}%
\pgfsetstrokecolor{currentstroke}%
\pgfsetdash{}{0pt}%
\pgfpathmoveto{\pgfqpoint{1.732115in}{0.641446in}}%
\pgfpathcurveto{\pgfqpoint{1.743010in}{0.641446in}}{\pgfqpoint{1.753461in}{0.645774in}}{\pgfqpoint{1.761165in}{0.653479in}}%
\pgfpathcurveto{\pgfqpoint{1.768870in}{0.661183in}}{\pgfqpoint{1.773199in}{0.671634in}}{\pgfqpoint{1.773199in}{0.682529in}}%
\pgfpathcurveto{\pgfqpoint{1.773199in}{0.693425in}}{\pgfqpoint{1.768870in}{0.703876in}}{\pgfqpoint{1.761165in}{0.711580in}}%
\pgfpathcurveto{\pgfqpoint{1.753461in}{0.719285in}}{\pgfqpoint{1.743010in}{0.723613in}}{\pgfqpoint{1.732115in}{0.723613in}}%
\pgfpathcurveto{\pgfqpoint{1.721219in}{0.723613in}}{\pgfqpoint{1.710768in}{0.719285in}}{\pgfqpoint{1.703064in}{0.711580in}}%
\pgfpathcurveto{\pgfqpoint{1.695360in}{0.703876in}}{\pgfqpoint{1.691031in}{0.693425in}}{\pgfqpoint{1.691031in}{0.682529in}}%
\pgfpathcurveto{\pgfqpoint{1.691031in}{0.671634in}}{\pgfqpoint{1.695360in}{0.661183in}}{\pgfqpoint{1.703064in}{0.653479in}}%
\pgfpathcurveto{\pgfqpoint{1.710768in}{0.645774in}}{\pgfqpoint{1.721219in}{0.641446in}}{\pgfqpoint{1.732115in}{0.641446in}}%
\pgfusepath{stroke}%
\end{pgfscope}%
\begin{pgfscope}%
\pgfpathrectangle{\pgfqpoint{0.688192in}{0.670138in}}{\pgfqpoint{7.111808in}{5.061530in}}%
\pgfusepath{clip}%
\pgfsetbuttcap%
\pgfsetroundjoin%
\pgfsetlinewidth{1.003750pt}%
\definecolor{currentstroke}{rgb}{0.000000,0.000000,0.000000}%
\pgfsetstrokecolor{currentstroke}%
\pgfsetdash{}{0pt}%
\pgfpathmoveto{\pgfqpoint{1.399923in}{0.643066in}}%
\pgfpathcurveto{\pgfqpoint{1.410818in}{0.643066in}}{\pgfqpoint{1.421269in}{0.647395in}}{\pgfqpoint{1.428973in}{0.655099in}}%
\pgfpathcurveto{\pgfqpoint{1.436678in}{0.662804in}}{\pgfqpoint{1.441006in}{0.673254in}}{\pgfqpoint{1.441006in}{0.684150in}}%
\pgfpathcurveto{\pgfqpoint{1.441006in}{0.695046in}}{\pgfqpoint{1.436678in}{0.705496in}}{\pgfqpoint{1.428973in}{0.713201in}}%
\pgfpathcurveto{\pgfqpoint{1.421269in}{0.720905in}}{\pgfqpoint{1.410818in}{0.725234in}}{\pgfqpoint{1.399923in}{0.725234in}}%
\pgfpathcurveto{\pgfqpoint{1.389027in}{0.725234in}}{\pgfqpoint{1.378576in}{0.720905in}}{\pgfqpoint{1.370872in}{0.713201in}}%
\pgfpathcurveto{\pgfqpoint{1.363168in}{0.705496in}}{\pgfqpoint{1.358839in}{0.695046in}}{\pgfqpoint{1.358839in}{0.684150in}}%
\pgfpathcurveto{\pgfqpoint{1.358839in}{0.673254in}}{\pgfqpoint{1.363168in}{0.662804in}}{\pgfqpoint{1.370872in}{0.655099in}}%
\pgfpathcurveto{\pgfqpoint{1.378576in}{0.647395in}}{\pgfqpoint{1.389027in}{0.643066in}}{\pgfqpoint{1.399923in}{0.643066in}}%
\pgfusepath{stroke}%
\end{pgfscope}%
\begin{pgfscope}%
\pgfpathrectangle{\pgfqpoint{0.688192in}{0.670138in}}{\pgfqpoint{7.111808in}{5.061530in}}%
\pgfusepath{clip}%
\pgfsetbuttcap%
\pgfsetroundjoin%
\pgfsetlinewidth{1.003750pt}%
\definecolor{currentstroke}{rgb}{0.000000,0.000000,0.000000}%
\pgfsetstrokecolor{currentstroke}%
\pgfsetdash{}{0pt}%
\pgfpathmoveto{\pgfqpoint{2.794622in}{0.693227in}}%
\pgfpathcurveto{\pgfqpoint{2.805518in}{0.693227in}}{\pgfqpoint{2.815968in}{0.697556in}}{\pgfqpoint{2.823673in}{0.705261in}}%
\pgfpathcurveto{\pgfqpoint{2.831377in}{0.712965in}}{\pgfqpoint{2.835706in}{0.723416in}}{\pgfqpoint{2.835706in}{0.734311in}}%
\pgfpathcurveto{\pgfqpoint{2.835706in}{0.745207in}}{\pgfqpoint{2.831377in}{0.755658in}}{\pgfqpoint{2.823673in}{0.763362in}}%
\pgfpathcurveto{\pgfqpoint{2.815968in}{0.771066in}}{\pgfqpoint{2.805518in}{0.775395in}}{\pgfqpoint{2.794622in}{0.775395in}}%
\pgfpathcurveto{\pgfqpoint{2.783726in}{0.775395in}}{\pgfqpoint{2.773276in}{0.771066in}}{\pgfqpoint{2.765571in}{0.763362in}}%
\pgfpathcurveto{\pgfqpoint{2.757867in}{0.755658in}}{\pgfqpoint{2.753538in}{0.745207in}}{\pgfqpoint{2.753538in}{0.734311in}}%
\pgfpathcurveto{\pgfqpoint{2.753538in}{0.723416in}}{\pgfqpoint{2.757867in}{0.712965in}}{\pgfqpoint{2.765571in}{0.705261in}}%
\pgfpathcurveto{\pgfqpoint{2.773276in}{0.697556in}}{\pgfqpoint{2.783726in}{0.693227in}}{\pgfqpoint{2.794622in}{0.693227in}}%
\pgfpathlineto{\pgfqpoint{2.794622in}{0.693227in}}%
\pgfpathclose%
\pgfusepath{stroke}%
\end{pgfscope}%
\begin{pgfscope}%
\pgfpathrectangle{\pgfqpoint{0.688192in}{0.670138in}}{\pgfqpoint{7.111808in}{5.061530in}}%
\pgfusepath{clip}%
\pgfsetbuttcap%
\pgfsetroundjoin%
\pgfsetlinewidth{1.003750pt}%
\definecolor{currentstroke}{rgb}{0.000000,0.000000,0.000000}%
\pgfsetstrokecolor{currentstroke}%
\pgfsetdash{}{0pt}%
\pgfpathmoveto{\pgfqpoint{2.458239in}{0.697123in}}%
\pgfpathcurveto{\pgfqpoint{2.469134in}{0.697123in}}{\pgfqpoint{2.479585in}{0.701452in}}{\pgfqpoint{2.487289in}{0.709157in}}%
\pgfpathcurveto{\pgfqpoint{2.494994in}{0.716861in}}{\pgfqpoint{2.499323in}{0.727312in}}{\pgfqpoint{2.499323in}{0.738207in}}%
\pgfpathcurveto{\pgfqpoint{2.499323in}{0.749103in}}{\pgfqpoint{2.494994in}{0.759554in}}{\pgfqpoint{2.487289in}{0.767258in}}%
\pgfpathcurveto{\pgfqpoint{2.479585in}{0.774962in}}{\pgfqpoint{2.469134in}{0.779291in}}{\pgfqpoint{2.458239in}{0.779291in}}%
\pgfpathcurveto{\pgfqpoint{2.447343in}{0.779291in}}{\pgfqpoint{2.436892in}{0.774962in}}{\pgfqpoint{2.429188in}{0.767258in}}%
\pgfpathcurveto{\pgfqpoint{2.421484in}{0.759554in}}{\pgfqpoint{2.417155in}{0.749103in}}{\pgfqpoint{2.417155in}{0.738207in}}%
\pgfpathcurveto{\pgfqpoint{2.417155in}{0.727312in}}{\pgfqpoint{2.421484in}{0.716861in}}{\pgfqpoint{2.429188in}{0.709157in}}%
\pgfpathcurveto{\pgfqpoint{2.436892in}{0.701452in}}{\pgfqpoint{2.447343in}{0.697123in}}{\pgfqpoint{2.458239in}{0.697123in}}%
\pgfpathlineto{\pgfqpoint{2.458239in}{0.697123in}}%
\pgfpathclose%
\pgfusepath{stroke}%
\end{pgfscope}%
\begin{pgfscope}%
\pgfpathrectangle{\pgfqpoint{0.688192in}{0.670138in}}{\pgfqpoint{7.111808in}{5.061530in}}%
\pgfusepath{clip}%
\pgfsetbuttcap%
\pgfsetroundjoin%
\pgfsetlinewidth{1.003750pt}%
\definecolor{currentstroke}{rgb}{0.000000,0.000000,0.000000}%
\pgfsetstrokecolor{currentstroke}%
\pgfsetdash{}{0pt}%
\pgfpathmoveto{\pgfqpoint{0.749254in}{1.404065in}}%
\pgfpathcurveto{\pgfqpoint{0.760150in}{1.404065in}}{\pgfqpoint{0.770600in}{1.408394in}}{\pgfqpoint{0.778305in}{1.416099in}}%
\pgfpathcurveto{\pgfqpoint{0.786009in}{1.423803in}}{\pgfqpoint{0.790338in}{1.434254in}}{\pgfqpoint{0.790338in}{1.445149in}}%
\pgfpathcurveto{\pgfqpoint{0.790338in}{1.456045in}}{\pgfqpoint{0.786009in}{1.466496in}}{\pgfqpoint{0.778305in}{1.474200in}}%
\pgfpathcurveto{\pgfqpoint{0.770600in}{1.481904in}}{\pgfqpoint{0.760150in}{1.486233in}}{\pgfqpoint{0.749254in}{1.486233in}}%
\pgfpathcurveto{\pgfqpoint{0.738358in}{1.486233in}}{\pgfqpoint{0.727908in}{1.481904in}}{\pgfqpoint{0.720203in}{1.474200in}}%
\pgfpathcurveto{\pgfqpoint{0.712499in}{1.466496in}}{\pgfqpoint{0.708170in}{1.456045in}}{\pgfqpoint{0.708170in}{1.445149in}}%
\pgfpathcurveto{\pgfqpoint{0.708170in}{1.434254in}}{\pgfqpoint{0.712499in}{1.423803in}}{\pgfqpoint{0.720203in}{1.416099in}}%
\pgfpathcurveto{\pgfqpoint{0.727908in}{1.408394in}}{\pgfqpoint{0.738358in}{1.404065in}}{\pgfqpoint{0.749254in}{1.404065in}}%
\pgfpathlineto{\pgfqpoint{0.749254in}{1.404065in}}%
\pgfpathclose%
\pgfusepath{stroke}%
\end{pgfscope}%
\begin{pgfscope}%
\pgfpathrectangle{\pgfqpoint{0.688192in}{0.670138in}}{\pgfqpoint{7.111808in}{5.061530in}}%
\pgfusepath{clip}%
\pgfsetbuttcap%
\pgfsetroundjoin%
\pgfsetlinewidth{1.003750pt}%
\definecolor{currentstroke}{rgb}{0.000000,0.000000,0.000000}%
\pgfsetstrokecolor{currentstroke}%
\pgfsetdash{}{0pt}%
\pgfpathmoveto{\pgfqpoint{4.420185in}{4.806141in}}%
\pgfpathcurveto{\pgfqpoint{4.431080in}{4.806141in}}{\pgfqpoint{4.441531in}{4.810469in}}{\pgfqpoint{4.449235in}{4.818174in}}%
\pgfpathcurveto{\pgfqpoint{4.456940in}{4.825878in}}{\pgfqpoint{4.461269in}{4.836329in}}{\pgfqpoint{4.461269in}{4.847224in}}%
\pgfpathcurveto{\pgfqpoint{4.461269in}{4.858120in}}{\pgfqpoint{4.456940in}{4.868571in}}{\pgfqpoint{4.449235in}{4.876275in}}%
\pgfpathcurveto{\pgfqpoint{4.441531in}{4.883980in}}{\pgfqpoint{4.431080in}{4.888308in}}{\pgfqpoint{4.420185in}{4.888308in}}%
\pgfpathcurveto{\pgfqpoint{4.409289in}{4.888308in}}{\pgfqpoint{4.398838in}{4.883980in}}{\pgfqpoint{4.391134in}{4.876275in}}%
\pgfpathcurveto{\pgfqpoint{4.383430in}{4.868571in}}{\pgfqpoint{4.379101in}{4.858120in}}{\pgfqpoint{4.379101in}{4.847224in}}%
\pgfpathcurveto{\pgfqpoint{4.379101in}{4.836329in}}{\pgfqpoint{4.383430in}{4.825878in}}{\pgfqpoint{4.391134in}{4.818174in}}%
\pgfpathcurveto{\pgfqpoint{4.398838in}{4.810469in}}{\pgfqpoint{4.409289in}{4.806141in}}{\pgfqpoint{4.420185in}{4.806141in}}%
\pgfpathlineto{\pgfqpoint{4.420185in}{4.806141in}}%
\pgfpathclose%
\pgfusepath{stroke}%
\end{pgfscope}%
\begin{pgfscope}%
\pgfpathrectangle{\pgfqpoint{0.688192in}{0.670138in}}{\pgfqpoint{7.111808in}{5.061530in}}%
\pgfusepath{clip}%
\pgfsetbuttcap%
\pgfsetroundjoin%
\pgfsetlinewidth{1.003750pt}%
\definecolor{currentstroke}{rgb}{0.000000,0.000000,0.000000}%
\pgfsetstrokecolor{currentstroke}%
\pgfsetdash{}{0pt}%
\pgfpathmoveto{\pgfqpoint{2.911373in}{0.635871in}}%
\pgfpathcurveto{\pgfqpoint{2.922269in}{0.635871in}}{\pgfqpoint{2.932719in}{0.640200in}}{\pgfqpoint{2.940424in}{0.647905in}}%
\pgfpathcurveto{\pgfqpoint{2.948128in}{0.655609in}}{\pgfqpoint{2.952457in}{0.666060in}}{\pgfqpoint{2.952457in}{0.676955in}}%
\pgfpathcurveto{\pgfqpoint{2.952457in}{0.687851in}}{\pgfqpoint{2.948128in}{0.698302in}}{\pgfqpoint{2.940424in}{0.706006in}}%
\pgfpathcurveto{\pgfqpoint{2.932719in}{0.713710in}}{\pgfqpoint{2.922269in}{0.718039in}}{\pgfqpoint{2.911373in}{0.718039in}}%
\pgfpathcurveto{\pgfqpoint{2.900477in}{0.718039in}}{\pgfqpoint{2.890027in}{0.713710in}}{\pgfqpoint{2.882322in}{0.706006in}}%
\pgfpathcurveto{\pgfqpoint{2.874618in}{0.698302in}}{\pgfqpoint{2.870289in}{0.687851in}}{\pgfqpoint{2.870289in}{0.676955in}}%
\pgfpathcurveto{\pgfqpoint{2.870289in}{0.666060in}}{\pgfqpoint{2.874618in}{0.655609in}}{\pgfqpoint{2.882322in}{0.647905in}}%
\pgfpathcurveto{\pgfqpoint{2.890027in}{0.640200in}}{\pgfqpoint{2.900477in}{0.635871in}}{\pgfqpoint{2.911373in}{0.635871in}}%
\pgfusepath{stroke}%
\end{pgfscope}%
\begin{pgfscope}%
\pgfpathrectangle{\pgfqpoint{0.688192in}{0.670138in}}{\pgfqpoint{7.111808in}{5.061530in}}%
\pgfusepath{clip}%
\pgfsetbuttcap%
\pgfsetroundjoin%
\pgfsetlinewidth{1.003750pt}%
\definecolor{currentstroke}{rgb}{0.000000,0.000000,0.000000}%
\pgfsetstrokecolor{currentstroke}%
\pgfsetdash{}{0pt}%
\pgfpathmoveto{\pgfqpoint{3.290259in}{0.889197in}}%
\pgfpathcurveto{\pgfqpoint{3.301154in}{0.889197in}}{\pgfqpoint{3.311605in}{0.893526in}}{\pgfqpoint{3.319309in}{0.901230in}}%
\pgfpathcurveto{\pgfqpoint{3.327014in}{0.908935in}}{\pgfqpoint{3.331342in}{0.919385in}}{\pgfqpoint{3.331342in}{0.930281in}}%
\pgfpathcurveto{\pgfqpoint{3.331342in}{0.941177in}}{\pgfqpoint{3.327014in}{0.951627in}}{\pgfqpoint{3.319309in}{0.959332in}}%
\pgfpathcurveto{\pgfqpoint{3.311605in}{0.967036in}}{\pgfqpoint{3.301154in}{0.971365in}}{\pgfqpoint{3.290259in}{0.971365in}}%
\pgfpathcurveto{\pgfqpoint{3.279363in}{0.971365in}}{\pgfqpoint{3.268912in}{0.967036in}}{\pgfqpoint{3.261208in}{0.959332in}}%
\pgfpathcurveto{\pgfqpoint{3.253504in}{0.951627in}}{\pgfqpoint{3.249175in}{0.941177in}}{\pgfqpoint{3.249175in}{0.930281in}}%
\pgfpathcurveto{\pgfqpoint{3.249175in}{0.919385in}}{\pgfqpoint{3.253504in}{0.908935in}}{\pgfqpoint{3.261208in}{0.901230in}}%
\pgfpathcurveto{\pgfqpoint{3.268912in}{0.893526in}}{\pgfqpoint{3.279363in}{0.889197in}}{\pgfqpoint{3.290259in}{0.889197in}}%
\pgfpathlineto{\pgfqpoint{3.290259in}{0.889197in}}%
\pgfpathclose%
\pgfusepath{stroke}%
\end{pgfscope}%
\begin{pgfscope}%
\pgfpathrectangle{\pgfqpoint{0.688192in}{0.670138in}}{\pgfqpoint{7.111808in}{5.061530in}}%
\pgfusepath{clip}%
\pgfsetbuttcap%
\pgfsetroundjoin%
\pgfsetlinewidth{1.003750pt}%
\definecolor{currentstroke}{rgb}{0.000000,0.000000,0.000000}%
\pgfsetstrokecolor{currentstroke}%
\pgfsetdash{}{0pt}%
\pgfpathmoveto{\pgfqpoint{2.056022in}{1.075198in}}%
\pgfpathcurveto{\pgfqpoint{2.066917in}{1.075198in}}{\pgfqpoint{2.077368in}{1.079527in}}{\pgfqpoint{2.085072in}{1.087231in}}%
\pgfpathcurveto{\pgfqpoint{2.092777in}{1.094936in}}{\pgfqpoint{2.097106in}{1.105386in}}{\pgfqpoint{2.097106in}{1.116282in}}%
\pgfpathcurveto{\pgfqpoint{2.097106in}{1.127178in}}{\pgfqpoint{2.092777in}{1.137628in}}{\pgfqpoint{2.085072in}{1.145333in}}%
\pgfpathcurveto{\pgfqpoint{2.077368in}{1.153037in}}{\pgfqpoint{2.066917in}{1.157366in}}{\pgfqpoint{2.056022in}{1.157366in}}%
\pgfpathcurveto{\pgfqpoint{2.045126in}{1.157366in}}{\pgfqpoint{2.034675in}{1.153037in}}{\pgfqpoint{2.026971in}{1.145333in}}%
\pgfpathcurveto{\pgfqpoint{2.019267in}{1.137628in}}{\pgfqpoint{2.014938in}{1.127178in}}{\pgfqpoint{2.014938in}{1.116282in}}%
\pgfpathcurveto{\pgfqpoint{2.014938in}{1.105386in}}{\pgfqpoint{2.019267in}{1.094936in}}{\pgfqpoint{2.026971in}{1.087231in}}%
\pgfpathcurveto{\pgfqpoint{2.034675in}{1.079527in}}{\pgfqpoint{2.045126in}{1.075198in}}{\pgfqpoint{2.056022in}{1.075198in}}%
\pgfpathlineto{\pgfqpoint{2.056022in}{1.075198in}}%
\pgfpathclose%
\pgfusepath{stroke}%
\end{pgfscope}%
\begin{pgfscope}%
\pgfpathrectangle{\pgfqpoint{0.688192in}{0.670138in}}{\pgfqpoint{7.111808in}{5.061530in}}%
\pgfusepath{clip}%
\pgfsetbuttcap%
\pgfsetroundjoin%
\pgfsetlinewidth{1.003750pt}%
\definecolor{currentstroke}{rgb}{0.000000,0.000000,0.000000}%
\pgfsetstrokecolor{currentstroke}%
\pgfsetdash{}{0pt}%
\pgfpathmoveto{\pgfqpoint{2.233371in}{1.183790in}}%
\pgfpathcurveto{\pgfqpoint{2.244266in}{1.183790in}}{\pgfqpoint{2.254717in}{1.188119in}}{\pgfqpoint{2.262421in}{1.195823in}}%
\pgfpathcurveto{\pgfqpoint{2.270126in}{1.203527in}}{\pgfqpoint{2.274454in}{1.213978in}}{\pgfqpoint{2.274454in}{1.224874in}}%
\pgfpathcurveto{\pgfqpoint{2.274454in}{1.235769in}}{\pgfqpoint{2.270126in}{1.246220in}}{\pgfqpoint{2.262421in}{1.253924in}}%
\pgfpathcurveto{\pgfqpoint{2.254717in}{1.261629in}}{\pgfqpoint{2.244266in}{1.265958in}}{\pgfqpoint{2.233371in}{1.265958in}}%
\pgfpathcurveto{\pgfqpoint{2.222475in}{1.265958in}}{\pgfqpoint{2.212024in}{1.261629in}}{\pgfqpoint{2.204320in}{1.253924in}}%
\pgfpathcurveto{\pgfqpoint{2.196615in}{1.246220in}}{\pgfqpoint{2.192287in}{1.235769in}}{\pgfqpoint{2.192287in}{1.224874in}}%
\pgfpathcurveto{\pgfqpoint{2.192287in}{1.213978in}}{\pgfqpoint{2.196615in}{1.203527in}}{\pgfqpoint{2.204320in}{1.195823in}}%
\pgfpathcurveto{\pgfqpoint{2.212024in}{1.188119in}}{\pgfqpoint{2.222475in}{1.183790in}}{\pgfqpoint{2.233371in}{1.183790in}}%
\pgfpathlineto{\pgfqpoint{2.233371in}{1.183790in}}%
\pgfpathclose%
\pgfusepath{stroke}%
\end{pgfscope}%
\begin{pgfscope}%
\pgfpathrectangle{\pgfqpoint{0.688192in}{0.670138in}}{\pgfqpoint{7.111808in}{5.061530in}}%
\pgfusepath{clip}%
\pgfsetbuttcap%
\pgfsetroundjoin%
\pgfsetlinewidth{1.003750pt}%
\definecolor{currentstroke}{rgb}{0.000000,0.000000,0.000000}%
\pgfsetstrokecolor{currentstroke}%
\pgfsetdash{}{0pt}%
\pgfpathmoveto{\pgfqpoint{1.102529in}{0.648268in}}%
\pgfpathcurveto{\pgfqpoint{1.113424in}{0.648268in}}{\pgfqpoint{1.123875in}{0.652597in}}{\pgfqpoint{1.131579in}{0.660301in}}%
\pgfpathcurveto{\pgfqpoint{1.139284in}{0.668005in}}{\pgfqpoint{1.143613in}{0.678456in}}{\pgfqpoint{1.143613in}{0.689352in}}%
\pgfpathcurveto{\pgfqpoint{1.143613in}{0.700247in}}{\pgfqpoint{1.139284in}{0.710698in}}{\pgfqpoint{1.131579in}{0.718402in}}%
\pgfpathcurveto{\pgfqpoint{1.123875in}{0.726107in}}{\pgfqpoint{1.113424in}{0.730436in}}{\pgfqpoint{1.102529in}{0.730436in}}%
\pgfpathcurveto{\pgfqpoint{1.091633in}{0.730436in}}{\pgfqpoint{1.081182in}{0.726107in}}{\pgfqpoint{1.073478in}{0.718402in}}%
\pgfpathcurveto{\pgfqpoint{1.065774in}{0.710698in}}{\pgfqpoint{1.061445in}{0.700247in}}{\pgfqpoint{1.061445in}{0.689352in}}%
\pgfpathcurveto{\pgfqpoint{1.061445in}{0.678456in}}{\pgfqpoint{1.065774in}{0.668005in}}{\pgfqpoint{1.073478in}{0.660301in}}%
\pgfpathcurveto{\pgfqpoint{1.081182in}{0.652597in}}{\pgfqpoint{1.091633in}{0.648268in}}{\pgfqpoint{1.102529in}{0.648268in}}%
\pgfusepath{stroke}%
\end{pgfscope}%
\begin{pgfscope}%
\pgfpathrectangle{\pgfqpoint{0.688192in}{0.670138in}}{\pgfqpoint{7.111808in}{5.061530in}}%
\pgfusepath{clip}%
\pgfsetbuttcap%
\pgfsetroundjoin%
\pgfsetlinewidth{1.003750pt}%
\definecolor{currentstroke}{rgb}{0.000000,0.000000,0.000000}%
\pgfsetstrokecolor{currentstroke}%
\pgfsetdash{}{0pt}%
\pgfpathmoveto{\pgfqpoint{0.850270in}{0.697314in}}%
\pgfpathcurveto{\pgfqpoint{0.861166in}{0.697314in}}{\pgfqpoint{0.871617in}{0.701642in}}{\pgfqpoint{0.879321in}{0.709347in}}%
\pgfpathcurveto{\pgfqpoint{0.887025in}{0.717051in}}{\pgfqpoint{0.891354in}{0.727502in}}{\pgfqpoint{0.891354in}{0.738398in}}%
\pgfpathcurveto{\pgfqpoint{0.891354in}{0.749293in}}{\pgfqpoint{0.887025in}{0.759744in}}{\pgfqpoint{0.879321in}{0.767448in}}%
\pgfpathcurveto{\pgfqpoint{0.871617in}{0.775153in}}{\pgfqpoint{0.861166in}{0.779481in}}{\pgfqpoint{0.850270in}{0.779481in}}%
\pgfpathcurveto{\pgfqpoint{0.839375in}{0.779481in}}{\pgfqpoint{0.828924in}{0.775153in}}{\pgfqpoint{0.821220in}{0.767448in}}%
\pgfpathcurveto{\pgfqpoint{0.813515in}{0.759744in}}{\pgfqpoint{0.809186in}{0.749293in}}{\pgfqpoint{0.809186in}{0.738398in}}%
\pgfpathcurveto{\pgfqpoint{0.809186in}{0.727502in}}{\pgfqpoint{0.813515in}{0.717051in}}{\pgfqpoint{0.821220in}{0.709347in}}%
\pgfpathcurveto{\pgfqpoint{0.828924in}{0.701642in}}{\pgfqpoint{0.839375in}{0.697314in}}{\pgfqpoint{0.850270in}{0.697314in}}%
\pgfpathlineto{\pgfqpoint{0.850270in}{0.697314in}}%
\pgfpathclose%
\pgfusepath{stroke}%
\end{pgfscope}%
\begin{pgfscope}%
\pgfpathrectangle{\pgfqpoint{0.688192in}{0.670138in}}{\pgfqpoint{7.111808in}{5.061530in}}%
\pgfusepath{clip}%
\pgfsetbuttcap%
\pgfsetroundjoin%
\pgfsetlinewidth{1.003750pt}%
\definecolor{currentstroke}{rgb}{0.000000,0.000000,0.000000}%
\pgfsetstrokecolor{currentstroke}%
\pgfsetdash{}{0pt}%
\pgfpathmoveto{\pgfqpoint{1.693830in}{2.533599in}}%
\pgfpathcurveto{\pgfqpoint{1.704726in}{2.533599in}}{\pgfqpoint{1.715177in}{2.537928in}}{\pgfqpoint{1.722881in}{2.545632in}}%
\pgfpathcurveto{\pgfqpoint{1.730585in}{2.553337in}}{\pgfqpoint{1.734914in}{2.563787in}}{\pgfqpoint{1.734914in}{2.574683in}}%
\pgfpathcurveto{\pgfqpoint{1.734914in}{2.585579in}}{\pgfqpoint{1.730585in}{2.596029in}}{\pgfqpoint{1.722881in}{2.603734in}}%
\pgfpathcurveto{\pgfqpoint{1.715177in}{2.611438in}}{\pgfqpoint{1.704726in}{2.615767in}}{\pgfqpoint{1.693830in}{2.615767in}}%
\pgfpathcurveto{\pgfqpoint{1.682935in}{2.615767in}}{\pgfqpoint{1.672484in}{2.611438in}}{\pgfqpoint{1.664779in}{2.603734in}}%
\pgfpathcurveto{\pgfqpoint{1.657075in}{2.596029in}}{\pgfqpoint{1.652746in}{2.585579in}}{\pgfqpoint{1.652746in}{2.574683in}}%
\pgfpathcurveto{\pgfqpoint{1.652746in}{2.563787in}}{\pgfqpoint{1.657075in}{2.553337in}}{\pgfqpoint{1.664779in}{2.545632in}}%
\pgfpathcurveto{\pgfqpoint{1.672484in}{2.537928in}}{\pgfqpoint{1.682935in}{2.533599in}}{\pgfqpoint{1.693830in}{2.533599in}}%
\pgfpathlineto{\pgfqpoint{1.693830in}{2.533599in}}%
\pgfpathclose%
\pgfusepath{stroke}%
\end{pgfscope}%
\begin{pgfscope}%
\pgfpathrectangle{\pgfqpoint{0.688192in}{0.670138in}}{\pgfqpoint{7.111808in}{5.061530in}}%
\pgfusepath{clip}%
\pgfsetbuttcap%
\pgfsetroundjoin%
\pgfsetlinewidth{1.003750pt}%
\definecolor{currentstroke}{rgb}{0.000000,0.000000,0.000000}%
\pgfsetstrokecolor{currentstroke}%
\pgfsetdash{}{0pt}%
\pgfpathmoveto{\pgfqpoint{1.672396in}{2.153838in}}%
\pgfpathcurveto{\pgfqpoint{1.683291in}{2.153838in}}{\pgfqpoint{1.693742in}{2.158167in}}{\pgfqpoint{1.701446in}{2.165871in}}%
\pgfpathcurveto{\pgfqpoint{1.709151in}{2.173575in}}{\pgfqpoint{1.713480in}{2.184026in}}{\pgfqpoint{1.713480in}{2.194922in}}%
\pgfpathcurveto{\pgfqpoint{1.713480in}{2.205817in}}{\pgfqpoint{1.709151in}{2.216268in}}{\pgfqpoint{1.701446in}{2.223972in}}%
\pgfpathcurveto{\pgfqpoint{1.693742in}{2.231677in}}{\pgfqpoint{1.683291in}{2.236005in}}{\pgfqpoint{1.672396in}{2.236005in}}%
\pgfpathcurveto{\pgfqpoint{1.661500in}{2.236005in}}{\pgfqpoint{1.651049in}{2.231677in}}{\pgfqpoint{1.643345in}{2.223972in}}%
\pgfpathcurveto{\pgfqpoint{1.635641in}{2.216268in}}{\pgfqpoint{1.631312in}{2.205817in}}{\pgfqpoint{1.631312in}{2.194922in}}%
\pgfpathcurveto{\pgfqpoint{1.631312in}{2.184026in}}{\pgfqpoint{1.635641in}{2.173575in}}{\pgfqpoint{1.643345in}{2.165871in}}%
\pgfpathcurveto{\pgfqpoint{1.651049in}{2.158167in}}{\pgfqpoint{1.661500in}{2.153838in}}{\pgfqpoint{1.672396in}{2.153838in}}%
\pgfpathlineto{\pgfqpoint{1.672396in}{2.153838in}}%
\pgfpathclose%
\pgfusepath{stroke}%
\end{pgfscope}%
\begin{pgfscope}%
\pgfpathrectangle{\pgfqpoint{0.688192in}{0.670138in}}{\pgfqpoint{7.111808in}{5.061530in}}%
\pgfusepath{clip}%
\pgfsetbuttcap%
\pgfsetroundjoin%
\pgfsetlinewidth{1.003750pt}%
\definecolor{currentstroke}{rgb}{0.000000,0.000000,0.000000}%
\pgfsetstrokecolor{currentstroke}%
\pgfsetdash{}{0pt}%
\pgfpathmoveto{\pgfqpoint{1.761288in}{0.640905in}}%
\pgfpathcurveto{\pgfqpoint{1.772184in}{0.640905in}}{\pgfqpoint{1.782635in}{0.645234in}}{\pgfqpoint{1.790339in}{0.652938in}}%
\pgfpathcurveto{\pgfqpoint{1.798043in}{0.660642in}}{\pgfqpoint{1.802372in}{0.671093in}}{\pgfqpoint{1.802372in}{0.681989in}}%
\pgfpathcurveto{\pgfqpoint{1.802372in}{0.692884in}}{\pgfqpoint{1.798043in}{0.703335in}}{\pgfqpoint{1.790339in}{0.711039in}}%
\pgfpathcurveto{\pgfqpoint{1.782635in}{0.718744in}}{\pgfqpoint{1.772184in}{0.723073in}}{\pgfqpoint{1.761288in}{0.723073in}}%
\pgfpathcurveto{\pgfqpoint{1.750393in}{0.723073in}}{\pgfqpoint{1.739942in}{0.718744in}}{\pgfqpoint{1.732237in}{0.711039in}}%
\pgfpathcurveto{\pgfqpoint{1.724533in}{0.703335in}}{\pgfqpoint{1.720204in}{0.692884in}}{\pgfqpoint{1.720204in}{0.681989in}}%
\pgfpathcurveto{\pgfqpoint{1.720204in}{0.671093in}}{\pgfqpoint{1.724533in}{0.660642in}}{\pgfqpoint{1.732237in}{0.652938in}}%
\pgfpathcurveto{\pgfqpoint{1.739942in}{0.645234in}}{\pgfqpoint{1.750393in}{0.640905in}}{\pgfqpoint{1.761288in}{0.640905in}}%
\pgfusepath{stroke}%
\end{pgfscope}%
\begin{pgfscope}%
\pgfpathrectangle{\pgfqpoint{0.688192in}{0.670138in}}{\pgfqpoint{7.111808in}{5.061530in}}%
\pgfusepath{clip}%
\pgfsetbuttcap%
\pgfsetroundjoin%
\pgfsetlinewidth{1.003750pt}%
\definecolor{currentstroke}{rgb}{0.000000,0.000000,0.000000}%
\pgfsetstrokecolor{currentstroke}%
\pgfsetdash{}{0pt}%
\pgfpathmoveto{\pgfqpoint{4.449664in}{0.700004in}}%
\pgfpathcurveto{\pgfqpoint{4.460560in}{0.700004in}}{\pgfqpoint{4.471010in}{0.704333in}}{\pgfqpoint{4.478715in}{0.712037in}}%
\pgfpathcurveto{\pgfqpoint{4.486419in}{0.719741in}}{\pgfqpoint{4.490748in}{0.730192in}}{\pgfqpoint{4.490748in}{0.741088in}}%
\pgfpathcurveto{\pgfqpoint{4.490748in}{0.751983in}}{\pgfqpoint{4.486419in}{0.762434in}}{\pgfqpoint{4.478715in}{0.770138in}}%
\pgfpathcurveto{\pgfqpoint{4.471010in}{0.777843in}}{\pgfqpoint{4.460560in}{0.782172in}}{\pgfqpoint{4.449664in}{0.782172in}}%
\pgfpathcurveto{\pgfqpoint{4.438768in}{0.782172in}}{\pgfqpoint{4.428318in}{0.777843in}}{\pgfqpoint{4.420613in}{0.770138in}}%
\pgfpathcurveto{\pgfqpoint{4.412909in}{0.762434in}}{\pgfqpoint{4.408580in}{0.751983in}}{\pgfqpoint{4.408580in}{0.741088in}}%
\pgfpathcurveto{\pgfqpoint{4.408580in}{0.730192in}}{\pgfqpoint{4.412909in}{0.719741in}}{\pgfqpoint{4.420613in}{0.712037in}}%
\pgfpathcurveto{\pgfqpoint{4.428318in}{0.704333in}}{\pgfqpoint{4.438768in}{0.700004in}}{\pgfqpoint{4.449664in}{0.700004in}}%
\pgfpathlineto{\pgfqpoint{4.449664in}{0.700004in}}%
\pgfpathclose%
\pgfusepath{stroke}%
\end{pgfscope}%
\begin{pgfscope}%
\pgfpathrectangle{\pgfqpoint{0.688192in}{0.670138in}}{\pgfqpoint{7.111808in}{5.061530in}}%
\pgfusepath{clip}%
\pgfsetbuttcap%
\pgfsetroundjoin%
\pgfsetlinewidth{1.003750pt}%
\definecolor{currentstroke}{rgb}{0.000000,0.000000,0.000000}%
\pgfsetstrokecolor{currentstroke}%
\pgfsetdash{}{0pt}%
\pgfpathmoveto{\pgfqpoint{0.842811in}{0.703924in}}%
\pgfpathcurveto{\pgfqpoint{0.853707in}{0.703924in}}{\pgfqpoint{0.864158in}{0.708253in}}{\pgfqpoint{0.871862in}{0.715957in}}%
\pgfpathcurveto{\pgfqpoint{0.879566in}{0.723662in}}{\pgfqpoint{0.883895in}{0.734112in}}{\pgfqpoint{0.883895in}{0.745008in}}%
\pgfpathcurveto{\pgfqpoint{0.883895in}{0.755904in}}{\pgfqpoint{0.879566in}{0.766354in}}{\pgfqpoint{0.871862in}{0.774059in}}%
\pgfpathcurveto{\pgfqpoint{0.864158in}{0.781763in}}{\pgfqpoint{0.853707in}{0.786092in}}{\pgfqpoint{0.842811in}{0.786092in}}%
\pgfpathcurveto{\pgfqpoint{0.831916in}{0.786092in}}{\pgfqpoint{0.821465in}{0.781763in}}{\pgfqpoint{0.813760in}{0.774059in}}%
\pgfpathcurveto{\pgfqpoint{0.806056in}{0.766354in}}{\pgfqpoint{0.801727in}{0.755904in}}{\pgfqpoint{0.801727in}{0.745008in}}%
\pgfpathcurveto{\pgfqpoint{0.801727in}{0.734112in}}{\pgfqpoint{0.806056in}{0.723662in}}{\pgfqpoint{0.813760in}{0.715957in}}%
\pgfpathcurveto{\pgfqpoint{0.821465in}{0.708253in}}{\pgfqpoint{0.831916in}{0.703924in}}{\pgfqpoint{0.842811in}{0.703924in}}%
\pgfpathlineto{\pgfqpoint{0.842811in}{0.703924in}}%
\pgfpathclose%
\pgfusepath{stroke}%
\end{pgfscope}%
\begin{pgfscope}%
\pgfpathrectangle{\pgfqpoint{0.688192in}{0.670138in}}{\pgfqpoint{7.111808in}{5.061530in}}%
\pgfusepath{clip}%
\pgfsetbuttcap%
\pgfsetroundjoin%
\pgfsetlinewidth{1.003750pt}%
\definecolor{currentstroke}{rgb}{0.000000,0.000000,0.000000}%
\pgfsetstrokecolor{currentstroke}%
\pgfsetdash{}{0pt}%
\pgfpathmoveto{\pgfqpoint{1.364139in}{0.643320in}}%
\pgfpathcurveto{\pgfqpoint{1.375035in}{0.643320in}}{\pgfqpoint{1.385486in}{0.647649in}}{\pgfqpoint{1.393190in}{0.655353in}}%
\pgfpathcurveto{\pgfqpoint{1.400894in}{0.663057in}}{\pgfqpoint{1.405223in}{0.673508in}}{\pgfqpoint{1.405223in}{0.684404in}}%
\pgfpathcurveto{\pgfqpoint{1.405223in}{0.695299in}}{\pgfqpoint{1.400894in}{0.705750in}}{\pgfqpoint{1.393190in}{0.713454in}}%
\pgfpathcurveto{\pgfqpoint{1.385486in}{0.721159in}}{\pgfqpoint{1.375035in}{0.725488in}}{\pgfqpoint{1.364139in}{0.725488in}}%
\pgfpathcurveto{\pgfqpoint{1.353244in}{0.725488in}}{\pgfqpoint{1.342793in}{0.721159in}}{\pgfqpoint{1.335089in}{0.713454in}}%
\pgfpathcurveto{\pgfqpoint{1.327384in}{0.705750in}}{\pgfqpoint{1.323055in}{0.695299in}}{\pgfqpoint{1.323055in}{0.684404in}}%
\pgfpathcurveto{\pgfqpoint{1.323055in}{0.673508in}}{\pgfqpoint{1.327384in}{0.663057in}}{\pgfqpoint{1.335089in}{0.655353in}}%
\pgfpathcurveto{\pgfqpoint{1.342793in}{0.647649in}}{\pgfqpoint{1.353244in}{0.643320in}}{\pgfqpoint{1.364139in}{0.643320in}}%
\pgfusepath{stroke}%
\end{pgfscope}%
\begin{pgfscope}%
\pgfpathrectangle{\pgfqpoint{0.688192in}{0.670138in}}{\pgfqpoint{7.111808in}{5.061530in}}%
\pgfusepath{clip}%
\pgfsetbuttcap%
\pgfsetroundjoin%
\pgfsetlinewidth{1.003750pt}%
\definecolor{currentstroke}{rgb}{0.000000,0.000000,0.000000}%
\pgfsetstrokecolor{currentstroke}%
\pgfsetdash{}{0pt}%
\pgfpathmoveto{\pgfqpoint{0.945763in}{0.669750in}}%
\pgfpathcurveto{\pgfqpoint{0.956659in}{0.669750in}}{\pgfqpoint{0.967110in}{0.674079in}}{\pgfqpoint{0.974814in}{0.681783in}}%
\pgfpathcurveto{\pgfqpoint{0.982518in}{0.689488in}}{\pgfqpoint{0.986847in}{0.699938in}}{\pgfqpoint{0.986847in}{0.710834in}}%
\pgfpathcurveto{\pgfqpoint{0.986847in}{0.721730in}}{\pgfqpoint{0.982518in}{0.732180in}}{\pgfqpoint{0.974814in}{0.739885in}}%
\pgfpathcurveto{\pgfqpoint{0.967110in}{0.747589in}}{\pgfqpoint{0.956659in}{0.751918in}}{\pgfqpoint{0.945763in}{0.751918in}}%
\pgfpathcurveto{\pgfqpoint{0.934868in}{0.751918in}}{\pgfqpoint{0.924417in}{0.747589in}}{\pgfqpoint{0.916713in}{0.739885in}}%
\pgfpathcurveto{\pgfqpoint{0.909008in}{0.732180in}}{\pgfqpoint{0.904679in}{0.721730in}}{\pgfqpoint{0.904679in}{0.710834in}}%
\pgfpathcurveto{\pgfqpoint{0.904679in}{0.699938in}}{\pgfqpoint{0.909008in}{0.689488in}}{\pgfqpoint{0.916713in}{0.681783in}}%
\pgfpathcurveto{\pgfqpoint{0.924417in}{0.674079in}}{\pgfqpoint{0.934868in}{0.669750in}}{\pgfqpoint{0.945763in}{0.669750in}}%
\pgfpathlineto{\pgfqpoint{0.945763in}{0.669750in}}%
\pgfpathclose%
\pgfusepath{stroke}%
\end{pgfscope}%
\begin{pgfscope}%
\pgfpathrectangle{\pgfqpoint{0.688192in}{0.670138in}}{\pgfqpoint{7.111808in}{5.061530in}}%
\pgfusepath{clip}%
\pgfsetbuttcap%
\pgfsetroundjoin%
\pgfsetlinewidth{1.003750pt}%
\definecolor{currentstroke}{rgb}{0.000000,0.000000,0.000000}%
\pgfsetstrokecolor{currentstroke}%
\pgfsetdash{}{0pt}%
\pgfpathmoveto{\pgfqpoint{1.903233in}{0.640199in}}%
\pgfpathcurveto{\pgfqpoint{1.914129in}{0.640199in}}{\pgfqpoint{1.924579in}{0.644527in}}{\pgfqpoint{1.932284in}{0.652232in}}%
\pgfpathcurveto{\pgfqpoint{1.939988in}{0.659936in}}{\pgfqpoint{1.944317in}{0.670387in}}{\pgfqpoint{1.944317in}{0.681282in}}%
\pgfpathcurveto{\pgfqpoint{1.944317in}{0.692178in}}{\pgfqpoint{1.939988in}{0.702629in}}{\pgfqpoint{1.932284in}{0.710333in}}%
\pgfpathcurveto{\pgfqpoint{1.924579in}{0.718038in}}{\pgfqpoint{1.914129in}{0.722366in}}{\pgfqpoint{1.903233in}{0.722366in}}%
\pgfpathcurveto{\pgfqpoint{1.892337in}{0.722366in}}{\pgfqpoint{1.881887in}{0.718038in}}{\pgfqpoint{1.874182in}{0.710333in}}%
\pgfpathcurveto{\pgfqpoint{1.866478in}{0.702629in}}{\pgfqpoint{1.862149in}{0.692178in}}{\pgfqpoint{1.862149in}{0.681282in}}%
\pgfpathcurveto{\pgfqpoint{1.862149in}{0.670387in}}{\pgfqpoint{1.866478in}{0.659936in}}{\pgfqpoint{1.874182in}{0.652232in}}%
\pgfpathcurveto{\pgfqpoint{1.881887in}{0.644527in}}{\pgfqpoint{1.892337in}{0.640199in}}{\pgfqpoint{1.903233in}{0.640199in}}%
\pgfusepath{stroke}%
\end{pgfscope}%
\begin{pgfscope}%
\pgfpathrectangle{\pgfqpoint{0.688192in}{0.670138in}}{\pgfqpoint{7.111808in}{5.061530in}}%
\pgfusepath{clip}%
\pgfsetbuttcap%
\pgfsetroundjoin%
\pgfsetlinewidth{1.003750pt}%
\definecolor{currentstroke}{rgb}{0.000000,0.000000,0.000000}%
\pgfsetstrokecolor{currentstroke}%
\pgfsetdash{}{0pt}%
\pgfpathmoveto{\pgfqpoint{2.500957in}{2.186214in}}%
\pgfpathcurveto{\pgfqpoint{2.511852in}{2.186214in}}{\pgfqpoint{2.522303in}{2.190543in}}{\pgfqpoint{2.530007in}{2.198247in}}%
\pgfpathcurveto{\pgfqpoint{2.537712in}{2.205951in}}{\pgfqpoint{2.542041in}{2.216402in}}{\pgfqpoint{2.542041in}{2.227298in}}%
\pgfpathcurveto{\pgfqpoint{2.542041in}{2.238193in}}{\pgfqpoint{2.537712in}{2.248644in}}{\pgfqpoint{2.530007in}{2.256349in}}%
\pgfpathcurveto{\pgfqpoint{2.522303in}{2.264053in}}{\pgfqpoint{2.511852in}{2.268382in}}{\pgfqpoint{2.500957in}{2.268382in}}%
\pgfpathcurveto{\pgfqpoint{2.490061in}{2.268382in}}{\pgfqpoint{2.479610in}{2.264053in}}{\pgfqpoint{2.471906in}{2.256349in}}%
\pgfpathcurveto{\pgfqpoint{2.464202in}{2.248644in}}{\pgfqpoint{2.459873in}{2.238193in}}{\pgfqpoint{2.459873in}{2.227298in}}%
\pgfpathcurveto{\pgfqpoint{2.459873in}{2.216402in}}{\pgfqpoint{2.464202in}{2.205951in}}{\pgfqpoint{2.471906in}{2.198247in}}%
\pgfpathcurveto{\pgfqpoint{2.479610in}{2.190543in}}{\pgfqpoint{2.490061in}{2.186214in}}{\pgfqpoint{2.500957in}{2.186214in}}%
\pgfpathlineto{\pgfqpoint{2.500957in}{2.186214in}}%
\pgfpathclose%
\pgfusepath{stroke}%
\end{pgfscope}%
\begin{pgfscope}%
\pgfpathrectangle{\pgfqpoint{0.688192in}{0.670138in}}{\pgfqpoint{7.111808in}{5.061530in}}%
\pgfusepath{clip}%
\pgfsetbuttcap%
\pgfsetroundjoin%
\pgfsetlinewidth{1.003750pt}%
\definecolor{currentstroke}{rgb}{0.000000,0.000000,0.000000}%
\pgfsetstrokecolor{currentstroke}%
\pgfsetdash{}{0pt}%
\pgfpathmoveto{\pgfqpoint{1.663205in}{0.641981in}}%
\pgfpathcurveto{\pgfqpoint{1.674101in}{0.641981in}}{\pgfqpoint{1.684552in}{0.646310in}}{\pgfqpoint{1.692256in}{0.654014in}}%
\pgfpathcurveto{\pgfqpoint{1.699960in}{0.661719in}}{\pgfqpoint{1.704289in}{0.672169in}}{\pgfqpoint{1.704289in}{0.683065in}}%
\pgfpathcurveto{\pgfqpoint{1.704289in}{0.693961in}}{\pgfqpoint{1.699960in}{0.704411in}}{\pgfqpoint{1.692256in}{0.712116in}}%
\pgfpathcurveto{\pgfqpoint{1.684552in}{0.719820in}}{\pgfqpoint{1.674101in}{0.724149in}}{\pgfqpoint{1.663205in}{0.724149in}}%
\pgfpathcurveto{\pgfqpoint{1.652310in}{0.724149in}}{\pgfqpoint{1.641859in}{0.719820in}}{\pgfqpoint{1.634155in}{0.712116in}}%
\pgfpathcurveto{\pgfqpoint{1.626450in}{0.704411in}}{\pgfqpoint{1.622121in}{0.693961in}}{\pgfqpoint{1.622121in}{0.683065in}}%
\pgfpathcurveto{\pgfqpoint{1.622121in}{0.672169in}}{\pgfqpoint{1.626450in}{0.661719in}}{\pgfqpoint{1.634155in}{0.654014in}}%
\pgfpathcurveto{\pgfqpoint{1.641859in}{0.646310in}}{\pgfqpoint{1.652310in}{0.641981in}}{\pgfqpoint{1.663205in}{0.641981in}}%
\pgfusepath{stroke}%
\end{pgfscope}%
\begin{pgfscope}%
\pgfpathrectangle{\pgfqpoint{0.688192in}{0.670138in}}{\pgfqpoint{7.111808in}{5.061530in}}%
\pgfusepath{clip}%
\pgfsetbuttcap%
\pgfsetroundjoin%
\pgfsetlinewidth{1.003750pt}%
\definecolor{currentstroke}{rgb}{0.000000,0.000000,0.000000}%
\pgfsetstrokecolor{currentstroke}%
\pgfsetdash{}{0pt}%
\pgfpathmoveto{\pgfqpoint{3.224983in}{2.235929in}}%
\pgfpathcurveto{\pgfqpoint{3.235878in}{2.235929in}}{\pgfqpoint{3.246329in}{2.240258in}}{\pgfqpoint{3.254033in}{2.247962in}}%
\pgfpathcurveto{\pgfqpoint{3.261738in}{2.255666in}}{\pgfqpoint{3.266067in}{2.266117in}}{\pgfqpoint{3.266067in}{2.277013in}}%
\pgfpathcurveto{\pgfqpoint{3.266067in}{2.287908in}}{\pgfqpoint{3.261738in}{2.298359in}}{\pgfqpoint{3.254033in}{2.306063in}}%
\pgfpathcurveto{\pgfqpoint{3.246329in}{2.313768in}}{\pgfqpoint{3.235878in}{2.318096in}}{\pgfqpoint{3.224983in}{2.318096in}}%
\pgfpathcurveto{\pgfqpoint{3.214087in}{2.318096in}}{\pgfqpoint{3.203636in}{2.313768in}}{\pgfqpoint{3.195932in}{2.306063in}}%
\pgfpathcurveto{\pgfqpoint{3.188228in}{2.298359in}}{\pgfqpoint{3.183899in}{2.287908in}}{\pgfqpoint{3.183899in}{2.277013in}}%
\pgfpathcurveto{\pgfqpoint{3.183899in}{2.266117in}}{\pgfqpoint{3.188228in}{2.255666in}}{\pgfqpoint{3.195932in}{2.247962in}}%
\pgfpathcurveto{\pgfqpoint{3.203636in}{2.240258in}}{\pgfqpoint{3.214087in}{2.235929in}}{\pgfqpoint{3.224983in}{2.235929in}}%
\pgfpathlineto{\pgfqpoint{3.224983in}{2.235929in}}%
\pgfpathclose%
\pgfusepath{stroke}%
\end{pgfscope}%
\begin{pgfscope}%
\pgfpathrectangle{\pgfqpoint{0.688192in}{0.670138in}}{\pgfqpoint{7.111808in}{5.061530in}}%
\pgfusepath{clip}%
\pgfsetbuttcap%
\pgfsetroundjoin%
\pgfsetlinewidth{1.003750pt}%
\definecolor{currentstroke}{rgb}{0.000000,0.000000,0.000000}%
\pgfsetstrokecolor{currentstroke}%
\pgfsetdash{}{0pt}%
\pgfpathmoveto{\pgfqpoint{5.252788in}{2.395935in}}%
\pgfpathcurveto{\pgfqpoint{5.263684in}{2.395935in}}{\pgfqpoint{5.274134in}{2.400263in}}{\pgfqpoint{5.281839in}{2.407968in}}%
\pgfpathcurveto{\pgfqpoint{5.289543in}{2.415672in}}{\pgfqpoint{5.293872in}{2.426123in}}{\pgfqpoint{5.293872in}{2.437018in}}%
\pgfpathcurveto{\pgfqpoint{5.293872in}{2.447914in}}{\pgfqpoint{5.289543in}{2.458365in}}{\pgfqpoint{5.281839in}{2.466069in}}%
\pgfpathcurveto{\pgfqpoint{5.274134in}{2.473774in}}{\pgfqpoint{5.263684in}{2.478102in}}{\pgfqpoint{5.252788in}{2.478102in}}%
\pgfpathcurveto{\pgfqpoint{5.241893in}{2.478102in}}{\pgfqpoint{5.231442in}{2.473774in}}{\pgfqpoint{5.223737in}{2.466069in}}%
\pgfpathcurveto{\pgfqpoint{5.216033in}{2.458365in}}{\pgfqpoint{5.211704in}{2.447914in}}{\pgfqpoint{5.211704in}{2.437018in}}%
\pgfpathcurveto{\pgfqpoint{5.211704in}{2.426123in}}{\pgfqpoint{5.216033in}{2.415672in}}{\pgfqpoint{5.223737in}{2.407968in}}%
\pgfpathcurveto{\pgfqpoint{5.231442in}{2.400263in}}{\pgfqpoint{5.241893in}{2.395935in}}{\pgfqpoint{5.252788in}{2.395935in}}%
\pgfpathlineto{\pgfqpoint{5.252788in}{2.395935in}}%
\pgfpathclose%
\pgfusepath{stroke}%
\end{pgfscope}%
\begin{pgfscope}%
\pgfpathrectangle{\pgfqpoint{0.688192in}{0.670138in}}{\pgfqpoint{7.111808in}{5.061530in}}%
\pgfusepath{clip}%
\pgfsetbuttcap%
\pgfsetroundjoin%
\pgfsetlinewidth{1.003750pt}%
\definecolor{currentstroke}{rgb}{0.000000,0.000000,0.000000}%
\pgfsetstrokecolor{currentstroke}%
\pgfsetdash{}{0pt}%
\pgfpathmoveto{\pgfqpoint{1.444241in}{0.642951in}}%
\pgfpathcurveto{\pgfqpoint{1.455136in}{0.642951in}}{\pgfqpoint{1.465587in}{0.647280in}}{\pgfqpoint{1.473291in}{0.654984in}}%
\pgfpathcurveto{\pgfqpoint{1.480996in}{0.662688in}}{\pgfqpoint{1.485325in}{0.673139in}}{\pgfqpoint{1.485325in}{0.684035in}}%
\pgfpathcurveto{\pgfqpoint{1.485325in}{0.694930in}}{\pgfqpoint{1.480996in}{0.705381in}}{\pgfqpoint{1.473291in}{0.713085in}}%
\pgfpathcurveto{\pgfqpoint{1.465587in}{0.720790in}}{\pgfqpoint{1.455136in}{0.725119in}}{\pgfqpoint{1.444241in}{0.725119in}}%
\pgfpathcurveto{\pgfqpoint{1.433345in}{0.725119in}}{\pgfqpoint{1.422894in}{0.720790in}}{\pgfqpoint{1.415190in}{0.713085in}}%
\pgfpathcurveto{\pgfqpoint{1.407486in}{0.705381in}}{\pgfqpoint{1.403157in}{0.694930in}}{\pgfqpoint{1.403157in}{0.684035in}}%
\pgfpathcurveto{\pgfqpoint{1.403157in}{0.673139in}}{\pgfqpoint{1.407486in}{0.662688in}}{\pgfqpoint{1.415190in}{0.654984in}}%
\pgfpathcurveto{\pgfqpoint{1.422894in}{0.647280in}}{\pgfqpoint{1.433345in}{0.642951in}}{\pgfqpoint{1.444241in}{0.642951in}}%
\pgfusepath{stroke}%
\end{pgfscope}%
\begin{pgfscope}%
\pgfpathrectangle{\pgfqpoint{0.688192in}{0.670138in}}{\pgfqpoint{7.111808in}{5.061530in}}%
\pgfusepath{clip}%
\pgfsetbuttcap%
\pgfsetroundjoin%
\pgfsetlinewidth{1.003750pt}%
\definecolor{currentstroke}{rgb}{0.000000,0.000000,0.000000}%
\pgfsetstrokecolor{currentstroke}%
\pgfsetdash{}{0pt}%
\pgfpathmoveto{\pgfqpoint{1.885375in}{3.246009in}}%
\pgfpathcurveto{\pgfqpoint{1.896270in}{3.246009in}}{\pgfqpoint{1.906721in}{3.250338in}}{\pgfqpoint{1.914425in}{3.258042in}}%
\pgfpathcurveto{\pgfqpoint{1.922130in}{3.265746in}}{\pgfqpoint{1.926458in}{3.276197in}}{\pgfqpoint{1.926458in}{3.287093in}}%
\pgfpathcurveto{\pgfqpoint{1.926458in}{3.297988in}}{\pgfqpoint{1.922130in}{3.308439in}}{\pgfqpoint{1.914425in}{3.316144in}}%
\pgfpathcurveto{\pgfqpoint{1.906721in}{3.323848in}}{\pgfqpoint{1.896270in}{3.328177in}}{\pgfqpoint{1.885375in}{3.328177in}}%
\pgfpathcurveto{\pgfqpoint{1.874479in}{3.328177in}}{\pgfqpoint{1.864028in}{3.323848in}}{\pgfqpoint{1.856324in}{3.316144in}}%
\pgfpathcurveto{\pgfqpoint{1.848620in}{3.308439in}}{\pgfqpoint{1.844291in}{3.297988in}}{\pgfqpoint{1.844291in}{3.287093in}}%
\pgfpathcurveto{\pgfqpoint{1.844291in}{3.276197in}}{\pgfqpoint{1.848620in}{3.265746in}}{\pgfqpoint{1.856324in}{3.258042in}}%
\pgfpathcurveto{\pgfqpoint{1.864028in}{3.250338in}}{\pgfqpoint{1.874479in}{3.246009in}}{\pgfqpoint{1.885375in}{3.246009in}}%
\pgfpathlineto{\pgfqpoint{1.885375in}{3.246009in}}%
\pgfpathclose%
\pgfusepath{stroke}%
\end{pgfscope}%
\begin{pgfscope}%
\pgfpathrectangle{\pgfqpoint{0.688192in}{0.670138in}}{\pgfqpoint{7.111808in}{5.061530in}}%
\pgfusepath{clip}%
\pgfsetbuttcap%
\pgfsetroundjoin%
\pgfsetlinewidth{1.003750pt}%
\definecolor{currentstroke}{rgb}{0.000000,0.000000,0.000000}%
\pgfsetstrokecolor{currentstroke}%
\pgfsetdash{}{0pt}%
\pgfpathmoveto{\pgfqpoint{4.353375in}{1.924389in}}%
\pgfpathcurveto{\pgfqpoint{4.364271in}{1.924389in}}{\pgfqpoint{4.374721in}{1.928718in}}{\pgfqpoint{4.382426in}{1.936422in}}%
\pgfpathcurveto{\pgfqpoint{4.390130in}{1.944126in}}{\pgfqpoint{4.394459in}{1.954577in}}{\pgfqpoint{4.394459in}{1.965473in}}%
\pgfpathcurveto{\pgfqpoint{4.394459in}{1.976368in}}{\pgfqpoint{4.390130in}{1.986819in}}{\pgfqpoint{4.382426in}{1.994524in}}%
\pgfpathcurveto{\pgfqpoint{4.374721in}{2.002228in}}{\pgfqpoint{4.364271in}{2.006557in}}{\pgfqpoint{4.353375in}{2.006557in}}%
\pgfpathcurveto{\pgfqpoint{4.342480in}{2.006557in}}{\pgfqpoint{4.332029in}{2.002228in}}{\pgfqpoint{4.324324in}{1.994524in}}%
\pgfpathcurveto{\pgfqpoint{4.316620in}{1.986819in}}{\pgfqpoint{4.312291in}{1.976368in}}{\pgfqpoint{4.312291in}{1.965473in}}%
\pgfpathcurveto{\pgfqpoint{4.312291in}{1.954577in}}{\pgfqpoint{4.316620in}{1.944126in}}{\pgfqpoint{4.324324in}{1.936422in}}%
\pgfpathcurveto{\pgfqpoint{4.332029in}{1.928718in}}{\pgfqpoint{4.342480in}{1.924389in}}{\pgfqpoint{4.353375in}{1.924389in}}%
\pgfpathlineto{\pgfqpoint{4.353375in}{1.924389in}}%
\pgfpathclose%
\pgfusepath{stroke}%
\end{pgfscope}%
\begin{pgfscope}%
\pgfpathrectangle{\pgfqpoint{0.688192in}{0.670138in}}{\pgfqpoint{7.111808in}{5.061530in}}%
\pgfusepath{clip}%
\pgfsetbuttcap%
\pgfsetroundjoin%
\pgfsetlinewidth{1.003750pt}%
\definecolor{currentstroke}{rgb}{0.000000,0.000000,0.000000}%
\pgfsetstrokecolor{currentstroke}%
\pgfsetdash{}{0pt}%
\pgfpathmoveto{\pgfqpoint{1.407033in}{0.642986in}}%
\pgfpathcurveto{\pgfqpoint{1.417928in}{0.642986in}}{\pgfqpoint{1.428379in}{0.647315in}}{\pgfqpoint{1.436083in}{0.655020in}}%
\pgfpathcurveto{\pgfqpoint{1.443788in}{0.662724in}}{\pgfqpoint{1.448117in}{0.673175in}}{\pgfqpoint{1.448117in}{0.684070in}}%
\pgfpathcurveto{\pgfqpoint{1.448117in}{0.694966in}}{\pgfqpoint{1.443788in}{0.705417in}}{\pgfqpoint{1.436083in}{0.713121in}}%
\pgfpathcurveto{\pgfqpoint{1.428379in}{0.720825in}}{\pgfqpoint{1.417928in}{0.725154in}}{\pgfqpoint{1.407033in}{0.725154in}}%
\pgfpathcurveto{\pgfqpoint{1.396137in}{0.725154in}}{\pgfqpoint{1.385686in}{0.720825in}}{\pgfqpoint{1.377982in}{0.713121in}}%
\pgfpathcurveto{\pgfqpoint{1.370278in}{0.705417in}}{\pgfqpoint{1.365949in}{0.694966in}}{\pgfqpoint{1.365949in}{0.684070in}}%
\pgfpathcurveto{\pgfqpoint{1.365949in}{0.673175in}}{\pgfqpoint{1.370278in}{0.662724in}}{\pgfqpoint{1.377982in}{0.655020in}}%
\pgfpathcurveto{\pgfqpoint{1.385686in}{0.647315in}}{\pgfqpoint{1.396137in}{0.642986in}}{\pgfqpoint{1.407033in}{0.642986in}}%
\pgfusepath{stroke}%
\end{pgfscope}%
\begin{pgfscope}%
\pgfpathrectangle{\pgfqpoint{0.688192in}{0.670138in}}{\pgfqpoint{7.111808in}{5.061530in}}%
\pgfusepath{clip}%
\pgfsetbuttcap%
\pgfsetroundjoin%
\pgfsetlinewidth{1.003750pt}%
\definecolor{currentstroke}{rgb}{0.000000,0.000000,0.000000}%
\pgfsetstrokecolor{currentstroke}%
\pgfsetdash{}{0pt}%
\pgfpathmoveto{\pgfqpoint{0.942062in}{0.671110in}}%
\pgfpathcurveto{\pgfqpoint{0.952958in}{0.671110in}}{\pgfqpoint{0.963408in}{0.675439in}}{\pgfqpoint{0.971113in}{0.683143in}}%
\pgfpathcurveto{\pgfqpoint{0.978817in}{0.690847in}}{\pgfqpoint{0.983146in}{0.701298in}}{\pgfqpoint{0.983146in}{0.712194in}}%
\pgfpathcurveto{\pgfqpoint{0.983146in}{0.723089in}}{\pgfqpoint{0.978817in}{0.733540in}}{\pgfqpoint{0.971113in}{0.741244in}}%
\pgfpathcurveto{\pgfqpoint{0.963408in}{0.748949in}}{\pgfqpoint{0.952958in}{0.753278in}}{\pgfqpoint{0.942062in}{0.753278in}}%
\pgfpathcurveto{\pgfqpoint{0.931166in}{0.753278in}}{\pgfqpoint{0.920716in}{0.748949in}}{\pgfqpoint{0.913011in}{0.741244in}}%
\pgfpathcurveto{\pgfqpoint{0.905307in}{0.733540in}}{\pgfqpoint{0.900978in}{0.723089in}}{\pgfqpoint{0.900978in}{0.712194in}}%
\pgfpathcurveto{\pgfqpoint{0.900978in}{0.701298in}}{\pgfqpoint{0.905307in}{0.690847in}}{\pgfqpoint{0.913011in}{0.683143in}}%
\pgfpathcurveto{\pgfqpoint{0.920716in}{0.675439in}}{\pgfqpoint{0.931166in}{0.671110in}}{\pgfqpoint{0.942062in}{0.671110in}}%
\pgfpathlineto{\pgfqpoint{0.942062in}{0.671110in}}%
\pgfpathclose%
\pgfusepath{stroke}%
\end{pgfscope}%
\begin{pgfscope}%
\pgfpathrectangle{\pgfqpoint{0.688192in}{0.670138in}}{\pgfqpoint{7.111808in}{5.061530in}}%
\pgfusepath{clip}%
\pgfsetbuttcap%
\pgfsetroundjoin%
\pgfsetlinewidth{1.003750pt}%
\definecolor{currentstroke}{rgb}{0.000000,0.000000,0.000000}%
\pgfsetstrokecolor{currentstroke}%
\pgfsetdash{}{0pt}%
\pgfpathmoveto{\pgfqpoint{5.254247in}{0.690349in}}%
\pgfpathcurveto{\pgfqpoint{5.265142in}{0.690349in}}{\pgfqpoint{5.275593in}{0.694678in}}{\pgfqpoint{5.283298in}{0.702382in}}%
\pgfpathcurveto{\pgfqpoint{5.291002in}{0.710086in}}{\pgfqpoint{5.295331in}{0.720537in}}{\pgfqpoint{5.295331in}{0.731433in}}%
\pgfpathcurveto{\pgfqpoint{5.295331in}{0.742328in}}{\pgfqpoint{5.291002in}{0.752779in}}{\pgfqpoint{5.283298in}{0.760484in}}%
\pgfpathcurveto{\pgfqpoint{5.275593in}{0.768188in}}{\pgfqpoint{5.265142in}{0.772517in}}{\pgfqpoint{5.254247in}{0.772517in}}%
\pgfpathcurveto{\pgfqpoint{5.243351in}{0.772517in}}{\pgfqpoint{5.232901in}{0.768188in}}{\pgfqpoint{5.225196in}{0.760484in}}%
\pgfpathcurveto{\pgfqpoint{5.217492in}{0.752779in}}{\pgfqpoint{5.213163in}{0.742328in}}{\pgfqpoint{5.213163in}{0.731433in}}%
\pgfpathcurveto{\pgfqpoint{5.213163in}{0.720537in}}{\pgfqpoint{5.217492in}{0.710086in}}{\pgfqpoint{5.225196in}{0.702382in}}%
\pgfpathcurveto{\pgfqpoint{5.232901in}{0.694678in}}{\pgfqpoint{5.243351in}{0.690349in}}{\pgfqpoint{5.254247in}{0.690349in}}%
\pgfpathlineto{\pgfqpoint{5.254247in}{0.690349in}}%
\pgfpathclose%
\pgfusepath{stroke}%
\end{pgfscope}%
\begin{pgfscope}%
\pgfpathrectangle{\pgfqpoint{0.688192in}{0.670138in}}{\pgfqpoint{7.111808in}{5.061530in}}%
\pgfusepath{clip}%
\pgfsetbuttcap%
\pgfsetroundjoin%
\pgfsetlinewidth{1.003750pt}%
\definecolor{currentstroke}{rgb}{0.000000,0.000000,0.000000}%
\pgfsetstrokecolor{currentstroke}%
\pgfsetdash{}{0pt}%
\pgfpathmoveto{\pgfqpoint{3.169391in}{2.270687in}}%
\pgfpathcurveto{\pgfqpoint{3.180287in}{2.270687in}}{\pgfqpoint{3.190737in}{2.275016in}}{\pgfqpoint{3.198442in}{2.282720in}}%
\pgfpathcurveto{\pgfqpoint{3.206146in}{2.290424in}}{\pgfqpoint{3.210475in}{2.300875in}}{\pgfqpoint{3.210475in}{2.311771in}}%
\pgfpathcurveto{\pgfqpoint{3.210475in}{2.322666in}}{\pgfqpoint{3.206146in}{2.333117in}}{\pgfqpoint{3.198442in}{2.340821in}}%
\pgfpathcurveto{\pgfqpoint{3.190737in}{2.348526in}}{\pgfqpoint{3.180287in}{2.352854in}}{\pgfqpoint{3.169391in}{2.352854in}}%
\pgfpathcurveto{\pgfqpoint{3.158495in}{2.352854in}}{\pgfqpoint{3.148045in}{2.348526in}}{\pgfqpoint{3.140340in}{2.340821in}}%
\pgfpathcurveto{\pgfqpoint{3.132636in}{2.333117in}}{\pgfqpoint{3.128307in}{2.322666in}}{\pgfqpoint{3.128307in}{2.311771in}}%
\pgfpathcurveto{\pgfqpoint{3.128307in}{2.300875in}}{\pgfqpoint{3.132636in}{2.290424in}}{\pgfqpoint{3.140340in}{2.282720in}}%
\pgfpathcurveto{\pgfqpoint{3.148045in}{2.275016in}}{\pgfqpoint{3.158495in}{2.270687in}}{\pgfqpoint{3.169391in}{2.270687in}}%
\pgfpathlineto{\pgfqpoint{3.169391in}{2.270687in}}%
\pgfpathclose%
\pgfusepath{stroke}%
\end{pgfscope}%
\begin{pgfscope}%
\pgfpathrectangle{\pgfqpoint{0.688192in}{0.670138in}}{\pgfqpoint{7.111808in}{5.061530in}}%
\pgfusepath{clip}%
\pgfsetbuttcap%
\pgfsetroundjoin%
\pgfsetlinewidth{1.003750pt}%
\definecolor{currentstroke}{rgb}{0.000000,0.000000,0.000000}%
\pgfsetstrokecolor{currentstroke}%
\pgfsetdash{}{0pt}%
\pgfpathmoveto{\pgfqpoint{1.477331in}{0.642791in}}%
\pgfpathcurveto{\pgfqpoint{1.488227in}{0.642791in}}{\pgfqpoint{1.498677in}{0.647120in}}{\pgfqpoint{1.506382in}{0.654824in}}%
\pgfpathcurveto{\pgfqpoint{1.514086in}{0.662528in}}{\pgfqpoint{1.518415in}{0.672979in}}{\pgfqpoint{1.518415in}{0.683875in}}%
\pgfpathcurveto{\pgfqpoint{1.518415in}{0.694770in}}{\pgfqpoint{1.514086in}{0.705221in}}{\pgfqpoint{1.506382in}{0.712926in}}%
\pgfpathcurveto{\pgfqpoint{1.498677in}{0.720630in}}{\pgfqpoint{1.488227in}{0.724959in}}{\pgfqpoint{1.477331in}{0.724959in}}%
\pgfpathcurveto{\pgfqpoint{1.466435in}{0.724959in}}{\pgfqpoint{1.455985in}{0.720630in}}{\pgfqpoint{1.448280in}{0.712926in}}%
\pgfpathcurveto{\pgfqpoint{1.440576in}{0.705221in}}{\pgfqpoint{1.436247in}{0.694770in}}{\pgfqpoint{1.436247in}{0.683875in}}%
\pgfpathcurveto{\pgfqpoint{1.436247in}{0.672979in}}{\pgfqpoint{1.440576in}{0.662528in}}{\pgfqpoint{1.448280in}{0.654824in}}%
\pgfpathcurveto{\pgfqpoint{1.455985in}{0.647120in}}{\pgfqpoint{1.466435in}{0.642791in}}{\pgfqpoint{1.477331in}{0.642791in}}%
\pgfusepath{stroke}%
\end{pgfscope}%
\begin{pgfscope}%
\pgfpathrectangle{\pgfqpoint{0.688192in}{0.670138in}}{\pgfqpoint{7.111808in}{5.061530in}}%
\pgfusepath{clip}%
\pgfsetbuttcap%
\pgfsetroundjoin%
\pgfsetlinewidth{1.003750pt}%
\definecolor{currentstroke}{rgb}{0.000000,0.000000,0.000000}%
\pgfsetstrokecolor{currentstroke}%
\pgfsetdash{}{0pt}%
\pgfpathmoveto{\pgfqpoint{4.957579in}{2.334771in}}%
\pgfpathcurveto{\pgfqpoint{4.968474in}{2.334771in}}{\pgfqpoint{4.978925in}{2.339100in}}{\pgfqpoint{4.986629in}{2.346804in}}%
\pgfpathcurveto{\pgfqpoint{4.994334in}{2.354509in}}{\pgfqpoint{4.998662in}{2.364959in}}{\pgfqpoint{4.998662in}{2.375855in}}%
\pgfpathcurveto{\pgfqpoint{4.998662in}{2.386750in}}{\pgfqpoint{4.994334in}{2.397201in}}{\pgfqpoint{4.986629in}{2.404906in}}%
\pgfpathcurveto{\pgfqpoint{4.978925in}{2.412610in}}{\pgfqpoint{4.968474in}{2.416939in}}{\pgfqpoint{4.957579in}{2.416939in}}%
\pgfpathcurveto{\pgfqpoint{4.946683in}{2.416939in}}{\pgfqpoint{4.936232in}{2.412610in}}{\pgfqpoint{4.928528in}{2.404906in}}%
\pgfpathcurveto{\pgfqpoint{4.920824in}{2.397201in}}{\pgfqpoint{4.916495in}{2.386750in}}{\pgfqpoint{4.916495in}{2.375855in}}%
\pgfpathcurveto{\pgfqpoint{4.916495in}{2.364959in}}{\pgfqpoint{4.920824in}{2.354509in}}{\pgfqpoint{4.928528in}{2.346804in}}%
\pgfpathcurveto{\pgfqpoint{4.936232in}{2.339100in}}{\pgfqpoint{4.946683in}{2.334771in}}{\pgfqpoint{4.957579in}{2.334771in}}%
\pgfpathlineto{\pgfqpoint{4.957579in}{2.334771in}}%
\pgfpathclose%
\pgfusepath{stroke}%
\end{pgfscope}%
\begin{pgfscope}%
\pgfpathrectangle{\pgfqpoint{0.688192in}{0.670138in}}{\pgfqpoint{7.111808in}{5.061530in}}%
\pgfusepath{clip}%
\pgfsetbuttcap%
\pgfsetroundjoin%
\pgfsetlinewidth{1.003750pt}%
\definecolor{currentstroke}{rgb}{0.000000,0.000000,0.000000}%
\pgfsetstrokecolor{currentstroke}%
\pgfsetdash{}{0pt}%
\pgfpathmoveto{\pgfqpoint{1.738010in}{0.641235in}}%
\pgfpathcurveto{\pgfqpoint{1.748906in}{0.641235in}}{\pgfqpoint{1.759357in}{0.645564in}}{\pgfqpoint{1.767061in}{0.653269in}}%
\pgfpathcurveto{\pgfqpoint{1.774765in}{0.660973in}}{\pgfqpoint{1.779094in}{0.671424in}}{\pgfqpoint{1.779094in}{0.682319in}}%
\pgfpathcurveto{\pgfqpoint{1.779094in}{0.693215in}}{\pgfqpoint{1.774765in}{0.703666in}}{\pgfqpoint{1.767061in}{0.711370in}}%
\pgfpathcurveto{\pgfqpoint{1.759357in}{0.719074in}}{\pgfqpoint{1.748906in}{0.723403in}}{\pgfqpoint{1.738010in}{0.723403in}}%
\pgfpathcurveto{\pgfqpoint{1.727115in}{0.723403in}}{\pgfqpoint{1.716664in}{0.719074in}}{\pgfqpoint{1.708960in}{0.711370in}}%
\pgfpathcurveto{\pgfqpoint{1.701255in}{0.703666in}}{\pgfqpoint{1.696926in}{0.693215in}}{\pgfqpoint{1.696926in}{0.682319in}}%
\pgfpathcurveto{\pgfqpoint{1.696926in}{0.671424in}}{\pgfqpoint{1.701255in}{0.660973in}}{\pgfqpoint{1.708960in}{0.653269in}}%
\pgfpathcurveto{\pgfqpoint{1.716664in}{0.645564in}}{\pgfqpoint{1.727115in}{0.641235in}}{\pgfqpoint{1.738010in}{0.641235in}}%
\pgfusepath{stroke}%
\end{pgfscope}%
\begin{pgfscope}%
\pgfpathrectangle{\pgfqpoint{0.688192in}{0.670138in}}{\pgfqpoint{7.111808in}{5.061530in}}%
\pgfusepath{clip}%
\pgfsetbuttcap%
\pgfsetroundjoin%
\pgfsetlinewidth{1.003750pt}%
\definecolor{currentstroke}{rgb}{0.000000,0.000000,0.000000}%
\pgfsetstrokecolor{currentstroke}%
\pgfsetdash{}{0pt}%
\pgfpathmoveto{\pgfqpoint{1.826952in}{0.640664in}}%
\pgfpathcurveto{\pgfqpoint{1.837847in}{0.640664in}}{\pgfqpoint{1.848298in}{0.644992in}}{\pgfqpoint{1.856002in}{0.652697in}}%
\pgfpathcurveto{\pgfqpoint{1.863707in}{0.660401in}}{\pgfqpoint{1.868035in}{0.670852in}}{\pgfqpoint{1.868035in}{0.681747in}}%
\pgfpathcurveto{\pgfqpoint{1.868035in}{0.692643in}}{\pgfqpoint{1.863707in}{0.703094in}}{\pgfqpoint{1.856002in}{0.710798in}}%
\pgfpathcurveto{\pgfqpoint{1.848298in}{0.718502in}}{\pgfqpoint{1.837847in}{0.722831in}}{\pgfqpoint{1.826952in}{0.722831in}}%
\pgfpathcurveto{\pgfqpoint{1.816056in}{0.722831in}}{\pgfqpoint{1.805605in}{0.718502in}}{\pgfqpoint{1.797901in}{0.710798in}}%
\pgfpathcurveto{\pgfqpoint{1.790197in}{0.703094in}}{\pgfqpoint{1.785868in}{0.692643in}}{\pgfqpoint{1.785868in}{0.681747in}}%
\pgfpathcurveto{\pgfqpoint{1.785868in}{0.670852in}}{\pgfqpoint{1.790197in}{0.660401in}}{\pgfqpoint{1.797901in}{0.652697in}}%
\pgfpathcurveto{\pgfqpoint{1.805605in}{0.644992in}}{\pgfqpoint{1.816056in}{0.640664in}}{\pgfqpoint{1.826952in}{0.640664in}}%
\pgfusepath{stroke}%
\end{pgfscope}%
\begin{pgfscope}%
\pgfpathrectangle{\pgfqpoint{0.688192in}{0.670138in}}{\pgfqpoint{7.111808in}{5.061530in}}%
\pgfusepath{clip}%
\pgfsetbuttcap%
\pgfsetroundjoin%
\pgfsetlinewidth{1.003750pt}%
\definecolor{currentstroke}{rgb}{0.000000,0.000000,0.000000}%
\pgfsetstrokecolor{currentstroke}%
\pgfsetdash{}{0pt}%
\pgfpathmoveto{\pgfqpoint{5.274608in}{0.629748in}}%
\pgfpathcurveto{\pgfqpoint{5.285503in}{0.629748in}}{\pgfqpoint{5.295954in}{0.634077in}}{\pgfqpoint{5.303659in}{0.641781in}}%
\pgfpathcurveto{\pgfqpoint{5.311363in}{0.649486in}}{\pgfqpoint{5.315692in}{0.659937in}}{\pgfqpoint{5.315692in}{0.670832in}}%
\pgfpathcurveto{\pgfqpoint{5.315692in}{0.681728in}}{\pgfqpoint{5.311363in}{0.692179in}}{\pgfqpoint{5.303659in}{0.699883in}}%
\pgfpathcurveto{\pgfqpoint{5.295954in}{0.707587in}}{\pgfqpoint{5.285503in}{0.711916in}}{\pgfqpoint{5.274608in}{0.711916in}}%
\pgfpathcurveto{\pgfqpoint{5.263712in}{0.711916in}}{\pgfqpoint{5.253261in}{0.707587in}}{\pgfqpoint{5.245557in}{0.699883in}}%
\pgfpathcurveto{\pgfqpoint{5.237853in}{0.692179in}}{\pgfqpoint{5.233524in}{0.681728in}}{\pgfqpoint{5.233524in}{0.670832in}}%
\pgfpathcurveto{\pgfqpoint{5.233524in}{0.659937in}}{\pgfqpoint{5.237853in}{0.649486in}}{\pgfqpoint{5.245557in}{0.641781in}}%
\pgfpathcurveto{\pgfqpoint{5.253261in}{0.634077in}}{\pgfqpoint{5.263712in}{0.629748in}}{\pgfqpoint{5.274608in}{0.629748in}}%
\pgfusepath{stroke}%
\end{pgfscope}%
\begin{pgfscope}%
\pgfpathrectangle{\pgfqpoint{0.688192in}{0.670138in}}{\pgfqpoint{7.111808in}{5.061530in}}%
\pgfusepath{clip}%
\pgfsetbuttcap%
\pgfsetroundjoin%
\pgfsetlinewidth{1.003750pt}%
\definecolor{currentstroke}{rgb}{0.000000,0.000000,0.000000}%
\pgfsetstrokecolor{currentstroke}%
\pgfsetdash{}{0pt}%
\pgfpathmoveto{\pgfqpoint{4.965126in}{0.630283in}}%
\pgfpathcurveto{\pgfqpoint{4.976022in}{0.630283in}}{\pgfqpoint{4.986472in}{0.634611in}}{\pgfqpoint{4.994177in}{0.642316in}}%
\pgfpathcurveto{\pgfqpoint{5.001881in}{0.650020in}}{\pgfqpoint{5.006210in}{0.660471in}}{\pgfqpoint{5.006210in}{0.671366in}}%
\pgfpathcurveto{\pgfqpoint{5.006210in}{0.682262in}}{\pgfqpoint{5.001881in}{0.692713in}}{\pgfqpoint{4.994177in}{0.700417in}}%
\pgfpathcurveto{\pgfqpoint{4.986472in}{0.708122in}}{\pgfqpoint{4.976022in}{0.712450in}}{\pgfqpoint{4.965126in}{0.712450in}}%
\pgfpathcurveto{\pgfqpoint{4.954230in}{0.712450in}}{\pgfqpoint{4.943780in}{0.708122in}}{\pgfqpoint{4.936075in}{0.700417in}}%
\pgfpathcurveto{\pgfqpoint{4.928371in}{0.692713in}}{\pgfqpoint{4.924042in}{0.682262in}}{\pgfqpoint{4.924042in}{0.671366in}}%
\pgfpathcurveto{\pgfqpoint{4.924042in}{0.660471in}}{\pgfqpoint{4.928371in}{0.650020in}}{\pgfqpoint{4.936075in}{0.642316in}}%
\pgfpathcurveto{\pgfqpoint{4.943780in}{0.634611in}}{\pgfqpoint{4.954230in}{0.630283in}}{\pgfqpoint{4.965126in}{0.630283in}}%
\pgfusepath{stroke}%
\end{pgfscope}%
\begin{pgfscope}%
\pgfpathrectangle{\pgfqpoint{0.688192in}{0.670138in}}{\pgfqpoint{7.111808in}{5.061530in}}%
\pgfusepath{clip}%
\pgfsetbuttcap%
\pgfsetroundjoin%
\pgfsetlinewidth{1.003750pt}%
\definecolor{currentstroke}{rgb}{0.000000,0.000000,0.000000}%
\pgfsetstrokecolor{currentstroke}%
\pgfsetdash{}{0pt}%
\pgfpathmoveto{\pgfqpoint{6.914205in}{3.400049in}}%
\pgfpathcurveto{\pgfqpoint{6.925101in}{3.400049in}}{\pgfqpoint{6.935552in}{3.404378in}}{\pgfqpoint{6.943256in}{3.412082in}}%
\pgfpathcurveto{\pgfqpoint{6.950960in}{3.419786in}}{\pgfqpoint{6.955289in}{3.430237in}}{\pgfqpoint{6.955289in}{3.441133in}}%
\pgfpathcurveto{\pgfqpoint{6.955289in}{3.452028in}}{\pgfqpoint{6.950960in}{3.462479in}}{\pgfqpoint{6.943256in}{3.470183in}}%
\pgfpathcurveto{\pgfqpoint{6.935552in}{3.477888in}}{\pgfqpoint{6.925101in}{3.482217in}}{\pgfqpoint{6.914205in}{3.482217in}}%
\pgfpathcurveto{\pgfqpoint{6.903310in}{3.482217in}}{\pgfqpoint{6.892859in}{3.477888in}}{\pgfqpoint{6.885155in}{3.470183in}}%
\pgfpathcurveto{\pgfqpoint{6.877450in}{3.462479in}}{\pgfqpoint{6.873121in}{3.452028in}}{\pgfqpoint{6.873121in}{3.441133in}}%
\pgfpathcurveto{\pgfqpoint{6.873121in}{3.430237in}}{\pgfqpoint{6.877450in}{3.419786in}}{\pgfqpoint{6.885155in}{3.412082in}}%
\pgfpathcurveto{\pgfqpoint{6.892859in}{3.404378in}}{\pgfqpoint{6.903310in}{3.400049in}}{\pgfqpoint{6.914205in}{3.400049in}}%
\pgfpathlineto{\pgfqpoint{6.914205in}{3.400049in}}%
\pgfpathclose%
\pgfusepath{stroke}%
\end{pgfscope}%
\begin{pgfscope}%
\pgfpathrectangle{\pgfqpoint{0.688192in}{0.670138in}}{\pgfqpoint{7.111808in}{5.061530in}}%
\pgfusepath{clip}%
\pgfsetbuttcap%
\pgfsetroundjoin%
\pgfsetlinewidth{1.003750pt}%
\definecolor{currentstroke}{rgb}{0.000000,0.000000,0.000000}%
\pgfsetstrokecolor{currentstroke}%
\pgfsetdash{}{0pt}%
\pgfpathmoveto{\pgfqpoint{0.990837in}{0.662198in}}%
\pgfpathcurveto{\pgfqpoint{1.001733in}{0.662198in}}{\pgfqpoint{1.012184in}{0.666527in}}{\pgfqpoint{1.019888in}{0.674231in}}%
\pgfpathcurveto{\pgfqpoint{1.027592in}{0.681936in}}{\pgfqpoint{1.031921in}{0.692386in}}{\pgfqpoint{1.031921in}{0.703282in}}%
\pgfpathcurveto{\pgfqpoint{1.031921in}{0.714178in}}{\pgfqpoint{1.027592in}{0.724628in}}{\pgfqpoint{1.019888in}{0.732333in}}%
\pgfpathcurveto{\pgfqpoint{1.012184in}{0.740037in}}{\pgfqpoint{1.001733in}{0.744366in}}{\pgfqpoint{0.990837in}{0.744366in}}%
\pgfpathcurveto{\pgfqpoint{0.979942in}{0.744366in}}{\pgfqpoint{0.969491in}{0.740037in}}{\pgfqpoint{0.961787in}{0.732333in}}%
\pgfpathcurveto{\pgfqpoint{0.954082in}{0.724628in}}{\pgfqpoint{0.949753in}{0.714178in}}{\pgfqpoint{0.949753in}{0.703282in}}%
\pgfpathcurveto{\pgfqpoint{0.949753in}{0.692386in}}{\pgfqpoint{0.954082in}{0.681936in}}{\pgfqpoint{0.961787in}{0.674231in}}%
\pgfpathcurveto{\pgfqpoint{0.969491in}{0.666527in}}{\pgfqpoint{0.979942in}{0.662198in}}{\pgfqpoint{0.990837in}{0.662198in}}%
\pgfusepath{stroke}%
\end{pgfscope}%
\begin{pgfscope}%
\pgfpathrectangle{\pgfqpoint{0.688192in}{0.670138in}}{\pgfqpoint{7.111808in}{5.061530in}}%
\pgfusepath{clip}%
\pgfsetbuttcap%
\pgfsetroundjoin%
\pgfsetlinewidth{1.003750pt}%
\definecolor{currentstroke}{rgb}{0.000000,0.000000,0.000000}%
\pgfsetstrokecolor{currentstroke}%
\pgfsetdash{}{0pt}%
\pgfpathmoveto{\pgfqpoint{6.070398in}{2.022877in}}%
\pgfpathcurveto{\pgfqpoint{6.081294in}{2.022877in}}{\pgfqpoint{6.091745in}{2.027206in}}{\pgfqpoint{6.099449in}{2.034910in}}%
\pgfpathcurveto{\pgfqpoint{6.107153in}{2.042614in}}{\pgfqpoint{6.111482in}{2.053065in}}{\pgfqpoint{6.111482in}{2.063961in}}%
\pgfpathcurveto{\pgfqpoint{6.111482in}{2.074856in}}{\pgfqpoint{6.107153in}{2.085307in}}{\pgfqpoint{6.099449in}{2.093011in}}%
\pgfpathcurveto{\pgfqpoint{6.091745in}{2.100716in}}{\pgfqpoint{6.081294in}{2.105045in}}{\pgfqpoint{6.070398in}{2.105045in}}%
\pgfpathcurveto{\pgfqpoint{6.059503in}{2.105045in}}{\pgfqpoint{6.049052in}{2.100716in}}{\pgfqpoint{6.041348in}{2.093011in}}%
\pgfpathcurveto{\pgfqpoint{6.033643in}{2.085307in}}{\pgfqpoint{6.029314in}{2.074856in}}{\pgfqpoint{6.029314in}{2.063961in}}%
\pgfpathcurveto{\pgfqpoint{6.029314in}{2.053065in}}{\pgfqpoint{6.033643in}{2.042614in}}{\pgfqpoint{6.041348in}{2.034910in}}%
\pgfpathcurveto{\pgfqpoint{6.049052in}{2.027206in}}{\pgfqpoint{6.059503in}{2.022877in}}{\pgfqpoint{6.070398in}{2.022877in}}%
\pgfpathlineto{\pgfqpoint{6.070398in}{2.022877in}}%
\pgfpathclose%
\pgfusepath{stroke}%
\end{pgfscope}%
\begin{pgfscope}%
\pgfpathrectangle{\pgfqpoint{0.688192in}{0.670138in}}{\pgfqpoint{7.111808in}{5.061530in}}%
\pgfusepath{clip}%
\pgfsetbuttcap%
\pgfsetroundjoin%
\pgfsetlinewidth{1.003750pt}%
\definecolor{currentstroke}{rgb}{0.000000,0.000000,0.000000}%
\pgfsetstrokecolor{currentstroke}%
\pgfsetdash{}{0pt}%
\pgfpathmoveto{\pgfqpoint{3.780438in}{0.873903in}}%
\pgfpathcurveto{\pgfqpoint{3.791333in}{0.873903in}}{\pgfqpoint{3.801784in}{0.878232in}}{\pgfqpoint{3.809488in}{0.885937in}}%
\pgfpathcurveto{\pgfqpoint{3.817193in}{0.893641in}}{\pgfqpoint{3.821522in}{0.904092in}}{\pgfqpoint{3.821522in}{0.914987in}}%
\pgfpathcurveto{\pgfqpoint{3.821522in}{0.925883in}}{\pgfqpoint{3.817193in}{0.936334in}}{\pgfqpoint{3.809488in}{0.944038in}}%
\pgfpathcurveto{\pgfqpoint{3.801784in}{0.951742in}}{\pgfqpoint{3.791333in}{0.956071in}}{\pgfqpoint{3.780438in}{0.956071in}}%
\pgfpathcurveto{\pgfqpoint{3.769542in}{0.956071in}}{\pgfqpoint{3.759091in}{0.951742in}}{\pgfqpoint{3.751387in}{0.944038in}}%
\pgfpathcurveto{\pgfqpoint{3.743683in}{0.936334in}}{\pgfqpoint{3.739354in}{0.925883in}}{\pgfqpoint{3.739354in}{0.914987in}}%
\pgfpathcurveto{\pgfqpoint{3.739354in}{0.904092in}}{\pgfqpoint{3.743683in}{0.893641in}}{\pgfqpoint{3.751387in}{0.885937in}}%
\pgfpathcurveto{\pgfqpoint{3.759091in}{0.878232in}}{\pgfqpoint{3.769542in}{0.873903in}}{\pgfqpoint{3.780438in}{0.873903in}}%
\pgfpathlineto{\pgfqpoint{3.780438in}{0.873903in}}%
\pgfpathclose%
\pgfusepath{stroke}%
\end{pgfscope}%
\begin{pgfscope}%
\pgfpathrectangle{\pgfqpoint{0.688192in}{0.670138in}}{\pgfqpoint{7.111808in}{5.061530in}}%
\pgfusepath{clip}%
\pgfsetbuttcap%
\pgfsetroundjoin%
\pgfsetlinewidth{1.003750pt}%
\definecolor{currentstroke}{rgb}{0.000000,0.000000,0.000000}%
\pgfsetstrokecolor{currentstroke}%
\pgfsetdash{}{0pt}%
\pgfpathmoveto{\pgfqpoint{1.407033in}{0.642986in}}%
\pgfpathcurveto{\pgfqpoint{1.417928in}{0.642986in}}{\pgfqpoint{1.428379in}{0.647315in}}{\pgfqpoint{1.436083in}{0.655020in}}%
\pgfpathcurveto{\pgfqpoint{1.443788in}{0.662724in}}{\pgfqpoint{1.448117in}{0.673175in}}{\pgfqpoint{1.448117in}{0.684070in}}%
\pgfpathcurveto{\pgfqpoint{1.448117in}{0.694966in}}{\pgfqpoint{1.443788in}{0.705417in}}{\pgfqpoint{1.436083in}{0.713121in}}%
\pgfpathcurveto{\pgfqpoint{1.428379in}{0.720825in}}{\pgfqpoint{1.417928in}{0.725154in}}{\pgfqpoint{1.407033in}{0.725154in}}%
\pgfpathcurveto{\pgfqpoint{1.396137in}{0.725154in}}{\pgfqpoint{1.385686in}{0.720825in}}{\pgfqpoint{1.377982in}{0.713121in}}%
\pgfpathcurveto{\pgfqpoint{1.370278in}{0.705417in}}{\pgfqpoint{1.365949in}{0.694966in}}{\pgfqpoint{1.365949in}{0.684070in}}%
\pgfpathcurveto{\pgfqpoint{1.365949in}{0.673175in}}{\pgfqpoint{1.370278in}{0.662724in}}{\pgfqpoint{1.377982in}{0.655020in}}%
\pgfpathcurveto{\pgfqpoint{1.385686in}{0.647315in}}{\pgfqpoint{1.396137in}{0.642986in}}{\pgfqpoint{1.407033in}{0.642986in}}%
\pgfusepath{stroke}%
\end{pgfscope}%
\begin{pgfscope}%
\pgfpathrectangle{\pgfqpoint{0.688192in}{0.670138in}}{\pgfqpoint{7.111808in}{5.061530in}}%
\pgfusepath{clip}%
\pgfsetbuttcap%
\pgfsetroundjoin%
\pgfsetlinewidth{1.003750pt}%
\definecolor{currentstroke}{rgb}{0.000000,0.000000,0.000000}%
\pgfsetstrokecolor{currentstroke}%
\pgfsetdash{}{0pt}%
\pgfpathmoveto{\pgfqpoint{5.135185in}{0.630032in}}%
\pgfpathcurveto{\pgfqpoint{5.146081in}{0.630032in}}{\pgfqpoint{5.156532in}{0.634361in}}{\pgfqpoint{5.164236in}{0.642065in}}%
\pgfpathcurveto{\pgfqpoint{5.171940in}{0.649770in}}{\pgfqpoint{5.176269in}{0.660220in}}{\pgfqpoint{5.176269in}{0.671116in}}%
\pgfpathcurveto{\pgfqpoint{5.176269in}{0.682011in}}{\pgfqpoint{5.171940in}{0.692462in}}{\pgfqpoint{5.164236in}{0.700167in}}%
\pgfpathcurveto{\pgfqpoint{5.156532in}{0.707871in}}{\pgfqpoint{5.146081in}{0.712200in}}{\pgfqpoint{5.135185in}{0.712200in}}%
\pgfpathcurveto{\pgfqpoint{5.124290in}{0.712200in}}{\pgfqpoint{5.113839in}{0.707871in}}{\pgfqpoint{5.106135in}{0.700167in}}%
\pgfpathcurveto{\pgfqpoint{5.098430in}{0.692462in}}{\pgfqpoint{5.094101in}{0.682011in}}{\pgfqpoint{5.094101in}{0.671116in}}%
\pgfpathcurveto{\pgfqpoint{5.094101in}{0.660220in}}{\pgfqpoint{5.098430in}{0.649770in}}{\pgfqpoint{5.106135in}{0.642065in}}%
\pgfpathcurveto{\pgfqpoint{5.113839in}{0.634361in}}{\pgfqpoint{5.124290in}{0.630032in}}{\pgfqpoint{5.135185in}{0.630032in}}%
\pgfusepath{stroke}%
\end{pgfscope}%
\begin{pgfscope}%
\pgfpathrectangle{\pgfqpoint{0.688192in}{0.670138in}}{\pgfqpoint{7.111808in}{5.061530in}}%
\pgfusepath{clip}%
\pgfsetbuttcap%
\pgfsetroundjoin%
\pgfsetlinewidth{1.003750pt}%
\definecolor{currentstroke}{rgb}{0.000000,0.000000,0.000000}%
\pgfsetstrokecolor{currentstroke}%
\pgfsetdash{}{0pt}%
\pgfpathmoveto{\pgfqpoint{5.140992in}{0.629909in}}%
\pgfpathcurveto{\pgfqpoint{5.151888in}{0.629909in}}{\pgfqpoint{5.162338in}{0.634237in}}{\pgfqpoint{5.170043in}{0.641942in}}%
\pgfpathcurveto{\pgfqpoint{5.177747in}{0.649646in}}{\pgfqpoint{5.182076in}{0.660097in}}{\pgfqpoint{5.182076in}{0.670993in}}%
\pgfpathcurveto{\pgfqpoint{5.182076in}{0.681888in}}{\pgfqpoint{5.177747in}{0.692339in}}{\pgfqpoint{5.170043in}{0.700043in}}%
\pgfpathcurveto{\pgfqpoint{5.162338in}{0.707748in}}{\pgfqpoint{5.151888in}{0.712076in}}{\pgfqpoint{5.140992in}{0.712076in}}%
\pgfpathcurveto{\pgfqpoint{5.130097in}{0.712076in}}{\pgfqpoint{5.119646in}{0.707748in}}{\pgfqpoint{5.111941in}{0.700043in}}%
\pgfpathcurveto{\pgfqpoint{5.104237in}{0.692339in}}{\pgfqpoint{5.099908in}{0.681888in}}{\pgfqpoint{5.099908in}{0.670993in}}%
\pgfpathcurveto{\pgfqpoint{5.099908in}{0.660097in}}{\pgfqpoint{5.104237in}{0.649646in}}{\pgfqpoint{5.111941in}{0.641942in}}%
\pgfpathcurveto{\pgfqpoint{5.119646in}{0.634237in}}{\pgfqpoint{5.130097in}{0.629909in}}{\pgfqpoint{5.140992in}{0.629909in}}%
\pgfusepath{stroke}%
\end{pgfscope}%
\begin{pgfscope}%
\pgfpathrectangle{\pgfqpoint{0.688192in}{0.670138in}}{\pgfqpoint{7.111808in}{5.061530in}}%
\pgfusepath{clip}%
\pgfsetbuttcap%
\pgfsetroundjoin%
\pgfsetlinewidth{1.003750pt}%
\definecolor{currentstroke}{rgb}{0.000000,0.000000,0.000000}%
\pgfsetstrokecolor{currentstroke}%
\pgfsetdash{}{0pt}%
\pgfpathmoveto{\pgfqpoint{1.159390in}{0.644542in}}%
\pgfpathcurveto{\pgfqpoint{1.170286in}{0.644542in}}{\pgfqpoint{1.180737in}{0.648870in}}{\pgfqpoint{1.188441in}{0.656575in}}%
\pgfpathcurveto{\pgfqpoint{1.196145in}{0.664279in}}{\pgfqpoint{1.200474in}{0.674730in}}{\pgfqpoint{1.200474in}{0.685625in}}%
\pgfpathcurveto{\pgfqpoint{1.200474in}{0.696521in}}{\pgfqpoint{1.196145in}{0.706972in}}{\pgfqpoint{1.188441in}{0.714676in}}%
\pgfpathcurveto{\pgfqpoint{1.180737in}{0.722380in}}{\pgfqpoint{1.170286in}{0.726709in}}{\pgfqpoint{1.159390in}{0.726709in}}%
\pgfpathcurveto{\pgfqpoint{1.148495in}{0.726709in}}{\pgfqpoint{1.138044in}{0.722380in}}{\pgfqpoint{1.130340in}{0.714676in}}%
\pgfpathcurveto{\pgfqpoint{1.122635in}{0.706972in}}{\pgfqpoint{1.118306in}{0.696521in}}{\pgfqpoint{1.118306in}{0.685625in}}%
\pgfpathcurveto{\pgfqpoint{1.118306in}{0.674730in}}{\pgfqpoint{1.122635in}{0.664279in}}{\pgfqpoint{1.130340in}{0.656575in}}%
\pgfpathcurveto{\pgfqpoint{1.138044in}{0.648870in}}{\pgfqpoint{1.148495in}{0.644542in}}{\pgfqpoint{1.159390in}{0.644542in}}%
\pgfusepath{stroke}%
\end{pgfscope}%
\begin{pgfscope}%
\pgfpathrectangle{\pgfqpoint{0.688192in}{0.670138in}}{\pgfqpoint{7.111808in}{5.061530in}}%
\pgfusepath{clip}%
\pgfsetbuttcap%
\pgfsetroundjoin%
\pgfsetlinewidth{1.003750pt}%
\definecolor{currentstroke}{rgb}{0.000000,0.000000,0.000000}%
\pgfsetstrokecolor{currentstroke}%
\pgfsetdash{}{0pt}%
\pgfpathmoveto{\pgfqpoint{3.863667in}{0.632041in}}%
\pgfpathcurveto{\pgfqpoint{3.874562in}{0.632041in}}{\pgfqpoint{3.885013in}{0.636370in}}{\pgfqpoint{3.892717in}{0.644075in}}%
\pgfpathcurveto{\pgfqpoint{3.900422in}{0.651779in}}{\pgfqpoint{3.904751in}{0.662230in}}{\pgfqpoint{3.904751in}{0.673125in}}%
\pgfpathcurveto{\pgfqpoint{3.904751in}{0.684021in}}{\pgfqpoint{3.900422in}{0.694472in}}{\pgfqpoint{3.892717in}{0.702176in}}%
\pgfpathcurveto{\pgfqpoint{3.885013in}{0.709880in}}{\pgfqpoint{3.874562in}{0.714209in}}{\pgfqpoint{3.863667in}{0.714209in}}%
\pgfpathcurveto{\pgfqpoint{3.852771in}{0.714209in}}{\pgfqpoint{3.842320in}{0.709880in}}{\pgfqpoint{3.834616in}{0.702176in}}%
\pgfpathcurveto{\pgfqpoint{3.826912in}{0.694472in}}{\pgfqpoint{3.822583in}{0.684021in}}{\pgfqpoint{3.822583in}{0.673125in}}%
\pgfpathcurveto{\pgfqpoint{3.822583in}{0.662230in}}{\pgfqpoint{3.826912in}{0.651779in}}{\pgfqpoint{3.834616in}{0.644075in}}%
\pgfpathcurveto{\pgfqpoint{3.842320in}{0.636370in}}{\pgfqpoint{3.852771in}{0.632041in}}{\pgfqpoint{3.863667in}{0.632041in}}%
\pgfusepath{stroke}%
\end{pgfscope}%
\begin{pgfscope}%
\pgfpathrectangle{\pgfqpoint{0.688192in}{0.670138in}}{\pgfqpoint{7.111808in}{5.061530in}}%
\pgfusepath{clip}%
\pgfsetbuttcap%
\pgfsetroundjoin%
\pgfsetlinewidth{1.003750pt}%
\definecolor{currentstroke}{rgb}{0.000000,0.000000,0.000000}%
\pgfsetstrokecolor{currentstroke}%
\pgfsetdash{}{0pt}%
\pgfpathmoveto{\pgfqpoint{0.749254in}{1.404065in}}%
\pgfpathcurveto{\pgfqpoint{0.760150in}{1.404065in}}{\pgfqpoint{0.770600in}{1.408394in}}{\pgfqpoint{0.778305in}{1.416099in}}%
\pgfpathcurveto{\pgfqpoint{0.786009in}{1.423803in}}{\pgfqpoint{0.790338in}{1.434254in}}{\pgfqpoint{0.790338in}{1.445149in}}%
\pgfpathcurveto{\pgfqpoint{0.790338in}{1.456045in}}{\pgfqpoint{0.786009in}{1.466496in}}{\pgfqpoint{0.778305in}{1.474200in}}%
\pgfpathcurveto{\pgfqpoint{0.770600in}{1.481904in}}{\pgfqpoint{0.760150in}{1.486233in}}{\pgfqpoint{0.749254in}{1.486233in}}%
\pgfpathcurveto{\pgfqpoint{0.738358in}{1.486233in}}{\pgfqpoint{0.727908in}{1.481904in}}{\pgfqpoint{0.720203in}{1.474200in}}%
\pgfpathcurveto{\pgfqpoint{0.712499in}{1.466496in}}{\pgfqpoint{0.708170in}{1.456045in}}{\pgfqpoint{0.708170in}{1.445149in}}%
\pgfpathcurveto{\pgfqpoint{0.708170in}{1.434254in}}{\pgfqpoint{0.712499in}{1.423803in}}{\pgfqpoint{0.720203in}{1.416099in}}%
\pgfpathcurveto{\pgfqpoint{0.727908in}{1.408394in}}{\pgfqpoint{0.738358in}{1.404065in}}{\pgfqpoint{0.749254in}{1.404065in}}%
\pgfpathlineto{\pgfqpoint{0.749254in}{1.404065in}}%
\pgfpathclose%
\pgfusepath{stroke}%
\end{pgfscope}%
\begin{pgfscope}%
\pgfpathrectangle{\pgfqpoint{0.688192in}{0.670138in}}{\pgfqpoint{7.111808in}{5.061530in}}%
\pgfusepath{clip}%
\pgfsetbuttcap%
\pgfsetroundjoin%
\pgfsetlinewidth{1.003750pt}%
\definecolor{currentstroke}{rgb}{0.000000,0.000000,0.000000}%
\pgfsetstrokecolor{currentstroke}%
\pgfsetdash{}{0pt}%
\pgfpathmoveto{\pgfqpoint{1.503177in}{0.642365in}}%
\pgfpathcurveto{\pgfqpoint{1.514073in}{0.642365in}}{\pgfqpoint{1.524523in}{0.646694in}}{\pgfqpoint{1.532228in}{0.654398in}}%
\pgfpathcurveto{\pgfqpoint{1.539932in}{0.662103in}}{\pgfqpoint{1.544261in}{0.672553in}}{\pgfqpoint{1.544261in}{0.683449in}}%
\pgfpathcurveto{\pgfqpoint{1.544261in}{0.694345in}}{\pgfqpoint{1.539932in}{0.704795in}}{\pgfqpoint{1.532228in}{0.712500in}}%
\pgfpathcurveto{\pgfqpoint{1.524523in}{0.720204in}}{\pgfqpoint{1.514073in}{0.724533in}}{\pgfqpoint{1.503177in}{0.724533in}}%
\pgfpathcurveto{\pgfqpoint{1.492281in}{0.724533in}}{\pgfqpoint{1.481831in}{0.720204in}}{\pgfqpoint{1.474126in}{0.712500in}}%
\pgfpathcurveto{\pgfqpoint{1.466422in}{0.704795in}}{\pgfqpoint{1.462093in}{0.694345in}}{\pgfqpoint{1.462093in}{0.683449in}}%
\pgfpathcurveto{\pgfqpoint{1.462093in}{0.672553in}}{\pgfqpoint{1.466422in}{0.662103in}}{\pgfqpoint{1.474126in}{0.654398in}}%
\pgfpathcurveto{\pgfqpoint{1.481831in}{0.646694in}}{\pgfqpoint{1.492281in}{0.642365in}}{\pgfqpoint{1.503177in}{0.642365in}}%
\pgfusepath{stroke}%
\end{pgfscope}%
\begin{pgfscope}%
\pgfpathrectangle{\pgfqpoint{0.688192in}{0.670138in}}{\pgfqpoint{7.111808in}{5.061530in}}%
\pgfusepath{clip}%
\pgfsetbuttcap%
\pgfsetroundjoin%
\pgfsetlinewidth{1.003750pt}%
\definecolor{currentstroke}{rgb}{0.000000,0.000000,0.000000}%
\pgfsetstrokecolor{currentstroke}%
\pgfsetdash{}{0pt}%
\pgfpathmoveto{\pgfqpoint{2.526063in}{2.676387in}}%
\pgfpathcurveto{\pgfqpoint{2.536958in}{2.676387in}}{\pgfqpoint{2.547409in}{2.680716in}}{\pgfqpoint{2.555114in}{2.688420in}}%
\pgfpathcurveto{\pgfqpoint{2.562818in}{2.696125in}}{\pgfqpoint{2.567147in}{2.706575in}}{\pgfqpoint{2.567147in}{2.717471in}}%
\pgfpathcurveto{\pgfqpoint{2.567147in}{2.728367in}}{\pgfqpoint{2.562818in}{2.738817in}}{\pgfqpoint{2.555114in}{2.746522in}}%
\pgfpathcurveto{\pgfqpoint{2.547409in}{2.754226in}}{\pgfqpoint{2.536958in}{2.758555in}}{\pgfqpoint{2.526063in}{2.758555in}}%
\pgfpathcurveto{\pgfqpoint{2.515167in}{2.758555in}}{\pgfqpoint{2.504716in}{2.754226in}}{\pgfqpoint{2.497012in}{2.746522in}}%
\pgfpathcurveto{\pgfqpoint{2.489308in}{2.738817in}}{\pgfqpoint{2.484979in}{2.728367in}}{\pgfqpoint{2.484979in}{2.717471in}}%
\pgfpathcurveto{\pgfqpoint{2.484979in}{2.706575in}}{\pgfqpoint{2.489308in}{2.696125in}}{\pgfqpoint{2.497012in}{2.688420in}}%
\pgfpathcurveto{\pgfqpoint{2.504716in}{2.680716in}}{\pgfqpoint{2.515167in}{2.676387in}}{\pgfqpoint{2.526063in}{2.676387in}}%
\pgfpathlineto{\pgfqpoint{2.526063in}{2.676387in}}%
\pgfpathclose%
\pgfusepath{stroke}%
\end{pgfscope}%
\begin{pgfscope}%
\pgfpathrectangle{\pgfqpoint{0.688192in}{0.670138in}}{\pgfqpoint{7.111808in}{5.061530in}}%
\pgfusepath{clip}%
\pgfsetbuttcap%
\pgfsetroundjoin%
\pgfsetlinewidth{1.003750pt}%
\definecolor{currentstroke}{rgb}{0.000000,0.000000,0.000000}%
\pgfsetstrokecolor{currentstroke}%
\pgfsetdash{}{0pt}%
\pgfpathmoveto{\pgfqpoint{4.207141in}{0.631198in}}%
\pgfpathcurveto{\pgfqpoint{4.218036in}{0.631198in}}{\pgfqpoint{4.228487in}{0.635527in}}{\pgfqpoint{4.236191in}{0.643231in}}%
\pgfpathcurveto{\pgfqpoint{4.243896in}{0.650936in}}{\pgfqpoint{4.248225in}{0.661387in}}{\pgfqpoint{4.248225in}{0.672282in}}%
\pgfpathcurveto{\pgfqpoint{4.248225in}{0.683178in}}{\pgfqpoint{4.243896in}{0.693629in}}{\pgfqpoint{4.236191in}{0.701333in}}%
\pgfpathcurveto{\pgfqpoint{4.228487in}{0.709037in}}{\pgfqpoint{4.218036in}{0.713366in}}{\pgfqpoint{4.207141in}{0.713366in}}%
\pgfpathcurveto{\pgfqpoint{4.196245in}{0.713366in}}{\pgfqpoint{4.185794in}{0.709037in}}{\pgfqpoint{4.178090in}{0.701333in}}%
\pgfpathcurveto{\pgfqpoint{4.170386in}{0.693629in}}{\pgfqpoint{4.166057in}{0.683178in}}{\pgfqpoint{4.166057in}{0.672282in}}%
\pgfpathcurveto{\pgfqpoint{4.166057in}{0.661387in}}{\pgfqpoint{4.170386in}{0.650936in}}{\pgfqpoint{4.178090in}{0.643231in}}%
\pgfpathcurveto{\pgfqpoint{4.185794in}{0.635527in}}{\pgfqpoint{4.196245in}{0.631198in}}{\pgfqpoint{4.207141in}{0.631198in}}%
\pgfusepath{stroke}%
\end{pgfscope}%
\begin{pgfscope}%
\pgfpathrectangle{\pgfqpoint{0.688192in}{0.670138in}}{\pgfqpoint{7.111808in}{5.061530in}}%
\pgfusepath{clip}%
\pgfsetbuttcap%
\pgfsetroundjoin%
\pgfsetlinewidth{1.003750pt}%
\definecolor{currentstroke}{rgb}{0.000000,0.000000,0.000000}%
\pgfsetstrokecolor{currentstroke}%
\pgfsetdash{}{0pt}%
\pgfpathmoveto{\pgfqpoint{1.503177in}{0.642365in}}%
\pgfpathcurveto{\pgfqpoint{1.514073in}{0.642365in}}{\pgfqpoint{1.524523in}{0.646694in}}{\pgfqpoint{1.532228in}{0.654398in}}%
\pgfpathcurveto{\pgfqpoint{1.539932in}{0.662103in}}{\pgfqpoint{1.544261in}{0.672553in}}{\pgfqpoint{1.544261in}{0.683449in}}%
\pgfpathcurveto{\pgfqpoint{1.544261in}{0.694345in}}{\pgfqpoint{1.539932in}{0.704795in}}{\pgfqpoint{1.532228in}{0.712500in}}%
\pgfpathcurveto{\pgfqpoint{1.524523in}{0.720204in}}{\pgfqpoint{1.514073in}{0.724533in}}{\pgfqpoint{1.503177in}{0.724533in}}%
\pgfpathcurveto{\pgfqpoint{1.492281in}{0.724533in}}{\pgfqpoint{1.481831in}{0.720204in}}{\pgfqpoint{1.474126in}{0.712500in}}%
\pgfpathcurveto{\pgfqpoint{1.466422in}{0.704795in}}{\pgfqpoint{1.462093in}{0.694345in}}{\pgfqpoint{1.462093in}{0.683449in}}%
\pgfpathcurveto{\pgfqpoint{1.462093in}{0.672553in}}{\pgfqpoint{1.466422in}{0.662103in}}{\pgfqpoint{1.474126in}{0.654398in}}%
\pgfpathcurveto{\pgfqpoint{1.481831in}{0.646694in}}{\pgfqpoint{1.492281in}{0.642365in}}{\pgfqpoint{1.503177in}{0.642365in}}%
\pgfusepath{stroke}%
\end{pgfscope}%
\begin{pgfscope}%
\pgfpathrectangle{\pgfqpoint{0.688192in}{0.670138in}}{\pgfqpoint{7.111808in}{5.061530in}}%
\pgfusepath{clip}%
\pgfsetbuttcap%
\pgfsetroundjoin%
\pgfsetlinewidth{1.003750pt}%
\definecolor{currentstroke}{rgb}{0.000000,0.000000,0.000000}%
\pgfsetstrokecolor{currentstroke}%
\pgfsetdash{}{0pt}%
\pgfpathmoveto{\pgfqpoint{2.410547in}{1.281058in}}%
\pgfpathcurveto{\pgfqpoint{2.421442in}{1.281058in}}{\pgfqpoint{2.431893in}{1.285387in}}{\pgfqpoint{2.439597in}{1.293091in}}%
\pgfpathcurveto{\pgfqpoint{2.447302in}{1.300795in}}{\pgfqpoint{2.451631in}{1.311246in}}{\pgfqpoint{2.451631in}{1.322142in}}%
\pgfpathcurveto{\pgfqpoint{2.451631in}{1.333037in}}{\pgfqpoint{2.447302in}{1.343488in}}{\pgfqpoint{2.439597in}{1.351192in}}%
\pgfpathcurveto{\pgfqpoint{2.431893in}{1.358897in}}{\pgfqpoint{2.421442in}{1.363226in}}{\pgfqpoint{2.410547in}{1.363226in}}%
\pgfpathcurveto{\pgfqpoint{2.399651in}{1.363226in}}{\pgfqpoint{2.389200in}{1.358897in}}{\pgfqpoint{2.381496in}{1.351192in}}%
\pgfpathcurveto{\pgfqpoint{2.373792in}{1.343488in}}{\pgfqpoint{2.369463in}{1.333037in}}{\pgfqpoint{2.369463in}{1.322142in}}%
\pgfpathcurveto{\pgfqpoint{2.369463in}{1.311246in}}{\pgfqpoint{2.373792in}{1.300795in}}{\pgfqpoint{2.381496in}{1.293091in}}%
\pgfpathcurveto{\pgfqpoint{2.389200in}{1.285387in}}{\pgfqpoint{2.399651in}{1.281058in}}{\pgfqpoint{2.410547in}{1.281058in}}%
\pgfpathlineto{\pgfqpoint{2.410547in}{1.281058in}}%
\pgfpathclose%
\pgfusepath{stroke}%
\end{pgfscope}%
\begin{pgfscope}%
\pgfpathrectangle{\pgfqpoint{0.688192in}{0.670138in}}{\pgfqpoint{7.111808in}{5.061530in}}%
\pgfusepath{clip}%
\pgfsetbuttcap%
\pgfsetroundjoin%
\pgfsetlinewidth{1.003750pt}%
\definecolor{currentstroke}{rgb}{0.000000,0.000000,0.000000}%
\pgfsetstrokecolor{currentstroke}%
\pgfsetdash{}{0pt}%
\pgfpathmoveto{\pgfqpoint{6.294942in}{0.759522in}}%
\pgfpathcurveto{\pgfqpoint{6.305837in}{0.759522in}}{\pgfqpoint{6.316288in}{0.763851in}}{\pgfqpoint{6.323992in}{0.771556in}}%
\pgfpathcurveto{\pgfqpoint{6.331697in}{0.779260in}}{\pgfqpoint{6.336026in}{0.789711in}}{\pgfqpoint{6.336026in}{0.800606in}}%
\pgfpathcurveto{\pgfqpoint{6.336026in}{0.811502in}}{\pgfqpoint{6.331697in}{0.821953in}}{\pgfqpoint{6.323992in}{0.829657in}}%
\pgfpathcurveto{\pgfqpoint{6.316288in}{0.837361in}}{\pgfqpoint{6.305837in}{0.841690in}}{\pgfqpoint{6.294942in}{0.841690in}}%
\pgfpathcurveto{\pgfqpoint{6.284046in}{0.841690in}}{\pgfqpoint{6.273595in}{0.837361in}}{\pgfqpoint{6.265891in}{0.829657in}}%
\pgfpathcurveto{\pgfqpoint{6.258187in}{0.821953in}}{\pgfqpoint{6.253858in}{0.811502in}}{\pgfqpoint{6.253858in}{0.800606in}}%
\pgfpathcurveto{\pgfqpoint{6.253858in}{0.789711in}}{\pgfqpoint{6.258187in}{0.779260in}}{\pgfqpoint{6.265891in}{0.771556in}}%
\pgfpathcurveto{\pgfqpoint{6.273595in}{0.763851in}}{\pgfqpoint{6.284046in}{0.759522in}}{\pgfqpoint{6.294942in}{0.759522in}}%
\pgfpathlineto{\pgfqpoint{6.294942in}{0.759522in}}%
\pgfpathclose%
\pgfusepath{stroke}%
\end{pgfscope}%
\begin{pgfscope}%
\pgfpathrectangle{\pgfqpoint{0.688192in}{0.670138in}}{\pgfqpoint{7.111808in}{5.061530in}}%
\pgfusepath{clip}%
\pgfsetbuttcap%
\pgfsetroundjoin%
\pgfsetlinewidth{1.003750pt}%
\definecolor{currentstroke}{rgb}{0.000000,0.000000,0.000000}%
\pgfsetstrokecolor{currentstroke}%
\pgfsetdash{}{0pt}%
\pgfpathmoveto{\pgfqpoint{1.503177in}{0.642365in}}%
\pgfpathcurveto{\pgfqpoint{1.514073in}{0.642365in}}{\pgfqpoint{1.524523in}{0.646694in}}{\pgfqpoint{1.532228in}{0.654398in}}%
\pgfpathcurveto{\pgfqpoint{1.539932in}{0.662103in}}{\pgfqpoint{1.544261in}{0.672553in}}{\pgfqpoint{1.544261in}{0.683449in}}%
\pgfpathcurveto{\pgfqpoint{1.544261in}{0.694345in}}{\pgfqpoint{1.539932in}{0.704795in}}{\pgfqpoint{1.532228in}{0.712500in}}%
\pgfpathcurveto{\pgfqpoint{1.524523in}{0.720204in}}{\pgfqpoint{1.514073in}{0.724533in}}{\pgfqpoint{1.503177in}{0.724533in}}%
\pgfpathcurveto{\pgfqpoint{1.492281in}{0.724533in}}{\pgfqpoint{1.481831in}{0.720204in}}{\pgfqpoint{1.474126in}{0.712500in}}%
\pgfpathcurveto{\pgfqpoint{1.466422in}{0.704795in}}{\pgfqpoint{1.462093in}{0.694345in}}{\pgfqpoint{1.462093in}{0.683449in}}%
\pgfpathcurveto{\pgfqpoint{1.462093in}{0.672553in}}{\pgfqpoint{1.466422in}{0.662103in}}{\pgfqpoint{1.474126in}{0.654398in}}%
\pgfpathcurveto{\pgfqpoint{1.481831in}{0.646694in}}{\pgfqpoint{1.492281in}{0.642365in}}{\pgfqpoint{1.503177in}{0.642365in}}%
\pgfusepath{stroke}%
\end{pgfscope}%
\begin{pgfscope}%
\pgfpathrectangle{\pgfqpoint{0.688192in}{0.670138in}}{\pgfqpoint{7.111808in}{5.061530in}}%
\pgfusepath{clip}%
\pgfsetbuttcap%
\pgfsetroundjoin%
\pgfsetlinewidth{1.003750pt}%
\definecolor{currentstroke}{rgb}{0.000000,0.000000,0.000000}%
\pgfsetstrokecolor{currentstroke}%
\pgfsetdash{}{0pt}%
\pgfpathmoveto{\pgfqpoint{1.198965in}{0.644318in}}%
\pgfpathcurveto{\pgfqpoint{1.209860in}{0.644318in}}{\pgfqpoint{1.220311in}{0.648647in}}{\pgfqpoint{1.228016in}{0.656352in}}%
\pgfpathcurveto{\pgfqpoint{1.235720in}{0.664056in}}{\pgfqpoint{1.240049in}{0.674507in}}{\pgfqpoint{1.240049in}{0.685402in}}%
\pgfpathcurveto{\pgfqpoint{1.240049in}{0.696298in}}{\pgfqpoint{1.235720in}{0.706749in}}{\pgfqpoint{1.228016in}{0.714453in}}%
\pgfpathcurveto{\pgfqpoint{1.220311in}{0.722157in}}{\pgfqpoint{1.209860in}{0.726486in}}{\pgfqpoint{1.198965in}{0.726486in}}%
\pgfpathcurveto{\pgfqpoint{1.188069in}{0.726486in}}{\pgfqpoint{1.177619in}{0.722157in}}{\pgfqpoint{1.169914in}{0.714453in}}%
\pgfpathcurveto{\pgfqpoint{1.162210in}{0.706749in}}{\pgfqpoint{1.157881in}{0.696298in}}{\pgfqpoint{1.157881in}{0.685402in}}%
\pgfpathcurveto{\pgfqpoint{1.157881in}{0.674507in}}{\pgfqpoint{1.162210in}{0.664056in}}{\pgfqpoint{1.169914in}{0.656352in}}%
\pgfpathcurveto{\pgfqpoint{1.177619in}{0.648647in}}{\pgfqpoint{1.188069in}{0.644318in}}{\pgfqpoint{1.198965in}{0.644318in}}%
\pgfusepath{stroke}%
\end{pgfscope}%
\begin{pgfscope}%
\pgfpathrectangle{\pgfqpoint{0.688192in}{0.670138in}}{\pgfqpoint{7.111808in}{5.061530in}}%
\pgfusepath{clip}%
\pgfsetbuttcap%
\pgfsetroundjoin%
\pgfsetlinewidth{1.003750pt}%
\definecolor{currentstroke}{rgb}{0.000000,0.000000,0.000000}%
\pgfsetstrokecolor{currentstroke}%
\pgfsetdash{}{0pt}%
\pgfpathmoveto{\pgfqpoint{4.726657in}{0.630576in}}%
\pgfpathcurveto{\pgfqpoint{4.737552in}{0.630576in}}{\pgfqpoint{4.748003in}{0.634905in}}{\pgfqpoint{4.755707in}{0.642610in}}%
\pgfpathcurveto{\pgfqpoint{4.763412in}{0.650314in}}{\pgfqpoint{4.767741in}{0.660765in}}{\pgfqpoint{4.767741in}{0.671660in}}%
\pgfpathcurveto{\pgfqpoint{4.767741in}{0.682556in}}{\pgfqpoint{4.763412in}{0.693007in}}{\pgfqpoint{4.755707in}{0.700711in}}%
\pgfpathcurveto{\pgfqpoint{4.748003in}{0.708415in}}{\pgfqpoint{4.737552in}{0.712744in}}{\pgfqpoint{4.726657in}{0.712744in}}%
\pgfpathcurveto{\pgfqpoint{4.715761in}{0.712744in}}{\pgfqpoint{4.705310in}{0.708415in}}{\pgfqpoint{4.697606in}{0.700711in}}%
\pgfpathcurveto{\pgfqpoint{4.689902in}{0.693007in}}{\pgfqpoint{4.685573in}{0.682556in}}{\pgfqpoint{4.685573in}{0.671660in}}%
\pgfpathcurveto{\pgfqpoint{4.685573in}{0.660765in}}{\pgfqpoint{4.689902in}{0.650314in}}{\pgfqpoint{4.697606in}{0.642610in}}%
\pgfpathcurveto{\pgfqpoint{4.705310in}{0.634905in}}{\pgfqpoint{4.715761in}{0.630576in}}{\pgfqpoint{4.726657in}{0.630576in}}%
\pgfusepath{stroke}%
\end{pgfscope}%
\begin{pgfscope}%
\pgfpathrectangle{\pgfqpoint{0.688192in}{0.670138in}}{\pgfqpoint{7.111808in}{5.061530in}}%
\pgfusepath{clip}%
\pgfsetbuttcap%
\pgfsetroundjoin%
\pgfsetlinewidth{1.003750pt}%
\definecolor{currentstroke}{rgb}{0.000000,0.000000,0.000000}%
\pgfsetstrokecolor{currentstroke}%
\pgfsetdash{}{0pt}%
\pgfpathmoveto{\pgfqpoint{4.948155in}{0.663796in}}%
\pgfpathcurveto{\pgfqpoint{4.959050in}{0.663796in}}{\pgfqpoint{4.969501in}{0.668125in}}{\pgfqpoint{4.977205in}{0.675829in}}%
\pgfpathcurveto{\pgfqpoint{4.984910in}{0.683533in}}{\pgfqpoint{4.989239in}{0.693984in}}{\pgfqpoint{4.989239in}{0.704880in}}%
\pgfpathcurveto{\pgfqpoint{4.989239in}{0.715775in}}{\pgfqpoint{4.984910in}{0.726226in}}{\pgfqpoint{4.977205in}{0.733930in}}%
\pgfpathcurveto{\pgfqpoint{4.969501in}{0.741635in}}{\pgfqpoint{4.959050in}{0.745964in}}{\pgfqpoint{4.948155in}{0.745964in}}%
\pgfpathcurveto{\pgfqpoint{4.937259in}{0.745964in}}{\pgfqpoint{4.926808in}{0.741635in}}{\pgfqpoint{4.919104in}{0.733930in}}%
\pgfpathcurveto{\pgfqpoint{4.911400in}{0.726226in}}{\pgfqpoint{4.907071in}{0.715775in}}{\pgfqpoint{4.907071in}{0.704880in}}%
\pgfpathcurveto{\pgfqpoint{4.907071in}{0.693984in}}{\pgfqpoint{4.911400in}{0.683533in}}{\pgfqpoint{4.919104in}{0.675829in}}%
\pgfpathcurveto{\pgfqpoint{4.926808in}{0.668125in}}{\pgfqpoint{4.937259in}{0.663796in}}{\pgfqpoint{4.948155in}{0.663796in}}%
\pgfusepath{stroke}%
\end{pgfscope}%
\begin{pgfscope}%
\pgfpathrectangle{\pgfqpoint{0.688192in}{0.670138in}}{\pgfqpoint{7.111808in}{5.061530in}}%
\pgfusepath{clip}%
\pgfsetbuttcap%
\pgfsetroundjoin%
\pgfsetlinewidth{1.003750pt}%
\definecolor{currentstroke}{rgb}{0.000000,0.000000,0.000000}%
\pgfsetstrokecolor{currentstroke}%
\pgfsetdash{}{0pt}%
\pgfpathmoveto{\pgfqpoint{0.946217in}{0.669736in}}%
\pgfpathcurveto{\pgfqpoint{0.957113in}{0.669736in}}{\pgfqpoint{0.967564in}{0.674065in}}{\pgfqpoint{0.975268in}{0.681769in}}%
\pgfpathcurveto{\pgfqpoint{0.982972in}{0.689473in}}{\pgfqpoint{0.987301in}{0.699924in}}{\pgfqpoint{0.987301in}{0.710820in}}%
\pgfpathcurveto{\pgfqpoint{0.987301in}{0.721715in}}{\pgfqpoint{0.982972in}{0.732166in}}{\pgfqpoint{0.975268in}{0.739870in}}%
\pgfpathcurveto{\pgfqpoint{0.967564in}{0.747575in}}{\pgfqpoint{0.957113in}{0.751904in}}{\pgfqpoint{0.946217in}{0.751904in}}%
\pgfpathcurveto{\pgfqpoint{0.935322in}{0.751904in}}{\pgfqpoint{0.924871in}{0.747575in}}{\pgfqpoint{0.917167in}{0.739870in}}%
\pgfpathcurveto{\pgfqpoint{0.909462in}{0.732166in}}{\pgfqpoint{0.905134in}{0.721715in}}{\pgfqpoint{0.905134in}{0.710820in}}%
\pgfpathcurveto{\pgfqpoint{0.905134in}{0.699924in}}{\pgfqpoint{0.909462in}{0.689473in}}{\pgfqpoint{0.917167in}{0.681769in}}%
\pgfpathcurveto{\pgfqpoint{0.924871in}{0.674065in}}{\pgfqpoint{0.935322in}{0.669736in}}{\pgfqpoint{0.946217in}{0.669736in}}%
\pgfpathlineto{\pgfqpoint{0.946217in}{0.669736in}}%
\pgfpathclose%
\pgfusepath{stroke}%
\end{pgfscope}%
\begin{pgfscope}%
\pgfpathrectangle{\pgfqpoint{0.688192in}{0.670138in}}{\pgfqpoint{7.111808in}{5.061530in}}%
\pgfusepath{clip}%
\pgfsetbuttcap%
\pgfsetroundjoin%
\pgfsetlinewidth{1.003750pt}%
\definecolor{currentstroke}{rgb}{0.000000,0.000000,0.000000}%
\pgfsetstrokecolor{currentstroke}%
\pgfsetdash{}{0pt}%
\pgfpathmoveto{\pgfqpoint{4.945557in}{0.670217in}}%
\pgfpathcurveto{\pgfqpoint{4.956453in}{0.670217in}}{\pgfqpoint{4.966903in}{0.674546in}}{\pgfqpoint{4.974608in}{0.682251in}}%
\pgfpathcurveto{\pgfqpoint{4.982312in}{0.689955in}}{\pgfqpoint{4.986641in}{0.700406in}}{\pgfqpoint{4.986641in}{0.711301in}}%
\pgfpathcurveto{\pgfqpoint{4.986641in}{0.722197in}}{\pgfqpoint{4.982312in}{0.732648in}}{\pgfqpoint{4.974608in}{0.740352in}}%
\pgfpathcurveto{\pgfqpoint{4.966903in}{0.748056in}}{\pgfqpoint{4.956453in}{0.752385in}}{\pgfqpoint{4.945557in}{0.752385in}}%
\pgfpathcurveto{\pgfqpoint{4.934662in}{0.752385in}}{\pgfqpoint{4.924211in}{0.748056in}}{\pgfqpoint{4.916506in}{0.740352in}}%
\pgfpathcurveto{\pgfqpoint{4.908802in}{0.732648in}}{\pgfqpoint{4.904473in}{0.722197in}}{\pgfqpoint{4.904473in}{0.711301in}}%
\pgfpathcurveto{\pgfqpoint{4.904473in}{0.700406in}}{\pgfqpoint{4.908802in}{0.689955in}}{\pgfqpoint{4.916506in}{0.682251in}}%
\pgfpathcurveto{\pgfqpoint{4.924211in}{0.674546in}}{\pgfqpoint{4.934662in}{0.670217in}}{\pgfqpoint{4.945557in}{0.670217in}}%
\pgfpathlineto{\pgfqpoint{4.945557in}{0.670217in}}%
\pgfpathclose%
\pgfusepath{stroke}%
\end{pgfscope}%
\begin{pgfscope}%
\pgfpathrectangle{\pgfqpoint{0.688192in}{0.670138in}}{\pgfqpoint{7.111808in}{5.061530in}}%
\pgfusepath{clip}%
\pgfsetbuttcap%
\pgfsetroundjoin%
\pgfsetlinewidth{1.003750pt}%
\definecolor{currentstroke}{rgb}{0.000000,0.000000,0.000000}%
\pgfsetstrokecolor{currentstroke}%
\pgfsetdash{}{0pt}%
\pgfpathmoveto{\pgfqpoint{2.494145in}{0.637313in}}%
\pgfpathcurveto{\pgfqpoint{2.505041in}{0.637313in}}{\pgfqpoint{2.515491in}{0.641642in}}{\pgfqpoint{2.523196in}{0.649346in}}%
\pgfpathcurveto{\pgfqpoint{2.530900in}{0.657051in}}{\pgfqpoint{2.535229in}{0.667501in}}{\pgfqpoint{2.535229in}{0.678397in}}%
\pgfpathcurveto{\pgfqpoint{2.535229in}{0.689293in}}{\pgfqpoint{2.530900in}{0.699743in}}{\pgfqpoint{2.523196in}{0.707448in}}%
\pgfpathcurveto{\pgfqpoint{2.515491in}{0.715152in}}{\pgfqpoint{2.505041in}{0.719481in}}{\pgfqpoint{2.494145in}{0.719481in}}%
\pgfpathcurveto{\pgfqpoint{2.483249in}{0.719481in}}{\pgfqpoint{2.472799in}{0.715152in}}{\pgfqpoint{2.465094in}{0.707448in}}%
\pgfpathcurveto{\pgfqpoint{2.457390in}{0.699743in}}{\pgfqpoint{2.453061in}{0.689293in}}{\pgfqpoint{2.453061in}{0.678397in}}%
\pgfpathcurveto{\pgfqpoint{2.453061in}{0.667501in}}{\pgfqpoint{2.457390in}{0.657051in}}{\pgfqpoint{2.465094in}{0.649346in}}%
\pgfpathcurveto{\pgfqpoint{2.472799in}{0.641642in}}{\pgfqpoint{2.483249in}{0.637313in}}{\pgfqpoint{2.494145in}{0.637313in}}%
\pgfusepath{stroke}%
\end{pgfscope}%
\begin{pgfscope}%
\pgfpathrectangle{\pgfqpoint{0.688192in}{0.670138in}}{\pgfqpoint{7.111808in}{5.061530in}}%
\pgfusepath{clip}%
\pgfsetbuttcap%
\pgfsetroundjoin%
\pgfsetlinewidth{1.003750pt}%
\definecolor{currentstroke}{rgb}{0.000000,0.000000,0.000000}%
\pgfsetstrokecolor{currentstroke}%
\pgfsetdash{}{0pt}%
\pgfpathmoveto{\pgfqpoint{2.452193in}{3.677100in}}%
\pgfpathcurveto{\pgfqpoint{2.463088in}{3.677100in}}{\pgfqpoint{2.473539in}{3.681429in}}{\pgfqpoint{2.481243in}{3.689133in}}%
\pgfpathcurveto{\pgfqpoint{2.488948in}{3.696837in}}{\pgfqpoint{2.493276in}{3.707288in}}{\pgfqpoint{2.493276in}{3.718184in}}%
\pgfpathcurveto{\pgfqpoint{2.493276in}{3.729079in}}{\pgfqpoint{2.488948in}{3.739530in}}{\pgfqpoint{2.481243in}{3.747234in}}%
\pgfpathcurveto{\pgfqpoint{2.473539in}{3.754939in}}{\pgfqpoint{2.463088in}{3.759268in}}{\pgfqpoint{2.452193in}{3.759268in}}%
\pgfpathcurveto{\pgfqpoint{2.441297in}{3.759268in}}{\pgfqpoint{2.430846in}{3.754939in}}{\pgfqpoint{2.423142in}{3.747234in}}%
\pgfpathcurveto{\pgfqpoint{2.415438in}{3.739530in}}{\pgfqpoint{2.411109in}{3.729079in}}{\pgfqpoint{2.411109in}{3.718184in}}%
\pgfpathcurveto{\pgfqpoint{2.411109in}{3.707288in}}{\pgfqpoint{2.415438in}{3.696837in}}{\pgfqpoint{2.423142in}{3.689133in}}%
\pgfpathcurveto{\pgfqpoint{2.430846in}{3.681429in}}{\pgfqpoint{2.441297in}{3.677100in}}{\pgfqpoint{2.452193in}{3.677100in}}%
\pgfpathlineto{\pgfqpoint{2.452193in}{3.677100in}}%
\pgfpathclose%
\pgfusepath{stroke}%
\end{pgfscope}%
\begin{pgfscope}%
\pgfpathrectangle{\pgfqpoint{0.688192in}{0.670138in}}{\pgfqpoint{7.111808in}{5.061530in}}%
\pgfusepath{clip}%
\pgfsetbuttcap%
\pgfsetroundjoin%
\pgfsetlinewidth{1.003750pt}%
\definecolor{currentstroke}{rgb}{0.000000,0.000000,0.000000}%
\pgfsetstrokecolor{currentstroke}%
\pgfsetdash{}{0pt}%
\pgfpathmoveto{\pgfqpoint{0.939729in}{0.677113in}}%
\pgfpathcurveto{\pgfqpoint{0.950625in}{0.677113in}}{\pgfqpoint{0.961075in}{0.681442in}}{\pgfqpoint{0.968780in}{0.689146in}}%
\pgfpathcurveto{\pgfqpoint{0.976484in}{0.696850in}}{\pgfqpoint{0.980813in}{0.707301in}}{\pgfqpoint{0.980813in}{0.718197in}}%
\pgfpathcurveto{\pgfqpoint{0.980813in}{0.729092in}}{\pgfqpoint{0.976484in}{0.739543in}}{\pgfqpoint{0.968780in}{0.747247in}}%
\pgfpathcurveto{\pgfqpoint{0.961075in}{0.754952in}}{\pgfqpoint{0.950625in}{0.759281in}}{\pgfqpoint{0.939729in}{0.759281in}}%
\pgfpathcurveto{\pgfqpoint{0.928834in}{0.759281in}}{\pgfqpoint{0.918383in}{0.754952in}}{\pgfqpoint{0.910678in}{0.747247in}}%
\pgfpathcurveto{\pgfqpoint{0.902974in}{0.739543in}}{\pgfqpoint{0.898645in}{0.729092in}}{\pgfqpoint{0.898645in}{0.718197in}}%
\pgfpathcurveto{\pgfqpoint{0.898645in}{0.707301in}}{\pgfqpoint{0.902974in}{0.696850in}}{\pgfqpoint{0.910678in}{0.689146in}}%
\pgfpathcurveto{\pgfqpoint{0.918383in}{0.681442in}}{\pgfqpoint{0.928834in}{0.677113in}}{\pgfqpoint{0.939729in}{0.677113in}}%
\pgfpathlineto{\pgfqpoint{0.939729in}{0.677113in}}%
\pgfpathclose%
\pgfusepath{stroke}%
\end{pgfscope}%
\begin{pgfscope}%
\pgfpathrectangle{\pgfqpoint{0.688192in}{0.670138in}}{\pgfqpoint{7.111808in}{5.061530in}}%
\pgfusepath{clip}%
\pgfsetbuttcap%
\pgfsetroundjoin%
\pgfsetlinewidth{1.003750pt}%
\definecolor{currentstroke}{rgb}{0.000000,0.000000,0.000000}%
\pgfsetstrokecolor{currentstroke}%
\pgfsetdash{}{0pt}%
\pgfpathmoveto{\pgfqpoint{5.359877in}{0.629627in}}%
\pgfpathcurveto{\pgfqpoint{5.370772in}{0.629627in}}{\pgfqpoint{5.381223in}{0.633955in}}{\pgfqpoint{5.388928in}{0.641660in}}%
\pgfpathcurveto{\pgfqpoint{5.396632in}{0.649364in}}{\pgfqpoint{5.400961in}{0.659815in}}{\pgfqpoint{5.400961in}{0.670710in}}%
\pgfpathcurveto{\pgfqpoint{5.400961in}{0.681606in}}{\pgfqpoint{5.396632in}{0.692057in}}{\pgfqpoint{5.388928in}{0.699761in}}%
\pgfpathcurveto{\pgfqpoint{5.381223in}{0.707465in}}{\pgfqpoint{5.370772in}{0.711794in}}{\pgfqpoint{5.359877in}{0.711794in}}%
\pgfpathcurveto{\pgfqpoint{5.348981in}{0.711794in}}{\pgfqpoint{5.338530in}{0.707465in}}{\pgfqpoint{5.330826in}{0.699761in}}%
\pgfpathcurveto{\pgfqpoint{5.323122in}{0.692057in}}{\pgfqpoint{5.318793in}{0.681606in}}{\pgfqpoint{5.318793in}{0.670710in}}%
\pgfpathcurveto{\pgfqpoint{5.318793in}{0.659815in}}{\pgfqpoint{5.323122in}{0.649364in}}{\pgfqpoint{5.330826in}{0.641660in}}%
\pgfpathcurveto{\pgfqpoint{5.338530in}{0.633955in}}{\pgfqpoint{5.348981in}{0.629627in}}{\pgfqpoint{5.359877in}{0.629627in}}%
\pgfusepath{stroke}%
\end{pgfscope}%
\begin{pgfscope}%
\pgfpathrectangle{\pgfqpoint{0.688192in}{0.670138in}}{\pgfqpoint{7.111808in}{5.061530in}}%
\pgfusepath{clip}%
\pgfsetbuttcap%
\pgfsetroundjoin%
\pgfsetlinewidth{1.003750pt}%
\definecolor{currentstroke}{rgb}{0.000000,0.000000,0.000000}%
\pgfsetstrokecolor{currentstroke}%
\pgfsetdash{}{0pt}%
\pgfpathmoveto{\pgfqpoint{1.473926in}{0.642857in}}%
\pgfpathcurveto{\pgfqpoint{1.484821in}{0.642857in}}{\pgfqpoint{1.495272in}{0.647186in}}{\pgfqpoint{1.502976in}{0.654890in}}%
\pgfpathcurveto{\pgfqpoint{1.510681in}{0.662595in}}{\pgfqpoint{1.515009in}{0.673045in}}{\pgfqpoint{1.515009in}{0.683941in}}%
\pgfpathcurveto{\pgfqpoint{1.515009in}{0.694837in}}{\pgfqpoint{1.510681in}{0.705287in}}{\pgfqpoint{1.502976in}{0.712992in}}%
\pgfpathcurveto{\pgfqpoint{1.495272in}{0.720696in}}{\pgfqpoint{1.484821in}{0.725025in}}{\pgfqpoint{1.473926in}{0.725025in}}%
\pgfpathcurveto{\pgfqpoint{1.463030in}{0.725025in}}{\pgfqpoint{1.452579in}{0.720696in}}{\pgfqpoint{1.444875in}{0.712992in}}%
\pgfpathcurveto{\pgfqpoint{1.437171in}{0.705287in}}{\pgfqpoint{1.432842in}{0.694837in}}{\pgfqpoint{1.432842in}{0.683941in}}%
\pgfpathcurveto{\pgfqpoint{1.432842in}{0.673045in}}{\pgfqpoint{1.437171in}{0.662595in}}{\pgfqpoint{1.444875in}{0.654890in}}%
\pgfpathcurveto{\pgfqpoint{1.452579in}{0.647186in}}{\pgfqpoint{1.463030in}{0.642857in}}{\pgfqpoint{1.473926in}{0.642857in}}%
\pgfusepath{stroke}%
\end{pgfscope}%
\begin{pgfscope}%
\pgfpathrectangle{\pgfqpoint{0.688192in}{0.670138in}}{\pgfqpoint{7.111808in}{5.061530in}}%
\pgfusepath{clip}%
\pgfsetbuttcap%
\pgfsetroundjoin%
\pgfsetlinewidth{1.003750pt}%
\definecolor{currentstroke}{rgb}{0.000000,0.000000,0.000000}%
\pgfsetstrokecolor{currentstroke}%
\pgfsetdash{}{0pt}%
\pgfpathmoveto{\pgfqpoint{2.707541in}{0.659864in}}%
\pgfpathcurveto{\pgfqpoint{2.718437in}{0.659864in}}{\pgfqpoint{2.728888in}{0.664193in}}{\pgfqpoint{2.736592in}{0.671897in}}%
\pgfpathcurveto{\pgfqpoint{2.744297in}{0.679601in}}{\pgfqpoint{2.748625in}{0.690052in}}{\pgfqpoint{2.748625in}{0.700948in}}%
\pgfpathcurveto{\pgfqpoint{2.748625in}{0.711843in}}{\pgfqpoint{2.744297in}{0.722294in}}{\pgfqpoint{2.736592in}{0.729998in}}%
\pgfpathcurveto{\pgfqpoint{2.728888in}{0.737703in}}{\pgfqpoint{2.718437in}{0.742031in}}{\pgfqpoint{2.707541in}{0.742031in}}%
\pgfpathcurveto{\pgfqpoint{2.696646in}{0.742031in}}{\pgfqpoint{2.686195in}{0.737703in}}{\pgfqpoint{2.678491in}{0.729998in}}%
\pgfpathcurveto{\pgfqpoint{2.670786in}{0.722294in}}{\pgfqpoint{2.666458in}{0.711843in}}{\pgfqpoint{2.666458in}{0.700948in}}%
\pgfpathcurveto{\pgfqpoint{2.666458in}{0.690052in}}{\pgfqpoint{2.670786in}{0.679601in}}{\pgfqpoint{2.678491in}{0.671897in}}%
\pgfpathcurveto{\pgfqpoint{2.686195in}{0.664193in}}{\pgfqpoint{2.696646in}{0.659864in}}{\pgfqpoint{2.707541in}{0.659864in}}%
\pgfusepath{stroke}%
\end{pgfscope}%
\begin{pgfscope}%
\pgfpathrectangle{\pgfqpoint{0.688192in}{0.670138in}}{\pgfqpoint{7.111808in}{5.061530in}}%
\pgfusepath{clip}%
\pgfsetbuttcap%
\pgfsetroundjoin%
\pgfsetlinewidth{1.003750pt}%
\definecolor{currentstroke}{rgb}{0.000000,0.000000,0.000000}%
\pgfsetstrokecolor{currentstroke}%
\pgfsetdash{}{0pt}%
\pgfpathmoveto{\pgfqpoint{1.178009in}{0.644363in}}%
\pgfpathcurveto{\pgfqpoint{1.188905in}{0.644363in}}{\pgfqpoint{1.199355in}{0.648692in}}{\pgfqpoint{1.207060in}{0.656396in}}%
\pgfpathcurveto{\pgfqpoint{1.214764in}{0.664100in}}{\pgfqpoint{1.219093in}{0.674551in}}{\pgfqpoint{1.219093in}{0.685447in}}%
\pgfpathcurveto{\pgfqpoint{1.219093in}{0.696342in}}{\pgfqpoint{1.214764in}{0.706793in}}{\pgfqpoint{1.207060in}{0.714497in}}%
\pgfpathcurveto{\pgfqpoint{1.199355in}{0.722202in}}{\pgfqpoint{1.188905in}{0.726531in}}{\pgfqpoint{1.178009in}{0.726531in}}%
\pgfpathcurveto{\pgfqpoint{1.167113in}{0.726531in}}{\pgfqpoint{1.156663in}{0.722202in}}{\pgfqpoint{1.148958in}{0.714497in}}%
\pgfpathcurveto{\pgfqpoint{1.141254in}{0.706793in}}{\pgfqpoint{1.136925in}{0.696342in}}{\pgfqpoint{1.136925in}{0.685447in}}%
\pgfpathcurveto{\pgfqpoint{1.136925in}{0.674551in}}{\pgfqpoint{1.141254in}{0.664100in}}{\pgfqpoint{1.148958in}{0.656396in}}%
\pgfpathcurveto{\pgfqpoint{1.156663in}{0.648692in}}{\pgfqpoint{1.167113in}{0.644363in}}{\pgfqpoint{1.178009in}{0.644363in}}%
\pgfusepath{stroke}%
\end{pgfscope}%
\begin{pgfscope}%
\pgfpathrectangle{\pgfqpoint{0.688192in}{0.670138in}}{\pgfqpoint{7.111808in}{5.061530in}}%
\pgfusepath{clip}%
\pgfsetbuttcap%
\pgfsetroundjoin%
\pgfsetlinewidth{1.003750pt}%
\definecolor{currentstroke}{rgb}{0.000000,0.000000,0.000000}%
\pgfsetstrokecolor{currentstroke}%
\pgfsetdash{}{0pt}%
\pgfpathmoveto{\pgfqpoint{1.178009in}{0.644363in}}%
\pgfpathcurveto{\pgfqpoint{1.188905in}{0.644363in}}{\pgfqpoint{1.199355in}{0.648692in}}{\pgfqpoint{1.207060in}{0.656396in}}%
\pgfpathcurveto{\pgfqpoint{1.214764in}{0.664100in}}{\pgfqpoint{1.219093in}{0.674551in}}{\pgfqpoint{1.219093in}{0.685447in}}%
\pgfpathcurveto{\pgfqpoint{1.219093in}{0.696342in}}{\pgfqpoint{1.214764in}{0.706793in}}{\pgfqpoint{1.207060in}{0.714497in}}%
\pgfpathcurveto{\pgfqpoint{1.199355in}{0.722202in}}{\pgfqpoint{1.188905in}{0.726531in}}{\pgfqpoint{1.178009in}{0.726531in}}%
\pgfpathcurveto{\pgfqpoint{1.167113in}{0.726531in}}{\pgfqpoint{1.156663in}{0.722202in}}{\pgfqpoint{1.148958in}{0.714497in}}%
\pgfpathcurveto{\pgfqpoint{1.141254in}{0.706793in}}{\pgfqpoint{1.136925in}{0.696342in}}{\pgfqpoint{1.136925in}{0.685447in}}%
\pgfpathcurveto{\pgfqpoint{1.136925in}{0.674551in}}{\pgfqpoint{1.141254in}{0.664100in}}{\pgfqpoint{1.148958in}{0.656396in}}%
\pgfpathcurveto{\pgfqpoint{1.156663in}{0.648692in}}{\pgfqpoint{1.167113in}{0.644363in}}{\pgfqpoint{1.178009in}{0.644363in}}%
\pgfusepath{stroke}%
\end{pgfscope}%
\begin{pgfscope}%
\pgfpathrectangle{\pgfqpoint{0.688192in}{0.670138in}}{\pgfqpoint{7.111808in}{5.061530in}}%
\pgfusepath{clip}%
\pgfsetbuttcap%
\pgfsetroundjoin%
\pgfsetlinewidth{1.003750pt}%
\definecolor{currentstroke}{rgb}{0.000000,0.000000,0.000000}%
\pgfsetstrokecolor{currentstroke}%
\pgfsetdash{}{0pt}%
\pgfpathmoveto{\pgfqpoint{4.979341in}{0.630236in}}%
\pgfpathcurveto{\pgfqpoint{4.990237in}{0.630236in}}{\pgfqpoint{5.000687in}{0.634565in}}{\pgfqpoint{5.008392in}{0.642269in}}%
\pgfpathcurveto{\pgfqpoint{5.016096in}{0.649973in}}{\pgfqpoint{5.020425in}{0.660424in}}{\pgfqpoint{5.020425in}{0.671320in}}%
\pgfpathcurveto{\pgfqpoint{5.020425in}{0.682215in}}{\pgfqpoint{5.016096in}{0.692666in}}{\pgfqpoint{5.008392in}{0.700370in}}%
\pgfpathcurveto{\pgfqpoint{5.000687in}{0.708075in}}{\pgfqpoint{4.990237in}{0.712403in}}{\pgfqpoint{4.979341in}{0.712403in}}%
\pgfpathcurveto{\pgfqpoint{4.968446in}{0.712403in}}{\pgfqpoint{4.957995in}{0.708075in}}{\pgfqpoint{4.950290in}{0.700370in}}%
\pgfpathcurveto{\pgfqpoint{4.942586in}{0.692666in}}{\pgfqpoint{4.938257in}{0.682215in}}{\pgfqpoint{4.938257in}{0.671320in}}%
\pgfpathcurveto{\pgfqpoint{4.938257in}{0.660424in}}{\pgfqpoint{4.942586in}{0.649973in}}{\pgfqpoint{4.950290in}{0.642269in}}%
\pgfpathcurveto{\pgfqpoint{4.957995in}{0.634565in}}{\pgfqpoint{4.968446in}{0.630236in}}{\pgfqpoint{4.979341in}{0.630236in}}%
\pgfusepath{stroke}%
\end{pgfscope}%
\begin{pgfscope}%
\pgfpathrectangle{\pgfqpoint{0.688192in}{0.670138in}}{\pgfqpoint{7.111808in}{5.061530in}}%
\pgfusepath{clip}%
\pgfsetbuttcap%
\pgfsetroundjoin%
\pgfsetlinewidth{1.003750pt}%
\definecolor{currentstroke}{rgb}{0.000000,0.000000,0.000000}%
\pgfsetstrokecolor{currentstroke}%
\pgfsetdash{}{0pt}%
\pgfpathmoveto{\pgfqpoint{0.954596in}{0.669135in}}%
\pgfpathcurveto{\pgfqpoint{0.965492in}{0.669135in}}{\pgfqpoint{0.975942in}{0.673464in}}{\pgfqpoint{0.983647in}{0.681169in}}%
\pgfpathcurveto{\pgfqpoint{0.991351in}{0.688873in}}{\pgfqpoint{0.995680in}{0.699324in}}{\pgfqpoint{0.995680in}{0.710219in}}%
\pgfpathcurveto{\pgfqpoint{0.995680in}{0.721115in}}{\pgfqpoint{0.991351in}{0.731566in}}{\pgfqpoint{0.983647in}{0.739270in}}%
\pgfpathcurveto{\pgfqpoint{0.975942in}{0.746974in}}{\pgfqpoint{0.965492in}{0.751303in}}{\pgfqpoint{0.954596in}{0.751303in}}%
\pgfpathcurveto{\pgfqpoint{0.943700in}{0.751303in}}{\pgfqpoint{0.933250in}{0.746974in}}{\pgfqpoint{0.925545in}{0.739270in}}%
\pgfpathcurveto{\pgfqpoint{0.917841in}{0.731566in}}{\pgfqpoint{0.913512in}{0.721115in}}{\pgfqpoint{0.913512in}{0.710219in}}%
\pgfpathcurveto{\pgfqpoint{0.913512in}{0.699324in}}{\pgfqpoint{0.917841in}{0.688873in}}{\pgfqpoint{0.925545in}{0.681169in}}%
\pgfpathcurveto{\pgfqpoint{0.933250in}{0.673464in}}{\pgfqpoint{0.943700in}{0.669135in}}{\pgfqpoint{0.954596in}{0.669135in}}%
\pgfpathlineto{\pgfqpoint{0.954596in}{0.669135in}}%
\pgfpathclose%
\pgfusepath{stroke}%
\end{pgfscope}%
\begin{pgfscope}%
\pgfpathrectangle{\pgfqpoint{0.688192in}{0.670138in}}{\pgfqpoint{7.111808in}{5.061530in}}%
\pgfusepath{clip}%
\pgfsetbuttcap%
\pgfsetroundjoin%
\pgfsetlinewidth{1.003750pt}%
\definecolor{currentstroke}{rgb}{0.000000,0.000000,0.000000}%
\pgfsetstrokecolor{currentstroke}%
\pgfsetdash{}{0pt}%
\pgfpathmoveto{\pgfqpoint{2.768315in}{3.990841in}}%
\pgfpathcurveto{\pgfqpoint{2.779211in}{3.990841in}}{\pgfqpoint{2.789661in}{3.995170in}}{\pgfqpoint{2.797366in}{4.002874in}}%
\pgfpathcurveto{\pgfqpoint{2.805070in}{4.010579in}}{\pgfqpoint{2.809399in}{4.021029in}}{\pgfqpoint{2.809399in}{4.031925in}}%
\pgfpathcurveto{\pgfqpoint{2.809399in}{4.042821in}}{\pgfqpoint{2.805070in}{4.053271in}}{\pgfqpoint{2.797366in}{4.060976in}}%
\pgfpathcurveto{\pgfqpoint{2.789661in}{4.068680in}}{\pgfqpoint{2.779211in}{4.073009in}}{\pgfqpoint{2.768315in}{4.073009in}}%
\pgfpathcurveto{\pgfqpoint{2.757419in}{4.073009in}}{\pgfqpoint{2.746969in}{4.068680in}}{\pgfqpoint{2.739264in}{4.060976in}}%
\pgfpathcurveto{\pgfqpoint{2.731560in}{4.053271in}}{\pgfqpoint{2.727231in}{4.042821in}}{\pgfqpoint{2.727231in}{4.031925in}}%
\pgfpathcurveto{\pgfqpoint{2.727231in}{4.021029in}}{\pgfqpoint{2.731560in}{4.010579in}}{\pgfqpoint{2.739264in}{4.002874in}}%
\pgfpathcurveto{\pgfqpoint{2.746969in}{3.995170in}}{\pgfqpoint{2.757419in}{3.990841in}}{\pgfqpoint{2.768315in}{3.990841in}}%
\pgfpathlineto{\pgfqpoint{2.768315in}{3.990841in}}%
\pgfpathclose%
\pgfusepath{stroke}%
\end{pgfscope}%
\begin{pgfscope}%
\pgfpathrectangle{\pgfqpoint{0.688192in}{0.670138in}}{\pgfqpoint{7.111808in}{5.061530in}}%
\pgfusepath{clip}%
\pgfsetbuttcap%
\pgfsetroundjoin%
\pgfsetlinewidth{1.003750pt}%
\definecolor{currentstroke}{rgb}{0.000000,0.000000,0.000000}%
\pgfsetstrokecolor{currentstroke}%
\pgfsetdash{}{0pt}%
\pgfpathmoveto{\pgfqpoint{0.954596in}{0.669135in}}%
\pgfpathcurveto{\pgfqpoint{0.965492in}{0.669135in}}{\pgfqpoint{0.975942in}{0.673464in}}{\pgfqpoint{0.983647in}{0.681169in}}%
\pgfpathcurveto{\pgfqpoint{0.991351in}{0.688873in}}{\pgfqpoint{0.995680in}{0.699324in}}{\pgfqpoint{0.995680in}{0.710219in}}%
\pgfpathcurveto{\pgfqpoint{0.995680in}{0.721115in}}{\pgfqpoint{0.991351in}{0.731566in}}{\pgfqpoint{0.983647in}{0.739270in}}%
\pgfpathcurveto{\pgfqpoint{0.975942in}{0.746974in}}{\pgfqpoint{0.965492in}{0.751303in}}{\pgfqpoint{0.954596in}{0.751303in}}%
\pgfpathcurveto{\pgfqpoint{0.943700in}{0.751303in}}{\pgfqpoint{0.933250in}{0.746974in}}{\pgfqpoint{0.925545in}{0.739270in}}%
\pgfpathcurveto{\pgfqpoint{0.917841in}{0.731566in}}{\pgfqpoint{0.913512in}{0.721115in}}{\pgfqpoint{0.913512in}{0.710219in}}%
\pgfpathcurveto{\pgfqpoint{0.913512in}{0.699324in}}{\pgfqpoint{0.917841in}{0.688873in}}{\pgfqpoint{0.925545in}{0.681169in}}%
\pgfpathcurveto{\pgfqpoint{0.933250in}{0.673464in}}{\pgfqpoint{0.943700in}{0.669135in}}{\pgfqpoint{0.954596in}{0.669135in}}%
\pgfpathlineto{\pgfqpoint{0.954596in}{0.669135in}}%
\pgfpathclose%
\pgfusepath{stroke}%
\end{pgfscope}%
\begin{pgfscope}%
\pgfpathrectangle{\pgfqpoint{0.688192in}{0.670138in}}{\pgfqpoint{7.111808in}{5.061530in}}%
\pgfusepath{clip}%
\pgfsetbuttcap%
\pgfsetroundjoin%
\pgfsetlinewidth{1.003750pt}%
\definecolor{currentstroke}{rgb}{0.000000,0.000000,0.000000}%
\pgfsetstrokecolor{currentstroke}%
\pgfsetdash{}{0pt}%
\pgfpathmoveto{\pgfqpoint{4.207141in}{0.631198in}}%
\pgfpathcurveto{\pgfqpoint{4.218036in}{0.631198in}}{\pgfqpoint{4.228487in}{0.635527in}}{\pgfqpoint{4.236191in}{0.643231in}}%
\pgfpathcurveto{\pgfqpoint{4.243896in}{0.650936in}}{\pgfqpoint{4.248225in}{0.661387in}}{\pgfqpoint{4.248225in}{0.672282in}}%
\pgfpathcurveto{\pgfqpoint{4.248225in}{0.683178in}}{\pgfqpoint{4.243896in}{0.693629in}}{\pgfqpoint{4.236191in}{0.701333in}}%
\pgfpathcurveto{\pgfqpoint{4.228487in}{0.709037in}}{\pgfqpoint{4.218036in}{0.713366in}}{\pgfqpoint{4.207141in}{0.713366in}}%
\pgfpathcurveto{\pgfqpoint{4.196245in}{0.713366in}}{\pgfqpoint{4.185794in}{0.709037in}}{\pgfqpoint{4.178090in}{0.701333in}}%
\pgfpathcurveto{\pgfqpoint{4.170386in}{0.693629in}}{\pgfqpoint{4.166057in}{0.683178in}}{\pgfqpoint{4.166057in}{0.672282in}}%
\pgfpathcurveto{\pgfqpoint{4.166057in}{0.661387in}}{\pgfqpoint{4.170386in}{0.650936in}}{\pgfqpoint{4.178090in}{0.643231in}}%
\pgfpathcurveto{\pgfqpoint{4.185794in}{0.635527in}}{\pgfqpoint{4.196245in}{0.631198in}}{\pgfqpoint{4.207141in}{0.631198in}}%
\pgfusepath{stroke}%
\end{pgfscope}%
\begin{pgfscope}%
\pgfpathrectangle{\pgfqpoint{0.688192in}{0.670138in}}{\pgfqpoint{7.111808in}{5.061530in}}%
\pgfusepath{clip}%
\pgfsetbuttcap%
\pgfsetroundjoin%
\pgfsetlinewidth{1.003750pt}%
\definecolor{currentstroke}{rgb}{0.000000,0.000000,0.000000}%
\pgfsetstrokecolor{currentstroke}%
\pgfsetdash{}{0pt}%
\pgfpathmoveto{\pgfqpoint{2.033007in}{1.968580in}}%
\pgfpathcurveto{\pgfqpoint{2.043903in}{1.968580in}}{\pgfqpoint{2.054354in}{1.972909in}}{\pgfqpoint{2.062058in}{1.980613in}}%
\pgfpathcurveto{\pgfqpoint{2.069762in}{1.988317in}}{\pgfqpoint{2.074091in}{1.998768in}}{\pgfqpoint{2.074091in}{2.009664in}}%
\pgfpathcurveto{\pgfqpoint{2.074091in}{2.020559in}}{\pgfqpoint{2.069762in}{2.031010in}}{\pgfqpoint{2.062058in}{2.038714in}}%
\pgfpathcurveto{\pgfqpoint{2.054354in}{2.046419in}}{\pgfqpoint{2.043903in}{2.050748in}}{\pgfqpoint{2.033007in}{2.050748in}}%
\pgfpathcurveto{\pgfqpoint{2.022112in}{2.050748in}}{\pgfqpoint{2.011661in}{2.046419in}}{\pgfqpoint{2.003957in}{2.038714in}}%
\pgfpathcurveto{\pgfqpoint{1.996252in}{2.031010in}}{\pgfqpoint{1.991923in}{2.020559in}}{\pgfqpoint{1.991923in}{2.009664in}}%
\pgfpathcurveto{\pgfqpoint{1.991923in}{1.998768in}}{\pgfqpoint{1.996252in}{1.988317in}}{\pgfqpoint{2.003957in}{1.980613in}}%
\pgfpathcurveto{\pgfqpoint{2.011661in}{1.972909in}}{\pgfqpoint{2.022112in}{1.968580in}}{\pgfqpoint{2.033007in}{1.968580in}}%
\pgfpathlineto{\pgfqpoint{2.033007in}{1.968580in}}%
\pgfpathclose%
\pgfusepath{stroke}%
\end{pgfscope}%
\begin{pgfscope}%
\pgfpathrectangle{\pgfqpoint{0.688192in}{0.670138in}}{\pgfqpoint{7.111808in}{5.061530in}}%
\pgfusepath{clip}%
\pgfsetbuttcap%
\pgfsetroundjoin%
\pgfsetlinewidth{1.003750pt}%
\definecolor{currentstroke}{rgb}{0.000000,0.000000,0.000000}%
\pgfsetstrokecolor{currentstroke}%
\pgfsetdash{}{0pt}%
\pgfpathmoveto{\pgfqpoint{4.965126in}{0.630283in}}%
\pgfpathcurveto{\pgfqpoint{4.976022in}{0.630283in}}{\pgfqpoint{4.986472in}{0.634611in}}{\pgfqpoint{4.994177in}{0.642316in}}%
\pgfpathcurveto{\pgfqpoint{5.001881in}{0.650020in}}{\pgfqpoint{5.006210in}{0.660471in}}{\pgfqpoint{5.006210in}{0.671366in}}%
\pgfpathcurveto{\pgfqpoint{5.006210in}{0.682262in}}{\pgfqpoint{5.001881in}{0.692713in}}{\pgfqpoint{4.994177in}{0.700417in}}%
\pgfpathcurveto{\pgfqpoint{4.986472in}{0.708122in}}{\pgfqpoint{4.976022in}{0.712450in}}{\pgfqpoint{4.965126in}{0.712450in}}%
\pgfpathcurveto{\pgfqpoint{4.954230in}{0.712450in}}{\pgfqpoint{4.943780in}{0.708122in}}{\pgfqpoint{4.936075in}{0.700417in}}%
\pgfpathcurveto{\pgfqpoint{4.928371in}{0.692713in}}{\pgfqpoint{4.924042in}{0.682262in}}{\pgfqpoint{4.924042in}{0.671366in}}%
\pgfpathcurveto{\pgfqpoint{4.924042in}{0.660471in}}{\pgfqpoint{4.928371in}{0.650020in}}{\pgfqpoint{4.936075in}{0.642316in}}%
\pgfpathcurveto{\pgfqpoint{4.943780in}{0.634611in}}{\pgfqpoint{4.954230in}{0.630283in}}{\pgfqpoint{4.965126in}{0.630283in}}%
\pgfusepath{stroke}%
\end{pgfscope}%
\begin{pgfscope}%
\pgfpathrectangle{\pgfqpoint{0.688192in}{0.670138in}}{\pgfqpoint{7.111808in}{5.061530in}}%
\pgfusepath{clip}%
\pgfsetbuttcap%
\pgfsetroundjoin%
\pgfsetlinewidth{1.003750pt}%
\definecolor{currentstroke}{rgb}{0.000000,0.000000,0.000000}%
\pgfsetstrokecolor{currentstroke}%
\pgfsetdash{}{0pt}%
\pgfpathmoveto{\pgfqpoint{1.151625in}{0.644694in}}%
\pgfpathcurveto{\pgfqpoint{1.162521in}{0.644694in}}{\pgfqpoint{1.172972in}{0.649023in}}{\pgfqpoint{1.180676in}{0.656727in}}%
\pgfpathcurveto{\pgfqpoint{1.188380in}{0.664431in}}{\pgfqpoint{1.192709in}{0.674882in}}{\pgfqpoint{1.192709in}{0.685778in}}%
\pgfpathcurveto{\pgfqpoint{1.192709in}{0.696673in}}{\pgfqpoint{1.188380in}{0.707124in}}{\pgfqpoint{1.180676in}{0.714828in}}%
\pgfpathcurveto{\pgfqpoint{1.172972in}{0.722533in}}{\pgfqpoint{1.162521in}{0.726862in}}{\pgfqpoint{1.151625in}{0.726862in}}%
\pgfpathcurveto{\pgfqpoint{1.140730in}{0.726862in}}{\pgfqpoint{1.130279in}{0.722533in}}{\pgfqpoint{1.122575in}{0.714828in}}%
\pgfpathcurveto{\pgfqpoint{1.114870in}{0.707124in}}{\pgfqpoint{1.110542in}{0.696673in}}{\pgfqpoint{1.110542in}{0.685778in}}%
\pgfpathcurveto{\pgfqpoint{1.110542in}{0.674882in}}{\pgfqpoint{1.114870in}{0.664431in}}{\pgfqpoint{1.122575in}{0.656727in}}%
\pgfpathcurveto{\pgfqpoint{1.130279in}{0.649023in}}{\pgfqpoint{1.140730in}{0.644694in}}{\pgfqpoint{1.151625in}{0.644694in}}%
\pgfusepath{stroke}%
\end{pgfscope}%
\begin{pgfscope}%
\pgfpathrectangle{\pgfqpoint{0.688192in}{0.670138in}}{\pgfqpoint{7.111808in}{5.061530in}}%
\pgfusepath{clip}%
\pgfsetbuttcap%
\pgfsetroundjoin%
\pgfsetlinewidth{1.003750pt}%
\definecolor{currentstroke}{rgb}{0.000000,0.000000,0.000000}%
\pgfsetstrokecolor{currentstroke}%
\pgfsetdash{}{0pt}%
\pgfpathmoveto{\pgfqpoint{6.063101in}{1.389347in}}%
\pgfpathcurveto{\pgfqpoint{6.073997in}{1.389347in}}{\pgfqpoint{6.084448in}{1.393675in}}{\pgfqpoint{6.092152in}{1.401380in}}%
\pgfpathcurveto{\pgfqpoint{6.099856in}{1.409084in}}{\pgfqpoint{6.104185in}{1.419535in}}{\pgfqpoint{6.104185in}{1.430431in}}%
\pgfpathcurveto{\pgfqpoint{6.104185in}{1.441326in}}{\pgfqpoint{6.099856in}{1.451777in}}{\pgfqpoint{6.092152in}{1.459481in}}%
\pgfpathcurveto{\pgfqpoint{6.084448in}{1.467186in}}{\pgfqpoint{6.073997in}{1.471514in}}{\pgfqpoint{6.063101in}{1.471514in}}%
\pgfpathcurveto{\pgfqpoint{6.052206in}{1.471514in}}{\pgfqpoint{6.041755in}{1.467186in}}{\pgfqpoint{6.034051in}{1.459481in}}%
\pgfpathcurveto{\pgfqpoint{6.026346in}{1.451777in}}{\pgfqpoint{6.022018in}{1.441326in}}{\pgfqpoint{6.022018in}{1.430431in}}%
\pgfpathcurveto{\pgfqpoint{6.022018in}{1.419535in}}{\pgfqpoint{6.026346in}{1.409084in}}{\pgfqpoint{6.034051in}{1.401380in}}%
\pgfpathcurveto{\pgfqpoint{6.041755in}{1.393675in}}{\pgfqpoint{6.052206in}{1.389347in}}{\pgfqpoint{6.063101in}{1.389347in}}%
\pgfpathlineto{\pgfqpoint{6.063101in}{1.389347in}}%
\pgfpathclose%
\pgfusepath{stroke}%
\end{pgfscope}%
\begin{pgfscope}%
\pgfpathrectangle{\pgfqpoint{0.688192in}{0.670138in}}{\pgfqpoint{7.111808in}{5.061530in}}%
\pgfusepath{clip}%
\pgfsetbuttcap%
\pgfsetroundjoin%
\pgfsetlinewidth{1.003750pt}%
\definecolor{currentstroke}{rgb}{0.000000,0.000000,0.000000}%
\pgfsetstrokecolor{currentstroke}%
\pgfsetdash{}{0pt}%
\pgfpathmoveto{\pgfqpoint{1.407033in}{0.642986in}}%
\pgfpathcurveto{\pgfqpoint{1.417928in}{0.642986in}}{\pgfqpoint{1.428379in}{0.647315in}}{\pgfqpoint{1.436083in}{0.655020in}}%
\pgfpathcurveto{\pgfqpoint{1.443788in}{0.662724in}}{\pgfqpoint{1.448117in}{0.673175in}}{\pgfqpoint{1.448117in}{0.684070in}}%
\pgfpathcurveto{\pgfqpoint{1.448117in}{0.694966in}}{\pgfqpoint{1.443788in}{0.705417in}}{\pgfqpoint{1.436083in}{0.713121in}}%
\pgfpathcurveto{\pgfqpoint{1.428379in}{0.720825in}}{\pgfqpoint{1.417928in}{0.725154in}}{\pgfqpoint{1.407033in}{0.725154in}}%
\pgfpathcurveto{\pgfqpoint{1.396137in}{0.725154in}}{\pgfqpoint{1.385686in}{0.720825in}}{\pgfqpoint{1.377982in}{0.713121in}}%
\pgfpathcurveto{\pgfqpoint{1.370278in}{0.705417in}}{\pgfqpoint{1.365949in}{0.694966in}}{\pgfqpoint{1.365949in}{0.684070in}}%
\pgfpathcurveto{\pgfqpoint{1.365949in}{0.673175in}}{\pgfqpoint{1.370278in}{0.662724in}}{\pgfqpoint{1.377982in}{0.655020in}}%
\pgfpathcurveto{\pgfqpoint{1.385686in}{0.647315in}}{\pgfqpoint{1.396137in}{0.642986in}}{\pgfqpoint{1.407033in}{0.642986in}}%
\pgfusepath{stroke}%
\end{pgfscope}%
\begin{pgfscope}%
\pgfpathrectangle{\pgfqpoint{0.688192in}{0.670138in}}{\pgfqpoint{7.111808in}{5.061530in}}%
\pgfusepath{clip}%
\pgfsetbuttcap%
\pgfsetroundjoin%
\pgfsetlinewidth{1.003750pt}%
\definecolor{currentstroke}{rgb}{0.000000,0.000000,0.000000}%
\pgfsetstrokecolor{currentstroke}%
\pgfsetdash{}{0pt}%
\pgfpathmoveto{\pgfqpoint{2.932466in}{0.635354in}}%
\pgfpathcurveto{\pgfqpoint{2.943362in}{0.635354in}}{\pgfqpoint{2.953812in}{0.639683in}}{\pgfqpoint{2.961517in}{0.647387in}}%
\pgfpathcurveto{\pgfqpoint{2.969221in}{0.655091in}}{\pgfqpoint{2.973550in}{0.665542in}}{\pgfqpoint{2.973550in}{0.676438in}}%
\pgfpathcurveto{\pgfqpoint{2.973550in}{0.687333in}}{\pgfqpoint{2.969221in}{0.697784in}}{\pgfqpoint{2.961517in}{0.705488in}}%
\pgfpathcurveto{\pgfqpoint{2.953812in}{0.713193in}}{\pgfqpoint{2.943362in}{0.717522in}}{\pgfqpoint{2.932466in}{0.717522in}}%
\pgfpathcurveto{\pgfqpoint{2.921571in}{0.717522in}}{\pgfqpoint{2.911120in}{0.713193in}}{\pgfqpoint{2.903415in}{0.705488in}}%
\pgfpathcurveto{\pgfqpoint{2.895711in}{0.697784in}}{\pgfqpoint{2.891382in}{0.687333in}}{\pgfqpoint{2.891382in}{0.676438in}}%
\pgfpathcurveto{\pgfqpoint{2.891382in}{0.665542in}}{\pgfqpoint{2.895711in}{0.655091in}}{\pgfqpoint{2.903415in}{0.647387in}}%
\pgfpathcurveto{\pgfqpoint{2.911120in}{0.639683in}}{\pgfqpoint{2.921571in}{0.635354in}}{\pgfqpoint{2.932466in}{0.635354in}}%
\pgfusepath{stroke}%
\end{pgfscope}%
\begin{pgfscope}%
\pgfpathrectangle{\pgfqpoint{0.688192in}{0.670138in}}{\pgfqpoint{7.111808in}{5.061530in}}%
\pgfusepath{clip}%
\pgfsetbuttcap%
\pgfsetroundjoin%
\pgfsetlinewidth{1.003750pt}%
\definecolor{currentstroke}{rgb}{0.000000,0.000000,0.000000}%
\pgfsetstrokecolor{currentstroke}%
\pgfsetdash{}{0pt}%
\pgfpathmoveto{\pgfqpoint{1.473551in}{0.642864in}}%
\pgfpathcurveto{\pgfqpoint{1.484447in}{0.642864in}}{\pgfqpoint{1.494898in}{0.647193in}}{\pgfqpoint{1.502602in}{0.654897in}}%
\pgfpathcurveto{\pgfqpoint{1.510306in}{0.662602in}}{\pgfqpoint{1.514635in}{0.673052in}}{\pgfqpoint{1.514635in}{0.683948in}}%
\pgfpathcurveto{\pgfqpoint{1.514635in}{0.694844in}}{\pgfqpoint{1.510306in}{0.705294in}}{\pgfqpoint{1.502602in}{0.712999in}}%
\pgfpathcurveto{\pgfqpoint{1.494898in}{0.720703in}}{\pgfqpoint{1.484447in}{0.725032in}}{\pgfqpoint{1.473551in}{0.725032in}}%
\pgfpathcurveto{\pgfqpoint{1.462656in}{0.725032in}}{\pgfqpoint{1.452205in}{0.720703in}}{\pgfqpoint{1.444500in}{0.712999in}}%
\pgfpathcurveto{\pgfqpoint{1.436796in}{0.705294in}}{\pgfqpoint{1.432467in}{0.694844in}}{\pgfqpoint{1.432467in}{0.683948in}}%
\pgfpathcurveto{\pgfqpoint{1.432467in}{0.673052in}}{\pgfqpoint{1.436796in}{0.662602in}}{\pgfqpoint{1.444500in}{0.654897in}}%
\pgfpathcurveto{\pgfqpoint{1.452205in}{0.647193in}}{\pgfqpoint{1.462656in}{0.642864in}}{\pgfqpoint{1.473551in}{0.642864in}}%
\pgfusepath{stroke}%
\end{pgfscope}%
\begin{pgfscope}%
\pgfpathrectangle{\pgfqpoint{0.688192in}{0.670138in}}{\pgfqpoint{7.111808in}{5.061530in}}%
\pgfusepath{clip}%
\pgfsetbuttcap%
\pgfsetroundjoin%
\pgfsetlinewidth{1.003750pt}%
\definecolor{currentstroke}{rgb}{0.000000,0.000000,0.000000}%
\pgfsetstrokecolor{currentstroke}%
\pgfsetdash{}{0pt}%
\pgfpathmoveto{\pgfqpoint{1.732115in}{0.641446in}}%
\pgfpathcurveto{\pgfqpoint{1.743010in}{0.641446in}}{\pgfqpoint{1.753461in}{0.645774in}}{\pgfqpoint{1.761165in}{0.653479in}}%
\pgfpathcurveto{\pgfqpoint{1.768870in}{0.661183in}}{\pgfqpoint{1.773199in}{0.671634in}}{\pgfqpoint{1.773199in}{0.682529in}}%
\pgfpathcurveto{\pgfqpoint{1.773199in}{0.693425in}}{\pgfqpoint{1.768870in}{0.703876in}}{\pgfqpoint{1.761165in}{0.711580in}}%
\pgfpathcurveto{\pgfqpoint{1.753461in}{0.719285in}}{\pgfqpoint{1.743010in}{0.723613in}}{\pgfqpoint{1.732115in}{0.723613in}}%
\pgfpathcurveto{\pgfqpoint{1.721219in}{0.723613in}}{\pgfqpoint{1.710768in}{0.719285in}}{\pgfqpoint{1.703064in}{0.711580in}}%
\pgfpathcurveto{\pgfqpoint{1.695360in}{0.703876in}}{\pgfqpoint{1.691031in}{0.693425in}}{\pgfqpoint{1.691031in}{0.682529in}}%
\pgfpathcurveto{\pgfqpoint{1.691031in}{0.671634in}}{\pgfqpoint{1.695360in}{0.661183in}}{\pgfqpoint{1.703064in}{0.653479in}}%
\pgfpathcurveto{\pgfqpoint{1.710768in}{0.645774in}}{\pgfqpoint{1.721219in}{0.641446in}}{\pgfqpoint{1.732115in}{0.641446in}}%
\pgfusepath{stroke}%
\end{pgfscope}%
\begin{pgfscope}%
\pgfpathrectangle{\pgfqpoint{0.688192in}{0.670138in}}{\pgfqpoint{7.111808in}{5.061530in}}%
\pgfusepath{clip}%
\pgfsetbuttcap%
\pgfsetroundjoin%
\pgfsetlinewidth{1.003750pt}%
\definecolor{currentstroke}{rgb}{0.000000,0.000000,0.000000}%
\pgfsetstrokecolor{currentstroke}%
\pgfsetdash{}{0pt}%
\pgfpathmoveto{\pgfqpoint{0.749254in}{1.404065in}}%
\pgfpathcurveto{\pgfqpoint{0.760150in}{1.404065in}}{\pgfqpoint{0.770600in}{1.408394in}}{\pgfqpoint{0.778305in}{1.416099in}}%
\pgfpathcurveto{\pgfqpoint{0.786009in}{1.423803in}}{\pgfqpoint{0.790338in}{1.434254in}}{\pgfqpoint{0.790338in}{1.445149in}}%
\pgfpathcurveto{\pgfqpoint{0.790338in}{1.456045in}}{\pgfqpoint{0.786009in}{1.466496in}}{\pgfqpoint{0.778305in}{1.474200in}}%
\pgfpathcurveto{\pgfqpoint{0.770600in}{1.481904in}}{\pgfqpoint{0.760150in}{1.486233in}}{\pgfqpoint{0.749254in}{1.486233in}}%
\pgfpathcurveto{\pgfqpoint{0.738358in}{1.486233in}}{\pgfqpoint{0.727908in}{1.481904in}}{\pgfqpoint{0.720203in}{1.474200in}}%
\pgfpathcurveto{\pgfqpoint{0.712499in}{1.466496in}}{\pgfqpoint{0.708170in}{1.456045in}}{\pgfqpoint{0.708170in}{1.445149in}}%
\pgfpathcurveto{\pgfqpoint{0.708170in}{1.434254in}}{\pgfqpoint{0.712499in}{1.423803in}}{\pgfqpoint{0.720203in}{1.416099in}}%
\pgfpathcurveto{\pgfqpoint{0.727908in}{1.408394in}}{\pgfqpoint{0.738358in}{1.404065in}}{\pgfqpoint{0.749254in}{1.404065in}}%
\pgfpathlineto{\pgfqpoint{0.749254in}{1.404065in}}%
\pgfpathclose%
\pgfusepath{stroke}%
\end{pgfscope}%
\begin{pgfscope}%
\pgfpathrectangle{\pgfqpoint{0.688192in}{0.670138in}}{\pgfqpoint{7.111808in}{5.061530in}}%
\pgfusepath{clip}%
\pgfsetbuttcap%
\pgfsetroundjoin%
\pgfsetlinewidth{1.003750pt}%
\definecolor{currentstroke}{rgb}{0.000000,0.000000,0.000000}%
\pgfsetstrokecolor{currentstroke}%
\pgfsetdash{}{0pt}%
\pgfpathmoveto{\pgfqpoint{2.768315in}{3.990841in}}%
\pgfpathcurveto{\pgfqpoint{2.779211in}{3.990841in}}{\pgfqpoint{2.789661in}{3.995170in}}{\pgfqpoint{2.797366in}{4.002874in}}%
\pgfpathcurveto{\pgfqpoint{2.805070in}{4.010579in}}{\pgfqpoint{2.809399in}{4.021029in}}{\pgfqpoint{2.809399in}{4.031925in}}%
\pgfpathcurveto{\pgfqpoint{2.809399in}{4.042821in}}{\pgfqpoint{2.805070in}{4.053271in}}{\pgfqpoint{2.797366in}{4.060976in}}%
\pgfpathcurveto{\pgfqpoint{2.789661in}{4.068680in}}{\pgfqpoint{2.779211in}{4.073009in}}{\pgfqpoint{2.768315in}{4.073009in}}%
\pgfpathcurveto{\pgfqpoint{2.757419in}{4.073009in}}{\pgfqpoint{2.746969in}{4.068680in}}{\pgfqpoint{2.739264in}{4.060976in}}%
\pgfpathcurveto{\pgfqpoint{2.731560in}{4.053271in}}{\pgfqpoint{2.727231in}{4.042821in}}{\pgfqpoint{2.727231in}{4.031925in}}%
\pgfpathcurveto{\pgfqpoint{2.727231in}{4.021029in}}{\pgfqpoint{2.731560in}{4.010579in}}{\pgfqpoint{2.739264in}{4.002874in}}%
\pgfpathcurveto{\pgfqpoint{2.746969in}{3.995170in}}{\pgfqpoint{2.757419in}{3.990841in}}{\pgfqpoint{2.768315in}{3.990841in}}%
\pgfpathlineto{\pgfqpoint{2.768315in}{3.990841in}}%
\pgfpathclose%
\pgfusepath{stroke}%
\end{pgfscope}%
\begin{pgfscope}%
\pgfpathrectangle{\pgfqpoint{0.688192in}{0.670138in}}{\pgfqpoint{7.111808in}{5.061530in}}%
\pgfusepath{clip}%
\pgfsetbuttcap%
\pgfsetroundjoin%
\pgfsetlinewidth{1.003750pt}%
\definecolor{currentstroke}{rgb}{0.000000,0.000000,0.000000}%
\pgfsetstrokecolor{currentstroke}%
\pgfsetdash{}{0pt}%
\pgfpathmoveto{\pgfqpoint{4.141773in}{2.223656in}}%
\pgfpathcurveto{\pgfqpoint{4.152669in}{2.223656in}}{\pgfqpoint{4.163119in}{2.227985in}}{\pgfqpoint{4.170824in}{2.235689in}}%
\pgfpathcurveto{\pgfqpoint{4.178528in}{2.243394in}}{\pgfqpoint{4.182857in}{2.253845in}}{\pgfqpoint{4.182857in}{2.264740in}}%
\pgfpathcurveto{\pgfqpoint{4.182857in}{2.275636in}}{\pgfqpoint{4.178528in}{2.286086in}}{\pgfqpoint{4.170824in}{2.293791in}}%
\pgfpathcurveto{\pgfqpoint{4.163119in}{2.301495in}}{\pgfqpoint{4.152669in}{2.305824in}}{\pgfqpoint{4.141773in}{2.305824in}}%
\pgfpathcurveto{\pgfqpoint{4.130877in}{2.305824in}}{\pgfqpoint{4.120427in}{2.301495in}}{\pgfqpoint{4.112722in}{2.293791in}}%
\pgfpathcurveto{\pgfqpoint{4.105018in}{2.286086in}}{\pgfqpoint{4.100689in}{2.275636in}}{\pgfqpoint{4.100689in}{2.264740in}}%
\pgfpathcurveto{\pgfqpoint{4.100689in}{2.253845in}}{\pgfqpoint{4.105018in}{2.243394in}}{\pgfqpoint{4.112722in}{2.235689in}}%
\pgfpathcurveto{\pgfqpoint{4.120427in}{2.227985in}}{\pgfqpoint{4.130877in}{2.223656in}}{\pgfqpoint{4.141773in}{2.223656in}}%
\pgfpathlineto{\pgfqpoint{4.141773in}{2.223656in}}%
\pgfpathclose%
\pgfusepath{stroke}%
\end{pgfscope}%
\begin{pgfscope}%
\pgfpathrectangle{\pgfqpoint{0.688192in}{0.670138in}}{\pgfqpoint{7.111808in}{5.061530in}}%
\pgfusepath{clip}%
\pgfsetbuttcap%
\pgfsetroundjoin%
\pgfsetlinewidth{1.003750pt}%
\definecolor{currentstroke}{rgb}{0.000000,0.000000,0.000000}%
\pgfsetstrokecolor{currentstroke}%
\pgfsetdash{}{0pt}%
\pgfpathmoveto{\pgfqpoint{1.967310in}{0.639999in}}%
\pgfpathcurveto{\pgfqpoint{1.978206in}{0.639999in}}{\pgfqpoint{1.988657in}{0.644328in}}{\pgfqpoint{1.996361in}{0.652032in}}%
\pgfpathcurveto{\pgfqpoint{2.004065in}{0.659736in}}{\pgfqpoint{2.008394in}{0.670187in}}{\pgfqpoint{2.008394in}{0.681083in}}%
\pgfpathcurveto{\pgfqpoint{2.008394in}{0.691978in}}{\pgfqpoint{2.004065in}{0.702429in}}{\pgfqpoint{1.996361in}{0.710133in}}%
\pgfpathcurveto{\pgfqpoint{1.988657in}{0.717838in}}{\pgfqpoint{1.978206in}{0.722166in}}{\pgfqpoint{1.967310in}{0.722166in}}%
\pgfpathcurveto{\pgfqpoint{1.956415in}{0.722166in}}{\pgfqpoint{1.945964in}{0.717838in}}{\pgfqpoint{1.938259in}{0.710133in}}%
\pgfpathcurveto{\pgfqpoint{1.930555in}{0.702429in}}{\pgfqpoint{1.926226in}{0.691978in}}{\pgfqpoint{1.926226in}{0.681083in}}%
\pgfpathcurveto{\pgfqpoint{1.926226in}{0.670187in}}{\pgfqpoint{1.930555in}{0.659736in}}{\pgfqpoint{1.938259in}{0.652032in}}%
\pgfpathcurveto{\pgfqpoint{1.945964in}{0.644328in}}{\pgfqpoint{1.956415in}{0.639999in}}{\pgfqpoint{1.967310in}{0.639999in}}%
\pgfusepath{stroke}%
\end{pgfscope}%
\begin{pgfscope}%
\pgfpathrectangle{\pgfqpoint{0.688192in}{0.670138in}}{\pgfqpoint{7.111808in}{5.061530in}}%
\pgfusepath{clip}%
\pgfsetbuttcap%
\pgfsetroundjoin%
\pgfsetlinewidth{1.003750pt}%
\definecolor{currentstroke}{rgb}{0.000000,0.000000,0.000000}%
\pgfsetstrokecolor{currentstroke}%
\pgfsetdash{}{0pt}%
\pgfpathmoveto{\pgfqpoint{4.170511in}{1.865457in}}%
\pgfpathcurveto{\pgfqpoint{4.181407in}{1.865457in}}{\pgfqpoint{4.191857in}{1.869786in}}{\pgfqpoint{4.199562in}{1.877490in}}%
\pgfpathcurveto{\pgfqpoint{4.207266in}{1.885195in}}{\pgfqpoint{4.211595in}{1.895645in}}{\pgfqpoint{4.211595in}{1.906541in}}%
\pgfpathcurveto{\pgfqpoint{4.211595in}{1.917436in}}{\pgfqpoint{4.207266in}{1.927887in}}{\pgfqpoint{4.199562in}{1.935592in}}%
\pgfpathcurveto{\pgfqpoint{4.191857in}{1.943296in}}{\pgfqpoint{4.181407in}{1.947625in}}{\pgfqpoint{4.170511in}{1.947625in}}%
\pgfpathcurveto{\pgfqpoint{4.159615in}{1.947625in}}{\pgfqpoint{4.149165in}{1.943296in}}{\pgfqpoint{4.141460in}{1.935592in}}%
\pgfpathcurveto{\pgfqpoint{4.133756in}{1.927887in}}{\pgfqpoint{4.129427in}{1.917436in}}{\pgfqpoint{4.129427in}{1.906541in}}%
\pgfpathcurveto{\pgfqpoint{4.129427in}{1.895645in}}{\pgfqpoint{4.133756in}{1.885195in}}{\pgfqpoint{4.141460in}{1.877490in}}%
\pgfpathcurveto{\pgfqpoint{4.149165in}{1.869786in}}{\pgfqpoint{4.159615in}{1.865457in}}{\pgfqpoint{4.170511in}{1.865457in}}%
\pgfpathlineto{\pgfqpoint{4.170511in}{1.865457in}}%
\pgfpathclose%
\pgfusepath{stroke}%
\end{pgfscope}%
\begin{pgfscope}%
\pgfpathrectangle{\pgfqpoint{0.688192in}{0.670138in}}{\pgfqpoint{7.111808in}{5.061530in}}%
\pgfusepath{clip}%
\pgfsetbuttcap%
\pgfsetroundjoin%
\pgfsetlinewidth{1.003750pt}%
\definecolor{currentstroke}{rgb}{0.000000,0.000000,0.000000}%
\pgfsetstrokecolor{currentstroke}%
\pgfsetdash{}{0pt}%
\pgfpathmoveto{\pgfqpoint{6.733668in}{3.149748in}}%
\pgfpathcurveto{\pgfqpoint{6.744563in}{3.149748in}}{\pgfqpoint{6.755014in}{3.154077in}}{\pgfqpoint{6.762719in}{3.161781in}}%
\pgfpathcurveto{\pgfqpoint{6.770423in}{3.169485in}}{\pgfqpoint{6.774752in}{3.179936in}}{\pgfqpoint{6.774752in}{3.190832in}}%
\pgfpathcurveto{\pgfqpoint{6.774752in}{3.201727in}}{\pgfqpoint{6.770423in}{3.212178in}}{\pgfqpoint{6.762719in}{3.219882in}}%
\pgfpathcurveto{\pgfqpoint{6.755014in}{3.227587in}}{\pgfqpoint{6.744563in}{3.231916in}}{\pgfqpoint{6.733668in}{3.231916in}}%
\pgfpathcurveto{\pgfqpoint{6.722772in}{3.231916in}}{\pgfqpoint{6.712322in}{3.227587in}}{\pgfqpoint{6.704617in}{3.219882in}}%
\pgfpathcurveto{\pgfqpoint{6.696913in}{3.212178in}}{\pgfqpoint{6.692584in}{3.201727in}}{\pgfqpoint{6.692584in}{3.190832in}}%
\pgfpathcurveto{\pgfqpoint{6.692584in}{3.179936in}}{\pgfqpoint{6.696913in}{3.169485in}}{\pgfqpoint{6.704617in}{3.161781in}}%
\pgfpathcurveto{\pgfqpoint{6.712322in}{3.154077in}}{\pgfqpoint{6.722772in}{3.149748in}}{\pgfqpoint{6.733668in}{3.149748in}}%
\pgfpathlineto{\pgfqpoint{6.733668in}{3.149748in}}%
\pgfpathclose%
\pgfusepath{stroke}%
\end{pgfscope}%
\begin{pgfscope}%
\pgfpathrectangle{\pgfqpoint{0.688192in}{0.670138in}}{\pgfqpoint{7.111808in}{5.061530in}}%
\pgfusepath{clip}%
\pgfsetbuttcap%
\pgfsetroundjoin%
\pgfsetlinewidth{1.003750pt}%
\definecolor{currentstroke}{rgb}{0.000000,0.000000,0.000000}%
\pgfsetstrokecolor{currentstroke}%
\pgfsetdash{}{0pt}%
\pgfpathmoveto{\pgfqpoint{3.291859in}{0.633837in}}%
\pgfpathcurveto{\pgfqpoint{3.302755in}{0.633837in}}{\pgfqpoint{3.313206in}{0.638166in}}{\pgfqpoint{3.320910in}{0.645871in}}%
\pgfpathcurveto{\pgfqpoint{3.328614in}{0.653575in}}{\pgfqpoint{3.332943in}{0.664026in}}{\pgfqpoint{3.332943in}{0.674921in}}%
\pgfpathcurveto{\pgfqpoint{3.332943in}{0.685817in}}{\pgfqpoint{3.328614in}{0.696268in}}{\pgfqpoint{3.320910in}{0.703972in}}%
\pgfpathcurveto{\pgfqpoint{3.313206in}{0.711676in}}{\pgfqpoint{3.302755in}{0.716005in}}{\pgfqpoint{3.291859in}{0.716005in}}%
\pgfpathcurveto{\pgfqpoint{3.280964in}{0.716005in}}{\pgfqpoint{3.270513in}{0.711676in}}{\pgfqpoint{3.262809in}{0.703972in}}%
\pgfpathcurveto{\pgfqpoint{3.255104in}{0.696268in}}{\pgfqpoint{3.250775in}{0.685817in}}{\pgfqpoint{3.250775in}{0.674921in}}%
\pgfpathcurveto{\pgfqpoint{3.250775in}{0.664026in}}{\pgfqpoint{3.255104in}{0.653575in}}{\pgfqpoint{3.262809in}{0.645871in}}%
\pgfpathcurveto{\pgfqpoint{3.270513in}{0.638166in}}{\pgfqpoint{3.280964in}{0.633837in}}{\pgfqpoint{3.291859in}{0.633837in}}%
\pgfusepath{stroke}%
\end{pgfscope}%
\begin{pgfscope}%
\pgfpathrectangle{\pgfqpoint{0.688192in}{0.670138in}}{\pgfqpoint{7.111808in}{5.061530in}}%
\pgfusepath{clip}%
\pgfsetbuttcap%
\pgfsetroundjoin%
\pgfsetlinewidth{1.003750pt}%
\definecolor{currentstroke}{rgb}{0.000000,0.000000,0.000000}%
\pgfsetstrokecolor{currentstroke}%
\pgfsetdash{}{0pt}%
\pgfpathmoveto{\pgfqpoint{0.850270in}{0.697314in}}%
\pgfpathcurveto{\pgfqpoint{0.861166in}{0.697314in}}{\pgfqpoint{0.871617in}{0.701642in}}{\pgfqpoint{0.879321in}{0.709347in}}%
\pgfpathcurveto{\pgfqpoint{0.887025in}{0.717051in}}{\pgfqpoint{0.891354in}{0.727502in}}{\pgfqpoint{0.891354in}{0.738398in}}%
\pgfpathcurveto{\pgfqpoint{0.891354in}{0.749293in}}{\pgfqpoint{0.887025in}{0.759744in}}{\pgfqpoint{0.879321in}{0.767448in}}%
\pgfpathcurveto{\pgfqpoint{0.871617in}{0.775153in}}{\pgfqpoint{0.861166in}{0.779481in}}{\pgfqpoint{0.850270in}{0.779481in}}%
\pgfpathcurveto{\pgfqpoint{0.839375in}{0.779481in}}{\pgfqpoint{0.828924in}{0.775153in}}{\pgfqpoint{0.821220in}{0.767448in}}%
\pgfpathcurveto{\pgfqpoint{0.813515in}{0.759744in}}{\pgfqpoint{0.809186in}{0.749293in}}{\pgfqpoint{0.809186in}{0.738398in}}%
\pgfpathcurveto{\pgfqpoint{0.809186in}{0.727502in}}{\pgfqpoint{0.813515in}{0.717051in}}{\pgfqpoint{0.821220in}{0.709347in}}%
\pgfpathcurveto{\pgfqpoint{0.828924in}{0.701642in}}{\pgfqpoint{0.839375in}{0.697314in}}{\pgfqpoint{0.850270in}{0.697314in}}%
\pgfpathlineto{\pgfqpoint{0.850270in}{0.697314in}}%
\pgfpathclose%
\pgfusepath{stroke}%
\end{pgfscope}%
\begin{pgfscope}%
\pgfpathrectangle{\pgfqpoint{0.688192in}{0.670138in}}{\pgfqpoint{7.111808in}{5.061530in}}%
\pgfusepath{clip}%
\pgfsetbuttcap%
\pgfsetroundjoin%
\pgfsetlinewidth{1.003750pt}%
\definecolor{currentstroke}{rgb}{0.000000,0.000000,0.000000}%
\pgfsetstrokecolor{currentstroke}%
\pgfsetdash{}{0pt}%
\pgfpathmoveto{\pgfqpoint{1.732115in}{0.641446in}}%
\pgfpathcurveto{\pgfqpoint{1.743010in}{0.641446in}}{\pgfqpoint{1.753461in}{0.645774in}}{\pgfqpoint{1.761165in}{0.653479in}}%
\pgfpathcurveto{\pgfqpoint{1.768870in}{0.661183in}}{\pgfqpoint{1.773199in}{0.671634in}}{\pgfqpoint{1.773199in}{0.682529in}}%
\pgfpathcurveto{\pgfqpoint{1.773199in}{0.693425in}}{\pgfqpoint{1.768870in}{0.703876in}}{\pgfqpoint{1.761165in}{0.711580in}}%
\pgfpathcurveto{\pgfqpoint{1.753461in}{0.719285in}}{\pgfqpoint{1.743010in}{0.723613in}}{\pgfqpoint{1.732115in}{0.723613in}}%
\pgfpathcurveto{\pgfqpoint{1.721219in}{0.723613in}}{\pgfqpoint{1.710768in}{0.719285in}}{\pgfqpoint{1.703064in}{0.711580in}}%
\pgfpathcurveto{\pgfqpoint{1.695360in}{0.703876in}}{\pgfqpoint{1.691031in}{0.693425in}}{\pgfqpoint{1.691031in}{0.682529in}}%
\pgfpathcurveto{\pgfqpoint{1.691031in}{0.671634in}}{\pgfqpoint{1.695360in}{0.661183in}}{\pgfqpoint{1.703064in}{0.653479in}}%
\pgfpathcurveto{\pgfqpoint{1.710768in}{0.645774in}}{\pgfqpoint{1.721219in}{0.641446in}}{\pgfqpoint{1.732115in}{0.641446in}}%
\pgfusepath{stroke}%
\end{pgfscope}%
\begin{pgfscope}%
\pgfpathrectangle{\pgfqpoint{0.688192in}{0.670138in}}{\pgfqpoint{7.111808in}{5.061530in}}%
\pgfusepath{clip}%
\pgfsetbuttcap%
\pgfsetroundjoin%
\pgfsetlinewidth{1.003750pt}%
\definecolor{currentstroke}{rgb}{0.000000,0.000000,0.000000}%
\pgfsetstrokecolor{currentstroke}%
\pgfsetdash{}{0pt}%
\pgfpathmoveto{\pgfqpoint{1.178009in}{0.644363in}}%
\pgfpathcurveto{\pgfqpoint{1.188905in}{0.644363in}}{\pgfqpoint{1.199355in}{0.648692in}}{\pgfqpoint{1.207060in}{0.656396in}}%
\pgfpathcurveto{\pgfqpoint{1.214764in}{0.664100in}}{\pgfqpoint{1.219093in}{0.674551in}}{\pgfqpoint{1.219093in}{0.685447in}}%
\pgfpathcurveto{\pgfqpoint{1.219093in}{0.696342in}}{\pgfqpoint{1.214764in}{0.706793in}}{\pgfqpoint{1.207060in}{0.714497in}}%
\pgfpathcurveto{\pgfqpoint{1.199355in}{0.722202in}}{\pgfqpoint{1.188905in}{0.726531in}}{\pgfqpoint{1.178009in}{0.726531in}}%
\pgfpathcurveto{\pgfqpoint{1.167113in}{0.726531in}}{\pgfqpoint{1.156663in}{0.722202in}}{\pgfqpoint{1.148958in}{0.714497in}}%
\pgfpathcurveto{\pgfqpoint{1.141254in}{0.706793in}}{\pgfqpoint{1.136925in}{0.696342in}}{\pgfqpoint{1.136925in}{0.685447in}}%
\pgfpathcurveto{\pgfqpoint{1.136925in}{0.674551in}}{\pgfqpoint{1.141254in}{0.664100in}}{\pgfqpoint{1.148958in}{0.656396in}}%
\pgfpathcurveto{\pgfqpoint{1.156663in}{0.648692in}}{\pgfqpoint{1.167113in}{0.644363in}}{\pgfqpoint{1.178009in}{0.644363in}}%
\pgfusepath{stroke}%
\end{pgfscope}%
\begin{pgfscope}%
\pgfpathrectangle{\pgfqpoint{0.688192in}{0.670138in}}{\pgfqpoint{7.111808in}{5.061530in}}%
\pgfusepath{clip}%
\pgfsetbuttcap%
\pgfsetroundjoin%
\pgfsetlinewidth{1.003750pt}%
\definecolor{currentstroke}{rgb}{0.000000,0.000000,0.000000}%
\pgfsetstrokecolor{currentstroke}%
\pgfsetdash{}{0pt}%
\pgfpathmoveto{\pgfqpoint{5.816581in}{0.663595in}}%
\pgfpathcurveto{\pgfqpoint{5.827477in}{0.663595in}}{\pgfqpoint{5.837927in}{0.667924in}}{\pgfqpoint{5.845632in}{0.675628in}}%
\pgfpathcurveto{\pgfqpoint{5.853336in}{0.683333in}}{\pgfqpoint{5.857665in}{0.693783in}}{\pgfqpoint{5.857665in}{0.704679in}}%
\pgfpathcurveto{\pgfqpoint{5.857665in}{0.715575in}}{\pgfqpoint{5.853336in}{0.726025in}}{\pgfqpoint{5.845632in}{0.733730in}}%
\pgfpathcurveto{\pgfqpoint{5.837927in}{0.741434in}}{\pgfqpoint{5.827477in}{0.745763in}}{\pgfqpoint{5.816581in}{0.745763in}}%
\pgfpathcurveto{\pgfqpoint{5.805685in}{0.745763in}}{\pgfqpoint{5.795235in}{0.741434in}}{\pgfqpoint{5.787530in}{0.733730in}}%
\pgfpathcurveto{\pgfqpoint{5.779826in}{0.726025in}}{\pgfqpoint{5.775497in}{0.715575in}}{\pgfqpoint{5.775497in}{0.704679in}}%
\pgfpathcurveto{\pgfqpoint{5.775497in}{0.693783in}}{\pgfqpoint{5.779826in}{0.683333in}}{\pgfqpoint{5.787530in}{0.675628in}}%
\pgfpathcurveto{\pgfqpoint{5.795235in}{0.667924in}}{\pgfqpoint{5.805685in}{0.663595in}}{\pgfqpoint{5.816581in}{0.663595in}}%
\pgfusepath{stroke}%
\end{pgfscope}%
\begin{pgfscope}%
\pgfpathrectangle{\pgfqpoint{0.688192in}{0.670138in}}{\pgfqpoint{7.111808in}{5.061530in}}%
\pgfusepath{clip}%
\pgfsetbuttcap%
\pgfsetroundjoin%
\pgfsetlinewidth{1.003750pt}%
\definecolor{currentstroke}{rgb}{0.000000,0.000000,0.000000}%
\pgfsetstrokecolor{currentstroke}%
\pgfsetdash{}{0pt}%
\pgfpathmoveto{\pgfqpoint{5.159789in}{0.629772in}}%
\pgfpathcurveto{\pgfqpoint{5.170685in}{0.629772in}}{\pgfqpoint{5.181136in}{0.634100in}}{\pgfqpoint{5.188840in}{0.641805in}}%
\pgfpathcurveto{\pgfqpoint{5.196544in}{0.649509in}}{\pgfqpoint{5.200873in}{0.659960in}}{\pgfqpoint{5.200873in}{0.670855in}}%
\pgfpathcurveto{\pgfqpoint{5.200873in}{0.681751in}}{\pgfqpoint{5.196544in}{0.692202in}}{\pgfqpoint{5.188840in}{0.699906in}}%
\pgfpathcurveto{\pgfqpoint{5.181136in}{0.707610in}}{\pgfqpoint{5.170685in}{0.711939in}}{\pgfqpoint{5.159789in}{0.711939in}}%
\pgfpathcurveto{\pgfqpoint{5.148894in}{0.711939in}}{\pgfqpoint{5.138443in}{0.707610in}}{\pgfqpoint{5.130739in}{0.699906in}}%
\pgfpathcurveto{\pgfqpoint{5.123034in}{0.692202in}}{\pgfqpoint{5.118705in}{0.681751in}}{\pgfqpoint{5.118705in}{0.670855in}}%
\pgfpathcurveto{\pgfqpoint{5.118705in}{0.659960in}}{\pgfqpoint{5.123034in}{0.649509in}}{\pgfqpoint{5.130739in}{0.641805in}}%
\pgfpathcurveto{\pgfqpoint{5.138443in}{0.634100in}}{\pgfqpoint{5.148894in}{0.629772in}}{\pgfqpoint{5.159789in}{0.629772in}}%
\pgfusepath{stroke}%
\end{pgfscope}%
\begin{pgfscope}%
\pgfpathrectangle{\pgfqpoint{0.688192in}{0.670138in}}{\pgfqpoint{7.111808in}{5.061530in}}%
\pgfusepath{clip}%
\pgfsetbuttcap%
\pgfsetroundjoin%
\pgfsetlinewidth{1.003750pt}%
\definecolor{currentstroke}{rgb}{0.000000,0.000000,0.000000}%
\pgfsetstrokecolor{currentstroke}%
\pgfsetdash{}{0pt}%
\pgfpathmoveto{\pgfqpoint{4.296409in}{1.792126in}}%
\pgfpathcurveto{\pgfqpoint{4.307305in}{1.792126in}}{\pgfqpoint{4.317755in}{1.796455in}}{\pgfqpoint{4.325460in}{1.804159in}}%
\pgfpathcurveto{\pgfqpoint{4.333164in}{1.811863in}}{\pgfqpoint{4.337493in}{1.822314in}}{\pgfqpoint{4.337493in}{1.833210in}}%
\pgfpathcurveto{\pgfqpoint{4.337493in}{1.844105in}}{\pgfqpoint{4.333164in}{1.854556in}}{\pgfqpoint{4.325460in}{1.862260in}}%
\pgfpathcurveto{\pgfqpoint{4.317755in}{1.869965in}}{\pgfqpoint{4.307305in}{1.874294in}}{\pgfqpoint{4.296409in}{1.874294in}}%
\pgfpathcurveto{\pgfqpoint{4.285513in}{1.874294in}}{\pgfqpoint{4.275063in}{1.869965in}}{\pgfqpoint{4.267358in}{1.862260in}}%
\pgfpathcurveto{\pgfqpoint{4.259654in}{1.854556in}}{\pgfqpoint{4.255325in}{1.844105in}}{\pgfqpoint{4.255325in}{1.833210in}}%
\pgfpathcurveto{\pgfqpoint{4.255325in}{1.822314in}}{\pgfqpoint{4.259654in}{1.811863in}}{\pgfqpoint{4.267358in}{1.804159in}}%
\pgfpathcurveto{\pgfqpoint{4.275063in}{1.796455in}}{\pgfqpoint{4.285513in}{1.792126in}}{\pgfqpoint{4.296409in}{1.792126in}}%
\pgfpathlineto{\pgfqpoint{4.296409in}{1.792126in}}%
\pgfpathclose%
\pgfusepath{stroke}%
\end{pgfscope}%
\begin{pgfscope}%
\pgfpathrectangle{\pgfqpoint{0.688192in}{0.670138in}}{\pgfqpoint{7.111808in}{5.061530in}}%
\pgfusepath{clip}%
\pgfsetbuttcap%
\pgfsetroundjoin%
\pgfsetlinewidth{1.003750pt}%
\definecolor{currentstroke}{rgb}{0.000000,0.000000,0.000000}%
\pgfsetstrokecolor{currentstroke}%
\pgfsetdash{}{0pt}%
\pgfpathmoveto{\pgfqpoint{2.500957in}{2.186214in}}%
\pgfpathcurveto{\pgfqpoint{2.511852in}{2.186214in}}{\pgfqpoint{2.522303in}{2.190543in}}{\pgfqpoint{2.530007in}{2.198247in}}%
\pgfpathcurveto{\pgfqpoint{2.537712in}{2.205951in}}{\pgfqpoint{2.542041in}{2.216402in}}{\pgfqpoint{2.542041in}{2.227298in}}%
\pgfpathcurveto{\pgfqpoint{2.542041in}{2.238193in}}{\pgfqpoint{2.537712in}{2.248644in}}{\pgfqpoint{2.530007in}{2.256349in}}%
\pgfpathcurveto{\pgfqpoint{2.522303in}{2.264053in}}{\pgfqpoint{2.511852in}{2.268382in}}{\pgfqpoint{2.500957in}{2.268382in}}%
\pgfpathcurveto{\pgfqpoint{2.490061in}{2.268382in}}{\pgfqpoint{2.479610in}{2.264053in}}{\pgfqpoint{2.471906in}{2.256349in}}%
\pgfpathcurveto{\pgfqpoint{2.464202in}{2.248644in}}{\pgfqpoint{2.459873in}{2.238193in}}{\pgfqpoint{2.459873in}{2.227298in}}%
\pgfpathcurveto{\pgfqpoint{2.459873in}{2.216402in}}{\pgfqpoint{2.464202in}{2.205951in}}{\pgfqpoint{2.471906in}{2.198247in}}%
\pgfpathcurveto{\pgfqpoint{2.479610in}{2.190543in}}{\pgfqpoint{2.490061in}{2.186214in}}{\pgfqpoint{2.500957in}{2.186214in}}%
\pgfpathlineto{\pgfqpoint{2.500957in}{2.186214in}}%
\pgfpathclose%
\pgfusepath{stroke}%
\end{pgfscope}%
\begin{pgfscope}%
\pgfpathrectangle{\pgfqpoint{0.688192in}{0.670138in}}{\pgfqpoint{7.111808in}{5.061530in}}%
\pgfusepath{clip}%
\pgfsetbuttcap%
\pgfsetroundjoin%
\pgfsetlinewidth{1.003750pt}%
\definecolor{currentstroke}{rgb}{0.000000,0.000000,0.000000}%
\pgfsetstrokecolor{currentstroke}%
\pgfsetdash{}{0pt}%
\pgfpathmoveto{\pgfqpoint{5.557238in}{0.629442in}}%
\pgfpathcurveto{\pgfqpoint{5.568133in}{0.629442in}}{\pgfqpoint{5.578584in}{0.633771in}}{\pgfqpoint{5.586288in}{0.641475in}}%
\pgfpathcurveto{\pgfqpoint{5.593993in}{0.649179in}}{\pgfqpoint{5.598322in}{0.659630in}}{\pgfqpoint{5.598322in}{0.670526in}}%
\pgfpathcurveto{\pgfqpoint{5.598322in}{0.681421in}}{\pgfqpoint{5.593993in}{0.691872in}}{\pgfqpoint{5.586288in}{0.699576in}}%
\pgfpathcurveto{\pgfqpoint{5.578584in}{0.707281in}}{\pgfqpoint{5.568133in}{0.711609in}}{\pgfqpoint{5.557238in}{0.711609in}}%
\pgfpathcurveto{\pgfqpoint{5.546342in}{0.711609in}}{\pgfqpoint{5.535891in}{0.707281in}}{\pgfqpoint{5.528187in}{0.699576in}}%
\pgfpathcurveto{\pgfqpoint{5.520483in}{0.691872in}}{\pgfqpoint{5.516154in}{0.681421in}}{\pgfqpoint{5.516154in}{0.670526in}}%
\pgfpathcurveto{\pgfqpoint{5.516154in}{0.659630in}}{\pgfqpoint{5.520483in}{0.649179in}}{\pgfqpoint{5.528187in}{0.641475in}}%
\pgfpathcurveto{\pgfqpoint{5.535891in}{0.633771in}}{\pgfqpoint{5.546342in}{0.629442in}}{\pgfqpoint{5.557238in}{0.629442in}}%
\pgfusepath{stroke}%
\end{pgfscope}%
\begin{pgfscope}%
\pgfpathrectangle{\pgfqpoint{0.688192in}{0.670138in}}{\pgfqpoint{7.111808in}{5.061530in}}%
\pgfusepath{clip}%
\pgfsetbuttcap%
\pgfsetroundjoin%
\pgfsetlinewidth{1.003750pt}%
\definecolor{currentstroke}{rgb}{0.000000,0.000000,0.000000}%
\pgfsetstrokecolor{currentstroke}%
\pgfsetdash{}{0pt}%
\pgfpathmoveto{\pgfqpoint{0.945763in}{0.669750in}}%
\pgfpathcurveto{\pgfqpoint{0.956659in}{0.669750in}}{\pgfqpoint{0.967110in}{0.674079in}}{\pgfqpoint{0.974814in}{0.681783in}}%
\pgfpathcurveto{\pgfqpoint{0.982518in}{0.689488in}}{\pgfqpoint{0.986847in}{0.699938in}}{\pgfqpoint{0.986847in}{0.710834in}}%
\pgfpathcurveto{\pgfqpoint{0.986847in}{0.721730in}}{\pgfqpoint{0.982518in}{0.732180in}}{\pgfqpoint{0.974814in}{0.739885in}}%
\pgfpathcurveto{\pgfqpoint{0.967110in}{0.747589in}}{\pgfqpoint{0.956659in}{0.751918in}}{\pgfqpoint{0.945763in}{0.751918in}}%
\pgfpathcurveto{\pgfqpoint{0.934868in}{0.751918in}}{\pgfqpoint{0.924417in}{0.747589in}}{\pgfqpoint{0.916713in}{0.739885in}}%
\pgfpathcurveto{\pgfqpoint{0.909008in}{0.732180in}}{\pgfqpoint{0.904679in}{0.721730in}}{\pgfqpoint{0.904679in}{0.710834in}}%
\pgfpathcurveto{\pgfqpoint{0.904679in}{0.699938in}}{\pgfqpoint{0.909008in}{0.689488in}}{\pgfqpoint{0.916713in}{0.681783in}}%
\pgfpathcurveto{\pgfqpoint{0.924417in}{0.674079in}}{\pgfqpoint{0.934868in}{0.669750in}}{\pgfqpoint{0.945763in}{0.669750in}}%
\pgfpathlineto{\pgfqpoint{0.945763in}{0.669750in}}%
\pgfpathclose%
\pgfusepath{stroke}%
\end{pgfscope}%
\begin{pgfscope}%
\pgfpathrectangle{\pgfqpoint{0.688192in}{0.670138in}}{\pgfqpoint{7.111808in}{5.061530in}}%
\pgfusepath{clip}%
\pgfsetbuttcap%
\pgfsetroundjoin%
\pgfsetlinewidth{1.003750pt}%
\definecolor{currentstroke}{rgb}{0.000000,0.000000,0.000000}%
\pgfsetstrokecolor{currentstroke}%
\pgfsetdash{}{0pt}%
\pgfpathmoveto{\pgfqpoint{1.469414in}{0.642878in}}%
\pgfpathcurveto{\pgfqpoint{1.480310in}{0.642878in}}{\pgfqpoint{1.490760in}{0.647207in}}{\pgfqpoint{1.498465in}{0.654911in}}%
\pgfpathcurveto{\pgfqpoint{1.506169in}{0.662615in}}{\pgfqpoint{1.510498in}{0.673066in}}{\pgfqpoint{1.510498in}{0.683962in}}%
\pgfpathcurveto{\pgfqpoint{1.510498in}{0.694857in}}{\pgfqpoint{1.506169in}{0.705308in}}{\pgfqpoint{1.498465in}{0.713012in}}%
\pgfpathcurveto{\pgfqpoint{1.490760in}{0.720717in}}{\pgfqpoint{1.480310in}{0.725046in}}{\pgfqpoint{1.469414in}{0.725046in}}%
\pgfpathcurveto{\pgfqpoint{1.458519in}{0.725046in}}{\pgfqpoint{1.448068in}{0.720717in}}{\pgfqpoint{1.440363in}{0.713012in}}%
\pgfpathcurveto{\pgfqpoint{1.432659in}{0.705308in}}{\pgfqpoint{1.428330in}{0.694857in}}{\pgfqpoint{1.428330in}{0.683962in}}%
\pgfpathcurveto{\pgfqpoint{1.428330in}{0.673066in}}{\pgfqpoint{1.432659in}{0.662615in}}{\pgfqpoint{1.440363in}{0.654911in}}%
\pgfpathcurveto{\pgfqpoint{1.448068in}{0.647207in}}{\pgfqpoint{1.458519in}{0.642878in}}{\pgfqpoint{1.469414in}{0.642878in}}%
\pgfusepath{stroke}%
\end{pgfscope}%
\begin{pgfscope}%
\pgfpathrectangle{\pgfqpoint{0.688192in}{0.670138in}}{\pgfqpoint{7.111808in}{5.061530in}}%
\pgfusepath{clip}%
\pgfsetbuttcap%
\pgfsetroundjoin%
\pgfsetlinewidth{1.003750pt}%
\definecolor{currentstroke}{rgb}{0.000000,0.000000,0.000000}%
\pgfsetstrokecolor{currentstroke}%
\pgfsetdash{}{0pt}%
\pgfpathmoveto{\pgfqpoint{2.439145in}{4.149268in}}%
\pgfpathcurveto{\pgfqpoint{2.450040in}{4.149268in}}{\pgfqpoint{2.460491in}{4.153597in}}{\pgfqpoint{2.468195in}{4.161301in}}%
\pgfpathcurveto{\pgfqpoint{2.475900in}{4.169006in}}{\pgfqpoint{2.480229in}{4.179456in}}{\pgfqpoint{2.480229in}{4.190352in}}%
\pgfpathcurveto{\pgfqpoint{2.480229in}{4.201248in}}{\pgfqpoint{2.475900in}{4.211698in}}{\pgfqpoint{2.468195in}{4.219403in}}%
\pgfpathcurveto{\pgfqpoint{2.460491in}{4.227107in}}{\pgfqpoint{2.450040in}{4.231436in}}{\pgfqpoint{2.439145in}{4.231436in}}%
\pgfpathcurveto{\pgfqpoint{2.428249in}{4.231436in}}{\pgfqpoint{2.417798in}{4.227107in}}{\pgfqpoint{2.410094in}{4.219403in}}%
\pgfpathcurveto{\pgfqpoint{2.402390in}{4.211698in}}{\pgfqpoint{2.398061in}{4.201248in}}{\pgfqpoint{2.398061in}{4.190352in}}%
\pgfpathcurveto{\pgfqpoint{2.398061in}{4.179456in}}{\pgfqpoint{2.402390in}{4.169006in}}{\pgfqpoint{2.410094in}{4.161301in}}%
\pgfpathcurveto{\pgfqpoint{2.417798in}{4.153597in}}{\pgfqpoint{2.428249in}{4.149268in}}{\pgfqpoint{2.439145in}{4.149268in}}%
\pgfpathlineto{\pgfqpoint{2.439145in}{4.149268in}}%
\pgfpathclose%
\pgfusepath{stroke}%
\end{pgfscope}%
\begin{pgfscope}%
\pgfpathrectangle{\pgfqpoint{0.688192in}{0.670138in}}{\pgfqpoint{7.111808in}{5.061530in}}%
\pgfusepath{clip}%
\pgfsetbuttcap%
\pgfsetroundjoin%
\pgfsetlinewidth{1.003750pt}%
\definecolor{currentstroke}{rgb}{0.000000,0.000000,0.000000}%
\pgfsetstrokecolor{currentstroke}%
\pgfsetdash{}{0pt}%
\pgfpathmoveto{\pgfqpoint{2.500957in}{2.186214in}}%
\pgfpathcurveto{\pgfqpoint{2.511852in}{2.186214in}}{\pgfqpoint{2.522303in}{2.190543in}}{\pgfqpoint{2.530007in}{2.198247in}}%
\pgfpathcurveto{\pgfqpoint{2.537712in}{2.205951in}}{\pgfqpoint{2.542041in}{2.216402in}}{\pgfqpoint{2.542041in}{2.227298in}}%
\pgfpathcurveto{\pgfqpoint{2.542041in}{2.238193in}}{\pgfqpoint{2.537712in}{2.248644in}}{\pgfqpoint{2.530007in}{2.256349in}}%
\pgfpathcurveto{\pgfqpoint{2.522303in}{2.264053in}}{\pgfqpoint{2.511852in}{2.268382in}}{\pgfqpoint{2.500957in}{2.268382in}}%
\pgfpathcurveto{\pgfqpoint{2.490061in}{2.268382in}}{\pgfqpoint{2.479610in}{2.264053in}}{\pgfqpoint{2.471906in}{2.256349in}}%
\pgfpathcurveto{\pgfqpoint{2.464202in}{2.248644in}}{\pgfqpoint{2.459873in}{2.238193in}}{\pgfqpoint{2.459873in}{2.227298in}}%
\pgfpathcurveto{\pgfqpoint{2.459873in}{2.216402in}}{\pgfqpoint{2.464202in}{2.205951in}}{\pgfqpoint{2.471906in}{2.198247in}}%
\pgfpathcurveto{\pgfqpoint{2.479610in}{2.190543in}}{\pgfqpoint{2.490061in}{2.186214in}}{\pgfqpoint{2.500957in}{2.186214in}}%
\pgfpathlineto{\pgfqpoint{2.500957in}{2.186214in}}%
\pgfpathclose%
\pgfusepath{stroke}%
\end{pgfscope}%
\begin{pgfscope}%
\pgfpathrectangle{\pgfqpoint{0.688192in}{0.670138in}}{\pgfqpoint{7.111808in}{5.061530in}}%
\pgfusepath{clip}%
\pgfsetbuttcap%
\pgfsetroundjoin%
\pgfsetlinewidth{1.003750pt}%
\definecolor{currentstroke}{rgb}{0.000000,0.000000,0.000000}%
\pgfsetstrokecolor{currentstroke}%
\pgfsetdash{}{0pt}%
\pgfpathmoveto{\pgfqpoint{4.965126in}{0.630283in}}%
\pgfpathcurveto{\pgfqpoint{4.976022in}{0.630283in}}{\pgfqpoint{4.986472in}{0.634611in}}{\pgfqpoint{4.994177in}{0.642316in}}%
\pgfpathcurveto{\pgfqpoint{5.001881in}{0.650020in}}{\pgfqpoint{5.006210in}{0.660471in}}{\pgfqpoint{5.006210in}{0.671366in}}%
\pgfpathcurveto{\pgfqpoint{5.006210in}{0.682262in}}{\pgfqpoint{5.001881in}{0.692713in}}{\pgfqpoint{4.994177in}{0.700417in}}%
\pgfpathcurveto{\pgfqpoint{4.986472in}{0.708122in}}{\pgfqpoint{4.976022in}{0.712450in}}{\pgfqpoint{4.965126in}{0.712450in}}%
\pgfpathcurveto{\pgfqpoint{4.954230in}{0.712450in}}{\pgfqpoint{4.943780in}{0.708122in}}{\pgfqpoint{4.936075in}{0.700417in}}%
\pgfpathcurveto{\pgfqpoint{4.928371in}{0.692713in}}{\pgfqpoint{4.924042in}{0.682262in}}{\pgfqpoint{4.924042in}{0.671366in}}%
\pgfpathcurveto{\pgfqpoint{4.924042in}{0.660471in}}{\pgfqpoint{4.928371in}{0.650020in}}{\pgfqpoint{4.936075in}{0.642316in}}%
\pgfpathcurveto{\pgfqpoint{4.943780in}{0.634611in}}{\pgfqpoint{4.954230in}{0.630283in}}{\pgfqpoint{4.965126in}{0.630283in}}%
\pgfusepath{stroke}%
\end{pgfscope}%
\begin{pgfscope}%
\pgfpathrectangle{\pgfqpoint{0.688192in}{0.670138in}}{\pgfqpoint{7.111808in}{5.061530in}}%
\pgfusepath{clip}%
\pgfsetbuttcap%
\pgfsetroundjoin%
\pgfsetlinewidth{1.003750pt}%
\definecolor{currentstroke}{rgb}{0.000000,0.000000,0.000000}%
\pgfsetstrokecolor{currentstroke}%
\pgfsetdash{}{0pt}%
\pgfpathmoveto{\pgfqpoint{2.158002in}{3.347740in}}%
\pgfpathcurveto{\pgfqpoint{2.168897in}{3.347740in}}{\pgfqpoint{2.179348in}{3.352069in}}{\pgfqpoint{2.187052in}{3.359773in}}%
\pgfpathcurveto{\pgfqpoint{2.194757in}{3.367478in}}{\pgfqpoint{2.199086in}{3.377928in}}{\pgfqpoint{2.199086in}{3.388824in}}%
\pgfpathcurveto{\pgfqpoint{2.199086in}{3.399720in}}{\pgfqpoint{2.194757in}{3.410170in}}{\pgfqpoint{2.187052in}{3.417875in}}%
\pgfpathcurveto{\pgfqpoint{2.179348in}{3.425579in}}{\pgfqpoint{2.168897in}{3.429908in}}{\pgfqpoint{2.158002in}{3.429908in}}%
\pgfpathcurveto{\pgfqpoint{2.147106in}{3.429908in}}{\pgfqpoint{2.136655in}{3.425579in}}{\pgfqpoint{2.128951in}{3.417875in}}%
\pgfpathcurveto{\pgfqpoint{2.121247in}{3.410170in}}{\pgfqpoint{2.116918in}{3.399720in}}{\pgfqpoint{2.116918in}{3.388824in}}%
\pgfpathcurveto{\pgfqpoint{2.116918in}{3.377928in}}{\pgfqpoint{2.121247in}{3.367478in}}{\pgfqpoint{2.128951in}{3.359773in}}%
\pgfpathcurveto{\pgfqpoint{2.136655in}{3.352069in}}{\pgfqpoint{2.147106in}{3.347740in}}{\pgfqpoint{2.158002in}{3.347740in}}%
\pgfpathlineto{\pgfqpoint{2.158002in}{3.347740in}}%
\pgfpathclose%
\pgfusepath{stroke}%
\end{pgfscope}%
\begin{pgfscope}%
\pgfpathrectangle{\pgfqpoint{0.688192in}{0.670138in}}{\pgfqpoint{7.111808in}{5.061530in}}%
\pgfusepath{clip}%
\pgfsetbuttcap%
\pgfsetroundjoin%
\pgfsetlinewidth{1.003750pt}%
\definecolor{currentstroke}{rgb}{0.000000,0.000000,0.000000}%
\pgfsetstrokecolor{currentstroke}%
\pgfsetdash{}{0pt}%
\pgfpathmoveto{\pgfqpoint{2.512089in}{1.884528in}}%
\pgfpathcurveto{\pgfqpoint{2.522985in}{1.884528in}}{\pgfqpoint{2.533436in}{1.888857in}}{\pgfqpoint{2.541140in}{1.896561in}}%
\pgfpathcurveto{\pgfqpoint{2.548844in}{1.904265in}}{\pgfqpoint{2.553173in}{1.914716in}}{\pgfqpoint{2.553173in}{1.925612in}}%
\pgfpathcurveto{\pgfqpoint{2.553173in}{1.936507in}}{\pgfqpoint{2.548844in}{1.946958in}}{\pgfqpoint{2.541140in}{1.954662in}}%
\pgfpathcurveto{\pgfqpoint{2.533436in}{1.962367in}}{\pgfqpoint{2.522985in}{1.966696in}}{\pgfqpoint{2.512089in}{1.966696in}}%
\pgfpathcurveto{\pgfqpoint{2.501194in}{1.966696in}}{\pgfqpoint{2.490743in}{1.962367in}}{\pgfqpoint{2.483039in}{1.954662in}}%
\pgfpathcurveto{\pgfqpoint{2.475334in}{1.946958in}}{\pgfqpoint{2.471005in}{1.936507in}}{\pgfqpoint{2.471005in}{1.925612in}}%
\pgfpathcurveto{\pgfqpoint{2.471005in}{1.914716in}}{\pgfqpoint{2.475334in}{1.904265in}}{\pgfqpoint{2.483039in}{1.896561in}}%
\pgfpathcurveto{\pgfqpoint{2.490743in}{1.888857in}}{\pgfqpoint{2.501194in}{1.884528in}}{\pgfqpoint{2.512089in}{1.884528in}}%
\pgfpathlineto{\pgfqpoint{2.512089in}{1.884528in}}%
\pgfpathclose%
\pgfusepath{stroke}%
\end{pgfscope}%
\begin{pgfscope}%
\pgfpathrectangle{\pgfqpoint{0.688192in}{0.670138in}}{\pgfqpoint{7.111808in}{5.061530in}}%
\pgfusepath{clip}%
\pgfsetbuttcap%
\pgfsetroundjoin%
\pgfsetlinewidth{1.003750pt}%
\definecolor{currentstroke}{rgb}{0.000000,0.000000,0.000000}%
\pgfsetstrokecolor{currentstroke}%
\pgfsetdash{}{0pt}%
\pgfpathmoveto{\pgfqpoint{0.810327in}{0.725375in}}%
\pgfpathcurveto{\pgfqpoint{0.821223in}{0.725375in}}{\pgfqpoint{0.831674in}{0.729703in}}{\pgfqpoint{0.839378in}{0.737408in}}%
\pgfpathcurveto{\pgfqpoint{0.847082in}{0.745112in}}{\pgfqpoint{0.851411in}{0.755563in}}{\pgfqpoint{0.851411in}{0.766458in}}%
\pgfpathcurveto{\pgfqpoint{0.851411in}{0.777354in}}{\pgfqpoint{0.847082in}{0.787805in}}{\pgfqpoint{0.839378in}{0.795509in}}%
\pgfpathcurveto{\pgfqpoint{0.831674in}{0.803213in}}{\pgfqpoint{0.821223in}{0.807542in}}{\pgfqpoint{0.810327in}{0.807542in}}%
\pgfpathcurveto{\pgfqpoint{0.799432in}{0.807542in}}{\pgfqpoint{0.788981in}{0.803213in}}{\pgfqpoint{0.781277in}{0.795509in}}%
\pgfpathcurveto{\pgfqpoint{0.773572in}{0.787805in}}{\pgfqpoint{0.769244in}{0.777354in}}{\pgfqpoint{0.769244in}{0.766458in}}%
\pgfpathcurveto{\pgfqpoint{0.769244in}{0.755563in}}{\pgfqpoint{0.773572in}{0.745112in}}{\pgfqpoint{0.781277in}{0.737408in}}%
\pgfpathcurveto{\pgfqpoint{0.788981in}{0.729703in}}{\pgfqpoint{0.799432in}{0.725375in}}{\pgfqpoint{0.810327in}{0.725375in}}%
\pgfpathlineto{\pgfqpoint{0.810327in}{0.725375in}}%
\pgfpathclose%
\pgfusepath{stroke}%
\end{pgfscope}%
\begin{pgfscope}%
\pgfpathrectangle{\pgfqpoint{0.688192in}{0.670138in}}{\pgfqpoint{7.111808in}{5.061530in}}%
\pgfusepath{clip}%
\pgfsetbuttcap%
\pgfsetroundjoin%
\pgfsetlinewidth{1.003750pt}%
\definecolor{currentstroke}{rgb}{0.000000,0.000000,0.000000}%
\pgfsetstrokecolor{currentstroke}%
\pgfsetdash{}{0pt}%
\pgfpathmoveto{\pgfqpoint{1.672396in}{2.153838in}}%
\pgfpathcurveto{\pgfqpoint{1.683291in}{2.153838in}}{\pgfqpoint{1.693742in}{2.158167in}}{\pgfqpoint{1.701446in}{2.165871in}}%
\pgfpathcurveto{\pgfqpoint{1.709151in}{2.173575in}}{\pgfqpoint{1.713480in}{2.184026in}}{\pgfqpoint{1.713480in}{2.194922in}}%
\pgfpathcurveto{\pgfqpoint{1.713480in}{2.205817in}}{\pgfqpoint{1.709151in}{2.216268in}}{\pgfqpoint{1.701446in}{2.223972in}}%
\pgfpathcurveto{\pgfqpoint{1.693742in}{2.231677in}}{\pgfqpoint{1.683291in}{2.236005in}}{\pgfqpoint{1.672396in}{2.236005in}}%
\pgfpathcurveto{\pgfqpoint{1.661500in}{2.236005in}}{\pgfqpoint{1.651049in}{2.231677in}}{\pgfqpoint{1.643345in}{2.223972in}}%
\pgfpathcurveto{\pgfqpoint{1.635641in}{2.216268in}}{\pgfqpoint{1.631312in}{2.205817in}}{\pgfqpoint{1.631312in}{2.194922in}}%
\pgfpathcurveto{\pgfqpoint{1.631312in}{2.184026in}}{\pgfqpoint{1.635641in}{2.173575in}}{\pgfqpoint{1.643345in}{2.165871in}}%
\pgfpathcurveto{\pgfqpoint{1.651049in}{2.158167in}}{\pgfqpoint{1.661500in}{2.153838in}}{\pgfqpoint{1.672396in}{2.153838in}}%
\pgfpathlineto{\pgfqpoint{1.672396in}{2.153838in}}%
\pgfpathclose%
\pgfusepath{stroke}%
\end{pgfscope}%
\begin{pgfscope}%
\pgfpathrectangle{\pgfqpoint{0.688192in}{0.670138in}}{\pgfqpoint{7.111808in}{5.061530in}}%
\pgfusepath{clip}%
\pgfsetbuttcap%
\pgfsetroundjoin%
\pgfsetlinewidth{1.003750pt}%
\definecolor{currentstroke}{rgb}{0.000000,0.000000,0.000000}%
\pgfsetstrokecolor{currentstroke}%
\pgfsetdash{}{0pt}%
\pgfpathmoveto{\pgfqpoint{0.854347in}{0.695577in}}%
\pgfpathcurveto{\pgfqpoint{0.865242in}{0.695577in}}{\pgfqpoint{0.875693in}{0.699906in}}{\pgfqpoint{0.883397in}{0.707611in}}%
\pgfpathcurveto{\pgfqpoint{0.891102in}{0.715315in}}{\pgfqpoint{0.895431in}{0.725766in}}{\pgfqpoint{0.895431in}{0.736661in}}%
\pgfpathcurveto{\pgfqpoint{0.895431in}{0.747557in}}{\pgfqpoint{0.891102in}{0.758008in}}{\pgfqpoint{0.883397in}{0.765712in}}%
\pgfpathcurveto{\pgfqpoint{0.875693in}{0.773416in}}{\pgfqpoint{0.865242in}{0.777745in}}{\pgfqpoint{0.854347in}{0.777745in}}%
\pgfpathcurveto{\pgfqpoint{0.843451in}{0.777745in}}{\pgfqpoint{0.833000in}{0.773416in}}{\pgfqpoint{0.825296in}{0.765712in}}%
\pgfpathcurveto{\pgfqpoint{0.817592in}{0.758008in}}{\pgfqpoint{0.813263in}{0.747557in}}{\pgfqpoint{0.813263in}{0.736661in}}%
\pgfpathcurveto{\pgfqpoint{0.813263in}{0.725766in}}{\pgfqpoint{0.817592in}{0.715315in}}{\pgfqpoint{0.825296in}{0.707611in}}%
\pgfpathcurveto{\pgfqpoint{0.833000in}{0.699906in}}{\pgfqpoint{0.843451in}{0.695577in}}{\pgfqpoint{0.854347in}{0.695577in}}%
\pgfpathlineto{\pgfqpoint{0.854347in}{0.695577in}}%
\pgfpathclose%
\pgfusepath{stroke}%
\end{pgfscope}%
\begin{pgfscope}%
\pgfpathrectangle{\pgfqpoint{0.688192in}{0.670138in}}{\pgfqpoint{7.111808in}{5.061530in}}%
\pgfusepath{clip}%
\pgfsetbuttcap%
\pgfsetroundjoin%
\pgfsetlinewidth{1.003750pt}%
\definecolor{currentstroke}{rgb}{0.000000,0.000000,0.000000}%
\pgfsetstrokecolor{currentstroke}%
\pgfsetdash{}{0pt}%
\pgfpathmoveto{\pgfqpoint{2.790066in}{0.669908in}}%
\pgfpathcurveto{\pgfqpoint{2.800961in}{0.669908in}}{\pgfqpoint{2.811412in}{0.674236in}}{\pgfqpoint{2.819116in}{0.681941in}}%
\pgfpathcurveto{\pgfqpoint{2.826821in}{0.689645in}}{\pgfqpoint{2.831149in}{0.700096in}}{\pgfqpoint{2.831149in}{0.710992in}}%
\pgfpathcurveto{\pgfqpoint{2.831149in}{0.721887in}}{\pgfqpoint{2.826821in}{0.732338in}}{\pgfqpoint{2.819116in}{0.740042in}}%
\pgfpathcurveto{\pgfqpoint{2.811412in}{0.747747in}}{\pgfqpoint{2.800961in}{0.752075in}}{\pgfqpoint{2.790066in}{0.752075in}}%
\pgfpathcurveto{\pgfqpoint{2.779170in}{0.752075in}}{\pgfqpoint{2.768719in}{0.747747in}}{\pgfqpoint{2.761015in}{0.740042in}}%
\pgfpathcurveto{\pgfqpoint{2.753311in}{0.732338in}}{\pgfqpoint{2.748982in}{0.721887in}}{\pgfqpoint{2.748982in}{0.710992in}}%
\pgfpathcurveto{\pgfqpoint{2.748982in}{0.700096in}}{\pgfqpoint{2.753311in}{0.689645in}}{\pgfqpoint{2.761015in}{0.681941in}}%
\pgfpathcurveto{\pgfqpoint{2.768719in}{0.674236in}}{\pgfqpoint{2.779170in}{0.669908in}}{\pgfqpoint{2.790066in}{0.669908in}}%
\pgfpathlineto{\pgfqpoint{2.790066in}{0.669908in}}%
\pgfpathclose%
\pgfusepath{stroke}%
\end{pgfscope}%
\begin{pgfscope}%
\pgfpathrectangle{\pgfqpoint{0.688192in}{0.670138in}}{\pgfqpoint{7.111808in}{5.061530in}}%
\pgfusepath{clip}%
\pgfsetbuttcap%
\pgfsetroundjoin%
\pgfsetlinewidth{1.003750pt}%
\definecolor{currentstroke}{rgb}{0.000000,0.000000,0.000000}%
\pgfsetstrokecolor{currentstroke}%
\pgfsetdash{}{0pt}%
\pgfpathmoveto{\pgfqpoint{5.792143in}{2.727751in}}%
\pgfpathcurveto{\pgfqpoint{5.803038in}{2.727751in}}{\pgfqpoint{5.813489in}{2.732079in}}{\pgfqpoint{5.821194in}{2.739784in}}%
\pgfpathcurveto{\pgfqpoint{5.828898in}{2.747488in}}{\pgfqpoint{5.833227in}{2.757939in}}{\pgfqpoint{5.833227in}{2.768835in}}%
\pgfpathcurveto{\pgfqpoint{5.833227in}{2.779730in}}{\pgfqpoint{5.828898in}{2.790181in}}{\pgfqpoint{5.821194in}{2.797885in}}%
\pgfpathcurveto{\pgfqpoint{5.813489in}{2.805590in}}{\pgfqpoint{5.803038in}{2.809918in}}{\pgfqpoint{5.792143in}{2.809918in}}%
\pgfpathcurveto{\pgfqpoint{5.781247in}{2.809918in}}{\pgfqpoint{5.770797in}{2.805590in}}{\pgfqpoint{5.763092in}{2.797885in}}%
\pgfpathcurveto{\pgfqpoint{5.755388in}{2.790181in}}{\pgfqpoint{5.751059in}{2.779730in}}{\pgfqpoint{5.751059in}{2.768835in}}%
\pgfpathcurveto{\pgfqpoint{5.751059in}{2.757939in}}{\pgfqpoint{5.755388in}{2.747488in}}{\pgfqpoint{5.763092in}{2.739784in}}%
\pgfpathcurveto{\pgfqpoint{5.770797in}{2.732079in}}{\pgfqpoint{5.781247in}{2.727751in}}{\pgfqpoint{5.792143in}{2.727751in}}%
\pgfpathlineto{\pgfqpoint{5.792143in}{2.727751in}}%
\pgfpathclose%
\pgfusepath{stroke}%
\end{pgfscope}%
\begin{pgfscope}%
\pgfpathrectangle{\pgfqpoint{0.688192in}{0.670138in}}{\pgfqpoint{7.111808in}{5.061530in}}%
\pgfusepath{clip}%
\pgfsetbuttcap%
\pgfsetroundjoin%
\pgfsetlinewidth{1.003750pt}%
\definecolor{currentstroke}{rgb}{0.000000,0.000000,0.000000}%
\pgfsetstrokecolor{currentstroke}%
\pgfsetdash{}{0pt}%
\pgfpathmoveto{\pgfqpoint{7.063065in}{2.756002in}}%
\pgfpathcurveto{\pgfqpoint{7.073961in}{2.756002in}}{\pgfqpoint{7.084412in}{2.760331in}}{\pgfqpoint{7.092116in}{2.768036in}}%
\pgfpathcurveto{\pgfqpoint{7.099820in}{2.775740in}}{\pgfqpoint{7.104149in}{2.786191in}}{\pgfqpoint{7.104149in}{2.797086in}}%
\pgfpathcurveto{\pgfqpoint{7.104149in}{2.807982in}}{\pgfqpoint{7.099820in}{2.818433in}}{\pgfqpoint{7.092116in}{2.826137in}}%
\pgfpathcurveto{\pgfqpoint{7.084412in}{2.833841in}}{\pgfqpoint{7.073961in}{2.838170in}}{\pgfqpoint{7.063065in}{2.838170in}}%
\pgfpathcurveto{\pgfqpoint{7.052170in}{2.838170in}}{\pgfqpoint{7.041719in}{2.833841in}}{\pgfqpoint{7.034014in}{2.826137in}}%
\pgfpathcurveto{\pgfqpoint{7.026310in}{2.818433in}}{\pgfqpoint{7.021981in}{2.807982in}}{\pgfqpoint{7.021981in}{2.797086in}}%
\pgfpathcurveto{\pgfqpoint{7.021981in}{2.786191in}}{\pgfqpoint{7.026310in}{2.775740in}}{\pgfqpoint{7.034014in}{2.768036in}}%
\pgfpathcurveto{\pgfqpoint{7.041719in}{2.760331in}}{\pgfqpoint{7.052170in}{2.756002in}}{\pgfqpoint{7.063065in}{2.756002in}}%
\pgfpathlineto{\pgfqpoint{7.063065in}{2.756002in}}%
\pgfpathclose%
\pgfusepath{stroke}%
\end{pgfscope}%
\begin{pgfscope}%
\pgfpathrectangle{\pgfqpoint{0.688192in}{0.670138in}}{\pgfqpoint{7.111808in}{5.061530in}}%
\pgfusepath{clip}%
\pgfsetbuttcap%
\pgfsetroundjoin%
\pgfsetlinewidth{1.003750pt}%
\definecolor{currentstroke}{rgb}{0.000000,0.000000,0.000000}%
\pgfsetstrokecolor{currentstroke}%
\pgfsetdash{}{0pt}%
\pgfpathmoveto{\pgfqpoint{0.996565in}{0.661960in}}%
\pgfpathcurveto{\pgfqpoint{1.007461in}{0.661960in}}{\pgfqpoint{1.017912in}{0.666289in}}{\pgfqpoint{1.025616in}{0.673993in}}%
\pgfpathcurveto{\pgfqpoint{1.033320in}{0.681697in}}{\pgfqpoint{1.037649in}{0.692148in}}{\pgfqpoint{1.037649in}{0.703044in}}%
\pgfpathcurveto{\pgfqpoint{1.037649in}{0.713939in}}{\pgfqpoint{1.033320in}{0.724390in}}{\pgfqpoint{1.025616in}{0.732095in}}%
\pgfpathcurveto{\pgfqpoint{1.017912in}{0.739799in}}{\pgfqpoint{1.007461in}{0.744128in}}{\pgfqpoint{0.996565in}{0.744128in}}%
\pgfpathcurveto{\pgfqpoint{0.985670in}{0.744128in}}{\pgfqpoint{0.975219in}{0.739799in}}{\pgfqpoint{0.967515in}{0.732095in}}%
\pgfpathcurveto{\pgfqpoint{0.959810in}{0.724390in}}{\pgfqpoint{0.955481in}{0.713939in}}{\pgfqpoint{0.955481in}{0.703044in}}%
\pgfpathcurveto{\pgfqpoint{0.955481in}{0.692148in}}{\pgfqpoint{0.959810in}{0.681697in}}{\pgfqpoint{0.967515in}{0.673993in}}%
\pgfpathcurveto{\pgfqpoint{0.975219in}{0.666289in}}{\pgfqpoint{0.985670in}{0.661960in}}{\pgfqpoint{0.996565in}{0.661960in}}%
\pgfusepath{stroke}%
\end{pgfscope}%
\begin{pgfscope}%
\pgfpathrectangle{\pgfqpoint{0.688192in}{0.670138in}}{\pgfqpoint{7.111808in}{5.061530in}}%
\pgfusepath{clip}%
\pgfsetbuttcap%
\pgfsetroundjoin%
\pgfsetlinewidth{1.003750pt}%
\definecolor{currentstroke}{rgb}{0.000000,0.000000,0.000000}%
\pgfsetstrokecolor{currentstroke}%
\pgfsetdash{}{0pt}%
\pgfpathmoveto{\pgfqpoint{4.979341in}{0.630236in}}%
\pgfpathcurveto{\pgfqpoint{4.990237in}{0.630236in}}{\pgfqpoint{5.000687in}{0.634565in}}{\pgfqpoint{5.008392in}{0.642269in}}%
\pgfpathcurveto{\pgfqpoint{5.016096in}{0.649973in}}{\pgfqpoint{5.020425in}{0.660424in}}{\pgfqpoint{5.020425in}{0.671320in}}%
\pgfpathcurveto{\pgfqpoint{5.020425in}{0.682215in}}{\pgfqpoint{5.016096in}{0.692666in}}{\pgfqpoint{5.008392in}{0.700370in}}%
\pgfpathcurveto{\pgfqpoint{5.000687in}{0.708075in}}{\pgfqpoint{4.990237in}{0.712403in}}{\pgfqpoint{4.979341in}{0.712403in}}%
\pgfpathcurveto{\pgfqpoint{4.968446in}{0.712403in}}{\pgfqpoint{4.957995in}{0.708075in}}{\pgfqpoint{4.950290in}{0.700370in}}%
\pgfpathcurveto{\pgfqpoint{4.942586in}{0.692666in}}{\pgfqpoint{4.938257in}{0.682215in}}{\pgfqpoint{4.938257in}{0.671320in}}%
\pgfpathcurveto{\pgfqpoint{4.938257in}{0.660424in}}{\pgfqpoint{4.942586in}{0.649973in}}{\pgfqpoint{4.950290in}{0.642269in}}%
\pgfpathcurveto{\pgfqpoint{4.957995in}{0.634565in}}{\pgfqpoint{4.968446in}{0.630236in}}{\pgfqpoint{4.979341in}{0.630236in}}%
\pgfusepath{stroke}%
\end{pgfscope}%
\begin{pgfscope}%
\pgfpathrectangle{\pgfqpoint{0.688192in}{0.670138in}}{\pgfqpoint{7.111808in}{5.061530in}}%
\pgfusepath{clip}%
\pgfsetbuttcap%
\pgfsetroundjoin%
\pgfsetlinewidth{1.003750pt}%
\definecolor{currentstroke}{rgb}{0.000000,0.000000,0.000000}%
\pgfsetstrokecolor{currentstroke}%
\pgfsetdash{}{0pt}%
\pgfpathmoveto{\pgfqpoint{7.370154in}{3.896576in}}%
\pgfpathcurveto{\pgfqpoint{7.381049in}{3.896576in}}{\pgfqpoint{7.391500in}{3.900905in}}{\pgfqpoint{7.399204in}{3.908609in}}%
\pgfpathcurveto{\pgfqpoint{7.406909in}{3.916314in}}{\pgfqpoint{7.411238in}{3.926764in}}{\pgfqpoint{7.411238in}{3.937660in}}%
\pgfpathcurveto{\pgfqpoint{7.411238in}{3.948555in}}{\pgfqpoint{7.406909in}{3.959006in}}{\pgfqpoint{7.399204in}{3.966711in}}%
\pgfpathcurveto{\pgfqpoint{7.391500in}{3.974415in}}{\pgfqpoint{7.381049in}{3.978744in}}{\pgfqpoint{7.370154in}{3.978744in}}%
\pgfpathcurveto{\pgfqpoint{7.359258in}{3.978744in}}{\pgfqpoint{7.348807in}{3.974415in}}{\pgfqpoint{7.341103in}{3.966711in}}%
\pgfpathcurveto{\pgfqpoint{7.333399in}{3.959006in}}{\pgfqpoint{7.329070in}{3.948555in}}{\pgfqpoint{7.329070in}{3.937660in}}%
\pgfpathcurveto{\pgfqpoint{7.329070in}{3.926764in}}{\pgfqpoint{7.333399in}{3.916314in}}{\pgfqpoint{7.341103in}{3.908609in}}%
\pgfpathcurveto{\pgfqpoint{7.348807in}{3.900905in}}{\pgfqpoint{7.359258in}{3.896576in}}{\pgfqpoint{7.370154in}{3.896576in}}%
\pgfpathlineto{\pgfqpoint{7.370154in}{3.896576in}}%
\pgfpathclose%
\pgfusepath{stroke}%
\end{pgfscope}%
\begin{pgfscope}%
\pgfpathrectangle{\pgfqpoint{0.688192in}{0.670138in}}{\pgfqpoint{7.111808in}{5.061530in}}%
\pgfusepath{clip}%
\pgfsetbuttcap%
\pgfsetroundjoin%
\pgfsetlinewidth{1.003750pt}%
\definecolor{currentstroke}{rgb}{0.000000,0.000000,0.000000}%
\pgfsetstrokecolor{currentstroke}%
\pgfsetdash{}{0pt}%
\pgfpathmoveto{\pgfqpoint{4.726657in}{0.630576in}}%
\pgfpathcurveto{\pgfqpoint{4.737552in}{0.630576in}}{\pgfqpoint{4.748003in}{0.634905in}}{\pgfqpoint{4.755707in}{0.642610in}}%
\pgfpathcurveto{\pgfqpoint{4.763412in}{0.650314in}}{\pgfqpoint{4.767741in}{0.660765in}}{\pgfqpoint{4.767741in}{0.671660in}}%
\pgfpathcurveto{\pgfqpoint{4.767741in}{0.682556in}}{\pgfqpoint{4.763412in}{0.693007in}}{\pgfqpoint{4.755707in}{0.700711in}}%
\pgfpathcurveto{\pgfqpoint{4.748003in}{0.708415in}}{\pgfqpoint{4.737552in}{0.712744in}}{\pgfqpoint{4.726657in}{0.712744in}}%
\pgfpathcurveto{\pgfqpoint{4.715761in}{0.712744in}}{\pgfqpoint{4.705310in}{0.708415in}}{\pgfqpoint{4.697606in}{0.700711in}}%
\pgfpathcurveto{\pgfqpoint{4.689902in}{0.693007in}}{\pgfqpoint{4.685573in}{0.682556in}}{\pgfqpoint{4.685573in}{0.671660in}}%
\pgfpathcurveto{\pgfqpoint{4.685573in}{0.660765in}}{\pgfqpoint{4.689902in}{0.650314in}}{\pgfqpoint{4.697606in}{0.642610in}}%
\pgfpathcurveto{\pgfqpoint{4.705310in}{0.634905in}}{\pgfqpoint{4.715761in}{0.630576in}}{\pgfqpoint{4.726657in}{0.630576in}}%
\pgfusepath{stroke}%
\end{pgfscope}%
\begin{pgfscope}%
\pgfpathrectangle{\pgfqpoint{0.688192in}{0.670138in}}{\pgfqpoint{7.111808in}{5.061530in}}%
\pgfusepath{clip}%
\pgfsetbuttcap%
\pgfsetroundjoin%
\pgfsetlinewidth{1.003750pt}%
\definecolor{currentstroke}{rgb}{0.000000,0.000000,0.000000}%
\pgfsetstrokecolor{currentstroke}%
\pgfsetdash{}{0pt}%
\pgfpathmoveto{\pgfqpoint{2.474846in}{3.185560in}}%
\pgfpathcurveto{\pgfqpoint{2.485742in}{3.185560in}}{\pgfqpoint{2.496193in}{3.189889in}}{\pgfqpoint{2.503897in}{3.197593in}}%
\pgfpathcurveto{\pgfqpoint{2.511601in}{3.205298in}}{\pgfqpoint{2.515930in}{3.215748in}}{\pgfqpoint{2.515930in}{3.226644in}}%
\pgfpathcurveto{\pgfqpoint{2.515930in}{3.237539in}}{\pgfqpoint{2.511601in}{3.247990in}}{\pgfqpoint{2.503897in}{3.255695in}}%
\pgfpathcurveto{\pgfqpoint{2.496193in}{3.263399in}}{\pgfqpoint{2.485742in}{3.267728in}}{\pgfqpoint{2.474846in}{3.267728in}}%
\pgfpathcurveto{\pgfqpoint{2.463951in}{3.267728in}}{\pgfqpoint{2.453500in}{3.263399in}}{\pgfqpoint{2.445796in}{3.255695in}}%
\pgfpathcurveto{\pgfqpoint{2.438091in}{3.247990in}}{\pgfqpoint{2.433762in}{3.237539in}}{\pgfqpoint{2.433762in}{3.226644in}}%
\pgfpathcurveto{\pgfqpoint{2.433762in}{3.215748in}}{\pgfqpoint{2.438091in}{3.205298in}}{\pgfqpoint{2.445796in}{3.197593in}}%
\pgfpathcurveto{\pgfqpoint{2.453500in}{3.189889in}}{\pgfqpoint{2.463951in}{3.185560in}}{\pgfqpoint{2.474846in}{3.185560in}}%
\pgfpathlineto{\pgfqpoint{2.474846in}{3.185560in}}%
\pgfpathclose%
\pgfusepath{stroke}%
\end{pgfscope}%
\begin{pgfscope}%
\pgfpathrectangle{\pgfqpoint{0.688192in}{0.670138in}}{\pgfqpoint{7.111808in}{5.061530in}}%
\pgfusepath{clip}%
\pgfsetbuttcap%
\pgfsetroundjoin%
\pgfsetlinewidth{1.003750pt}%
\definecolor{currentstroke}{rgb}{0.000000,0.000000,0.000000}%
\pgfsetstrokecolor{currentstroke}%
\pgfsetdash{}{0pt}%
\pgfpathmoveto{\pgfqpoint{2.911373in}{0.635871in}}%
\pgfpathcurveto{\pgfqpoint{2.922269in}{0.635871in}}{\pgfqpoint{2.932719in}{0.640200in}}{\pgfqpoint{2.940424in}{0.647905in}}%
\pgfpathcurveto{\pgfqpoint{2.948128in}{0.655609in}}{\pgfqpoint{2.952457in}{0.666060in}}{\pgfqpoint{2.952457in}{0.676955in}}%
\pgfpathcurveto{\pgfqpoint{2.952457in}{0.687851in}}{\pgfqpoint{2.948128in}{0.698302in}}{\pgfqpoint{2.940424in}{0.706006in}}%
\pgfpathcurveto{\pgfqpoint{2.932719in}{0.713710in}}{\pgfqpoint{2.922269in}{0.718039in}}{\pgfqpoint{2.911373in}{0.718039in}}%
\pgfpathcurveto{\pgfqpoint{2.900477in}{0.718039in}}{\pgfqpoint{2.890027in}{0.713710in}}{\pgfqpoint{2.882322in}{0.706006in}}%
\pgfpathcurveto{\pgfqpoint{2.874618in}{0.698302in}}{\pgfqpoint{2.870289in}{0.687851in}}{\pgfqpoint{2.870289in}{0.676955in}}%
\pgfpathcurveto{\pgfqpoint{2.870289in}{0.666060in}}{\pgfqpoint{2.874618in}{0.655609in}}{\pgfqpoint{2.882322in}{0.647905in}}%
\pgfpathcurveto{\pgfqpoint{2.890027in}{0.640200in}}{\pgfqpoint{2.900477in}{0.635871in}}{\pgfqpoint{2.911373in}{0.635871in}}%
\pgfusepath{stroke}%
\end{pgfscope}%
\begin{pgfscope}%
\pgfpathrectangle{\pgfqpoint{0.688192in}{0.670138in}}{\pgfqpoint{7.111808in}{5.061530in}}%
\pgfusepath{clip}%
\pgfsetbuttcap%
\pgfsetroundjoin%
\pgfsetlinewidth{1.003750pt}%
\definecolor{currentstroke}{rgb}{0.000000,0.000000,0.000000}%
\pgfsetstrokecolor{currentstroke}%
\pgfsetdash{}{0pt}%
\pgfpathmoveto{\pgfqpoint{1.075000in}{0.651535in}}%
\pgfpathcurveto{\pgfqpoint{1.085896in}{0.651535in}}{\pgfqpoint{1.096346in}{0.655863in}}{\pgfqpoint{1.104051in}{0.663568in}}%
\pgfpathcurveto{\pgfqpoint{1.111755in}{0.671272in}}{\pgfqpoint{1.116084in}{0.681723in}}{\pgfqpoint{1.116084in}{0.692618in}}%
\pgfpathcurveto{\pgfqpoint{1.116084in}{0.703514in}}{\pgfqpoint{1.111755in}{0.713965in}}{\pgfqpoint{1.104051in}{0.721669in}}%
\pgfpathcurveto{\pgfqpoint{1.096346in}{0.729373in}}{\pgfqpoint{1.085896in}{0.733702in}}{\pgfqpoint{1.075000in}{0.733702in}}%
\pgfpathcurveto{\pgfqpoint{1.064105in}{0.733702in}}{\pgfqpoint{1.053654in}{0.729373in}}{\pgfqpoint{1.045949in}{0.721669in}}%
\pgfpathcurveto{\pgfqpoint{1.038245in}{0.713965in}}{\pgfqpoint{1.033916in}{0.703514in}}{\pgfqpoint{1.033916in}{0.692618in}}%
\pgfpathcurveto{\pgfqpoint{1.033916in}{0.681723in}}{\pgfqpoint{1.038245in}{0.671272in}}{\pgfqpoint{1.045949in}{0.663568in}}%
\pgfpathcurveto{\pgfqpoint{1.053654in}{0.655863in}}{\pgfqpoint{1.064105in}{0.651535in}}{\pgfqpoint{1.075000in}{0.651535in}}%
\pgfusepath{stroke}%
\end{pgfscope}%
\begin{pgfscope}%
\pgfpathrectangle{\pgfqpoint{0.688192in}{0.670138in}}{\pgfqpoint{7.111808in}{5.061530in}}%
\pgfusepath{clip}%
\pgfsetbuttcap%
\pgfsetroundjoin%
\pgfsetlinewidth{1.003750pt}%
\definecolor{currentstroke}{rgb}{0.000000,0.000000,0.000000}%
\pgfsetstrokecolor{currentstroke}%
\pgfsetdash{}{0pt}%
\pgfpathmoveto{\pgfqpoint{0.940682in}{0.674925in}}%
\pgfpathcurveto{\pgfqpoint{0.951577in}{0.674925in}}{\pgfqpoint{0.962028in}{0.679254in}}{\pgfqpoint{0.969732in}{0.686959in}}%
\pgfpathcurveto{\pgfqpoint{0.977437in}{0.694663in}}{\pgfqpoint{0.981766in}{0.705114in}}{\pgfqpoint{0.981766in}{0.716009in}}%
\pgfpathcurveto{\pgfqpoint{0.981766in}{0.726905in}}{\pgfqpoint{0.977437in}{0.737356in}}{\pgfqpoint{0.969732in}{0.745060in}}%
\pgfpathcurveto{\pgfqpoint{0.962028in}{0.752764in}}{\pgfqpoint{0.951577in}{0.757093in}}{\pgfqpoint{0.940682in}{0.757093in}}%
\pgfpathcurveto{\pgfqpoint{0.929786in}{0.757093in}}{\pgfqpoint{0.919335in}{0.752764in}}{\pgfqpoint{0.911631in}{0.745060in}}%
\pgfpathcurveto{\pgfqpoint{0.903927in}{0.737356in}}{\pgfqpoint{0.899598in}{0.726905in}}{\pgfqpoint{0.899598in}{0.716009in}}%
\pgfpathcurveto{\pgfqpoint{0.899598in}{0.705114in}}{\pgfqpoint{0.903927in}{0.694663in}}{\pgfqpoint{0.911631in}{0.686959in}}%
\pgfpathcurveto{\pgfqpoint{0.919335in}{0.679254in}}{\pgfqpoint{0.929786in}{0.674925in}}{\pgfqpoint{0.940682in}{0.674925in}}%
\pgfpathlineto{\pgfqpoint{0.940682in}{0.674925in}}%
\pgfpathclose%
\pgfusepath{stroke}%
\end{pgfscope}%
\begin{pgfscope}%
\pgfpathrectangle{\pgfqpoint{0.688192in}{0.670138in}}{\pgfqpoint{7.111808in}{5.061530in}}%
\pgfusepath{clip}%
\pgfsetbuttcap%
\pgfsetroundjoin%
\pgfsetlinewidth{1.003750pt}%
\definecolor{currentstroke}{rgb}{0.000000,0.000000,0.000000}%
\pgfsetstrokecolor{currentstroke}%
\pgfsetdash{}{0pt}%
\pgfpathmoveto{\pgfqpoint{2.768315in}{3.990841in}}%
\pgfpathcurveto{\pgfqpoint{2.779211in}{3.990841in}}{\pgfqpoint{2.789661in}{3.995170in}}{\pgfqpoint{2.797366in}{4.002874in}}%
\pgfpathcurveto{\pgfqpoint{2.805070in}{4.010579in}}{\pgfqpoint{2.809399in}{4.021029in}}{\pgfqpoint{2.809399in}{4.031925in}}%
\pgfpathcurveto{\pgfqpoint{2.809399in}{4.042821in}}{\pgfqpoint{2.805070in}{4.053271in}}{\pgfqpoint{2.797366in}{4.060976in}}%
\pgfpathcurveto{\pgfqpoint{2.789661in}{4.068680in}}{\pgfqpoint{2.779211in}{4.073009in}}{\pgfqpoint{2.768315in}{4.073009in}}%
\pgfpathcurveto{\pgfqpoint{2.757419in}{4.073009in}}{\pgfqpoint{2.746969in}{4.068680in}}{\pgfqpoint{2.739264in}{4.060976in}}%
\pgfpathcurveto{\pgfqpoint{2.731560in}{4.053271in}}{\pgfqpoint{2.727231in}{4.042821in}}{\pgfqpoint{2.727231in}{4.031925in}}%
\pgfpathcurveto{\pgfqpoint{2.727231in}{4.021029in}}{\pgfqpoint{2.731560in}{4.010579in}}{\pgfqpoint{2.739264in}{4.002874in}}%
\pgfpathcurveto{\pgfqpoint{2.746969in}{3.995170in}}{\pgfqpoint{2.757419in}{3.990841in}}{\pgfqpoint{2.768315in}{3.990841in}}%
\pgfpathlineto{\pgfqpoint{2.768315in}{3.990841in}}%
\pgfpathclose%
\pgfusepath{stroke}%
\end{pgfscope}%
\begin{pgfscope}%
\pgfpathrectangle{\pgfqpoint{0.688192in}{0.670138in}}{\pgfqpoint{7.111808in}{5.061530in}}%
\pgfusepath{clip}%
\pgfsetbuttcap%
\pgfsetroundjoin%
\pgfsetlinewidth{1.003750pt}%
\definecolor{currentstroke}{rgb}{0.000000,0.000000,0.000000}%
\pgfsetstrokecolor{currentstroke}%
\pgfsetdash{}{0pt}%
\pgfpathmoveto{\pgfqpoint{5.820103in}{3.024991in}}%
\pgfpathcurveto{\pgfqpoint{5.830998in}{3.024991in}}{\pgfqpoint{5.841449in}{3.029320in}}{\pgfqpoint{5.849153in}{3.037025in}}%
\pgfpathcurveto{\pgfqpoint{5.856858in}{3.044729in}}{\pgfqpoint{5.861187in}{3.055180in}}{\pgfqpoint{5.861187in}{3.066075in}}%
\pgfpathcurveto{\pgfqpoint{5.861187in}{3.076971in}}{\pgfqpoint{5.856858in}{3.087422in}}{\pgfqpoint{5.849153in}{3.095126in}}%
\pgfpathcurveto{\pgfqpoint{5.841449in}{3.102830in}}{\pgfqpoint{5.830998in}{3.107159in}}{\pgfqpoint{5.820103in}{3.107159in}}%
\pgfpathcurveto{\pgfqpoint{5.809207in}{3.107159in}}{\pgfqpoint{5.798756in}{3.102830in}}{\pgfqpoint{5.791052in}{3.095126in}}%
\pgfpathcurveto{\pgfqpoint{5.783348in}{3.087422in}}{\pgfqpoint{5.779019in}{3.076971in}}{\pgfqpoint{5.779019in}{3.066075in}}%
\pgfpathcurveto{\pgfqpoint{5.779019in}{3.055180in}}{\pgfqpoint{5.783348in}{3.044729in}}{\pgfqpoint{5.791052in}{3.037025in}}%
\pgfpathcurveto{\pgfqpoint{5.798756in}{3.029320in}}{\pgfqpoint{5.809207in}{3.024991in}}{\pgfqpoint{5.820103in}{3.024991in}}%
\pgfpathlineto{\pgfqpoint{5.820103in}{3.024991in}}%
\pgfpathclose%
\pgfusepath{stroke}%
\end{pgfscope}%
\begin{pgfscope}%
\pgfpathrectangle{\pgfqpoint{0.688192in}{0.670138in}}{\pgfqpoint{7.111808in}{5.061530in}}%
\pgfusepath{clip}%
\pgfsetbuttcap%
\pgfsetroundjoin%
\pgfsetlinewidth{1.003750pt}%
\definecolor{currentstroke}{rgb}{0.000000,0.000000,0.000000}%
\pgfsetstrokecolor{currentstroke}%
\pgfsetdash{}{0pt}%
\pgfpathmoveto{\pgfqpoint{2.225009in}{0.638936in}}%
\pgfpathcurveto{\pgfqpoint{2.235905in}{0.638936in}}{\pgfqpoint{2.246356in}{0.643265in}}{\pgfqpoint{2.254060in}{0.650969in}}%
\pgfpathcurveto{\pgfqpoint{2.261764in}{0.658674in}}{\pgfqpoint{2.266093in}{0.669125in}}{\pgfqpoint{2.266093in}{0.680020in}}%
\pgfpathcurveto{\pgfqpoint{2.266093in}{0.690916in}}{\pgfqpoint{2.261764in}{0.701366in}}{\pgfqpoint{2.254060in}{0.709071in}}%
\pgfpathcurveto{\pgfqpoint{2.246356in}{0.716775in}}{\pgfqpoint{2.235905in}{0.721104in}}{\pgfqpoint{2.225009in}{0.721104in}}%
\pgfpathcurveto{\pgfqpoint{2.214114in}{0.721104in}}{\pgfqpoint{2.203663in}{0.716775in}}{\pgfqpoint{2.195959in}{0.709071in}}%
\pgfpathcurveto{\pgfqpoint{2.188254in}{0.701366in}}{\pgfqpoint{2.183926in}{0.690916in}}{\pgfqpoint{2.183926in}{0.680020in}}%
\pgfpathcurveto{\pgfqpoint{2.183926in}{0.669125in}}{\pgfqpoint{2.188254in}{0.658674in}}{\pgfqpoint{2.195959in}{0.650969in}}%
\pgfpathcurveto{\pgfqpoint{2.203663in}{0.643265in}}{\pgfqpoint{2.214114in}{0.638936in}}{\pgfqpoint{2.225009in}{0.638936in}}%
\pgfusepath{stroke}%
\end{pgfscope}%
\begin{pgfscope}%
\pgfpathrectangle{\pgfqpoint{0.688192in}{0.670138in}}{\pgfqpoint{7.111808in}{5.061530in}}%
\pgfusepath{clip}%
\pgfsetbuttcap%
\pgfsetroundjoin%
\pgfsetlinewidth{1.003750pt}%
\definecolor{currentstroke}{rgb}{0.000000,0.000000,0.000000}%
\pgfsetstrokecolor{currentstroke}%
\pgfsetdash{}{0pt}%
\pgfpathmoveto{\pgfqpoint{4.572123in}{1.653121in}}%
\pgfpathcurveto{\pgfqpoint{4.583018in}{1.653121in}}{\pgfqpoint{4.593469in}{1.657450in}}{\pgfqpoint{4.601173in}{1.665154in}}%
\pgfpathcurveto{\pgfqpoint{4.608878in}{1.672858in}}{\pgfqpoint{4.613207in}{1.683309in}}{\pgfqpoint{4.613207in}{1.694205in}}%
\pgfpathcurveto{\pgfqpoint{4.613207in}{1.705100in}}{\pgfqpoint{4.608878in}{1.715551in}}{\pgfqpoint{4.601173in}{1.723255in}}%
\pgfpathcurveto{\pgfqpoint{4.593469in}{1.730960in}}{\pgfqpoint{4.583018in}{1.735288in}}{\pgfqpoint{4.572123in}{1.735288in}}%
\pgfpathcurveto{\pgfqpoint{4.561227in}{1.735288in}}{\pgfqpoint{4.550776in}{1.730960in}}{\pgfqpoint{4.543072in}{1.723255in}}%
\pgfpathcurveto{\pgfqpoint{4.535368in}{1.715551in}}{\pgfqpoint{4.531039in}{1.705100in}}{\pgfqpoint{4.531039in}{1.694205in}}%
\pgfpathcurveto{\pgfqpoint{4.531039in}{1.683309in}}{\pgfqpoint{4.535368in}{1.672858in}}{\pgfqpoint{4.543072in}{1.665154in}}%
\pgfpathcurveto{\pgfqpoint{4.550776in}{1.657450in}}{\pgfqpoint{4.561227in}{1.653121in}}{\pgfqpoint{4.572123in}{1.653121in}}%
\pgfpathlineto{\pgfqpoint{4.572123in}{1.653121in}}%
\pgfpathclose%
\pgfusepath{stroke}%
\end{pgfscope}%
\begin{pgfscope}%
\pgfpathrectangle{\pgfqpoint{0.688192in}{0.670138in}}{\pgfqpoint{7.111808in}{5.061530in}}%
\pgfusepath{clip}%
\pgfsetbuttcap%
\pgfsetroundjoin%
\pgfsetlinewidth{1.003750pt}%
\definecolor{currentstroke}{rgb}{0.000000,0.000000,0.000000}%
\pgfsetstrokecolor{currentstroke}%
\pgfsetdash{}{0pt}%
\pgfpathmoveto{\pgfqpoint{5.370147in}{1.305328in}}%
\pgfpathcurveto{\pgfqpoint{5.381042in}{1.305328in}}{\pgfqpoint{5.391493in}{1.309656in}}{\pgfqpoint{5.399198in}{1.317361in}}%
\pgfpathcurveto{\pgfqpoint{5.406902in}{1.325065in}}{\pgfqpoint{5.411231in}{1.335516in}}{\pgfqpoint{5.411231in}{1.346411in}}%
\pgfpathcurveto{\pgfqpoint{5.411231in}{1.357307in}}{\pgfqpoint{5.406902in}{1.367758in}}{\pgfqpoint{5.399198in}{1.375462in}}%
\pgfpathcurveto{\pgfqpoint{5.391493in}{1.383166in}}{\pgfqpoint{5.381042in}{1.387495in}}{\pgfqpoint{5.370147in}{1.387495in}}%
\pgfpathcurveto{\pgfqpoint{5.359251in}{1.387495in}}{\pgfqpoint{5.348801in}{1.383166in}}{\pgfqpoint{5.341096in}{1.375462in}}%
\pgfpathcurveto{\pgfqpoint{5.333392in}{1.367758in}}{\pgfqpoint{5.329063in}{1.357307in}}{\pgfqpoint{5.329063in}{1.346411in}}%
\pgfpathcurveto{\pgfqpoint{5.329063in}{1.335516in}}{\pgfqpoint{5.333392in}{1.325065in}}{\pgfqpoint{5.341096in}{1.317361in}}%
\pgfpathcurveto{\pgfqpoint{5.348801in}{1.309656in}}{\pgfqpoint{5.359251in}{1.305328in}}{\pgfqpoint{5.370147in}{1.305328in}}%
\pgfpathlineto{\pgfqpoint{5.370147in}{1.305328in}}%
\pgfpathclose%
\pgfusepath{stroke}%
\end{pgfscope}%
\begin{pgfscope}%
\pgfpathrectangle{\pgfqpoint{0.688192in}{0.670138in}}{\pgfqpoint{7.111808in}{5.061530in}}%
\pgfusepath{clip}%
\pgfsetbuttcap%
\pgfsetroundjoin%
\pgfsetlinewidth{1.003750pt}%
\definecolor{currentstroke}{rgb}{0.000000,0.000000,0.000000}%
\pgfsetstrokecolor{currentstroke}%
\pgfsetdash{}{0pt}%
\pgfpathmoveto{\pgfqpoint{1.903233in}{0.640199in}}%
\pgfpathcurveto{\pgfqpoint{1.914129in}{0.640199in}}{\pgfqpoint{1.924579in}{0.644527in}}{\pgfqpoint{1.932284in}{0.652232in}}%
\pgfpathcurveto{\pgfqpoint{1.939988in}{0.659936in}}{\pgfqpoint{1.944317in}{0.670387in}}{\pgfqpoint{1.944317in}{0.681282in}}%
\pgfpathcurveto{\pgfqpoint{1.944317in}{0.692178in}}{\pgfqpoint{1.939988in}{0.702629in}}{\pgfqpoint{1.932284in}{0.710333in}}%
\pgfpathcurveto{\pgfqpoint{1.924579in}{0.718038in}}{\pgfqpoint{1.914129in}{0.722366in}}{\pgfqpoint{1.903233in}{0.722366in}}%
\pgfpathcurveto{\pgfqpoint{1.892337in}{0.722366in}}{\pgfqpoint{1.881887in}{0.718038in}}{\pgfqpoint{1.874182in}{0.710333in}}%
\pgfpathcurveto{\pgfqpoint{1.866478in}{0.702629in}}{\pgfqpoint{1.862149in}{0.692178in}}{\pgfqpoint{1.862149in}{0.681282in}}%
\pgfpathcurveto{\pgfqpoint{1.862149in}{0.670387in}}{\pgfqpoint{1.866478in}{0.659936in}}{\pgfqpoint{1.874182in}{0.652232in}}%
\pgfpathcurveto{\pgfqpoint{1.881887in}{0.644527in}}{\pgfqpoint{1.892337in}{0.640199in}}{\pgfqpoint{1.903233in}{0.640199in}}%
\pgfusepath{stroke}%
\end{pgfscope}%
\begin{pgfscope}%
\pgfpathrectangle{\pgfqpoint{0.688192in}{0.670138in}}{\pgfqpoint{7.111808in}{5.061530in}}%
\pgfusepath{clip}%
\pgfsetbuttcap%
\pgfsetroundjoin%
\pgfsetlinewidth{1.003750pt}%
\definecolor{currentstroke}{rgb}{0.000000,0.000000,0.000000}%
\pgfsetstrokecolor{currentstroke}%
\pgfsetdash{}{0pt}%
\pgfpathmoveto{\pgfqpoint{3.169391in}{2.270687in}}%
\pgfpathcurveto{\pgfqpoint{3.180287in}{2.270687in}}{\pgfqpoint{3.190737in}{2.275016in}}{\pgfqpoint{3.198442in}{2.282720in}}%
\pgfpathcurveto{\pgfqpoint{3.206146in}{2.290424in}}{\pgfqpoint{3.210475in}{2.300875in}}{\pgfqpoint{3.210475in}{2.311771in}}%
\pgfpathcurveto{\pgfqpoint{3.210475in}{2.322666in}}{\pgfqpoint{3.206146in}{2.333117in}}{\pgfqpoint{3.198442in}{2.340821in}}%
\pgfpathcurveto{\pgfqpoint{3.190737in}{2.348526in}}{\pgfqpoint{3.180287in}{2.352854in}}{\pgfqpoint{3.169391in}{2.352854in}}%
\pgfpathcurveto{\pgfqpoint{3.158495in}{2.352854in}}{\pgfqpoint{3.148045in}{2.348526in}}{\pgfqpoint{3.140340in}{2.340821in}}%
\pgfpathcurveto{\pgfqpoint{3.132636in}{2.333117in}}{\pgfqpoint{3.128307in}{2.322666in}}{\pgfqpoint{3.128307in}{2.311771in}}%
\pgfpathcurveto{\pgfqpoint{3.128307in}{2.300875in}}{\pgfqpoint{3.132636in}{2.290424in}}{\pgfqpoint{3.140340in}{2.282720in}}%
\pgfpathcurveto{\pgfqpoint{3.148045in}{2.275016in}}{\pgfqpoint{3.158495in}{2.270687in}}{\pgfqpoint{3.169391in}{2.270687in}}%
\pgfpathlineto{\pgfqpoint{3.169391in}{2.270687in}}%
\pgfpathclose%
\pgfusepath{stroke}%
\end{pgfscope}%
\begin{pgfscope}%
\pgfpathrectangle{\pgfqpoint{0.688192in}{0.670138in}}{\pgfqpoint{7.111808in}{5.061530in}}%
\pgfusepath{clip}%
\pgfsetbuttcap%
\pgfsetroundjoin%
\pgfsetlinewidth{1.003750pt}%
\definecolor{currentstroke}{rgb}{0.000000,0.000000,0.000000}%
\pgfsetstrokecolor{currentstroke}%
\pgfsetdash{}{0pt}%
\pgfpathmoveto{\pgfqpoint{1.407033in}{0.642986in}}%
\pgfpathcurveto{\pgfqpoint{1.417928in}{0.642986in}}{\pgfqpoint{1.428379in}{0.647315in}}{\pgfqpoint{1.436083in}{0.655020in}}%
\pgfpathcurveto{\pgfqpoint{1.443788in}{0.662724in}}{\pgfqpoint{1.448117in}{0.673175in}}{\pgfqpoint{1.448117in}{0.684070in}}%
\pgfpathcurveto{\pgfqpoint{1.448117in}{0.694966in}}{\pgfqpoint{1.443788in}{0.705417in}}{\pgfqpoint{1.436083in}{0.713121in}}%
\pgfpathcurveto{\pgfqpoint{1.428379in}{0.720825in}}{\pgfqpoint{1.417928in}{0.725154in}}{\pgfqpoint{1.407033in}{0.725154in}}%
\pgfpathcurveto{\pgfqpoint{1.396137in}{0.725154in}}{\pgfqpoint{1.385686in}{0.720825in}}{\pgfqpoint{1.377982in}{0.713121in}}%
\pgfpathcurveto{\pgfqpoint{1.370278in}{0.705417in}}{\pgfqpoint{1.365949in}{0.694966in}}{\pgfqpoint{1.365949in}{0.684070in}}%
\pgfpathcurveto{\pgfqpoint{1.365949in}{0.673175in}}{\pgfqpoint{1.370278in}{0.662724in}}{\pgfqpoint{1.377982in}{0.655020in}}%
\pgfpathcurveto{\pgfqpoint{1.385686in}{0.647315in}}{\pgfqpoint{1.396137in}{0.642986in}}{\pgfqpoint{1.407033in}{0.642986in}}%
\pgfusepath{stroke}%
\end{pgfscope}%
\begin{pgfscope}%
\pgfpathrectangle{\pgfqpoint{0.688192in}{0.670138in}}{\pgfqpoint{7.111808in}{5.061530in}}%
\pgfusepath{clip}%
\pgfsetbuttcap%
\pgfsetroundjoin%
\pgfsetlinewidth{1.003750pt}%
\definecolor{currentstroke}{rgb}{0.000000,0.000000,0.000000}%
\pgfsetstrokecolor{currentstroke}%
\pgfsetdash{}{0pt}%
\pgfpathmoveto{\pgfqpoint{0.749254in}{1.404065in}}%
\pgfpathcurveto{\pgfqpoint{0.760150in}{1.404065in}}{\pgfqpoint{0.770600in}{1.408394in}}{\pgfqpoint{0.778305in}{1.416099in}}%
\pgfpathcurveto{\pgfqpoint{0.786009in}{1.423803in}}{\pgfqpoint{0.790338in}{1.434254in}}{\pgfqpoint{0.790338in}{1.445149in}}%
\pgfpathcurveto{\pgfqpoint{0.790338in}{1.456045in}}{\pgfqpoint{0.786009in}{1.466496in}}{\pgfqpoint{0.778305in}{1.474200in}}%
\pgfpathcurveto{\pgfqpoint{0.770600in}{1.481904in}}{\pgfqpoint{0.760150in}{1.486233in}}{\pgfqpoint{0.749254in}{1.486233in}}%
\pgfpathcurveto{\pgfqpoint{0.738358in}{1.486233in}}{\pgfqpoint{0.727908in}{1.481904in}}{\pgfqpoint{0.720203in}{1.474200in}}%
\pgfpathcurveto{\pgfqpoint{0.712499in}{1.466496in}}{\pgfqpoint{0.708170in}{1.456045in}}{\pgfqpoint{0.708170in}{1.445149in}}%
\pgfpathcurveto{\pgfqpoint{0.708170in}{1.434254in}}{\pgfqpoint{0.712499in}{1.423803in}}{\pgfqpoint{0.720203in}{1.416099in}}%
\pgfpathcurveto{\pgfqpoint{0.727908in}{1.408394in}}{\pgfqpoint{0.738358in}{1.404065in}}{\pgfqpoint{0.749254in}{1.404065in}}%
\pgfpathlineto{\pgfqpoint{0.749254in}{1.404065in}}%
\pgfpathclose%
\pgfusepath{stroke}%
\end{pgfscope}%
\begin{pgfscope}%
\pgfpathrectangle{\pgfqpoint{0.688192in}{0.670138in}}{\pgfqpoint{7.111808in}{5.061530in}}%
\pgfusepath{clip}%
\pgfsetbuttcap%
\pgfsetroundjoin%
\pgfsetlinewidth{1.003750pt}%
\definecolor{currentstroke}{rgb}{0.000000,0.000000,0.000000}%
\pgfsetstrokecolor{currentstroke}%
\pgfsetdash{}{0pt}%
\pgfpathmoveto{\pgfqpoint{5.223268in}{0.784445in}}%
\pgfpathcurveto{\pgfqpoint{5.234163in}{0.784445in}}{\pgfqpoint{5.244614in}{0.788774in}}{\pgfqpoint{5.252318in}{0.796478in}}%
\pgfpathcurveto{\pgfqpoint{5.260023in}{0.804183in}}{\pgfqpoint{5.264352in}{0.814634in}}{\pgfqpoint{5.264352in}{0.825529in}}%
\pgfpathcurveto{\pgfqpoint{5.264352in}{0.836425in}}{\pgfqpoint{5.260023in}{0.846875in}}{\pgfqpoint{5.252318in}{0.854580in}}%
\pgfpathcurveto{\pgfqpoint{5.244614in}{0.862284in}}{\pgfqpoint{5.234163in}{0.866613in}}{\pgfqpoint{5.223268in}{0.866613in}}%
\pgfpathcurveto{\pgfqpoint{5.212372in}{0.866613in}}{\pgfqpoint{5.201921in}{0.862284in}}{\pgfqpoint{5.194217in}{0.854580in}}%
\pgfpathcurveto{\pgfqpoint{5.186513in}{0.846875in}}{\pgfqpoint{5.182184in}{0.836425in}}{\pgfqpoint{5.182184in}{0.825529in}}%
\pgfpathcurveto{\pgfqpoint{5.182184in}{0.814634in}}{\pgfqpoint{5.186513in}{0.804183in}}{\pgfqpoint{5.194217in}{0.796478in}}%
\pgfpathcurveto{\pgfqpoint{5.201921in}{0.788774in}}{\pgfqpoint{5.212372in}{0.784445in}}{\pgfqpoint{5.223268in}{0.784445in}}%
\pgfpathlineto{\pgfqpoint{5.223268in}{0.784445in}}%
\pgfpathclose%
\pgfusepath{stroke}%
\end{pgfscope}%
\begin{pgfscope}%
\pgfpathrectangle{\pgfqpoint{0.688192in}{0.670138in}}{\pgfqpoint{7.111808in}{5.061530in}}%
\pgfusepath{clip}%
\pgfsetbuttcap%
\pgfsetroundjoin%
\pgfsetlinewidth{1.003750pt}%
\definecolor{currentstroke}{rgb}{0.000000,0.000000,0.000000}%
\pgfsetstrokecolor{currentstroke}%
\pgfsetdash{}{0pt}%
\pgfpathmoveto{\pgfqpoint{1.151625in}{0.644694in}}%
\pgfpathcurveto{\pgfqpoint{1.162521in}{0.644694in}}{\pgfqpoint{1.172972in}{0.649023in}}{\pgfqpoint{1.180676in}{0.656727in}}%
\pgfpathcurveto{\pgfqpoint{1.188380in}{0.664431in}}{\pgfqpoint{1.192709in}{0.674882in}}{\pgfqpoint{1.192709in}{0.685778in}}%
\pgfpathcurveto{\pgfqpoint{1.192709in}{0.696673in}}{\pgfqpoint{1.188380in}{0.707124in}}{\pgfqpoint{1.180676in}{0.714828in}}%
\pgfpathcurveto{\pgfqpoint{1.172972in}{0.722533in}}{\pgfqpoint{1.162521in}{0.726862in}}{\pgfqpoint{1.151625in}{0.726862in}}%
\pgfpathcurveto{\pgfqpoint{1.140730in}{0.726862in}}{\pgfqpoint{1.130279in}{0.722533in}}{\pgfqpoint{1.122575in}{0.714828in}}%
\pgfpathcurveto{\pgfqpoint{1.114870in}{0.707124in}}{\pgfqpoint{1.110542in}{0.696673in}}{\pgfqpoint{1.110542in}{0.685778in}}%
\pgfpathcurveto{\pgfqpoint{1.110542in}{0.674882in}}{\pgfqpoint{1.114870in}{0.664431in}}{\pgfqpoint{1.122575in}{0.656727in}}%
\pgfpathcurveto{\pgfqpoint{1.130279in}{0.649023in}}{\pgfqpoint{1.140730in}{0.644694in}}{\pgfqpoint{1.151625in}{0.644694in}}%
\pgfusepath{stroke}%
\end{pgfscope}%
\begin{pgfscope}%
\pgfpathrectangle{\pgfqpoint{0.688192in}{0.670138in}}{\pgfqpoint{7.111808in}{5.061530in}}%
\pgfusepath{clip}%
\pgfsetbuttcap%
\pgfsetroundjoin%
\pgfsetlinewidth{1.003750pt}%
\definecolor{currentstroke}{rgb}{0.000000,0.000000,0.000000}%
\pgfsetstrokecolor{currentstroke}%
\pgfsetdash{}{0pt}%
\pgfpathmoveto{\pgfqpoint{2.265105in}{0.638474in}}%
\pgfpathcurveto{\pgfqpoint{2.276001in}{0.638474in}}{\pgfqpoint{2.286451in}{0.642803in}}{\pgfqpoint{2.294156in}{0.650507in}}%
\pgfpathcurveto{\pgfqpoint{2.301860in}{0.658211in}}{\pgfqpoint{2.306189in}{0.668662in}}{\pgfqpoint{2.306189in}{0.679558in}}%
\pgfpathcurveto{\pgfqpoint{2.306189in}{0.690453in}}{\pgfqpoint{2.301860in}{0.700904in}}{\pgfqpoint{2.294156in}{0.708608in}}%
\pgfpathcurveto{\pgfqpoint{2.286451in}{0.716313in}}{\pgfqpoint{2.276001in}{0.720642in}}{\pgfqpoint{2.265105in}{0.720642in}}%
\pgfpathcurveto{\pgfqpoint{2.254209in}{0.720642in}}{\pgfqpoint{2.243759in}{0.716313in}}{\pgfqpoint{2.236054in}{0.708608in}}%
\pgfpathcurveto{\pgfqpoint{2.228350in}{0.700904in}}{\pgfqpoint{2.224021in}{0.690453in}}{\pgfqpoint{2.224021in}{0.679558in}}%
\pgfpathcurveto{\pgfqpoint{2.224021in}{0.668662in}}{\pgfqpoint{2.228350in}{0.658211in}}{\pgfqpoint{2.236054in}{0.650507in}}%
\pgfpathcurveto{\pgfqpoint{2.243759in}{0.642803in}}{\pgfqpoint{2.254209in}{0.638474in}}{\pgfqpoint{2.265105in}{0.638474in}}%
\pgfusepath{stroke}%
\end{pgfscope}%
\begin{pgfscope}%
\pgfpathrectangle{\pgfqpoint{0.688192in}{0.670138in}}{\pgfqpoint{7.111808in}{5.061530in}}%
\pgfusepath{clip}%
\pgfsetbuttcap%
\pgfsetroundjoin%
\pgfsetlinewidth{1.003750pt}%
\definecolor{currentstroke}{rgb}{0.000000,0.000000,0.000000}%
\pgfsetstrokecolor{currentstroke}%
\pgfsetdash{}{0pt}%
\pgfpathmoveto{\pgfqpoint{1.495917in}{0.642463in}}%
\pgfpathcurveto{\pgfqpoint{1.506813in}{0.642463in}}{\pgfqpoint{1.517263in}{0.646792in}}{\pgfqpoint{1.524968in}{0.654497in}}%
\pgfpathcurveto{\pgfqpoint{1.532672in}{0.662201in}}{\pgfqpoint{1.537001in}{0.672652in}}{\pgfqpoint{1.537001in}{0.683547in}}%
\pgfpathcurveto{\pgfqpoint{1.537001in}{0.694443in}}{\pgfqpoint{1.532672in}{0.704894in}}{\pgfqpoint{1.524968in}{0.712598in}}%
\pgfpathcurveto{\pgfqpoint{1.517263in}{0.720302in}}{\pgfqpoint{1.506813in}{0.724631in}}{\pgfqpoint{1.495917in}{0.724631in}}%
\pgfpathcurveto{\pgfqpoint{1.485021in}{0.724631in}}{\pgfqpoint{1.474571in}{0.720302in}}{\pgfqpoint{1.466866in}{0.712598in}}%
\pgfpathcurveto{\pgfqpoint{1.459162in}{0.704894in}}{\pgfqpoint{1.454833in}{0.694443in}}{\pgfqpoint{1.454833in}{0.683547in}}%
\pgfpathcurveto{\pgfqpoint{1.454833in}{0.672652in}}{\pgfqpoint{1.459162in}{0.662201in}}{\pgfqpoint{1.466866in}{0.654497in}}%
\pgfpathcurveto{\pgfqpoint{1.474571in}{0.646792in}}{\pgfqpoint{1.485021in}{0.642463in}}{\pgfqpoint{1.495917in}{0.642463in}}%
\pgfusepath{stroke}%
\end{pgfscope}%
\begin{pgfscope}%
\pgfpathrectangle{\pgfqpoint{0.688192in}{0.670138in}}{\pgfqpoint{7.111808in}{5.061530in}}%
\pgfusepath{clip}%
\pgfsetbuttcap%
\pgfsetroundjoin%
\pgfsetlinewidth{1.003750pt}%
\definecolor{currentstroke}{rgb}{0.000000,0.000000,0.000000}%
\pgfsetstrokecolor{currentstroke}%
\pgfsetdash{}{0pt}%
\pgfpathmoveto{\pgfqpoint{2.429320in}{0.637441in}}%
\pgfpathcurveto{\pgfqpoint{2.440215in}{0.637441in}}{\pgfqpoint{2.450666in}{0.641770in}}{\pgfqpoint{2.458370in}{0.649474in}}%
\pgfpathcurveto{\pgfqpoint{2.466075in}{0.657178in}}{\pgfqpoint{2.470404in}{0.667629in}}{\pgfqpoint{2.470404in}{0.678525in}}%
\pgfpathcurveto{\pgfqpoint{2.470404in}{0.689420in}}{\pgfqpoint{2.466075in}{0.699871in}}{\pgfqpoint{2.458370in}{0.707575in}}%
\pgfpathcurveto{\pgfqpoint{2.450666in}{0.715280in}}{\pgfqpoint{2.440215in}{0.719609in}}{\pgfqpoint{2.429320in}{0.719609in}}%
\pgfpathcurveto{\pgfqpoint{2.418424in}{0.719609in}}{\pgfqpoint{2.407973in}{0.715280in}}{\pgfqpoint{2.400269in}{0.707575in}}%
\pgfpathcurveto{\pgfqpoint{2.392565in}{0.699871in}}{\pgfqpoint{2.388236in}{0.689420in}}{\pgfqpoint{2.388236in}{0.678525in}}%
\pgfpathcurveto{\pgfqpoint{2.388236in}{0.667629in}}{\pgfqpoint{2.392565in}{0.657178in}}{\pgfqpoint{2.400269in}{0.649474in}}%
\pgfpathcurveto{\pgfqpoint{2.407973in}{0.641770in}}{\pgfqpoint{2.418424in}{0.637441in}}{\pgfqpoint{2.429320in}{0.637441in}}%
\pgfusepath{stroke}%
\end{pgfscope}%
\begin{pgfscope}%
\pgfpathrectangle{\pgfqpoint{0.688192in}{0.670138in}}{\pgfqpoint{7.111808in}{5.061530in}}%
\pgfusepath{clip}%
\pgfsetbuttcap%
\pgfsetroundjoin%
\pgfsetlinewidth{1.003750pt}%
\definecolor{currentstroke}{rgb}{0.000000,0.000000,0.000000}%
\pgfsetstrokecolor{currentstroke}%
\pgfsetdash{}{0pt}%
\pgfpathmoveto{\pgfqpoint{5.542141in}{2.935103in}}%
\pgfpathcurveto{\pgfqpoint{5.553036in}{2.935103in}}{\pgfqpoint{5.563487in}{2.939432in}}{\pgfqpoint{5.571191in}{2.947136in}}%
\pgfpathcurveto{\pgfqpoint{5.578896in}{2.954840in}}{\pgfqpoint{5.583224in}{2.965291in}}{\pgfqpoint{5.583224in}{2.976187in}}%
\pgfpathcurveto{\pgfqpoint{5.583224in}{2.987082in}}{\pgfqpoint{5.578896in}{2.997533in}}{\pgfqpoint{5.571191in}{3.005237in}}%
\pgfpathcurveto{\pgfqpoint{5.563487in}{3.012942in}}{\pgfqpoint{5.553036in}{3.017271in}}{\pgfqpoint{5.542141in}{3.017271in}}%
\pgfpathcurveto{\pgfqpoint{5.531245in}{3.017271in}}{\pgfqpoint{5.520794in}{3.012942in}}{\pgfqpoint{5.513090in}{3.005237in}}%
\pgfpathcurveto{\pgfqpoint{5.505386in}{2.997533in}}{\pgfqpoint{5.501057in}{2.987082in}}{\pgfqpoint{5.501057in}{2.976187in}}%
\pgfpathcurveto{\pgfqpoint{5.501057in}{2.965291in}}{\pgfqpoint{5.505386in}{2.954840in}}{\pgfqpoint{5.513090in}{2.947136in}}%
\pgfpathcurveto{\pgfqpoint{5.520794in}{2.939432in}}{\pgfqpoint{5.531245in}{2.935103in}}{\pgfqpoint{5.542141in}{2.935103in}}%
\pgfpathlineto{\pgfqpoint{5.542141in}{2.935103in}}%
\pgfpathclose%
\pgfusepath{stroke}%
\end{pgfscope}%
\begin{pgfscope}%
\pgfpathrectangle{\pgfqpoint{0.688192in}{0.670138in}}{\pgfqpoint{7.111808in}{5.061530in}}%
\pgfusepath{clip}%
\pgfsetbuttcap%
\pgfsetroundjoin%
\pgfsetlinewidth{1.003750pt}%
\definecolor{currentstroke}{rgb}{0.000000,0.000000,0.000000}%
\pgfsetstrokecolor{currentstroke}%
\pgfsetdash{}{0pt}%
\pgfpathmoveto{\pgfqpoint{1.495917in}{0.642463in}}%
\pgfpathcurveto{\pgfqpoint{1.506813in}{0.642463in}}{\pgfqpoint{1.517263in}{0.646792in}}{\pgfqpoint{1.524968in}{0.654497in}}%
\pgfpathcurveto{\pgfqpoint{1.532672in}{0.662201in}}{\pgfqpoint{1.537001in}{0.672652in}}{\pgfqpoint{1.537001in}{0.683547in}}%
\pgfpathcurveto{\pgfqpoint{1.537001in}{0.694443in}}{\pgfqpoint{1.532672in}{0.704894in}}{\pgfqpoint{1.524968in}{0.712598in}}%
\pgfpathcurveto{\pgfqpoint{1.517263in}{0.720302in}}{\pgfqpoint{1.506813in}{0.724631in}}{\pgfqpoint{1.495917in}{0.724631in}}%
\pgfpathcurveto{\pgfqpoint{1.485021in}{0.724631in}}{\pgfqpoint{1.474571in}{0.720302in}}{\pgfqpoint{1.466866in}{0.712598in}}%
\pgfpathcurveto{\pgfqpoint{1.459162in}{0.704894in}}{\pgfqpoint{1.454833in}{0.694443in}}{\pgfqpoint{1.454833in}{0.683547in}}%
\pgfpathcurveto{\pgfqpoint{1.454833in}{0.672652in}}{\pgfqpoint{1.459162in}{0.662201in}}{\pgfqpoint{1.466866in}{0.654497in}}%
\pgfpathcurveto{\pgfqpoint{1.474571in}{0.646792in}}{\pgfqpoint{1.485021in}{0.642463in}}{\pgfqpoint{1.495917in}{0.642463in}}%
\pgfusepath{stroke}%
\end{pgfscope}%
\begin{pgfscope}%
\pgfpathrectangle{\pgfqpoint{0.688192in}{0.670138in}}{\pgfqpoint{7.111808in}{5.061530in}}%
\pgfusepath{clip}%
\pgfsetbuttcap%
\pgfsetroundjoin%
\pgfsetlinewidth{1.003750pt}%
\definecolor{currentstroke}{rgb}{0.000000,0.000000,0.000000}%
\pgfsetstrokecolor{currentstroke}%
\pgfsetdash{}{0pt}%
\pgfpathmoveto{\pgfqpoint{2.588475in}{0.663743in}}%
\pgfpathcurveto{\pgfqpoint{2.599371in}{0.663743in}}{\pgfqpoint{2.609822in}{0.668072in}}{\pgfqpoint{2.617526in}{0.675776in}}%
\pgfpathcurveto{\pgfqpoint{2.625231in}{0.683480in}}{\pgfqpoint{2.629559in}{0.693931in}}{\pgfqpoint{2.629559in}{0.704827in}}%
\pgfpathcurveto{\pgfqpoint{2.629559in}{0.715722in}}{\pgfqpoint{2.625231in}{0.726173in}}{\pgfqpoint{2.617526in}{0.733877in}}%
\pgfpathcurveto{\pgfqpoint{2.609822in}{0.741582in}}{\pgfqpoint{2.599371in}{0.745911in}}{\pgfqpoint{2.588475in}{0.745911in}}%
\pgfpathcurveto{\pgfqpoint{2.577580in}{0.745911in}}{\pgfqpoint{2.567129in}{0.741582in}}{\pgfqpoint{2.559425in}{0.733877in}}%
\pgfpathcurveto{\pgfqpoint{2.551720in}{0.726173in}}{\pgfqpoint{2.547392in}{0.715722in}}{\pgfqpoint{2.547392in}{0.704827in}}%
\pgfpathcurveto{\pgfqpoint{2.547392in}{0.693931in}}{\pgfqpoint{2.551720in}{0.683480in}}{\pgfqpoint{2.559425in}{0.675776in}}%
\pgfpathcurveto{\pgfqpoint{2.567129in}{0.668072in}}{\pgfqpoint{2.577580in}{0.663743in}}{\pgfqpoint{2.588475in}{0.663743in}}%
\pgfusepath{stroke}%
\end{pgfscope}%
\begin{pgfscope}%
\pgfpathrectangle{\pgfqpoint{0.688192in}{0.670138in}}{\pgfqpoint{7.111808in}{5.061530in}}%
\pgfusepath{clip}%
\pgfsetbuttcap%
\pgfsetroundjoin%
\pgfsetlinewidth{1.003750pt}%
\definecolor{currentstroke}{rgb}{0.000000,0.000000,0.000000}%
\pgfsetstrokecolor{currentstroke}%
\pgfsetdash{}{0pt}%
\pgfpathmoveto{\pgfqpoint{2.383761in}{0.938718in}}%
\pgfpathcurveto{\pgfqpoint{2.394657in}{0.938718in}}{\pgfqpoint{2.405108in}{0.943047in}}{\pgfqpoint{2.412812in}{0.950751in}}%
\pgfpathcurveto{\pgfqpoint{2.420516in}{0.958456in}}{\pgfqpoint{2.424845in}{0.968907in}}{\pgfqpoint{2.424845in}{0.979802in}}%
\pgfpathcurveto{\pgfqpoint{2.424845in}{0.990698in}}{\pgfqpoint{2.420516in}{1.001148in}}{\pgfqpoint{2.412812in}{1.008853in}}%
\pgfpathcurveto{\pgfqpoint{2.405108in}{1.016557in}}{\pgfqpoint{2.394657in}{1.020886in}}{\pgfqpoint{2.383761in}{1.020886in}}%
\pgfpathcurveto{\pgfqpoint{2.372866in}{1.020886in}}{\pgfqpoint{2.362415in}{1.016557in}}{\pgfqpoint{2.354711in}{1.008853in}}%
\pgfpathcurveto{\pgfqpoint{2.347006in}{1.001148in}}{\pgfqpoint{2.342677in}{0.990698in}}{\pgfqpoint{2.342677in}{0.979802in}}%
\pgfpathcurveto{\pgfqpoint{2.342677in}{0.968907in}}{\pgfqpoint{2.347006in}{0.958456in}}{\pgfqpoint{2.354711in}{0.950751in}}%
\pgfpathcurveto{\pgfqpoint{2.362415in}{0.943047in}}{\pgfqpoint{2.372866in}{0.938718in}}{\pgfqpoint{2.383761in}{0.938718in}}%
\pgfpathlineto{\pgfqpoint{2.383761in}{0.938718in}}%
\pgfpathclose%
\pgfusepath{stroke}%
\end{pgfscope}%
\begin{pgfscope}%
\pgfpathrectangle{\pgfqpoint{0.688192in}{0.670138in}}{\pgfqpoint{7.111808in}{5.061530in}}%
\pgfusepath{clip}%
\pgfsetbuttcap%
\pgfsetroundjoin%
\pgfsetlinewidth{1.003750pt}%
\definecolor{currentstroke}{rgb}{0.000000,0.000000,0.000000}%
\pgfsetstrokecolor{currentstroke}%
\pgfsetdash{}{0pt}%
\pgfpathmoveto{\pgfqpoint{6.712361in}{2.552477in}}%
\pgfpathcurveto{\pgfqpoint{6.723257in}{2.552477in}}{\pgfqpoint{6.733708in}{2.556806in}}{\pgfqpoint{6.741412in}{2.564511in}}%
\pgfpathcurveto{\pgfqpoint{6.749116in}{2.572215in}}{\pgfqpoint{6.753445in}{2.582666in}}{\pgfqpoint{6.753445in}{2.593561in}}%
\pgfpathcurveto{\pgfqpoint{6.753445in}{2.604457in}}{\pgfqpoint{6.749116in}{2.614908in}}{\pgfqpoint{6.741412in}{2.622612in}}%
\pgfpathcurveto{\pgfqpoint{6.733708in}{2.630316in}}{\pgfqpoint{6.723257in}{2.634645in}}{\pgfqpoint{6.712361in}{2.634645in}}%
\pgfpathcurveto{\pgfqpoint{6.701466in}{2.634645in}}{\pgfqpoint{6.691015in}{2.630316in}}{\pgfqpoint{6.683311in}{2.622612in}}%
\pgfpathcurveto{\pgfqpoint{6.675606in}{2.614908in}}{\pgfqpoint{6.671277in}{2.604457in}}{\pgfqpoint{6.671277in}{2.593561in}}%
\pgfpathcurveto{\pgfqpoint{6.671277in}{2.582666in}}{\pgfqpoint{6.675606in}{2.572215in}}{\pgfqpoint{6.683311in}{2.564511in}}%
\pgfpathcurveto{\pgfqpoint{6.691015in}{2.556806in}}{\pgfqpoint{6.701466in}{2.552477in}}{\pgfqpoint{6.712361in}{2.552477in}}%
\pgfpathlineto{\pgfqpoint{6.712361in}{2.552477in}}%
\pgfpathclose%
\pgfusepath{stroke}%
\end{pgfscope}%
\begin{pgfscope}%
\pgfpathrectangle{\pgfqpoint{0.688192in}{0.670138in}}{\pgfqpoint{7.111808in}{5.061530in}}%
\pgfusepath{clip}%
\pgfsetbuttcap%
\pgfsetroundjoin%
\pgfsetlinewidth{1.003750pt}%
\definecolor{currentstroke}{rgb}{0.000000,0.000000,0.000000}%
\pgfsetstrokecolor{currentstroke}%
\pgfsetdash{}{0pt}%
\pgfpathmoveto{\pgfqpoint{2.526063in}{2.676387in}}%
\pgfpathcurveto{\pgfqpoint{2.536958in}{2.676387in}}{\pgfqpoint{2.547409in}{2.680716in}}{\pgfqpoint{2.555114in}{2.688420in}}%
\pgfpathcurveto{\pgfqpoint{2.562818in}{2.696125in}}{\pgfqpoint{2.567147in}{2.706575in}}{\pgfqpoint{2.567147in}{2.717471in}}%
\pgfpathcurveto{\pgfqpoint{2.567147in}{2.728367in}}{\pgfqpoint{2.562818in}{2.738817in}}{\pgfqpoint{2.555114in}{2.746522in}}%
\pgfpathcurveto{\pgfqpoint{2.547409in}{2.754226in}}{\pgfqpoint{2.536958in}{2.758555in}}{\pgfqpoint{2.526063in}{2.758555in}}%
\pgfpathcurveto{\pgfqpoint{2.515167in}{2.758555in}}{\pgfqpoint{2.504716in}{2.754226in}}{\pgfqpoint{2.497012in}{2.746522in}}%
\pgfpathcurveto{\pgfqpoint{2.489308in}{2.738817in}}{\pgfqpoint{2.484979in}{2.728367in}}{\pgfqpoint{2.484979in}{2.717471in}}%
\pgfpathcurveto{\pgfqpoint{2.484979in}{2.706575in}}{\pgfqpoint{2.489308in}{2.696125in}}{\pgfqpoint{2.497012in}{2.688420in}}%
\pgfpathcurveto{\pgfqpoint{2.504716in}{2.680716in}}{\pgfqpoint{2.515167in}{2.676387in}}{\pgfqpoint{2.526063in}{2.676387in}}%
\pgfpathlineto{\pgfqpoint{2.526063in}{2.676387in}}%
\pgfpathclose%
\pgfusepath{stroke}%
\end{pgfscope}%
\begin{pgfscope}%
\pgfpathrectangle{\pgfqpoint{0.688192in}{0.670138in}}{\pgfqpoint{7.111808in}{5.061530in}}%
\pgfusepath{clip}%
\pgfsetbuttcap%
\pgfsetroundjoin%
\pgfsetlinewidth{1.003750pt}%
\definecolor{currentstroke}{rgb}{0.000000,0.000000,0.000000}%
\pgfsetstrokecolor{currentstroke}%
\pgfsetdash{}{0pt}%
\pgfpathmoveto{\pgfqpoint{0.962397in}{1.603179in}}%
\pgfpathcurveto{\pgfqpoint{0.973292in}{1.603179in}}{\pgfqpoint{0.983743in}{1.607508in}}{\pgfqpoint{0.991448in}{1.615212in}}%
\pgfpathcurveto{\pgfqpoint{0.999152in}{1.622916in}}{\pgfqpoint{1.003481in}{1.633367in}}{\pgfqpoint{1.003481in}{1.644263in}}%
\pgfpathcurveto{\pgfqpoint{1.003481in}{1.655158in}}{\pgfqpoint{0.999152in}{1.665609in}}{\pgfqpoint{0.991448in}{1.673313in}}%
\pgfpathcurveto{\pgfqpoint{0.983743in}{1.681018in}}{\pgfqpoint{0.973292in}{1.685347in}}{\pgfqpoint{0.962397in}{1.685347in}}%
\pgfpathcurveto{\pgfqpoint{0.951501in}{1.685347in}}{\pgfqpoint{0.941050in}{1.681018in}}{\pgfqpoint{0.933346in}{1.673313in}}%
\pgfpathcurveto{\pgfqpoint{0.925642in}{1.665609in}}{\pgfqpoint{0.921313in}{1.655158in}}{\pgfqpoint{0.921313in}{1.644263in}}%
\pgfpathcurveto{\pgfqpoint{0.921313in}{1.633367in}}{\pgfqpoint{0.925642in}{1.622916in}}{\pgfqpoint{0.933346in}{1.615212in}}%
\pgfpathcurveto{\pgfqpoint{0.941050in}{1.607508in}}{\pgfqpoint{0.951501in}{1.603179in}}{\pgfqpoint{0.962397in}{1.603179in}}%
\pgfpathlineto{\pgfqpoint{0.962397in}{1.603179in}}%
\pgfpathclose%
\pgfusepath{stroke}%
\end{pgfscope}%
\begin{pgfscope}%
\pgfpathrectangle{\pgfqpoint{0.688192in}{0.670138in}}{\pgfqpoint{7.111808in}{5.061530in}}%
\pgfusepath{clip}%
\pgfsetbuttcap%
\pgfsetroundjoin%
\pgfsetlinewidth{1.003750pt}%
\definecolor{currentstroke}{rgb}{0.000000,0.000000,0.000000}%
\pgfsetstrokecolor{currentstroke}%
\pgfsetdash{}{0pt}%
\pgfpathmoveto{\pgfqpoint{5.692400in}{3.514494in}}%
\pgfpathcurveto{\pgfqpoint{5.703296in}{3.514494in}}{\pgfqpoint{5.713747in}{3.518823in}}{\pgfqpoint{5.721451in}{3.526527in}}%
\pgfpathcurveto{\pgfqpoint{5.729155in}{3.534232in}}{\pgfqpoint{5.733484in}{3.544683in}}{\pgfqpoint{5.733484in}{3.555578in}}%
\pgfpathcurveto{\pgfqpoint{5.733484in}{3.566474in}}{\pgfqpoint{5.729155in}{3.576924in}}{\pgfqpoint{5.721451in}{3.584629in}}%
\pgfpathcurveto{\pgfqpoint{5.713747in}{3.592333in}}{\pgfqpoint{5.703296in}{3.596662in}}{\pgfqpoint{5.692400in}{3.596662in}}%
\pgfpathcurveto{\pgfqpoint{5.681505in}{3.596662in}}{\pgfqpoint{5.671054in}{3.592333in}}{\pgfqpoint{5.663350in}{3.584629in}}%
\pgfpathcurveto{\pgfqpoint{5.655645in}{3.576924in}}{\pgfqpoint{5.651316in}{3.566474in}}{\pgfqpoint{5.651316in}{3.555578in}}%
\pgfpathcurveto{\pgfqpoint{5.651316in}{3.544683in}}{\pgfqpoint{5.655645in}{3.534232in}}{\pgfqpoint{5.663350in}{3.526527in}}%
\pgfpathcurveto{\pgfqpoint{5.671054in}{3.518823in}}{\pgfqpoint{5.681505in}{3.514494in}}{\pgfqpoint{5.692400in}{3.514494in}}%
\pgfpathlineto{\pgfqpoint{5.692400in}{3.514494in}}%
\pgfpathclose%
\pgfusepath{stroke}%
\end{pgfscope}%
\begin{pgfscope}%
\pgfpathrectangle{\pgfqpoint{0.688192in}{0.670138in}}{\pgfqpoint{7.111808in}{5.061530in}}%
\pgfusepath{clip}%
\pgfsetbuttcap%
\pgfsetroundjoin%
\pgfsetlinewidth{1.003750pt}%
\definecolor{currentstroke}{rgb}{0.000000,0.000000,0.000000}%
\pgfsetstrokecolor{currentstroke}%
\pgfsetdash{}{0pt}%
\pgfpathmoveto{\pgfqpoint{0.987244in}{0.664981in}}%
\pgfpathcurveto{\pgfqpoint{0.998140in}{0.664981in}}{\pgfqpoint{1.008591in}{0.669310in}}{\pgfqpoint{1.016295in}{0.677014in}}%
\pgfpathcurveto{\pgfqpoint{1.023999in}{0.684719in}}{\pgfqpoint{1.028328in}{0.695169in}}{\pgfqpoint{1.028328in}{0.706065in}}%
\pgfpathcurveto{\pgfqpoint{1.028328in}{0.716961in}}{\pgfqpoint{1.023999in}{0.727411in}}{\pgfqpoint{1.016295in}{0.735116in}}%
\pgfpathcurveto{\pgfqpoint{1.008591in}{0.742820in}}{\pgfqpoint{0.998140in}{0.747149in}}{\pgfqpoint{0.987244in}{0.747149in}}%
\pgfpathcurveto{\pgfqpoint{0.976349in}{0.747149in}}{\pgfqpoint{0.965898in}{0.742820in}}{\pgfqpoint{0.958194in}{0.735116in}}%
\pgfpathcurveto{\pgfqpoint{0.950489in}{0.727411in}}{\pgfqpoint{0.946161in}{0.716961in}}{\pgfqpoint{0.946161in}{0.706065in}}%
\pgfpathcurveto{\pgfqpoint{0.946161in}{0.695169in}}{\pgfqpoint{0.950489in}{0.684719in}}{\pgfqpoint{0.958194in}{0.677014in}}%
\pgfpathcurveto{\pgfqpoint{0.965898in}{0.669310in}}{\pgfqpoint{0.976349in}{0.664981in}}{\pgfqpoint{0.987244in}{0.664981in}}%
\pgfusepath{stroke}%
\end{pgfscope}%
\begin{pgfscope}%
\pgfpathrectangle{\pgfqpoint{0.688192in}{0.670138in}}{\pgfqpoint{7.111808in}{5.061530in}}%
\pgfusepath{clip}%
\pgfsetbuttcap%
\pgfsetroundjoin%
\pgfsetlinewidth{1.003750pt}%
\definecolor{currentstroke}{rgb}{0.000000,0.000000,0.000000}%
\pgfsetstrokecolor{currentstroke}%
\pgfsetdash{}{0pt}%
\pgfpathmoveto{\pgfqpoint{2.768315in}{3.990841in}}%
\pgfpathcurveto{\pgfqpoint{2.779211in}{3.990841in}}{\pgfqpoint{2.789661in}{3.995170in}}{\pgfqpoint{2.797366in}{4.002874in}}%
\pgfpathcurveto{\pgfqpoint{2.805070in}{4.010579in}}{\pgfqpoint{2.809399in}{4.021029in}}{\pgfqpoint{2.809399in}{4.031925in}}%
\pgfpathcurveto{\pgfqpoint{2.809399in}{4.042821in}}{\pgfqpoint{2.805070in}{4.053271in}}{\pgfqpoint{2.797366in}{4.060976in}}%
\pgfpathcurveto{\pgfqpoint{2.789661in}{4.068680in}}{\pgfqpoint{2.779211in}{4.073009in}}{\pgfqpoint{2.768315in}{4.073009in}}%
\pgfpathcurveto{\pgfqpoint{2.757419in}{4.073009in}}{\pgfqpoint{2.746969in}{4.068680in}}{\pgfqpoint{2.739264in}{4.060976in}}%
\pgfpathcurveto{\pgfqpoint{2.731560in}{4.053271in}}{\pgfqpoint{2.727231in}{4.042821in}}{\pgfqpoint{2.727231in}{4.031925in}}%
\pgfpathcurveto{\pgfqpoint{2.727231in}{4.021029in}}{\pgfqpoint{2.731560in}{4.010579in}}{\pgfqpoint{2.739264in}{4.002874in}}%
\pgfpathcurveto{\pgfqpoint{2.746969in}{3.995170in}}{\pgfqpoint{2.757419in}{3.990841in}}{\pgfqpoint{2.768315in}{3.990841in}}%
\pgfpathlineto{\pgfqpoint{2.768315in}{3.990841in}}%
\pgfpathclose%
\pgfusepath{stroke}%
\end{pgfscope}%
\begin{pgfscope}%
\pgfpathrectangle{\pgfqpoint{0.688192in}{0.670138in}}{\pgfqpoint{7.111808in}{5.061530in}}%
\pgfusepath{clip}%
\pgfsetbuttcap%
\pgfsetroundjoin%
\pgfsetlinewidth{1.003750pt}%
\definecolor{currentstroke}{rgb}{0.000000,0.000000,0.000000}%
\pgfsetstrokecolor{currentstroke}%
\pgfsetdash{}{0pt}%
\pgfpathmoveto{\pgfqpoint{2.929728in}{1.495747in}}%
\pgfpathcurveto{\pgfqpoint{2.940624in}{1.495747in}}{\pgfqpoint{2.951074in}{1.500075in}}{\pgfqpoint{2.958779in}{1.507780in}}%
\pgfpathcurveto{\pgfqpoint{2.966483in}{1.515484in}}{\pgfqpoint{2.970812in}{1.525935in}}{\pgfqpoint{2.970812in}{1.536830in}}%
\pgfpathcurveto{\pgfqpoint{2.970812in}{1.547726in}}{\pgfqpoint{2.966483in}{1.558177in}}{\pgfqpoint{2.958779in}{1.565881in}}%
\pgfpathcurveto{\pgfqpoint{2.951074in}{1.573585in}}{\pgfqpoint{2.940624in}{1.577914in}}{\pgfqpoint{2.929728in}{1.577914in}}%
\pgfpathcurveto{\pgfqpoint{2.918833in}{1.577914in}}{\pgfqpoint{2.908382in}{1.573585in}}{\pgfqpoint{2.900677in}{1.565881in}}%
\pgfpathcurveto{\pgfqpoint{2.892973in}{1.558177in}}{\pgfqpoint{2.888644in}{1.547726in}}{\pgfqpoint{2.888644in}{1.536830in}}%
\pgfpathcurveto{\pgfqpoint{2.888644in}{1.525935in}}{\pgfqpoint{2.892973in}{1.515484in}}{\pgfqpoint{2.900677in}{1.507780in}}%
\pgfpathcurveto{\pgfqpoint{2.908382in}{1.500075in}}{\pgfqpoint{2.918833in}{1.495747in}}{\pgfqpoint{2.929728in}{1.495747in}}%
\pgfpathlineto{\pgfqpoint{2.929728in}{1.495747in}}%
\pgfpathclose%
\pgfusepath{stroke}%
\end{pgfscope}%
\begin{pgfscope}%
\pgfpathrectangle{\pgfqpoint{0.688192in}{0.670138in}}{\pgfqpoint{7.111808in}{5.061530in}}%
\pgfusepath{clip}%
\pgfsetbuttcap%
\pgfsetroundjoin%
\pgfsetlinewidth{1.003750pt}%
\definecolor{currentstroke}{rgb}{0.000000,0.000000,0.000000}%
\pgfsetstrokecolor{currentstroke}%
\pgfsetdash{}{0pt}%
\pgfpathmoveto{\pgfqpoint{1.719213in}{0.641640in}}%
\pgfpathcurveto{\pgfqpoint{1.730109in}{0.641640in}}{\pgfqpoint{1.740559in}{0.645969in}}{\pgfqpoint{1.748264in}{0.653673in}}%
\pgfpathcurveto{\pgfqpoint{1.755968in}{0.661377in}}{\pgfqpoint{1.760297in}{0.671828in}}{\pgfqpoint{1.760297in}{0.682724in}}%
\pgfpathcurveto{\pgfqpoint{1.760297in}{0.693619in}}{\pgfqpoint{1.755968in}{0.704070in}}{\pgfqpoint{1.748264in}{0.711774in}}%
\pgfpathcurveto{\pgfqpoint{1.740559in}{0.719479in}}{\pgfqpoint{1.730109in}{0.723807in}}{\pgfqpoint{1.719213in}{0.723807in}}%
\pgfpathcurveto{\pgfqpoint{1.708317in}{0.723807in}}{\pgfqpoint{1.697867in}{0.719479in}}{\pgfqpoint{1.690162in}{0.711774in}}%
\pgfpathcurveto{\pgfqpoint{1.682458in}{0.704070in}}{\pgfqpoint{1.678129in}{0.693619in}}{\pgfqpoint{1.678129in}{0.682724in}}%
\pgfpathcurveto{\pgfqpoint{1.678129in}{0.671828in}}{\pgfqpoint{1.682458in}{0.661377in}}{\pgfqpoint{1.690162in}{0.653673in}}%
\pgfpathcurveto{\pgfqpoint{1.697867in}{0.645969in}}{\pgfqpoint{1.708317in}{0.641640in}}{\pgfqpoint{1.719213in}{0.641640in}}%
\pgfusepath{stroke}%
\end{pgfscope}%
\begin{pgfscope}%
\pgfpathrectangle{\pgfqpoint{0.688192in}{0.670138in}}{\pgfqpoint{7.111808in}{5.061530in}}%
\pgfusepath{clip}%
\pgfsetbuttcap%
\pgfsetroundjoin%
\pgfsetlinewidth{1.003750pt}%
\definecolor{currentstroke}{rgb}{0.000000,0.000000,0.000000}%
\pgfsetstrokecolor{currentstroke}%
\pgfsetdash{}{0pt}%
\pgfpathmoveto{\pgfqpoint{0.945316in}{0.670387in}}%
\pgfpathcurveto{\pgfqpoint{0.956212in}{0.670387in}}{\pgfqpoint{0.966663in}{0.674716in}}{\pgfqpoint{0.974367in}{0.682420in}}%
\pgfpathcurveto{\pgfqpoint{0.982072in}{0.690125in}}{\pgfqpoint{0.986400in}{0.700576in}}{\pgfqpoint{0.986400in}{0.711471in}}%
\pgfpathcurveto{\pgfqpoint{0.986400in}{0.722367in}}{\pgfqpoint{0.982072in}{0.732818in}}{\pgfqpoint{0.974367in}{0.740522in}}%
\pgfpathcurveto{\pgfqpoint{0.966663in}{0.748226in}}{\pgfqpoint{0.956212in}{0.752555in}}{\pgfqpoint{0.945316in}{0.752555in}}%
\pgfpathcurveto{\pgfqpoint{0.934421in}{0.752555in}}{\pgfqpoint{0.923970in}{0.748226in}}{\pgfqpoint{0.916266in}{0.740522in}}%
\pgfpathcurveto{\pgfqpoint{0.908561in}{0.732818in}}{\pgfqpoint{0.904233in}{0.722367in}}{\pgfqpoint{0.904233in}{0.711471in}}%
\pgfpathcurveto{\pgfqpoint{0.904233in}{0.700576in}}{\pgfqpoint{0.908561in}{0.690125in}}{\pgfqpoint{0.916266in}{0.682420in}}%
\pgfpathcurveto{\pgfqpoint{0.923970in}{0.674716in}}{\pgfqpoint{0.934421in}{0.670387in}}{\pgfqpoint{0.945316in}{0.670387in}}%
\pgfpathlineto{\pgfqpoint{0.945316in}{0.670387in}}%
\pgfpathclose%
\pgfusepath{stroke}%
\end{pgfscope}%
\begin{pgfscope}%
\pgfpathrectangle{\pgfqpoint{0.688192in}{0.670138in}}{\pgfqpoint{7.111808in}{5.061530in}}%
\pgfusepath{clip}%
\pgfsetbuttcap%
\pgfsetroundjoin%
\pgfsetlinewidth{1.003750pt}%
\definecolor{currentstroke}{rgb}{0.000000,0.000000,0.000000}%
\pgfsetstrokecolor{currentstroke}%
\pgfsetdash{}{0pt}%
\pgfpathmoveto{\pgfqpoint{4.644178in}{1.702285in}}%
\pgfpathcurveto{\pgfqpoint{4.655074in}{1.702285in}}{\pgfqpoint{4.665525in}{1.706614in}}{\pgfqpoint{4.673229in}{1.714318in}}%
\pgfpathcurveto{\pgfqpoint{4.680933in}{1.722023in}}{\pgfqpoint{4.685262in}{1.732473in}}{\pgfqpoint{4.685262in}{1.743369in}}%
\pgfpathcurveto{\pgfqpoint{4.685262in}{1.754265in}}{\pgfqpoint{4.680933in}{1.764715in}}{\pgfqpoint{4.673229in}{1.772420in}}%
\pgfpathcurveto{\pgfqpoint{4.665525in}{1.780124in}}{\pgfqpoint{4.655074in}{1.784453in}}{\pgfqpoint{4.644178in}{1.784453in}}%
\pgfpathcurveto{\pgfqpoint{4.633283in}{1.784453in}}{\pgfqpoint{4.622832in}{1.780124in}}{\pgfqpoint{4.615128in}{1.772420in}}%
\pgfpathcurveto{\pgfqpoint{4.607423in}{1.764715in}}{\pgfqpoint{4.603094in}{1.754265in}}{\pgfqpoint{4.603094in}{1.743369in}}%
\pgfpathcurveto{\pgfqpoint{4.603094in}{1.732473in}}{\pgfqpoint{4.607423in}{1.722023in}}{\pgfqpoint{4.615128in}{1.714318in}}%
\pgfpathcurveto{\pgfqpoint{4.622832in}{1.706614in}}{\pgfqpoint{4.633283in}{1.702285in}}{\pgfqpoint{4.644178in}{1.702285in}}%
\pgfpathlineto{\pgfqpoint{4.644178in}{1.702285in}}%
\pgfpathclose%
\pgfusepath{stroke}%
\end{pgfscope}%
\begin{pgfscope}%
\pgfpathrectangle{\pgfqpoint{0.688192in}{0.670138in}}{\pgfqpoint{7.111808in}{5.061530in}}%
\pgfusepath{clip}%
\pgfsetbuttcap%
\pgfsetroundjoin%
\pgfsetlinewidth{1.003750pt}%
\definecolor{currentstroke}{rgb}{0.000000,0.000000,0.000000}%
\pgfsetstrokecolor{currentstroke}%
\pgfsetdash{}{0pt}%
\pgfpathmoveto{\pgfqpoint{2.244939in}{0.638531in}}%
\pgfpathcurveto{\pgfqpoint{2.255835in}{0.638531in}}{\pgfqpoint{2.266286in}{0.642860in}}{\pgfqpoint{2.273990in}{0.650564in}}%
\pgfpathcurveto{\pgfqpoint{2.281695in}{0.658269in}}{\pgfqpoint{2.286023in}{0.668719in}}{\pgfqpoint{2.286023in}{0.679615in}}%
\pgfpathcurveto{\pgfqpoint{2.286023in}{0.690511in}}{\pgfqpoint{2.281695in}{0.700961in}}{\pgfqpoint{2.273990in}{0.708666in}}%
\pgfpathcurveto{\pgfqpoint{2.266286in}{0.716370in}}{\pgfqpoint{2.255835in}{0.720699in}}{\pgfqpoint{2.244939in}{0.720699in}}%
\pgfpathcurveto{\pgfqpoint{2.234044in}{0.720699in}}{\pgfqpoint{2.223593in}{0.716370in}}{\pgfqpoint{2.215889in}{0.708666in}}%
\pgfpathcurveto{\pgfqpoint{2.208184in}{0.700961in}}{\pgfqpoint{2.203856in}{0.690511in}}{\pgfqpoint{2.203856in}{0.679615in}}%
\pgfpathcurveto{\pgfqpoint{2.203856in}{0.668719in}}{\pgfqpoint{2.208184in}{0.658269in}}{\pgfqpoint{2.215889in}{0.650564in}}%
\pgfpathcurveto{\pgfqpoint{2.223593in}{0.642860in}}{\pgfqpoint{2.234044in}{0.638531in}}{\pgfqpoint{2.244939in}{0.638531in}}%
\pgfusepath{stroke}%
\end{pgfscope}%
\begin{pgfscope}%
\pgfpathrectangle{\pgfqpoint{0.688192in}{0.670138in}}{\pgfqpoint{7.111808in}{5.061530in}}%
\pgfusepath{clip}%
\pgfsetbuttcap%
\pgfsetroundjoin%
\pgfsetlinewidth{1.003750pt}%
\definecolor{currentstroke}{rgb}{0.000000,0.000000,0.000000}%
\pgfsetstrokecolor{currentstroke}%
\pgfsetdash{}{0pt}%
\pgfpathmoveto{\pgfqpoint{3.280946in}{0.634044in}}%
\pgfpathcurveto{\pgfqpoint{3.291842in}{0.634044in}}{\pgfqpoint{3.302293in}{0.638373in}}{\pgfqpoint{3.309997in}{0.646078in}}%
\pgfpathcurveto{\pgfqpoint{3.317701in}{0.653782in}}{\pgfqpoint{3.322030in}{0.664233in}}{\pgfqpoint{3.322030in}{0.675128in}}%
\pgfpathcurveto{\pgfqpoint{3.322030in}{0.686024in}}{\pgfqpoint{3.317701in}{0.696475in}}{\pgfqpoint{3.309997in}{0.704179in}}%
\pgfpathcurveto{\pgfqpoint{3.302293in}{0.711883in}}{\pgfqpoint{3.291842in}{0.716212in}}{\pgfqpoint{3.280946in}{0.716212in}}%
\pgfpathcurveto{\pgfqpoint{3.270051in}{0.716212in}}{\pgfqpoint{3.259600in}{0.711883in}}{\pgfqpoint{3.251896in}{0.704179in}}%
\pgfpathcurveto{\pgfqpoint{3.244191in}{0.696475in}}{\pgfqpoint{3.239862in}{0.686024in}}{\pgfqpoint{3.239862in}{0.675128in}}%
\pgfpathcurveto{\pgfqpoint{3.239862in}{0.664233in}}{\pgfqpoint{3.244191in}{0.653782in}}{\pgfqpoint{3.251896in}{0.646078in}}%
\pgfpathcurveto{\pgfqpoint{3.259600in}{0.638373in}}{\pgfqpoint{3.270051in}{0.634044in}}{\pgfqpoint{3.280946in}{0.634044in}}%
\pgfusepath{stroke}%
\end{pgfscope}%
\begin{pgfscope}%
\pgfpathrectangle{\pgfqpoint{0.688192in}{0.670138in}}{\pgfqpoint{7.111808in}{5.061530in}}%
\pgfusepath{clip}%
\pgfsetbuttcap%
\pgfsetroundjoin%
\pgfsetlinewidth{1.003750pt}%
\definecolor{currentstroke}{rgb}{0.000000,0.000000,0.000000}%
\pgfsetstrokecolor{currentstroke}%
\pgfsetdash{}{0pt}%
\pgfpathmoveto{\pgfqpoint{0.946217in}{0.669736in}}%
\pgfpathcurveto{\pgfqpoint{0.957113in}{0.669736in}}{\pgfqpoint{0.967564in}{0.674065in}}{\pgfqpoint{0.975268in}{0.681769in}}%
\pgfpathcurveto{\pgfqpoint{0.982972in}{0.689473in}}{\pgfqpoint{0.987301in}{0.699924in}}{\pgfqpoint{0.987301in}{0.710820in}}%
\pgfpathcurveto{\pgfqpoint{0.987301in}{0.721715in}}{\pgfqpoint{0.982972in}{0.732166in}}{\pgfqpoint{0.975268in}{0.739870in}}%
\pgfpathcurveto{\pgfqpoint{0.967564in}{0.747575in}}{\pgfqpoint{0.957113in}{0.751904in}}{\pgfqpoint{0.946217in}{0.751904in}}%
\pgfpathcurveto{\pgfqpoint{0.935322in}{0.751904in}}{\pgfqpoint{0.924871in}{0.747575in}}{\pgfqpoint{0.917167in}{0.739870in}}%
\pgfpathcurveto{\pgfqpoint{0.909462in}{0.732166in}}{\pgfqpoint{0.905134in}{0.721715in}}{\pgfqpoint{0.905134in}{0.710820in}}%
\pgfpathcurveto{\pgfqpoint{0.905134in}{0.699924in}}{\pgfqpoint{0.909462in}{0.689473in}}{\pgfqpoint{0.917167in}{0.681769in}}%
\pgfpathcurveto{\pgfqpoint{0.924871in}{0.674065in}}{\pgfqpoint{0.935322in}{0.669736in}}{\pgfqpoint{0.946217in}{0.669736in}}%
\pgfpathlineto{\pgfqpoint{0.946217in}{0.669736in}}%
\pgfpathclose%
\pgfusepath{stroke}%
\end{pgfscope}%
\begin{pgfscope}%
\pgfpathrectangle{\pgfqpoint{0.688192in}{0.670138in}}{\pgfqpoint{7.111808in}{5.061530in}}%
\pgfusepath{clip}%
\pgfsetbuttcap%
\pgfsetroundjoin%
\pgfsetlinewidth{1.003750pt}%
\definecolor{currentstroke}{rgb}{0.000000,0.000000,0.000000}%
\pgfsetstrokecolor{currentstroke}%
\pgfsetdash{}{0pt}%
\pgfpathmoveto{\pgfqpoint{1.334128in}{0.643434in}}%
\pgfpathcurveto{\pgfqpoint{1.345023in}{0.643434in}}{\pgfqpoint{1.355474in}{0.647762in}}{\pgfqpoint{1.363178in}{0.655467in}}%
\pgfpathcurveto{\pgfqpoint{1.370883in}{0.663171in}}{\pgfqpoint{1.375211in}{0.673622in}}{\pgfqpoint{1.375211in}{0.684517in}}%
\pgfpathcurveto{\pgfqpoint{1.375211in}{0.695413in}}{\pgfqpoint{1.370883in}{0.705864in}}{\pgfqpoint{1.363178in}{0.713568in}}%
\pgfpathcurveto{\pgfqpoint{1.355474in}{0.721272in}}{\pgfqpoint{1.345023in}{0.725601in}}{\pgfqpoint{1.334128in}{0.725601in}}%
\pgfpathcurveto{\pgfqpoint{1.323232in}{0.725601in}}{\pgfqpoint{1.312781in}{0.721272in}}{\pgfqpoint{1.305077in}{0.713568in}}%
\pgfpathcurveto{\pgfqpoint{1.297373in}{0.705864in}}{\pgfqpoint{1.293044in}{0.695413in}}{\pgfqpoint{1.293044in}{0.684517in}}%
\pgfpathcurveto{\pgfqpoint{1.293044in}{0.673622in}}{\pgfqpoint{1.297373in}{0.663171in}}{\pgfqpoint{1.305077in}{0.655467in}}%
\pgfpathcurveto{\pgfqpoint{1.312781in}{0.647762in}}{\pgfqpoint{1.323232in}{0.643434in}}{\pgfqpoint{1.334128in}{0.643434in}}%
\pgfusepath{stroke}%
\end{pgfscope}%
\begin{pgfscope}%
\pgfpathrectangle{\pgfqpoint{0.688192in}{0.670138in}}{\pgfqpoint{7.111808in}{5.061530in}}%
\pgfusepath{clip}%
\pgfsetbuttcap%
\pgfsetroundjoin%
\pgfsetlinewidth{1.003750pt}%
\definecolor{currentstroke}{rgb}{0.000000,0.000000,0.000000}%
\pgfsetstrokecolor{currentstroke}%
\pgfsetdash{}{0pt}%
\pgfpathmoveto{\pgfqpoint{4.971049in}{1.047044in}}%
\pgfpathcurveto{\pgfqpoint{4.981945in}{1.047044in}}{\pgfqpoint{4.992396in}{1.051373in}}{\pgfqpoint{5.000100in}{1.059077in}}%
\pgfpathcurveto{\pgfqpoint{5.007804in}{1.066782in}}{\pgfqpoint{5.012133in}{1.077233in}}{\pgfqpoint{5.012133in}{1.088128in}}%
\pgfpathcurveto{\pgfqpoint{5.012133in}{1.099024in}}{\pgfqpoint{5.007804in}{1.109474in}}{\pgfqpoint{5.000100in}{1.117179in}}%
\pgfpathcurveto{\pgfqpoint{4.992396in}{1.124883in}}{\pgfqpoint{4.981945in}{1.129212in}}{\pgfqpoint{4.971049in}{1.129212in}}%
\pgfpathcurveto{\pgfqpoint{4.960154in}{1.129212in}}{\pgfqpoint{4.949703in}{1.124883in}}{\pgfqpoint{4.941998in}{1.117179in}}%
\pgfpathcurveto{\pgfqpoint{4.934294in}{1.109474in}}{\pgfqpoint{4.929965in}{1.099024in}}{\pgfqpoint{4.929965in}{1.088128in}}%
\pgfpathcurveto{\pgfqpoint{4.929965in}{1.077233in}}{\pgfqpoint{4.934294in}{1.066782in}}{\pgfqpoint{4.941998in}{1.059077in}}%
\pgfpathcurveto{\pgfqpoint{4.949703in}{1.051373in}}{\pgfqpoint{4.960154in}{1.047044in}}{\pgfqpoint{4.971049in}{1.047044in}}%
\pgfpathlineto{\pgfqpoint{4.971049in}{1.047044in}}%
\pgfpathclose%
\pgfusepath{stroke}%
\end{pgfscope}%
\begin{pgfscope}%
\pgfpathrectangle{\pgfqpoint{0.688192in}{0.670138in}}{\pgfqpoint{7.111808in}{5.061530in}}%
\pgfusepath{clip}%
\pgfsetbuttcap%
\pgfsetroundjoin%
\pgfsetlinewidth{1.003750pt}%
\definecolor{currentstroke}{rgb}{0.000000,0.000000,0.000000}%
\pgfsetstrokecolor{currentstroke}%
\pgfsetdash{}{0pt}%
\pgfpathmoveto{\pgfqpoint{2.346757in}{1.590411in}}%
\pgfpathcurveto{\pgfqpoint{2.357653in}{1.590411in}}{\pgfqpoint{2.368104in}{1.594740in}}{\pgfqpoint{2.375808in}{1.602444in}}%
\pgfpathcurveto{\pgfqpoint{2.383512in}{1.610149in}}{\pgfqpoint{2.387841in}{1.620599in}}{\pgfqpoint{2.387841in}{1.631495in}}%
\pgfpathcurveto{\pgfqpoint{2.387841in}{1.642391in}}{\pgfqpoint{2.383512in}{1.652841in}}{\pgfqpoint{2.375808in}{1.660546in}}%
\pgfpathcurveto{\pgfqpoint{2.368104in}{1.668250in}}{\pgfqpoint{2.357653in}{1.672579in}}{\pgfqpoint{2.346757in}{1.672579in}}%
\pgfpathcurveto{\pgfqpoint{2.335862in}{1.672579in}}{\pgfqpoint{2.325411in}{1.668250in}}{\pgfqpoint{2.317707in}{1.660546in}}%
\pgfpathcurveto{\pgfqpoint{2.310002in}{1.652841in}}{\pgfqpoint{2.305674in}{1.642391in}}{\pgfqpoint{2.305674in}{1.631495in}}%
\pgfpathcurveto{\pgfqpoint{2.305674in}{1.620599in}}{\pgfqpoint{2.310002in}{1.610149in}}{\pgfqpoint{2.317707in}{1.602444in}}%
\pgfpathcurveto{\pgfqpoint{2.325411in}{1.594740in}}{\pgfqpoint{2.335862in}{1.590411in}}{\pgfqpoint{2.346757in}{1.590411in}}%
\pgfpathlineto{\pgfqpoint{2.346757in}{1.590411in}}%
\pgfpathclose%
\pgfusepath{stroke}%
\end{pgfscope}%
\begin{pgfscope}%
\pgfpathrectangle{\pgfqpoint{0.688192in}{0.670138in}}{\pgfqpoint{7.111808in}{5.061530in}}%
\pgfusepath{clip}%
\pgfsetbuttcap%
\pgfsetroundjoin%
\pgfsetlinewidth{1.003750pt}%
\definecolor{currentstroke}{rgb}{0.000000,0.000000,0.000000}%
\pgfsetstrokecolor{currentstroke}%
\pgfsetdash{}{0pt}%
\pgfpathmoveto{\pgfqpoint{1.495917in}{0.642463in}}%
\pgfpathcurveto{\pgfqpoint{1.506813in}{0.642463in}}{\pgfqpoint{1.517263in}{0.646792in}}{\pgfqpoint{1.524968in}{0.654497in}}%
\pgfpathcurveto{\pgfqpoint{1.532672in}{0.662201in}}{\pgfqpoint{1.537001in}{0.672652in}}{\pgfqpoint{1.537001in}{0.683547in}}%
\pgfpathcurveto{\pgfqpoint{1.537001in}{0.694443in}}{\pgfqpoint{1.532672in}{0.704894in}}{\pgfqpoint{1.524968in}{0.712598in}}%
\pgfpathcurveto{\pgfqpoint{1.517263in}{0.720302in}}{\pgfqpoint{1.506813in}{0.724631in}}{\pgfqpoint{1.495917in}{0.724631in}}%
\pgfpathcurveto{\pgfqpoint{1.485021in}{0.724631in}}{\pgfqpoint{1.474571in}{0.720302in}}{\pgfqpoint{1.466866in}{0.712598in}}%
\pgfpathcurveto{\pgfqpoint{1.459162in}{0.704894in}}{\pgfqpoint{1.454833in}{0.694443in}}{\pgfqpoint{1.454833in}{0.683547in}}%
\pgfpathcurveto{\pgfqpoint{1.454833in}{0.672652in}}{\pgfqpoint{1.459162in}{0.662201in}}{\pgfqpoint{1.466866in}{0.654497in}}%
\pgfpathcurveto{\pgfqpoint{1.474571in}{0.646792in}}{\pgfqpoint{1.485021in}{0.642463in}}{\pgfqpoint{1.495917in}{0.642463in}}%
\pgfusepath{stroke}%
\end{pgfscope}%
\begin{pgfscope}%
\pgfpathrectangle{\pgfqpoint{0.688192in}{0.670138in}}{\pgfqpoint{7.111808in}{5.061530in}}%
\pgfusepath{clip}%
\pgfsetbuttcap%
\pgfsetroundjoin%
\pgfsetlinewidth{1.003750pt}%
\definecolor{currentstroke}{rgb}{0.000000,0.000000,0.000000}%
\pgfsetstrokecolor{currentstroke}%
\pgfsetdash{}{0pt}%
\pgfpathmoveto{\pgfqpoint{1.928513in}{0.640089in}}%
\pgfpathcurveto{\pgfqpoint{1.939409in}{0.640089in}}{\pgfqpoint{1.949859in}{0.644417in}}{\pgfqpoint{1.957564in}{0.652122in}}%
\pgfpathcurveto{\pgfqpoint{1.965268in}{0.659826in}}{\pgfqpoint{1.969597in}{0.670277in}}{\pgfqpoint{1.969597in}{0.681173in}}%
\pgfpathcurveto{\pgfqpoint{1.969597in}{0.692068in}}{\pgfqpoint{1.965268in}{0.702519in}}{\pgfqpoint{1.957564in}{0.710223in}}%
\pgfpathcurveto{\pgfqpoint{1.949859in}{0.717928in}}{\pgfqpoint{1.939409in}{0.722256in}}{\pgfqpoint{1.928513in}{0.722256in}}%
\pgfpathcurveto{\pgfqpoint{1.917618in}{0.722256in}}{\pgfqpoint{1.907167in}{0.717928in}}{\pgfqpoint{1.899462in}{0.710223in}}%
\pgfpathcurveto{\pgfqpoint{1.891758in}{0.702519in}}{\pgfqpoint{1.887429in}{0.692068in}}{\pgfqpoint{1.887429in}{0.681173in}}%
\pgfpathcurveto{\pgfqpoint{1.887429in}{0.670277in}}{\pgfqpoint{1.891758in}{0.659826in}}{\pgfqpoint{1.899462in}{0.652122in}}%
\pgfpathcurveto{\pgfqpoint{1.907167in}{0.644417in}}{\pgfqpoint{1.917618in}{0.640089in}}{\pgfqpoint{1.928513in}{0.640089in}}%
\pgfusepath{stroke}%
\end{pgfscope}%
\begin{pgfscope}%
\pgfpathrectangle{\pgfqpoint{0.688192in}{0.670138in}}{\pgfqpoint{7.111808in}{5.061530in}}%
\pgfusepath{clip}%
\pgfsetbuttcap%
\pgfsetroundjoin%
\pgfsetlinewidth{1.003750pt}%
\definecolor{currentstroke}{rgb}{0.000000,0.000000,0.000000}%
\pgfsetstrokecolor{currentstroke}%
\pgfsetdash{}{0pt}%
\pgfpathmoveto{\pgfqpoint{2.244939in}{0.638531in}}%
\pgfpathcurveto{\pgfqpoint{2.255835in}{0.638531in}}{\pgfqpoint{2.266286in}{0.642860in}}{\pgfqpoint{2.273990in}{0.650564in}}%
\pgfpathcurveto{\pgfqpoint{2.281695in}{0.658269in}}{\pgfqpoint{2.286023in}{0.668719in}}{\pgfqpoint{2.286023in}{0.679615in}}%
\pgfpathcurveto{\pgfqpoint{2.286023in}{0.690511in}}{\pgfqpoint{2.281695in}{0.700961in}}{\pgfqpoint{2.273990in}{0.708666in}}%
\pgfpathcurveto{\pgfqpoint{2.266286in}{0.716370in}}{\pgfqpoint{2.255835in}{0.720699in}}{\pgfqpoint{2.244939in}{0.720699in}}%
\pgfpathcurveto{\pgfqpoint{2.234044in}{0.720699in}}{\pgfqpoint{2.223593in}{0.716370in}}{\pgfqpoint{2.215889in}{0.708666in}}%
\pgfpathcurveto{\pgfqpoint{2.208184in}{0.700961in}}{\pgfqpoint{2.203856in}{0.690511in}}{\pgfqpoint{2.203856in}{0.679615in}}%
\pgfpathcurveto{\pgfqpoint{2.203856in}{0.668719in}}{\pgfqpoint{2.208184in}{0.658269in}}{\pgfqpoint{2.215889in}{0.650564in}}%
\pgfpathcurveto{\pgfqpoint{2.223593in}{0.642860in}}{\pgfqpoint{2.234044in}{0.638531in}}{\pgfqpoint{2.244939in}{0.638531in}}%
\pgfusepath{stroke}%
\end{pgfscope}%
\begin{pgfscope}%
\pgfpathrectangle{\pgfqpoint{0.688192in}{0.670138in}}{\pgfqpoint{7.111808in}{5.061530in}}%
\pgfusepath{clip}%
\pgfsetbuttcap%
\pgfsetroundjoin%
\pgfsetlinewidth{1.003750pt}%
\definecolor{currentstroke}{rgb}{0.000000,0.000000,0.000000}%
\pgfsetstrokecolor{currentstroke}%
\pgfsetdash{}{0pt}%
\pgfpathmoveto{\pgfqpoint{0.945316in}{0.670387in}}%
\pgfpathcurveto{\pgfqpoint{0.956212in}{0.670387in}}{\pgfqpoint{0.966663in}{0.674716in}}{\pgfqpoint{0.974367in}{0.682420in}}%
\pgfpathcurveto{\pgfqpoint{0.982072in}{0.690125in}}{\pgfqpoint{0.986400in}{0.700576in}}{\pgfqpoint{0.986400in}{0.711471in}}%
\pgfpathcurveto{\pgfqpoint{0.986400in}{0.722367in}}{\pgfqpoint{0.982072in}{0.732818in}}{\pgfqpoint{0.974367in}{0.740522in}}%
\pgfpathcurveto{\pgfqpoint{0.966663in}{0.748226in}}{\pgfqpoint{0.956212in}{0.752555in}}{\pgfqpoint{0.945316in}{0.752555in}}%
\pgfpathcurveto{\pgfqpoint{0.934421in}{0.752555in}}{\pgfqpoint{0.923970in}{0.748226in}}{\pgfqpoint{0.916266in}{0.740522in}}%
\pgfpathcurveto{\pgfqpoint{0.908561in}{0.732818in}}{\pgfqpoint{0.904233in}{0.722367in}}{\pgfqpoint{0.904233in}{0.711471in}}%
\pgfpathcurveto{\pgfqpoint{0.904233in}{0.700576in}}{\pgfqpoint{0.908561in}{0.690125in}}{\pgfqpoint{0.916266in}{0.682420in}}%
\pgfpathcurveto{\pgfqpoint{0.923970in}{0.674716in}}{\pgfqpoint{0.934421in}{0.670387in}}{\pgfqpoint{0.945316in}{0.670387in}}%
\pgfpathlineto{\pgfqpoint{0.945316in}{0.670387in}}%
\pgfpathclose%
\pgfusepath{stroke}%
\end{pgfscope}%
\begin{pgfscope}%
\pgfpathrectangle{\pgfqpoint{0.688192in}{0.670138in}}{\pgfqpoint{7.111808in}{5.061530in}}%
\pgfusepath{clip}%
\pgfsetbuttcap%
\pgfsetroundjoin%
\pgfsetlinewidth{1.003750pt}%
\definecolor{currentstroke}{rgb}{0.000000,0.000000,0.000000}%
\pgfsetstrokecolor{currentstroke}%
\pgfsetdash{}{0pt}%
\pgfpathmoveto{\pgfqpoint{3.005363in}{2.932506in}}%
\pgfpathcurveto{\pgfqpoint{3.016258in}{2.932506in}}{\pgfqpoint{3.026709in}{2.936835in}}{\pgfqpoint{3.034413in}{2.944540in}}%
\pgfpathcurveto{\pgfqpoint{3.042118in}{2.952244in}}{\pgfqpoint{3.046447in}{2.962695in}}{\pgfqpoint{3.046447in}{2.973590in}}%
\pgfpathcurveto{\pgfqpoint{3.046447in}{2.984486in}}{\pgfqpoint{3.042118in}{2.994937in}}{\pgfqpoint{3.034413in}{3.002641in}}%
\pgfpathcurveto{\pgfqpoint{3.026709in}{3.010345in}}{\pgfqpoint{3.016258in}{3.014674in}}{\pgfqpoint{3.005363in}{3.014674in}}%
\pgfpathcurveto{\pgfqpoint{2.994467in}{3.014674in}}{\pgfqpoint{2.984016in}{3.010345in}}{\pgfqpoint{2.976312in}{3.002641in}}%
\pgfpathcurveto{\pgfqpoint{2.968608in}{2.994937in}}{\pgfqpoint{2.964279in}{2.984486in}}{\pgfqpoint{2.964279in}{2.973590in}}%
\pgfpathcurveto{\pgfqpoint{2.964279in}{2.962695in}}{\pgfqpoint{2.968608in}{2.952244in}}{\pgfqpoint{2.976312in}{2.944540in}}%
\pgfpathcurveto{\pgfqpoint{2.984016in}{2.936835in}}{\pgfqpoint{2.994467in}{2.932506in}}{\pgfqpoint{3.005363in}{2.932506in}}%
\pgfpathlineto{\pgfqpoint{3.005363in}{2.932506in}}%
\pgfpathclose%
\pgfusepath{stroke}%
\end{pgfscope}%
\begin{pgfscope}%
\pgfpathrectangle{\pgfqpoint{0.688192in}{0.670138in}}{\pgfqpoint{7.111808in}{5.061530in}}%
\pgfusepath{clip}%
\pgfsetbuttcap%
\pgfsetroundjoin%
\pgfsetlinewidth{1.003750pt}%
\definecolor{currentstroke}{rgb}{0.000000,0.000000,0.000000}%
\pgfsetstrokecolor{currentstroke}%
\pgfsetdash{}{0pt}%
\pgfpathmoveto{\pgfqpoint{0.781884in}{0.847588in}}%
\pgfpathcurveto{\pgfqpoint{0.792780in}{0.847588in}}{\pgfqpoint{0.803230in}{0.851917in}}{\pgfqpoint{0.810935in}{0.859621in}}%
\pgfpathcurveto{\pgfqpoint{0.818639in}{0.867326in}}{\pgfqpoint{0.822968in}{0.877776in}}{\pgfqpoint{0.822968in}{0.888672in}}%
\pgfpathcurveto{\pgfqpoint{0.822968in}{0.899568in}}{\pgfqpoint{0.818639in}{0.910018in}}{\pgfqpoint{0.810935in}{0.917723in}}%
\pgfpathcurveto{\pgfqpoint{0.803230in}{0.925427in}}{\pgfqpoint{0.792780in}{0.929756in}}{\pgfqpoint{0.781884in}{0.929756in}}%
\pgfpathcurveto{\pgfqpoint{0.770989in}{0.929756in}}{\pgfqpoint{0.760538in}{0.925427in}}{\pgfqpoint{0.752833in}{0.917723in}}%
\pgfpathcurveto{\pgfqpoint{0.745129in}{0.910018in}}{\pgfqpoint{0.740800in}{0.899568in}}{\pgfqpoint{0.740800in}{0.888672in}}%
\pgfpathcurveto{\pgfqpoint{0.740800in}{0.877776in}}{\pgfqpoint{0.745129in}{0.867326in}}{\pgfqpoint{0.752833in}{0.859621in}}%
\pgfpathcurveto{\pgfqpoint{0.760538in}{0.851917in}}{\pgfqpoint{0.770989in}{0.847588in}}{\pgfqpoint{0.781884in}{0.847588in}}%
\pgfpathlineto{\pgfqpoint{0.781884in}{0.847588in}}%
\pgfpathclose%
\pgfusepath{stroke}%
\end{pgfscope}%
\begin{pgfscope}%
\pgfpathrectangle{\pgfqpoint{0.688192in}{0.670138in}}{\pgfqpoint{7.111808in}{5.061530in}}%
\pgfusepath{clip}%
\pgfsetbuttcap%
\pgfsetroundjoin%
\pgfsetlinewidth{1.003750pt}%
\definecolor{currentstroke}{rgb}{0.000000,0.000000,0.000000}%
\pgfsetstrokecolor{currentstroke}%
\pgfsetdash{}{0pt}%
\pgfpathmoveto{\pgfqpoint{0.939729in}{0.677113in}}%
\pgfpathcurveto{\pgfqpoint{0.950625in}{0.677113in}}{\pgfqpoint{0.961075in}{0.681442in}}{\pgfqpoint{0.968780in}{0.689146in}}%
\pgfpathcurveto{\pgfqpoint{0.976484in}{0.696850in}}{\pgfqpoint{0.980813in}{0.707301in}}{\pgfqpoint{0.980813in}{0.718197in}}%
\pgfpathcurveto{\pgfqpoint{0.980813in}{0.729092in}}{\pgfqpoint{0.976484in}{0.739543in}}{\pgfqpoint{0.968780in}{0.747247in}}%
\pgfpathcurveto{\pgfqpoint{0.961075in}{0.754952in}}{\pgfqpoint{0.950625in}{0.759281in}}{\pgfqpoint{0.939729in}{0.759281in}}%
\pgfpathcurveto{\pgfqpoint{0.928834in}{0.759281in}}{\pgfqpoint{0.918383in}{0.754952in}}{\pgfqpoint{0.910678in}{0.747247in}}%
\pgfpathcurveto{\pgfqpoint{0.902974in}{0.739543in}}{\pgfqpoint{0.898645in}{0.729092in}}{\pgfqpoint{0.898645in}{0.718197in}}%
\pgfpathcurveto{\pgfqpoint{0.898645in}{0.707301in}}{\pgfqpoint{0.902974in}{0.696850in}}{\pgfqpoint{0.910678in}{0.689146in}}%
\pgfpathcurveto{\pgfqpoint{0.918383in}{0.681442in}}{\pgfqpoint{0.928834in}{0.677113in}}{\pgfqpoint{0.939729in}{0.677113in}}%
\pgfpathlineto{\pgfqpoint{0.939729in}{0.677113in}}%
\pgfpathclose%
\pgfusepath{stroke}%
\end{pgfscope}%
\begin{pgfscope}%
\pgfpathrectangle{\pgfqpoint{0.688192in}{0.670138in}}{\pgfqpoint{7.111808in}{5.061530in}}%
\pgfusepath{clip}%
\pgfsetbuttcap%
\pgfsetroundjoin%
\pgfsetlinewidth{1.003750pt}%
\definecolor{currentstroke}{rgb}{0.000000,0.000000,0.000000}%
\pgfsetstrokecolor{currentstroke}%
\pgfsetdash{}{0pt}%
\pgfpathmoveto{\pgfqpoint{3.593895in}{4.545535in}}%
\pgfpathcurveto{\pgfqpoint{3.604791in}{4.545535in}}{\pgfqpoint{3.615242in}{4.549864in}}{\pgfqpoint{3.622946in}{4.557569in}}%
\pgfpathcurveto{\pgfqpoint{3.630650in}{4.565273in}}{\pgfqpoint{3.634979in}{4.575724in}}{\pgfqpoint{3.634979in}{4.586619in}}%
\pgfpathcurveto{\pgfqpoint{3.634979in}{4.597515in}}{\pgfqpoint{3.630650in}{4.607966in}}{\pgfqpoint{3.622946in}{4.615670in}}%
\pgfpathcurveto{\pgfqpoint{3.615242in}{4.623374in}}{\pgfqpoint{3.604791in}{4.627703in}}{\pgfqpoint{3.593895in}{4.627703in}}%
\pgfpathcurveto{\pgfqpoint{3.583000in}{4.627703in}}{\pgfqpoint{3.572549in}{4.623374in}}{\pgfqpoint{3.564844in}{4.615670in}}%
\pgfpathcurveto{\pgfqpoint{3.557140in}{4.607966in}}{\pgfqpoint{3.552811in}{4.597515in}}{\pgfqpoint{3.552811in}{4.586619in}}%
\pgfpathcurveto{\pgfqpoint{3.552811in}{4.575724in}}{\pgfqpoint{3.557140in}{4.565273in}}{\pgfqpoint{3.564844in}{4.557569in}}%
\pgfpathcurveto{\pgfqpoint{3.572549in}{4.549864in}}{\pgfqpoint{3.583000in}{4.545535in}}{\pgfqpoint{3.593895in}{4.545535in}}%
\pgfpathlineto{\pgfqpoint{3.593895in}{4.545535in}}%
\pgfpathclose%
\pgfusepath{stroke}%
\end{pgfscope}%
\begin{pgfscope}%
\pgfpathrectangle{\pgfqpoint{0.688192in}{0.670138in}}{\pgfqpoint{7.111808in}{5.061530in}}%
\pgfusepath{clip}%
\pgfsetbuttcap%
\pgfsetroundjoin%
\pgfsetlinewidth{1.003750pt}%
\definecolor{currentstroke}{rgb}{0.000000,0.000000,0.000000}%
\pgfsetstrokecolor{currentstroke}%
\pgfsetdash{}{0pt}%
\pgfpathmoveto{\pgfqpoint{2.864031in}{0.664569in}}%
\pgfpathcurveto{\pgfqpoint{2.874926in}{0.664569in}}{\pgfqpoint{2.885377in}{0.668898in}}{\pgfqpoint{2.893081in}{0.676602in}}%
\pgfpathcurveto{\pgfqpoint{2.900786in}{0.684306in}}{\pgfqpoint{2.905115in}{0.694757in}}{\pgfqpoint{2.905115in}{0.705653in}}%
\pgfpathcurveto{\pgfqpoint{2.905115in}{0.716548in}}{\pgfqpoint{2.900786in}{0.726999in}}{\pgfqpoint{2.893081in}{0.734703in}}%
\pgfpathcurveto{\pgfqpoint{2.885377in}{0.742408in}}{\pgfqpoint{2.874926in}{0.746736in}}{\pgfqpoint{2.864031in}{0.746736in}}%
\pgfpathcurveto{\pgfqpoint{2.853135in}{0.746736in}}{\pgfqpoint{2.842684in}{0.742408in}}{\pgfqpoint{2.834980in}{0.734703in}}%
\pgfpathcurveto{\pgfqpoint{2.827276in}{0.726999in}}{\pgfqpoint{2.822947in}{0.716548in}}{\pgfqpoint{2.822947in}{0.705653in}}%
\pgfpathcurveto{\pgfqpoint{2.822947in}{0.694757in}}{\pgfqpoint{2.827276in}{0.684306in}}{\pgfqpoint{2.834980in}{0.676602in}}%
\pgfpathcurveto{\pgfqpoint{2.842684in}{0.668898in}}{\pgfqpoint{2.853135in}{0.664569in}}{\pgfqpoint{2.864031in}{0.664569in}}%
\pgfusepath{stroke}%
\end{pgfscope}%
\begin{pgfscope}%
\pgfpathrectangle{\pgfqpoint{0.688192in}{0.670138in}}{\pgfqpoint{7.111808in}{5.061530in}}%
\pgfusepath{clip}%
\pgfsetbuttcap%
\pgfsetroundjoin%
\pgfsetlinewidth{1.003750pt}%
\definecolor{currentstroke}{rgb}{0.000000,0.000000,0.000000}%
\pgfsetstrokecolor{currentstroke}%
\pgfsetdash{}{0pt}%
\pgfpathmoveto{\pgfqpoint{0.849953in}{0.698273in}}%
\pgfpathcurveto{\pgfqpoint{0.860849in}{0.698273in}}{\pgfqpoint{0.871300in}{0.702602in}}{\pgfqpoint{0.879004in}{0.710307in}}%
\pgfpathcurveto{\pgfqpoint{0.886709in}{0.718011in}}{\pgfqpoint{0.891037in}{0.728462in}}{\pgfqpoint{0.891037in}{0.739357in}}%
\pgfpathcurveto{\pgfqpoint{0.891037in}{0.750253in}}{\pgfqpoint{0.886709in}{0.760704in}}{\pgfqpoint{0.879004in}{0.768408in}}%
\pgfpathcurveto{\pgfqpoint{0.871300in}{0.776112in}}{\pgfqpoint{0.860849in}{0.780441in}}{\pgfqpoint{0.849953in}{0.780441in}}%
\pgfpathcurveto{\pgfqpoint{0.839058in}{0.780441in}}{\pgfqpoint{0.828607in}{0.776112in}}{\pgfqpoint{0.820903in}{0.768408in}}%
\pgfpathcurveto{\pgfqpoint{0.813198in}{0.760704in}}{\pgfqpoint{0.808870in}{0.750253in}}{\pgfqpoint{0.808870in}{0.739357in}}%
\pgfpathcurveto{\pgfqpoint{0.808870in}{0.728462in}}{\pgfqpoint{0.813198in}{0.718011in}}{\pgfqpoint{0.820903in}{0.710307in}}%
\pgfpathcurveto{\pgfqpoint{0.828607in}{0.702602in}}{\pgfqpoint{0.839058in}{0.698273in}}{\pgfqpoint{0.849953in}{0.698273in}}%
\pgfpathlineto{\pgfqpoint{0.849953in}{0.698273in}}%
\pgfpathclose%
\pgfusepath{stroke}%
\end{pgfscope}%
\begin{pgfscope}%
\pgfpathrectangle{\pgfqpoint{0.688192in}{0.670138in}}{\pgfqpoint{7.111808in}{5.061530in}}%
\pgfusepath{clip}%
\pgfsetbuttcap%
\pgfsetroundjoin%
\pgfsetlinewidth{1.003750pt}%
\definecolor{currentstroke}{rgb}{0.000000,0.000000,0.000000}%
\pgfsetstrokecolor{currentstroke}%
\pgfsetdash{}{0pt}%
\pgfpathmoveto{\pgfqpoint{6.574362in}{1.061159in}}%
\pgfpathcurveto{\pgfqpoint{6.585258in}{1.061159in}}{\pgfqpoint{6.595709in}{1.065488in}}{\pgfqpoint{6.603413in}{1.073192in}}%
\pgfpathcurveto{\pgfqpoint{6.611117in}{1.080896in}}{\pgfqpoint{6.615446in}{1.091347in}}{\pgfqpoint{6.615446in}{1.102243in}}%
\pgfpathcurveto{\pgfqpoint{6.615446in}{1.113138in}}{\pgfqpoint{6.611117in}{1.123589in}}{\pgfqpoint{6.603413in}{1.131293in}}%
\pgfpathcurveto{\pgfqpoint{6.595709in}{1.138998in}}{\pgfqpoint{6.585258in}{1.143327in}}{\pgfqpoint{6.574362in}{1.143327in}}%
\pgfpathcurveto{\pgfqpoint{6.563467in}{1.143327in}}{\pgfqpoint{6.553016in}{1.138998in}}{\pgfqpoint{6.545311in}{1.131293in}}%
\pgfpathcurveto{\pgfqpoint{6.537607in}{1.123589in}}{\pgfqpoint{6.533278in}{1.113138in}}{\pgfqpoint{6.533278in}{1.102243in}}%
\pgfpathcurveto{\pgfqpoint{6.533278in}{1.091347in}}{\pgfqpoint{6.537607in}{1.080896in}}{\pgfqpoint{6.545311in}{1.073192in}}%
\pgfpathcurveto{\pgfqpoint{6.553016in}{1.065488in}}{\pgfqpoint{6.563467in}{1.061159in}}{\pgfqpoint{6.574362in}{1.061159in}}%
\pgfpathlineto{\pgfqpoint{6.574362in}{1.061159in}}%
\pgfpathclose%
\pgfusepath{stroke}%
\end{pgfscope}%
\begin{pgfscope}%
\pgfpathrectangle{\pgfqpoint{0.688192in}{0.670138in}}{\pgfqpoint{7.111808in}{5.061530in}}%
\pgfusepath{clip}%
\pgfsetbuttcap%
\pgfsetroundjoin%
\pgfsetlinewidth{1.003750pt}%
\definecolor{currentstroke}{rgb}{0.000000,0.000000,0.000000}%
\pgfsetstrokecolor{currentstroke}%
\pgfsetdash{}{0pt}%
\pgfpathmoveto{\pgfqpoint{0.754090in}{1.113315in}}%
\pgfpathcurveto{\pgfqpoint{0.764986in}{1.113315in}}{\pgfqpoint{0.775436in}{1.117644in}}{\pgfqpoint{0.783141in}{1.125348in}}%
\pgfpathcurveto{\pgfqpoint{0.790845in}{1.133052in}}{\pgfqpoint{0.795174in}{1.143503in}}{\pgfqpoint{0.795174in}{1.154399in}}%
\pgfpathcurveto{\pgfqpoint{0.795174in}{1.165294in}}{\pgfqpoint{0.790845in}{1.175745in}}{\pgfqpoint{0.783141in}{1.183450in}}%
\pgfpathcurveto{\pgfqpoint{0.775436in}{1.191154in}}{\pgfqpoint{0.764986in}{1.195483in}}{\pgfqpoint{0.754090in}{1.195483in}}%
\pgfpathcurveto{\pgfqpoint{0.743194in}{1.195483in}}{\pgfqpoint{0.732744in}{1.191154in}}{\pgfqpoint{0.725039in}{1.183450in}}%
\pgfpathcurveto{\pgfqpoint{0.717335in}{1.175745in}}{\pgfqpoint{0.713006in}{1.165294in}}{\pgfqpoint{0.713006in}{1.154399in}}%
\pgfpathcurveto{\pgfqpoint{0.713006in}{1.143503in}}{\pgfqpoint{0.717335in}{1.133052in}}{\pgfqpoint{0.725039in}{1.125348in}}%
\pgfpathcurveto{\pgfqpoint{0.732744in}{1.117644in}}{\pgfqpoint{0.743194in}{1.113315in}}{\pgfqpoint{0.754090in}{1.113315in}}%
\pgfpathlineto{\pgfqpoint{0.754090in}{1.113315in}}%
\pgfpathclose%
\pgfusepath{stroke}%
\end{pgfscope}%
\begin{pgfscope}%
\pgfpathrectangle{\pgfqpoint{0.688192in}{0.670138in}}{\pgfqpoint{7.111808in}{5.061530in}}%
\pgfusepath{clip}%
\pgfsetbuttcap%
\pgfsetroundjoin%
\pgfsetlinewidth{1.003750pt}%
\definecolor{currentstroke}{rgb}{0.000000,0.000000,0.000000}%
\pgfsetstrokecolor{currentstroke}%
\pgfsetdash{}{0pt}%
\pgfpathmoveto{\pgfqpoint{5.658790in}{1.473202in}}%
\pgfpathcurveto{\pgfqpoint{5.669686in}{1.473202in}}{\pgfqpoint{5.680137in}{1.477531in}}{\pgfqpoint{5.687841in}{1.485235in}}%
\pgfpathcurveto{\pgfqpoint{5.695545in}{1.492940in}}{\pgfqpoint{5.699874in}{1.503390in}}{\pgfqpoint{5.699874in}{1.514286in}}%
\pgfpathcurveto{\pgfqpoint{5.699874in}{1.525182in}}{\pgfqpoint{5.695545in}{1.535632in}}{\pgfqpoint{5.687841in}{1.543337in}}%
\pgfpathcurveto{\pgfqpoint{5.680137in}{1.551041in}}{\pgfqpoint{5.669686in}{1.555370in}}{\pgfqpoint{5.658790in}{1.555370in}}%
\pgfpathcurveto{\pgfqpoint{5.647895in}{1.555370in}}{\pgfqpoint{5.637444in}{1.551041in}}{\pgfqpoint{5.629740in}{1.543337in}}%
\pgfpathcurveto{\pgfqpoint{5.622035in}{1.535632in}}{\pgfqpoint{5.617707in}{1.525182in}}{\pgfqpoint{5.617707in}{1.514286in}}%
\pgfpathcurveto{\pgfqpoint{5.617707in}{1.503390in}}{\pgfqpoint{5.622035in}{1.492940in}}{\pgfqpoint{5.629740in}{1.485235in}}%
\pgfpathcurveto{\pgfqpoint{5.637444in}{1.477531in}}{\pgfqpoint{5.647895in}{1.473202in}}{\pgfqpoint{5.658790in}{1.473202in}}%
\pgfpathlineto{\pgfqpoint{5.658790in}{1.473202in}}%
\pgfpathclose%
\pgfusepath{stroke}%
\end{pgfscope}%
\begin{pgfscope}%
\pgfpathrectangle{\pgfqpoint{0.688192in}{0.670138in}}{\pgfqpoint{7.111808in}{5.061530in}}%
\pgfusepath{clip}%
\pgfsetbuttcap%
\pgfsetroundjoin%
\pgfsetlinewidth{1.003750pt}%
\definecolor{currentstroke}{rgb}{0.000000,0.000000,0.000000}%
\pgfsetstrokecolor{currentstroke}%
\pgfsetdash{}{0pt}%
\pgfpathmoveto{\pgfqpoint{6.128769in}{3.921881in}}%
\pgfpathcurveto{\pgfqpoint{6.139665in}{3.921881in}}{\pgfqpoint{6.150115in}{3.926210in}}{\pgfqpoint{6.157820in}{3.933914in}}%
\pgfpathcurveto{\pgfqpoint{6.165524in}{3.941619in}}{\pgfqpoint{6.169853in}{3.952069in}}{\pgfqpoint{6.169853in}{3.962965in}}%
\pgfpathcurveto{\pgfqpoint{6.169853in}{3.973861in}}{\pgfqpoint{6.165524in}{3.984311in}}{\pgfqpoint{6.157820in}{3.992016in}}%
\pgfpathcurveto{\pgfqpoint{6.150115in}{3.999720in}}{\pgfqpoint{6.139665in}{4.004049in}}{\pgfqpoint{6.128769in}{4.004049in}}%
\pgfpathcurveto{\pgfqpoint{6.117873in}{4.004049in}}{\pgfqpoint{6.107423in}{3.999720in}}{\pgfqpoint{6.099718in}{3.992016in}}%
\pgfpathcurveto{\pgfqpoint{6.092014in}{3.984311in}}{\pgfqpoint{6.087685in}{3.973861in}}{\pgfqpoint{6.087685in}{3.962965in}}%
\pgfpathcurveto{\pgfqpoint{6.087685in}{3.952069in}}{\pgfqpoint{6.092014in}{3.941619in}}{\pgfqpoint{6.099718in}{3.933914in}}%
\pgfpathcurveto{\pgfqpoint{6.107423in}{3.926210in}}{\pgfqpoint{6.117873in}{3.921881in}}{\pgfqpoint{6.128769in}{3.921881in}}%
\pgfpathlineto{\pgfqpoint{6.128769in}{3.921881in}}%
\pgfpathclose%
\pgfusepath{stroke}%
\end{pgfscope}%
\begin{pgfscope}%
\pgfpathrectangle{\pgfqpoint{0.688192in}{0.670138in}}{\pgfqpoint{7.111808in}{5.061530in}}%
\pgfusepath{clip}%
\pgfsetbuttcap%
\pgfsetroundjoin%
\pgfsetlinewidth{1.003750pt}%
\definecolor{currentstroke}{rgb}{0.000000,0.000000,0.000000}%
\pgfsetstrokecolor{currentstroke}%
\pgfsetdash{}{0pt}%
\pgfpathmoveto{\pgfqpoint{1.334128in}{0.643434in}}%
\pgfpathcurveto{\pgfqpoint{1.345023in}{0.643434in}}{\pgfqpoint{1.355474in}{0.647762in}}{\pgfqpoint{1.363178in}{0.655467in}}%
\pgfpathcurveto{\pgfqpoint{1.370883in}{0.663171in}}{\pgfqpoint{1.375211in}{0.673622in}}{\pgfqpoint{1.375211in}{0.684517in}}%
\pgfpathcurveto{\pgfqpoint{1.375211in}{0.695413in}}{\pgfqpoint{1.370883in}{0.705864in}}{\pgfqpoint{1.363178in}{0.713568in}}%
\pgfpathcurveto{\pgfqpoint{1.355474in}{0.721272in}}{\pgfqpoint{1.345023in}{0.725601in}}{\pgfqpoint{1.334128in}{0.725601in}}%
\pgfpathcurveto{\pgfqpoint{1.323232in}{0.725601in}}{\pgfqpoint{1.312781in}{0.721272in}}{\pgfqpoint{1.305077in}{0.713568in}}%
\pgfpathcurveto{\pgfqpoint{1.297373in}{0.705864in}}{\pgfqpoint{1.293044in}{0.695413in}}{\pgfqpoint{1.293044in}{0.684517in}}%
\pgfpathcurveto{\pgfqpoint{1.293044in}{0.673622in}}{\pgfqpoint{1.297373in}{0.663171in}}{\pgfqpoint{1.305077in}{0.655467in}}%
\pgfpathcurveto{\pgfqpoint{1.312781in}{0.647762in}}{\pgfqpoint{1.323232in}{0.643434in}}{\pgfqpoint{1.334128in}{0.643434in}}%
\pgfusepath{stroke}%
\end{pgfscope}%
\begin{pgfscope}%
\pgfpathrectangle{\pgfqpoint{0.688192in}{0.670138in}}{\pgfqpoint{7.111808in}{5.061530in}}%
\pgfusepath{clip}%
\pgfsetbuttcap%
\pgfsetroundjoin%
\pgfsetlinewidth{1.003750pt}%
\definecolor{currentstroke}{rgb}{0.000000,0.000000,0.000000}%
\pgfsetstrokecolor{currentstroke}%
\pgfsetdash{}{0pt}%
\pgfpathmoveto{\pgfqpoint{7.054099in}{2.908145in}}%
\pgfpathcurveto{\pgfqpoint{7.064994in}{2.908145in}}{\pgfqpoint{7.075445in}{2.912474in}}{\pgfqpoint{7.083149in}{2.920178in}}%
\pgfpathcurveto{\pgfqpoint{7.090854in}{2.927883in}}{\pgfqpoint{7.095183in}{2.938333in}}{\pgfqpoint{7.095183in}{2.949229in}}%
\pgfpathcurveto{\pgfqpoint{7.095183in}{2.960125in}}{\pgfqpoint{7.090854in}{2.970575in}}{\pgfqpoint{7.083149in}{2.978280in}}%
\pgfpathcurveto{\pgfqpoint{7.075445in}{2.985984in}}{\pgfqpoint{7.064994in}{2.990313in}}{\pgfqpoint{7.054099in}{2.990313in}}%
\pgfpathcurveto{\pgfqpoint{7.043203in}{2.990313in}}{\pgfqpoint{7.032752in}{2.985984in}}{\pgfqpoint{7.025048in}{2.978280in}}%
\pgfpathcurveto{\pgfqpoint{7.017344in}{2.970575in}}{\pgfqpoint{7.013015in}{2.960125in}}{\pgfqpoint{7.013015in}{2.949229in}}%
\pgfpathcurveto{\pgfqpoint{7.013015in}{2.938333in}}{\pgfqpoint{7.017344in}{2.927883in}}{\pgfqpoint{7.025048in}{2.920178in}}%
\pgfpathcurveto{\pgfqpoint{7.032752in}{2.912474in}}{\pgfqpoint{7.043203in}{2.908145in}}{\pgfqpoint{7.054099in}{2.908145in}}%
\pgfpathlineto{\pgfqpoint{7.054099in}{2.908145in}}%
\pgfpathclose%
\pgfusepath{stroke}%
\end{pgfscope}%
\begin{pgfscope}%
\pgfpathrectangle{\pgfqpoint{0.688192in}{0.670138in}}{\pgfqpoint{7.111808in}{5.061530in}}%
\pgfusepath{clip}%
\pgfsetbuttcap%
\pgfsetroundjoin%
\pgfsetlinewidth{1.003750pt}%
\definecolor{currentstroke}{rgb}{0.000000,0.000000,0.000000}%
\pgfsetstrokecolor{currentstroke}%
\pgfsetdash{}{0pt}%
\pgfpathmoveto{\pgfqpoint{6.063101in}{1.389347in}}%
\pgfpathcurveto{\pgfqpoint{6.073997in}{1.389347in}}{\pgfqpoint{6.084448in}{1.393675in}}{\pgfqpoint{6.092152in}{1.401380in}}%
\pgfpathcurveto{\pgfqpoint{6.099856in}{1.409084in}}{\pgfqpoint{6.104185in}{1.419535in}}{\pgfqpoint{6.104185in}{1.430431in}}%
\pgfpathcurveto{\pgfqpoint{6.104185in}{1.441326in}}{\pgfqpoint{6.099856in}{1.451777in}}{\pgfqpoint{6.092152in}{1.459481in}}%
\pgfpathcurveto{\pgfqpoint{6.084448in}{1.467186in}}{\pgfqpoint{6.073997in}{1.471514in}}{\pgfqpoint{6.063101in}{1.471514in}}%
\pgfpathcurveto{\pgfqpoint{6.052206in}{1.471514in}}{\pgfqpoint{6.041755in}{1.467186in}}{\pgfqpoint{6.034051in}{1.459481in}}%
\pgfpathcurveto{\pgfqpoint{6.026346in}{1.451777in}}{\pgfqpoint{6.022018in}{1.441326in}}{\pgfqpoint{6.022018in}{1.430431in}}%
\pgfpathcurveto{\pgfqpoint{6.022018in}{1.419535in}}{\pgfqpoint{6.026346in}{1.409084in}}{\pgfqpoint{6.034051in}{1.401380in}}%
\pgfpathcurveto{\pgfqpoint{6.041755in}{1.393675in}}{\pgfqpoint{6.052206in}{1.389347in}}{\pgfqpoint{6.063101in}{1.389347in}}%
\pgfpathlineto{\pgfqpoint{6.063101in}{1.389347in}}%
\pgfpathclose%
\pgfusepath{stroke}%
\end{pgfscope}%
\begin{pgfscope}%
\pgfpathrectangle{\pgfqpoint{0.688192in}{0.670138in}}{\pgfqpoint{7.111808in}{5.061530in}}%
\pgfusepath{clip}%
\pgfsetbuttcap%
\pgfsetroundjoin%
\pgfsetlinewidth{1.003750pt}%
\definecolor{currentstroke}{rgb}{0.000000,0.000000,0.000000}%
\pgfsetstrokecolor{currentstroke}%
\pgfsetdash{}{0pt}%
\pgfpathmoveto{\pgfqpoint{1.164071in}{0.644502in}}%
\pgfpathcurveto{\pgfqpoint{1.174967in}{0.644502in}}{\pgfqpoint{1.185418in}{0.648831in}}{\pgfqpoint{1.193122in}{0.656535in}}%
\pgfpathcurveto{\pgfqpoint{1.200826in}{0.664239in}}{\pgfqpoint{1.205155in}{0.674690in}}{\pgfqpoint{1.205155in}{0.685586in}}%
\pgfpathcurveto{\pgfqpoint{1.205155in}{0.696481in}}{\pgfqpoint{1.200826in}{0.706932in}}{\pgfqpoint{1.193122in}{0.714637in}}%
\pgfpathcurveto{\pgfqpoint{1.185418in}{0.722341in}}{\pgfqpoint{1.174967in}{0.726670in}}{\pgfqpoint{1.164071in}{0.726670in}}%
\pgfpathcurveto{\pgfqpoint{1.153176in}{0.726670in}}{\pgfqpoint{1.142725in}{0.722341in}}{\pgfqpoint{1.135021in}{0.714637in}}%
\pgfpathcurveto{\pgfqpoint{1.127316in}{0.706932in}}{\pgfqpoint{1.122988in}{0.696481in}}{\pgfqpoint{1.122988in}{0.685586in}}%
\pgfpathcurveto{\pgfqpoint{1.122988in}{0.674690in}}{\pgfqpoint{1.127316in}{0.664239in}}{\pgfqpoint{1.135021in}{0.656535in}}%
\pgfpathcurveto{\pgfqpoint{1.142725in}{0.648831in}}{\pgfqpoint{1.153176in}{0.644502in}}{\pgfqpoint{1.164071in}{0.644502in}}%
\pgfusepath{stroke}%
\end{pgfscope}%
\begin{pgfscope}%
\pgfpathrectangle{\pgfqpoint{0.688192in}{0.670138in}}{\pgfqpoint{7.111808in}{5.061530in}}%
\pgfusepath{clip}%
\pgfsetbuttcap%
\pgfsetroundjoin%
\pgfsetlinewidth{1.003750pt}%
\definecolor{currentstroke}{rgb}{0.000000,0.000000,0.000000}%
\pgfsetstrokecolor{currentstroke}%
\pgfsetdash{}{0pt}%
\pgfpathmoveto{\pgfqpoint{1.811432in}{0.640843in}}%
\pgfpathcurveto{\pgfqpoint{1.822328in}{0.640843in}}{\pgfqpoint{1.832779in}{0.645172in}}{\pgfqpoint{1.840483in}{0.652876in}}%
\pgfpathcurveto{\pgfqpoint{1.848187in}{0.660580in}}{\pgfqpoint{1.852516in}{0.671031in}}{\pgfqpoint{1.852516in}{0.681927in}}%
\pgfpathcurveto{\pgfqpoint{1.852516in}{0.692822in}}{\pgfqpoint{1.848187in}{0.703273in}}{\pgfqpoint{1.840483in}{0.710978in}}%
\pgfpathcurveto{\pgfqpoint{1.832779in}{0.718682in}}{\pgfqpoint{1.822328in}{0.723011in}}{\pgfqpoint{1.811432in}{0.723011in}}%
\pgfpathcurveto{\pgfqpoint{1.800537in}{0.723011in}}{\pgfqpoint{1.790086in}{0.718682in}}{\pgfqpoint{1.782381in}{0.710978in}}%
\pgfpathcurveto{\pgfqpoint{1.774677in}{0.703273in}}{\pgfqpoint{1.770348in}{0.692822in}}{\pgfqpoint{1.770348in}{0.681927in}}%
\pgfpathcurveto{\pgfqpoint{1.770348in}{0.671031in}}{\pgfqpoint{1.774677in}{0.660580in}}{\pgfqpoint{1.782381in}{0.652876in}}%
\pgfpathcurveto{\pgfqpoint{1.790086in}{0.645172in}}{\pgfqpoint{1.800537in}{0.640843in}}{\pgfqpoint{1.811432in}{0.640843in}}%
\pgfusepath{stroke}%
\end{pgfscope}%
\begin{pgfscope}%
\pgfpathrectangle{\pgfqpoint{0.688192in}{0.670138in}}{\pgfqpoint{7.111808in}{5.061530in}}%
\pgfusepath{clip}%
\pgfsetbuttcap%
\pgfsetroundjoin%
\pgfsetlinewidth{1.003750pt}%
\definecolor{currentstroke}{rgb}{0.000000,0.000000,0.000000}%
\pgfsetstrokecolor{currentstroke}%
\pgfsetdash{}{0pt}%
\pgfpathmoveto{\pgfqpoint{1.678208in}{0.641792in}}%
\pgfpathcurveto{\pgfqpoint{1.689103in}{0.641792in}}{\pgfqpoint{1.699554in}{0.646121in}}{\pgfqpoint{1.707258in}{0.653826in}}%
\pgfpathcurveto{\pgfqpoint{1.714963in}{0.661530in}}{\pgfqpoint{1.719292in}{0.671981in}}{\pgfqpoint{1.719292in}{0.682876in}}%
\pgfpathcurveto{\pgfqpoint{1.719292in}{0.693772in}}{\pgfqpoint{1.714963in}{0.704223in}}{\pgfqpoint{1.707258in}{0.711927in}}%
\pgfpathcurveto{\pgfqpoint{1.699554in}{0.719631in}}{\pgfqpoint{1.689103in}{0.723960in}}{\pgfqpoint{1.678208in}{0.723960in}}%
\pgfpathcurveto{\pgfqpoint{1.667312in}{0.723960in}}{\pgfqpoint{1.656861in}{0.719631in}}{\pgfqpoint{1.649157in}{0.711927in}}%
\pgfpathcurveto{\pgfqpoint{1.641453in}{0.704223in}}{\pgfqpoint{1.637124in}{0.693772in}}{\pgfqpoint{1.637124in}{0.682876in}}%
\pgfpathcurveto{\pgfqpoint{1.637124in}{0.671981in}}{\pgfqpoint{1.641453in}{0.661530in}}{\pgfqpoint{1.649157in}{0.653826in}}%
\pgfpathcurveto{\pgfqpoint{1.656861in}{0.646121in}}{\pgfqpoint{1.667312in}{0.641792in}}{\pgfqpoint{1.678208in}{0.641792in}}%
\pgfusepath{stroke}%
\end{pgfscope}%
\begin{pgfscope}%
\pgfpathrectangle{\pgfqpoint{0.688192in}{0.670138in}}{\pgfqpoint{7.111808in}{5.061530in}}%
\pgfusepath{clip}%
\pgfsetbuttcap%
\pgfsetroundjoin%
\pgfsetlinewidth{1.003750pt}%
\definecolor{currentstroke}{rgb}{0.000000,0.000000,0.000000}%
\pgfsetstrokecolor{currentstroke}%
\pgfsetdash{}{0pt}%
\pgfpathmoveto{\pgfqpoint{3.055486in}{1.754496in}}%
\pgfpathcurveto{\pgfqpoint{3.066382in}{1.754496in}}{\pgfqpoint{3.076833in}{1.758824in}}{\pgfqpoint{3.084537in}{1.766529in}}%
\pgfpathcurveto{\pgfqpoint{3.092241in}{1.774233in}}{\pgfqpoint{3.096570in}{1.784684in}}{\pgfqpoint{3.096570in}{1.795579in}}%
\pgfpathcurveto{\pgfqpoint{3.096570in}{1.806475in}}{\pgfqpoint{3.092241in}{1.816926in}}{\pgfqpoint{3.084537in}{1.824630in}}%
\pgfpathcurveto{\pgfqpoint{3.076833in}{1.832335in}}{\pgfqpoint{3.066382in}{1.836663in}}{\pgfqpoint{3.055486in}{1.836663in}}%
\pgfpathcurveto{\pgfqpoint{3.044591in}{1.836663in}}{\pgfqpoint{3.034140in}{1.832335in}}{\pgfqpoint{3.026436in}{1.824630in}}%
\pgfpathcurveto{\pgfqpoint{3.018731in}{1.816926in}}{\pgfqpoint{3.014402in}{1.806475in}}{\pgfqpoint{3.014402in}{1.795579in}}%
\pgfpathcurveto{\pgfqpoint{3.014402in}{1.784684in}}{\pgfqpoint{3.018731in}{1.774233in}}{\pgfqpoint{3.026436in}{1.766529in}}%
\pgfpathcurveto{\pgfqpoint{3.034140in}{1.758824in}}{\pgfqpoint{3.044591in}{1.754496in}}{\pgfqpoint{3.055486in}{1.754496in}}%
\pgfpathlineto{\pgfqpoint{3.055486in}{1.754496in}}%
\pgfpathclose%
\pgfusepath{stroke}%
\end{pgfscope}%
\begin{pgfscope}%
\pgfpathrectangle{\pgfqpoint{0.688192in}{0.670138in}}{\pgfqpoint{7.111808in}{5.061530in}}%
\pgfusepath{clip}%
\pgfsetbuttcap%
\pgfsetroundjoin%
\pgfsetlinewidth{1.003750pt}%
\definecolor{currentstroke}{rgb}{0.000000,0.000000,0.000000}%
\pgfsetstrokecolor{currentstroke}%
\pgfsetdash{}{0pt}%
\pgfpathmoveto{\pgfqpoint{1.469414in}{0.642878in}}%
\pgfpathcurveto{\pgfqpoint{1.480310in}{0.642878in}}{\pgfqpoint{1.490760in}{0.647207in}}{\pgfqpoint{1.498465in}{0.654911in}}%
\pgfpathcurveto{\pgfqpoint{1.506169in}{0.662615in}}{\pgfqpoint{1.510498in}{0.673066in}}{\pgfqpoint{1.510498in}{0.683962in}}%
\pgfpathcurveto{\pgfqpoint{1.510498in}{0.694857in}}{\pgfqpoint{1.506169in}{0.705308in}}{\pgfqpoint{1.498465in}{0.713012in}}%
\pgfpathcurveto{\pgfqpoint{1.490760in}{0.720717in}}{\pgfqpoint{1.480310in}{0.725046in}}{\pgfqpoint{1.469414in}{0.725046in}}%
\pgfpathcurveto{\pgfqpoint{1.458519in}{0.725046in}}{\pgfqpoint{1.448068in}{0.720717in}}{\pgfqpoint{1.440363in}{0.713012in}}%
\pgfpathcurveto{\pgfqpoint{1.432659in}{0.705308in}}{\pgfqpoint{1.428330in}{0.694857in}}{\pgfqpoint{1.428330in}{0.683962in}}%
\pgfpathcurveto{\pgfqpoint{1.428330in}{0.673066in}}{\pgfqpoint{1.432659in}{0.662615in}}{\pgfqpoint{1.440363in}{0.654911in}}%
\pgfpathcurveto{\pgfqpoint{1.448068in}{0.647207in}}{\pgfqpoint{1.458519in}{0.642878in}}{\pgfqpoint{1.469414in}{0.642878in}}%
\pgfusepath{stroke}%
\end{pgfscope}%
\begin{pgfscope}%
\pgfpathrectangle{\pgfqpoint{0.688192in}{0.670138in}}{\pgfqpoint{7.111808in}{5.061530in}}%
\pgfusepath{clip}%
\pgfsetbuttcap%
\pgfsetroundjoin%
\pgfsetlinewidth{1.003750pt}%
\definecolor{currentstroke}{rgb}{0.000000,0.000000,0.000000}%
\pgfsetstrokecolor{currentstroke}%
\pgfsetdash{}{0pt}%
\pgfpathmoveto{\pgfqpoint{1.719213in}{0.641640in}}%
\pgfpathcurveto{\pgfqpoint{1.730109in}{0.641640in}}{\pgfqpoint{1.740559in}{0.645969in}}{\pgfqpoint{1.748264in}{0.653673in}}%
\pgfpathcurveto{\pgfqpoint{1.755968in}{0.661377in}}{\pgfqpoint{1.760297in}{0.671828in}}{\pgfqpoint{1.760297in}{0.682724in}}%
\pgfpathcurveto{\pgfqpoint{1.760297in}{0.693619in}}{\pgfqpoint{1.755968in}{0.704070in}}{\pgfqpoint{1.748264in}{0.711774in}}%
\pgfpathcurveto{\pgfqpoint{1.740559in}{0.719479in}}{\pgfqpoint{1.730109in}{0.723807in}}{\pgfqpoint{1.719213in}{0.723807in}}%
\pgfpathcurveto{\pgfqpoint{1.708317in}{0.723807in}}{\pgfqpoint{1.697867in}{0.719479in}}{\pgfqpoint{1.690162in}{0.711774in}}%
\pgfpathcurveto{\pgfqpoint{1.682458in}{0.704070in}}{\pgfqpoint{1.678129in}{0.693619in}}{\pgfqpoint{1.678129in}{0.682724in}}%
\pgfpathcurveto{\pgfqpoint{1.678129in}{0.671828in}}{\pgfqpoint{1.682458in}{0.661377in}}{\pgfqpoint{1.690162in}{0.653673in}}%
\pgfpathcurveto{\pgfqpoint{1.697867in}{0.645969in}}{\pgfqpoint{1.708317in}{0.641640in}}{\pgfqpoint{1.719213in}{0.641640in}}%
\pgfusepath{stroke}%
\end{pgfscope}%
\begin{pgfscope}%
\pgfpathrectangle{\pgfqpoint{0.688192in}{0.670138in}}{\pgfqpoint{7.111808in}{5.061530in}}%
\pgfusepath{clip}%
\pgfsetbuttcap%
\pgfsetroundjoin%
\pgfsetlinewidth{1.003750pt}%
\definecolor{currentstroke}{rgb}{0.000000,0.000000,0.000000}%
\pgfsetstrokecolor{currentstroke}%
\pgfsetdash{}{0pt}%
\pgfpathmoveto{\pgfqpoint{0.842696in}{0.704078in}}%
\pgfpathcurveto{\pgfqpoint{0.853592in}{0.704078in}}{\pgfqpoint{0.864043in}{0.708407in}}{\pgfqpoint{0.871747in}{0.716112in}}%
\pgfpathcurveto{\pgfqpoint{0.879451in}{0.723816in}}{\pgfqpoint{0.883780in}{0.734267in}}{\pgfqpoint{0.883780in}{0.745162in}}%
\pgfpathcurveto{\pgfqpoint{0.883780in}{0.756058in}}{\pgfqpoint{0.879451in}{0.766509in}}{\pgfqpoint{0.871747in}{0.774213in}}%
\pgfpathcurveto{\pgfqpoint{0.864043in}{0.781917in}}{\pgfqpoint{0.853592in}{0.786246in}}{\pgfqpoint{0.842696in}{0.786246in}}%
\pgfpathcurveto{\pgfqpoint{0.831801in}{0.786246in}}{\pgfqpoint{0.821350in}{0.781917in}}{\pgfqpoint{0.813646in}{0.774213in}}%
\pgfpathcurveto{\pgfqpoint{0.805941in}{0.766509in}}{\pgfqpoint{0.801612in}{0.756058in}}{\pgfqpoint{0.801612in}{0.745162in}}%
\pgfpathcurveto{\pgfqpoint{0.801612in}{0.734267in}}{\pgfqpoint{0.805941in}{0.723816in}}{\pgfqpoint{0.813646in}{0.716112in}}%
\pgfpathcurveto{\pgfqpoint{0.821350in}{0.708407in}}{\pgfqpoint{0.831801in}{0.704078in}}{\pgfqpoint{0.842696in}{0.704078in}}%
\pgfpathlineto{\pgfqpoint{0.842696in}{0.704078in}}%
\pgfpathclose%
\pgfusepath{stroke}%
\end{pgfscope}%
\begin{pgfscope}%
\pgfpathrectangle{\pgfqpoint{0.688192in}{0.670138in}}{\pgfqpoint{7.111808in}{5.061530in}}%
\pgfusepath{clip}%
\pgfsetbuttcap%
\pgfsetroundjoin%
\pgfsetlinewidth{1.003750pt}%
\definecolor{currentstroke}{rgb}{0.000000,0.000000,0.000000}%
\pgfsetstrokecolor{currentstroke}%
\pgfsetdash{}{0pt}%
\pgfpathmoveto{\pgfqpoint{3.014199in}{3.949746in}}%
\pgfpathcurveto{\pgfqpoint{3.025094in}{3.949746in}}{\pgfqpoint{3.035545in}{3.954075in}}{\pgfqpoint{3.043249in}{3.961779in}}%
\pgfpathcurveto{\pgfqpoint{3.050954in}{3.969483in}}{\pgfqpoint{3.055283in}{3.979934in}}{\pgfqpoint{3.055283in}{3.990830in}}%
\pgfpathcurveto{\pgfqpoint{3.055283in}{4.001725in}}{\pgfqpoint{3.050954in}{4.012176in}}{\pgfqpoint{3.043249in}{4.019880in}}%
\pgfpathcurveto{\pgfqpoint{3.035545in}{4.027585in}}{\pgfqpoint{3.025094in}{4.031914in}}{\pgfqpoint{3.014199in}{4.031914in}}%
\pgfpathcurveto{\pgfqpoint{3.003303in}{4.031914in}}{\pgfqpoint{2.992852in}{4.027585in}}{\pgfqpoint{2.985148in}{4.019880in}}%
\pgfpathcurveto{\pgfqpoint{2.977444in}{4.012176in}}{\pgfqpoint{2.973115in}{4.001725in}}{\pgfqpoint{2.973115in}{3.990830in}}%
\pgfpathcurveto{\pgfqpoint{2.973115in}{3.979934in}}{\pgfqpoint{2.977444in}{3.969483in}}{\pgfqpoint{2.985148in}{3.961779in}}%
\pgfpathcurveto{\pgfqpoint{2.992852in}{3.954075in}}{\pgfqpoint{3.003303in}{3.949746in}}{\pgfqpoint{3.014199in}{3.949746in}}%
\pgfpathlineto{\pgfqpoint{3.014199in}{3.949746in}}%
\pgfpathclose%
\pgfusepath{stroke}%
\end{pgfscope}%
\begin{pgfscope}%
\pgfpathrectangle{\pgfqpoint{0.688192in}{0.670138in}}{\pgfqpoint{7.111808in}{5.061530in}}%
\pgfusepath{clip}%
\pgfsetbuttcap%
\pgfsetroundjoin%
\pgfsetlinewidth{1.003750pt}%
\definecolor{currentstroke}{rgb}{0.000000,0.000000,0.000000}%
\pgfsetstrokecolor{currentstroke}%
\pgfsetdash{}{0pt}%
\pgfpathmoveto{\pgfqpoint{1.198965in}{0.644318in}}%
\pgfpathcurveto{\pgfqpoint{1.209860in}{0.644318in}}{\pgfqpoint{1.220311in}{0.648647in}}{\pgfqpoint{1.228016in}{0.656352in}}%
\pgfpathcurveto{\pgfqpoint{1.235720in}{0.664056in}}{\pgfqpoint{1.240049in}{0.674507in}}{\pgfqpoint{1.240049in}{0.685402in}}%
\pgfpathcurveto{\pgfqpoint{1.240049in}{0.696298in}}{\pgfqpoint{1.235720in}{0.706749in}}{\pgfqpoint{1.228016in}{0.714453in}}%
\pgfpathcurveto{\pgfqpoint{1.220311in}{0.722157in}}{\pgfqpoint{1.209860in}{0.726486in}}{\pgfqpoint{1.198965in}{0.726486in}}%
\pgfpathcurveto{\pgfqpoint{1.188069in}{0.726486in}}{\pgfqpoint{1.177619in}{0.722157in}}{\pgfqpoint{1.169914in}{0.714453in}}%
\pgfpathcurveto{\pgfqpoint{1.162210in}{0.706749in}}{\pgfqpoint{1.157881in}{0.696298in}}{\pgfqpoint{1.157881in}{0.685402in}}%
\pgfpathcurveto{\pgfqpoint{1.157881in}{0.674507in}}{\pgfqpoint{1.162210in}{0.664056in}}{\pgfqpoint{1.169914in}{0.656352in}}%
\pgfpathcurveto{\pgfqpoint{1.177619in}{0.648647in}}{\pgfqpoint{1.188069in}{0.644318in}}{\pgfqpoint{1.198965in}{0.644318in}}%
\pgfusepath{stroke}%
\end{pgfscope}%
\begin{pgfscope}%
\pgfpathrectangle{\pgfqpoint{0.688192in}{0.670138in}}{\pgfqpoint{7.111808in}{5.061530in}}%
\pgfusepath{clip}%
\pgfsetbuttcap%
\pgfsetroundjoin%
\pgfsetlinewidth{1.003750pt}%
\definecolor{currentstroke}{rgb}{0.000000,0.000000,0.000000}%
\pgfsetstrokecolor{currentstroke}%
\pgfsetdash{}{0pt}%
\pgfpathmoveto{\pgfqpoint{1.967310in}{0.639999in}}%
\pgfpathcurveto{\pgfqpoint{1.978206in}{0.639999in}}{\pgfqpoint{1.988657in}{0.644328in}}{\pgfqpoint{1.996361in}{0.652032in}}%
\pgfpathcurveto{\pgfqpoint{2.004065in}{0.659736in}}{\pgfqpoint{2.008394in}{0.670187in}}{\pgfqpoint{2.008394in}{0.681083in}}%
\pgfpathcurveto{\pgfqpoint{2.008394in}{0.691978in}}{\pgfqpoint{2.004065in}{0.702429in}}{\pgfqpoint{1.996361in}{0.710133in}}%
\pgfpathcurveto{\pgfqpoint{1.988657in}{0.717838in}}{\pgfqpoint{1.978206in}{0.722166in}}{\pgfqpoint{1.967310in}{0.722166in}}%
\pgfpathcurveto{\pgfqpoint{1.956415in}{0.722166in}}{\pgfqpoint{1.945964in}{0.717838in}}{\pgfqpoint{1.938259in}{0.710133in}}%
\pgfpathcurveto{\pgfqpoint{1.930555in}{0.702429in}}{\pgfqpoint{1.926226in}{0.691978in}}{\pgfqpoint{1.926226in}{0.681083in}}%
\pgfpathcurveto{\pgfqpoint{1.926226in}{0.670187in}}{\pgfqpoint{1.930555in}{0.659736in}}{\pgfqpoint{1.938259in}{0.652032in}}%
\pgfpathcurveto{\pgfqpoint{1.945964in}{0.644328in}}{\pgfqpoint{1.956415in}{0.639999in}}{\pgfqpoint{1.967310in}{0.639999in}}%
\pgfusepath{stroke}%
\end{pgfscope}%
\begin{pgfscope}%
\pgfpathrectangle{\pgfqpoint{0.688192in}{0.670138in}}{\pgfqpoint{7.111808in}{5.061530in}}%
\pgfusepath{clip}%
\pgfsetbuttcap%
\pgfsetroundjoin%
\pgfsetlinewidth{1.003750pt}%
\definecolor{currentstroke}{rgb}{0.000000,0.000000,0.000000}%
\pgfsetstrokecolor{currentstroke}%
\pgfsetdash{}{0pt}%
\pgfpathmoveto{\pgfqpoint{4.491507in}{0.631138in}}%
\pgfpathcurveto{\pgfqpoint{4.502402in}{0.631138in}}{\pgfqpoint{4.512853in}{0.635467in}}{\pgfqpoint{4.520557in}{0.643171in}}%
\pgfpathcurveto{\pgfqpoint{4.528262in}{0.650875in}}{\pgfqpoint{4.532590in}{0.661326in}}{\pgfqpoint{4.532590in}{0.672222in}}%
\pgfpathcurveto{\pgfqpoint{4.532590in}{0.683117in}}{\pgfqpoint{4.528262in}{0.693568in}}{\pgfqpoint{4.520557in}{0.701272in}}%
\pgfpathcurveto{\pgfqpoint{4.512853in}{0.708977in}}{\pgfqpoint{4.502402in}{0.713305in}}{\pgfqpoint{4.491507in}{0.713305in}}%
\pgfpathcurveto{\pgfqpoint{4.480611in}{0.713305in}}{\pgfqpoint{4.470160in}{0.708977in}}{\pgfqpoint{4.462456in}{0.701272in}}%
\pgfpathcurveto{\pgfqpoint{4.454752in}{0.693568in}}{\pgfqpoint{4.450423in}{0.683117in}}{\pgfqpoint{4.450423in}{0.672222in}}%
\pgfpathcurveto{\pgfqpoint{4.450423in}{0.661326in}}{\pgfqpoint{4.454752in}{0.650875in}}{\pgfqpoint{4.462456in}{0.643171in}}%
\pgfpathcurveto{\pgfqpoint{4.470160in}{0.635467in}}{\pgfqpoint{4.480611in}{0.631138in}}{\pgfqpoint{4.491507in}{0.631138in}}%
\pgfusepath{stroke}%
\end{pgfscope}%
\begin{pgfscope}%
\pgfpathrectangle{\pgfqpoint{0.688192in}{0.670138in}}{\pgfqpoint{7.111808in}{5.061530in}}%
\pgfusepath{clip}%
\pgfsetbuttcap%
\pgfsetroundjoin%
\pgfsetlinewidth{1.003750pt}%
\definecolor{currentstroke}{rgb}{0.000000,0.000000,0.000000}%
\pgfsetstrokecolor{currentstroke}%
\pgfsetdash{}{0pt}%
\pgfpathmoveto{\pgfqpoint{2.429320in}{0.637441in}}%
\pgfpathcurveto{\pgfqpoint{2.440215in}{0.637441in}}{\pgfqpoint{2.450666in}{0.641770in}}{\pgfqpoint{2.458370in}{0.649474in}}%
\pgfpathcurveto{\pgfqpoint{2.466075in}{0.657178in}}{\pgfqpoint{2.470404in}{0.667629in}}{\pgfqpoint{2.470404in}{0.678525in}}%
\pgfpathcurveto{\pgfqpoint{2.470404in}{0.689420in}}{\pgfqpoint{2.466075in}{0.699871in}}{\pgfqpoint{2.458370in}{0.707575in}}%
\pgfpathcurveto{\pgfqpoint{2.450666in}{0.715280in}}{\pgfqpoint{2.440215in}{0.719609in}}{\pgfqpoint{2.429320in}{0.719609in}}%
\pgfpathcurveto{\pgfqpoint{2.418424in}{0.719609in}}{\pgfqpoint{2.407973in}{0.715280in}}{\pgfqpoint{2.400269in}{0.707575in}}%
\pgfpathcurveto{\pgfqpoint{2.392565in}{0.699871in}}{\pgfqpoint{2.388236in}{0.689420in}}{\pgfqpoint{2.388236in}{0.678525in}}%
\pgfpathcurveto{\pgfqpoint{2.388236in}{0.667629in}}{\pgfqpoint{2.392565in}{0.657178in}}{\pgfqpoint{2.400269in}{0.649474in}}%
\pgfpathcurveto{\pgfqpoint{2.407973in}{0.641770in}}{\pgfqpoint{2.418424in}{0.637441in}}{\pgfqpoint{2.429320in}{0.637441in}}%
\pgfusepath{stroke}%
\end{pgfscope}%
\begin{pgfscope}%
\pgfpathrectangle{\pgfqpoint{0.688192in}{0.670138in}}{\pgfqpoint{7.111808in}{5.061530in}}%
\pgfusepath{clip}%
\pgfsetbuttcap%
\pgfsetroundjoin%
\pgfsetlinewidth{1.003750pt}%
\definecolor{currentstroke}{rgb}{0.000000,0.000000,0.000000}%
\pgfsetstrokecolor{currentstroke}%
\pgfsetdash{}{0pt}%
\pgfpathmoveto{\pgfqpoint{2.763928in}{2.477598in}}%
\pgfpathcurveto{\pgfqpoint{2.774823in}{2.477598in}}{\pgfqpoint{2.785274in}{2.481927in}}{\pgfqpoint{2.792978in}{2.489632in}}%
\pgfpathcurveto{\pgfqpoint{2.800683in}{2.497336in}}{\pgfqpoint{2.805012in}{2.507787in}}{\pgfqpoint{2.805012in}{2.518682in}}%
\pgfpathcurveto{\pgfqpoint{2.805012in}{2.529578in}}{\pgfqpoint{2.800683in}{2.540029in}}{\pgfqpoint{2.792978in}{2.547733in}}%
\pgfpathcurveto{\pgfqpoint{2.785274in}{2.555437in}}{\pgfqpoint{2.774823in}{2.559766in}}{\pgfqpoint{2.763928in}{2.559766in}}%
\pgfpathcurveto{\pgfqpoint{2.753032in}{2.559766in}}{\pgfqpoint{2.742581in}{2.555437in}}{\pgfqpoint{2.734877in}{2.547733in}}%
\pgfpathcurveto{\pgfqpoint{2.727173in}{2.540029in}}{\pgfqpoint{2.722844in}{2.529578in}}{\pgfqpoint{2.722844in}{2.518682in}}%
\pgfpathcurveto{\pgfqpoint{2.722844in}{2.507787in}}{\pgfqpoint{2.727173in}{2.497336in}}{\pgfqpoint{2.734877in}{2.489632in}}%
\pgfpathcurveto{\pgfqpoint{2.742581in}{2.481927in}}{\pgfqpoint{2.753032in}{2.477598in}}{\pgfqpoint{2.763928in}{2.477598in}}%
\pgfpathlineto{\pgfqpoint{2.763928in}{2.477598in}}%
\pgfpathclose%
\pgfusepath{stroke}%
\end{pgfscope}%
\begin{pgfscope}%
\pgfpathrectangle{\pgfqpoint{0.688192in}{0.670138in}}{\pgfqpoint{7.111808in}{5.061530in}}%
\pgfusepath{clip}%
\pgfsetbuttcap%
\pgfsetroundjoin%
\pgfsetlinewidth{1.003750pt}%
\definecolor{currentstroke}{rgb}{0.000000,0.000000,0.000000}%
\pgfsetstrokecolor{currentstroke}%
\pgfsetdash{}{0pt}%
\pgfpathmoveto{\pgfqpoint{5.135185in}{0.630032in}}%
\pgfpathcurveto{\pgfqpoint{5.146081in}{0.630032in}}{\pgfqpoint{5.156532in}{0.634361in}}{\pgfqpoint{5.164236in}{0.642065in}}%
\pgfpathcurveto{\pgfqpoint{5.171940in}{0.649770in}}{\pgfqpoint{5.176269in}{0.660220in}}{\pgfqpoint{5.176269in}{0.671116in}}%
\pgfpathcurveto{\pgfqpoint{5.176269in}{0.682011in}}{\pgfqpoint{5.171940in}{0.692462in}}{\pgfqpoint{5.164236in}{0.700167in}}%
\pgfpathcurveto{\pgfqpoint{5.156532in}{0.707871in}}{\pgfqpoint{5.146081in}{0.712200in}}{\pgfqpoint{5.135185in}{0.712200in}}%
\pgfpathcurveto{\pgfqpoint{5.124290in}{0.712200in}}{\pgfqpoint{5.113839in}{0.707871in}}{\pgfqpoint{5.106135in}{0.700167in}}%
\pgfpathcurveto{\pgfqpoint{5.098430in}{0.692462in}}{\pgfqpoint{5.094101in}{0.682011in}}{\pgfqpoint{5.094101in}{0.671116in}}%
\pgfpathcurveto{\pgfqpoint{5.094101in}{0.660220in}}{\pgfqpoint{5.098430in}{0.649770in}}{\pgfqpoint{5.106135in}{0.642065in}}%
\pgfpathcurveto{\pgfqpoint{5.113839in}{0.634361in}}{\pgfqpoint{5.124290in}{0.630032in}}{\pgfqpoint{5.135185in}{0.630032in}}%
\pgfusepath{stroke}%
\end{pgfscope}%
\begin{pgfscope}%
\pgfpathrectangle{\pgfqpoint{0.688192in}{0.670138in}}{\pgfqpoint{7.111808in}{5.061530in}}%
\pgfusepath{clip}%
\pgfsetbuttcap%
\pgfsetroundjoin%
\pgfsetlinewidth{1.003750pt}%
\definecolor{currentstroke}{rgb}{0.000000,0.000000,0.000000}%
\pgfsetstrokecolor{currentstroke}%
\pgfsetdash{}{0pt}%
\pgfpathmoveto{\pgfqpoint{1.103303in}{0.648067in}}%
\pgfpathcurveto{\pgfqpoint{1.114198in}{0.648067in}}{\pgfqpoint{1.124649in}{0.652396in}}{\pgfqpoint{1.132353in}{0.660101in}}%
\pgfpathcurveto{\pgfqpoint{1.140058in}{0.667805in}}{\pgfqpoint{1.144386in}{0.678256in}}{\pgfqpoint{1.144386in}{0.689151in}}%
\pgfpathcurveto{\pgfqpoint{1.144386in}{0.700047in}}{\pgfqpoint{1.140058in}{0.710498in}}{\pgfqpoint{1.132353in}{0.718202in}}%
\pgfpathcurveto{\pgfqpoint{1.124649in}{0.725906in}}{\pgfqpoint{1.114198in}{0.730235in}}{\pgfqpoint{1.103303in}{0.730235in}}%
\pgfpathcurveto{\pgfqpoint{1.092407in}{0.730235in}}{\pgfqpoint{1.081956in}{0.725906in}}{\pgfqpoint{1.074252in}{0.718202in}}%
\pgfpathcurveto{\pgfqpoint{1.066548in}{0.710498in}}{\pgfqpoint{1.062219in}{0.700047in}}{\pgfqpoint{1.062219in}{0.689151in}}%
\pgfpathcurveto{\pgfqpoint{1.062219in}{0.678256in}}{\pgfqpoint{1.066548in}{0.667805in}}{\pgfqpoint{1.074252in}{0.660101in}}%
\pgfpathcurveto{\pgfqpoint{1.081956in}{0.652396in}}{\pgfqpoint{1.092407in}{0.648067in}}{\pgfqpoint{1.103303in}{0.648067in}}%
\pgfusepath{stroke}%
\end{pgfscope}%
\begin{pgfscope}%
\pgfpathrectangle{\pgfqpoint{0.688192in}{0.670138in}}{\pgfqpoint{7.111808in}{5.061530in}}%
\pgfusepath{clip}%
\pgfsetbuttcap%
\pgfsetroundjoin%
\pgfsetlinewidth{1.003750pt}%
\definecolor{currentstroke}{rgb}{0.000000,0.000000,0.000000}%
\pgfsetstrokecolor{currentstroke}%
\pgfsetdash{}{0pt}%
\pgfpathmoveto{\pgfqpoint{1.681839in}{0.641743in}}%
\pgfpathcurveto{\pgfqpoint{1.692735in}{0.641743in}}{\pgfqpoint{1.703185in}{0.646071in}}{\pgfqpoint{1.710890in}{0.653776in}}%
\pgfpathcurveto{\pgfqpoint{1.718594in}{0.661480in}}{\pgfqpoint{1.722923in}{0.671931in}}{\pgfqpoint{1.722923in}{0.682826in}}%
\pgfpathcurveto{\pgfqpoint{1.722923in}{0.693722in}}{\pgfqpoint{1.718594in}{0.704173in}}{\pgfqpoint{1.710890in}{0.711877in}}%
\pgfpathcurveto{\pgfqpoint{1.703185in}{0.719581in}}{\pgfqpoint{1.692735in}{0.723910in}}{\pgfqpoint{1.681839in}{0.723910in}}%
\pgfpathcurveto{\pgfqpoint{1.670943in}{0.723910in}}{\pgfqpoint{1.660493in}{0.719581in}}{\pgfqpoint{1.652788in}{0.711877in}}%
\pgfpathcurveto{\pgfqpoint{1.645084in}{0.704173in}}{\pgfqpoint{1.640755in}{0.693722in}}{\pgfqpoint{1.640755in}{0.682826in}}%
\pgfpathcurveto{\pgfqpoint{1.640755in}{0.671931in}}{\pgfqpoint{1.645084in}{0.661480in}}{\pgfqpoint{1.652788in}{0.653776in}}%
\pgfpathcurveto{\pgfqpoint{1.660493in}{0.646071in}}{\pgfqpoint{1.670943in}{0.641743in}}{\pgfqpoint{1.681839in}{0.641743in}}%
\pgfusepath{stroke}%
\end{pgfscope}%
\begin{pgfscope}%
\pgfpathrectangle{\pgfqpoint{0.688192in}{0.670138in}}{\pgfqpoint{7.111808in}{5.061530in}}%
\pgfusepath{clip}%
\pgfsetbuttcap%
\pgfsetroundjoin%
\pgfsetlinewidth{1.003750pt}%
\definecolor{currentstroke}{rgb}{0.000000,0.000000,0.000000}%
\pgfsetstrokecolor{currentstroke}%
\pgfsetdash{}{0pt}%
\pgfpathmoveto{\pgfqpoint{1.108617in}{0.647679in}}%
\pgfpathcurveto{\pgfqpoint{1.119513in}{0.647679in}}{\pgfqpoint{1.129964in}{0.652008in}}{\pgfqpoint{1.137668in}{0.659712in}}%
\pgfpathcurveto{\pgfqpoint{1.145372in}{0.667417in}}{\pgfqpoint{1.149701in}{0.677868in}}{\pgfqpoint{1.149701in}{0.688763in}}%
\pgfpathcurveto{\pgfqpoint{1.149701in}{0.699659in}}{\pgfqpoint{1.145372in}{0.710109in}}{\pgfqpoint{1.137668in}{0.717814in}}%
\pgfpathcurveto{\pgfqpoint{1.129964in}{0.725518in}}{\pgfqpoint{1.119513in}{0.729847in}}{\pgfqpoint{1.108617in}{0.729847in}}%
\pgfpathcurveto{\pgfqpoint{1.097722in}{0.729847in}}{\pgfqpoint{1.087271in}{0.725518in}}{\pgfqpoint{1.079567in}{0.717814in}}%
\pgfpathcurveto{\pgfqpoint{1.071862in}{0.710109in}}{\pgfqpoint{1.067534in}{0.699659in}}{\pgfqpoint{1.067534in}{0.688763in}}%
\pgfpathcurveto{\pgfqpoint{1.067534in}{0.677868in}}{\pgfqpoint{1.071862in}{0.667417in}}{\pgfqpoint{1.079567in}{0.659712in}}%
\pgfpathcurveto{\pgfqpoint{1.087271in}{0.652008in}}{\pgfqpoint{1.097722in}{0.647679in}}{\pgfqpoint{1.108617in}{0.647679in}}%
\pgfusepath{stroke}%
\end{pgfscope}%
\begin{pgfscope}%
\pgfpathrectangle{\pgfqpoint{0.688192in}{0.670138in}}{\pgfqpoint{7.111808in}{5.061530in}}%
\pgfusepath{clip}%
\pgfsetbuttcap%
\pgfsetroundjoin%
\pgfsetlinewidth{1.003750pt}%
\definecolor{currentstroke}{rgb}{0.000000,0.000000,0.000000}%
\pgfsetstrokecolor{currentstroke}%
\pgfsetdash{}{0pt}%
\pgfpathmoveto{\pgfqpoint{4.661935in}{0.630846in}}%
\pgfpathcurveto{\pgfqpoint{4.672830in}{0.630846in}}{\pgfqpoint{4.683281in}{0.635175in}}{\pgfqpoint{4.690985in}{0.642879in}}%
\pgfpathcurveto{\pgfqpoint{4.698690in}{0.650583in}}{\pgfqpoint{4.703018in}{0.661034in}}{\pgfqpoint{4.703018in}{0.671930in}}%
\pgfpathcurveto{\pgfqpoint{4.703018in}{0.682825in}}{\pgfqpoint{4.698690in}{0.693276in}}{\pgfqpoint{4.690985in}{0.700980in}}%
\pgfpathcurveto{\pgfqpoint{4.683281in}{0.708685in}}{\pgfqpoint{4.672830in}{0.713014in}}{\pgfqpoint{4.661935in}{0.713014in}}%
\pgfpathcurveto{\pgfqpoint{4.651039in}{0.713014in}}{\pgfqpoint{4.640588in}{0.708685in}}{\pgfqpoint{4.632884in}{0.700980in}}%
\pgfpathcurveto{\pgfqpoint{4.625180in}{0.693276in}}{\pgfqpoint{4.620851in}{0.682825in}}{\pgfqpoint{4.620851in}{0.671930in}}%
\pgfpathcurveto{\pgfqpoint{4.620851in}{0.661034in}}{\pgfqpoint{4.625180in}{0.650583in}}{\pgfqpoint{4.632884in}{0.642879in}}%
\pgfpathcurveto{\pgfqpoint{4.640588in}{0.635175in}}{\pgfqpoint{4.651039in}{0.630846in}}{\pgfqpoint{4.661935in}{0.630846in}}%
\pgfusepath{stroke}%
\end{pgfscope}%
\begin{pgfscope}%
\pgfpathrectangle{\pgfqpoint{0.688192in}{0.670138in}}{\pgfqpoint{7.111808in}{5.061530in}}%
\pgfusepath{clip}%
\pgfsetbuttcap%
\pgfsetroundjoin%
\pgfsetlinewidth{1.003750pt}%
\definecolor{currentstroke}{rgb}{0.000000,0.000000,0.000000}%
\pgfsetstrokecolor{currentstroke}%
\pgfsetdash{}{0pt}%
\pgfpathmoveto{\pgfqpoint{1.334128in}{0.643434in}}%
\pgfpathcurveto{\pgfqpoint{1.345023in}{0.643434in}}{\pgfqpoint{1.355474in}{0.647762in}}{\pgfqpoint{1.363178in}{0.655467in}}%
\pgfpathcurveto{\pgfqpoint{1.370883in}{0.663171in}}{\pgfqpoint{1.375211in}{0.673622in}}{\pgfqpoint{1.375211in}{0.684517in}}%
\pgfpathcurveto{\pgfqpoint{1.375211in}{0.695413in}}{\pgfqpoint{1.370883in}{0.705864in}}{\pgfqpoint{1.363178in}{0.713568in}}%
\pgfpathcurveto{\pgfqpoint{1.355474in}{0.721272in}}{\pgfqpoint{1.345023in}{0.725601in}}{\pgfqpoint{1.334128in}{0.725601in}}%
\pgfpathcurveto{\pgfqpoint{1.323232in}{0.725601in}}{\pgfqpoint{1.312781in}{0.721272in}}{\pgfqpoint{1.305077in}{0.713568in}}%
\pgfpathcurveto{\pgfqpoint{1.297373in}{0.705864in}}{\pgfqpoint{1.293044in}{0.695413in}}{\pgfqpoint{1.293044in}{0.684517in}}%
\pgfpathcurveto{\pgfqpoint{1.293044in}{0.673622in}}{\pgfqpoint{1.297373in}{0.663171in}}{\pgfqpoint{1.305077in}{0.655467in}}%
\pgfpathcurveto{\pgfqpoint{1.312781in}{0.647762in}}{\pgfqpoint{1.323232in}{0.643434in}}{\pgfqpoint{1.334128in}{0.643434in}}%
\pgfusepath{stroke}%
\end{pgfscope}%
\begin{pgfscope}%
\pgfpathrectangle{\pgfqpoint{0.688192in}{0.670138in}}{\pgfqpoint{7.111808in}{5.061530in}}%
\pgfusepath{clip}%
\pgfsetbuttcap%
\pgfsetroundjoin%
\pgfsetlinewidth{1.003750pt}%
\definecolor{currentstroke}{rgb}{0.000000,0.000000,0.000000}%
\pgfsetstrokecolor{currentstroke}%
\pgfsetdash{}{0pt}%
\pgfpathmoveto{\pgfqpoint{1.178009in}{0.644363in}}%
\pgfpathcurveto{\pgfqpoint{1.188905in}{0.644363in}}{\pgfqpoint{1.199355in}{0.648692in}}{\pgfqpoint{1.207060in}{0.656396in}}%
\pgfpathcurveto{\pgfqpoint{1.214764in}{0.664100in}}{\pgfqpoint{1.219093in}{0.674551in}}{\pgfqpoint{1.219093in}{0.685447in}}%
\pgfpathcurveto{\pgfqpoint{1.219093in}{0.696342in}}{\pgfqpoint{1.214764in}{0.706793in}}{\pgfqpoint{1.207060in}{0.714497in}}%
\pgfpathcurveto{\pgfqpoint{1.199355in}{0.722202in}}{\pgfqpoint{1.188905in}{0.726531in}}{\pgfqpoint{1.178009in}{0.726531in}}%
\pgfpathcurveto{\pgfqpoint{1.167113in}{0.726531in}}{\pgfqpoint{1.156663in}{0.722202in}}{\pgfqpoint{1.148958in}{0.714497in}}%
\pgfpathcurveto{\pgfqpoint{1.141254in}{0.706793in}}{\pgfqpoint{1.136925in}{0.696342in}}{\pgfqpoint{1.136925in}{0.685447in}}%
\pgfpathcurveto{\pgfqpoint{1.136925in}{0.674551in}}{\pgfqpoint{1.141254in}{0.664100in}}{\pgfqpoint{1.148958in}{0.656396in}}%
\pgfpathcurveto{\pgfqpoint{1.156663in}{0.648692in}}{\pgfqpoint{1.167113in}{0.644363in}}{\pgfqpoint{1.178009in}{0.644363in}}%
\pgfusepath{stroke}%
\end{pgfscope}%
\begin{pgfscope}%
\pgfpathrectangle{\pgfqpoint{0.688192in}{0.670138in}}{\pgfqpoint{7.111808in}{5.061530in}}%
\pgfusepath{clip}%
\pgfsetbuttcap%
\pgfsetroundjoin%
\pgfsetlinewidth{1.003750pt}%
\definecolor{currentstroke}{rgb}{0.000000,0.000000,0.000000}%
\pgfsetstrokecolor{currentstroke}%
\pgfsetdash{}{0pt}%
\pgfpathmoveto{\pgfqpoint{1.109998in}{0.647141in}}%
\pgfpathcurveto{\pgfqpoint{1.120894in}{0.647141in}}{\pgfqpoint{1.131345in}{0.651470in}}{\pgfqpoint{1.139049in}{0.659174in}}%
\pgfpathcurveto{\pgfqpoint{1.146753in}{0.666879in}}{\pgfqpoint{1.151082in}{0.677329in}}{\pgfqpoint{1.151082in}{0.688225in}}%
\pgfpathcurveto{\pgfqpoint{1.151082in}{0.699120in}}{\pgfqpoint{1.146753in}{0.709571in}}{\pgfqpoint{1.139049in}{0.717276in}}%
\pgfpathcurveto{\pgfqpoint{1.131345in}{0.724980in}}{\pgfqpoint{1.120894in}{0.729309in}}{\pgfqpoint{1.109998in}{0.729309in}}%
\pgfpathcurveto{\pgfqpoint{1.099103in}{0.729309in}}{\pgfqpoint{1.088652in}{0.724980in}}{\pgfqpoint{1.080948in}{0.717276in}}%
\pgfpathcurveto{\pgfqpoint{1.073243in}{0.709571in}}{\pgfqpoint{1.068914in}{0.699120in}}{\pgfqpoint{1.068914in}{0.688225in}}%
\pgfpathcurveto{\pgfqpoint{1.068914in}{0.677329in}}{\pgfqpoint{1.073243in}{0.666879in}}{\pgfqpoint{1.080948in}{0.659174in}}%
\pgfpathcurveto{\pgfqpoint{1.088652in}{0.651470in}}{\pgfqpoint{1.099103in}{0.647141in}}{\pgfqpoint{1.109998in}{0.647141in}}%
\pgfusepath{stroke}%
\end{pgfscope}%
\begin{pgfscope}%
\pgfpathrectangle{\pgfqpoint{0.688192in}{0.670138in}}{\pgfqpoint{7.111808in}{5.061530in}}%
\pgfusepath{clip}%
\pgfsetbuttcap%
\pgfsetroundjoin%
\pgfsetlinewidth{1.003750pt}%
\definecolor{currentstroke}{rgb}{0.000000,0.000000,0.000000}%
\pgfsetstrokecolor{currentstroke}%
\pgfsetdash{}{0pt}%
\pgfpathmoveto{\pgfqpoint{0.945763in}{0.669750in}}%
\pgfpathcurveto{\pgfqpoint{0.956659in}{0.669750in}}{\pgfqpoint{0.967110in}{0.674079in}}{\pgfqpoint{0.974814in}{0.681783in}}%
\pgfpathcurveto{\pgfqpoint{0.982518in}{0.689488in}}{\pgfqpoint{0.986847in}{0.699938in}}{\pgfqpoint{0.986847in}{0.710834in}}%
\pgfpathcurveto{\pgfqpoint{0.986847in}{0.721730in}}{\pgfqpoint{0.982518in}{0.732180in}}{\pgfqpoint{0.974814in}{0.739885in}}%
\pgfpathcurveto{\pgfqpoint{0.967110in}{0.747589in}}{\pgfqpoint{0.956659in}{0.751918in}}{\pgfqpoint{0.945763in}{0.751918in}}%
\pgfpathcurveto{\pgfqpoint{0.934868in}{0.751918in}}{\pgfqpoint{0.924417in}{0.747589in}}{\pgfqpoint{0.916713in}{0.739885in}}%
\pgfpathcurveto{\pgfqpoint{0.909008in}{0.732180in}}{\pgfqpoint{0.904679in}{0.721730in}}{\pgfqpoint{0.904679in}{0.710834in}}%
\pgfpathcurveto{\pgfqpoint{0.904679in}{0.699938in}}{\pgfqpoint{0.909008in}{0.689488in}}{\pgfqpoint{0.916713in}{0.681783in}}%
\pgfpathcurveto{\pgfqpoint{0.924417in}{0.674079in}}{\pgfqpoint{0.934868in}{0.669750in}}{\pgfqpoint{0.945763in}{0.669750in}}%
\pgfpathlineto{\pgfqpoint{0.945763in}{0.669750in}}%
\pgfpathclose%
\pgfusepath{stroke}%
\end{pgfscope}%
\begin{pgfscope}%
\pgfpathrectangle{\pgfqpoint{0.688192in}{0.670138in}}{\pgfqpoint{7.111808in}{5.061530in}}%
\pgfusepath{clip}%
\pgfsetbuttcap%
\pgfsetroundjoin%
\pgfsetlinewidth{1.003750pt}%
\definecolor{currentstroke}{rgb}{0.000000,0.000000,0.000000}%
\pgfsetstrokecolor{currentstroke}%
\pgfsetdash{}{0pt}%
\pgfpathmoveto{\pgfqpoint{4.463558in}{5.010906in}}%
\pgfpathcurveto{\pgfqpoint{4.474453in}{5.010906in}}{\pgfqpoint{4.484904in}{5.015235in}}{\pgfqpoint{4.492608in}{5.022939in}}%
\pgfpathcurveto{\pgfqpoint{4.500313in}{5.030643in}}{\pgfqpoint{4.504641in}{5.041094in}}{\pgfqpoint{4.504641in}{5.051990in}}%
\pgfpathcurveto{\pgfqpoint{4.504641in}{5.062885in}}{\pgfqpoint{4.500313in}{5.073336in}}{\pgfqpoint{4.492608in}{5.081040in}}%
\pgfpathcurveto{\pgfqpoint{4.484904in}{5.088745in}}{\pgfqpoint{4.474453in}{5.093073in}}{\pgfqpoint{4.463558in}{5.093073in}}%
\pgfpathcurveto{\pgfqpoint{4.452662in}{5.093073in}}{\pgfqpoint{4.442211in}{5.088745in}}{\pgfqpoint{4.434507in}{5.081040in}}%
\pgfpathcurveto{\pgfqpoint{4.426803in}{5.073336in}}{\pgfqpoint{4.422474in}{5.062885in}}{\pgfqpoint{4.422474in}{5.051990in}}%
\pgfpathcurveto{\pgfqpoint{4.422474in}{5.041094in}}{\pgfqpoint{4.426803in}{5.030643in}}{\pgfqpoint{4.434507in}{5.022939in}}%
\pgfpathcurveto{\pgfqpoint{4.442211in}{5.015235in}}{\pgfqpoint{4.452662in}{5.010906in}}{\pgfqpoint{4.463558in}{5.010906in}}%
\pgfpathlineto{\pgfqpoint{4.463558in}{5.010906in}}%
\pgfpathclose%
\pgfusepath{stroke}%
\end{pgfscope}%
\begin{pgfscope}%
\pgfpathrectangle{\pgfqpoint{0.688192in}{0.670138in}}{\pgfqpoint{7.111808in}{5.061530in}}%
\pgfusepath{clip}%
\pgfsetbuttcap%
\pgfsetroundjoin%
\pgfsetlinewidth{1.003750pt}%
\definecolor{currentstroke}{rgb}{0.000000,0.000000,0.000000}%
\pgfsetstrokecolor{currentstroke}%
\pgfsetdash{}{0pt}%
\pgfpathmoveto{\pgfqpoint{1.826952in}{0.640664in}}%
\pgfpathcurveto{\pgfqpoint{1.837847in}{0.640664in}}{\pgfqpoint{1.848298in}{0.644992in}}{\pgfqpoint{1.856002in}{0.652697in}}%
\pgfpathcurveto{\pgfqpoint{1.863707in}{0.660401in}}{\pgfqpoint{1.868035in}{0.670852in}}{\pgfqpoint{1.868035in}{0.681747in}}%
\pgfpathcurveto{\pgfqpoint{1.868035in}{0.692643in}}{\pgfqpoint{1.863707in}{0.703094in}}{\pgfqpoint{1.856002in}{0.710798in}}%
\pgfpathcurveto{\pgfqpoint{1.848298in}{0.718502in}}{\pgfqpoint{1.837847in}{0.722831in}}{\pgfqpoint{1.826952in}{0.722831in}}%
\pgfpathcurveto{\pgfqpoint{1.816056in}{0.722831in}}{\pgfqpoint{1.805605in}{0.718502in}}{\pgfqpoint{1.797901in}{0.710798in}}%
\pgfpathcurveto{\pgfqpoint{1.790197in}{0.703094in}}{\pgfqpoint{1.785868in}{0.692643in}}{\pgfqpoint{1.785868in}{0.681747in}}%
\pgfpathcurveto{\pgfqpoint{1.785868in}{0.670852in}}{\pgfqpoint{1.790197in}{0.660401in}}{\pgfqpoint{1.797901in}{0.652697in}}%
\pgfpathcurveto{\pgfqpoint{1.805605in}{0.644992in}}{\pgfqpoint{1.816056in}{0.640664in}}{\pgfqpoint{1.826952in}{0.640664in}}%
\pgfusepath{stroke}%
\end{pgfscope}%
\begin{pgfscope}%
\pgfpathrectangle{\pgfqpoint{0.688192in}{0.670138in}}{\pgfqpoint{7.111808in}{5.061530in}}%
\pgfusepath{clip}%
\pgfsetbuttcap%
\pgfsetroundjoin%
\pgfsetlinewidth{1.003750pt}%
\definecolor{currentstroke}{rgb}{0.000000,0.000000,0.000000}%
\pgfsetstrokecolor{currentstroke}%
\pgfsetdash{}{0pt}%
\pgfpathmoveto{\pgfqpoint{1.148995in}{0.644808in}}%
\pgfpathcurveto{\pgfqpoint{1.159890in}{0.644808in}}{\pgfqpoint{1.170341in}{0.649137in}}{\pgfqpoint{1.178045in}{0.656841in}}%
\pgfpathcurveto{\pgfqpoint{1.185750in}{0.664545in}}{\pgfqpoint{1.190079in}{0.674996in}}{\pgfqpoint{1.190079in}{0.685892in}}%
\pgfpathcurveto{\pgfqpoint{1.190079in}{0.696787in}}{\pgfqpoint{1.185750in}{0.707238in}}{\pgfqpoint{1.178045in}{0.714942in}}%
\pgfpathcurveto{\pgfqpoint{1.170341in}{0.722647in}}{\pgfqpoint{1.159890in}{0.726975in}}{\pgfqpoint{1.148995in}{0.726975in}}%
\pgfpathcurveto{\pgfqpoint{1.138099in}{0.726975in}}{\pgfqpoint{1.127648in}{0.722647in}}{\pgfqpoint{1.119944in}{0.714942in}}%
\pgfpathcurveto{\pgfqpoint{1.112240in}{0.707238in}}{\pgfqpoint{1.107911in}{0.696787in}}{\pgfqpoint{1.107911in}{0.685892in}}%
\pgfpathcurveto{\pgfqpoint{1.107911in}{0.674996in}}{\pgfqpoint{1.112240in}{0.664545in}}{\pgfqpoint{1.119944in}{0.656841in}}%
\pgfpathcurveto{\pgfqpoint{1.127648in}{0.649137in}}{\pgfqpoint{1.138099in}{0.644808in}}{\pgfqpoint{1.148995in}{0.644808in}}%
\pgfusepath{stroke}%
\end{pgfscope}%
\begin{pgfscope}%
\pgfpathrectangle{\pgfqpoint{0.688192in}{0.670138in}}{\pgfqpoint{7.111808in}{5.061530in}}%
\pgfusepath{clip}%
\pgfsetbuttcap%
\pgfsetroundjoin%
\pgfsetlinewidth{1.003750pt}%
\definecolor{currentstroke}{rgb}{0.000000,0.000000,0.000000}%
\pgfsetstrokecolor{currentstroke}%
\pgfsetdash{}{0pt}%
\pgfpathmoveto{\pgfqpoint{0.872029in}{0.690616in}}%
\pgfpathcurveto{\pgfqpoint{0.882925in}{0.690616in}}{\pgfqpoint{0.893376in}{0.694944in}}{\pgfqpoint{0.901080in}{0.702649in}}%
\pgfpathcurveto{\pgfqpoint{0.908784in}{0.710353in}}{\pgfqpoint{0.913113in}{0.720804in}}{\pgfqpoint{0.913113in}{0.731700in}}%
\pgfpathcurveto{\pgfqpoint{0.913113in}{0.742595in}}{\pgfqpoint{0.908784in}{0.753046in}}{\pgfqpoint{0.901080in}{0.760750in}}%
\pgfpathcurveto{\pgfqpoint{0.893376in}{0.768455in}}{\pgfqpoint{0.882925in}{0.772783in}}{\pgfqpoint{0.872029in}{0.772783in}}%
\pgfpathcurveto{\pgfqpoint{0.861134in}{0.772783in}}{\pgfqpoint{0.850683in}{0.768455in}}{\pgfqpoint{0.842979in}{0.760750in}}%
\pgfpathcurveto{\pgfqpoint{0.835274in}{0.753046in}}{\pgfqpoint{0.830945in}{0.742595in}}{\pgfqpoint{0.830945in}{0.731700in}}%
\pgfpathcurveto{\pgfqpoint{0.830945in}{0.720804in}}{\pgfqpoint{0.835274in}{0.710353in}}{\pgfqpoint{0.842979in}{0.702649in}}%
\pgfpathcurveto{\pgfqpoint{0.850683in}{0.694944in}}{\pgfqpoint{0.861134in}{0.690616in}}{\pgfqpoint{0.872029in}{0.690616in}}%
\pgfpathlineto{\pgfqpoint{0.872029in}{0.690616in}}%
\pgfpathclose%
\pgfusepath{stroke}%
\end{pgfscope}%
\begin{pgfscope}%
\pgfpathrectangle{\pgfqpoint{0.688192in}{0.670138in}}{\pgfqpoint{7.111808in}{5.061530in}}%
\pgfusepath{clip}%
\pgfsetbuttcap%
\pgfsetroundjoin%
\pgfsetlinewidth{1.003750pt}%
\definecolor{currentstroke}{rgb}{0.000000,0.000000,0.000000}%
\pgfsetstrokecolor{currentstroke}%
\pgfsetdash{}{0pt}%
\pgfpathmoveto{\pgfqpoint{0.842696in}{0.704078in}}%
\pgfpathcurveto{\pgfqpoint{0.853592in}{0.704078in}}{\pgfqpoint{0.864043in}{0.708407in}}{\pgfqpoint{0.871747in}{0.716112in}}%
\pgfpathcurveto{\pgfqpoint{0.879451in}{0.723816in}}{\pgfqpoint{0.883780in}{0.734267in}}{\pgfqpoint{0.883780in}{0.745162in}}%
\pgfpathcurveto{\pgfqpoint{0.883780in}{0.756058in}}{\pgfqpoint{0.879451in}{0.766509in}}{\pgfqpoint{0.871747in}{0.774213in}}%
\pgfpathcurveto{\pgfqpoint{0.864043in}{0.781917in}}{\pgfqpoint{0.853592in}{0.786246in}}{\pgfqpoint{0.842696in}{0.786246in}}%
\pgfpathcurveto{\pgfqpoint{0.831801in}{0.786246in}}{\pgfqpoint{0.821350in}{0.781917in}}{\pgfqpoint{0.813646in}{0.774213in}}%
\pgfpathcurveto{\pgfqpoint{0.805941in}{0.766509in}}{\pgfqpoint{0.801612in}{0.756058in}}{\pgfqpoint{0.801612in}{0.745162in}}%
\pgfpathcurveto{\pgfqpoint{0.801612in}{0.734267in}}{\pgfqpoint{0.805941in}{0.723816in}}{\pgfqpoint{0.813646in}{0.716112in}}%
\pgfpathcurveto{\pgfqpoint{0.821350in}{0.708407in}}{\pgfqpoint{0.831801in}{0.704078in}}{\pgfqpoint{0.842696in}{0.704078in}}%
\pgfpathlineto{\pgfqpoint{0.842696in}{0.704078in}}%
\pgfpathclose%
\pgfusepath{stroke}%
\end{pgfscope}%
\begin{pgfscope}%
\pgfpathrectangle{\pgfqpoint{0.688192in}{0.670138in}}{\pgfqpoint{7.111808in}{5.061530in}}%
\pgfusepath{clip}%
\pgfsetbuttcap%
\pgfsetroundjoin%
\pgfsetlinewidth{1.003750pt}%
\definecolor{currentstroke}{rgb}{0.000000,0.000000,0.000000}%
\pgfsetstrokecolor{currentstroke}%
\pgfsetdash{}{0pt}%
\pgfpathmoveto{\pgfqpoint{3.280946in}{0.634044in}}%
\pgfpathcurveto{\pgfqpoint{3.291842in}{0.634044in}}{\pgfqpoint{3.302293in}{0.638373in}}{\pgfqpoint{3.309997in}{0.646078in}}%
\pgfpathcurveto{\pgfqpoint{3.317701in}{0.653782in}}{\pgfqpoint{3.322030in}{0.664233in}}{\pgfqpoint{3.322030in}{0.675128in}}%
\pgfpathcurveto{\pgfqpoint{3.322030in}{0.686024in}}{\pgfqpoint{3.317701in}{0.696475in}}{\pgfqpoint{3.309997in}{0.704179in}}%
\pgfpathcurveto{\pgfqpoint{3.302293in}{0.711883in}}{\pgfqpoint{3.291842in}{0.716212in}}{\pgfqpoint{3.280946in}{0.716212in}}%
\pgfpathcurveto{\pgfqpoint{3.270051in}{0.716212in}}{\pgfqpoint{3.259600in}{0.711883in}}{\pgfqpoint{3.251896in}{0.704179in}}%
\pgfpathcurveto{\pgfqpoint{3.244191in}{0.696475in}}{\pgfqpoint{3.239862in}{0.686024in}}{\pgfqpoint{3.239862in}{0.675128in}}%
\pgfpathcurveto{\pgfqpoint{3.239862in}{0.664233in}}{\pgfqpoint{3.244191in}{0.653782in}}{\pgfqpoint{3.251896in}{0.646078in}}%
\pgfpathcurveto{\pgfqpoint{3.259600in}{0.638373in}}{\pgfqpoint{3.270051in}{0.634044in}}{\pgfqpoint{3.280946in}{0.634044in}}%
\pgfusepath{stroke}%
\end{pgfscope}%
\begin{pgfscope}%
\pgfpathrectangle{\pgfqpoint{0.688192in}{0.670138in}}{\pgfqpoint{7.111808in}{5.061530in}}%
\pgfusepath{clip}%
\pgfsetbuttcap%
\pgfsetroundjoin%
\pgfsetlinewidth{1.003750pt}%
\definecolor{currentstroke}{rgb}{0.000000,0.000000,0.000000}%
\pgfsetstrokecolor{currentstroke}%
\pgfsetdash{}{0pt}%
\pgfpathmoveto{\pgfqpoint{7.631530in}{2.110783in}}%
\pgfpathcurveto{\pgfqpoint{7.642425in}{2.110783in}}{\pgfqpoint{7.652876in}{2.115112in}}{\pgfqpoint{7.660580in}{2.122816in}}%
\pgfpathcurveto{\pgfqpoint{7.668285in}{2.130520in}}{\pgfqpoint{7.672614in}{2.140971in}}{\pgfqpoint{7.672614in}{2.151867in}}%
\pgfpathcurveto{\pgfqpoint{7.672614in}{2.162762in}}{\pgfqpoint{7.668285in}{2.173213in}}{\pgfqpoint{7.660580in}{2.180918in}}%
\pgfpathcurveto{\pgfqpoint{7.652876in}{2.188622in}}{\pgfqpoint{7.642425in}{2.192951in}}{\pgfqpoint{7.631530in}{2.192951in}}%
\pgfpathcurveto{\pgfqpoint{7.620634in}{2.192951in}}{\pgfqpoint{7.610183in}{2.188622in}}{\pgfqpoint{7.602479in}{2.180918in}}%
\pgfpathcurveto{\pgfqpoint{7.594775in}{2.173213in}}{\pgfqpoint{7.590446in}{2.162762in}}{\pgfqpoint{7.590446in}{2.151867in}}%
\pgfpathcurveto{\pgfqpoint{7.590446in}{2.140971in}}{\pgfqpoint{7.594775in}{2.130520in}}{\pgfqpoint{7.602479in}{2.122816in}}%
\pgfpathcurveto{\pgfqpoint{7.610183in}{2.115112in}}{\pgfqpoint{7.620634in}{2.110783in}}{\pgfqpoint{7.631530in}{2.110783in}}%
\pgfpathlineto{\pgfqpoint{7.631530in}{2.110783in}}%
\pgfpathclose%
\pgfusepath{stroke}%
\end{pgfscope}%
\begin{pgfscope}%
\pgfpathrectangle{\pgfqpoint{0.688192in}{0.670138in}}{\pgfqpoint{7.111808in}{5.061530in}}%
\pgfusepath{clip}%
\pgfsetbuttcap%
\pgfsetroundjoin%
\pgfsetlinewidth{1.003750pt}%
\definecolor{currentstroke}{rgb}{0.000000,0.000000,0.000000}%
\pgfsetstrokecolor{currentstroke}%
\pgfsetdash{}{0pt}%
\pgfpathmoveto{\pgfqpoint{3.809782in}{0.707059in}}%
\pgfpathcurveto{\pgfqpoint{3.820677in}{0.707059in}}{\pgfqpoint{3.831128in}{0.711388in}}{\pgfqpoint{3.838832in}{0.719093in}}%
\pgfpathcurveto{\pgfqpoint{3.846537in}{0.726797in}}{\pgfqpoint{3.850866in}{0.737248in}}{\pgfqpoint{3.850866in}{0.748143in}}%
\pgfpathcurveto{\pgfqpoint{3.850866in}{0.759039in}}{\pgfqpoint{3.846537in}{0.769490in}}{\pgfqpoint{3.838832in}{0.777194in}}%
\pgfpathcurveto{\pgfqpoint{3.831128in}{0.784898in}}{\pgfqpoint{3.820677in}{0.789227in}}{\pgfqpoint{3.809782in}{0.789227in}}%
\pgfpathcurveto{\pgfqpoint{3.798886in}{0.789227in}}{\pgfqpoint{3.788435in}{0.784898in}}{\pgfqpoint{3.780731in}{0.777194in}}%
\pgfpathcurveto{\pgfqpoint{3.773027in}{0.769490in}}{\pgfqpoint{3.768698in}{0.759039in}}{\pgfqpoint{3.768698in}{0.748143in}}%
\pgfpathcurveto{\pgfqpoint{3.768698in}{0.737248in}}{\pgfqpoint{3.773027in}{0.726797in}}{\pgfqpoint{3.780731in}{0.719093in}}%
\pgfpathcurveto{\pgfqpoint{3.788435in}{0.711388in}}{\pgfqpoint{3.798886in}{0.707059in}}{\pgfqpoint{3.809782in}{0.707059in}}%
\pgfpathlineto{\pgfqpoint{3.809782in}{0.707059in}}%
\pgfpathclose%
\pgfusepath{stroke}%
\end{pgfscope}%
\begin{pgfscope}%
\pgfpathrectangle{\pgfqpoint{0.688192in}{0.670138in}}{\pgfqpoint{7.111808in}{5.061530in}}%
\pgfusepath{clip}%
\pgfsetbuttcap%
\pgfsetroundjoin%
\pgfsetlinewidth{1.003750pt}%
\definecolor{currentstroke}{rgb}{0.000000,0.000000,0.000000}%
\pgfsetstrokecolor{currentstroke}%
\pgfsetdash{}{0pt}%
\pgfpathmoveto{\pgfqpoint{0.958858in}{0.667930in}}%
\pgfpathcurveto{\pgfqpoint{0.969753in}{0.667930in}}{\pgfqpoint{0.980204in}{0.672259in}}{\pgfqpoint{0.987908in}{0.679963in}}%
\pgfpathcurveto{\pgfqpoint{0.995613in}{0.687667in}}{\pgfqpoint{0.999941in}{0.698118in}}{\pgfqpoint{0.999941in}{0.709014in}}%
\pgfpathcurveto{\pgfqpoint{0.999941in}{0.719909in}}{\pgfqpoint{0.995613in}{0.730360in}}{\pgfqpoint{0.987908in}{0.738065in}}%
\pgfpathcurveto{\pgfqpoint{0.980204in}{0.745769in}}{\pgfqpoint{0.969753in}{0.750098in}}{\pgfqpoint{0.958858in}{0.750098in}}%
\pgfpathcurveto{\pgfqpoint{0.947962in}{0.750098in}}{\pgfqpoint{0.937511in}{0.745769in}}{\pgfqpoint{0.929807in}{0.738065in}}%
\pgfpathcurveto{\pgfqpoint{0.922102in}{0.730360in}}{\pgfqpoint{0.917774in}{0.719909in}}{\pgfqpoint{0.917774in}{0.709014in}}%
\pgfpathcurveto{\pgfqpoint{0.917774in}{0.698118in}}{\pgfqpoint{0.922102in}{0.687667in}}{\pgfqpoint{0.929807in}{0.679963in}}%
\pgfpathcurveto{\pgfqpoint{0.937511in}{0.672259in}}{\pgfqpoint{0.947962in}{0.667930in}}{\pgfqpoint{0.958858in}{0.667930in}}%
\pgfpathlineto{\pgfqpoint{0.958858in}{0.667930in}}%
\pgfpathclose%
\pgfusepath{stroke}%
\end{pgfscope}%
\begin{pgfscope}%
\pgfpathrectangle{\pgfqpoint{0.688192in}{0.670138in}}{\pgfqpoint{7.111808in}{5.061530in}}%
\pgfusepath{clip}%
\pgfsetbuttcap%
\pgfsetroundjoin%
\pgfsetlinewidth{1.003750pt}%
\definecolor{currentstroke}{rgb}{0.000000,0.000000,0.000000}%
\pgfsetstrokecolor{currentstroke}%
\pgfsetdash{}{0pt}%
\pgfpathmoveto{\pgfqpoint{1.731531in}{0.641456in}}%
\pgfpathcurveto{\pgfqpoint{1.742427in}{0.641456in}}{\pgfqpoint{1.752878in}{0.645785in}}{\pgfqpoint{1.760582in}{0.653489in}}%
\pgfpathcurveto{\pgfqpoint{1.768286in}{0.661194in}}{\pgfqpoint{1.772615in}{0.671645in}}{\pgfqpoint{1.772615in}{0.682540in}}%
\pgfpathcurveto{\pgfqpoint{1.772615in}{0.693436in}}{\pgfqpoint{1.768286in}{0.703886in}}{\pgfqpoint{1.760582in}{0.711591in}}%
\pgfpathcurveto{\pgfqpoint{1.752878in}{0.719295in}}{\pgfqpoint{1.742427in}{0.723624in}}{\pgfqpoint{1.731531in}{0.723624in}}%
\pgfpathcurveto{\pgfqpoint{1.720636in}{0.723624in}}{\pgfqpoint{1.710185in}{0.719295in}}{\pgfqpoint{1.702481in}{0.711591in}}%
\pgfpathcurveto{\pgfqpoint{1.694776in}{0.703886in}}{\pgfqpoint{1.690447in}{0.693436in}}{\pgfqpoint{1.690447in}{0.682540in}}%
\pgfpathcurveto{\pgfqpoint{1.690447in}{0.671645in}}{\pgfqpoint{1.694776in}{0.661194in}}{\pgfqpoint{1.702481in}{0.653489in}}%
\pgfpathcurveto{\pgfqpoint{1.710185in}{0.645785in}}{\pgfqpoint{1.720636in}{0.641456in}}{\pgfqpoint{1.731531in}{0.641456in}}%
\pgfusepath{stroke}%
\end{pgfscope}%
\begin{pgfscope}%
\pgfpathrectangle{\pgfqpoint{0.688192in}{0.670138in}}{\pgfqpoint{7.111808in}{5.061530in}}%
\pgfusepath{clip}%
\pgfsetbuttcap%
\pgfsetroundjoin%
\pgfsetlinewidth{1.003750pt}%
\definecolor{currentstroke}{rgb}{0.000000,0.000000,0.000000}%
\pgfsetstrokecolor{currentstroke}%
\pgfsetdash{}{0pt}%
\pgfpathmoveto{\pgfqpoint{1.198965in}{0.644318in}}%
\pgfpathcurveto{\pgfqpoint{1.209860in}{0.644318in}}{\pgfqpoint{1.220311in}{0.648647in}}{\pgfqpoint{1.228016in}{0.656352in}}%
\pgfpathcurveto{\pgfqpoint{1.235720in}{0.664056in}}{\pgfqpoint{1.240049in}{0.674507in}}{\pgfqpoint{1.240049in}{0.685402in}}%
\pgfpathcurveto{\pgfqpoint{1.240049in}{0.696298in}}{\pgfqpoint{1.235720in}{0.706749in}}{\pgfqpoint{1.228016in}{0.714453in}}%
\pgfpathcurveto{\pgfqpoint{1.220311in}{0.722157in}}{\pgfqpoint{1.209860in}{0.726486in}}{\pgfqpoint{1.198965in}{0.726486in}}%
\pgfpathcurveto{\pgfqpoint{1.188069in}{0.726486in}}{\pgfqpoint{1.177619in}{0.722157in}}{\pgfqpoint{1.169914in}{0.714453in}}%
\pgfpathcurveto{\pgfqpoint{1.162210in}{0.706749in}}{\pgfqpoint{1.157881in}{0.696298in}}{\pgfqpoint{1.157881in}{0.685402in}}%
\pgfpathcurveto{\pgfqpoint{1.157881in}{0.674507in}}{\pgfqpoint{1.162210in}{0.664056in}}{\pgfqpoint{1.169914in}{0.656352in}}%
\pgfpathcurveto{\pgfqpoint{1.177619in}{0.648647in}}{\pgfqpoint{1.188069in}{0.644318in}}{\pgfqpoint{1.198965in}{0.644318in}}%
\pgfusepath{stroke}%
\end{pgfscope}%
\begin{pgfscope}%
\pgfpathrectangle{\pgfqpoint{0.688192in}{0.670138in}}{\pgfqpoint{7.111808in}{5.061530in}}%
\pgfusepath{clip}%
\pgfsetbuttcap%
\pgfsetroundjoin%
\pgfsetlinewidth{1.003750pt}%
\definecolor{currentstroke}{rgb}{0.000000,0.000000,0.000000}%
\pgfsetstrokecolor{currentstroke}%
\pgfsetdash{}{0pt}%
\pgfpathmoveto{\pgfqpoint{2.603035in}{0.731524in}}%
\pgfpathcurveto{\pgfqpoint{2.613931in}{0.731524in}}{\pgfqpoint{2.624382in}{0.735853in}}{\pgfqpoint{2.632086in}{0.743557in}}%
\pgfpathcurveto{\pgfqpoint{2.639790in}{0.751261in}}{\pgfqpoint{2.644119in}{0.761712in}}{\pgfqpoint{2.644119in}{0.772608in}}%
\pgfpathcurveto{\pgfqpoint{2.644119in}{0.783503in}}{\pgfqpoint{2.639790in}{0.793954in}}{\pgfqpoint{2.632086in}{0.801659in}}%
\pgfpathcurveto{\pgfqpoint{2.624382in}{0.809363in}}{\pgfqpoint{2.613931in}{0.813692in}}{\pgfqpoint{2.603035in}{0.813692in}}%
\pgfpathcurveto{\pgfqpoint{2.592140in}{0.813692in}}{\pgfqpoint{2.581689in}{0.809363in}}{\pgfqpoint{2.573985in}{0.801659in}}%
\pgfpathcurveto{\pgfqpoint{2.566280in}{0.793954in}}{\pgfqpoint{2.561951in}{0.783503in}}{\pgfqpoint{2.561951in}{0.772608in}}%
\pgfpathcurveto{\pgfqpoint{2.561951in}{0.761712in}}{\pgfqpoint{2.566280in}{0.751261in}}{\pgfqpoint{2.573985in}{0.743557in}}%
\pgfpathcurveto{\pgfqpoint{2.581689in}{0.735853in}}{\pgfqpoint{2.592140in}{0.731524in}}{\pgfqpoint{2.603035in}{0.731524in}}%
\pgfpathlineto{\pgfqpoint{2.603035in}{0.731524in}}%
\pgfpathclose%
\pgfusepath{stroke}%
\end{pgfscope}%
\begin{pgfscope}%
\pgfpathrectangle{\pgfqpoint{0.688192in}{0.670138in}}{\pgfqpoint{7.111808in}{5.061530in}}%
\pgfusepath{clip}%
\pgfsetbuttcap%
\pgfsetroundjoin%
\pgfsetlinewidth{1.003750pt}%
\definecolor{currentstroke}{rgb}{0.000000,0.000000,0.000000}%
\pgfsetstrokecolor{currentstroke}%
\pgfsetdash{}{0pt}%
\pgfpathmoveto{\pgfqpoint{1.477331in}{0.642791in}}%
\pgfpathcurveto{\pgfqpoint{1.488227in}{0.642791in}}{\pgfqpoint{1.498677in}{0.647120in}}{\pgfqpoint{1.506382in}{0.654824in}}%
\pgfpathcurveto{\pgfqpoint{1.514086in}{0.662528in}}{\pgfqpoint{1.518415in}{0.672979in}}{\pgfqpoint{1.518415in}{0.683875in}}%
\pgfpathcurveto{\pgfqpoint{1.518415in}{0.694770in}}{\pgfqpoint{1.514086in}{0.705221in}}{\pgfqpoint{1.506382in}{0.712926in}}%
\pgfpathcurveto{\pgfqpoint{1.498677in}{0.720630in}}{\pgfqpoint{1.488227in}{0.724959in}}{\pgfqpoint{1.477331in}{0.724959in}}%
\pgfpathcurveto{\pgfqpoint{1.466435in}{0.724959in}}{\pgfqpoint{1.455985in}{0.720630in}}{\pgfqpoint{1.448280in}{0.712926in}}%
\pgfpathcurveto{\pgfqpoint{1.440576in}{0.705221in}}{\pgfqpoint{1.436247in}{0.694770in}}{\pgfqpoint{1.436247in}{0.683875in}}%
\pgfpathcurveto{\pgfqpoint{1.436247in}{0.672979in}}{\pgfqpoint{1.440576in}{0.662528in}}{\pgfqpoint{1.448280in}{0.654824in}}%
\pgfpathcurveto{\pgfqpoint{1.455985in}{0.647120in}}{\pgfqpoint{1.466435in}{0.642791in}}{\pgfqpoint{1.477331in}{0.642791in}}%
\pgfusepath{stroke}%
\end{pgfscope}%
\begin{pgfscope}%
\pgfpathrectangle{\pgfqpoint{0.688192in}{0.670138in}}{\pgfqpoint{7.111808in}{5.061530in}}%
\pgfusepath{clip}%
\pgfsetbuttcap%
\pgfsetroundjoin%
\pgfsetlinewidth{1.003750pt}%
\definecolor{currentstroke}{rgb}{0.000000,0.000000,0.000000}%
\pgfsetstrokecolor{currentstroke}%
\pgfsetdash{}{0pt}%
\pgfpathmoveto{\pgfqpoint{5.658790in}{1.473202in}}%
\pgfpathcurveto{\pgfqpoint{5.669686in}{1.473202in}}{\pgfqpoint{5.680137in}{1.477531in}}{\pgfqpoint{5.687841in}{1.485235in}}%
\pgfpathcurveto{\pgfqpoint{5.695545in}{1.492940in}}{\pgfqpoint{5.699874in}{1.503390in}}{\pgfqpoint{5.699874in}{1.514286in}}%
\pgfpathcurveto{\pgfqpoint{5.699874in}{1.525182in}}{\pgfqpoint{5.695545in}{1.535632in}}{\pgfqpoint{5.687841in}{1.543337in}}%
\pgfpathcurveto{\pgfqpoint{5.680137in}{1.551041in}}{\pgfqpoint{5.669686in}{1.555370in}}{\pgfqpoint{5.658790in}{1.555370in}}%
\pgfpathcurveto{\pgfqpoint{5.647895in}{1.555370in}}{\pgfqpoint{5.637444in}{1.551041in}}{\pgfqpoint{5.629740in}{1.543337in}}%
\pgfpathcurveto{\pgfqpoint{5.622035in}{1.535632in}}{\pgfqpoint{5.617707in}{1.525182in}}{\pgfqpoint{5.617707in}{1.514286in}}%
\pgfpathcurveto{\pgfqpoint{5.617707in}{1.503390in}}{\pgfqpoint{5.622035in}{1.492940in}}{\pgfqpoint{5.629740in}{1.485235in}}%
\pgfpathcurveto{\pgfqpoint{5.637444in}{1.477531in}}{\pgfqpoint{5.647895in}{1.473202in}}{\pgfqpoint{5.658790in}{1.473202in}}%
\pgfpathlineto{\pgfqpoint{5.658790in}{1.473202in}}%
\pgfpathclose%
\pgfusepath{stroke}%
\end{pgfscope}%
\begin{pgfscope}%
\pgfpathrectangle{\pgfqpoint{0.688192in}{0.670138in}}{\pgfqpoint{7.111808in}{5.061530in}}%
\pgfusepath{clip}%
\pgfsetbuttcap%
\pgfsetroundjoin%
\pgfsetlinewidth{1.003750pt}%
\definecolor{currentstroke}{rgb}{0.000000,0.000000,0.000000}%
\pgfsetstrokecolor{currentstroke}%
\pgfsetdash{}{0pt}%
\pgfpathmoveto{\pgfqpoint{0.946217in}{0.669736in}}%
\pgfpathcurveto{\pgfqpoint{0.957113in}{0.669736in}}{\pgfqpoint{0.967564in}{0.674065in}}{\pgfqpoint{0.975268in}{0.681769in}}%
\pgfpathcurveto{\pgfqpoint{0.982972in}{0.689473in}}{\pgfqpoint{0.987301in}{0.699924in}}{\pgfqpoint{0.987301in}{0.710820in}}%
\pgfpathcurveto{\pgfqpoint{0.987301in}{0.721715in}}{\pgfqpoint{0.982972in}{0.732166in}}{\pgfqpoint{0.975268in}{0.739870in}}%
\pgfpathcurveto{\pgfqpoint{0.967564in}{0.747575in}}{\pgfqpoint{0.957113in}{0.751904in}}{\pgfqpoint{0.946217in}{0.751904in}}%
\pgfpathcurveto{\pgfqpoint{0.935322in}{0.751904in}}{\pgfqpoint{0.924871in}{0.747575in}}{\pgfqpoint{0.917167in}{0.739870in}}%
\pgfpathcurveto{\pgfqpoint{0.909462in}{0.732166in}}{\pgfqpoint{0.905134in}{0.721715in}}{\pgfqpoint{0.905134in}{0.710820in}}%
\pgfpathcurveto{\pgfqpoint{0.905134in}{0.699924in}}{\pgfqpoint{0.909462in}{0.689473in}}{\pgfqpoint{0.917167in}{0.681769in}}%
\pgfpathcurveto{\pgfqpoint{0.924871in}{0.674065in}}{\pgfqpoint{0.935322in}{0.669736in}}{\pgfqpoint{0.946217in}{0.669736in}}%
\pgfpathlineto{\pgfqpoint{0.946217in}{0.669736in}}%
\pgfpathclose%
\pgfusepath{stroke}%
\end{pgfscope}%
\begin{pgfscope}%
\pgfpathrectangle{\pgfqpoint{0.688192in}{0.670138in}}{\pgfqpoint{7.111808in}{5.061530in}}%
\pgfusepath{clip}%
\pgfsetbuttcap%
\pgfsetroundjoin%
\pgfsetlinewidth{1.003750pt}%
\definecolor{currentstroke}{rgb}{0.000000,0.000000,0.000000}%
\pgfsetstrokecolor{currentstroke}%
\pgfsetdash{}{0pt}%
\pgfpathmoveto{\pgfqpoint{2.911373in}{0.635871in}}%
\pgfpathcurveto{\pgfqpoint{2.922269in}{0.635871in}}{\pgfqpoint{2.932719in}{0.640200in}}{\pgfqpoint{2.940424in}{0.647905in}}%
\pgfpathcurveto{\pgfqpoint{2.948128in}{0.655609in}}{\pgfqpoint{2.952457in}{0.666060in}}{\pgfqpoint{2.952457in}{0.676955in}}%
\pgfpathcurveto{\pgfqpoint{2.952457in}{0.687851in}}{\pgfqpoint{2.948128in}{0.698302in}}{\pgfqpoint{2.940424in}{0.706006in}}%
\pgfpathcurveto{\pgfqpoint{2.932719in}{0.713710in}}{\pgfqpoint{2.922269in}{0.718039in}}{\pgfqpoint{2.911373in}{0.718039in}}%
\pgfpathcurveto{\pgfqpoint{2.900477in}{0.718039in}}{\pgfqpoint{2.890027in}{0.713710in}}{\pgfqpoint{2.882322in}{0.706006in}}%
\pgfpathcurveto{\pgfqpoint{2.874618in}{0.698302in}}{\pgfqpoint{2.870289in}{0.687851in}}{\pgfqpoint{2.870289in}{0.676955in}}%
\pgfpathcurveto{\pgfqpoint{2.870289in}{0.666060in}}{\pgfqpoint{2.874618in}{0.655609in}}{\pgfqpoint{2.882322in}{0.647905in}}%
\pgfpathcurveto{\pgfqpoint{2.890027in}{0.640200in}}{\pgfqpoint{2.900477in}{0.635871in}}{\pgfqpoint{2.911373in}{0.635871in}}%
\pgfusepath{stroke}%
\end{pgfscope}%
\begin{pgfscope}%
\pgfpathrectangle{\pgfqpoint{0.688192in}{0.670138in}}{\pgfqpoint{7.111808in}{5.061530in}}%
\pgfusepath{clip}%
\pgfsetbuttcap%
\pgfsetroundjoin%
\pgfsetlinewidth{1.003750pt}%
\definecolor{currentstroke}{rgb}{0.000000,0.000000,0.000000}%
\pgfsetstrokecolor{currentstroke}%
\pgfsetdash{}{0pt}%
\pgfpathmoveto{\pgfqpoint{1.469414in}{0.642878in}}%
\pgfpathcurveto{\pgfqpoint{1.480310in}{0.642878in}}{\pgfqpoint{1.490760in}{0.647207in}}{\pgfqpoint{1.498465in}{0.654911in}}%
\pgfpathcurveto{\pgfqpoint{1.506169in}{0.662615in}}{\pgfqpoint{1.510498in}{0.673066in}}{\pgfqpoint{1.510498in}{0.683962in}}%
\pgfpathcurveto{\pgfqpoint{1.510498in}{0.694857in}}{\pgfqpoint{1.506169in}{0.705308in}}{\pgfqpoint{1.498465in}{0.713012in}}%
\pgfpathcurveto{\pgfqpoint{1.490760in}{0.720717in}}{\pgfqpoint{1.480310in}{0.725046in}}{\pgfqpoint{1.469414in}{0.725046in}}%
\pgfpathcurveto{\pgfqpoint{1.458519in}{0.725046in}}{\pgfqpoint{1.448068in}{0.720717in}}{\pgfqpoint{1.440363in}{0.713012in}}%
\pgfpathcurveto{\pgfqpoint{1.432659in}{0.705308in}}{\pgfqpoint{1.428330in}{0.694857in}}{\pgfqpoint{1.428330in}{0.683962in}}%
\pgfpathcurveto{\pgfqpoint{1.428330in}{0.673066in}}{\pgfqpoint{1.432659in}{0.662615in}}{\pgfqpoint{1.440363in}{0.654911in}}%
\pgfpathcurveto{\pgfqpoint{1.448068in}{0.647207in}}{\pgfqpoint{1.458519in}{0.642878in}}{\pgfqpoint{1.469414in}{0.642878in}}%
\pgfusepath{stroke}%
\end{pgfscope}%
\begin{pgfscope}%
\pgfpathrectangle{\pgfqpoint{0.688192in}{0.670138in}}{\pgfqpoint{7.111808in}{5.061530in}}%
\pgfusepath{clip}%
\pgfsetbuttcap%
\pgfsetroundjoin%
\pgfsetlinewidth{1.003750pt}%
\definecolor{currentstroke}{rgb}{0.000000,0.000000,0.000000}%
\pgfsetstrokecolor{currentstroke}%
\pgfsetdash{}{0pt}%
\pgfpathmoveto{\pgfqpoint{2.745823in}{2.382906in}}%
\pgfpathcurveto{\pgfqpoint{2.756718in}{2.382906in}}{\pgfqpoint{2.767169in}{2.387235in}}{\pgfqpoint{2.774874in}{2.394939in}}%
\pgfpathcurveto{\pgfqpoint{2.782578in}{2.402644in}}{\pgfqpoint{2.786907in}{2.413095in}}{\pgfqpoint{2.786907in}{2.423990in}}%
\pgfpathcurveto{\pgfqpoint{2.786907in}{2.434886in}}{\pgfqpoint{2.782578in}{2.445337in}}{\pgfqpoint{2.774874in}{2.453041in}}%
\pgfpathcurveto{\pgfqpoint{2.767169in}{2.460745in}}{\pgfqpoint{2.756718in}{2.465074in}}{\pgfqpoint{2.745823in}{2.465074in}}%
\pgfpathcurveto{\pgfqpoint{2.734927in}{2.465074in}}{\pgfqpoint{2.724477in}{2.460745in}}{\pgfqpoint{2.716772in}{2.453041in}}%
\pgfpathcurveto{\pgfqpoint{2.709068in}{2.445337in}}{\pgfqpoint{2.704739in}{2.434886in}}{\pgfqpoint{2.704739in}{2.423990in}}%
\pgfpathcurveto{\pgfqpoint{2.704739in}{2.413095in}}{\pgfqpoint{2.709068in}{2.402644in}}{\pgfqpoint{2.716772in}{2.394939in}}%
\pgfpathcurveto{\pgfqpoint{2.724477in}{2.387235in}}{\pgfqpoint{2.734927in}{2.382906in}}{\pgfqpoint{2.745823in}{2.382906in}}%
\pgfpathlineto{\pgfqpoint{2.745823in}{2.382906in}}%
\pgfpathclose%
\pgfusepath{stroke}%
\end{pgfscope}%
\begin{pgfscope}%
\pgfpathrectangle{\pgfqpoint{0.688192in}{0.670138in}}{\pgfqpoint{7.111808in}{5.061530in}}%
\pgfusepath{clip}%
\pgfsetbuttcap%
\pgfsetroundjoin%
\pgfsetlinewidth{1.003750pt}%
\definecolor{currentstroke}{rgb}{0.000000,0.000000,0.000000}%
\pgfsetstrokecolor{currentstroke}%
\pgfsetdash{}{0pt}%
\pgfpathmoveto{\pgfqpoint{1.198965in}{0.644318in}}%
\pgfpathcurveto{\pgfqpoint{1.209860in}{0.644318in}}{\pgfqpoint{1.220311in}{0.648647in}}{\pgfqpoint{1.228016in}{0.656352in}}%
\pgfpathcurveto{\pgfqpoint{1.235720in}{0.664056in}}{\pgfqpoint{1.240049in}{0.674507in}}{\pgfqpoint{1.240049in}{0.685402in}}%
\pgfpathcurveto{\pgfqpoint{1.240049in}{0.696298in}}{\pgfqpoint{1.235720in}{0.706749in}}{\pgfqpoint{1.228016in}{0.714453in}}%
\pgfpathcurveto{\pgfqpoint{1.220311in}{0.722157in}}{\pgfqpoint{1.209860in}{0.726486in}}{\pgfqpoint{1.198965in}{0.726486in}}%
\pgfpathcurveto{\pgfqpoint{1.188069in}{0.726486in}}{\pgfqpoint{1.177619in}{0.722157in}}{\pgfqpoint{1.169914in}{0.714453in}}%
\pgfpathcurveto{\pgfqpoint{1.162210in}{0.706749in}}{\pgfqpoint{1.157881in}{0.696298in}}{\pgfqpoint{1.157881in}{0.685402in}}%
\pgfpathcurveto{\pgfqpoint{1.157881in}{0.674507in}}{\pgfqpoint{1.162210in}{0.664056in}}{\pgfqpoint{1.169914in}{0.656352in}}%
\pgfpathcurveto{\pgfqpoint{1.177619in}{0.648647in}}{\pgfqpoint{1.188069in}{0.644318in}}{\pgfqpoint{1.198965in}{0.644318in}}%
\pgfusepath{stroke}%
\end{pgfscope}%
\begin{pgfscope}%
\pgfpathrectangle{\pgfqpoint{0.688192in}{0.670138in}}{\pgfqpoint{7.111808in}{5.061530in}}%
\pgfusepath{clip}%
\pgfsetbuttcap%
\pgfsetroundjoin%
\pgfsetlinewidth{1.003750pt}%
\definecolor{currentstroke}{rgb}{0.000000,0.000000,0.000000}%
\pgfsetstrokecolor{currentstroke}%
\pgfsetdash{}{0pt}%
\pgfpathmoveto{\pgfqpoint{1.948909in}{2.490221in}}%
\pgfpathcurveto{\pgfqpoint{1.959804in}{2.490221in}}{\pgfqpoint{1.970255in}{2.494550in}}{\pgfqpoint{1.977959in}{2.502254in}}%
\pgfpathcurveto{\pgfqpoint{1.985664in}{2.509958in}}{\pgfqpoint{1.989992in}{2.520409in}}{\pgfqpoint{1.989992in}{2.531305in}}%
\pgfpathcurveto{\pgfqpoint{1.989992in}{2.542200in}}{\pgfqpoint{1.985664in}{2.552651in}}{\pgfqpoint{1.977959in}{2.560355in}}%
\pgfpathcurveto{\pgfqpoint{1.970255in}{2.568060in}}{\pgfqpoint{1.959804in}{2.572389in}}{\pgfqpoint{1.948909in}{2.572389in}}%
\pgfpathcurveto{\pgfqpoint{1.938013in}{2.572389in}}{\pgfqpoint{1.927562in}{2.568060in}}{\pgfqpoint{1.919858in}{2.560355in}}%
\pgfpathcurveto{\pgfqpoint{1.912153in}{2.552651in}}{\pgfqpoint{1.907825in}{2.542200in}}{\pgfqpoint{1.907825in}{2.531305in}}%
\pgfpathcurveto{\pgfqpoint{1.907825in}{2.520409in}}{\pgfqpoint{1.912153in}{2.509958in}}{\pgfqpoint{1.919858in}{2.502254in}}%
\pgfpathcurveto{\pgfqpoint{1.927562in}{2.494550in}}{\pgfqpoint{1.938013in}{2.490221in}}{\pgfqpoint{1.948909in}{2.490221in}}%
\pgfpathlineto{\pgfqpoint{1.948909in}{2.490221in}}%
\pgfpathclose%
\pgfusepath{stroke}%
\end{pgfscope}%
\begin{pgfscope}%
\pgfpathrectangle{\pgfqpoint{0.688192in}{0.670138in}}{\pgfqpoint{7.111808in}{5.061530in}}%
\pgfusepath{clip}%
\pgfsetbuttcap%
\pgfsetroundjoin%
\pgfsetlinewidth{1.003750pt}%
\definecolor{currentstroke}{rgb}{0.000000,0.000000,0.000000}%
\pgfsetstrokecolor{currentstroke}%
\pgfsetdash{}{0pt}%
\pgfpathmoveto{\pgfqpoint{0.939729in}{0.677113in}}%
\pgfpathcurveto{\pgfqpoint{0.950625in}{0.677113in}}{\pgfqpoint{0.961075in}{0.681442in}}{\pgfqpoint{0.968780in}{0.689146in}}%
\pgfpathcurveto{\pgfqpoint{0.976484in}{0.696850in}}{\pgfqpoint{0.980813in}{0.707301in}}{\pgfqpoint{0.980813in}{0.718197in}}%
\pgfpathcurveto{\pgfqpoint{0.980813in}{0.729092in}}{\pgfqpoint{0.976484in}{0.739543in}}{\pgfqpoint{0.968780in}{0.747247in}}%
\pgfpathcurveto{\pgfqpoint{0.961075in}{0.754952in}}{\pgfqpoint{0.950625in}{0.759281in}}{\pgfqpoint{0.939729in}{0.759281in}}%
\pgfpathcurveto{\pgfqpoint{0.928834in}{0.759281in}}{\pgfqpoint{0.918383in}{0.754952in}}{\pgfqpoint{0.910678in}{0.747247in}}%
\pgfpathcurveto{\pgfqpoint{0.902974in}{0.739543in}}{\pgfqpoint{0.898645in}{0.729092in}}{\pgfqpoint{0.898645in}{0.718197in}}%
\pgfpathcurveto{\pgfqpoint{0.898645in}{0.707301in}}{\pgfqpoint{0.902974in}{0.696850in}}{\pgfqpoint{0.910678in}{0.689146in}}%
\pgfpathcurveto{\pgfqpoint{0.918383in}{0.681442in}}{\pgfqpoint{0.928834in}{0.677113in}}{\pgfqpoint{0.939729in}{0.677113in}}%
\pgfpathlineto{\pgfqpoint{0.939729in}{0.677113in}}%
\pgfpathclose%
\pgfusepath{stroke}%
\end{pgfscope}%
\begin{pgfscope}%
\pgfpathrectangle{\pgfqpoint{0.688192in}{0.670138in}}{\pgfqpoint{7.111808in}{5.061530in}}%
\pgfusepath{clip}%
\pgfsetbuttcap%
\pgfsetroundjoin%
\pgfsetlinewidth{1.003750pt}%
\definecolor{currentstroke}{rgb}{0.000000,0.000000,0.000000}%
\pgfsetstrokecolor{currentstroke}%
\pgfsetdash{}{0pt}%
\pgfpathmoveto{\pgfqpoint{0.795978in}{0.759945in}}%
\pgfpathcurveto{\pgfqpoint{0.806874in}{0.759945in}}{\pgfqpoint{0.817325in}{0.764274in}}{\pgfqpoint{0.825029in}{0.771978in}}%
\pgfpathcurveto{\pgfqpoint{0.832733in}{0.779683in}}{\pgfqpoint{0.837062in}{0.790133in}}{\pgfqpoint{0.837062in}{0.801029in}}%
\pgfpathcurveto{\pgfqpoint{0.837062in}{0.811925in}}{\pgfqpoint{0.832733in}{0.822375in}}{\pgfqpoint{0.825029in}{0.830080in}}%
\pgfpathcurveto{\pgfqpoint{0.817325in}{0.837784in}}{\pgfqpoint{0.806874in}{0.842113in}}{\pgfqpoint{0.795978in}{0.842113in}}%
\pgfpathcurveto{\pgfqpoint{0.785083in}{0.842113in}}{\pgfqpoint{0.774632in}{0.837784in}}{\pgfqpoint{0.766928in}{0.830080in}}%
\pgfpathcurveto{\pgfqpoint{0.759223in}{0.822375in}}{\pgfqpoint{0.754894in}{0.811925in}}{\pgfqpoint{0.754894in}{0.801029in}}%
\pgfpathcurveto{\pgfqpoint{0.754894in}{0.790133in}}{\pgfqpoint{0.759223in}{0.779683in}}{\pgfqpoint{0.766928in}{0.771978in}}%
\pgfpathcurveto{\pgfqpoint{0.774632in}{0.764274in}}{\pgfqpoint{0.785083in}{0.759945in}}{\pgfqpoint{0.795978in}{0.759945in}}%
\pgfpathlineto{\pgfqpoint{0.795978in}{0.759945in}}%
\pgfpathclose%
\pgfusepath{stroke}%
\end{pgfscope}%
\begin{pgfscope}%
\pgfpathrectangle{\pgfqpoint{0.688192in}{0.670138in}}{\pgfqpoint{7.111808in}{5.061530in}}%
\pgfusepath{clip}%
\pgfsetbuttcap%
\pgfsetroundjoin%
\pgfsetlinewidth{1.003750pt}%
\definecolor{currentstroke}{rgb}{0.000000,0.000000,0.000000}%
\pgfsetstrokecolor{currentstroke}%
\pgfsetdash{}{0pt}%
\pgfpathmoveto{\pgfqpoint{0.749254in}{1.404065in}}%
\pgfpathcurveto{\pgfqpoint{0.760150in}{1.404065in}}{\pgfqpoint{0.770600in}{1.408394in}}{\pgfqpoint{0.778305in}{1.416099in}}%
\pgfpathcurveto{\pgfqpoint{0.786009in}{1.423803in}}{\pgfqpoint{0.790338in}{1.434254in}}{\pgfqpoint{0.790338in}{1.445149in}}%
\pgfpathcurveto{\pgfqpoint{0.790338in}{1.456045in}}{\pgfqpoint{0.786009in}{1.466496in}}{\pgfqpoint{0.778305in}{1.474200in}}%
\pgfpathcurveto{\pgfqpoint{0.770600in}{1.481904in}}{\pgfqpoint{0.760150in}{1.486233in}}{\pgfqpoint{0.749254in}{1.486233in}}%
\pgfpathcurveto{\pgfqpoint{0.738358in}{1.486233in}}{\pgfqpoint{0.727908in}{1.481904in}}{\pgfqpoint{0.720203in}{1.474200in}}%
\pgfpathcurveto{\pgfqpoint{0.712499in}{1.466496in}}{\pgfqpoint{0.708170in}{1.456045in}}{\pgfqpoint{0.708170in}{1.445149in}}%
\pgfpathcurveto{\pgfqpoint{0.708170in}{1.434254in}}{\pgfqpoint{0.712499in}{1.423803in}}{\pgfqpoint{0.720203in}{1.416099in}}%
\pgfpathcurveto{\pgfqpoint{0.727908in}{1.408394in}}{\pgfqpoint{0.738358in}{1.404065in}}{\pgfqpoint{0.749254in}{1.404065in}}%
\pgfpathlineto{\pgfqpoint{0.749254in}{1.404065in}}%
\pgfpathclose%
\pgfusepath{stroke}%
\end{pgfscope}%
\begin{pgfscope}%
\pgfpathrectangle{\pgfqpoint{0.688192in}{0.670138in}}{\pgfqpoint{7.111808in}{5.061530in}}%
\pgfusepath{clip}%
\pgfsetbuttcap%
\pgfsetroundjoin%
\pgfsetlinewidth{1.003750pt}%
\definecolor{currentstroke}{rgb}{0.000000,0.000000,0.000000}%
\pgfsetstrokecolor{currentstroke}%
\pgfsetdash{}{0pt}%
\pgfpathmoveto{\pgfqpoint{1.444241in}{0.642951in}}%
\pgfpathcurveto{\pgfqpoint{1.455136in}{0.642951in}}{\pgfqpoint{1.465587in}{0.647280in}}{\pgfqpoint{1.473291in}{0.654984in}}%
\pgfpathcurveto{\pgfqpoint{1.480996in}{0.662688in}}{\pgfqpoint{1.485325in}{0.673139in}}{\pgfqpoint{1.485325in}{0.684035in}}%
\pgfpathcurveto{\pgfqpoint{1.485325in}{0.694930in}}{\pgfqpoint{1.480996in}{0.705381in}}{\pgfqpoint{1.473291in}{0.713085in}}%
\pgfpathcurveto{\pgfqpoint{1.465587in}{0.720790in}}{\pgfqpoint{1.455136in}{0.725119in}}{\pgfqpoint{1.444241in}{0.725119in}}%
\pgfpathcurveto{\pgfqpoint{1.433345in}{0.725119in}}{\pgfqpoint{1.422894in}{0.720790in}}{\pgfqpoint{1.415190in}{0.713085in}}%
\pgfpathcurveto{\pgfqpoint{1.407486in}{0.705381in}}{\pgfqpoint{1.403157in}{0.694930in}}{\pgfqpoint{1.403157in}{0.684035in}}%
\pgfpathcurveto{\pgfqpoint{1.403157in}{0.673139in}}{\pgfqpoint{1.407486in}{0.662688in}}{\pgfqpoint{1.415190in}{0.654984in}}%
\pgfpathcurveto{\pgfqpoint{1.422894in}{0.647280in}}{\pgfqpoint{1.433345in}{0.642951in}}{\pgfqpoint{1.444241in}{0.642951in}}%
\pgfusepath{stroke}%
\end{pgfscope}%
\begin{pgfscope}%
\pgfpathrectangle{\pgfqpoint{0.688192in}{0.670138in}}{\pgfqpoint{7.111808in}{5.061530in}}%
\pgfusepath{clip}%
\pgfsetbuttcap%
\pgfsetroundjoin%
\pgfsetlinewidth{1.003750pt}%
\definecolor{currentstroke}{rgb}{0.000000,0.000000,0.000000}%
\pgfsetstrokecolor{currentstroke}%
\pgfsetdash{}{0pt}%
\pgfpathmoveto{\pgfqpoint{1.148995in}{0.644808in}}%
\pgfpathcurveto{\pgfqpoint{1.159890in}{0.644808in}}{\pgfqpoint{1.170341in}{0.649137in}}{\pgfqpoint{1.178045in}{0.656841in}}%
\pgfpathcurveto{\pgfqpoint{1.185750in}{0.664545in}}{\pgfqpoint{1.190079in}{0.674996in}}{\pgfqpoint{1.190079in}{0.685892in}}%
\pgfpathcurveto{\pgfqpoint{1.190079in}{0.696787in}}{\pgfqpoint{1.185750in}{0.707238in}}{\pgfqpoint{1.178045in}{0.714942in}}%
\pgfpathcurveto{\pgfqpoint{1.170341in}{0.722647in}}{\pgfqpoint{1.159890in}{0.726975in}}{\pgfqpoint{1.148995in}{0.726975in}}%
\pgfpathcurveto{\pgfqpoint{1.138099in}{0.726975in}}{\pgfqpoint{1.127648in}{0.722647in}}{\pgfqpoint{1.119944in}{0.714942in}}%
\pgfpathcurveto{\pgfqpoint{1.112240in}{0.707238in}}{\pgfqpoint{1.107911in}{0.696787in}}{\pgfqpoint{1.107911in}{0.685892in}}%
\pgfpathcurveto{\pgfqpoint{1.107911in}{0.674996in}}{\pgfqpoint{1.112240in}{0.664545in}}{\pgfqpoint{1.119944in}{0.656841in}}%
\pgfpathcurveto{\pgfqpoint{1.127648in}{0.649137in}}{\pgfqpoint{1.138099in}{0.644808in}}{\pgfqpoint{1.148995in}{0.644808in}}%
\pgfusepath{stroke}%
\end{pgfscope}%
\begin{pgfscope}%
\pgfpathrectangle{\pgfqpoint{0.688192in}{0.670138in}}{\pgfqpoint{7.111808in}{5.061530in}}%
\pgfusepath{clip}%
\pgfsetbuttcap%
\pgfsetroundjoin%
\pgfsetlinewidth{1.003750pt}%
\definecolor{currentstroke}{rgb}{0.000000,0.000000,0.000000}%
\pgfsetstrokecolor{currentstroke}%
\pgfsetdash{}{0pt}%
\pgfpathmoveto{\pgfqpoint{4.965126in}{0.630283in}}%
\pgfpathcurveto{\pgfqpoint{4.976022in}{0.630283in}}{\pgfqpoint{4.986472in}{0.634611in}}{\pgfqpoint{4.994177in}{0.642316in}}%
\pgfpathcurveto{\pgfqpoint{5.001881in}{0.650020in}}{\pgfqpoint{5.006210in}{0.660471in}}{\pgfqpoint{5.006210in}{0.671366in}}%
\pgfpathcurveto{\pgfqpoint{5.006210in}{0.682262in}}{\pgfqpoint{5.001881in}{0.692713in}}{\pgfqpoint{4.994177in}{0.700417in}}%
\pgfpathcurveto{\pgfqpoint{4.986472in}{0.708122in}}{\pgfqpoint{4.976022in}{0.712450in}}{\pgfqpoint{4.965126in}{0.712450in}}%
\pgfpathcurveto{\pgfqpoint{4.954230in}{0.712450in}}{\pgfqpoint{4.943780in}{0.708122in}}{\pgfqpoint{4.936075in}{0.700417in}}%
\pgfpathcurveto{\pgfqpoint{4.928371in}{0.692713in}}{\pgfqpoint{4.924042in}{0.682262in}}{\pgfqpoint{4.924042in}{0.671366in}}%
\pgfpathcurveto{\pgfqpoint{4.924042in}{0.660471in}}{\pgfqpoint{4.928371in}{0.650020in}}{\pgfqpoint{4.936075in}{0.642316in}}%
\pgfpathcurveto{\pgfqpoint{4.943780in}{0.634611in}}{\pgfqpoint{4.954230in}{0.630283in}}{\pgfqpoint{4.965126in}{0.630283in}}%
\pgfusepath{stroke}%
\end{pgfscope}%
\begin{pgfscope}%
\pgfpathrectangle{\pgfqpoint{0.688192in}{0.670138in}}{\pgfqpoint{7.111808in}{5.061530in}}%
\pgfusepath{clip}%
\pgfsetbuttcap%
\pgfsetroundjoin%
\pgfsetlinewidth{1.003750pt}%
\definecolor{currentstroke}{rgb}{0.000000,0.000000,0.000000}%
\pgfsetstrokecolor{currentstroke}%
\pgfsetdash{}{0pt}%
\pgfpathmoveto{\pgfqpoint{1.681839in}{0.641743in}}%
\pgfpathcurveto{\pgfqpoint{1.692735in}{0.641743in}}{\pgfqpoint{1.703185in}{0.646071in}}{\pgfqpoint{1.710890in}{0.653776in}}%
\pgfpathcurveto{\pgfqpoint{1.718594in}{0.661480in}}{\pgfqpoint{1.722923in}{0.671931in}}{\pgfqpoint{1.722923in}{0.682826in}}%
\pgfpathcurveto{\pgfqpoint{1.722923in}{0.693722in}}{\pgfqpoint{1.718594in}{0.704173in}}{\pgfqpoint{1.710890in}{0.711877in}}%
\pgfpathcurveto{\pgfqpoint{1.703185in}{0.719581in}}{\pgfqpoint{1.692735in}{0.723910in}}{\pgfqpoint{1.681839in}{0.723910in}}%
\pgfpathcurveto{\pgfqpoint{1.670943in}{0.723910in}}{\pgfqpoint{1.660493in}{0.719581in}}{\pgfqpoint{1.652788in}{0.711877in}}%
\pgfpathcurveto{\pgfqpoint{1.645084in}{0.704173in}}{\pgfqpoint{1.640755in}{0.693722in}}{\pgfqpoint{1.640755in}{0.682826in}}%
\pgfpathcurveto{\pgfqpoint{1.640755in}{0.671931in}}{\pgfqpoint{1.645084in}{0.661480in}}{\pgfqpoint{1.652788in}{0.653776in}}%
\pgfpathcurveto{\pgfqpoint{1.660493in}{0.646071in}}{\pgfqpoint{1.670943in}{0.641743in}}{\pgfqpoint{1.681839in}{0.641743in}}%
\pgfusepath{stroke}%
\end{pgfscope}%
\begin{pgfscope}%
\pgfpathrectangle{\pgfqpoint{0.688192in}{0.670138in}}{\pgfqpoint{7.111808in}{5.061530in}}%
\pgfusepath{clip}%
\pgfsetbuttcap%
\pgfsetroundjoin%
\pgfsetlinewidth{1.003750pt}%
\definecolor{currentstroke}{rgb}{0.000000,0.000000,0.000000}%
\pgfsetstrokecolor{currentstroke}%
\pgfsetdash{}{0pt}%
\pgfpathmoveto{\pgfqpoint{1.103303in}{0.648067in}}%
\pgfpathcurveto{\pgfqpoint{1.114198in}{0.648067in}}{\pgfqpoint{1.124649in}{0.652396in}}{\pgfqpoint{1.132353in}{0.660101in}}%
\pgfpathcurveto{\pgfqpoint{1.140058in}{0.667805in}}{\pgfqpoint{1.144386in}{0.678256in}}{\pgfqpoint{1.144386in}{0.689151in}}%
\pgfpathcurveto{\pgfqpoint{1.144386in}{0.700047in}}{\pgfqpoint{1.140058in}{0.710498in}}{\pgfqpoint{1.132353in}{0.718202in}}%
\pgfpathcurveto{\pgfqpoint{1.124649in}{0.725906in}}{\pgfqpoint{1.114198in}{0.730235in}}{\pgfqpoint{1.103303in}{0.730235in}}%
\pgfpathcurveto{\pgfqpoint{1.092407in}{0.730235in}}{\pgfqpoint{1.081956in}{0.725906in}}{\pgfqpoint{1.074252in}{0.718202in}}%
\pgfpathcurveto{\pgfqpoint{1.066548in}{0.710498in}}{\pgfqpoint{1.062219in}{0.700047in}}{\pgfqpoint{1.062219in}{0.689151in}}%
\pgfpathcurveto{\pgfqpoint{1.062219in}{0.678256in}}{\pgfqpoint{1.066548in}{0.667805in}}{\pgfqpoint{1.074252in}{0.660101in}}%
\pgfpathcurveto{\pgfqpoint{1.081956in}{0.652396in}}{\pgfqpoint{1.092407in}{0.648067in}}{\pgfqpoint{1.103303in}{0.648067in}}%
\pgfusepath{stroke}%
\end{pgfscope}%
\begin{pgfscope}%
\pgfpathrectangle{\pgfqpoint{0.688192in}{0.670138in}}{\pgfqpoint{7.111808in}{5.061530in}}%
\pgfusepath{clip}%
\pgfsetbuttcap%
\pgfsetroundjoin%
\pgfsetlinewidth{1.003750pt}%
\definecolor{currentstroke}{rgb}{0.000000,0.000000,0.000000}%
\pgfsetstrokecolor{currentstroke}%
\pgfsetdash{}{0pt}%
\pgfpathmoveto{\pgfqpoint{4.421026in}{0.631171in}}%
\pgfpathcurveto{\pgfqpoint{4.431921in}{0.631171in}}{\pgfqpoint{4.442372in}{0.635500in}}{\pgfqpoint{4.450077in}{0.643204in}}%
\pgfpathcurveto{\pgfqpoint{4.457781in}{0.650909in}}{\pgfqpoint{4.462110in}{0.661359in}}{\pgfqpoint{4.462110in}{0.672255in}}%
\pgfpathcurveto{\pgfqpoint{4.462110in}{0.683150in}}{\pgfqpoint{4.457781in}{0.693601in}}{\pgfqpoint{4.450077in}{0.701306in}}%
\pgfpathcurveto{\pgfqpoint{4.442372in}{0.709010in}}{\pgfqpoint{4.431921in}{0.713339in}}{\pgfqpoint{4.421026in}{0.713339in}}%
\pgfpathcurveto{\pgfqpoint{4.410130in}{0.713339in}}{\pgfqpoint{4.399680in}{0.709010in}}{\pgfqpoint{4.391975in}{0.701306in}}%
\pgfpathcurveto{\pgfqpoint{4.384271in}{0.693601in}}{\pgfqpoint{4.379942in}{0.683150in}}{\pgfqpoint{4.379942in}{0.672255in}}%
\pgfpathcurveto{\pgfqpoint{4.379942in}{0.661359in}}{\pgfqpoint{4.384271in}{0.650909in}}{\pgfqpoint{4.391975in}{0.643204in}}%
\pgfpathcurveto{\pgfqpoint{4.399680in}{0.635500in}}{\pgfqpoint{4.410130in}{0.631171in}}{\pgfqpoint{4.421026in}{0.631171in}}%
\pgfusepath{stroke}%
\end{pgfscope}%
\begin{pgfscope}%
\pgfpathrectangle{\pgfqpoint{0.688192in}{0.670138in}}{\pgfqpoint{7.111808in}{5.061530in}}%
\pgfusepath{clip}%
\pgfsetbuttcap%
\pgfsetroundjoin%
\pgfsetlinewidth{1.003750pt}%
\definecolor{currentstroke}{rgb}{0.000000,0.000000,0.000000}%
\pgfsetstrokecolor{currentstroke}%
\pgfsetdash{}{0pt}%
\pgfpathmoveto{\pgfqpoint{1.970905in}{0.639806in}}%
\pgfpathcurveto{\pgfqpoint{1.981801in}{0.639806in}}{\pgfqpoint{1.992251in}{0.644135in}}{\pgfqpoint{1.999956in}{0.651839in}}%
\pgfpathcurveto{\pgfqpoint{2.007660in}{0.659543in}}{\pgfqpoint{2.011989in}{0.669994in}}{\pgfqpoint{2.011989in}{0.680890in}}%
\pgfpathcurveto{\pgfqpoint{2.011989in}{0.691785in}}{\pgfqpoint{2.007660in}{0.702236in}}{\pgfqpoint{1.999956in}{0.709941in}}%
\pgfpathcurveto{\pgfqpoint{1.992251in}{0.717645in}}{\pgfqpoint{1.981801in}{0.721974in}}{\pgfqpoint{1.970905in}{0.721974in}}%
\pgfpathcurveto{\pgfqpoint{1.960009in}{0.721974in}}{\pgfqpoint{1.949559in}{0.717645in}}{\pgfqpoint{1.941854in}{0.709941in}}%
\pgfpathcurveto{\pgfqpoint{1.934150in}{0.702236in}}{\pgfqpoint{1.929821in}{0.691785in}}{\pgfqpoint{1.929821in}{0.680890in}}%
\pgfpathcurveto{\pgfqpoint{1.929821in}{0.669994in}}{\pgfqpoint{1.934150in}{0.659543in}}{\pgfqpoint{1.941854in}{0.651839in}}%
\pgfpathcurveto{\pgfqpoint{1.949559in}{0.644135in}}{\pgfqpoint{1.960009in}{0.639806in}}{\pgfqpoint{1.970905in}{0.639806in}}%
\pgfusepath{stroke}%
\end{pgfscope}%
\begin{pgfscope}%
\pgfpathrectangle{\pgfqpoint{0.688192in}{0.670138in}}{\pgfqpoint{7.111808in}{5.061530in}}%
\pgfusepath{clip}%
\pgfsetbuttcap%
\pgfsetroundjoin%
\pgfsetlinewidth{1.003750pt}%
\definecolor{currentstroke}{rgb}{0.000000,0.000000,0.000000}%
\pgfsetstrokecolor{currentstroke}%
\pgfsetdash{}{0pt}%
\pgfpathmoveto{\pgfqpoint{0.827191in}{0.718716in}}%
\pgfpathcurveto{\pgfqpoint{0.838087in}{0.718716in}}{\pgfqpoint{0.848538in}{0.723045in}}{\pgfqpoint{0.856242in}{0.730749in}}%
\pgfpathcurveto{\pgfqpoint{0.863946in}{0.738453in}}{\pgfqpoint{0.868275in}{0.748904in}}{\pgfqpoint{0.868275in}{0.759800in}}%
\pgfpathcurveto{\pgfqpoint{0.868275in}{0.770695in}}{\pgfqpoint{0.863946in}{0.781146in}}{\pgfqpoint{0.856242in}{0.788850in}}%
\pgfpathcurveto{\pgfqpoint{0.848538in}{0.796555in}}{\pgfqpoint{0.838087in}{0.800884in}}{\pgfqpoint{0.827191in}{0.800884in}}%
\pgfpathcurveto{\pgfqpoint{0.816296in}{0.800884in}}{\pgfqpoint{0.805845in}{0.796555in}}{\pgfqpoint{0.798140in}{0.788850in}}%
\pgfpathcurveto{\pgfqpoint{0.790436in}{0.781146in}}{\pgfqpoint{0.786107in}{0.770695in}}{\pgfqpoint{0.786107in}{0.759800in}}%
\pgfpathcurveto{\pgfqpoint{0.786107in}{0.748904in}}{\pgfqpoint{0.790436in}{0.738453in}}{\pgfqpoint{0.798140in}{0.730749in}}%
\pgfpathcurveto{\pgfqpoint{0.805845in}{0.723045in}}{\pgfqpoint{0.816296in}{0.718716in}}{\pgfqpoint{0.827191in}{0.718716in}}%
\pgfpathlineto{\pgfqpoint{0.827191in}{0.718716in}}%
\pgfpathclose%
\pgfusepath{stroke}%
\end{pgfscope}%
\begin{pgfscope}%
\pgfpathrectangle{\pgfqpoint{0.688192in}{0.670138in}}{\pgfqpoint{7.111808in}{5.061530in}}%
\pgfusepath{clip}%
\pgfsetbuttcap%
\pgfsetroundjoin%
\pgfsetlinewidth{1.003750pt}%
\definecolor{currentstroke}{rgb}{0.000000,0.000000,0.000000}%
\pgfsetstrokecolor{currentstroke}%
\pgfsetdash{}{0pt}%
\pgfpathmoveto{\pgfqpoint{1.503177in}{0.642365in}}%
\pgfpathcurveto{\pgfqpoint{1.514073in}{0.642365in}}{\pgfqpoint{1.524523in}{0.646694in}}{\pgfqpoint{1.532228in}{0.654398in}}%
\pgfpathcurveto{\pgfqpoint{1.539932in}{0.662103in}}{\pgfqpoint{1.544261in}{0.672553in}}{\pgfqpoint{1.544261in}{0.683449in}}%
\pgfpathcurveto{\pgfqpoint{1.544261in}{0.694345in}}{\pgfqpoint{1.539932in}{0.704795in}}{\pgfqpoint{1.532228in}{0.712500in}}%
\pgfpathcurveto{\pgfqpoint{1.524523in}{0.720204in}}{\pgfqpoint{1.514073in}{0.724533in}}{\pgfqpoint{1.503177in}{0.724533in}}%
\pgfpathcurveto{\pgfqpoint{1.492281in}{0.724533in}}{\pgfqpoint{1.481831in}{0.720204in}}{\pgfqpoint{1.474126in}{0.712500in}}%
\pgfpathcurveto{\pgfqpoint{1.466422in}{0.704795in}}{\pgfqpoint{1.462093in}{0.694345in}}{\pgfqpoint{1.462093in}{0.683449in}}%
\pgfpathcurveto{\pgfqpoint{1.462093in}{0.672553in}}{\pgfqpoint{1.466422in}{0.662103in}}{\pgfqpoint{1.474126in}{0.654398in}}%
\pgfpathcurveto{\pgfqpoint{1.481831in}{0.646694in}}{\pgfqpoint{1.492281in}{0.642365in}}{\pgfqpoint{1.503177in}{0.642365in}}%
\pgfusepath{stroke}%
\end{pgfscope}%
\begin{pgfscope}%
\pgfpathrectangle{\pgfqpoint{0.688192in}{0.670138in}}{\pgfqpoint{7.111808in}{5.061530in}}%
\pgfusepath{clip}%
\pgfsetbuttcap%
\pgfsetroundjoin%
\pgfsetlinewidth{1.003750pt}%
\definecolor{currentstroke}{rgb}{0.000000,0.000000,0.000000}%
\pgfsetstrokecolor{currentstroke}%
\pgfsetdash{}{0pt}%
\pgfpathmoveto{\pgfqpoint{2.439145in}{4.149268in}}%
\pgfpathcurveto{\pgfqpoint{2.450040in}{4.149268in}}{\pgfqpoint{2.460491in}{4.153597in}}{\pgfqpoint{2.468195in}{4.161301in}}%
\pgfpathcurveto{\pgfqpoint{2.475900in}{4.169006in}}{\pgfqpoint{2.480229in}{4.179456in}}{\pgfqpoint{2.480229in}{4.190352in}}%
\pgfpathcurveto{\pgfqpoint{2.480229in}{4.201247in}}{\pgfqpoint{2.475900in}{4.211698in}}{\pgfqpoint{2.468195in}{4.219403in}}%
\pgfpathcurveto{\pgfqpoint{2.460491in}{4.227107in}}{\pgfqpoint{2.450040in}{4.231436in}}{\pgfqpoint{2.439145in}{4.231436in}}%
\pgfpathcurveto{\pgfqpoint{2.428249in}{4.231436in}}{\pgfqpoint{2.417798in}{4.227107in}}{\pgfqpoint{2.410094in}{4.219403in}}%
\pgfpathcurveto{\pgfqpoint{2.402390in}{4.211698in}}{\pgfqpoint{2.398061in}{4.201247in}}{\pgfqpoint{2.398061in}{4.190352in}}%
\pgfpathcurveto{\pgfqpoint{2.398061in}{4.179456in}}{\pgfqpoint{2.402390in}{4.169006in}}{\pgfqpoint{2.410094in}{4.161301in}}%
\pgfpathcurveto{\pgfqpoint{2.417798in}{4.153597in}}{\pgfqpoint{2.428249in}{4.149268in}}{\pgfqpoint{2.439145in}{4.149268in}}%
\pgfpathlineto{\pgfqpoint{2.439145in}{4.149268in}}%
\pgfpathclose%
\pgfusepath{stroke}%
\end{pgfscope}%
\begin{pgfscope}%
\pgfpathrectangle{\pgfqpoint{0.688192in}{0.670138in}}{\pgfqpoint{7.111808in}{5.061530in}}%
\pgfusepath{clip}%
\pgfsetbuttcap%
\pgfsetroundjoin%
\pgfsetlinewidth{1.003750pt}%
\definecolor{currentstroke}{rgb}{0.000000,0.000000,0.000000}%
\pgfsetstrokecolor{currentstroke}%
\pgfsetdash{}{0pt}%
\pgfpathmoveto{\pgfqpoint{4.576929in}{0.631060in}}%
\pgfpathcurveto{\pgfqpoint{4.587825in}{0.631060in}}{\pgfqpoint{4.598276in}{0.635389in}}{\pgfqpoint{4.605980in}{0.643094in}}%
\pgfpathcurveto{\pgfqpoint{4.613684in}{0.650798in}}{\pgfqpoint{4.618013in}{0.661249in}}{\pgfqpoint{4.618013in}{0.672144in}}%
\pgfpathcurveto{\pgfqpoint{4.618013in}{0.683040in}}{\pgfqpoint{4.613684in}{0.693491in}}{\pgfqpoint{4.605980in}{0.701195in}}%
\pgfpathcurveto{\pgfqpoint{4.598276in}{0.708899in}}{\pgfqpoint{4.587825in}{0.713228in}}{\pgfqpoint{4.576929in}{0.713228in}}%
\pgfpathcurveto{\pgfqpoint{4.566034in}{0.713228in}}{\pgfqpoint{4.555583in}{0.708899in}}{\pgfqpoint{4.547878in}{0.701195in}}%
\pgfpathcurveto{\pgfqpoint{4.540174in}{0.693491in}}{\pgfqpoint{4.535845in}{0.683040in}}{\pgfqpoint{4.535845in}{0.672144in}}%
\pgfpathcurveto{\pgfqpoint{4.535845in}{0.661249in}}{\pgfqpoint{4.540174in}{0.650798in}}{\pgfqpoint{4.547878in}{0.643094in}}%
\pgfpathcurveto{\pgfqpoint{4.555583in}{0.635389in}}{\pgfqpoint{4.566034in}{0.631060in}}{\pgfqpoint{4.576929in}{0.631060in}}%
\pgfusepath{stroke}%
\end{pgfscope}%
\begin{pgfscope}%
\pgfpathrectangle{\pgfqpoint{0.688192in}{0.670138in}}{\pgfqpoint{7.111808in}{5.061530in}}%
\pgfusepath{clip}%
\pgfsetbuttcap%
\pgfsetroundjoin%
\pgfsetlinewidth{1.003750pt}%
\definecolor{currentstroke}{rgb}{0.000000,0.000000,0.000000}%
\pgfsetstrokecolor{currentstroke}%
\pgfsetdash{}{0pt}%
\pgfpathmoveto{\pgfqpoint{1.473926in}{0.642857in}}%
\pgfpathcurveto{\pgfqpoint{1.484821in}{0.642857in}}{\pgfqpoint{1.495272in}{0.647186in}}{\pgfqpoint{1.502976in}{0.654890in}}%
\pgfpathcurveto{\pgfqpoint{1.510681in}{0.662595in}}{\pgfqpoint{1.515009in}{0.673045in}}{\pgfqpoint{1.515009in}{0.683941in}}%
\pgfpathcurveto{\pgfqpoint{1.515009in}{0.694837in}}{\pgfqpoint{1.510681in}{0.705287in}}{\pgfqpoint{1.502976in}{0.712992in}}%
\pgfpathcurveto{\pgfqpoint{1.495272in}{0.720696in}}{\pgfqpoint{1.484821in}{0.725025in}}{\pgfqpoint{1.473926in}{0.725025in}}%
\pgfpathcurveto{\pgfqpoint{1.463030in}{0.725025in}}{\pgfqpoint{1.452579in}{0.720696in}}{\pgfqpoint{1.444875in}{0.712992in}}%
\pgfpathcurveto{\pgfqpoint{1.437171in}{0.705287in}}{\pgfqpoint{1.432842in}{0.694837in}}{\pgfqpoint{1.432842in}{0.683941in}}%
\pgfpathcurveto{\pgfqpoint{1.432842in}{0.673045in}}{\pgfqpoint{1.437171in}{0.662595in}}{\pgfqpoint{1.444875in}{0.654890in}}%
\pgfpathcurveto{\pgfqpoint{1.452579in}{0.647186in}}{\pgfqpoint{1.463030in}{0.642857in}}{\pgfqpoint{1.473926in}{0.642857in}}%
\pgfusepath{stroke}%
\end{pgfscope}%
\begin{pgfscope}%
\pgfpathrectangle{\pgfqpoint{0.688192in}{0.670138in}}{\pgfqpoint{7.111808in}{5.061530in}}%
\pgfusepath{clip}%
\pgfsetbuttcap%
\pgfsetroundjoin%
\pgfsetlinewidth{1.003750pt}%
\definecolor{currentstroke}{rgb}{0.000000,0.000000,0.000000}%
\pgfsetstrokecolor{currentstroke}%
\pgfsetdash{}{0pt}%
\pgfpathmoveto{\pgfqpoint{4.304331in}{1.472410in}}%
\pgfpathcurveto{\pgfqpoint{4.315227in}{1.472410in}}{\pgfqpoint{4.325678in}{1.476739in}}{\pgfqpoint{4.333382in}{1.484443in}}%
\pgfpathcurveto{\pgfqpoint{4.341086in}{1.492147in}}{\pgfqpoint{4.345415in}{1.502598in}}{\pgfqpoint{4.345415in}{1.513494in}}%
\pgfpathcurveto{\pgfqpoint{4.345415in}{1.524389in}}{\pgfqpoint{4.341086in}{1.534840in}}{\pgfqpoint{4.333382in}{1.542544in}}%
\pgfpathcurveto{\pgfqpoint{4.325678in}{1.550249in}}{\pgfqpoint{4.315227in}{1.554577in}}{\pgfqpoint{4.304331in}{1.554577in}}%
\pgfpathcurveto{\pgfqpoint{4.293436in}{1.554577in}}{\pgfqpoint{4.282985in}{1.550249in}}{\pgfqpoint{4.275281in}{1.542544in}}%
\pgfpathcurveto{\pgfqpoint{4.267576in}{1.534840in}}{\pgfqpoint{4.263247in}{1.524389in}}{\pgfqpoint{4.263247in}{1.513494in}}%
\pgfpathcurveto{\pgfqpoint{4.263247in}{1.502598in}}{\pgfqpoint{4.267576in}{1.492147in}}{\pgfqpoint{4.275281in}{1.484443in}}%
\pgfpathcurveto{\pgfqpoint{4.282985in}{1.476739in}}{\pgfqpoint{4.293436in}{1.472410in}}{\pgfqpoint{4.304331in}{1.472410in}}%
\pgfpathlineto{\pgfqpoint{4.304331in}{1.472410in}}%
\pgfpathclose%
\pgfusepath{stroke}%
\end{pgfscope}%
\begin{pgfscope}%
\pgfpathrectangle{\pgfqpoint{0.688192in}{0.670138in}}{\pgfqpoint{7.111808in}{5.061530in}}%
\pgfusepath{clip}%
\pgfsetbuttcap%
\pgfsetroundjoin%
\pgfsetlinewidth{1.003750pt}%
\definecolor{currentstroke}{rgb}{0.000000,0.000000,0.000000}%
\pgfsetstrokecolor{currentstroke}%
\pgfsetdash{}{0pt}%
\pgfpathmoveto{\pgfqpoint{3.396005in}{0.769646in}}%
\pgfpathcurveto{\pgfqpoint{3.406901in}{0.769646in}}{\pgfqpoint{3.417352in}{0.773975in}}{\pgfqpoint{3.425056in}{0.781679in}}%
\pgfpathcurveto{\pgfqpoint{3.432761in}{0.789383in}}{\pgfqpoint{3.437089in}{0.799834in}}{\pgfqpoint{3.437089in}{0.810730in}}%
\pgfpathcurveto{\pgfqpoint{3.437089in}{0.821625in}}{\pgfqpoint{3.432761in}{0.832076in}}{\pgfqpoint{3.425056in}{0.839781in}}%
\pgfpathcurveto{\pgfqpoint{3.417352in}{0.847485in}}{\pgfqpoint{3.406901in}{0.851814in}}{\pgfqpoint{3.396005in}{0.851814in}}%
\pgfpathcurveto{\pgfqpoint{3.385110in}{0.851814in}}{\pgfqpoint{3.374659in}{0.847485in}}{\pgfqpoint{3.366955in}{0.839781in}}%
\pgfpathcurveto{\pgfqpoint{3.359250in}{0.832076in}}{\pgfqpoint{3.354922in}{0.821625in}}{\pgfqpoint{3.354922in}{0.810730in}}%
\pgfpathcurveto{\pgfqpoint{3.354922in}{0.799834in}}{\pgfqpoint{3.359250in}{0.789383in}}{\pgfqpoint{3.366955in}{0.781679in}}%
\pgfpathcurveto{\pgfqpoint{3.374659in}{0.773975in}}{\pgfqpoint{3.385110in}{0.769646in}}{\pgfqpoint{3.396005in}{0.769646in}}%
\pgfpathlineto{\pgfqpoint{3.396005in}{0.769646in}}%
\pgfpathclose%
\pgfusepath{stroke}%
\end{pgfscope}%
\begin{pgfscope}%
\pgfpathrectangle{\pgfqpoint{0.688192in}{0.670138in}}{\pgfqpoint{7.111808in}{5.061530in}}%
\pgfusepath{clip}%
\pgfsetbuttcap%
\pgfsetroundjoin%
\pgfsetlinewidth{1.003750pt}%
\definecolor{currentstroke}{rgb}{0.000000,0.000000,0.000000}%
\pgfsetstrokecolor{currentstroke}%
\pgfsetdash{}{0pt}%
\pgfpathmoveto{\pgfqpoint{4.420185in}{4.806141in}}%
\pgfpathcurveto{\pgfqpoint{4.431080in}{4.806141in}}{\pgfqpoint{4.441531in}{4.810469in}}{\pgfqpoint{4.449235in}{4.818174in}}%
\pgfpathcurveto{\pgfqpoint{4.456940in}{4.825878in}}{\pgfqpoint{4.461269in}{4.836329in}}{\pgfqpoint{4.461269in}{4.847224in}}%
\pgfpathcurveto{\pgfqpoint{4.461269in}{4.858120in}}{\pgfqpoint{4.456940in}{4.868571in}}{\pgfqpoint{4.449235in}{4.876275in}}%
\pgfpathcurveto{\pgfqpoint{4.441531in}{4.883980in}}{\pgfqpoint{4.431080in}{4.888308in}}{\pgfqpoint{4.420185in}{4.888308in}}%
\pgfpathcurveto{\pgfqpoint{4.409289in}{4.888308in}}{\pgfqpoint{4.398838in}{4.883980in}}{\pgfqpoint{4.391134in}{4.876275in}}%
\pgfpathcurveto{\pgfqpoint{4.383430in}{4.868571in}}{\pgfqpoint{4.379101in}{4.858120in}}{\pgfqpoint{4.379101in}{4.847224in}}%
\pgfpathcurveto{\pgfqpoint{4.379101in}{4.836329in}}{\pgfqpoint{4.383430in}{4.825878in}}{\pgfqpoint{4.391134in}{4.818174in}}%
\pgfpathcurveto{\pgfqpoint{4.398838in}{4.810469in}}{\pgfqpoint{4.409289in}{4.806141in}}{\pgfqpoint{4.420185in}{4.806141in}}%
\pgfpathlineto{\pgfqpoint{4.420185in}{4.806141in}}%
\pgfpathclose%
\pgfusepath{stroke}%
\end{pgfscope}%
\begin{pgfscope}%
\pgfpathrectangle{\pgfqpoint{0.688192in}{0.670138in}}{\pgfqpoint{7.111808in}{5.061530in}}%
\pgfusepath{clip}%
\pgfsetbuttcap%
\pgfsetroundjoin%
\pgfsetlinewidth{1.003750pt}%
\definecolor{currentstroke}{rgb}{0.000000,0.000000,0.000000}%
\pgfsetstrokecolor{currentstroke}%
\pgfsetdash{}{0pt}%
\pgfpathmoveto{\pgfqpoint{1.126908in}{0.646085in}}%
\pgfpathcurveto{\pgfqpoint{1.137803in}{0.646085in}}{\pgfqpoint{1.148254in}{0.650414in}}{\pgfqpoint{1.155958in}{0.658118in}}%
\pgfpathcurveto{\pgfqpoint{1.163663in}{0.665822in}}{\pgfqpoint{1.167991in}{0.676273in}}{\pgfqpoint{1.167991in}{0.687169in}}%
\pgfpathcurveto{\pgfqpoint{1.167991in}{0.698064in}}{\pgfqpoint{1.163663in}{0.708515in}}{\pgfqpoint{1.155958in}{0.716220in}}%
\pgfpathcurveto{\pgfqpoint{1.148254in}{0.723924in}}{\pgfqpoint{1.137803in}{0.728253in}}{\pgfqpoint{1.126908in}{0.728253in}}%
\pgfpathcurveto{\pgfqpoint{1.116012in}{0.728253in}}{\pgfqpoint{1.105561in}{0.723924in}}{\pgfqpoint{1.097857in}{0.716220in}}%
\pgfpathcurveto{\pgfqpoint{1.090153in}{0.708515in}}{\pgfqpoint{1.085824in}{0.698064in}}{\pgfqpoint{1.085824in}{0.687169in}}%
\pgfpathcurveto{\pgfqpoint{1.085824in}{0.676273in}}{\pgfqpoint{1.090153in}{0.665822in}}{\pgfqpoint{1.097857in}{0.658118in}}%
\pgfpathcurveto{\pgfqpoint{1.105561in}{0.650414in}}{\pgfqpoint{1.116012in}{0.646085in}}{\pgfqpoint{1.126908in}{0.646085in}}%
\pgfusepath{stroke}%
\end{pgfscope}%
\begin{pgfscope}%
\pgfpathrectangle{\pgfqpoint{0.688192in}{0.670138in}}{\pgfqpoint{7.111808in}{5.061530in}}%
\pgfusepath{clip}%
\pgfsetbuttcap%
\pgfsetroundjoin%
\pgfsetlinewidth{1.003750pt}%
\definecolor{currentstroke}{rgb}{0.000000,0.000000,0.000000}%
\pgfsetstrokecolor{currentstroke}%
\pgfsetdash{}{0pt}%
\pgfpathmoveto{\pgfqpoint{3.927110in}{3.524891in}}%
\pgfpathcurveto{\pgfqpoint{3.938005in}{3.524891in}}{\pgfqpoint{3.948456in}{3.529220in}}{\pgfqpoint{3.956160in}{3.536924in}}%
\pgfpathcurveto{\pgfqpoint{3.963865in}{3.544628in}}{\pgfqpoint{3.968194in}{3.555079in}}{\pgfqpoint{3.968194in}{3.565975in}}%
\pgfpathcurveto{\pgfqpoint{3.968194in}{3.576870in}}{\pgfqpoint{3.963865in}{3.587321in}}{\pgfqpoint{3.956160in}{3.595025in}}%
\pgfpathcurveto{\pgfqpoint{3.948456in}{3.602730in}}{\pgfqpoint{3.938005in}{3.607059in}}{\pgfqpoint{3.927110in}{3.607059in}}%
\pgfpathcurveto{\pgfqpoint{3.916214in}{3.607059in}}{\pgfqpoint{3.905763in}{3.602730in}}{\pgfqpoint{3.898059in}{3.595025in}}%
\pgfpathcurveto{\pgfqpoint{3.890355in}{3.587321in}}{\pgfqpoint{3.886026in}{3.576870in}}{\pgfqpoint{3.886026in}{3.565975in}}%
\pgfpathcurveto{\pgfqpoint{3.886026in}{3.555079in}}{\pgfqpoint{3.890355in}{3.544628in}}{\pgfqpoint{3.898059in}{3.536924in}}%
\pgfpathcurveto{\pgfqpoint{3.905763in}{3.529220in}}{\pgfqpoint{3.916214in}{3.524891in}}{\pgfqpoint{3.927110in}{3.524891in}}%
\pgfpathlineto{\pgfqpoint{3.927110in}{3.524891in}}%
\pgfpathclose%
\pgfusepath{stroke}%
\end{pgfscope}%
\begin{pgfscope}%
\pgfpathrectangle{\pgfqpoint{0.688192in}{0.670138in}}{\pgfqpoint{7.111808in}{5.061530in}}%
\pgfusepath{clip}%
\pgfsetbuttcap%
\pgfsetroundjoin%
\pgfsetlinewidth{1.003750pt}%
\definecolor{currentstroke}{rgb}{0.000000,0.000000,0.000000}%
\pgfsetstrokecolor{currentstroke}%
\pgfsetdash{}{0pt}%
\pgfpathmoveto{\pgfqpoint{4.615336in}{2.432243in}}%
\pgfpathcurveto{\pgfqpoint{4.626231in}{2.432243in}}{\pgfqpoint{4.636682in}{2.436572in}}{\pgfqpoint{4.644386in}{2.444277in}}%
\pgfpathcurveto{\pgfqpoint{4.652091in}{2.451981in}}{\pgfqpoint{4.656420in}{2.462432in}}{\pgfqpoint{4.656420in}{2.473327in}}%
\pgfpathcurveto{\pgfqpoint{4.656420in}{2.484223in}}{\pgfqpoint{4.652091in}{2.494674in}}{\pgfqpoint{4.644386in}{2.502378in}}%
\pgfpathcurveto{\pgfqpoint{4.636682in}{2.510082in}}{\pgfqpoint{4.626231in}{2.514411in}}{\pgfqpoint{4.615336in}{2.514411in}}%
\pgfpathcurveto{\pgfqpoint{4.604440in}{2.514411in}}{\pgfqpoint{4.593989in}{2.510082in}}{\pgfqpoint{4.586285in}{2.502378in}}%
\pgfpathcurveto{\pgfqpoint{4.578581in}{2.494674in}}{\pgfqpoint{4.574252in}{2.484223in}}{\pgfqpoint{4.574252in}{2.473327in}}%
\pgfpathcurveto{\pgfqpoint{4.574252in}{2.462432in}}{\pgfqpoint{4.578581in}{2.451981in}}{\pgfqpoint{4.586285in}{2.444277in}}%
\pgfpathcurveto{\pgfqpoint{4.593989in}{2.436572in}}{\pgfqpoint{4.604440in}{2.432243in}}{\pgfqpoint{4.615336in}{2.432243in}}%
\pgfpathlineto{\pgfqpoint{4.615336in}{2.432243in}}%
\pgfpathclose%
\pgfusepath{stroke}%
\end{pgfscope}%
\begin{pgfscope}%
\pgfpathrectangle{\pgfqpoint{0.688192in}{0.670138in}}{\pgfqpoint{7.111808in}{5.061530in}}%
\pgfusepath{clip}%
\pgfsetbuttcap%
\pgfsetroundjoin%
\pgfsetlinewidth{1.003750pt}%
\definecolor{currentstroke}{rgb}{0.000000,0.000000,0.000000}%
\pgfsetstrokecolor{currentstroke}%
\pgfsetdash{}{0pt}%
\pgfpathmoveto{\pgfqpoint{4.304331in}{1.472410in}}%
\pgfpathcurveto{\pgfqpoint{4.315227in}{1.472410in}}{\pgfqpoint{4.325678in}{1.476739in}}{\pgfqpoint{4.333382in}{1.484443in}}%
\pgfpathcurveto{\pgfqpoint{4.341086in}{1.492147in}}{\pgfqpoint{4.345415in}{1.502598in}}{\pgfqpoint{4.345415in}{1.513494in}}%
\pgfpathcurveto{\pgfqpoint{4.345415in}{1.524389in}}{\pgfqpoint{4.341086in}{1.534840in}}{\pgfqpoint{4.333382in}{1.542544in}}%
\pgfpathcurveto{\pgfqpoint{4.325678in}{1.550249in}}{\pgfqpoint{4.315227in}{1.554577in}}{\pgfqpoint{4.304331in}{1.554577in}}%
\pgfpathcurveto{\pgfqpoint{4.293436in}{1.554577in}}{\pgfqpoint{4.282985in}{1.550249in}}{\pgfqpoint{4.275281in}{1.542544in}}%
\pgfpathcurveto{\pgfqpoint{4.267576in}{1.534840in}}{\pgfqpoint{4.263247in}{1.524389in}}{\pgfqpoint{4.263247in}{1.513494in}}%
\pgfpathcurveto{\pgfqpoint{4.263247in}{1.502598in}}{\pgfqpoint{4.267576in}{1.492147in}}{\pgfqpoint{4.275281in}{1.484443in}}%
\pgfpathcurveto{\pgfqpoint{4.282985in}{1.476739in}}{\pgfqpoint{4.293436in}{1.472410in}}{\pgfqpoint{4.304331in}{1.472410in}}%
\pgfpathlineto{\pgfqpoint{4.304331in}{1.472410in}}%
\pgfpathclose%
\pgfusepath{stroke}%
\end{pgfscope}%
\begin{pgfscope}%
\pgfpathrectangle{\pgfqpoint{0.688192in}{0.670138in}}{\pgfqpoint{7.111808in}{5.061530in}}%
\pgfusepath{clip}%
\pgfsetbuttcap%
\pgfsetroundjoin%
\pgfsetlinewidth{1.003750pt}%
\definecolor{currentstroke}{rgb}{0.000000,0.000000,0.000000}%
\pgfsetstrokecolor{currentstroke}%
\pgfsetdash{}{0pt}%
\pgfpathmoveto{\pgfqpoint{1.178009in}{0.644363in}}%
\pgfpathcurveto{\pgfqpoint{1.188905in}{0.644363in}}{\pgfqpoint{1.199355in}{0.648692in}}{\pgfqpoint{1.207060in}{0.656396in}}%
\pgfpathcurveto{\pgfqpoint{1.214764in}{0.664100in}}{\pgfqpoint{1.219093in}{0.674551in}}{\pgfqpoint{1.219093in}{0.685447in}}%
\pgfpathcurveto{\pgfqpoint{1.219093in}{0.696342in}}{\pgfqpoint{1.214764in}{0.706793in}}{\pgfqpoint{1.207060in}{0.714497in}}%
\pgfpathcurveto{\pgfqpoint{1.199355in}{0.722202in}}{\pgfqpoint{1.188905in}{0.726531in}}{\pgfqpoint{1.178009in}{0.726531in}}%
\pgfpathcurveto{\pgfqpoint{1.167113in}{0.726531in}}{\pgfqpoint{1.156663in}{0.722202in}}{\pgfqpoint{1.148958in}{0.714497in}}%
\pgfpathcurveto{\pgfqpoint{1.141254in}{0.706793in}}{\pgfqpoint{1.136925in}{0.696342in}}{\pgfqpoint{1.136925in}{0.685447in}}%
\pgfpathcurveto{\pgfqpoint{1.136925in}{0.674551in}}{\pgfqpoint{1.141254in}{0.664100in}}{\pgfqpoint{1.148958in}{0.656396in}}%
\pgfpathcurveto{\pgfqpoint{1.156663in}{0.648692in}}{\pgfqpoint{1.167113in}{0.644363in}}{\pgfqpoint{1.178009in}{0.644363in}}%
\pgfusepath{stroke}%
\end{pgfscope}%
\begin{pgfscope}%
\pgfpathrectangle{\pgfqpoint{0.688192in}{0.670138in}}{\pgfqpoint{7.111808in}{5.061530in}}%
\pgfusepath{clip}%
\pgfsetbuttcap%
\pgfsetroundjoin%
\pgfsetlinewidth{1.003750pt}%
\definecolor{currentstroke}{rgb}{0.000000,0.000000,0.000000}%
\pgfsetstrokecolor{currentstroke}%
\pgfsetdash{}{0pt}%
\pgfpathmoveto{\pgfqpoint{1.969227in}{2.439076in}}%
\pgfpathcurveto{\pgfqpoint{1.980123in}{2.439076in}}{\pgfqpoint{1.990574in}{2.443405in}}{\pgfqpoint{1.998278in}{2.451109in}}%
\pgfpathcurveto{\pgfqpoint{2.005982in}{2.458813in}}{\pgfqpoint{2.010311in}{2.469264in}}{\pgfqpoint{2.010311in}{2.480160in}}%
\pgfpathcurveto{\pgfqpoint{2.010311in}{2.491055in}}{\pgfqpoint{2.005982in}{2.501506in}}{\pgfqpoint{1.998278in}{2.509211in}}%
\pgfpathcurveto{\pgfqpoint{1.990574in}{2.516915in}}{\pgfqpoint{1.980123in}{2.521244in}}{\pgfqpoint{1.969227in}{2.521244in}}%
\pgfpathcurveto{\pgfqpoint{1.958332in}{2.521244in}}{\pgfqpoint{1.947881in}{2.516915in}}{\pgfqpoint{1.940177in}{2.509211in}}%
\pgfpathcurveto{\pgfqpoint{1.932472in}{2.501506in}}{\pgfqpoint{1.928144in}{2.491055in}}{\pgfqpoint{1.928144in}{2.480160in}}%
\pgfpathcurveto{\pgfqpoint{1.928144in}{2.469264in}}{\pgfqpoint{1.932472in}{2.458813in}}{\pgfqpoint{1.940177in}{2.451109in}}%
\pgfpathcurveto{\pgfqpoint{1.947881in}{2.443405in}}{\pgfqpoint{1.958332in}{2.439076in}}{\pgfqpoint{1.969227in}{2.439076in}}%
\pgfpathlineto{\pgfqpoint{1.969227in}{2.439076in}}%
\pgfpathclose%
\pgfusepath{stroke}%
\end{pgfscope}%
\begin{pgfscope}%
\pgfpathrectangle{\pgfqpoint{0.688192in}{0.670138in}}{\pgfqpoint{7.111808in}{5.061530in}}%
\pgfusepath{clip}%
\pgfsetbuttcap%
\pgfsetroundjoin%
\pgfsetlinewidth{1.003750pt}%
\definecolor{currentstroke}{rgb}{0.000000,0.000000,0.000000}%
\pgfsetstrokecolor{currentstroke}%
\pgfsetdash{}{0pt}%
\pgfpathmoveto{\pgfqpoint{1.198965in}{0.644318in}}%
\pgfpathcurveto{\pgfqpoint{1.209860in}{0.644318in}}{\pgfqpoint{1.220311in}{0.648647in}}{\pgfqpoint{1.228016in}{0.656352in}}%
\pgfpathcurveto{\pgfqpoint{1.235720in}{0.664056in}}{\pgfqpoint{1.240049in}{0.674507in}}{\pgfqpoint{1.240049in}{0.685402in}}%
\pgfpathcurveto{\pgfqpoint{1.240049in}{0.696298in}}{\pgfqpoint{1.235720in}{0.706749in}}{\pgfqpoint{1.228016in}{0.714453in}}%
\pgfpathcurveto{\pgfqpoint{1.220311in}{0.722157in}}{\pgfqpoint{1.209860in}{0.726486in}}{\pgfqpoint{1.198965in}{0.726486in}}%
\pgfpathcurveto{\pgfqpoint{1.188069in}{0.726486in}}{\pgfqpoint{1.177619in}{0.722157in}}{\pgfqpoint{1.169914in}{0.714453in}}%
\pgfpathcurveto{\pgfqpoint{1.162210in}{0.706749in}}{\pgfqpoint{1.157881in}{0.696298in}}{\pgfqpoint{1.157881in}{0.685402in}}%
\pgfpathcurveto{\pgfqpoint{1.157881in}{0.674507in}}{\pgfqpoint{1.162210in}{0.664056in}}{\pgfqpoint{1.169914in}{0.656352in}}%
\pgfpathcurveto{\pgfqpoint{1.177619in}{0.648647in}}{\pgfqpoint{1.188069in}{0.644318in}}{\pgfqpoint{1.198965in}{0.644318in}}%
\pgfusepath{stroke}%
\end{pgfscope}%
\begin{pgfscope}%
\pgfpathrectangle{\pgfqpoint{0.688192in}{0.670138in}}{\pgfqpoint{7.111808in}{5.061530in}}%
\pgfusepath{clip}%
\pgfsetbuttcap%
\pgfsetroundjoin%
\pgfsetlinewidth{1.003750pt}%
\definecolor{currentstroke}{rgb}{0.000000,0.000000,0.000000}%
\pgfsetstrokecolor{currentstroke}%
\pgfsetdash{}{0pt}%
\pgfpathmoveto{\pgfqpoint{1.128867in}{0.645415in}}%
\pgfpathcurveto{\pgfqpoint{1.139762in}{0.645415in}}{\pgfqpoint{1.150213in}{0.649744in}}{\pgfqpoint{1.157917in}{0.657449in}}%
\pgfpathcurveto{\pgfqpoint{1.165622in}{0.665153in}}{\pgfqpoint{1.169951in}{0.675604in}}{\pgfqpoint{1.169951in}{0.686499in}}%
\pgfpathcurveto{\pgfqpoint{1.169951in}{0.697395in}}{\pgfqpoint{1.165622in}{0.707846in}}{\pgfqpoint{1.157917in}{0.715550in}}%
\pgfpathcurveto{\pgfqpoint{1.150213in}{0.723254in}}{\pgfqpoint{1.139762in}{0.727583in}}{\pgfqpoint{1.128867in}{0.727583in}}%
\pgfpathcurveto{\pgfqpoint{1.117971in}{0.727583in}}{\pgfqpoint{1.107520in}{0.723254in}}{\pgfqpoint{1.099816in}{0.715550in}}%
\pgfpathcurveto{\pgfqpoint{1.092112in}{0.707846in}}{\pgfqpoint{1.087783in}{0.697395in}}{\pgfqpoint{1.087783in}{0.686499in}}%
\pgfpathcurveto{\pgfqpoint{1.087783in}{0.675604in}}{\pgfqpoint{1.092112in}{0.665153in}}{\pgfqpoint{1.099816in}{0.657449in}}%
\pgfpathcurveto{\pgfqpoint{1.107520in}{0.649744in}}{\pgfqpoint{1.117971in}{0.645415in}}{\pgfqpoint{1.128867in}{0.645415in}}%
\pgfusepath{stroke}%
\end{pgfscope}%
\begin{pgfscope}%
\pgfpathrectangle{\pgfqpoint{0.688192in}{0.670138in}}{\pgfqpoint{7.111808in}{5.061530in}}%
\pgfusepath{clip}%
\pgfsetbuttcap%
\pgfsetroundjoin%
\pgfsetlinewidth{1.003750pt}%
\definecolor{currentstroke}{rgb}{0.000000,0.000000,0.000000}%
\pgfsetstrokecolor{currentstroke}%
\pgfsetdash{}{0pt}%
\pgfpathmoveto{\pgfqpoint{1.444241in}{0.642951in}}%
\pgfpathcurveto{\pgfqpoint{1.455136in}{0.642951in}}{\pgfqpoint{1.465587in}{0.647280in}}{\pgfqpoint{1.473291in}{0.654984in}}%
\pgfpathcurveto{\pgfqpoint{1.480996in}{0.662688in}}{\pgfqpoint{1.485325in}{0.673139in}}{\pgfqpoint{1.485325in}{0.684035in}}%
\pgfpathcurveto{\pgfqpoint{1.485325in}{0.694930in}}{\pgfqpoint{1.480996in}{0.705381in}}{\pgfqpoint{1.473291in}{0.713085in}}%
\pgfpathcurveto{\pgfqpoint{1.465587in}{0.720790in}}{\pgfqpoint{1.455136in}{0.725119in}}{\pgfqpoint{1.444241in}{0.725119in}}%
\pgfpathcurveto{\pgfqpoint{1.433345in}{0.725119in}}{\pgfqpoint{1.422894in}{0.720790in}}{\pgfqpoint{1.415190in}{0.713085in}}%
\pgfpathcurveto{\pgfqpoint{1.407486in}{0.705381in}}{\pgfqpoint{1.403157in}{0.694930in}}{\pgfqpoint{1.403157in}{0.684035in}}%
\pgfpathcurveto{\pgfqpoint{1.403157in}{0.673139in}}{\pgfqpoint{1.407486in}{0.662688in}}{\pgfqpoint{1.415190in}{0.654984in}}%
\pgfpathcurveto{\pgfqpoint{1.422894in}{0.647280in}}{\pgfqpoint{1.433345in}{0.642951in}}{\pgfqpoint{1.444241in}{0.642951in}}%
\pgfusepath{stroke}%
\end{pgfscope}%
\begin{pgfscope}%
\pgfpathrectangle{\pgfqpoint{0.688192in}{0.670138in}}{\pgfqpoint{7.111808in}{5.061530in}}%
\pgfusepath{clip}%
\pgfsetbuttcap%
\pgfsetroundjoin%
\pgfsetlinewidth{1.003750pt}%
\definecolor{currentstroke}{rgb}{0.000000,0.000000,0.000000}%
\pgfsetstrokecolor{currentstroke}%
\pgfsetdash{}{0pt}%
\pgfpathmoveto{\pgfqpoint{2.449592in}{3.982971in}}%
\pgfpathcurveto{\pgfqpoint{2.460487in}{3.982971in}}{\pgfqpoint{2.470938in}{3.987300in}}{\pgfqpoint{2.478642in}{3.995004in}}%
\pgfpathcurveto{\pgfqpoint{2.486347in}{4.002709in}}{\pgfqpoint{2.490675in}{4.013159in}}{\pgfqpoint{2.490675in}{4.024055in}}%
\pgfpathcurveto{\pgfqpoint{2.490675in}{4.034951in}}{\pgfqpoint{2.486347in}{4.045401in}}{\pgfqpoint{2.478642in}{4.053106in}}%
\pgfpathcurveto{\pgfqpoint{2.470938in}{4.060810in}}{\pgfqpoint{2.460487in}{4.065139in}}{\pgfqpoint{2.449592in}{4.065139in}}%
\pgfpathcurveto{\pgfqpoint{2.438696in}{4.065139in}}{\pgfqpoint{2.428245in}{4.060810in}}{\pgfqpoint{2.420541in}{4.053106in}}%
\pgfpathcurveto{\pgfqpoint{2.412836in}{4.045401in}}{\pgfqpoint{2.408508in}{4.034951in}}{\pgfqpoint{2.408508in}{4.024055in}}%
\pgfpathcurveto{\pgfqpoint{2.408508in}{4.013159in}}{\pgfqpoint{2.412836in}{4.002709in}}{\pgfqpoint{2.420541in}{3.995004in}}%
\pgfpathcurveto{\pgfqpoint{2.428245in}{3.987300in}}{\pgfqpoint{2.438696in}{3.982971in}}{\pgfqpoint{2.449592in}{3.982971in}}%
\pgfpathlineto{\pgfqpoint{2.449592in}{3.982971in}}%
\pgfpathclose%
\pgfusepath{stroke}%
\end{pgfscope}%
\begin{pgfscope}%
\pgfpathrectangle{\pgfqpoint{0.688192in}{0.670138in}}{\pgfqpoint{7.111808in}{5.061530in}}%
\pgfusepath{clip}%
\pgfsetbuttcap%
\pgfsetroundjoin%
\pgfsetlinewidth{1.003750pt}%
\definecolor{currentstroke}{rgb}{0.000000,0.000000,0.000000}%
\pgfsetstrokecolor{currentstroke}%
\pgfsetdash{}{0pt}%
\pgfpathmoveto{\pgfqpoint{5.505780in}{1.507933in}}%
\pgfpathcurveto{\pgfqpoint{5.516676in}{1.507933in}}{\pgfqpoint{5.527127in}{1.512262in}}{\pgfqpoint{5.534831in}{1.519966in}}%
\pgfpathcurveto{\pgfqpoint{5.542535in}{1.527671in}}{\pgfqpoint{5.546864in}{1.538122in}}{\pgfqpoint{5.546864in}{1.549017in}}%
\pgfpathcurveto{\pgfqpoint{5.546864in}{1.559913in}}{\pgfqpoint{5.542535in}{1.570364in}}{\pgfqpoint{5.534831in}{1.578068in}}%
\pgfpathcurveto{\pgfqpoint{5.527127in}{1.585772in}}{\pgfqpoint{5.516676in}{1.590101in}}{\pgfqpoint{5.505780in}{1.590101in}}%
\pgfpathcurveto{\pgfqpoint{5.494885in}{1.590101in}}{\pgfqpoint{5.484434in}{1.585772in}}{\pgfqpoint{5.476730in}{1.578068in}}%
\pgfpathcurveto{\pgfqpoint{5.469025in}{1.570364in}}{\pgfqpoint{5.464697in}{1.559913in}}{\pgfqpoint{5.464697in}{1.549017in}}%
\pgfpathcurveto{\pgfqpoint{5.464697in}{1.538122in}}{\pgfqpoint{5.469025in}{1.527671in}}{\pgfqpoint{5.476730in}{1.519966in}}%
\pgfpathcurveto{\pgfqpoint{5.484434in}{1.512262in}}{\pgfqpoint{5.494885in}{1.507933in}}{\pgfqpoint{5.505780in}{1.507933in}}%
\pgfpathlineto{\pgfqpoint{5.505780in}{1.507933in}}%
\pgfpathclose%
\pgfusepath{stroke}%
\end{pgfscope}%
\begin{pgfscope}%
\pgfpathrectangle{\pgfqpoint{0.688192in}{0.670138in}}{\pgfqpoint{7.111808in}{5.061530in}}%
\pgfusepath{clip}%
\pgfsetbuttcap%
\pgfsetroundjoin%
\pgfsetlinewidth{1.003750pt}%
\definecolor{currentstroke}{rgb}{0.000000,0.000000,0.000000}%
\pgfsetstrokecolor{currentstroke}%
\pgfsetdash{}{0pt}%
\pgfpathmoveto{\pgfqpoint{3.560021in}{0.632771in}}%
\pgfpathcurveto{\pgfqpoint{3.570916in}{0.632771in}}{\pgfqpoint{3.581367in}{0.637099in}}{\pgfqpoint{3.589071in}{0.644804in}}%
\pgfpathcurveto{\pgfqpoint{3.596776in}{0.652508in}}{\pgfqpoint{3.601105in}{0.662959in}}{\pgfqpoint{3.601105in}{0.673854in}}%
\pgfpathcurveto{\pgfqpoint{3.601105in}{0.684750in}}{\pgfqpoint{3.596776in}{0.695201in}}{\pgfqpoint{3.589071in}{0.702905in}}%
\pgfpathcurveto{\pgfqpoint{3.581367in}{0.710609in}}{\pgfqpoint{3.570916in}{0.714938in}}{\pgfqpoint{3.560021in}{0.714938in}}%
\pgfpathcurveto{\pgfqpoint{3.549125in}{0.714938in}}{\pgfqpoint{3.538674in}{0.710609in}}{\pgfqpoint{3.530970in}{0.702905in}}%
\pgfpathcurveto{\pgfqpoint{3.523266in}{0.695201in}}{\pgfqpoint{3.518937in}{0.684750in}}{\pgfqpoint{3.518937in}{0.673854in}}%
\pgfpathcurveto{\pgfqpoint{3.518937in}{0.662959in}}{\pgfqpoint{3.523266in}{0.652508in}}{\pgfqpoint{3.530970in}{0.644804in}}%
\pgfpathcurveto{\pgfqpoint{3.538674in}{0.637099in}}{\pgfqpoint{3.549125in}{0.632771in}}{\pgfqpoint{3.560021in}{0.632771in}}%
\pgfusepath{stroke}%
\end{pgfscope}%
\begin{pgfscope}%
\pgfpathrectangle{\pgfqpoint{0.688192in}{0.670138in}}{\pgfqpoint{7.111808in}{5.061530in}}%
\pgfusepath{clip}%
\pgfsetbuttcap%
\pgfsetroundjoin%
\pgfsetlinewidth{1.003750pt}%
\definecolor{currentstroke}{rgb}{0.000000,0.000000,0.000000}%
\pgfsetstrokecolor{currentstroke}%
\pgfsetdash{}{0pt}%
\pgfpathmoveto{\pgfqpoint{1.903233in}{0.640199in}}%
\pgfpathcurveto{\pgfqpoint{1.914129in}{0.640199in}}{\pgfqpoint{1.924579in}{0.644527in}}{\pgfqpoint{1.932284in}{0.652232in}}%
\pgfpathcurveto{\pgfqpoint{1.939988in}{0.659936in}}{\pgfqpoint{1.944317in}{0.670387in}}{\pgfqpoint{1.944317in}{0.681282in}}%
\pgfpathcurveto{\pgfqpoint{1.944317in}{0.692178in}}{\pgfqpoint{1.939988in}{0.702629in}}{\pgfqpoint{1.932284in}{0.710333in}}%
\pgfpathcurveto{\pgfqpoint{1.924579in}{0.718038in}}{\pgfqpoint{1.914129in}{0.722366in}}{\pgfqpoint{1.903233in}{0.722366in}}%
\pgfpathcurveto{\pgfqpoint{1.892337in}{0.722366in}}{\pgfqpoint{1.881887in}{0.718038in}}{\pgfqpoint{1.874182in}{0.710333in}}%
\pgfpathcurveto{\pgfqpoint{1.866478in}{0.702629in}}{\pgfqpoint{1.862149in}{0.692178in}}{\pgfqpoint{1.862149in}{0.681282in}}%
\pgfpathcurveto{\pgfqpoint{1.862149in}{0.670387in}}{\pgfqpoint{1.866478in}{0.659936in}}{\pgfqpoint{1.874182in}{0.652232in}}%
\pgfpathcurveto{\pgfqpoint{1.881887in}{0.644527in}}{\pgfqpoint{1.892337in}{0.640199in}}{\pgfqpoint{1.903233in}{0.640199in}}%
\pgfusepath{stroke}%
\end{pgfscope}%
\begin{pgfscope}%
\pgfpathrectangle{\pgfqpoint{0.688192in}{0.670138in}}{\pgfqpoint{7.111808in}{5.061530in}}%
\pgfusepath{clip}%
\pgfsetbuttcap%
\pgfsetroundjoin%
\pgfsetlinewidth{1.003750pt}%
\definecolor{currentstroke}{rgb}{0.000000,0.000000,0.000000}%
\pgfsetstrokecolor{currentstroke}%
\pgfsetdash{}{0pt}%
\pgfpathmoveto{\pgfqpoint{4.576929in}{0.631060in}}%
\pgfpathcurveto{\pgfqpoint{4.587825in}{0.631060in}}{\pgfqpoint{4.598276in}{0.635389in}}{\pgfqpoint{4.605980in}{0.643094in}}%
\pgfpathcurveto{\pgfqpoint{4.613684in}{0.650798in}}{\pgfqpoint{4.618013in}{0.661249in}}{\pgfqpoint{4.618013in}{0.672144in}}%
\pgfpathcurveto{\pgfqpoint{4.618013in}{0.683040in}}{\pgfqpoint{4.613684in}{0.693491in}}{\pgfqpoint{4.605980in}{0.701195in}}%
\pgfpathcurveto{\pgfqpoint{4.598276in}{0.708899in}}{\pgfqpoint{4.587825in}{0.713228in}}{\pgfqpoint{4.576929in}{0.713228in}}%
\pgfpathcurveto{\pgfqpoint{4.566034in}{0.713228in}}{\pgfqpoint{4.555583in}{0.708899in}}{\pgfqpoint{4.547878in}{0.701195in}}%
\pgfpathcurveto{\pgfqpoint{4.540174in}{0.693491in}}{\pgfqpoint{4.535845in}{0.683040in}}{\pgfqpoint{4.535845in}{0.672144in}}%
\pgfpathcurveto{\pgfqpoint{4.535845in}{0.661249in}}{\pgfqpoint{4.540174in}{0.650798in}}{\pgfqpoint{4.547878in}{0.643094in}}%
\pgfpathcurveto{\pgfqpoint{4.555583in}{0.635389in}}{\pgfqpoint{4.566034in}{0.631060in}}{\pgfqpoint{4.576929in}{0.631060in}}%
\pgfusepath{stroke}%
\end{pgfscope}%
\begin{pgfscope}%
\pgfpathrectangle{\pgfqpoint{0.688192in}{0.670138in}}{\pgfqpoint{7.111808in}{5.061530in}}%
\pgfusepath{clip}%
\pgfsetbuttcap%
\pgfsetroundjoin%
\pgfsetlinewidth{1.003750pt}%
\definecolor{currentstroke}{rgb}{0.000000,0.000000,0.000000}%
\pgfsetstrokecolor{currentstroke}%
\pgfsetdash{}{0pt}%
\pgfpathmoveto{\pgfqpoint{2.294956in}{0.723409in}}%
\pgfpathcurveto{\pgfqpoint{2.305852in}{0.723409in}}{\pgfqpoint{2.316303in}{0.727738in}}{\pgfqpoint{2.324007in}{0.735442in}}%
\pgfpathcurveto{\pgfqpoint{2.331711in}{0.743147in}}{\pgfqpoint{2.336040in}{0.753598in}}{\pgfqpoint{2.336040in}{0.764493in}}%
\pgfpathcurveto{\pgfqpoint{2.336040in}{0.775389in}}{\pgfqpoint{2.331711in}{0.785839in}}{\pgfqpoint{2.324007in}{0.793544in}}%
\pgfpathcurveto{\pgfqpoint{2.316303in}{0.801248in}}{\pgfqpoint{2.305852in}{0.805577in}}{\pgfqpoint{2.294956in}{0.805577in}}%
\pgfpathcurveto{\pgfqpoint{2.284061in}{0.805577in}}{\pgfqpoint{2.273610in}{0.801248in}}{\pgfqpoint{2.265906in}{0.793544in}}%
\pgfpathcurveto{\pgfqpoint{2.258201in}{0.785839in}}{\pgfqpoint{2.253872in}{0.775389in}}{\pgfqpoint{2.253872in}{0.764493in}}%
\pgfpathcurveto{\pgfqpoint{2.253872in}{0.753598in}}{\pgfqpoint{2.258201in}{0.743147in}}{\pgfqpoint{2.265906in}{0.735442in}}%
\pgfpathcurveto{\pgfqpoint{2.273610in}{0.727738in}}{\pgfqpoint{2.284061in}{0.723409in}}{\pgfqpoint{2.294956in}{0.723409in}}%
\pgfpathlineto{\pgfqpoint{2.294956in}{0.723409in}}%
\pgfpathclose%
\pgfusepath{stroke}%
\end{pgfscope}%
\begin{pgfscope}%
\pgfpathrectangle{\pgfqpoint{0.688192in}{0.670138in}}{\pgfqpoint{7.111808in}{5.061530in}}%
\pgfusepath{clip}%
\pgfsetbuttcap%
\pgfsetroundjoin%
\pgfsetlinewidth{1.003750pt}%
\definecolor{currentstroke}{rgb}{0.000000,0.000000,0.000000}%
\pgfsetstrokecolor{currentstroke}%
\pgfsetdash{}{0pt}%
\pgfpathmoveto{\pgfqpoint{3.563680in}{0.682356in}}%
\pgfpathcurveto{\pgfqpoint{3.574576in}{0.682356in}}{\pgfqpoint{3.585026in}{0.686685in}}{\pgfqpoint{3.592731in}{0.694389in}}%
\pgfpathcurveto{\pgfqpoint{3.600435in}{0.702094in}}{\pgfqpoint{3.604764in}{0.712545in}}{\pgfqpoint{3.604764in}{0.723440in}}%
\pgfpathcurveto{\pgfqpoint{3.604764in}{0.734336in}}{\pgfqpoint{3.600435in}{0.744786in}}{\pgfqpoint{3.592731in}{0.752491in}}%
\pgfpathcurveto{\pgfqpoint{3.585026in}{0.760195in}}{\pgfqpoint{3.574576in}{0.764524in}}{\pgfqpoint{3.563680in}{0.764524in}}%
\pgfpathcurveto{\pgfqpoint{3.552784in}{0.764524in}}{\pgfqpoint{3.542334in}{0.760195in}}{\pgfqpoint{3.534629in}{0.752491in}}%
\pgfpathcurveto{\pgfqpoint{3.526925in}{0.744786in}}{\pgfqpoint{3.522596in}{0.734336in}}{\pgfqpoint{3.522596in}{0.723440in}}%
\pgfpathcurveto{\pgfqpoint{3.522596in}{0.712545in}}{\pgfqpoint{3.526925in}{0.702094in}}{\pgfqpoint{3.534629in}{0.694389in}}%
\pgfpathcurveto{\pgfqpoint{3.542334in}{0.686685in}}{\pgfqpoint{3.552784in}{0.682356in}}{\pgfqpoint{3.563680in}{0.682356in}}%
\pgfpathlineto{\pgfqpoint{3.563680in}{0.682356in}}%
\pgfpathclose%
\pgfusepath{stroke}%
\end{pgfscope}%
\begin{pgfscope}%
\pgfpathrectangle{\pgfqpoint{0.688192in}{0.670138in}}{\pgfqpoint{7.111808in}{5.061530in}}%
\pgfusepath{clip}%
\pgfsetbuttcap%
\pgfsetroundjoin%
\pgfsetlinewidth{1.003750pt}%
\definecolor{currentstroke}{rgb}{0.000000,0.000000,0.000000}%
\pgfsetstrokecolor{currentstroke}%
\pgfsetdash{}{0pt}%
\pgfpathmoveto{\pgfqpoint{7.628730in}{2.472549in}}%
\pgfpathcurveto{\pgfqpoint{7.639626in}{2.472549in}}{\pgfqpoint{7.650077in}{2.476878in}}{\pgfqpoint{7.657781in}{2.484582in}}%
\pgfpathcurveto{\pgfqpoint{7.665485in}{2.492286in}}{\pgfqpoint{7.669814in}{2.502737in}}{\pgfqpoint{7.669814in}{2.513633in}}%
\pgfpathcurveto{\pgfqpoint{7.669814in}{2.524528in}}{\pgfqpoint{7.665485in}{2.534979in}}{\pgfqpoint{7.657781in}{2.542683in}}%
\pgfpathcurveto{\pgfqpoint{7.650077in}{2.550388in}}{\pgfqpoint{7.639626in}{2.554716in}}{\pgfqpoint{7.628730in}{2.554716in}}%
\pgfpathcurveto{\pgfqpoint{7.617835in}{2.554716in}}{\pgfqpoint{7.607384in}{2.550388in}}{\pgfqpoint{7.599680in}{2.542683in}}%
\pgfpathcurveto{\pgfqpoint{7.591975in}{2.534979in}}{\pgfqpoint{7.587647in}{2.524528in}}{\pgfqpoint{7.587647in}{2.513633in}}%
\pgfpathcurveto{\pgfqpoint{7.587647in}{2.502737in}}{\pgfqpoint{7.591975in}{2.492286in}}{\pgfqpoint{7.599680in}{2.484582in}}%
\pgfpathcurveto{\pgfqpoint{7.607384in}{2.476878in}}{\pgfqpoint{7.617835in}{2.472549in}}{\pgfqpoint{7.628730in}{2.472549in}}%
\pgfpathlineto{\pgfqpoint{7.628730in}{2.472549in}}%
\pgfpathclose%
\pgfusepath{stroke}%
\end{pgfscope}%
\begin{pgfscope}%
\pgfpathrectangle{\pgfqpoint{0.688192in}{0.670138in}}{\pgfqpoint{7.111808in}{5.061530in}}%
\pgfusepath{clip}%
\pgfsetbuttcap%
\pgfsetroundjoin%
\pgfsetlinewidth{1.003750pt}%
\definecolor{currentstroke}{rgb}{0.000000,0.000000,0.000000}%
\pgfsetstrokecolor{currentstroke}%
\pgfsetdash{}{0pt}%
\pgfpathmoveto{\pgfqpoint{6.249740in}{0.748741in}}%
\pgfpathcurveto{\pgfqpoint{6.260636in}{0.748741in}}{\pgfqpoint{6.271087in}{0.753070in}}{\pgfqpoint{6.278791in}{0.760774in}}%
\pgfpathcurveto{\pgfqpoint{6.286495in}{0.768479in}}{\pgfqpoint{6.290824in}{0.778930in}}{\pgfqpoint{6.290824in}{0.789825in}}%
\pgfpathcurveto{\pgfqpoint{6.290824in}{0.800721in}}{\pgfqpoint{6.286495in}{0.811171in}}{\pgfqpoint{6.278791in}{0.818876in}}%
\pgfpathcurveto{\pgfqpoint{6.271087in}{0.826580in}}{\pgfqpoint{6.260636in}{0.830909in}}{\pgfqpoint{6.249740in}{0.830909in}}%
\pgfpathcurveto{\pgfqpoint{6.238845in}{0.830909in}}{\pgfqpoint{6.228394in}{0.826580in}}{\pgfqpoint{6.220690in}{0.818876in}}%
\pgfpathcurveto{\pgfqpoint{6.212985in}{0.811171in}}{\pgfqpoint{6.208656in}{0.800721in}}{\pgfqpoint{6.208656in}{0.789825in}}%
\pgfpathcurveto{\pgfqpoint{6.208656in}{0.778930in}}{\pgfqpoint{6.212985in}{0.768479in}}{\pgfqpoint{6.220690in}{0.760774in}}%
\pgfpathcurveto{\pgfqpoint{6.228394in}{0.753070in}}{\pgfqpoint{6.238845in}{0.748741in}}{\pgfqpoint{6.249740in}{0.748741in}}%
\pgfpathlineto{\pgfqpoint{6.249740in}{0.748741in}}%
\pgfpathclose%
\pgfusepath{stroke}%
\end{pgfscope}%
\begin{pgfscope}%
\pgfpathrectangle{\pgfqpoint{0.688192in}{0.670138in}}{\pgfqpoint{7.111808in}{5.061530in}}%
\pgfusepath{clip}%
\pgfsetbuttcap%
\pgfsetroundjoin%
\pgfsetlinewidth{1.003750pt}%
\definecolor{currentstroke}{rgb}{0.000000,0.000000,0.000000}%
\pgfsetstrokecolor{currentstroke}%
\pgfsetdash{}{0pt}%
\pgfpathmoveto{\pgfqpoint{5.946216in}{2.685863in}}%
\pgfpathcurveto{\pgfqpoint{5.957111in}{2.685863in}}{\pgfqpoint{5.967562in}{2.690192in}}{\pgfqpoint{5.975266in}{2.697896in}}%
\pgfpathcurveto{\pgfqpoint{5.982971in}{2.705601in}}{\pgfqpoint{5.987299in}{2.716051in}}{\pgfqpoint{5.987299in}{2.726947in}}%
\pgfpathcurveto{\pgfqpoint{5.987299in}{2.737843in}}{\pgfqpoint{5.982971in}{2.748293in}}{\pgfqpoint{5.975266in}{2.755998in}}%
\pgfpathcurveto{\pgfqpoint{5.967562in}{2.763702in}}{\pgfqpoint{5.957111in}{2.768031in}}{\pgfqpoint{5.946216in}{2.768031in}}%
\pgfpathcurveto{\pgfqpoint{5.935320in}{2.768031in}}{\pgfqpoint{5.924869in}{2.763702in}}{\pgfqpoint{5.917165in}{2.755998in}}%
\pgfpathcurveto{\pgfqpoint{5.909461in}{2.748293in}}{\pgfqpoint{5.905132in}{2.737843in}}{\pgfqpoint{5.905132in}{2.726947in}}%
\pgfpathcurveto{\pgfqpoint{5.905132in}{2.716051in}}{\pgfqpoint{5.909461in}{2.705601in}}{\pgfqpoint{5.917165in}{2.697896in}}%
\pgfpathcurveto{\pgfqpoint{5.924869in}{2.690192in}}{\pgfqpoint{5.935320in}{2.685863in}}{\pgfqpoint{5.946216in}{2.685863in}}%
\pgfpathlineto{\pgfqpoint{5.946216in}{2.685863in}}%
\pgfpathclose%
\pgfusepath{stroke}%
\end{pgfscope}%
\begin{pgfscope}%
\pgfpathrectangle{\pgfqpoint{0.688192in}{0.670138in}}{\pgfqpoint{7.111808in}{5.061530in}}%
\pgfusepath{clip}%
\pgfsetbuttcap%
\pgfsetroundjoin%
\pgfsetlinewidth{1.003750pt}%
\definecolor{currentstroke}{rgb}{0.000000,0.000000,0.000000}%
\pgfsetstrokecolor{currentstroke}%
\pgfsetdash{}{0pt}%
\pgfpathmoveto{\pgfqpoint{5.142949in}{0.659582in}}%
\pgfpathcurveto{\pgfqpoint{5.153844in}{0.659582in}}{\pgfqpoint{5.164295in}{0.663911in}}{\pgfqpoint{5.172000in}{0.671616in}}%
\pgfpathcurveto{\pgfqpoint{5.179704in}{0.679320in}}{\pgfqpoint{5.184033in}{0.689771in}}{\pgfqpoint{5.184033in}{0.700666in}}%
\pgfpathcurveto{\pgfqpoint{5.184033in}{0.711562in}}{\pgfqpoint{5.179704in}{0.722013in}}{\pgfqpoint{5.172000in}{0.729717in}}%
\pgfpathcurveto{\pgfqpoint{5.164295in}{0.737421in}}{\pgfqpoint{5.153844in}{0.741750in}}{\pgfqpoint{5.142949in}{0.741750in}}%
\pgfpathcurveto{\pgfqpoint{5.132053in}{0.741750in}}{\pgfqpoint{5.121603in}{0.737421in}}{\pgfqpoint{5.113898in}{0.729717in}}%
\pgfpathcurveto{\pgfqpoint{5.106194in}{0.722013in}}{\pgfqpoint{5.101865in}{0.711562in}}{\pgfqpoint{5.101865in}{0.700666in}}%
\pgfpathcurveto{\pgfqpoint{5.101865in}{0.689771in}}{\pgfqpoint{5.106194in}{0.679320in}}{\pgfqpoint{5.113898in}{0.671616in}}%
\pgfpathcurveto{\pgfqpoint{5.121603in}{0.663911in}}{\pgfqpoint{5.132053in}{0.659582in}}{\pgfqpoint{5.142949in}{0.659582in}}%
\pgfusepath{stroke}%
\end{pgfscope}%
\begin{pgfscope}%
\pgfpathrectangle{\pgfqpoint{0.688192in}{0.670138in}}{\pgfqpoint{7.111808in}{5.061530in}}%
\pgfusepath{clip}%
\pgfsetbuttcap%
\pgfsetroundjoin%
\pgfsetlinewidth{1.003750pt}%
\definecolor{currentstroke}{rgb}{0.000000,0.000000,0.000000}%
\pgfsetstrokecolor{currentstroke}%
\pgfsetdash{}{0pt}%
\pgfpathmoveto{\pgfqpoint{0.816256in}{0.719759in}}%
\pgfpathcurveto{\pgfqpoint{0.827152in}{0.719759in}}{\pgfqpoint{0.837603in}{0.724088in}}{\pgfqpoint{0.845307in}{0.731792in}}%
\pgfpathcurveto{\pgfqpoint{0.853011in}{0.739496in}}{\pgfqpoint{0.857340in}{0.749947in}}{\pgfqpoint{0.857340in}{0.760843in}}%
\pgfpathcurveto{\pgfqpoint{0.857340in}{0.771738in}}{\pgfqpoint{0.853011in}{0.782189in}}{\pgfqpoint{0.845307in}{0.789893in}}%
\pgfpathcurveto{\pgfqpoint{0.837603in}{0.797598in}}{\pgfqpoint{0.827152in}{0.801926in}}{\pgfqpoint{0.816256in}{0.801926in}}%
\pgfpathcurveto{\pgfqpoint{0.805361in}{0.801926in}}{\pgfqpoint{0.794910in}{0.797598in}}{\pgfqpoint{0.787205in}{0.789893in}}%
\pgfpathcurveto{\pgfqpoint{0.779501in}{0.782189in}}{\pgfqpoint{0.775172in}{0.771738in}}{\pgfqpoint{0.775172in}{0.760843in}}%
\pgfpathcurveto{\pgfqpoint{0.775172in}{0.749947in}}{\pgfqpoint{0.779501in}{0.739496in}}{\pgfqpoint{0.787205in}{0.731792in}}%
\pgfpathcurveto{\pgfqpoint{0.794910in}{0.724088in}}{\pgfqpoint{0.805361in}{0.719759in}}{\pgfqpoint{0.816256in}{0.719759in}}%
\pgfpathlineto{\pgfqpoint{0.816256in}{0.719759in}}%
\pgfpathclose%
\pgfusepath{stroke}%
\end{pgfscope}%
\begin{pgfscope}%
\pgfpathrectangle{\pgfqpoint{0.688192in}{0.670138in}}{\pgfqpoint{7.111808in}{5.061530in}}%
\pgfusepath{clip}%
\pgfsetbuttcap%
\pgfsetroundjoin%
\pgfsetlinewidth{1.003750pt}%
\definecolor{currentstroke}{rgb}{0.000000,0.000000,0.000000}%
\pgfsetstrokecolor{currentstroke}%
\pgfsetdash{}{0pt}%
\pgfpathmoveto{\pgfqpoint{1.376266in}{0.643207in}}%
\pgfpathcurveto{\pgfqpoint{1.387161in}{0.643207in}}{\pgfqpoint{1.397612in}{0.647536in}}{\pgfqpoint{1.405316in}{0.655240in}}%
\pgfpathcurveto{\pgfqpoint{1.413021in}{0.662944in}}{\pgfqpoint{1.417349in}{0.673395in}}{\pgfqpoint{1.417349in}{0.684291in}}%
\pgfpathcurveto{\pgfqpoint{1.417349in}{0.695186in}}{\pgfqpoint{1.413021in}{0.705637in}}{\pgfqpoint{1.405316in}{0.713341in}}%
\pgfpathcurveto{\pgfqpoint{1.397612in}{0.721046in}}{\pgfqpoint{1.387161in}{0.725375in}}{\pgfqpoint{1.376266in}{0.725375in}}%
\pgfpathcurveto{\pgfqpoint{1.365370in}{0.725375in}}{\pgfqpoint{1.354919in}{0.721046in}}{\pgfqpoint{1.347215in}{0.713341in}}%
\pgfpathcurveto{\pgfqpoint{1.339511in}{0.705637in}}{\pgfqpoint{1.335182in}{0.695186in}}{\pgfqpoint{1.335182in}{0.684291in}}%
\pgfpathcurveto{\pgfqpoint{1.335182in}{0.673395in}}{\pgfqpoint{1.339511in}{0.662944in}}{\pgfqpoint{1.347215in}{0.655240in}}%
\pgfpathcurveto{\pgfqpoint{1.354919in}{0.647536in}}{\pgfqpoint{1.365370in}{0.643207in}}{\pgfqpoint{1.376266in}{0.643207in}}%
\pgfusepath{stroke}%
\end{pgfscope}%
\begin{pgfscope}%
\pgfpathrectangle{\pgfqpoint{0.688192in}{0.670138in}}{\pgfqpoint{7.111808in}{5.061530in}}%
\pgfusepath{clip}%
\pgfsetbuttcap%
\pgfsetroundjoin%
\pgfsetlinewidth{1.003750pt}%
\definecolor{currentstroke}{rgb}{0.000000,0.000000,0.000000}%
\pgfsetstrokecolor{currentstroke}%
\pgfsetdash{}{0pt}%
\pgfpathmoveto{\pgfqpoint{0.940682in}{0.674925in}}%
\pgfpathcurveto{\pgfqpoint{0.951577in}{0.674925in}}{\pgfqpoint{0.962028in}{0.679254in}}{\pgfqpoint{0.969732in}{0.686959in}}%
\pgfpathcurveto{\pgfqpoint{0.977437in}{0.694663in}}{\pgfqpoint{0.981766in}{0.705114in}}{\pgfqpoint{0.981766in}{0.716009in}}%
\pgfpathcurveto{\pgfqpoint{0.981766in}{0.726905in}}{\pgfqpoint{0.977437in}{0.737356in}}{\pgfqpoint{0.969732in}{0.745060in}}%
\pgfpathcurveto{\pgfqpoint{0.962028in}{0.752764in}}{\pgfqpoint{0.951577in}{0.757093in}}{\pgfqpoint{0.940682in}{0.757093in}}%
\pgfpathcurveto{\pgfqpoint{0.929786in}{0.757093in}}{\pgfqpoint{0.919335in}{0.752764in}}{\pgfqpoint{0.911631in}{0.745060in}}%
\pgfpathcurveto{\pgfqpoint{0.903927in}{0.737356in}}{\pgfqpoint{0.899598in}{0.726905in}}{\pgfqpoint{0.899598in}{0.716009in}}%
\pgfpathcurveto{\pgfqpoint{0.899598in}{0.705114in}}{\pgfqpoint{0.903927in}{0.694663in}}{\pgfqpoint{0.911631in}{0.686959in}}%
\pgfpathcurveto{\pgfqpoint{0.919335in}{0.679254in}}{\pgfqpoint{0.929786in}{0.674925in}}{\pgfqpoint{0.940682in}{0.674925in}}%
\pgfpathlineto{\pgfqpoint{0.940682in}{0.674925in}}%
\pgfpathclose%
\pgfusepath{stroke}%
\end{pgfscope}%
\begin{pgfscope}%
\pgfpathrectangle{\pgfqpoint{0.688192in}{0.670138in}}{\pgfqpoint{7.111808in}{5.061530in}}%
\pgfusepath{clip}%
\pgfsetbuttcap%
\pgfsetroundjoin%
\pgfsetlinewidth{1.003750pt}%
\definecolor{currentstroke}{rgb}{0.000000,0.000000,0.000000}%
\pgfsetstrokecolor{currentstroke}%
\pgfsetdash{}{0pt}%
\pgfpathmoveto{\pgfqpoint{1.103303in}{0.648067in}}%
\pgfpathcurveto{\pgfqpoint{1.114198in}{0.648067in}}{\pgfqpoint{1.124649in}{0.652396in}}{\pgfqpoint{1.132353in}{0.660101in}}%
\pgfpathcurveto{\pgfqpoint{1.140058in}{0.667805in}}{\pgfqpoint{1.144386in}{0.678256in}}{\pgfqpoint{1.144386in}{0.689151in}}%
\pgfpathcurveto{\pgfqpoint{1.144386in}{0.700047in}}{\pgfqpoint{1.140058in}{0.710498in}}{\pgfqpoint{1.132353in}{0.718202in}}%
\pgfpathcurveto{\pgfqpoint{1.124649in}{0.725906in}}{\pgfqpoint{1.114198in}{0.730235in}}{\pgfqpoint{1.103303in}{0.730235in}}%
\pgfpathcurveto{\pgfqpoint{1.092407in}{0.730235in}}{\pgfqpoint{1.081956in}{0.725906in}}{\pgfqpoint{1.074252in}{0.718202in}}%
\pgfpathcurveto{\pgfqpoint{1.066548in}{0.710498in}}{\pgfqpoint{1.062219in}{0.700047in}}{\pgfqpoint{1.062219in}{0.689151in}}%
\pgfpathcurveto{\pgfqpoint{1.062219in}{0.678256in}}{\pgfqpoint{1.066548in}{0.667805in}}{\pgfqpoint{1.074252in}{0.660101in}}%
\pgfpathcurveto{\pgfqpoint{1.081956in}{0.652396in}}{\pgfqpoint{1.092407in}{0.648067in}}{\pgfqpoint{1.103303in}{0.648067in}}%
\pgfusepath{stroke}%
\end{pgfscope}%
\begin{pgfscope}%
\pgfpathrectangle{\pgfqpoint{0.688192in}{0.670138in}}{\pgfqpoint{7.111808in}{5.061530in}}%
\pgfusepath{clip}%
\pgfsetbuttcap%
\pgfsetroundjoin%
\pgfsetlinewidth{1.003750pt}%
\definecolor{currentstroke}{rgb}{0.000000,0.000000,0.000000}%
\pgfsetstrokecolor{currentstroke}%
\pgfsetdash{}{0pt}%
\pgfpathmoveto{\pgfqpoint{2.603035in}{0.731524in}}%
\pgfpathcurveto{\pgfqpoint{2.613931in}{0.731524in}}{\pgfqpoint{2.624382in}{0.735853in}}{\pgfqpoint{2.632086in}{0.743557in}}%
\pgfpathcurveto{\pgfqpoint{2.639790in}{0.751261in}}{\pgfqpoint{2.644119in}{0.761712in}}{\pgfqpoint{2.644119in}{0.772608in}}%
\pgfpathcurveto{\pgfqpoint{2.644119in}{0.783503in}}{\pgfqpoint{2.639790in}{0.793954in}}{\pgfqpoint{2.632086in}{0.801659in}}%
\pgfpathcurveto{\pgfqpoint{2.624382in}{0.809363in}}{\pgfqpoint{2.613931in}{0.813692in}}{\pgfqpoint{2.603035in}{0.813692in}}%
\pgfpathcurveto{\pgfqpoint{2.592140in}{0.813692in}}{\pgfqpoint{2.581689in}{0.809363in}}{\pgfqpoint{2.573985in}{0.801659in}}%
\pgfpathcurveto{\pgfqpoint{2.566280in}{0.793954in}}{\pgfqpoint{2.561951in}{0.783503in}}{\pgfqpoint{2.561951in}{0.772608in}}%
\pgfpathcurveto{\pgfqpoint{2.561951in}{0.761712in}}{\pgfqpoint{2.566280in}{0.751261in}}{\pgfqpoint{2.573985in}{0.743557in}}%
\pgfpathcurveto{\pgfqpoint{2.581689in}{0.735853in}}{\pgfqpoint{2.592140in}{0.731524in}}{\pgfqpoint{2.603035in}{0.731524in}}%
\pgfpathlineto{\pgfqpoint{2.603035in}{0.731524in}}%
\pgfpathclose%
\pgfusepath{stroke}%
\end{pgfscope}%
\begin{pgfscope}%
\pgfpathrectangle{\pgfqpoint{0.688192in}{0.670138in}}{\pgfqpoint{7.111808in}{5.061530in}}%
\pgfusepath{clip}%
\pgfsetbuttcap%
\pgfsetroundjoin%
\pgfsetlinewidth{1.003750pt}%
\definecolor{currentstroke}{rgb}{0.000000,0.000000,0.000000}%
\pgfsetstrokecolor{currentstroke}%
\pgfsetdash{}{0pt}%
\pgfpathmoveto{\pgfqpoint{5.283066in}{0.629702in}}%
\pgfpathcurveto{\pgfqpoint{5.293961in}{0.629702in}}{\pgfqpoint{5.304412in}{0.634031in}}{\pgfqpoint{5.312116in}{0.641735in}}%
\pgfpathcurveto{\pgfqpoint{5.319821in}{0.649439in}}{\pgfqpoint{5.324149in}{0.659890in}}{\pgfqpoint{5.324149in}{0.670786in}}%
\pgfpathcurveto{\pgfqpoint{5.324149in}{0.681681in}}{\pgfqpoint{5.319821in}{0.692132in}}{\pgfqpoint{5.312116in}{0.699837in}}%
\pgfpathcurveto{\pgfqpoint{5.304412in}{0.707541in}}{\pgfqpoint{5.293961in}{0.711870in}}{\pgfqpoint{5.283066in}{0.711870in}}%
\pgfpathcurveto{\pgfqpoint{5.272170in}{0.711870in}}{\pgfqpoint{5.261719in}{0.707541in}}{\pgfqpoint{5.254015in}{0.699837in}}%
\pgfpathcurveto{\pgfqpoint{5.246311in}{0.692132in}}{\pgfqpoint{5.241982in}{0.681681in}}{\pgfqpoint{5.241982in}{0.670786in}}%
\pgfpathcurveto{\pgfqpoint{5.241982in}{0.659890in}}{\pgfqpoint{5.246311in}{0.649439in}}{\pgfqpoint{5.254015in}{0.641735in}}%
\pgfpathcurveto{\pgfqpoint{5.261719in}{0.634031in}}{\pgfqpoint{5.272170in}{0.629702in}}{\pgfqpoint{5.283066in}{0.629702in}}%
\pgfusepath{stroke}%
\end{pgfscope}%
\begin{pgfscope}%
\pgfpathrectangle{\pgfqpoint{0.688192in}{0.670138in}}{\pgfqpoint{7.111808in}{5.061530in}}%
\pgfusepath{clip}%
\pgfsetbuttcap%
\pgfsetroundjoin%
\pgfsetlinewidth{1.003750pt}%
\definecolor{currentstroke}{rgb}{0.000000,0.000000,0.000000}%
\pgfsetstrokecolor{currentstroke}%
\pgfsetdash{}{0pt}%
\pgfpathmoveto{\pgfqpoint{4.495422in}{4.650068in}}%
\pgfpathcurveto{\pgfqpoint{4.506318in}{4.650068in}}{\pgfqpoint{4.516769in}{4.654397in}}{\pgfqpoint{4.524473in}{4.662101in}}%
\pgfpathcurveto{\pgfqpoint{4.532177in}{4.669805in}}{\pgfqpoint{4.536506in}{4.680256in}}{\pgfqpoint{4.536506in}{4.691152in}}%
\pgfpathcurveto{\pgfqpoint{4.536506in}{4.702047in}}{\pgfqpoint{4.532177in}{4.712498in}}{\pgfqpoint{4.524473in}{4.720203in}}%
\pgfpathcurveto{\pgfqpoint{4.516769in}{4.727907in}}{\pgfqpoint{4.506318in}{4.732236in}}{\pgfqpoint{4.495422in}{4.732236in}}%
\pgfpathcurveto{\pgfqpoint{4.484527in}{4.732236in}}{\pgfqpoint{4.474076in}{4.727907in}}{\pgfqpoint{4.466371in}{4.720203in}}%
\pgfpathcurveto{\pgfqpoint{4.458667in}{4.712498in}}{\pgfqpoint{4.454338in}{4.702047in}}{\pgfqpoint{4.454338in}{4.691152in}}%
\pgfpathcurveto{\pgfqpoint{4.454338in}{4.680256in}}{\pgfqpoint{4.458667in}{4.669805in}}{\pgfqpoint{4.466371in}{4.662101in}}%
\pgfpathcurveto{\pgfqpoint{4.474076in}{4.654397in}}{\pgfqpoint{4.484527in}{4.650068in}}{\pgfqpoint{4.495422in}{4.650068in}}%
\pgfpathlineto{\pgfqpoint{4.495422in}{4.650068in}}%
\pgfpathclose%
\pgfusepath{stroke}%
\end{pgfscope}%
\begin{pgfscope}%
\pgfpathrectangle{\pgfqpoint{0.688192in}{0.670138in}}{\pgfqpoint{7.111808in}{5.061530in}}%
\pgfusepath{clip}%
\pgfsetbuttcap%
\pgfsetroundjoin%
\pgfsetlinewidth{1.003750pt}%
\definecolor{currentstroke}{rgb}{0.000000,0.000000,0.000000}%
\pgfsetstrokecolor{currentstroke}%
\pgfsetdash{}{0pt}%
\pgfpathmoveto{\pgfqpoint{1.663205in}{0.641981in}}%
\pgfpathcurveto{\pgfqpoint{1.674101in}{0.641981in}}{\pgfqpoint{1.684552in}{0.646310in}}{\pgfqpoint{1.692256in}{0.654014in}}%
\pgfpathcurveto{\pgfqpoint{1.699960in}{0.661719in}}{\pgfqpoint{1.704289in}{0.672169in}}{\pgfqpoint{1.704289in}{0.683065in}}%
\pgfpathcurveto{\pgfqpoint{1.704289in}{0.693961in}}{\pgfqpoint{1.699960in}{0.704411in}}{\pgfqpoint{1.692256in}{0.712116in}}%
\pgfpathcurveto{\pgfqpoint{1.684552in}{0.719820in}}{\pgfqpoint{1.674101in}{0.724149in}}{\pgfqpoint{1.663205in}{0.724149in}}%
\pgfpathcurveto{\pgfqpoint{1.652310in}{0.724149in}}{\pgfqpoint{1.641859in}{0.719820in}}{\pgfqpoint{1.634155in}{0.712116in}}%
\pgfpathcurveto{\pgfqpoint{1.626450in}{0.704411in}}{\pgfqpoint{1.622121in}{0.693961in}}{\pgfqpoint{1.622121in}{0.683065in}}%
\pgfpathcurveto{\pgfqpoint{1.622121in}{0.672169in}}{\pgfqpoint{1.626450in}{0.661719in}}{\pgfqpoint{1.634155in}{0.654014in}}%
\pgfpathcurveto{\pgfqpoint{1.641859in}{0.646310in}}{\pgfqpoint{1.652310in}{0.641981in}}{\pgfqpoint{1.663205in}{0.641981in}}%
\pgfusepath{stroke}%
\end{pgfscope}%
\begin{pgfscope}%
\pgfpathrectangle{\pgfqpoint{0.688192in}{0.670138in}}{\pgfqpoint{7.111808in}{5.061530in}}%
\pgfusepath{clip}%
\pgfsetbuttcap%
\pgfsetroundjoin%
\pgfsetlinewidth{1.003750pt}%
\definecolor{currentstroke}{rgb}{0.000000,0.000000,0.000000}%
\pgfsetstrokecolor{currentstroke}%
\pgfsetdash{}{0pt}%
\pgfpathmoveto{\pgfqpoint{1.364139in}{0.643320in}}%
\pgfpathcurveto{\pgfqpoint{1.375035in}{0.643320in}}{\pgfqpoint{1.385486in}{0.647649in}}{\pgfqpoint{1.393190in}{0.655353in}}%
\pgfpathcurveto{\pgfqpoint{1.400894in}{0.663057in}}{\pgfqpoint{1.405223in}{0.673508in}}{\pgfqpoint{1.405223in}{0.684404in}}%
\pgfpathcurveto{\pgfqpoint{1.405223in}{0.695299in}}{\pgfqpoint{1.400894in}{0.705750in}}{\pgfqpoint{1.393190in}{0.713454in}}%
\pgfpathcurveto{\pgfqpoint{1.385486in}{0.721159in}}{\pgfqpoint{1.375035in}{0.725488in}}{\pgfqpoint{1.364139in}{0.725488in}}%
\pgfpathcurveto{\pgfqpoint{1.353244in}{0.725488in}}{\pgfqpoint{1.342793in}{0.721159in}}{\pgfqpoint{1.335089in}{0.713454in}}%
\pgfpathcurveto{\pgfqpoint{1.327384in}{0.705750in}}{\pgfqpoint{1.323055in}{0.695299in}}{\pgfqpoint{1.323055in}{0.684404in}}%
\pgfpathcurveto{\pgfqpoint{1.323055in}{0.673508in}}{\pgfqpoint{1.327384in}{0.663057in}}{\pgfqpoint{1.335089in}{0.655353in}}%
\pgfpathcurveto{\pgfqpoint{1.342793in}{0.647649in}}{\pgfqpoint{1.353244in}{0.643320in}}{\pgfqpoint{1.364139in}{0.643320in}}%
\pgfusepath{stroke}%
\end{pgfscope}%
\begin{pgfscope}%
\pgfpathrectangle{\pgfqpoint{0.688192in}{0.670138in}}{\pgfqpoint{7.111808in}{5.061530in}}%
\pgfusepath{clip}%
\pgfsetbuttcap%
\pgfsetroundjoin%
\pgfsetlinewidth{1.003750pt}%
\definecolor{currentstroke}{rgb}{0.000000,0.000000,0.000000}%
\pgfsetstrokecolor{currentstroke}%
\pgfsetdash{}{0pt}%
\pgfpathmoveto{\pgfqpoint{3.618759in}{0.654558in}}%
\pgfpathcurveto{\pgfqpoint{3.629655in}{0.654558in}}{\pgfqpoint{3.640106in}{0.658887in}}{\pgfqpoint{3.647810in}{0.666591in}}%
\pgfpathcurveto{\pgfqpoint{3.655514in}{0.674295in}}{\pgfqpoint{3.659843in}{0.684746in}}{\pgfqpoint{3.659843in}{0.695642in}}%
\pgfpathcurveto{\pgfqpoint{3.659843in}{0.706537in}}{\pgfqpoint{3.655514in}{0.716988in}}{\pgfqpoint{3.647810in}{0.724692in}}%
\pgfpathcurveto{\pgfqpoint{3.640106in}{0.732397in}}{\pgfqpoint{3.629655in}{0.736726in}}{\pgfqpoint{3.618759in}{0.736726in}}%
\pgfpathcurveto{\pgfqpoint{3.607864in}{0.736726in}}{\pgfqpoint{3.597413in}{0.732397in}}{\pgfqpoint{3.589708in}{0.724692in}}%
\pgfpathcurveto{\pgfqpoint{3.582004in}{0.716988in}}{\pgfqpoint{3.577675in}{0.706537in}}{\pgfqpoint{3.577675in}{0.695642in}}%
\pgfpathcurveto{\pgfqpoint{3.577675in}{0.684746in}}{\pgfqpoint{3.582004in}{0.674295in}}{\pgfqpoint{3.589708in}{0.666591in}}%
\pgfpathcurveto{\pgfqpoint{3.597413in}{0.658887in}}{\pgfqpoint{3.607864in}{0.654558in}}{\pgfqpoint{3.618759in}{0.654558in}}%
\pgfusepath{stroke}%
\end{pgfscope}%
\begin{pgfscope}%
\pgfpathrectangle{\pgfqpoint{0.688192in}{0.670138in}}{\pgfqpoint{7.111808in}{5.061530in}}%
\pgfusepath{clip}%
\pgfsetbuttcap%
\pgfsetroundjoin%
\pgfsetlinewidth{1.003750pt}%
\definecolor{currentstroke}{rgb}{0.000000,0.000000,0.000000}%
\pgfsetstrokecolor{currentstroke}%
\pgfsetdash{}{0pt}%
\pgfpathmoveto{\pgfqpoint{1.815579in}{0.640817in}}%
\pgfpathcurveto{\pgfqpoint{1.826475in}{0.640817in}}{\pgfqpoint{1.836926in}{0.645146in}}{\pgfqpoint{1.844630in}{0.652851in}}%
\pgfpathcurveto{\pgfqpoint{1.852334in}{0.660555in}}{\pgfqpoint{1.856663in}{0.671006in}}{\pgfqpoint{1.856663in}{0.681901in}}%
\pgfpathcurveto{\pgfqpoint{1.856663in}{0.692797in}}{\pgfqpoint{1.852334in}{0.703248in}}{\pgfqpoint{1.844630in}{0.710952in}}%
\pgfpathcurveto{\pgfqpoint{1.836926in}{0.718656in}}{\pgfqpoint{1.826475in}{0.722985in}}{\pgfqpoint{1.815579in}{0.722985in}}%
\pgfpathcurveto{\pgfqpoint{1.804684in}{0.722985in}}{\pgfqpoint{1.794233in}{0.718656in}}{\pgfqpoint{1.786528in}{0.710952in}}%
\pgfpathcurveto{\pgfqpoint{1.778824in}{0.703248in}}{\pgfqpoint{1.774495in}{0.692797in}}{\pgfqpoint{1.774495in}{0.681901in}}%
\pgfpathcurveto{\pgfqpoint{1.774495in}{0.671006in}}{\pgfqpoint{1.778824in}{0.660555in}}{\pgfqpoint{1.786528in}{0.652851in}}%
\pgfpathcurveto{\pgfqpoint{1.794233in}{0.645146in}}{\pgfqpoint{1.804684in}{0.640817in}}{\pgfqpoint{1.815579in}{0.640817in}}%
\pgfusepath{stroke}%
\end{pgfscope}%
\begin{pgfscope}%
\pgfpathrectangle{\pgfqpoint{0.688192in}{0.670138in}}{\pgfqpoint{7.111808in}{5.061530in}}%
\pgfusepath{clip}%
\pgfsetbuttcap%
\pgfsetroundjoin%
\pgfsetlinewidth{1.003750pt}%
\definecolor{currentstroke}{rgb}{0.000000,0.000000,0.000000}%
\pgfsetstrokecolor{currentstroke}%
\pgfsetdash{}{0pt}%
\pgfpathmoveto{\pgfqpoint{4.576929in}{0.631060in}}%
\pgfpathcurveto{\pgfqpoint{4.587825in}{0.631060in}}{\pgfqpoint{4.598276in}{0.635389in}}{\pgfqpoint{4.605980in}{0.643094in}}%
\pgfpathcurveto{\pgfqpoint{4.613684in}{0.650798in}}{\pgfqpoint{4.618013in}{0.661249in}}{\pgfqpoint{4.618013in}{0.672144in}}%
\pgfpathcurveto{\pgfqpoint{4.618013in}{0.683040in}}{\pgfqpoint{4.613684in}{0.693491in}}{\pgfqpoint{4.605980in}{0.701195in}}%
\pgfpathcurveto{\pgfqpoint{4.598276in}{0.708899in}}{\pgfqpoint{4.587825in}{0.713228in}}{\pgfqpoint{4.576929in}{0.713228in}}%
\pgfpathcurveto{\pgfqpoint{4.566034in}{0.713228in}}{\pgfqpoint{4.555583in}{0.708899in}}{\pgfqpoint{4.547878in}{0.701195in}}%
\pgfpathcurveto{\pgfqpoint{4.540174in}{0.693491in}}{\pgfqpoint{4.535845in}{0.683040in}}{\pgfqpoint{4.535845in}{0.672144in}}%
\pgfpathcurveto{\pgfqpoint{4.535845in}{0.661249in}}{\pgfqpoint{4.540174in}{0.650798in}}{\pgfqpoint{4.547878in}{0.643094in}}%
\pgfpathcurveto{\pgfqpoint{4.555583in}{0.635389in}}{\pgfqpoint{4.566034in}{0.631060in}}{\pgfqpoint{4.576929in}{0.631060in}}%
\pgfusepath{stroke}%
\end{pgfscope}%
\begin{pgfscope}%
\pgfpathrectangle{\pgfqpoint{0.688192in}{0.670138in}}{\pgfqpoint{7.111808in}{5.061530in}}%
\pgfusepath{clip}%
\pgfsetbuttcap%
\pgfsetroundjoin%
\pgfsetlinewidth{1.003750pt}%
\definecolor{currentstroke}{rgb}{0.000000,0.000000,0.000000}%
\pgfsetstrokecolor{currentstroke}%
\pgfsetdash{}{0pt}%
\pgfpathmoveto{\pgfqpoint{4.661935in}{0.630846in}}%
\pgfpathcurveto{\pgfqpoint{4.672830in}{0.630846in}}{\pgfqpoint{4.683281in}{0.635175in}}{\pgfqpoint{4.690985in}{0.642879in}}%
\pgfpathcurveto{\pgfqpoint{4.698690in}{0.650583in}}{\pgfqpoint{4.703018in}{0.661034in}}{\pgfqpoint{4.703018in}{0.671930in}}%
\pgfpathcurveto{\pgfqpoint{4.703018in}{0.682825in}}{\pgfqpoint{4.698690in}{0.693276in}}{\pgfqpoint{4.690985in}{0.700980in}}%
\pgfpathcurveto{\pgfqpoint{4.683281in}{0.708685in}}{\pgfqpoint{4.672830in}{0.713014in}}{\pgfqpoint{4.661935in}{0.713014in}}%
\pgfpathcurveto{\pgfqpoint{4.651039in}{0.713014in}}{\pgfqpoint{4.640588in}{0.708685in}}{\pgfqpoint{4.632884in}{0.700980in}}%
\pgfpathcurveto{\pgfqpoint{4.625180in}{0.693276in}}{\pgfqpoint{4.620851in}{0.682825in}}{\pgfqpoint{4.620851in}{0.671930in}}%
\pgfpathcurveto{\pgfqpoint{4.620851in}{0.661034in}}{\pgfqpoint{4.625180in}{0.650583in}}{\pgfqpoint{4.632884in}{0.642879in}}%
\pgfpathcurveto{\pgfqpoint{4.640588in}{0.635175in}}{\pgfqpoint{4.651039in}{0.630846in}}{\pgfqpoint{4.661935in}{0.630846in}}%
\pgfusepath{stroke}%
\end{pgfscope}%
\begin{pgfscope}%
\pgfpathrectangle{\pgfqpoint{0.688192in}{0.670138in}}{\pgfqpoint{7.111808in}{5.061530in}}%
\pgfusepath{clip}%
\pgfsetbuttcap%
\pgfsetroundjoin%
\pgfsetlinewidth{1.003750pt}%
\definecolor{currentstroke}{rgb}{0.000000,0.000000,0.000000}%
\pgfsetstrokecolor{currentstroke}%
\pgfsetdash{}{0pt}%
\pgfpathmoveto{\pgfqpoint{1.198965in}{0.644318in}}%
\pgfpathcurveto{\pgfqpoint{1.209860in}{0.644318in}}{\pgfqpoint{1.220311in}{0.648647in}}{\pgfqpoint{1.228016in}{0.656352in}}%
\pgfpathcurveto{\pgfqpoint{1.235720in}{0.664056in}}{\pgfqpoint{1.240049in}{0.674507in}}{\pgfqpoint{1.240049in}{0.685402in}}%
\pgfpathcurveto{\pgfqpoint{1.240049in}{0.696298in}}{\pgfqpoint{1.235720in}{0.706749in}}{\pgfqpoint{1.228016in}{0.714453in}}%
\pgfpathcurveto{\pgfqpoint{1.220311in}{0.722157in}}{\pgfqpoint{1.209860in}{0.726486in}}{\pgfqpoint{1.198965in}{0.726486in}}%
\pgfpathcurveto{\pgfqpoint{1.188069in}{0.726486in}}{\pgfqpoint{1.177619in}{0.722157in}}{\pgfqpoint{1.169914in}{0.714453in}}%
\pgfpathcurveto{\pgfqpoint{1.162210in}{0.706749in}}{\pgfqpoint{1.157881in}{0.696298in}}{\pgfqpoint{1.157881in}{0.685402in}}%
\pgfpathcurveto{\pgfqpoint{1.157881in}{0.674507in}}{\pgfqpoint{1.162210in}{0.664056in}}{\pgfqpoint{1.169914in}{0.656352in}}%
\pgfpathcurveto{\pgfqpoint{1.177619in}{0.648647in}}{\pgfqpoint{1.188069in}{0.644318in}}{\pgfqpoint{1.198965in}{0.644318in}}%
\pgfusepath{stroke}%
\end{pgfscope}%
\begin{pgfscope}%
\pgfpathrectangle{\pgfqpoint{0.688192in}{0.670138in}}{\pgfqpoint{7.111808in}{5.061530in}}%
\pgfusepath{clip}%
\pgfsetbuttcap%
\pgfsetroundjoin%
\pgfsetlinewidth{1.003750pt}%
\definecolor{currentstroke}{rgb}{0.000000,0.000000,0.000000}%
\pgfsetstrokecolor{currentstroke}%
\pgfsetdash{}{0pt}%
\pgfpathmoveto{\pgfqpoint{5.657223in}{0.629365in}}%
\pgfpathcurveto{\pgfqpoint{5.668119in}{0.629365in}}{\pgfqpoint{5.678570in}{0.633694in}}{\pgfqpoint{5.686274in}{0.641399in}}%
\pgfpathcurveto{\pgfqpoint{5.693979in}{0.649103in}}{\pgfqpoint{5.698307in}{0.659554in}}{\pgfqpoint{5.698307in}{0.670449in}}%
\pgfpathcurveto{\pgfqpoint{5.698307in}{0.681345in}}{\pgfqpoint{5.693979in}{0.691796in}}{\pgfqpoint{5.686274in}{0.699500in}}%
\pgfpathcurveto{\pgfqpoint{5.678570in}{0.707204in}}{\pgfqpoint{5.668119in}{0.711533in}}{\pgfqpoint{5.657223in}{0.711533in}}%
\pgfpathcurveto{\pgfqpoint{5.646328in}{0.711533in}}{\pgfqpoint{5.635877in}{0.707204in}}{\pgfqpoint{5.628173in}{0.699500in}}%
\pgfpathcurveto{\pgfqpoint{5.620468in}{0.691796in}}{\pgfqpoint{5.616140in}{0.681345in}}{\pgfqpoint{5.616140in}{0.670449in}}%
\pgfpathcurveto{\pgfqpoint{5.616140in}{0.659554in}}{\pgfqpoint{5.620468in}{0.649103in}}{\pgfqpoint{5.628173in}{0.641399in}}%
\pgfpathcurveto{\pgfqpoint{5.635877in}{0.633694in}}{\pgfqpoint{5.646328in}{0.629365in}}{\pgfqpoint{5.657223in}{0.629365in}}%
\pgfusepath{stroke}%
\end{pgfscope}%
\begin{pgfscope}%
\pgfpathrectangle{\pgfqpoint{0.688192in}{0.670138in}}{\pgfqpoint{7.111808in}{5.061530in}}%
\pgfusepath{clip}%
\pgfsetbuttcap%
\pgfsetroundjoin%
\pgfsetlinewidth{1.003750pt}%
\definecolor{currentstroke}{rgb}{0.000000,0.000000,0.000000}%
\pgfsetstrokecolor{currentstroke}%
\pgfsetdash{}{0pt}%
\pgfpathmoveto{\pgfqpoint{1.407033in}{0.642986in}}%
\pgfpathcurveto{\pgfqpoint{1.417928in}{0.642986in}}{\pgfqpoint{1.428379in}{0.647315in}}{\pgfqpoint{1.436083in}{0.655020in}}%
\pgfpathcurveto{\pgfqpoint{1.443788in}{0.662724in}}{\pgfqpoint{1.448117in}{0.673175in}}{\pgfqpoint{1.448117in}{0.684070in}}%
\pgfpathcurveto{\pgfqpoint{1.448117in}{0.694966in}}{\pgfqpoint{1.443788in}{0.705417in}}{\pgfqpoint{1.436083in}{0.713121in}}%
\pgfpathcurveto{\pgfqpoint{1.428379in}{0.720825in}}{\pgfqpoint{1.417928in}{0.725154in}}{\pgfqpoint{1.407033in}{0.725154in}}%
\pgfpathcurveto{\pgfqpoint{1.396137in}{0.725154in}}{\pgfqpoint{1.385686in}{0.720825in}}{\pgfqpoint{1.377982in}{0.713121in}}%
\pgfpathcurveto{\pgfqpoint{1.370278in}{0.705417in}}{\pgfqpoint{1.365949in}{0.694966in}}{\pgfqpoint{1.365949in}{0.684070in}}%
\pgfpathcurveto{\pgfqpoint{1.365949in}{0.673175in}}{\pgfqpoint{1.370278in}{0.662724in}}{\pgfqpoint{1.377982in}{0.655020in}}%
\pgfpathcurveto{\pgfqpoint{1.385686in}{0.647315in}}{\pgfqpoint{1.396137in}{0.642986in}}{\pgfqpoint{1.407033in}{0.642986in}}%
\pgfusepath{stroke}%
\end{pgfscope}%
\begin{pgfscope}%
\pgfpathrectangle{\pgfqpoint{0.688192in}{0.670138in}}{\pgfqpoint{7.111808in}{5.061530in}}%
\pgfusepath{clip}%
\pgfsetbuttcap%
\pgfsetroundjoin%
\pgfsetlinewidth{1.003750pt}%
\definecolor{currentstroke}{rgb}{0.000000,0.000000,0.000000}%
\pgfsetstrokecolor{currentstroke}%
\pgfsetdash{}{0pt}%
\pgfpathmoveto{\pgfqpoint{1.473551in}{0.642864in}}%
\pgfpathcurveto{\pgfqpoint{1.484447in}{0.642864in}}{\pgfqpoint{1.494898in}{0.647193in}}{\pgfqpoint{1.502602in}{0.654897in}}%
\pgfpathcurveto{\pgfqpoint{1.510306in}{0.662602in}}{\pgfqpoint{1.514635in}{0.673052in}}{\pgfqpoint{1.514635in}{0.683948in}}%
\pgfpathcurveto{\pgfqpoint{1.514635in}{0.694844in}}{\pgfqpoint{1.510306in}{0.705294in}}{\pgfqpoint{1.502602in}{0.712999in}}%
\pgfpathcurveto{\pgfqpoint{1.494898in}{0.720703in}}{\pgfqpoint{1.484447in}{0.725032in}}{\pgfqpoint{1.473551in}{0.725032in}}%
\pgfpathcurveto{\pgfqpoint{1.462656in}{0.725032in}}{\pgfqpoint{1.452205in}{0.720703in}}{\pgfqpoint{1.444500in}{0.712999in}}%
\pgfpathcurveto{\pgfqpoint{1.436796in}{0.705294in}}{\pgfqpoint{1.432467in}{0.694844in}}{\pgfqpoint{1.432467in}{0.683948in}}%
\pgfpathcurveto{\pgfqpoint{1.432467in}{0.673052in}}{\pgfqpoint{1.436796in}{0.662602in}}{\pgfqpoint{1.444500in}{0.654897in}}%
\pgfpathcurveto{\pgfqpoint{1.452205in}{0.647193in}}{\pgfqpoint{1.462656in}{0.642864in}}{\pgfqpoint{1.473551in}{0.642864in}}%
\pgfusepath{stroke}%
\end{pgfscope}%
\begin{pgfscope}%
\pgfpathrectangle{\pgfqpoint{0.688192in}{0.670138in}}{\pgfqpoint{7.111808in}{5.061530in}}%
\pgfusepath{clip}%
\pgfsetbuttcap%
\pgfsetroundjoin%
\pgfsetlinewidth{1.003750pt}%
\definecolor{currentstroke}{rgb}{0.000000,0.000000,0.000000}%
\pgfsetstrokecolor{currentstroke}%
\pgfsetdash{}{0pt}%
\pgfpathmoveto{\pgfqpoint{4.957070in}{0.660271in}}%
\pgfpathcurveto{\pgfqpoint{4.967966in}{0.660271in}}{\pgfqpoint{4.978417in}{0.664600in}}{\pgfqpoint{4.986121in}{0.672304in}}%
\pgfpathcurveto{\pgfqpoint{4.993825in}{0.680009in}}{\pgfqpoint{4.998154in}{0.690460in}}{\pgfqpoint{4.998154in}{0.701355in}}%
\pgfpathcurveto{\pgfqpoint{4.998154in}{0.712251in}}{\pgfqpoint{4.993825in}{0.722701in}}{\pgfqpoint{4.986121in}{0.730406in}}%
\pgfpathcurveto{\pgfqpoint{4.978417in}{0.738110in}}{\pgfqpoint{4.967966in}{0.742439in}}{\pgfqpoint{4.957070in}{0.742439in}}%
\pgfpathcurveto{\pgfqpoint{4.946175in}{0.742439in}}{\pgfqpoint{4.935724in}{0.738110in}}{\pgfqpoint{4.928020in}{0.730406in}}%
\pgfpathcurveto{\pgfqpoint{4.920315in}{0.722701in}}{\pgfqpoint{4.915986in}{0.712251in}}{\pgfqpoint{4.915986in}{0.701355in}}%
\pgfpathcurveto{\pgfqpoint{4.915986in}{0.690460in}}{\pgfqpoint{4.920315in}{0.680009in}}{\pgfqpoint{4.928020in}{0.672304in}}%
\pgfpathcurveto{\pgfqpoint{4.935724in}{0.664600in}}{\pgfqpoint{4.946175in}{0.660271in}}{\pgfqpoint{4.957070in}{0.660271in}}%
\pgfusepath{stroke}%
\end{pgfscope}%
\begin{pgfscope}%
\pgfpathrectangle{\pgfqpoint{0.688192in}{0.670138in}}{\pgfqpoint{7.111808in}{5.061530in}}%
\pgfusepath{clip}%
\pgfsetbuttcap%
\pgfsetroundjoin%
\pgfsetlinewidth{1.003750pt}%
\definecolor{currentstroke}{rgb}{0.000000,0.000000,0.000000}%
\pgfsetstrokecolor{currentstroke}%
\pgfsetdash{}{0pt}%
\pgfpathmoveto{\pgfqpoint{2.512089in}{1.884528in}}%
\pgfpathcurveto{\pgfqpoint{2.522985in}{1.884528in}}{\pgfqpoint{2.533436in}{1.888857in}}{\pgfqpoint{2.541140in}{1.896561in}}%
\pgfpathcurveto{\pgfqpoint{2.548844in}{1.904265in}}{\pgfqpoint{2.553173in}{1.914716in}}{\pgfqpoint{2.553173in}{1.925612in}}%
\pgfpathcurveto{\pgfqpoint{2.553173in}{1.936507in}}{\pgfqpoint{2.548844in}{1.946958in}}{\pgfqpoint{2.541140in}{1.954662in}}%
\pgfpathcurveto{\pgfqpoint{2.533436in}{1.962367in}}{\pgfqpoint{2.522985in}{1.966696in}}{\pgfqpoint{2.512089in}{1.966696in}}%
\pgfpathcurveto{\pgfqpoint{2.501194in}{1.966696in}}{\pgfqpoint{2.490743in}{1.962367in}}{\pgfqpoint{2.483039in}{1.954662in}}%
\pgfpathcurveto{\pgfqpoint{2.475334in}{1.946958in}}{\pgfqpoint{2.471005in}{1.936507in}}{\pgfqpoint{2.471005in}{1.925612in}}%
\pgfpathcurveto{\pgfqpoint{2.471005in}{1.914716in}}{\pgfqpoint{2.475334in}{1.904265in}}{\pgfqpoint{2.483039in}{1.896561in}}%
\pgfpathcurveto{\pgfqpoint{2.490743in}{1.888857in}}{\pgfqpoint{2.501194in}{1.884528in}}{\pgfqpoint{2.512089in}{1.884528in}}%
\pgfpathlineto{\pgfqpoint{2.512089in}{1.884528in}}%
\pgfpathclose%
\pgfusepath{stroke}%
\end{pgfscope}%
\begin{pgfscope}%
\pgfpathrectangle{\pgfqpoint{0.688192in}{0.670138in}}{\pgfqpoint{7.111808in}{5.061530in}}%
\pgfusepath{clip}%
\pgfsetbuttcap%
\pgfsetroundjoin%
\pgfsetlinewidth{1.003750pt}%
\definecolor{currentstroke}{rgb}{0.000000,0.000000,0.000000}%
\pgfsetstrokecolor{currentstroke}%
\pgfsetdash{}{0pt}%
\pgfpathmoveto{\pgfqpoint{4.170511in}{1.865457in}}%
\pgfpathcurveto{\pgfqpoint{4.181407in}{1.865457in}}{\pgfqpoint{4.191857in}{1.869786in}}{\pgfqpoint{4.199562in}{1.877490in}}%
\pgfpathcurveto{\pgfqpoint{4.207266in}{1.885195in}}{\pgfqpoint{4.211595in}{1.895645in}}{\pgfqpoint{4.211595in}{1.906541in}}%
\pgfpathcurveto{\pgfqpoint{4.211595in}{1.917436in}}{\pgfqpoint{4.207266in}{1.927887in}}{\pgfqpoint{4.199562in}{1.935592in}}%
\pgfpathcurveto{\pgfqpoint{4.191857in}{1.943296in}}{\pgfqpoint{4.181407in}{1.947625in}}{\pgfqpoint{4.170511in}{1.947625in}}%
\pgfpathcurveto{\pgfqpoint{4.159615in}{1.947625in}}{\pgfqpoint{4.149165in}{1.943296in}}{\pgfqpoint{4.141460in}{1.935592in}}%
\pgfpathcurveto{\pgfqpoint{4.133756in}{1.927887in}}{\pgfqpoint{4.129427in}{1.917436in}}{\pgfqpoint{4.129427in}{1.906541in}}%
\pgfpathcurveto{\pgfqpoint{4.129427in}{1.895645in}}{\pgfqpoint{4.133756in}{1.885195in}}{\pgfqpoint{4.141460in}{1.877490in}}%
\pgfpathcurveto{\pgfqpoint{4.149165in}{1.869786in}}{\pgfqpoint{4.159615in}{1.865457in}}{\pgfqpoint{4.170511in}{1.865457in}}%
\pgfpathlineto{\pgfqpoint{4.170511in}{1.865457in}}%
\pgfpathclose%
\pgfusepath{stroke}%
\end{pgfscope}%
\begin{pgfscope}%
\pgfpathrectangle{\pgfqpoint{0.688192in}{0.670138in}}{\pgfqpoint{7.111808in}{5.061530in}}%
\pgfusepath{clip}%
\pgfsetbuttcap%
\pgfsetroundjoin%
\pgfsetlinewidth{1.003750pt}%
\definecolor{currentstroke}{rgb}{0.000000,0.000000,0.000000}%
\pgfsetstrokecolor{currentstroke}%
\pgfsetdash{}{0pt}%
\pgfpathmoveto{\pgfqpoint{2.166232in}{1.289290in}}%
\pgfpathcurveto{\pgfqpoint{2.177128in}{1.289290in}}{\pgfqpoint{2.187578in}{1.293619in}}{\pgfqpoint{2.195283in}{1.301324in}}%
\pgfpathcurveto{\pgfqpoint{2.202987in}{1.309028in}}{\pgfqpoint{2.207316in}{1.319479in}}{\pgfqpoint{2.207316in}{1.330374in}}%
\pgfpathcurveto{\pgfqpoint{2.207316in}{1.341270in}}{\pgfqpoint{2.202987in}{1.351721in}}{\pgfqpoint{2.195283in}{1.359425in}}%
\pgfpathcurveto{\pgfqpoint{2.187578in}{1.367129in}}{\pgfqpoint{2.177128in}{1.371458in}}{\pgfqpoint{2.166232in}{1.371458in}}%
\pgfpathcurveto{\pgfqpoint{2.155336in}{1.371458in}}{\pgfqpoint{2.144886in}{1.367129in}}{\pgfqpoint{2.137181in}{1.359425in}}%
\pgfpathcurveto{\pgfqpoint{2.129477in}{1.351721in}}{\pgfqpoint{2.125148in}{1.341270in}}{\pgfqpoint{2.125148in}{1.330374in}}%
\pgfpathcurveto{\pgfqpoint{2.125148in}{1.319479in}}{\pgfqpoint{2.129477in}{1.309028in}}{\pgfqpoint{2.137181in}{1.301324in}}%
\pgfpathcurveto{\pgfqpoint{2.144886in}{1.293619in}}{\pgfqpoint{2.155336in}{1.289290in}}{\pgfqpoint{2.166232in}{1.289290in}}%
\pgfpathlineto{\pgfqpoint{2.166232in}{1.289290in}}%
\pgfpathclose%
\pgfusepath{stroke}%
\end{pgfscope}%
\begin{pgfscope}%
\pgfpathrectangle{\pgfqpoint{0.688192in}{0.670138in}}{\pgfqpoint{7.111808in}{5.061530in}}%
\pgfusepath{clip}%
\pgfsetbuttcap%
\pgfsetroundjoin%
\pgfsetlinewidth{1.003750pt}%
\definecolor{currentstroke}{rgb}{0.000000,0.000000,0.000000}%
\pgfsetstrokecolor{currentstroke}%
\pgfsetdash{}{0pt}%
\pgfpathmoveto{\pgfqpoint{0.939729in}{0.677113in}}%
\pgfpathcurveto{\pgfqpoint{0.950625in}{0.677113in}}{\pgfqpoint{0.961075in}{0.681442in}}{\pgfqpoint{0.968780in}{0.689146in}}%
\pgfpathcurveto{\pgfqpoint{0.976484in}{0.696850in}}{\pgfqpoint{0.980813in}{0.707301in}}{\pgfqpoint{0.980813in}{0.718197in}}%
\pgfpathcurveto{\pgfqpoint{0.980813in}{0.729092in}}{\pgfqpoint{0.976484in}{0.739543in}}{\pgfqpoint{0.968780in}{0.747247in}}%
\pgfpathcurveto{\pgfqpoint{0.961075in}{0.754952in}}{\pgfqpoint{0.950625in}{0.759281in}}{\pgfqpoint{0.939729in}{0.759281in}}%
\pgfpathcurveto{\pgfqpoint{0.928834in}{0.759281in}}{\pgfqpoint{0.918383in}{0.754952in}}{\pgfqpoint{0.910678in}{0.747247in}}%
\pgfpathcurveto{\pgfqpoint{0.902974in}{0.739543in}}{\pgfqpoint{0.898645in}{0.729092in}}{\pgfqpoint{0.898645in}{0.718197in}}%
\pgfpathcurveto{\pgfqpoint{0.898645in}{0.707301in}}{\pgfqpoint{0.902974in}{0.696850in}}{\pgfqpoint{0.910678in}{0.689146in}}%
\pgfpathcurveto{\pgfqpoint{0.918383in}{0.681442in}}{\pgfqpoint{0.928834in}{0.677113in}}{\pgfqpoint{0.939729in}{0.677113in}}%
\pgfpathlineto{\pgfqpoint{0.939729in}{0.677113in}}%
\pgfpathclose%
\pgfusepath{stroke}%
\end{pgfscope}%
\begin{pgfscope}%
\pgfpathrectangle{\pgfqpoint{0.688192in}{0.670138in}}{\pgfqpoint{7.111808in}{5.061530in}}%
\pgfusepath{clip}%
\pgfsetbuttcap%
\pgfsetroundjoin%
\pgfsetlinewidth{1.003750pt}%
\definecolor{currentstroke}{rgb}{0.000000,0.000000,0.000000}%
\pgfsetstrokecolor{currentstroke}%
\pgfsetdash{}{0pt}%
\pgfpathmoveto{\pgfqpoint{0.849953in}{0.698273in}}%
\pgfpathcurveto{\pgfqpoint{0.860849in}{0.698273in}}{\pgfqpoint{0.871300in}{0.702602in}}{\pgfqpoint{0.879004in}{0.710307in}}%
\pgfpathcurveto{\pgfqpoint{0.886709in}{0.718011in}}{\pgfqpoint{0.891037in}{0.728462in}}{\pgfqpoint{0.891037in}{0.739357in}}%
\pgfpathcurveto{\pgfqpoint{0.891037in}{0.750253in}}{\pgfqpoint{0.886709in}{0.760704in}}{\pgfqpoint{0.879004in}{0.768408in}}%
\pgfpathcurveto{\pgfqpoint{0.871300in}{0.776112in}}{\pgfqpoint{0.860849in}{0.780441in}}{\pgfqpoint{0.849953in}{0.780441in}}%
\pgfpathcurveto{\pgfqpoint{0.839058in}{0.780441in}}{\pgfqpoint{0.828607in}{0.776112in}}{\pgfqpoint{0.820903in}{0.768408in}}%
\pgfpathcurveto{\pgfqpoint{0.813198in}{0.760704in}}{\pgfqpoint{0.808870in}{0.750253in}}{\pgfqpoint{0.808870in}{0.739357in}}%
\pgfpathcurveto{\pgfqpoint{0.808870in}{0.728462in}}{\pgfqpoint{0.813198in}{0.718011in}}{\pgfqpoint{0.820903in}{0.710307in}}%
\pgfpathcurveto{\pgfqpoint{0.828607in}{0.702602in}}{\pgfqpoint{0.839058in}{0.698273in}}{\pgfqpoint{0.849953in}{0.698273in}}%
\pgfpathlineto{\pgfqpoint{0.849953in}{0.698273in}}%
\pgfpathclose%
\pgfusepath{stroke}%
\end{pgfscope}%
\begin{pgfscope}%
\pgfpathrectangle{\pgfqpoint{0.688192in}{0.670138in}}{\pgfqpoint{7.111808in}{5.061530in}}%
\pgfusepath{clip}%
\pgfsetbuttcap%
\pgfsetroundjoin%
\pgfsetlinewidth{1.003750pt}%
\definecolor{currentstroke}{rgb}{0.000000,0.000000,0.000000}%
\pgfsetstrokecolor{currentstroke}%
\pgfsetdash{}{0pt}%
\pgfpathmoveto{\pgfqpoint{4.491507in}{0.631138in}}%
\pgfpathcurveto{\pgfqpoint{4.502402in}{0.631138in}}{\pgfqpoint{4.512853in}{0.635467in}}{\pgfqpoint{4.520557in}{0.643171in}}%
\pgfpathcurveto{\pgfqpoint{4.528262in}{0.650875in}}{\pgfqpoint{4.532590in}{0.661326in}}{\pgfqpoint{4.532590in}{0.672222in}}%
\pgfpathcurveto{\pgfqpoint{4.532590in}{0.683117in}}{\pgfqpoint{4.528262in}{0.693568in}}{\pgfqpoint{4.520557in}{0.701272in}}%
\pgfpathcurveto{\pgfqpoint{4.512853in}{0.708977in}}{\pgfqpoint{4.502402in}{0.713305in}}{\pgfqpoint{4.491507in}{0.713305in}}%
\pgfpathcurveto{\pgfqpoint{4.480611in}{0.713305in}}{\pgfqpoint{4.470160in}{0.708977in}}{\pgfqpoint{4.462456in}{0.701272in}}%
\pgfpathcurveto{\pgfqpoint{4.454752in}{0.693568in}}{\pgfqpoint{4.450423in}{0.683117in}}{\pgfqpoint{4.450423in}{0.672222in}}%
\pgfpathcurveto{\pgfqpoint{4.450423in}{0.661326in}}{\pgfqpoint{4.454752in}{0.650875in}}{\pgfqpoint{4.462456in}{0.643171in}}%
\pgfpathcurveto{\pgfqpoint{4.470160in}{0.635467in}}{\pgfqpoint{4.480611in}{0.631138in}}{\pgfqpoint{4.491507in}{0.631138in}}%
\pgfusepath{stroke}%
\end{pgfscope}%
\begin{pgfscope}%
\pgfpathrectangle{\pgfqpoint{0.688192in}{0.670138in}}{\pgfqpoint{7.111808in}{5.061530in}}%
\pgfusepath{clip}%
\pgfsetbuttcap%
\pgfsetroundjoin%
\pgfsetlinewidth{1.003750pt}%
\definecolor{currentstroke}{rgb}{0.000000,0.000000,0.000000}%
\pgfsetstrokecolor{currentstroke}%
\pgfsetdash{}{0pt}%
\pgfpathmoveto{\pgfqpoint{6.618775in}{2.150438in}}%
\pgfpathcurveto{\pgfqpoint{6.629670in}{2.150438in}}{\pgfqpoint{6.640121in}{2.154766in}}{\pgfqpoint{6.647825in}{2.162471in}}%
\pgfpathcurveto{\pgfqpoint{6.655530in}{2.170175in}}{\pgfqpoint{6.659859in}{2.180626in}}{\pgfqpoint{6.659859in}{2.191521in}}%
\pgfpathcurveto{\pgfqpoint{6.659859in}{2.202417in}}{\pgfqpoint{6.655530in}{2.212868in}}{\pgfqpoint{6.647825in}{2.220572in}}%
\pgfpathcurveto{\pgfqpoint{6.640121in}{2.228277in}}{\pgfqpoint{6.629670in}{2.232605in}}{\pgfqpoint{6.618775in}{2.232605in}}%
\pgfpathcurveto{\pgfqpoint{6.607879in}{2.232605in}}{\pgfqpoint{6.597428in}{2.228277in}}{\pgfqpoint{6.589724in}{2.220572in}}%
\pgfpathcurveto{\pgfqpoint{6.582020in}{2.212868in}}{\pgfqpoint{6.577691in}{2.202417in}}{\pgfqpoint{6.577691in}{2.191521in}}%
\pgfpathcurveto{\pgfqpoint{6.577691in}{2.180626in}}{\pgfqpoint{6.582020in}{2.170175in}}{\pgfqpoint{6.589724in}{2.162471in}}%
\pgfpathcurveto{\pgfqpoint{6.597428in}{2.154766in}}{\pgfqpoint{6.607879in}{2.150438in}}{\pgfqpoint{6.618775in}{2.150438in}}%
\pgfpathlineto{\pgfqpoint{6.618775in}{2.150438in}}%
\pgfpathclose%
\pgfusepath{stroke}%
\end{pgfscope}%
\begin{pgfscope}%
\pgfpathrectangle{\pgfqpoint{0.688192in}{0.670138in}}{\pgfqpoint{7.111808in}{5.061530in}}%
\pgfusepath{clip}%
\pgfsetbuttcap%
\pgfsetroundjoin%
\pgfsetlinewidth{1.003750pt}%
\definecolor{currentstroke}{rgb}{0.000000,0.000000,0.000000}%
\pgfsetstrokecolor{currentstroke}%
\pgfsetdash{}{0pt}%
\pgfpathmoveto{\pgfqpoint{0.849953in}{0.698273in}}%
\pgfpathcurveto{\pgfqpoint{0.860849in}{0.698273in}}{\pgfqpoint{0.871300in}{0.702602in}}{\pgfqpoint{0.879004in}{0.710307in}}%
\pgfpathcurveto{\pgfqpoint{0.886709in}{0.718011in}}{\pgfqpoint{0.891037in}{0.728462in}}{\pgfqpoint{0.891037in}{0.739357in}}%
\pgfpathcurveto{\pgfqpoint{0.891037in}{0.750253in}}{\pgfqpoint{0.886709in}{0.760704in}}{\pgfqpoint{0.879004in}{0.768408in}}%
\pgfpathcurveto{\pgfqpoint{0.871300in}{0.776112in}}{\pgfqpoint{0.860849in}{0.780441in}}{\pgfqpoint{0.849953in}{0.780441in}}%
\pgfpathcurveto{\pgfqpoint{0.839058in}{0.780441in}}{\pgfqpoint{0.828607in}{0.776112in}}{\pgfqpoint{0.820903in}{0.768408in}}%
\pgfpathcurveto{\pgfqpoint{0.813198in}{0.760704in}}{\pgfqpoint{0.808870in}{0.750253in}}{\pgfqpoint{0.808870in}{0.739357in}}%
\pgfpathcurveto{\pgfqpoint{0.808870in}{0.728462in}}{\pgfqpoint{0.813198in}{0.718011in}}{\pgfqpoint{0.820903in}{0.710307in}}%
\pgfpathcurveto{\pgfqpoint{0.828607in}{0.702602in}}{\pgfqpoint{0.839058in}{0.698273in}}{\pgfqpoint{0.849953in}{0.698273in}}%
\pgfpathlineto{\pgfqpoint{0.849953in}{0.698273in}}%
\pgfpathclose%
\pgfusepath{stroke}%
\end{pgfscope}%
\begin{pgfscope}%
\pgfpathrectangle{\pgfqpoint{0.688192in}{0.670138in}}{\pgfqpoint{7.111808in}{5.061530in}}%
\pgfusepath{clip}%
\pgfsetbuttcap%
\pgfsetroundjoin%
\pgfsetlinewidth{1.003750pt}%
\definecolor{currentstroke}{rgb}{0.000000,0.000000,0.000000}%
\pgfsetstrokecolor{currentstroke}%
\pgfsetdash{}{0pt}%
\pgfpathmoveto{\pgfqpoint{1.364139in}{0.643320in}}%
\pgfpathcurveto{\pgfqpoint{1.375035in}{0.643320in}}{\pgfqpoint{1.385486in}{0.647649in}}{\pgfqpoint{1.393190in}{0.655353in}}%
\pgfpathcurveto{\pgfqpoint{1.400894in}{0.663057in}}{\pgfqpoint{1.405223in}{0.673508in}}{\pgfqpoint{1.405223in}{0.684404in}}%
\pgfpathcurveto{\pgfqpoint{1.405223in}{0.695299in}}{\pgfqpoint{1.400894in}{0.705750in}}{\pgfqpoint{1.393190in}{0.713454in}}%
\pgfpathcurveto{\pgfqpoint{1.385486in}{0.721159in}}{\pgfqpoint{1.375035in}{0.725488in}}{\pgfqpoint{1.364139in}{0.725488in}}%
\pgfpathcurveto{\pgfqpoint{1.353244in}{0.725488in}}{\pgfqpoint{1.342793in}{0.721159in}}{\pgfqpoint{1.335089in}{0.713454in}}%
\pgfpathcurveto{\pgfqpoint{1.327384in}{0.705750in}}{\pgfqpoint{1.323055in}{0.695299in}}{\pgfqpoint{1.323055in}{0.684404in}}%
\pgfpathcurveto{\pgfqpoint{1.323055in}{0.673508in}}{\pgfqpoint{1.327384in}{0.663057in}}{\pgfqpoint{1.335089in}{0.655353in}}%
\pgfpathcurveto{\pgfqpoint{1.342793in}{0.647649in}}{\pgfqpoint{1.353244in}{0.643320in}}{\pgfqpoint{1.364139in}{0.643320in}}%
\pgfusepath{stroke}%
\end{pgfscope}%
\begin{pgfscope}%
\pgfpathrectangle{\pgfqpoint{0.688192in}{0.670138in}}{\pgfqpoint{7.111808in}{5.061530in}}%
\pgfusepath{clip}%
\pgfsetbuttcap%
\pgfsetroundjoin%
\pgfsetlinewidth{1.003750pt}%
\definecolor{currentstroke}{rgb}{0.000000,0.000000,0.000000}%
\pgfsetstrokecolor{currentstroke}%
\pgfsetdash{}{0pt}%
\pgfpathmoveto{\pgfqpoint{4.726657in}{0.630576in}}%
\pgfpathcurveto{\pgfqpoint{4.737552in}{0.630576in}}{\pgfqpoint{4.748003in}{0.634905in}}{\pgfqpoint{4.755707in}{0.642610in}}%
\pgfpathcurveto{\pgfqpoint{4.763412in}{0.650314in}}{\pgfqpoint{4.767741in}{0.660765in}}{\pgfqpoint{4.767741in}{0.671660in}}%
\pgfpathcurveto{\pgfqpoint{4.767741in}{0.682556in}}{\pgfqpoint{4.763412in}{0.693007in}}{\pgfqpoint{4.755707in}{0.700711in}}%
\pgfpathcurveto{\pgfqpoint{4.748003in}{0.708415in}}{\pgfqpoint{4.737552in}{0.712744in}}{\pgfqpoint{4.726657in}{0.712744in}}%
\pgfpathcurveto{\pgfqpoint{4.715761in}{0.712744in}}{\pgfqpoint{4.705310in}{0.708415in}}{\pgfqpoint{4.697606in}{0.700711in}}%
\pgfpathcurveto{\pgfqpoint{4.689902in}{0.693007in}}{\pgfqpoint{4.685573in}{0.682556in}}{\pgfqpoint{4.685573in}{0.671660in}}%
\pgfpathcurveto{\pgfqpoint{4.685573in}{0.660765in}}{\pgfqpoint{4.689902in}{0.650314in}}{\pgfqpoint{4.697606in}{0.642610in}}%
\pgfpathcurveto{\pgfqpoint{4.705310in}{0.634905in}}{\pgfqpoint{4.715761in}{0.630576in}}{\pgfqpoint{4.726657in}{0.630576in}}%
\pgfusepath{stroke}%
\end{pgfscope}%
\begin{pgfscope}%
\pgfpathrectangle{\pgfqpoint{0.688192in}{0.670138in}}{\pgfqpoint{7.111808in}{5.061530in}}%
\pgfusepath{clip}%
\pgfsetbuttcap%
\pgfsetroundjoin%
\pgfsetlinewidth{1.003750pt}%
\definecolor{currentstroke}{rgb}{0.000000,0.000000,0.000000}%
\pgfsetstrokecolor{currentstroke}%
\pgfsetdash{}{0pt}%
\pgfpathmoveto{\pgfqpoint{1.364139in}{0.643320in}}%
\pgfpathcurveto{\pgfqpoint{1.375035in}{0.643320in}}{\pgfqpoint{1.385486in}{0.647649in}}{\pgfqpoint{1.393190in}{0.655353in}}%
\pgfpathcurveto{\pgfqpoint{1.400894in}{0.663057in}}{\pgfqpoint{1.405223in}{0.673508in}}{\pgfqpoint{1.405223in}{0.684404in}}%
\pgfpathcurveto{\pgfqpoint{1.405223in}{0.695299in}}{\pgfqpoint{1.400894in}{0.705750in}}{\pgfqpoint{1.393190in}{0.713454in}}%
\pgfpathcurveto{\pgfqpoint{1.385486in}{0.721159in}}{\pgfqpoint{1.375035in}{0.725488in}}{\pgfqpoint{1.364139in}{0.725488in}}%
\pgfpathcurveto{\pgfqpoint{1.353244in}{0.725488in}}{\pgfqpoint{1.342793in}{0.721159in}}{\pgfqpoint{1.335089in}{0.713454in}}%
\pgfpathcurveto{\pgfqpoint{1.327384in}{0.705750in}}{\pgfqpoint{1.323055in}{0.695299in}}{\pgfqpoint{1.323055in}{0.684404in}}%
\pgfpathcurveto{\pgfqpoint{1.323055in}{0.673508in}}{\pgfqpoint{1.327384in}{0.663057in}}{\pgfqpoint{1.335089in}{0.655353in}}%
\pgfpathcurveto{\pgfqpoint{1.342793in}{0.647649in}}{\pgfqpoint{1.353244in}{0.643320in}}{\pgfqpoint{1.364139in}{0.643320in}}%
\pgfusepath{stroke}%
\end{pgfscope}%
\begin{pgfscope}%
\pgfpathrectangle{\pgfqpoint{0.688192in}{0.670138in}}{\pgfqpoint{7.111808in}{5.061530in}}%
\pgfusepath{clip}%
\pgfsetbuttcap%
\pgfsetroundjoin%
\pgfsetlinewidth{1.003750pt}%
\definecolor{currentstroke}{rgb}{0.000000,0.000000,0.000000}%
\pgfsetstrokecolor{currentstroke}%
\pgfsetdash{}{0pt}%
\pgfpathmoveto{\pgfqpoint{6.271456in}{4.564028in}}%
\pgfpathcurveto{\pgfqpoint{6.282351in}{4.564028in}}{\pgfqpoint{6.292802in}{4.568357in}}{\pgfqpoint{6.300506in}{4.576061in}}%
\pgfpathcurveto{\pgfqpoint{6.308211in}{4.583765in}}{\pgfqpoint{6.312539in}{4.594216in}}{\pgfqpoint{6.312539in}{4.605112in}}%
\pgfpathcurveto{\pgfqpoint{6.312539in}{4.616007in}}{\pgfqpoint{6.308211in}{4.626458in}}{\pgfqpoint{6.300506in}{4.634162in}}%
\pgfpathcurveto{\pgfqpoint{6.292802in}{4.641867in}}{\pgfqpoint{6.282351in}{4.646196in}}{\pgfqpoint{6.271456in}{4.646196in}}%
\pgfpathcurveto{\pgfqpoint{6.260560in}{4.646196in}}{\pgfqpoint{6.250109in}{4.641867in}}{\pgfqpoint{6.242405in}{4.634162in}}%
\pgfpathcurveto{\pgfqpoint{6.234700in}{4.626458in}}{\pgfqpoint{6.230372in}{4.616007in}}{\pgfqpoint{6.230372in}{4.605112in}}%
\pgfpathcurveto{\pgfqpoint{6.230372in}{4.594216in}}{\pgfqpoint{6.234700in}{4.583765in}}{\pgfqpoint{6.242405in}{4.576061in}}%
\pgfpathcurveto{\pgfqpoint{6.250109in}{4.568357in}}{\pgfqpoint{6.260560in}{4.564028in}}{\pgfqpoint{6.271456in}{4.564028in}}%
\pgfpathlineto{\pgfqpoint{6.271456in}{4.564028in}}%
\pgfpathclose%
\pgfusepath{stroke}%
\end{pgfscope}%
\begin{pgfscope}%
\pgfpathrectangle{\pgfqpoint{0.688192in}{0.670138in}}{\pgfqpoint{7.111808in}{5.061530in}}%
\pgfusepath{clip}%
\pgfsetbuttcap%
\pgfsetroundjoin%
\pgfsetlinewidth{1.003750pt}%
\definecolor{currentstroke}{rgb}{0.000000,0.000000,0.000000}%
\pgfsetstrokecolor{currentstroke}%
\pgfsetdash{}{0pt}%
\pgfpathmoveto{\pgfqpoint{5.527297in}{0.629518in}}%
\pgfpathcurveto{\pgfqpoint{5.538192in}{0.629518in}}{\pgfqpoint{5.548643in}{0.633846in}}{\pgfqpoint{5.556347in}{0.641551in}}%
\pgfpathcurveto{\pgfqpoint{5.564052in}{0.649255in}}{\pgfqpoint{5.568380in}{0.659706in}}{\pgfqpoint{5.568380in}{0.670601in}}%
\pgfpathcurveto{\pgfqpoint{5.568380in}{0.681497in}}{\pgfqpoint{5.564052in}{0.691948in}}{\pgfqpoint{5.556347in}{0.699652in}}%
\pgfpathcurveto{\pgfqpoint{5.548643in}{0.707356in}}{\pgfqpoint{5.538192in}{0.711685in}}{\pgfqpoint{5.527297in}{0.711685in}}%
\pgfpathcurveto{\pgfqpoint{5.516401in}{0.711685in}}{\pgfqpoint{5.505950in}{0.707356in}}{\pgfqpoint{5.498246in}{0.699652in}}%
\pgfpathcurveto{\pgfqpoint{5.490541in}{0.691948in}}{\pgfqpoint{5.486213in}{0.681497in}}{\pgfqpoint{5.486213in}{0.670601in}}%
\pgfpathcurveto{\pgfqpoint{5.486213in}{0.659706in}}{\pgfqpoint{5.490541in}{0.649255in}}{\pgfqpoint{5.498246in}{0.641551in}}%
\pgfpathcurveto{\pgfqpoint{5.505950in}{0.633846in}}{\pgfqpoint{5.516401in}{0.629518in}}{\pgfqpoint{5.527297in}{0.629518in}}%
\pgfusepath{stroke}%
\end{pgfscope}%
\begin{pgfscope}%
\pgfpathrectangle{\pgfqpoint{0.688192in}{0.670138in}}{\pgfqpoint{7.111808in}{5.061530in}}%
\pgfusepath{clip}%
\pgfsetbuttcap%
\pgfsetroundjoin%
\pgfsetlinewidth{1.003750pt}%
\definecolor{currentstroke}{rgb}{0.000000,0.000000,0.000000}%
\pgfsetstrokecolor{currentstroke}%
\pgfsetdash{}{0pt}%
\pgfpathmoveto{\pgfqpoint{1.147700in}{0.645369in}}%
\pgfpathcurveto{\pgfqpoint{1.158595in}{0.645369in}}{\pgfqpoint{1.169046in}{0.649697in}}{\pgfqpoint{1.176750in}{0.657402in}}%
\pgfpathcurveto{\pgfqpoint{1.184455in}{0.665106in}}{\pgfqpoint{1.188784in}{0.675557in}}{\pgfqpoint{1.188784in}{0.686452in}}%
\pgfpathcurveto{\pgfqpoint{1.188784in}{0.697348in}}{\pgfqpoint{1.184455in}{0.707799in}}{\pgfqpoint{1.176750in}{0.715503in}}%
\pgfpathcurveto{\pgfqpoint{1.169046in}{0.723207in}}{\pgfqpoint{1.158595in}{0.727536in}}{\pgfqpoint{1.147700in}{0.727536in}}%
\pgfpathcurveto{\pgfqpoint{1.136804in}{0.727536in}}{\pgfqpoint{1.126353in}{0.723207in}}{\pgfqpoint{1.118649in}{0.715503in}}%
\pgfpathcurveto{\pgfqpoint{1.110945in}{0.707799in}}{\pgfqpoint{1.106616in}{0.697348in}}{\pgfqpoint{1.106616in}{0.686452in}}%
\pgfpathcurveto{\pgfqpoint{1.106616in}{0.675557in}}{\pgfqpoint{1.110945in}{0.665106in}}{\pgfqpoint{1.118649in}{0.657402in}}%
\pgfpathcurveto{\pgfqpoint{1.126353in}{0.649697in}}{\pgfqpoint{1.136804in}{0.645369in}}{\pgfqpoint{1.147700in}{0.645369in}}%
\pgfusepath{stroke}%
\end{pgfscope}%
\begin{pgfscope}%
\pgfpathrectangle{\pgfqpoint{0.688192in}{0.670138in}}{\pgfqpoint{7.111808in}{5.061530in}}%
\pgfusepath{clip}%
\pgfsetbuttcap%
\pgfsetroundjoin%
\pgfsetlinewidth{1.003750pt}%
\definecolor{currentstroke}{rgb}{0.000000,0.000000,0.000000}%
\pgfsetstrokecolor{currentstroke}%
\pgfsetdash{}{0pt}%
\pgfpathmoveto{\pgfqpoint{1.103303in}{0.648067in}}%
\pgfpathcurveto{\pgfqpoint{1.114198in}{0.648067in}}{\pgfqpoint{1.124649in}{0.652396in}}{\pgfqpoint{1.132353in}{0.660101in}}%
\pgfpathcurveto{\pgfqpoint{1.140058in}{0.667805in}}{\pgfqpoint{1.144386in}{0.678256in}}{\pgfqpoint{1.144386in}{0.689151in}}%
\pgfpathcurveto{\pgfqpoint{1.144386in}{0.700047in}}{\pgfqpoint{1.140058in}{0.710498in}}{\pgfqpoint{1.132353in}{0.718202in}}%
\pgfpathcurveto{\pgfqpoint{1.124649in}{0.725906in}}{\pgfqpoint{1.114198in}{0.730235in}}{\pgfqpoint{1.103303in}{0.730235in}}%
\pgfpathcurveto{\pgfqpoint{1.092407in}{0.730235in}}{\pgfqpoint{1.081956in}{0.725906in}}{\pgfqpoint{1.074252in}{0.718202in}}%
\pgfpathcurveto{\pgfqpoint{1.066548in}{0.710498in}}{\pgfqpoint{1.062219in}{0.700047in}}{\pgfqpoint{1.062219in}{0.689151in}}%
\pgfpathcurveto{\pgfqpoint{1.062219in}{0.678256in}}{\pgfqpoint{1.066548in}{0.667805in}}{\pgfqpoint{1.074252in}{0.660101in}}%
\pgfpathcurveto{\pgfqpoint{1.081956in}{0.652396in}}{\pgfqpoint{1.092407in}{0.648067in}}{\pgfqpoint{1.103303in}{0.648067in}}%
\pgfusepath{stroke}%
\end{pgfscope}%
\begin{pgfscope}%
\pgfpathrectangle{\pgfqpoint{0.688192in}{0.670138in}}{\pgfqpoint{7.111808in}{5.061530in}}%
\pgfusepath{clip}%
\pgfsetbuttcap%
\pgfsetroundjoin%
\pgfsetlinewidth{1.003750pt}%
\definecolor{currentstroke}{rgb}{0.000000,0.000000,0.000000}%
\pgfsetstrokecolor{currentstroke}%
\pgfsetdash{}{0pt}%
\pgfpathmoveto{\pgfqpoint{0.834764in}{0.708414in}}%
\pgfpathcurveto{\pgfqpoint{0.845660in}{0.708414in}}{\pgfqpoint{0.856111in}{0.712742in}}{\pgfqpoint{0.863815in}{0.720447in}}%
\pgfpathcurveto{\pgfqpoint{0.871519in}{0.728151in}}{\pgfqpoint{0.875848in}{0.738602in}}{\pgfqpoint{0.875848in}{0.749497in}}%
\pgfpathcurveto{\pgfqpoint{0.875848in}{0.760393in}}{\pgfqpoint{0.871519in}{0.770844in}}{\pgfqpoint{0.863815in}{0.778548in}}%
\pgfpathcurveto{\pgfqpoint{0.856111in}{0.786252in}}{\pgfqpoint{0.845660in}{0.790581in}}{\pgfqpoint{0.834764in}{0.790581in}}%
\pgfpathcurveto{\pgfqpoint{0.823869in}{0.790581in}}{\pgfqpoint{0.813418in}{0.786252in}}{\pgfqpoint{0.805714in}{0.778548in}}%
\pgfpathcurveto{\pgfqpoint{0.798009in}{0.770844in}}{\pgfqpoint{0.793681in}{0.760393in}}{\pgfqpoint{0.793681in}{0.749497in}}%
\pgfpathcurveto{\pgfqpoint{0.793681in}{0.738602in}}{\pgfqpoint{0.798009in}{0.728151in}}{\pgfqpoint{0.805714in}{0.720447in}}%
\pgfpathcurveto{\pgfqpoint{0.813418in}{0.712742in}}{\pgfqpoint{0.823869in}{0.708414in}}{\pgfqpoint{0.834764in}{0.708414in}}%
\pgfpathlineto{\pgfqpoint{0.834764in}{0.708414in}}%
\pgfpathclose%
\pgfusepath{stroke}%
\end{pgfscope}%
\begin{pgfscope}%
\pgfpathrectangle{\pgfqpoint{0.688192in}{0.670138in}}{\pgfqpoint{7.111808in}{5.061530in}}%
\pgfusepath{clip}%
\pgfsetbuttcap%
\pgfsetroundjoin%
\pgfsetlinewidth{1.003750pt}%
\definecolor{currentstroke}{rgb}{0.000000,0.000000,0.000000}%
\pgfsetstrokecolor{currentstroke}%
\pgfsetdash{}{0pt}%
\pgfpathmoveto{\pgfqpoint{2.509817in}{0.637055in}}%
\pgfpathcurveto{\pgfqpoint{2.520713in}{0.637055in}}{\pgfqpoint{2.531164in}{0.641384in}}{\pgfqpoint{2.538868in}{0.649088in}}%
\pgfpathcurveto{\pgfqpoint{2.546572in}{0.656792in}}{\pgfqpoint{2.550901in}{0.667243in}}{\pgfqpoint{2.550901in}{0.678139in}}%
\pgfpathcurveto{\pgfqpoint{2.550901in}{0.689034in}}{\pgfqpoint{2.546572in}{0.699485in}}{\pgfqpoint{2.538868in}{0.707189in}}%
\pgfpathcurveto{\pgfqpoint{2.531164in}{0.714894in}}{\pgfqpoint{2.520713in}{0.719223in}}{\pgfqpoint{2.509817in}{0.719223in}}%
\pgfpathcurveto{\pgfqpoint{2.498922in}{0.719223in}}{\pgfqpoint{2.488471in}{0.714894in}}{\pgfqpoint{2.480767in}{0.707189in}}%
\pgfpathcurveto{\pgfqpoint{2.473062in}{0.699485in}}{\pgfqpoint{2.468733in}{0.689034in}}{\pgfqpoint{2.468733in}{0.678139in}}%
\pgfpathcurveto{\pgfqpoint{2.468733in}{0.667243in}}{\pgfqpoint{2.473062in}{0.656792in}}{\pgfqpoint{2.480767in}{0.649088in}}%
\pgfpathcurveto{\pgfqpoint{2.488471in}{0.641384in}}{\pgfqpoint{2.498922in}{0.637055in}}{\pgfqpoint{2.509817in}{0.637055in}}%
\pgfusepath{stroke}%
\end{pgfscope}%
\begin{pgfscope}%
\pgfpathrectangle{\pgfqpoint{0.688192in}{0.670138in}}{\pgfqpoint{7.111808in}{5.061530in}}%
\pgfusepath{clip}%
\pgfsetbuttcap%
\pgfsetroundjoin%
\pgfsetlinewidth{1.003750pt}%
\definecolor{currentstroke}{rgb}{0.000000,0.000000,0.000000}%
\pgfsetstrokecolor{currentstroke}%
\pgfsetdash{}{0pt}%
\pgfpathmoveto{\pgfqpoint{2.542698in}{0.636859in}}%
\pgfpathcurveto{\pgfqpoint{2.553593in}{0.636859in}}{\pgfqpoint{2.564044in}{0.641188in}}{\pgfqpoint{2.571749in}{0.648892in}}%
\pgfpathcurveto{\pgfqpoint{2.579453in}{0.656597in}}{\pgfqpoint{2.583782in}{0.667048in}}{\pgfqpoint{2.583782in}{0.677943in}}%
\pgfpathcurveto{\pgfqpoint{2.583782in}{0.688839in}}{\pgfqpoint{2.579453in}{0.699289in}}{\pgfqpoint{2.571749in}{0.706994in}}%
\pgfpathcurveto{\pgfqpoint{2.564044in}{0.714698in}}{\pgfqpoint{2.553593in}{0.719027in}}{\pgfqpoint{2.542698in}{0.719027in}}%
\pgfpathcurveto{\pgfqpoint{2.531802in}{0.719027in}}{\pgfqpoint{2.521352in}{0.714698in}}{\pgfqpoint{2.513647in}{0.706994in}}%
\pgfpathcurveto{\pgfqpoint{2.505943in}{0.699289in}}{\pgfqpoint{2.501614in}{0.688839in}}{\pgfqpoint{2.501614in}{0.677943in}}%
\pgfpathcurveto{\pgfqpoint{2.501614in}{0.667048in}}{\pgfqpoint{2.505943in}{0.656597in}}{\pgfqpoint{2.513647in}{0.648892in}}%
\pgfpathcurveto{\pgfqpoint{2.521352in}{0.641188in}}{\pgfqpoint{2.531802in}{0.636859in}}{\pgfqpoint{2.542698in}{0.636859in}}%
\pgfusepath{stroke}%
\end{pgfscope}%
\begin{pgfscope}%
\pgfpathrectangle{\pgfqpoint{0.688192in}{0.670138in}}{\pgfqpoint{7.111808in}{5.061530in}}%
\pgfusepath{clip}%
\pgfsetbuttcap%
\pgfsetroundjoin%
\pgfsetlinewidth{1.003750pt}%
\definecolor{currentstroke}{rgb}{0.000000,0.000000,0.000000}%
\pgfsetstrokecolor{currentstroke}%
\pgfsetdash{}{0pt}%
\pgfpathmoveto{\pgfqpoint{1.444241in}{0.642951in}}%
\pgfpathcurveto{\pgfqpoint{1.455136in}{0.642951in}}{\pgfqpoint{1.465587in}{0.647280in}}{\pgfqpoint{1.473291in}{0.654984in}}%
\pgfpathcurveto{\pgfqpoint{1.480996in}{0.662688in}}{\pgfqpoint{1.485325in}{0.673139in}}{\pgfqpoint{1.485325in}{0.684035in}}%
\pgfpathcurveto{\pgfqpoint{1.485325in}{0.694930in}}{\pgfqpoint{1.480996in}{0.705381in}}{\pgfqpoint{1.473291in}{0.713085in}}%
\pgfpathcurveto{\pgfqpoint{1.465587in}{0.720790in}}{\pgfqpoint{1.455136in}{0.725119in}}{\pgfqpoint{1.444241in}{0.725119in}}%
\pgfpathcurveto{\pgfqpoint{1.433345in}{0.725119in}}{\pgfqpoint{1.422894in}{0.720790in}}{\pgfqpoint{1.415190in}{0.713085in}}%
\pgfpathcurveto{\pgfqpoint{1.407486in}{0.705381in}}{\pgfqpoint{1.403157in}{0.694930in}}{\pgfqpoint{1.403157in}{0.684035in}}%
\pgfpathcurveto{\pgfqpoint{1.403157in}{0.673139in}}{\pgfqpoint{1.407486in}{0.662688in}}{\pgfqpoint{1.415190in}{0.654984in}}%
\pgfpathcurveto{\pgfqpoint{1.422894in}{0.647280in}}{\pgfqpoint{1.433345in}{0.642951in}}{\pgfqpoint{1.444241in}{0.642951in}}%
\pgfusepath{stroke}%
\end{pgfscope}%
\begin{pgfscope}%
\pgfpathrectangle{\pgfqpoint{0.688192in}{0.670138in}}{\pgfqpoint{7.111808in}{5.061530in}}%
\pgfusepath{clip}%
\pgfsetbuttcap%
\pgfsetroundjoin%
\pgfsetlinewidth{1.003750pt}%
\definecolor{currentstroke}{rgb}{0.000000,0.000000,0.000000}%
\pgfsetstrokecolor{currentstroke}%
\pgfsetdash{}{0pt}%
\pgfpathmoveto{\pgfqpoint{1.469414in}{0.642878in}}%
\pgfpathcurveto{\pgfqpoint{1.480310in}{0.642878in}}{\pgfqpoint{1.490760in}{0.647207in}}{\pgfqpoint{1.498465in}{0.654911in}}%
\pgfpathcurveto{\pgfqpoint{1.506169in}{0.662615in}}{\pgfqpoint{1.510498in}{0.673066in}}{\pgfqpoint{1.510498in}{0.683962in}}%
\pgfpathcurveto{\pgfqpoint{1.510498in}{0.694857in}}{\pgfqpoint{1.506169in}{0.705308in}}{\pgfqpoint{1.498465in}{0.713012in}}%
\pgfpathcurveto{\pgfqpoint{1.490760in}{0.720717in}}{\pgfqpoint{1.480310in}{0.725046in}}{\pgfqpoint{1.469414in}{0.725046in}}%
\pgfpathcurveto{\pgfqpoint{1.458519in}{0.725046in}}{\pgfqpoint{1.448068in}{0.720717in}}{\pgfqpoint{1.440363in}{0.713012in}}%
\pgfpathcurveto{\pgfqpoint{1.432659in}{0.705308in}}{\pgfqpoint{1.428330in}{0.694857in}}{\pgfqpoint{1.428330in}{0.683962in}}%
\pgfpathcurveto{\pgfqpoint{1.428330in}{0.673066in}}{\pgfqpoint{1.432659in}{0.662615in}}{\pgfqpoint{1.440363in}{0.654911in}}%
\pgfpathcurveto{\pgfqpoint{1.448068in}{0.647207in}}{\pgfqpoint{1.458519in}{0.642878in}}{\pgfqpoint{1.469414in}{0.642878in}}%
\pgfusepath{stroke}%
\end{pgfscope}%
\begin{pgfscope}%
\pgfpathrectangle{\pgfqpoint{0.688192in}{0.670138in}}{\pgfqpoint{7.111808in}{5.061530in}}%
\pgfusepath{clip}%
\pgfsetbuttcap%
\pgfsetroundjoin%
\pgfsetlinewidth{1.003750pt}%
\definecolor{currentstroke}{rgb}{0.000000,0.000000,0.000000}%
\pgfsetstrokecolor{currentstroke}%
\pgfsetdash{}{0pt}%
\pgfpathmoveto{\pgfqpoint{4.447883in}{3.462388in}}%
\pgfpathcurveto{\pgfqpoint{4.458778in}{3.462388in}}{\pgfqpoint{4.469229in}{3.466717in}}{\pgfqpoint{4.476934in}{3.474421in}}%
\pgfpathcurveto{\pgfqpoint{4.484638in}{3.482125in}}{\pgfqpoint{4.488967in}{3.492576in}}{\pgfqpoint{4.488967in}{3.503472in}}%
\pgfpathcurveto{\pgfqpoint{4.488967in}{3.514367in}}{\pgfqpoint{4.484638in}{3.524818in}}{\pgfqpoint{4.476934in}{3.532522in}}%
\pgfpathcurveto{\pgfqpoint{4.469229in}{3.540227in}}{\pgfqpoint{4.458778in}{3.544556in}}{\pgfqpoint{4.447883in}{3.544556in}}%
\pgfpathcurveto{\pgfqpoint{4.436987in}{3.544556in}}{\pgfqpoint{4.426536in}{3.540227in}}{\pgfqpoint{4.418832in}{3.532522in}}%
\pgfpathcurveto{\pgfqpoint{4.411128in}{3.524818in}}{\pgfqpoint{4.406799in}{3.514367in}}{\pgfqpoint{4.406799in}{3.503472in}}%
\pgfpathcurveto{\pgfqpoint{4.406799in}{3.492576in}}{\pgfqpoint{4.411128in}{3.482125in}}{\pgfqpoint{4.418832in}{3.474421in}}%
\pgfpathcurveto{\pgfqpoint{4.426536in}{3.466717in}}{\pgfqpoint{4.436987in}{3.462388in}}{\pgfqpoint{4.447883in}{3.462388in}}%
\pgfpathlineto{\pgfqpoint{4.447883in}{3.462388in}}%
\pgfpathclose%
\pgfusepath{stroke}%
\end{pgfscope}%
\begin{pgfscope}%
\pgfpathrectangle{\pgfqpoint{0.688192in}{0.670138in}}{\pgfqpoint{7.111808in}{5.061530in}}%
\pgfusepath{clip}%
\pgfsetbuttcap%
\pgfsetroundjoin%
\pgfsetlinewidth{1.003750pt}%
\definecolor{currentstroke}{rgb}{0.000000,0.000000,0.000000}%
\pgfsetstrokecolor{currentstroke}%
\pgfsetdash{}{0pt}%
\pgfpathmoveto{\pgfqpoint{6.268478in}{0.731839in}}%
\pgfpathcurveto{\pgfqpoint{6.279374in}{0.731839in}}{\pgfqpoint{6.289825in}{0.736167in}}{\pgfqpoint{6.297529in}{0.743872in}}%
\pgfpathcurveto{\pgfqpoint{6.305233in}{0.751576in}}{\pgfqpoint{6.309562in}{0.762027in}}{\pgfqpoint{6.309562in}{0.772922in}}%
\pgfpathcurveto{\pgfqpoint{6.309562in}{0.783818in}}{\pgfqpoint{6.305233in}{0.794269in}}{\pgfqpoint{6.297529in}{0.801973in}}%
\pgfpathcurveto{\pgfqpoint{6.289825in}{0.809677in}}{\pgfqpoint{6.279374in}{0.814006in}}{\pgfqpoint{6.268478in}{0.814006in}}%
\pgfpathcurveto{\pgfqpoint{6.257583in}{0.814006in}}{\pgfqpoint{6.247132in}{0.809677in}}{\pgfqpoint{6.239428in}{0.801973in}}%
\pgfpathcurveto{\pgfqpoint{6.231723in}{0.794269in}}{\pgfqpoint{6.227394in}{0.783818in}}{\pgfqpoint{6.227394in}{0.772922in}}%
\pgfpathcurveto{\pgfqpoint{6.227394in}{0.762027in}}{\pgfqpoint{6.231723in}{0.751576in}}{\pgfqpoint{6.239428in}{0.743872in}}%
\pgfpathcurveto{\pgfqpoint{6.247132in}{0.736167in}}{\pgfqpoint{6.257583in}{0.731839in}}{\pgfqpoint{6.268478in}{0.731839in}}%
\pgfpathlineto{\pgfqpoint{6.268478in}{0.731839in}}%
\pgfpathclose%
\pgfusepath{stroke}%
\end{pgfscope}%
\begin{pgfscope}%
\pgfpathrectangle{\pgfqpoint{0.688192in}{0.670138in}}{\pgfqpoint{7.111808in}{5.061530in}}%
\pgfusepath{clip}%
\pgfsetbuttcap%
\pgfsetroundjoin%
\pgfsetlinewidth{1.003750pt}%
\definecolor{currentstroke}{rgb}{0.000000,0.000000,0.000000}%
\pgfsetstrokecolor{currentstroke}%
\pgfsetdash{}{0pt}%
\pgfpathmoveto{\pgfqpoint{1.102529in}{0.648268in}}%
\pgfpathcurveto{\pgfqpoint{1.113424in}{0.648268in}}{\pgfqpoint{1.123875in}{0.652597in}}{\pgfqpoint{1.131579in}{0.660301in}}%
\pgfpathcurveto{\pgfqpoint{1.139284in}{0.668005in}}{\pgfqpoint{1.143613in}{0.678456in}}{\pgfqpoint{1.143613in}{0.689352in}}%
\pgfpathcurveto{\pgfqpoint{1.143613in}{0.700247in}}{\pgfqpoint{1.139284in}{0.710698in}}{\pgfqpoint{1.131579in}{0.718402in}}%
\pgfpathcurveto{\pgfqpoint{1.123875in}{0.726107in}}{\pgfqpoint{1.113424in}{0.730436in}}{\pgfqpoint{1.102529in}{0.730436in}}%
\pgfpathcurveto{\pgfqpoint{1.091633in}{0.730436in}}{\pgfqpoint{1.081182in}{0.726107in}}{\pgfqpoint{1.073478in}{0.718402in}}%
\pgfpathcurveto{\pgfqpoint{1.065774in}{0.710698in}}{\pgfqpoint{1.061445in}{0.700247in}}{\pgfqpoint{1.061445in}{0.689352in}}%
\pgfpathcurveto{\pgfqpoint{1.061445in}{0.678456in}}{\pgfqpoint{1.065774in}{0.668005in}}{\pgfqpoint{1.073478in}{0.660301in}}%
\pgfpathcurveto{\pgfqpoint{1.081182in}{0.652597in}}{\pgfqpoint{1.091633in}{0.648268in}}{\pgfqpoint{1.102529in}{0.648268in}}%
\pgfusepath{stroke}%
\end{pgfscope}%
\begin{pgfscope}%
\pgfpathrectangle{\pgfqpoint{0.688192in}{0.670138in}}{\pgfqpoint{7.111808in}{5.061530in}}%
\pgfusepath{clip}%
\pgfsetbuttcap%
\pgfsetroundjoin%
\pgfsetlinewidth{1.003750pt}%
\definecolor{currentstroke}{rgb}{0.000000,0.000000,0.000000}%
\pgfsetstrokecolor{currentstroke}%
\pgfsetdash{}{0pt}%
\pgfpathmoveto{\pgfqpoint{2.926347in}{3.587432in}}%
\pgfpathcurveto{\pgfqpoint{2.937243in}{3.587432in}}{\pgfqpoint{2.947694in}{3.591760in}}{\pgfqpoint{2.955398in}{3.599465in}}%
\pgfpathcurveto{\pgfqpoint{2.963102in}{3.607169in}}{\pgfqpoint{2.967431in}{3.617620in}}{\pgfqpoint{2.967431in}{3.628515in}}%
\pgfpathcurveto{\pgfqpoint{2.967431in}{3.639411in}}{\pgfqpoint{2.963102in}{3.649862in}}{\pgfqpoint{2.955398in}{3.657566in}}%
\pgfpathcurveto{\pgfqpoint{2.947694in}{3.665270in}}{\pgfqpoint{2.937243in}{3.669599in}}{\pgfqpoint{2.926347in}{3.669599in}}%
\pgfpathcurveto{\pgfqpoint{2.915452in}{3.669599in}}{\pgfqpoint{2.905001in}{3.665270in}}{\pgfqpoint{2.897297in}{3.657566in}}%
\pgfpathcurveto{\pgfqpoint{2.889592in}{3.649862in}}{\pgfqpoint{2.885263in}{3.639411in}}{\pgfqpoint{2.885263in}{3.628515in}}%
\pgfpathcurveto{\pgfqpoint{2.885263in}{3.617620in}}{\pgfqpoint{2.889592in}{3.607169in}}{\pgfqpoint{2.897297in}{3.599465in}}%
\pgfpathcurveto{\pgfqpoint{2.905001in}{3.591760in}}{\pgfqpoint{2.915452in}{3.587432in}}{\pgfqpoint{2.926347in}{3.587432in}}%
\pgfpathlineto{\pgfqpoint{2.926347in}{3.587432in}}%
\pgfpathclose%
\pgfusepath{stroke}%
\end{pgfscope}%
\begin{pgfscope}%
\pgfpathrectangle{\pgfqpoint{0.688192in}{0.670138in}}{\pgfqpoint{7.111808in}{5.061530in}}%
\pgfusepath{clip}%
\pgfsetbuttcap%
\pgfsetroundjoin%
\pgfsetlinewidth{1.003750pt}%
\definecolor{currentstroke}{rgb}{0.000000,0.000000,0.000000}%
\pgfsetstrokecolor{currentstroke}%
\pgfsetdash{}{0pt}%
\pgfpathmoveto{\pgfqpoint{1.663205in}{0.641981in}}%
\pgfpathcurveto{\pgfqpoint{1.674101in}{0.641981in}}{\pgfqpoint{1.684552in}{0.646310in}}{\pgfqpoint{1.692256in}{0.654014in}}%
\pgfpathcurveto{\pgfqpoint{1.699960in}{0.661719in}}{\pgfqpoint{1.704289in}{0.672169in}}{\pgfqpoint{1.704289in}{0.683065in}}%
\pgfpathcurveto{\pgfqpoint{1.704289in}{0.693961in}}{\pgfqpoint{1.699960in}{0.704411in}}{\pgfqpoint{1.692256in}{0.712116in}}%
\pgfpathcurveto{\pgfqpoint{1.684552in}{0.719820in}}{\pgfqpoint{1.674101in}{0.724149in}}{\pgfqpoint{1.663205in}{0.724149in}}%
\pgfpathcurveto{\pgfqpoint{1.652310in}{0.724149in}}{\pgfqpoint{1.641859in}{0.719820in}}{\pgfqpoint{1.634155in}{0.712116in}}%
\pgfpathcurveto{\pgfqpoint{1.626450in}{0.704411in}}{\pgfqpoint{1.622121in}{0.693961in}}{\pgfqpoint{1.622121in}{0.683065in}}%
\pgfpathcurveto{\pgfqpoint{1.622121in}{0.672169in}}{\pgfqpoint{1.626450in}{0.661719in}}{\pgfqpoint{1.634155in}{0.654014in}}%
\pgfpathcurveto{\pgfqpoint{1.641859in}{0.646310in}}{\pgfqpoint{1.652310in}{0.641981in}}{\pgfqpoint{1.663205in}{0.641981in}}%
\pgfusepath{stroke}%
\end{pgfscope}%
\begin{pgfscope}%
\pgfpathrectangle{\pgfqpoint{0.688192in}{0.670138in}}{\pgfqpoint{7.111808in}{5.061530in}}%
\pgfusepath{clip}%
\pgfsetbuttcap%
\pgfsetroundjoin%
\pgfsetlinewidth{1.003750pt}%
\definecolor{currentstroke}{rgb}{0.000000,0.000000,0.000000}%
\pgfsetstrokecolor{currentstroke}%
\pgfsetdash{}{0pt}%
\pgfpathmoveto{\pgfqpoint{0.973315in}{1.722109in}}%
\pgfpathcurveto{\pgfqpoint{0.984211in}{1.722109in}}{\pgfqpoint{0.994662in}{1.726438in}}{\pgfqpoint{1.002366in}{1.734142in}}%
\pgfpathcurveto{\pgfqpoint{1.010070in}{1.741846in}}{\pgfqpoint{1.014399in}{1.752297in}}{\pgfqpoint{1.014399in}{1.763193in}}%
\pgfpathcurveto{\pgfqpoint{1.014399in}{1.774088in}}{\pgfqpoint{1.010070in}{1.784539in}}{\pgfqpoint{1.002366in}{1.792243in}}%
\pgfpathcurveto{\pgfqpoint{0.994662in}{1.799948in}}{\pgfqpoint{0.984211in}{1.804277in}}{\pgfqpoint{0.973315in}{1.804277in}}%
\pgfpathcurveto{\pgfqpoint{0.962420in}{1.804277in}}{\pgfqpoint{0.951969in}{1.799948in}}{\pgfqpoint{0.944264in}{1.792243in}}%
\pgfpathcurveto{\pgfqpoint{0.936560in}{1.784539in}}{\pgfqpoint{0.932231in}{1.774088in}}{\pgfqpoint{0.932231in}{1.763193in}}%
\pgfpathcurveto{\pgfqpoint{0.932231in}{1.752297in}}{\pgfqpoint{0.936560in}{1.741846in}}{\pgfqpoint{0.944264in}{1.734142in}}%
\pgfpathcurveto{\pgfqpoint{0.951969in}{1.726438in}}{\pgfqpoint{0.962420in}{1.722109in}}{\pgfqpoint{0.973315in}{1.722109in}}%
\pgfpathlineto{\pgfqpoint{0.973315in}{1.722109in}}%
\pgfpathclose%
\pgfusepath{stroke}%
\end{pgfscope}%
\begin{pgfscope}%
\pgfpathrectangle{\pgfqpoint{0.688192in}{0.670138in}}{\pgfqpoint{7.111808in}{5.061530in}}%
\pgfusepath{clip}%
\pgfsetbuttcap%
\pgfsetroundjoin%
\pgfsetlinewidth{1.003750pt}%
\definecolor{currentstroke}{rgb}{0.000000,0.000000,0.000000}%
\pgfsetstrokecolor{currentstroke}%
\pgfsetdash{}{0pt}%
\pgfpathmoveto{\pgfqpoint{1.178009in}{0.644363in}}%
\pgfpathcurveto{\pgfqpoint{1.188905in}{0.644363in}}{\pgfqpoint{1.199355in}{0.648692in}}{\pgfqpoint{1.207060in}{0.656396in}}%
\pgfpathcurveto{\pgfqpoint{1.214764in}{0.664100in}}{\pgfqpoint{1.219093in}{0.674551in}}{\pgfqpoint{1.219093in}{0.685447in}}%
\pgfpathcurveto{\pgfqpoint{1.219093in}{0.696342in}}{\pgfqpoint{1.214764in}{0.706793in}}{\pgfqpoint{1.207060in}{0.714497in}}%
\pgfpathcurveto{\pgfqpoint{1.199355in}{0.722202in}}{\pgfqpoint{1.188905in}{0.726531in}}{\pgfqpoint{1.178009in}{0.726531in}}%
\pgfpathcurveto{\pgfqpoint{1.167113in}{0.726531in}}{\pgfqpoint{1.156663in}{0.722202in}}{\pgfqpoint{1.148958in}{0.714497in}}%
\pgfpathcurveto{\pgfqpoint{1.141254in}{0.706793in}}{\pgfqpoint{1.136925in}{0.696342in}}{\pgfqpoint{1.136925in}{0.685447in}}%
\pgfpathcurveto{\pgfqpoint{1.136925in}{0.674551in}}{\pgfqpoint{1.141254in}{0.664100in}}{\pgfqpoint{1.148958in}{0.656396in}}%
\pgfpathcurveto{\pgfqpoint{1.156663in}{0.648692in}}{\pgfqpoint{1.167113in}{0.644363in}}{\pgfqpoint{1.178009in}{0.644363in}}%
\pgfusepath{stroke}%
\end{pgfscope}%
\begin{pgfscope}%
\pgfpathrectangle{\pgfqpoint{0.688192in}{0.670138in}}{\pgfqpoint{7.111808in}{5.061530in}}%
\pgfusepath{clip}%
\pgfsetbuttcap%
\pgfsetroundjoin%
\pgfsetlinewidth{1.003750pt}%
\definecolor{currentstroke}{rgb}{0.000000,0.000000,0.000000}%
\pgfsetstrokecolor{currentstroke}%
\pgfsetdash{}{0pt}%
\pgfpathmoveto{\pgfqpoint{5.527297in}{0.629518in}}%
\pgfpathcurveto{\pgfqpoint{5.538192in}{0.629518in}}{\pgfqpoint{5.548643in}{0.633846in}}{\pgfqpoint{5.556347in}{0.641551in}}%
\pgfpathcurveto{\pgfqpoint{5.564052in}{0.649255in}}{\pgfqpoint{5.568380in}{0.659706in}}{\pgfqpoint{5.568380in}{0.670601in}}%
\pgfpathcurveto{\pgfqpoint{5.568380in}{0.681497in}}{\pgfqpoint{5.564052in}{0.691948in}}{\pgfqpoint{5.556347in}{0.699652in}}%
\pgfpathcurveto{\pgfqpoint{5.548643in}{0.707356in}}{\pgfqpoint{5.538192in}{0.711685in}}{\pgfqpoint{5.527297in}{0.711685in}}%
\pgfpathcurveto{\pgfqpoint{5.516401in}{0.711685in}}{\pgfqpoint{5.505950in}{0.707356in}}{\pgfqpoint{5.498246in}{0.699652in}}%
\pgfpathcurveto{\pgfqpoint{5.490541in}{0.691948in}}{\pgfqpoint{5.486213in}{0.681497in}}{\pgfqpoint{5.486213in}{0.670601in}}%
\pgfpathcurveto{\pgfqpoint{5.486213in}{0.659706in}}{\pgfqpoint{5.490541in}{0.649255in}}{\pgfqpoint{5.498246in}{0.641551in}}%
\pgfpathcurveto{\pgfqpoint{5.505950in}{0.633846in}}{\pgfqpoint{5.516401in}{0.629518in}}{\pgfqpoint{5.527297in}{0.629518in}}%
\pgfusepath{stroke}%
\end{pgfscope}%
\begin{pgfscope}%
\pgfpathrectangle{\pgfqpoint{0.688192in}{0.670138in}}{\pgfqpoint{7.111808in}{5.061530in}}%
\pgfusepath{clip}%
\pgfsetbuttcap%
\pgfsetroundjoin%
\pgfsetlinewidth{1.003750pt}%
\definecolor{currentstroke}{rgb}{0.000000,0.000000,0.000000}%
\pgfsetstrokecolor{currentstroke}%
\pgfsetdash{}{0pt}%
\pgfpathmoveto{\pgfqpoint{2.452193in}{3.677100in}}%
\pgfpathcurveto{\pgfqpoint{2.463088in}{3.677100in}}{\pgfqpoint{2.473539in}{3.681429in}}{\pgfqpoint{2.481243in}{3.689133in}}%
\pgfpathcurveto{\pgfqpoint{2.488948in}{3.696837in}}{\pgfqpoint{2.493276in}{3.707288in}}{\pgfqpoint{2.493276in}{3.718184in}}%
\pgfpathcurveto{\pgfqpoint{2.493276in}{3.729079in}}{\pgfqpoint{2.488948in}{3.739530in}}{\pgfqpoint{2.481243in}{3.747234in}}%
\pgfpathcurveto{\pgfqpoint{2.473539in}{3.754939in}}{\pgfqpoint{2.463088in}{3.759268in}}{\pgfqpoint{2.452193in}{3.759268in}}%
\pgfpathcurveto{\pgfqpoint{2.441297in}{3.759268in}}{\pgfqpoint{2.430846in}{3.754939in}}{\pgfqpoint{2.423142in}{3.747234in}}%
\pgfpathcurveto{\pgfqpoint{2.415438in}{3.739530in}}{\pgfqpoint{2.411109in}{3.729079in}}{\pgfqpoint{2.411109in}{3.718184in}}%
\pgfpathcurveto{\pgfqpoint{2.411109in}{3.707288in}}{\pgfqpoint{2.415438in}{3.696837in}}{\pgfqpoint{2.423142in}{3.689133in}}%
\pgfpathcurveto{\pgfqpoint{2.430846in}{3.681429in}}{\pgfqpoint{2.441297in}{3.677100in}}{\pgfqpoint{2.452193in}{3.677100in}}%
\pgfpathlineto{\pgfqpoint{2.452193in}{3.677100in}}%
\pgfpathclose%
\pgfusepath{stroke}%
\end{pgfscope}%
\begin{pgfscope}%
\pgfpathrectangle{\pgfqpoint{0.688192in}{0.670138in}}{\pgfqpoint{7.111808in}{5.061530in}}%
\pgfusepath{clip}%
\pgfsetbuttcap%
\pgfsetroundjoin%
\pgfsetlinewidth{1.003750pt}%
\definecolor{currentstroke}{rgb}{0.000000,0.000000,0.000000}%
\pgfsetstrokecolor{currentstroke}%
\pgfsetdash{}{0pt}%
\pgfpathmoveto{\pgfqpoint{0.939729in}{0.677113in}}%
\pgfpathcurveto{\pgfqpoint{0.950625in}{0.677113in}}{\pgfqpoint{0.961075in}{0.681442in}}{\pgfqpoint{0.968780in}{0.689146in}}%
\pgfpathcurveto{\pgfqpoint{0.976484in}{0.696850in}}{\pgfqpoint{0.980813in}{0.707301in}}{\pgfqpoint{0.980813in}{0.718197in}}%
\pgfpathcurveto{\pgfqpoint{0.980813in}{0.729092in}}{\pgfqpoint{0.976484in}{0.739543in}}{\pgfqpoint{0.968780in}{0.747247in}}%
\pgfpathcurveto{\pgfqpoint{0.961075in}{0.754952in}}{\pgfqpoint{0.950625in}{0.759281in}}{\pgfqpoint{0.939729in}{0.759281in}}%
\pgfpathcurveto{\pgfqpoint{0.928834in}{0.759281in}}{\pgfqpoint{0.918383in}{0.754952in}}{\pgfqpoint{0.910678in}{0.747247in}}%
\pgfpathcurveto{\pgfqpoint{0.902974in}{0.739543in}}{\pgfqpoint{0.898645in}{0.729092in}}{\pgfqpoint{0.898645in}{0.718197in}}%
\pgfpathcurveto{\pgfqpoint{0.898645in}{0.707301in}}{\pgfqpoint{0.902974in}{0.696850in}}{\pgfqpoint{0.910678in}{0.689146in}}%
\pgfpathcurveto{\pgfqpoint{0.918383in}{0.681442in}}{\pgfqpoint{0.928834in}{0.677113in}}{\pgfqpoint{0.939729in}{0.677113in}}%
\pgfpathlineto{\pgfqpoint{0.939729in}{0.677113in}}%
\pgfpathclose%
\pgfusepath{stroke}%
\end{pgfscope}%
\begin{pgfscope}%
\pgfpathrectangle{\pgfqpoint{0.688192in}{0.670138in}}{\pgfqpoint{7.111808in}{5.061530in}}%
\pgfusepath{clip}%
\pgfsetbuttcap%
\pgfsetroundjoin%
\pgfsetlinewidth{1.003750pt}%
\definecolor{currentstroke}{rgb}{0.000000,0.000000,0.000000}%
\pgfsetstrokecolor{currentstroke}%
\pgfsetdash{}{0pt}%
\pgfpathmoveto{\pgfqpoint{7.818522in}{3.422332in}}%
\pgfpathcurveto{\pgfqpoint{7.829418in}{3.422332in}}{\pgfqpoint{7.839869in}{3.426661in}}{\pgfqpoint{7.847573in}{3.434365in}}%
\pgfpathcurveto{\pgfqpoint{7.855277in}{3.442069in}}{\pgfqpoint{7.859606in}{3.452520in}}{\pgfqpoint{7.859606in}{3.463416in}}%
\pgfpathcurveto{\pgfqpoint{7.859606in}{3.474311in}}{\pgfqpoint{7.855277in}{3.484762in}}{\pgfqpoint{7.847573in}{3.492466in}}%
\pgfpathcurveto{\pgfqpoint{7.839869in}{3.500171in}}{\pgfqpoint{7.829418in}{3.504499in}}{\pgfqpoint{7.818522in}{3.504499in}}%
\pgfpathcurveto{\pgfqpoint{7.807627in}{3.504499in}}{\pgfqpoint{7.797176in}{3.500171in}}{\pgfqpoint{7.789472in}{3.492466in}}%
\pgfpathcurveto{\pgfqpoint{7.781767in}{3.484762in}}{\pgfqpoint{7.777438in}{3.474311in}}{\pgfqpoint{7.777438in}{3.463416in}}%
\pgfpathcurveto{\pgfqpoint{7.777438in}{3.452520in}}{\pgfqpoint{7.781767in}{3.442069in}}{\pgfqpoint{7.789472in}{3.434365in}}%
\pgfpathcurveto{\pgfqpoint{7.797176in}{3.426661in}}{\pgfqpoint{7.807627in}{3.422332in}}{\pgfqpoint{7.818522in}{3.422332in}}%
\pgfusepath{stroke}%
\end{pgfscope}%
\begin{pgfscope}%
\pgfpathrectangle{\pgfqpoint{0.688192in}{0.670138in}}{\pgfqpoint{7.111808in}{5.061530in}}%
\pgfusepath{clip}%
\pgfsetbuttcap%
\pgfsetroundjoin%
\pgfsetlinewidth{1.003750pt}%
\definecolor{currentstroke}{rgb}{0.000000,0.000000,0.000000}%
\pgfsetstrokecolor{currentstroke}%
\pgfsetdash{}{0pt}%
\pgfpathmoveto{\pgfqpoint{0.793302in}{0.768515in}}%
\pgfpathcurveto{\pgfqpoint{0.804198in}{0.768515in}}{\pgfqpoint{0.814648in}{0.772844in}}{\pgfqpoint{0.822353in}{0.780548in}}%
\pgfpathcurveto{\pgfqpoint{0.830057in}{0.788253in}}{\pgfqpoint{0.834386in}{0.798703in}}{\pgfqpoint{0.834386in}{0.809599in}}%
\pgfpathcurveto{\pgfqpoint{0.834386in}{0.820494in}}{\pgfqpoint{0.830057in}{0.830945in}}{\pgfqpoint{0.822353in}{0.838650in}}%
\pgfpathcurveto{\pgfqpoint{0.814648in}{0.846354in}}{\pgfqpoint{0.804198in}{0.850683in}}{\pgfqpoint{0.793302in}{0.850683in}}%
\pgfpathcurveto{\pgfqpoint{0.782407in}{0.850683in}}{\pgfqpoint{0.771956in}{0.846354in}}{\pgfqpoint{0.764251in}{0.838650in}}%
\pgfpathcurveto{\pgfqpoint{0.756547in}{0.830945in}}{\pgfqpoint{0.752218in}{0.820494in}}{\pgfqpoint{0.752218in}{0.809599in}}%
\pgfpathcurveto{\pgfqpoint{0.752218in}{0.798703in}}{\pgfqpoint{0.756547in}{0.788253in}}{\pgfqpoint{0.764251in}{0.780548in}}%
\pgfpathcurveto{\pgfqpoint{0.771956in}{0.772844in}}{\pgfqpoint{0.782407in}{0.768515in}}{\pgfqpoint{0.793302in}{0.768515in}}%
\pgfpathlineto{\pgfqpoint{0.793302in}{0.768515in}}%
\pgfpathclose%
\pgfusepath{stroke}%
\end{pgfscope}%
\begin{pgfscope}%
\pgfpathrectangle{\pgfqpoint{0.688192in}{0.670138in}}{\pgfqpoint{7.111808in}{5.061530in}}%
\pgfusepath{clip}%
\pgfsetbuttcap%
\pgfsetroundjoin%
\pgfsetlinewidth{1.003750pt}%
\definecolor{currentstroke}{rgb}{0.000000,0.000000,0.000000}%
\pgfsetstrokecolor{currentstroke}%
\pgfsetdash{}{0pt}%
\pgfpathmoveto{\pgfqpoint{5.137647in}{4.419439in}}%
\pgfpathcurveto{\pgfqpoint{5.148543in}{4.419439in}}{\pgfqpoint{5.158994in}{4.423768in}}{\pgfqpoint{5.166698in}{4.431473in}}%
\pgfpathcurveto{\pgfqpoint{5.174402in}{4.439177in}}{\pgfqpoint{5.178731in}{4.449628in}}{\pgfqpoint{5.178731in}{4.460523in}}%
\pgfpathcurveto{\pgfqpoint{5.178731in}{4.471419in}}{\pgfqpoint{5.174402in}{4.481870in}}{\pgfqpoint{5.166698in}{4.489574in}}%
\pgfpathcurveto{\pgfqpoint{5.158994in}{4.497278in}}{\pgfqpoint{5.148543in}{4.501607in}}{\pgfqpoint{5.137647in}{4.501607in}}%
\pgfpathcurveto{\pgfqpoint{5.126752in}{4.501607in}}{\pgfqpoint{5.116301in}{4.497278in}}{\pgfqpoint{5.108597in}{4.489574in}}%
\pgfpathcurveto{\pgfqpoint{5.100892in}{4.481870in}}{\pgfqpoint{5.096563in}{4.471419in}}{\pgfqpoint{5.096563in}{4.460523in}}%
\pgfpathcurveto{\pgfqpoint{5.096563in}{4.449628in}}{\pgfqpoint{5.100892in}{4.439177in}}{\pgfqpoint{5.108597in}{4.431473in}}%
\pgfpathcurveto{\pgfqpoint{5.116301in}{4.423768in}}{\pgfqpoint{5.126752in}{4.419439in}}{\pgfqpoint{5.137647in}{4.419439in}}%
\pgfpathlineto{\pgfqpoint{5.137647in}{4.419439in}}%
\pgfpathclose%
\pgfusepath{stroke}%
\end{pgfscope}%
\begin{pgfscope}%
\pgfpathrectangle{\pgfqpoint{0.688192in}{0.670138in}}{\pgfqpoint{7.111808in}{5.061530in}}%
\pgfusepath{clip}%
\pgfsetbuttcap%
\pgfsetroundjoin%
\pgfsetlinewidth{1.003750pt}%
\definecolor{currentstroke}{rgb}{0.000000,0.000000,0.000000}%
\pgfsetstrokecolor{currentstroke}%
\pgfsetdash{}{0pt}%
\pgfpathmoveto{\pgfqpoint{1.719213in}{0.641640in}}%
\pgfpathcurveto{\pgfqpoint{1.730109in}{0.641640in}}{\pgfqpoint{1.740559in}{0.645969in}}{\pgfqpoint{1.748264in}{0.653673in}}%
\pgfpathcurveto{\pgfqpoint{1.755968in}{0.661377in}}{\pgfqpoint{1.760297in}{0.671828in}}{\pgfqpoint{1.760297in}{0.682724in}}%
\pgfpathcurveto{\pgfqpoint{1.760297in}{0.693619in}}{\pgfqpoint{1.755968in}{0.704070in}}{\pgfqpoint{1.748264in}{0.711774in}}%
\pgfpathcurveto{\pgfqpoint{1.740559in}{0.719479in}}{\pgfqpoint{1.730109in}{0.723807in}}{\pgfqpoint{1.719213in}{0.723807in}}%
\pgfpathcurveto{\pgfqpoint{1.708317in}{0.723807in}}{\pgfqpoint{1.697867in}{0.719479in}}{\pgfqpoint{1.690162in}{0.711774in}}%
\pgfpathcurveto{\pgfqpoint{1.682458in}{0.704070in}}{\pgfqpoint{1.678129in}{0.693619in}}{\pgfqpoint{1.678129in}{0.682724in}}%
\pgfpathcurveto{\pgfqpoint{1.678129in}{0.671828in}}{\pgfqpoint{1.682458in}{0.661377in}}{\pgfqpoint{1.690162in}{0.653673in}}%
\pgfpathcurveto{\pgfqpoint{1.697867in}{0.645969in}}{\pgfqpoint{1.708317in}{0.641640in}}{\pgfqpoint{1.719213in}{0.641640in}}%
\pgfusepath{stroke}%
\end{pgfscope}%
\begin{pgfscope}%
\pgfpathrectangle{\pgfqpoint{0.688192in}{0.670138in}}{\pgfqpoint{7.111808in}{5.061530in}}%
\pgfusepath{clip}%
\pgfsetbuttcap%
\pgfsetroundjoin%
\pgfsetlinewidth{1.003750pt}%
\definecolor{currentstroke}{rgb}{0.000000,0.000000,0.000000}%
\pgfsetstrokecolor{currentstroke}%
\pgfsetdash{}{0pt}%
\pgfpathmoveto{\pgfqpoint{1.128607in}{0.645422in}}%
\pgfpathcurveto{\pgfqpoint{1.139503in}{0.645422in}}{\pgfqpoint{1.149954in}{0.649751in}}{\pgfqpoint{1.157658in}{0.657455in}}%
\pgfpathcurveto{\pgfqpoint{1.165362in}{0.665160in}}{\pgfqpoint{1.169691in}{0.675610in}}{\pgfqpoint{1.169691in}{0.686506in}}%
\pgfpathcurveto{\pgfqpoint{1.169691in}{0.697402in}}{\pgfqpoint{1.165362in}{0.707852in}}{\pgfqpoint{1.157658in}{0.715557in}}%
\pgfpathcurveto{\pgfqpoint{1.149954in}{0.723261in}}{\pgfqpoint{1.139503in}{0.727590in}}{\pgfqpoint{1.128607in}{0.727590in}}%
\pgfpathcurveto{\pgfqpoint{1.117712in}{0.727590in}}{\pgfqpoint{1.107261in}{0.723261in}}{\pgfqpoint{1.099556in}{0.715557in}}%
\pgfpathcurveto{\pgfqpoint{1.091852in}{0.707852in}}{\pgfqpoint{1.087523in}{0.697402in}}{\pgfqpoint{1.087523in}{0.686506in}}%
\pgfpathcurveto{\pgfqpoint{1.087523in}{0.675610in}}{\pgfqpoint{1.091852in}{0.665160in}}{\pgfqpoint{1.099556in}{0.657455in}}%
\pgfpathcurveto{\pgfqpoint{1.107261in}{0.649751in}}{\pgfqpoint{1.117712in}{0.645422in}}{\pgfqpoint{1.128607in}{0.645422in}}%
\pgfusepath{stroke}%
\end{pgfscope}%
\begin{pgfscope}%
\pgfpathrectangle{\pgfqpoint{0.688192in}{0.670138in}}{\pgfqpoint{7.111808in}{5.061530in}}%
\pgfusepath{clip}%
\pgfsetbuttcap%
\pgfsetroundjoin%
\pgfsetlinewidth{1.003750pt}%
\definecolor{currentstroke}{rgb}{0.000000,0.000000,0.000000}%
\pgfsetstrokecolor{currentstroke}%
\pgfsetdash{}{0pt}%
\pgfpathmoveto{\pgfqpoint{1.108617in}{0.647679in}}%
\pgfpathcurveto{\pgfqpoint{1.119513in}{0.647679in}}{\pgfqpoint{1.129964in}{0.652008in}}{\pgfqpoint{1.137668in}{0.659712in}}%
\pgfpathcurveto{\pgfqpoint{1.145372in}{0.667417in}}{\pgfqpoint{1.149701in}{0.677868in}}{\pgfqpoint{1.149701in}{0.688763in}}%
\pgfpathcurveto{\pgfqpoint{1.149701in}{0.699659in}}{\pgfqpoint{1.145372in}{0.710109in}}{\pgfqpoint{1.137668in}{0.717814in}}%
\pgfpathcurveto{\pgfqpoint{1.129964in}{0.725518in}}{\pgfqpoint{1.119513in}{0.729847in}}{\pgfqpoint{1.108617in}{0.729847in}}%
\pgfpathcurveto{\pgfqpoint{1.097722in}{0.729847in}}{\pgfqpoint{1.087271in}{0.725518in}}{\pgfqpoint{1.079567in}{0.717814in}}%
\pgfpathcurveto{\pgfqpoint{1.071862in}{0.710109in}}{\pgfqpoint{1.067534in}{0.699659in}}{\pgfqpoint{1.067534in}{0.688763in}}%
\pgfpathcurveto{\pgfqpoint{1.067534in}{0.677868in}}{\pgfqpoint{1.071862in}{0.667417in}}{\pgfqpoint{1.079567in}{0.659712in}}%
\pgfpathcurveto{\pgfqpoint{1.087271in}{0.652008in}}{\pgfqpoint{1.097722in}{0.647679in}}{\pgfqpoint{1.108617in}{0.647679in}}%
\pgfusepath{stroke}%
\end{pgfscope}%
\begin{pgfscope}%
\pgfpathrectangle{\pgfqpoint{0.688192in}{0.670138in}}{\pgfqpoint{7.111808in}{5.061530in}}%
\pgfusepath{clip}%
\pgfsetbuttcap%
\pgfsetroundjoin%
\pgfsetlinewidth{1.003750pt}%
\definecolor{currentstroke}{rgb}{0.000000,0.000000,0.000000}%
\pgfsetstrokecolor{currentstroke}%
\pgfsetdash{}{0pt}%
\pgfpathmoveto{\pgfqpoint{4.871883in}{0.687604in}}%
\pgfpathcurveto{\pgfqpoint{4.882778in}{0.687604in}}{\pgfqpoint{4.893229in}{0.691933in}}{\pgfqpoint{4.900934in}{0.699637in}}%
\pgfpathcurveto{\pgfqpoint{4.908638in}{0.707342in}}{\pgfqpoint{4.912967in}{0.717792in}}{\pgfqpoint{4.912967in}{0.728688in}}%
\pgfpathcurveto{\pgfqpoint{4.912967in}{0.739583in}}{\pgfqpoint{4.908638in}{0.750034in}}{\pgfqpoint{4.900934in}{0.757739in}}%
\pgfpathcurveto{\pgfqpoint{4.893229in}{0.765443in}}{\pgfqpoint{4.882778in}{0.769772in}}{\pgfqpoint{4.871883in}{0.769772in}}%
\pgfpathcurveto{\pgfqpoint{4.860987in}{0.769772in}}{\pgfqpoint{4.850536in}{0.765443in}}{\pgfqpoint{4.842832in}{0.757739in}}%
\pgfpathcurveto{\pgfqpoint{4.835128in}{0.750034in}}{\pgfqpoint{4.830799in}{0.739583in}}{\pgfqpoint{4.830799in}{0.728688in}}%
\pgfpathcurveto{\pgfqpoint{4.830799in}{0.717792in}}{\pgfqpoint{4.835128in}{0.707342in}}{\pgfqpoint{4.842832in}{0.699637in}}%
\pgfpathcurveto{\pgfqpoint{4.850536in}{0.691933in}}{\pgfqpoint{4.860987in}{0.687604in}}{\pgfqpoint{4.871883in}{0.687604in}}%
\pgfpathlineto{\pgfqpoint{4.871883in}{0.687604in}}%
\pgfpathclose%
\pgfusepath{stroke}%
\end{pgfscope}%
\begin{pgfscope}%
\pgfpathrectangle{\pgfqpoint{0.688192in}{0.670138in}}{\pgfqpoint{7.111808in}{5.061530in}}%
\pgfusepath{clip}%
\pgfsetbuttcap%
\pgfsetroundjoin%
\pgfsetlinewidth{1.003750pt}%
\definecolor{currentstroke}{rgb}{0.000000,0.000000,0.000000}%
\pgfsetstrokecolor{currentstroke}%
\pgfsetdash{}{0pt}%
\pgfpathmoveto{\pgfqpoint{0.945316in}{0.670387in}}%
\pgfpathcurveto{\pgfqpoint{0.956212in}{0.670387in}}{\pgfqpoint{0.966663in}{0.674716in}}{\pgfqpoint{0.974367in}{0.682420in}}%
\pgfpathcurveto{\pgfqpoint{0.982072in}{0.690125in}}{\pgfqpoint{0.986400in}{0.700576in}}{\pgfqpoint{0.986400in}{0.711471in}}%
\pgfpathcurveto{\pgfqpoint{0.986400in}{0.722367in}}{\pgfqpoint{0.982072in}{0.732818in}}{\pgfqpoint{0.974367in}{0.740522in}}%
\pgfpathcurveto{\pgfqpoint{0.966663in}{0.748226in}}{\pgfqpoint{0.956212in}{0.752555in}}{\pgfqpoint{0.945316in}{0.752555in}}%
\pgfpathcurveto{\pgfqpoint{0.934421in}{0.752555in}}{\pgfqpoint{0.923970in}{0.748226in}}{\pgfqpoint{0.916266in}{0.740522in}}%
\pgfpathcurveto{\pgfqpoint{0.908561in}{0.732818in}}{\pgfqpoint{0.904233in}{0.722367in}}{\pgfqpoint{0.904233in}{0.711471in}}%
\pgfpathcurveto{\pgfqpoint{0.904233in}{0.700576in}}{\pgfqpoint{0.908561in}{0.690125in}}{\pgfqpoint{0.916266in}{0.682420in}}%
\pgfpathcurveto{\pgfqpoint{0.923970in}{0.674716in}}{\pgfqpoint{0.934421in}{0.670387in}}{\pgfqpoint{0.945316in}{0.670387in}}%
\pgfpathlineto{\pgfqpoint{0.945316in}{0.670387in}}%
\pgfpathclose%
\pgfusepath{stroke}%
\end{pgfscope}%
\begin{pgfscope}%
\pgfpathrectangle{\pgfqpoint{0.688192in}{0.670138in}}{\pgfqpoint{7.111808in}{5.061530in}}%
\pgfusepath{clip}%
\pgfsetbuttcap%
\pgfsetroundjoin%
\pgfsetlinewidth{1.003750pt}%
\definecolor{currentstroke}{rgb}{0.000000,0.000000,0.000000}%
\pgfsetstrokecolor{currentstroke}%
\pgfsetdash{}{0pt}%
\pgfpathmoveto{\pgfqpoint{3.910157in}{1.714449in}}%
\pgfpathcurveto{\pgfqpoint{3.921053in}{1.714449in}}{\pgfqpoint{3.931504in}{1.718778in}}{\pgfqpoint{3.939208in}{1.726482in}}%
\pgfpathcurveto{\pgfqpoint{3.946912in}{1.734187in}}{\pgfqpoint{3.951241in}{1.744637in}}{\pgfqpoint{3.951241in}{1.755533in}}%
\pgfpathcurveto{\pgfqpoint{3.951241in}{1.766429in}}{\pgfqpoint{3.946912in}{1.776879in}}{\pgfqpoint{3.939208in}{1.784584in}}%
\pgfpathcurveto{\pgfqpoint{3.931504in}{1.792288in}}{\pgfqpoint{3.921053in}{1.796617in}}{\pgfqpoint{3.910157in}{1.796617in}}%
\pgfpathcurveto{\pgfqpoint{3.899262in}{1.796617in}}{\pgfqpoint{3.888811in}{1.792288in}}{\pgfqpoint{3.881107in}{1.784584in}}%
\pgfpathcurveto{\pgfqpoint{3.873402in}{1.776879in}}{\pgfqpoint{3.869073in}{1.766429in}}{\pgfqpoint{3.869073in}{1.755533in}}%
\pgfpathcurveto{\pgfqpoint{3.869073in}{1.744637in}}{\pgfqpoint{3.873402in}{1.734187in}}{\pgfqpoint{3.881107in}{1.726482in}}%
\pgfpathcurveto{\pgfqpoint{3.888811in}{1.718778in}}{\pgfqpoint{3.899262in}{1.714449in}}{\pgfqpoint{3.910157in}{1.714449in}}%
\pgfpathlineto{\pgfqpoint{3.910157in}{1.714449in}}%
\pgfpathclose%
\pgfusepath{stroke}%
\end{pgfscope}%
\begin{pgfscope}%
\pgfpathrectangle{\pgfqpoint{0.688192in}{0.670138in}}{\pgfqpoint{7.111808in}{5.061530in}}%
\pgfusepath{clip}%
\pgfsetbuttcap%
\pgfsetroundjoin%
\pgfsetlinewidth{1.003750pt}%
\definecolor{currentstroke}{rgb}{0.000000,0.000000,0.000000}%
\pgfsetstrokecolor{currentstroke}%
\pgfsetdash{}{0pt}%
\pgfpathmoveto{\pgfqpoint{1.070825in}{0.656067in}}%
\pgfpathcurveto{\pgfqpoint{1.081720in}{0.656067in}}{\pgfqpoint{1.092171in}{0.660396in}}{\pgfqpoint{1.099876in}{0.668100in}}%
\pgfpathcurveto{\pgfqpoint{1.107580in}{0.675805in}}{\pgfqpoint{1.111909in}{0.686256in}}{\pgfqpoint{1.111909in}{0.697151in}}%
\pgfpathcurveto{\pgfqpoint{1.111909in}{0.708047in}}{\pgfqpoint{1.107580in}{0.718498in}}{\pgfqpoint{1.099876in}{0.726202in}}%
\pgfpathcurveto{\pgfqpoint{1.092171in}{0.733906in}}{\pgfqpoint{1.081720in}{0.738235in}}{\pgfqpoint{1.070825in}{0.738235in}}%
\pgfpathcurveto{\pgfqpoint{1.059929in}{0.738235in}}{\pgfqpoint{1.049478in}{0.733906in}}{\pgfqpoint{1.041774in}{0.726202in}}%
\pgfpathcurveto{\pgfqpoint{1.034070in}{0.718498in}}{\pgfqpoint{1.029741in}{0.708047in}}{\pgfqpoint{1.029741in}{0.697151in}}%
\pgfpathcurveto{\pgfqpoint{1.029741in}{0.686256in}}{\pgfqpoint{1.034070in}{0.675805in}}{\pgfqpoint{1.041774in}{0.668100in}}%
\pgfpathcurveto{\pgfqpoint{1.049478in}{0.660396in}}{\pgfqpoint{1.059929in}{0.656067in}}{\pgfqpoint{1.070825in}{0.656067in}}%
\pgfusepath{stroke}%
\end{pgfscope}%
\begin{pgfscope}%
\pgfpathrectangle{\pgfqpoint{0.688192in}{0.670138in}}{\pgfqpoint{7.111808in}{5.061530in}}%
\pgfusepath{clip}%
\pgfsetbuttcap%
\pgfsetroundjoin%
\pgfsetlinewidth{1.003750pt}%
\definecolor{currentstroke}{rgb}{0.000000,0.000000,0.000000}%
\pgfsetstrokecolor{currentstroke}%
\pgfsetdash{}{0pt}%
\pgfpathmoveto{\pgfqpoint{5.827103in}{2.421561in}}%
\pgfpathcurveto{\pgfqpoint{5.837998in}{2.421561in}}{\pgfqpoint{5.848449in}{2.425890in}}{\pgfqpoint{5.856153in}{2.433594in}}%
\pgfpathcurveto{\pgfqpoint{5.863858in}{2.441299in}}{\pgfqpoint{5.868187in}{2.451749in}}{\pgfqpoint{5.868187in}{2.462645in}}%
\pgfpathcurveto{\pgfqpoint{5.868187in}{2.473540in}}{\pgfqpoint{5.863858in}{2.483991in}}{\pgfqpoint{5.856153in}{2.491696in}}%
\pgfpathcurveto{\pgfqpoint{5.848449in}{2.499400in}}{\pgfqpoint{5.837998in}{2.503729in}}{\pgfqpoint{5.827103in}{2.503729in}}%
\pgfpathcurveto{\pgfqpoint{5.816207in}{2.503729in}}{\pgfqpoint{5.805756in}{2.499400in}}{\pgfqpoint{5.798052in}{2.491696in}}%
\pgfpathcurveto{\pgfqpoint{5.790348in}{2.483991in}}{\pgfqpoint{5.786019in}{2.473540in}}{\pgfqpoint{5.786019in}{2.462645in}}%
\pgfpathcurveto{\pgfqpoint{5.786019in}{2.451749in}}{\pgfqpoint{5.790348in}{2.441299in}}{\pgfqpoint{5.798052in}{2.433594in}}%
\pgfpathcurveto{\pgfqpoint{5.805756in}{2.425890in}}{\pgfqpoint{5.816207in}{2.421561in}}{\pgfqpoint{5.827103in}{2.421561in}}%
\pgfpathlineto{\pgfqpoint{5.827103in}{2.421561in}}%
\pgfpathclose%
\pgfusepath{stroke}%
\end{pgfscope}%
\begin{pgfscope}%
\pgfpathrectangle{\pgfqpoint{0.688192in}{0.670138in}}{\pgfqpoint{7.111808in}{5.061530in}}%
\pgfusepath{clip}%
\pgfsetbuttcap%
\pgfsetroundjoin%
\pgfsetlinewidth{1.003750pt}%
\definecolor{currentstroke}{rgb}{0.000000,0.000000,0.000000}%
\pgfsetstrokecolor{currentstroke}%
\pgfsetdash{}{0pt}%
\pgfpathmoveto{\pgfqpoint{6.248880in}{0.749609in}}%
\pgfpathcurveto{\pgfqpoint{6.259776in}{0.749609in}}{\pgfqpoint{6.270227in}{0.753937in}}{\pgfqpoint{6.277931in}{0.761642in}}%
\pgfpathcurveto{\pgfqpoint{6.285635in}{0.769346in}}{\pgfqpoint{6.289964in}{0.779797in}}{\pgfqpoint{6.289964in}{0.790692in}}%
\pgfpathcurveto{\pgfqpoint{6.289964in}{0.801588in}}{\pgfqpoint{6.285635in}{0.812039in}}{\pgfqpoint{6.277931in}{0.819743in}}%
\pgfpathcurveto{\pgfqpoint{6.270227in}{0.827448in}}{\pgfqpoint{6.259776in}{0.831776in}}{\pgfqpoint{6.248880in}{0.831776in}}%
\pgfpathcurveto{\pgfqpoint{6.237985in}{0.831776in}}{\pgfqpoint{6.227534in}{0.827448in}}{\pgfqpoint{6.219830in}{0.819743in}}%
\pgfpathcurveto{\pgfqpoint{6.212125in}{0.812039in}}{\pgfqpoint{6.207796in}{0.801588in}}{\pgfqpoint{6.207796in}{0.790692in}}%
\pgfpathcurveto{\pgfqpoint{6.207796in}{0.779797in}}{\pgfqpoint{6.212125in}{0.769346in}}{\pgfqpoint{6.219830in}{0.761642in}}%
\pgfpathcurveto{\pgfqpoint{6.227534in}{0.753937in}}{\pgfqpoint{6.237985in}{0.749609in}}{\pgfqpoint{6.248880in}{0.749609in}}%
\pgfpathlineto{\pgfqpoint{6.248880in}{0.749609in}}%
\pgfpathclose%
\pgfusepath{stroke}%
\end{pgfscope}%
\begin{pgfscope}%
\pgfpathrectangle{\pgfqpoint{0.688192in}{0.670138in}}{\pgfqpoint{7.111808in}{5.061530in}}%
\pgfusepath{clip}%
\pgfsetbuttcap%
\pgfsetroundjoin%
\pgfsetlinewidth{1.003750pt}%
\definecolor{currentstroke}{rgb}{0.000000,0.000000,0.000000}%
\pgfsetstrokecolor{currentstroke}%
\pgfsetdash{}{0pt}%
\pgfpathmoveto{\pgfqpoint{0.781884in}{0.847588in}}%
\pgfpathcurveto{\pgfqpoint{0.792780in}{0.847588in}}{\pgfqpoint{0.803230in}{0.851917in}}{\pgfqpoint{0.810935in}{0.859621in}}%
\pgfpathcurveto{\pgfqpoint{0.818639in}{0.867326in}}{\pgfqpoint{0.822968in}{0.877776in}}{\pgfqpoint{0.822968in}{0.888672in}}%
\pgfpathcurveto{\pgfqpoint{0.822968in}{0.899568in}}{\pgfqpoint{0.818639in}{0.910018in}}{\pgfqpoint{0.810935in}{0.917723in}}%
\pgfpathcurveto{\pgfqpoint{0.803230in}{0.925427in}}{\pgfqpoint{0.792780in}{0.929756in}}{\pgfqpoint{0.781884in}{0.929756in}}%
\pgfpathcurveto{\pgfqpoint{0.770989in}{0.929756in}}{\pgfqpoint{0.760538in}{0.925427in}}{\pgfqpoint{0.752833in}{0.917723in}}%
\pgfpathcurveto{\pgfqpoint{0.745129in}{0.910018in}}{\pgfqpoint{0.740800in}{0.899568in}}{\pgfqpoint{0.740800in}{0.888672in}}%
\pgfpathcurveto{\pgfqpoint{0.740800in}{0.877776in}}{\pgfqpoint{0.745129in}{0.867326in}}{\pgfqpoint{0.752833in}{0.859621in}}%
\pgfpathcurveto{\pgfqpoint{0.760538in}{0.851917in}}{\pgfqpoint{0.770989in}{0.847588in}}{\pgfqpoint{0.781884in}{0.847588in}}%
\pgfpathlineto{\pgfqpoint{0.781884in}{0.847588in}}%
\pgfpathclose%
\pgfusepath{stroke}%
\end{pgfscope}%
\begin{pgfscope}%
\pgfpathrectangle{\pgfqpoint{0.688192in}{0.670138in}}{\pgfqpoint{7.111808in}{5.061530in}}%
\pgfusepath{clip}%
\pgfsetbuttcap%
\pgfsetroundjoin%
\pgfsetlinewidth{1.003750pt}%
\definecolor{currentstroke}{rgb}{0.000000,0.000000,0.000000}%
\pgfsetstrokecolor{currentstroke}%
\pgfsetdash{}{0pt}%
\pgfpathmoveto{\pgfqpoint{1.732115in}{0.641446in}}%
\pgfpathcurveto{\pgfqpoint{1.743010in}{0.641446in}}{\pgfqpoint{1.753461in}{0.645774in}}{\pgfqpoint{1.761165in}{0.653479in}}%
\pgfpathcurveto{\pgfqpoint{1.768870in}{0.661183in}}{\pgfqpoint{1.773199in}{0.671634in}}{\pgfqpoint{1.773199in}{0.682529in}}%
\pgfpathcurveto{\pgfqpoint{1.773199in}{0.693425in}}{\pgfqpoint{1.768870in}{0.703876in}}{\pgfqpoint{1.761165in}{0.711580in}}%
\pgfpathcurveto{\pgfqpoint{1.753461in}{0.719285in}}{\pgfqpoint{1.743010in}{0.723613in}}{\pgfqpoint{1.732115in}{0.723613in}}%
\pgfpathcurveto{\pgfqpoint{1.721219in}{0.723613in}}{\pgfqpoint{1.710768in}{0.719285in}}{\pgfqpoint{1.703064in}{0.711580in}}%
\pgfpathcurveto{\pgfqpoint{1.695360in}{0.703876in}}{\pgfqpoint{1.691031in}{0.693425in}}{\pgfqpoint{1.691031in}{0.682529in}}%
\pgfpathcurveto{\pgfqpoint{1.691031in}{0.671634in}}{\pgfqpoint{1.695360in}{0.661183in}}{\pgfqpoint{1.703064in}{0.653479in}}%
\pgfpathcurveto{\pgfqpoint{1.710768in}{0.645774in}}{\pgfqpoint{1.721219in}{0.641446in}}{\pgfqpoint{1.732115in}{0.641446in}}%
\pgfusepath{stroke}%
\end{pgfscope}%
\begin{pgfscope}%
\pgfpathrectangle{\pgfqpoint{0.688192in}{0.670138in}}{\pgfqpoint{7.111808in}{5.061530in}}%
\pgfusepath{clip}%
\pgfsetbuttcap%
\pgfsetroundjoin%
\pgfsetlinewidth{1.003750pt}%
\definecolor{currentstroke}{rgb}{0.000000,0.000000,0.000000}%
\pgfsetstrokecolor{currentstroke}%
\pgfsetdash{}{0pt}%
\pgfpathmoveto{\pgfqpoint{0.946217in}{0.669736in}}%
\pgfpathcurveto{\pgfqpoint{0.957113in}{0.669736in}}{\pgfqpoint{0.967564in}{0.674065in}}{\pgfqpoint{0.975268in}{0.681769in}}%
\pgfpathcurveto{\pgfqpoint{0.982972in}{0.689473in}}{\pgfqpoint{0.987301in}{0.699924in}}{\pgfqpoint{0.987301in}{0.710820in}}%
\pgfpathcurveto{\pgfqpoint{0.987301in}{0.721715in}}{\pgfqpoint{0.982972in}{0.732166in}}{\pgfqpoint{0.975268in}{0.739870in}}%
\pgfpathcurveto{\pgfqpoint{0.967564in}{0.747575in}}{\pgfqpoint{0.957113in}{0.751904in}}{\pgfqpoint{0.946217in}{0.751904in}}%
\pgfpathcurveto{\pgfqpoint{0.935322in}{0.751904in}}{\pgfqpoint{0.924871in}{0.747575in}}{\pgfqpoint{0.917167in}{0.739870in}}%
\pgfpathcurveto{\pgfqpoint{0.909462in}{0.732166in}}{\pgfqpoint{0.905134in}{0.721715in}}{\pgfqpoint{0.905134in}{0.710820in}}%
\pgfpathcurveto{\pgfqpoint{0.905134in}{0.699924in}}{\pgfqpoint{0.909462in}{0.689473in}}{\pgfqpoint{0.917167in}{0.681769in}}%
\pgfpathcurveto{\pgfqpoint{0.924871in}{0.674065in}}{\pgfqpoint{0.935322in}{0.669736in}}{\pgfqpoint{0.946217in}{0.669736in}}%
\pgfpathlineto{\pgfqpoint{0.946217in}{0.669736in}}%
\pgfpathclose%
\pgfusepath{stroke}%
\end{pgfscope}%
\begin{pgfscope}%
\pgfpathrectangle{\pgfqpoint{0.688192in}{0.670138in}}{\pgfqpoint{7.111808in}{5.061530in}}%
\pgfusepath{clip}%
\pgfsetbuttcap%
\pgfsetroundjoin%
\pgfsetlinewidth{1.003750pt}%
\definecolor{currentstroke}{rgb}{0.000000,0.000000,0.000000}%
\pgfsetstrokecolor{currentstroke}%
\pgfsetdash{}{0pt}%
\pgfpathmoveto{\pgfqpoint{0.939729in}{0.677113in}}%
\pgfpathcurveto{\pgfqpoint{0.950625in}{0.677113in}}{\pgfqpoint{0.961075in}{0.681442in}}{\pgfqpoint{0.968780in}{0.689146in}}%
\pgfpathcurveto{\pgfqpoint{0.976484in}{0.696850in}}{\pgfqpoint{0.980813in}{0.707301in}}{\pgfqpoint{0.980813in}{0.718197in}}%
\pgfpathcurveto{\pgfqpoint{0.980813in}{0.729092in}}{\pgfqpoint{0.976484in}{0.739543in}}{\pgfqpoint{0.968780in}{0.747247in}}%
\pgfpathcurveto{\pgfqpoint{0.961075in}{0.754952in}}{\pgfqpoint{0.950625in}{0.759281in}}{\pgfqpoint{0.939729in}{0.759281in}}%
\pgfpathcurveto{\pgfqpoint{0.928834in}{0.759281in}}{\pgfqpoint{0.918383in}{0.754952in}}{\pgfqpoint{0.910678in}{0.747247in}}%
\pgfpathcurveto{\pgfqpoint{0.902974in}{0.739543in}}{\pgfqpoint{0.898645in}{0.729092in}}{\pgfqpoint{0.898645in}{0.718197in}}%
\pgfpathcurveto{\pgfqpoint{0.898645in}{0.707301in}}{\pgfqpoint{0.902974in}{0.696850in}}{\pgfqpoint{0.910678in}{0.689146in}}%
\pgfpathcurveto{\pgfqpoint{0.918383in}{0.681442in}}{\pgfqpoint{0.928834in}{0.677113in}}{\pgfqpoint{0.939729in}{0.677113in}}%
\pgfpathlineto{\pgfqpoint{0.939729in}{0.677113in}}%
\pgfpathclose%
\pgfusepath{stroke}%
\end{pgfscope}%
\begin{pgfscope}%
\pgfpathrectangle{\pgfqpoint{0.688192in}{0.670138in}}{\pgfqpoint{7.111808in}{5.061530in}}%
\pgfusepath{clip}%
\pgfsetbuttcap%
\pgfsetroundjoin%
\pgfsetlinewidth{1.003750pt}%
\definecolor{currentstroke}{rgb}{0.000000,0.000000,0.000000}%
\pgfsetstrokecolor{currentstroke}%
\pgfsetdash{}{0pt}%
\pgfpathmoveto{\pgfqpoint{4.282099in}{2.830946in}}%
\pgfpathcurveto{\pgfqpoint{4.292995in}{2.830946in}}{\pgfqpoint{4.303446in}{2.835275in}}{\pgfqpoint{4.311150in}{2.842979in}}%
\pgfpathcurveto{\pgfqpoint{4.318854in}{2.850683in}}{\pgfqpoint{4.323183in}{2.861134in}}{\pgfqpoint{4.323183in}{2.872030in}}%
\pgfpathcurveto{\pgfqpoint{4.323183in}{2.882925in}}{\pgfqpoint{4.318854in}{2.893376in}}{\pgfqpoint{4.311150in}{2.901080in}}%
\pgfpathcurveto{\pgfqpoint{4.303446in}{2.908785in}}{\pgfqpoint{4.292995in}{2.913113in}}{\pgfqpoint{4.282099in}{2.913113in}}%
\pgfpathcurveto{\pgfqpoint{4.271204in}{2.913113in}}{\pgfqpoint{4.260753in}{2.908785in}}{\pgfqpoint{4.253049in}{2.901080in}}%
\pgfpathcurveto{\pgfqpoint{4.245344in}{2.893376in}}{\pgfqpoint{4.241015in}{2.882925in}}{\pgfqpoint{4.241015in}{2.872030in}}%
\pgfpathcurveto{\pgfqpoint{4.241015in}{2.861134in}}{\pgfqpoint{4.245344in}{2.850683in}}{\pgfqpoint{4.253049in}{2.842979in}}%
\pgfpathcurveto{\pgfqpoint{4.260753in}{2.835275in}}{\pgfqpoint{4.271204in}{2.830946in}}{\pgfqpoint{4.282099in}{2.830946in}}%
\pgfpathlineto{\pgfqpoint{4.282099in}{2.830946in}}%
\pgfpathclose%
\pgfusepath{stroke}%
\end{pgfscope}%
\begin{pgfscope}%
\pgfpathrectangle{\pgfqpoint{0.688192in}{0.670138in}}{\pgfqpoint{7.111808in}{5.061530in}}%
\pgfusepath{clip}%
\pgfsetbuttcap%
\pgfsetroundjoin%
\pgfsetlinewidth{1.003750pt}%
\definecolor{currentstroke}{rgb}{0.000000,0.000000,0.000000}%
\pgfsetstrokecolor{currentstroke}%
\pgfsetdash{}{0pt}%
\pgfpathmoveto{\pgfqpoint{1.148995in}{0.644808in}}%
\pgfpathcurveto{\pgfqpoint{1.159890in}{0.644808in}}{\pgfqpoint{1.170341in}{0.649137in}}{\pgfqpoint{1.178045in}{0.656841in}}%
\pgfpathcurveto{\pgfqpoint{1.185750in}{0.664545in}}{\pgfqpoint{1.190079in}{0.674996in}}{\pgfqpoint{1.190079in}{0.685892in}}%
\pgfpathcurveto{\pgfqpoint{1.190079in}{0.696787in}}{\pgfqpoint{1.185750in}{0.707238in}}{\pgfqpoint{1.178045in}{0.714942in}}%
\pgfpathcurveto{\pgfqpoint{1.170341in}{0.722647in}}{\pgfqpoint{1.159890in}{0.726975in}}{\pgfqpoint{1.148995in}{0.726975in}}%
\pgfpathcurveto{\pgfqpoint{1.138099in}{0.726975in}}{\pgfqpoint{1.127648in}{0.722647in}}{\pgfqpoint{1.119944in}{0.714942in}}%
\pgfpathcurveto{\pgfqpoint{1.112240in}{0.707238in}}{\pgfqpoint{1.107911in}{0.696787in}}{\pgfqpoint{1.107911in}{0.685892in}}%
\pgfpathcurveto{\pgfqpoint{1.107911in}{0.674996in}}{\pgfqpoint{1.112240in}{0.664545in}}{\pgfqpoint{1.119944in}{0.656841in}}%
\pgfpathcurveto{\pgfqpoint{1.127648in}{0.649137in}}{\pgfqpoint{1.138099in}{0.644808in}}{\pgfqpoint{1.148995in}{0.644808in}}%
\pgfusepath{stroke}%
\end{pgfscope}%
\begin{pgfscope}%
\pgfpathrectangle{\pgfqpoint{0.688192in}{0.670138in}}{\pgfqpoint{7.111808in}{5.061530in}}%
\pgfusepath{clip}%
\pgfsetbuttcap%
\pgfsetroundjoin%
\pgfsetlinewidth{1.003750pt}%
\definecolor{currentstroke}{rgb}{0.000000,0.000000,0.000000}%
\pgfsetstrokecolor{currentstroke}%
\pgfsetdash{}{0pt}%
\pgfpathmoveto{\pgfqpoint{4.576929in}{0.631060in}}%
\pgfpathcurveto{\pgfqpoint{4.587825in}{0.631060in}}{\pgfqpoint{4.598276in}{0.635389in}}{\pgfqpoint{4.605980in}{0.643094in}}%
\pgfpathcurveto{\pgfqpoint{4.613684in}{0.650798in}}{\pgfqpoint{4.618013in}{0.661249in}}{\pgfqpoint{4.618013in}{0.672144in}}%
\pgfpathcurveto{\pgfqpoint{4.618013in}{0.683040in}}{\pgfqpoint{4.613684in}{0.693491in}}{\pgfqpoint{4.605980in}{0.701195in}}%
\pgfpathcurveto{\pgfqpoint{4.598276in}{0.708899in}}{\pgfqpoint{4.587825in}{0.713228in}}{\pgfqpoint{4.576929in}{0.713228in}}%
\pgfpathcurveto{\pgfqpoint{4.566034in}{0.713228in}}{\pgfqpoint{4.555583in}{0.708899in}}{\pgfqpoint{4.547878in}{0.701195in}}%
\pgfpathcurveto{\pgfqpoint{4.540174in}{0.693491in}}{\pgfqpoint{4.535845in}{0.683040in}}{\pgfqpoint{4.535845in}{0.672144in}}%
\pgfpathcurveto{\pgfqpoint{4.535845in}{0.661249in}}{\pgfqpoint{4.540174in}{0.650798in}}{\pgfqpoint{4.547878in}{0.643094in}}%
\pgfpathcurveto{\pgfqpoint{4.555583in}{0.635389in}}{\pgfqpoint{4.566034in}{0.631060in}}{\pgfqpoint{4.576929in}{0.631060in}}%
\pgfusepath{stroke}%
\end{pgfscope}%
\begin{pgfscope}%
\pgfpathrectangle{\pgfqpoint{0.688192in}{0.670138in}}{\pgfqpoint{7.111808in}{5.061530in}}%
\pgfusepath{clip}%
\pgfsetbuttcap%
\pgfsetroundjoin%
\pgfsetlinewidth{1.003750pt}%
\definecolor{currentstroke}{rgb}{0.000000,0.000000,0.000000}%
\pgfsetstrokecolor{currentstroke}%
\pgfsetdash{}{0pt}%
\pgfpathmoveto{\pgfqpoint{0.854347in}{0.695577in}}%
\pgfpathcurveto{\pgfqpoint{0.865242in}{0.695577in}}{\pgfqpoint{0.875693in}{0.699906in}}{\pgfqpoint{0.883397in}{0.707611in}}%
\pgfpathcurveto{\pgfqpoint{0.891102in}{0.715315in}}{\pgfqpoint{0.895431in}{0.725766in}}{\pgfqpoint{0.895431in}{0.736661in}}%
\pgfpathcurveto{\pgfqpoint{0.895431in}{0.747557in}}{\pgfqpoint{0.891102in}{0.758008in}}{\pgfqpoint{0.883397in}{0.765712in}}%
\pgfpathcurveto{\pgfqpoint{0.875693in}{0.773416in}}{\pgfqpoint{0.865242in}{0.777745in}}{\pgfqpoint{0.854347in}{0.777745in}}%
\pgfpathcurveto{\pgfqpoint{0.843451in}{0.777745in}}{\pgfqpoint{0.833000in}{0.773416in}}{\pgfqpoint{0.825296in}{0.765712in}}%
\pgfpathcurveto{\pgfqpoint{0.817592in}{0.758008in}}{\pgfqpoint{0.813263in}{0.747557in}}{\pgfqpoint{0.813263in}{0.736661in}}%
\pgfpathcurveto{\pgfqpoint{0.813263in}{0.725766in}}{\pgfqpoint{0.817592in}{0.715315in}}{\pgfqpoint{0.825296in}{0.707611in}}%
\pgfpathcurveto{\pgfqpoint{0.833000in}{0.699906in}}{\pgfqpoint{0.843451in}{0.695577in}}{\pgfqpoint{0.854347in}{0.695577in}}%
\pgfpathlineto{\pgfqpoint{0.854347in}{0.695577in}}%
\pgfpathclose%
\pgfusepath{stroke}%
\end{pgfscope}%
\begin{pgfscope}%
\pgfpathrectangle{\pgfqpoint{0.688192in}{0.670138in}}{\pgfqpoint{7.111808in}{5.061530in}}%
\pgfusepath{clip}%
\pgfsetbuttcap%
\pgfsetroundjoin%
\pgfsetlinewidth{1.003750pt}%
\definecolor{currentstroke}{rgb}{0.000000,0.000000,0.000000}%
\pgfsetstrokecolor{currentstroke}%
\pgfsetdash{}{0pt}%
\pgfpathmoveto{\pgfqpoint{5.260959in}{0.802663in}}%
\pgfpathcurveto{\pgfqpoint{5.271854in}{0.802663in}}{\pgfqpoint{5.282305in}{0.806992in}}{\pgfqpoint{5.290010in}{0.814696in}}%
\pgfpathcurveto{\pgfqpoint{5.297714in}{0.822400in}}{\pgfqpoint{5.302043in}{0.832851in}}{\pgfqpoint{5.302043in}{0.843747in}}%
\pgfpathcurveto{\pgfqpoint{5.302043in}{0.854642in}}{\pgfqpoint{5.297714in}{0.865093in}}{\pgfqpoint{5.290010in}{0.872798in}}%
\pgfpathcurveto{\pgfqpoint{5.282305in}{0.880502in}}{\pgfqpoint{5.271854in}{0.884831in}}{\pgfqpoint{5.260959in}{0.884831in}}%
\pgfpathcurveto{\pgfqpoint{5.250063in}{0.884831in}}{\pgfqpoint{5.239612in}{0.880502in}}{\pgfqpoint{5.231908in}{0.872798in}}%
\pgfpathcurveto{\pgfqpoint{5.224204in}{0.865093in}}{\pgfqpoint{5.219875in}{0.854642in}}{\pgfqpoint{5.219875in}{0.843747in}}%
\pgfpathcurveto{\pgfqpoint{5.219875in}{0.832851in}}{\pgfqpoint{5.224204in}{0.822400in}}{\pgfqpoint{5.231908in}{0.814696in}}%
\pgfpathcurveto{\pgfqpoint{5.239612in}{0.806992in}}{\pgfqpoint{5.250063in}{0.802663in}}{\pgfqpoint{5.260959in}{0.802663in}}%
\pgfpathlineto{\pgfqpoint{5.260959in}{0.802663in}}%
\pgfpathclose%
\pgfusepath{stroke}%
\end{pgfscope}%
\begin{pgfscope}%
\pgfpathrectangle{\pgfqpoint{0.688192in}{0.670138in}}{\pgfqpoint{7.111808in}{5.061530in}}%
\pgfusepath{clip}%
\pgfsetbuttcap%
\pgfsetroundjoin%
\pgfsetlinewidth{1.003750pt}%
\definecolor{currentstroke}{rgb}{0.000000,0.000000,0.000000}%
\pgfsetstrokecolor{currentstroke}%
\pgfsetdash{}{0pt}%
\pgfpathmoveto{\pgfqpoint{0.940682in}{0.674925in}}%
\pgfpathcurveto{\pgfqpoint{0.951577in}{0.674925in}}{\pgfqpoint{0.962028in}{0.679254in}}{\pgfqpoint{0.969732in}{0.686959in}}%
\pgfpathcurveto{\pgfqpoint{0.977437in}{0.694663in}}{\pgfqpoint{0.981766in}{0.705114in}}{\pgfqpoint{0.981766in}{0.716009in}}%
\pgfpathcurveto{\pgfqpoint{0.981766in}{0.726905in}}{\pgfqpoint{0.977437in}{0.737356in}}{\pgfqpoint{0.969732in}{0.745060in}}%
\pgfpathcurveto{\pgfqpoint{0.962028in}{0.752764in}}{\pgfqpoint{0.951577in}{0.757093in}}{\pgfqpoint{0.940682in}{0.757093in}}%
\pgfpathcurveto{\pgfqpoint{0.929786in}{0.757093in}}{\pgfqpoint{0.919335in}{0.752764in}}{\pgfqpoint{0.911631in}{0.745060in}}%
\pgfpathcurveto{\pgfqpoint{0.903927in}{0.737356in}}{\pgfqpoint{0.899598in}{0.726905in}}{\pgfqpoint{0.899598in}{0.716009in}}%
\pgfpathcurveto{\pgfqpoint{0.899598in}{0.705114in}}{\pgfqpoint{0.903927in}{0.694663in}}{\pgfqpoint{0.911631in}{0.686959in}}%
\pgfpathcurveto{\pgfqpoint{0.919335in}{0.679254in}}{\pgfqpoint{0.929786in}{0.674925in}}{\pgfqpoint{0.940682in}{0.674925in}}%
\pgfpathlineto{\pgfqpoint{0.940682in}{0.674925in}}%
\pgfpathclose%
\pgfusepath{stroke}%
\end{pgfscope}%
\begin{pgfscope}%
\pgfpathrectangle{\pgfqpoint{0.688192in}{0.670138in}}{\pgfqpoint{7.111808in}{5.061530in}}%
\pgfusepath{clip}%
\pgfsetbuttcap%
\pgfsetroundjoin%
\pgfsetlinewidth{1.003750pt}%
\definecolor{currentstroke}{rgb}{0.000000,0.000000,0.000000}%
\pgfsetstrokecolor{currentstroke}%
\pgfsetdash{}{0pt}%
\pgfpathmoveto{\pgfqpoint{4.491507in}{0.631138in}}%
\pgfpathcurveto{\pgfqpoint{4.502402in}{0.631138in}}{\pgfqpoint{4.512853in}{0.635467in}}{\pgfqpoint{4.520557in}{0.643171in}}%
\pgfpathcurveto{\pgfqpoint{4.528262in}{0.650875in}}{\pgfqpoint{4.532590in}{0.661326in}}{\pgfqpoint{4.532590in}{0.672222in}}%
\pgfpathcurveto{\pgfqpoint{4.532590in}{0.683117in}}{\pgfqpoint{4.528262in}{0.693568in}}{\pgfqpoint{4.520557in}{0.701272in}}%
\pgfpathcurveto{\pgfqpoint{4.512853in}{0.708977in}}{\pgfqpoint{4.502402in}{0.713305in}}{\pgfqpoint{4.491507in}{0.713305in}}%
\pgfpathcurveto{\pgfqpoint{4.480611in}{0.713305in}}{\pgfqpoint{4.470160in}{0.708977in}}{\pgfqpoint{4.462456in}{0.701272in}}%
\pgfpathcurveto{\pgfqpoint{4.454752in}{0.693568in}}{\pgfqpoint{4.450423in}{0.683117in}}{\pgfqpoint{4.450423in}{0.672222in}}%
\pgfpathcurveto{\pgfqpoint{4.450423in}{0.661326in}}{\pgfqpoint{4.454752in}{0.650875in}}{\pgfqpoint{4.462456in}{0.643171in}}%
\pgfpathcurveto{\pgfqpoint{4.470160in}{0.635467in}}{\pgfqpoint{4.480611in}{0.631138in}}{\pgfqpoint{4.491507in}{0.631138in}}%
\pgfusepath{stroke}%
\end{pgfscope}%
\begin{pgfscope}%
\pgfpathrectangle{\pgfqpoint{0.688192in}{0.670138in}}{\pgfqpoint{7.111808in}{5.061530in}}%
\pgfusepath{clip}%
\pgfsetbuttcap%
\pgfsetroundjoin%
\pgfsetlinewidth{1.003750pt}%
\definecolor{currentstroke}{rgb}{0.000000,0.000000,0.000000}%
\pgfsetstrokecolor{currentstroke}%
\pgfsetdash{}{0pt}%
\pgfpathmoveto{\pgfqpoint{6.801853in}{3.956709in}}%
\pgfpathcurveto{\pgfqpoint{6.812749in}{3.956709in}}{\pgfqpoint{6.823199in}{3.961038in}}{\pgfqpoint{6.830904in}{3.968742in}}%
\pgfpathcurveto{\pgfqpoint{6.838608in}{3.976447in}}{\pgfqpoint{6.842937in}{3.986897in}}{\pgfqpoint{6.842937in}{3.997793in}}%
\pgfpathcurveto{\pgfqpoint{6.842937in}{4.008689in}}{\pgfqpoint{6.838608in}{4.019139in}}{\pgfqpoint{6.830904in}{4.026844in}}%
\pgfpathcurveto{\pgfqpoint{6.823199in}{4.034548in}}{\pgfqpoint{6.812749in}{4.038877in}}{\pgfqpoint{6.801853in}{4.038877in}}%
\pgfpathcurveto{\pgfqpoint{6.790957in}{4.038877in}}{\pgfqpoint{6.780507in}{4.034548in}}{\pgfqpoint{6.772802in}{4.026844in}}%
\pgfpathcurveto{\pgfqpoint{6.765098in}{4.019139in}}{\pgfqpoint{6.760769in}{4.008689in}}{\pgfqpoint{6.760769in}{3.997793in}}%
\pgfpathcurveto{\pgfqpoint{6.760769in}{3.986897in}}{\pgfqpoint{6.765098in}{3.976447in}}{\pgfqpoint{6.772802in}{3.968742in}}%
\pgfpathcurveto{\pgfqpoint{6.780507in}{3.961038in}}{\pgfqpoint{6.790957in}{3.956709in}}{\pgfqpoint{6.801853in}{3.956709in}}%
\pgfpathlineto{\pgfqpoint{6.801853in}{3.956709in}}%
\pgfpathclose%
\pgfusepath{stroke}%
\end{pgfscope}%
\begin{pgfscope}%
\pgfpathrectangle{\pgfqpoint{0.688192in}{0.670138in}}{\pgfqpoint{7.111808in}{5.061530in}}%
\pgfusepath{clip}%
\pgfsetbuttcap%
\pgfsetroundjoin%
\pgfsetlinewidth{1.003750pt}%
\definecolor{currentstroke}{rgb}{0.000000,0.000000,0.000000}%
\pgfsetstrokecolor{currentstroke}%
\pgfsetdash{}{0pt}%
\pgfpathmoveto{\pgfqpoint{1.178009in}{0.644363in}}%
\pgfpathcurveto{\pgfqpoint{1.188905in}{0.644363in}}{\pgfqpoint{1.199355in}{0.648692in}}{\pgfqpoint{1.207060in}{0.656396in}}%
\pgfpathcurveto{\pgfqpoint{1.214764in}{0.664100in}}{\pgfqpoint{1.219093in}{0.674551in}}{\pgfqpoint{1.219093in}{0.685447in}}%
\pgfpathcurveto{\pgfqpoint{1.219093in}{0.696342in}}{\pgfqpoint{1.214764in}{0.706793in}}{\pgfqpoint{1.207060in}{0.714497in}}%
\pgfpathcurveto{\pgfqpoint{1.199355in}{0.722202in}}{\pgfqpoint{1.188905in}{0.726531in}}{\pgfqpoint{1.178009in}{0.726531in}}%
\pgfpathcurveto{\pgfqpoint{1.167113in}{0.726531in}}{\pgfqpoint{1.156663in}{0.722202in}}{\pgfqpoint{1.148958in}{0.714497in}}%
\pgfpathcurveto{\pgfqpoint{1.141254in}{0.706793in}}{\pgfqpoint{1.136925in}{0.696342in}}{\pgfqpoint{1.136925in}{0.685447in}}%
\pgfpathcurveto{\pgfqpoint{1.136925in}{0.674551in}}{\pgfqpoint{1.141254in}{0.664100in}}{\pgfqpoint{1.148958in}{0.656396in}}%
\pgfpathcurveto{\pgfqpoint{1.156663in}{0.648692in}}{\pgfqpoint{1.167113in}{0.644363in}}{\pgfqpoint{1.178009in}{0.644363in}}%
\pgfusepath{stroke}%
\end{pgfscope}%
\begin{pgfscope}%
\pgfpathrectangle{\pgfqpoint{0.688192in}{0.670138in}}{\pgfqpoint{7.111808in}{5.061530in}}%
\pgfusepath{clip}%
\pgfsetbuttcap%
\pgfsetroundjoin%
\pgfsetlinewidth{1.003750pt}%
\definecolor{currentstroke}{rgb}{0.000000,0.000000,0.000000}%
\pgfsetstrokecolor{currentstroke}%
\pgfsetdash{}{0pt}%
\pgfpathmoveto{\pgfqpoint{6.958187in}{0.740934in}}%
\pgfpathcurveto{\pgfqpoint{6.969082in}{0.740934in}}{\pgfqpoint{6.979533in}{0.745263in}}{\pgfqpoint{6.987238in}{0.752968in}}%
\pgfpathcurveto{\pgfqpoint{6.994942in}{0.760672in}}{\pgfqpoint{6.999271in}{0.771123in}}{\pgfqpoint{6.999271in}{0.782018in}}%
\pgfpathcurveto{\pgfqpoint{6.999271in}{0.792914in}}{\pgfqpoint{6.994942in}{0.803365in}}{\pgfqpoint{6.987238in}{0.811069in}}%
\pgfpathcurveto{\pgfqpoint{6.979533in}{0.818773in}}{\pgfqpoint{6.969082in}{0.823102in}}{\pgfqpoint{6.958187in}{0.823102in}}%
\pgfpathcurveto{\pgfqpoint{6.947291in}{0.823102in}}{\pgfqpoint{6.936841in}{0.818773in}}{\pgfqpoint{6.929136in}{0.811069in}}%
\pgfpathcurveto{\pgfqpoint{6.921432in}{0.803365in}}{\pgfqpoint{6.917103in}{0.792914in}}{\pgfqpoint{6.917103in}{0.782018in}}%
\pgfpathcurveto{\pgfqpoint{6.917103in}{0.771123in}}{\pgfqpoint{6.921432in}{0.760672in}}{\pgfqpoint{6.929136in}{0.752968in}}%
\pgfpathcurveto{\pgfqpoint{6.936841in}{0.745263in}}{\pgfqpoint{6.947291in}{0.740934in}}{\pgfqpoint{6.958187in}{0.740934in}}%
\pgfpathlineto{\pgfqpoint{6.958187in}{0.740934in}}%
\pgfpathclose%
\pgfusepath{stroke}%
\end{pgfscope}%
\begin{pgfscope}%
\pgfpathrectangle{\pgfqpoint{0.688192in}{0.670138in}}{\pgfqpoint{7.111808in}{5.061530in}}%
\pgfusepath{clip}%
\pgfsetbuttcap%
\pgfsetroundjoin%
\pgfsetlinewidth{1.003750pt}%
\definecolor{currentstroke}{rgb}{0.000000,0.000000,0.000000}%
\pgfsetstrokecolor{currentstroke}%
\pgfsetdash{}{0pt}%
\pgfpathmoveto{\pgfqpoint{2.378269in}{0.637968in}}%
\pgfpathcurveto{\pgfqpoint{2.389164in}{0.637968in}}{\pgfqpoint{2.399615in}{0.642297in}}{\pgfqpoint{2.407320in}{0.650002in}}%
\pgfpathcurveto{\pgfqpoint{2.415024in}{0.657706in}}{\pgfqpoint{2.419353in}{0.668157in}}{\pgfqpoint{2.419353in}{0.679052in}}%
\pgfpathcurveto{\pgfqpoint{2.419353in}{0.689948in}}{\pgfqpoint{2.415024in}{0.700399in}}{\pgfqpoint{2.407320in}{0.708103in}}%
\pgfpathcurveto{\pgfqpoint{2.399615in}{0.715807in}}{\pgfqpoint{2.389164in}{0.720136in}}{\pgfqpoint{2.378269in}{0.720136in}}%
\pgfpathcurveto{\pgfqpoint{2.367373in}{0.720136in}}{\pgfqpoint{2.356923in}{0.715807in}}{\pgfqpoint{2.349218in}{0.708103in}}%
\pgfpathcurveto{\pgfqpoint{2.341514in}{0.700399in}}{\pgfqpoint{2.337185in}{0.689948in}}{\pgfqpoint{2.337185in}{0.679052in}}%
\pgfpathcurveto{\pgfqpoint{2.337185in}{0.668157in}}{\pgfqpoint{2.341514in}{0.657706in}}{\pgfqpoint{2.349218in}{0.650002in}}%
\pgfpathcurveto{\pgfqpoint{2.356923in}{0.642297in}}{\pgfqpoint{2.367373in}{0.637968in}}{\pgfqpoint{2.378269in}{0.637968in}}%
\pgfusepath{stroke}%
\end{pgfscope}%
\begin{pgfscope}%
\pgfpathrectangle{\pgfqpoint{0.688192in}{0.670138in}}{\pgfqpoint{7.111808in}{5.061530in}}%
\pgfusepath{clip}%
\pgfsetbuttcap%
\pgfsetroundjoin%
\pgfsetlinewidth{1.003750pt}%
\definecolor{currentstroke}{rgb}{0.000000,0.000000,0.000000}%
\pgfsetstrokecolor{currentstroke}%
\pgfsetdash{}{0pt}%
\pgfpathmoveto{\pgfqpoint{5.700380in}{1.353933in}}%
\pgfpathcurveto{\pgfqpoint{5.711276in}{1.353933in}}{\pgfqpoint{5.721726in}{1.358262in}}{\pgfqpoint{5.729431in}{1.365966in}}%
\pgfpathcurveto{\pgfqpoint{5.737135in}{1.373670in}}{\pgfqpoint{5.741464in}{1.384121in}}{\pgfqpoint{5.741464in}{1.395017in}}%
\pgfpathcurveto{\pgfqpoint{5.741464in}{1.405912in}}{\pgfqpoint{5.737135in}{1.416363in}}{\pgfqpoint{5.729431in}{1.424067in}}%
\pgfpathcurveto{\pgfqpoint{5.721726in}{1.431772in}}{\pgfqpoint{5.711276in}{1.436101in}}{\pgfqpoint{5.700380in}{1.436101in}}%
\pgfpathcurveto{\pgfqpoint{5.689484in}{1.436101in}}{\pgfqpoint{5.679034in}{1.431772in}}{\pgfqpoint{5.671329in}{1.424067in}}%
\pgfpathcurveto{\pgfqpoint{5.663625in}{1.416363in}}{\pgfqpoint{5.659296in}{1.405912in}}{\pgfqpoint{5.659296in}{1.395017in}}%
\pgfpathcurveto{\pgfqpoint{5.659296in}{1.384121in}}{\pgfqpoint{5.663625in}{1.373670in}}{\pgfqpoint{5.671329in}{1.365966in}}%
\pgfpathcurveto{\pgfqpoint{5.679034in}{1.358262in}}{\pgfqpoint{5.689484in}{1.353933in}}{\pgfqpoint{5.700380in}{1.353933in}}%
\pgfpathlineto{\pgfqpoint{5.700380in}{1.353933in}}%
\pgfpathclose%
\pgfusepath{stroke}%
\end{pgfscope}%
\begin{pgfscope}%
\pgfpathrectangle{\pgfqpoint{0.688192in}{0.670138in}}{\pgfqpoint{7.111808in}{5.061530in}}%
\pgfusepath{clip}%
\pgfsetbuttcap%
\pgfsetroundjoin%
\pgfsetlinewidth{1.003750pt}%
\definecolor{currentstroke}{rgb}{0.000000,0.000000,0.000000}%
\pgfsetstrokecolor{currentstroke}%
\pgfsetdash{}{0pt}%
\pgfpathmoveto{\pgfqpoint{4.292716in}{0.683341in}}%
\pgfpathcurveto{\pgfqpoint{4.303612in}{0.683341in}}{\pgfqpoint{4.314062in}{0.687670in}}{\pgfqpoint{4.321767in}{0.695374in}}%
\pgfpathcurveto{\pgfqpoint{4.329471in}{0.703079in}}{\pgfqpoint{4.333800in}{0.713529in}}{\pgfqpoint{4.333800in}{0.724425in}}%
\pgfpathcurveto{\pgfqpoint{4.333800in}{0.735321in}}{\pgfqpoint{4.329471in}{0.745771in}}{\pgfqpoint{4.321767in}{0.753476in}}%
\pgfpathcurveto{\pgfqpoint{4.314062in}{0.761180in}}{\pgfqpoint{4.303612in}{0.765509in}}{\pgfqpoint{4.292716in}{0.765509in}}%
\pgfpathcurveto{\pgfqpoint{4.281820in}{0.765509in}}{\pgfqpoint{4.271370in}{0.761180in}}{\pgfqpoint{4.263665in}{0.753476in}}%
\pgfpathcurveto{\pgfqpoint{4.255961in}{0.745771in}}{\pgfqpoint{4.251632in}{0.735321in}}{\pgfqpoint{4.251632in}{0.724425in}}%
\pgfpathcurveto{\pgfqpoint{4.251632in}{0.713529in}}{\pgfqpoint{4.255961in}{0.703079in}}{\pgfqpoint{4.263665in}{0.695374in}}%
\pgfpathcurveto{\pgfqpoint{4.271370in}{0.687670in}}{\pgfqpoint{4.281820in}{0.683341in}}{\pgfqpoint{4.292716in}{0.683341in}}%
\pgfpathlineto{\pgfqpoint{4.292716in}{0.683341in}}%
\pgfpathclose%
\pgfusepath{stroke}%
\end{pgfscope}%
\begin{pgfscope}%
\pgfpathrectangle{\pgfqpoint{0.688192in}{0.670138in}}{\pgfqpoint{7.111808in}{5.061530in}}%
\pgfusepath{clip}%
\pgfsetbuttcap%
\pgfsetroundjoin%
\pgfsetlinewidth{1.003750pt}%
\definecolor{currentstroke}{rgb}{0.000000,0.000000,0.000000}%
\pgfsetstrokecolor{currentstroke}%
\pgfsetdash{}{0pt}%
\pgfpathmoveto{\pgfqpoint{0.990735in}{0.663640in}}%
\pgfpathcurveto{\pgfqpoint{1.001630in}{0.663640in}}{\pgfqpoint{1.012081in}{0.667969in}}{\pgfqpoint{1.019785in}{0.675673in}}%
\pgfpathcurveto{\pgfqpoint{1.027490in}{0.683378in}}{\pgfqpoint{1.031818in}{0.693828in}}{\pgfqpoint{1.031818in}{0.704724in}}%
\pgfpathcurveto{\pgfqpoint{1.031818in}{0.715620in}}{\pgfqpoint{1.027490in}{0.726070in}}{\pgfqpoint{1.019785in}{0.733775in}}%
\pgfpathcurveto{\pgfqpoint{1.012081in}{0.741479in}}{\pgfqpoint{1.001630in}{0.745808in}}{\pgfqpoint{0.990735in}{0.745808in}}%
\pgfpathcurveto{\pgfqpoint{0.979839in}{0.745808in}}{\pgfqpoint{0.969388in}{0.741479in}}{\pgfqpoint{0.961684in}{0.733775in}}%
\pgfpathcurveto{\pgfqpoint{0.953980in}{0.726070in}}{\pgfqpoint{0.949651in}{0.715620in}}{\pgfqpoint{0.949651in}{0.704724in}}%
\pgfpathcurveto{\pgfqpoint{0.949651in}{0.693828in}}{\pgfqpoint{0.953980in}{0.683378in}}{\pgfqpoint{0.961684in}{0.675673in}}%
\pgfpathcurveto{\pgfqpoint{0.969388in}{0.667969in}}{\pgfqpoint{0.979839in}{0.663640in}}{\pgfqpoint{0.990735in}{0.663640in}}%
\pgfusepath{stroke}%
\end{pgfscope}%
\begin{pgfscope}%
\pgfpathrectangle{\pgfqpoint{0.688192in}{0.670138in}}{\pgfqpoint{7.111808in}{5.061530in}}%
\pgfusepath{clip}%
\pgfsetbuttcap%
\pgfsetroundjoin%
\pgfsetlinewidth{1.003750pt}%
\definecolor{currentstroke}{rgb}{0.000000,0.000000,0.000000}%
\pgfsetstrokecolor{currentstroke}%
\pgfsetdash{}{0pt}%
\pgfpathmoveto{\pgfqpoint{1.103303in}{0.648067in}}%
\pgfpathcurveto{\pgfqpoint{1.114198in}{0.648067in}}{\pgfqpoint{1.124649in}{0.652396in}}{\pgfqpoint{1.132353in}{0.660101in}}%
\pgfpathcurveto{\pgfqpoint{1.140058in}{0.667805in}}{\pgfqpoint{1.144386in}{0.678256in}}{\pgfqpoint{1.144386in}{0.689151in}}%
\pgfpathcurveto{\pgfqpoint{1.144386in}{0.700047in}}{\pgfqpoint{1.140058in}{0.710498in}}{\pgfqpoint{1.132353in}{0.718202in}}%
\pgfpathcurveto{\pgfqpoint{1.124649in}{0.725906in}}{\pgfqpoint{1.114198in}{0.730235in}}{\pgfqpoint{1.103303in}{0.730235in}}%
\pgfpathcurveto{\pgfqpoint{1.092407in}{0.730235in}}{\pgfqpoint{1.081956in}{0.725906in}}{\pgfqpoint{1.074252in}{0.718202in}}%
\pgfpathcurveto{\pgfqpoint{1.066548in}{0.710498in}}{\pgfqpoint{1.062219in}{0.700047in}}{\pgfqpoint{1.062219in}{0.689151in}}%
\pgfpathcurveto{\pgfqpoint{1.062219in}{0.678256in}}{\pgfqpoint{1.066548in}{0.667805in}}{\pgfqpoint{1.074252in}{0.660101in}}%
\pgfpathcurveto{\pgfqpoint{1.081956in}{0.652396in}}{\pgfqpoint{1.092407in}{0.648067in}}{\pgfqpoint{1.103303in}{0.648067in}}%
\pgfusepath{stroke}%
\end{pgfscope}%
\begin{pgfscope}%
\pgfpathrectangle{\pgfqpoint{0.688192in}{0.670138in}}{\pgfqpoint{7.111808in}{5.061530in}}%
\pgfusepath{clip}%
\pgfsetbuttcap%
\pgfsetroundjoin%
\pgfsetlinewidth{1.003750pt}%
\definecolor{currentstroke}{rgb}{0.000000,0.000000,0.000000}%
\pgfsetstrokecolor{currentstroke}%
\pgfsetdash{}{0pt}%
\pgfpathmoveto{\pgfqpoint{4.994709in}{0.630180in}}%
\pgfpathcurveto{\pgfqpoint{5.005605in}{0.630180in}}{\pgfqpoint{5.016056in}{0.634509in}}{\pgfqpoint{5.023760in}{0.642213in}}%
\pgfpathcurveto{\pgfqpoint{5.031464in}{0.649918in}}{\pgfqpoint{5.035793in}{0.660368in}}{\pgfqpoint{5.035793in}{0.671264in}}%
\pgfpathcurveto{\pgfqpoint{5.035793in}{0.682160in}}{\pgfqpoint{5.031464in}{0.692610in}}{\pgfqpoint{5.023760in}{0.700315in}}%
\pgfpathcurveto{\pgfqpoint{5.016056in}{0.708019in}}{\pgfqpoint{5.005605in}{0.712348in}}{\pgfqpoint{4.994709in}{0.712348in}}%
\pgfpathcurveto{\pgfqpoint{4.983814in}{0.712348in}}{\pgfqpoint{4.973363in}{0.708019in}}{\pgfqpoint{4.965658in}{0.700315in}}%
\pgfpathcurveto{\pgfqpoint{4.957954in}{0.692610in}}{\pgfqpoint{4.953625in}{0.682160in}}{\pgfqpoint{4.953625in}{0.671264in}}%
\pgfpathcurveto{\pgfqpoint{4.953625in}{0.660368in}}{\pgfqpoint{4.957954in}{0.649918in}}{\pgfqpoint{4.965658in}{0.642213in}}%
\pgfpathcurveto{\pgfqpoint{4.973363in}{0.634509in}}{\pgfqpoint{4.983814in}{0.630180in}}{\pgfqpoint{4.994709in}{0.630180in}}%
\pgfusepath{stroke}%
\end{pgfscope}%
\begin{pgfscope}%
\pgfpathrectangle{\pgfqpoint{0.688192in}{0.670138in}}{\pgfqpoint{7.111808in}{5.061530in}}%
\pgfusepath{clip}%
\pgfsetbuttcap%
\pgfsetroundjoin%
\pgfsetlinewidth{1.003750pt}%
\definecolor{currentstroke}{rgb}{0.000000,0.000000,0.000000}%
\pgfsetstrokecolor{currentstroke}%
\pgfsetdash{}{0pt}%
\pgfpathmoveto{\pgfqpoint{2.095056in}{2.791135in}}%
\pgfpathcurveto{\pgfqpoint{2.105952in}{2.791135in}}{\pgfqpoint{2.116403in}{2.795464in}}{\pgfqpoint{2.124107in}{2.803168in}}%
\pgfpathcurveto{\pgfqpoint{2.131811in}{2.810873in}}{\pgfqpoint{2.136140in}{2.821323in}}{\pgfqpoint{2.136140in}{2.832219in}}%
\pgfpathcurveto{\pgfqpoint{2.136140in}{2.843114in}}{\pgfqpoint{2.131811in}{2.853565in}}{\pgfqpoint{2.124107in}{2.861270in}}%
\pgfpathcurveto{\pgfqpoint{2.116403in}{2.868974in}}{\pgfqpoint{2.105952in}{2.873303in}}{\pgfqpoint{2.095056in}{2.873303in}}%
\pgfpathcurveto{\pgfqpoint{2.084161in}{2.873303in}}{\pgfqpoint{2.073710in}{2.868974in}}{\pgfqpoint{2.066006in}{2.861270in}}%
\pgfpathcurveto{\pgfqpoint{2.058301in}{2.853565in}}{\pgfqpoint{2.053973in}{2.843114in}}{\pgfqpoint{2.053973in}{2.832219in}}%
\pgfpathcurveto{\pgfqpoint{2.053973in}{2.821323in}}{\pgfqpoint{2.058301in}{2.810873in}}{\pgfqpoint{2.066006in}{2.803168in}}%
\pgfpathcurveto{\pgfqpoint{2.073710in}{2.795464in}}{\pgfqpoint{2.084161in}{2.791135in}}{\pgfqpoint{2.095056in}{2.791135in}}%
\pgfpathlineto{\pgfqpoint{2.095056in}{2.791135in}}%
\pgfpathclose%
\pgfusepath{stroke}%
\end{pgfscope}%
\begin{pgfscope}%
\pgfpathrectangle{\pgfqpoint{0.688192in}{0.670138in}}{\pgfqpoint{7.111808in}{5.061530in}}%
\pgfusepath{clip}%
\pgfsetbuttcap%
\pgfsetroundjoin%
\pgfsetlinewidth{1.003750pt}%
\definecolor{currentstroke}{rgb}{0.000000,0.000000,0.000000}%
\pgfsetstrokecolor{currentstroke}%
\pgfsetdash{}{0pt}%
\pgfpathmoveto{\pgfqpoint{4.956446in}{3.235816in}}%
\pgfpathcurveto{\pgfqpoint{4.967342in}{3.235816in}}{\pgfqpoint{4.977793in}{3.240145in}}{\pgfqpoint{4.985497in}{3.247849in}}%
\pgfpathcurveto{\pgfqpoint{4.993201in}{3.255553in}}{\pgfqpoint{4.997530in}{3.266004in}}{\pgfqpoint{4.997530in}{3.276900in}}%
\pgfpathcurveto{\pgfqpoint{4.997530in}{3.287795in}}{\pgfqpoint{4.993201in}{3.298246in}}{\pgfqpoint{4.985497in}{3.305950in}}%
\pgfpathcurveto{\pgfqpoint{4.977793in}{3.313655in}}{\pgfqpoint{4.967342in}{3.317984in}}{\pgfqpoint{4.956446in}{3.317984in}}%
\pgfpathcurveto{\pgfqpoint{4.945551in}{3.317984in}}{\pgfqpoint{4.935100in}{3.313655in}}{\pgfqpoint{4.927396in}{3.305950in}}%
\pgfpathcurveto{\pgfqpoint{4.919691in}{3.298246in}}{\pgfqpoint{4.915363in}{3.287795in}}{\pgfqpoint{4.915363in}{3.276900in}}%
\pgfpathcurveto{\pgfqpoint{4.915363in}{3.266004in}}{\pgfqpoint{4.919691in}{3.255553in}}{\pgfqpoint{4.927396in}{3.247849in}}%
\pgfpathcurveto{\pgfqpoint{4.935100in}{3.240145in}}{\pgfqpoint{4.945551in}{3.235816in}}{\pgfqpoint{4.956446in}{3.235816in}}%
\pgfpathlineto{\pgfqpoint{4.956446in}{3.235816in}}%
\pgfpathclose%
\pgfusepath{stroke}%
\end{pgfscope}%
\begin{pgfscope}%
\pgfpathrectangle{\pgfqpoint{0.688192in}{0.670138in}}{\pgfqpoint{7.111808in}{5.061530in}}%
\pgfusepath{clip}%
\pgfsetbuttcap%
\pgfsetroundjoin%
\pgfsetlinewidth{1.003750pt}%
\definecolor{currentstroke}{rgb}{0.000000,0.000000,0.000000}%
\pgfsetstrokecolor{currentstroke}%
\pgfsetdash{}{0pt}%
\pgfpathmoveto{\pgfqpoint{1.437795in}{2.104006in}}%
\pgfpathcurveto{\pgfqpoint{1.448691in}{2.104006in}}{\pgfqpoint{1.459141in}{2.108335in}}{\pgfqpoint{1.466846in}{2.116039in}}%
\pgfpathcurveto{\pgfqpoint{1.474550in}{2.123743in}}{\pgfqpoint{1.478879in}{2.134194in}}{\pgfqpoint{1.478879in}{2.145090in}}%
\pgfpathcurveto{\pgfqpoint{1.478879in}{2.155985in}}{\pgfqpoint{1.474550in}{2.166436in}}{\pgfqpoint{1.466846in}{2.174140in}}%
\pgfpathcurveto{\pgfqpoint{1.459141in}{2.181845in}}{\pgfqpoint{1.448691in}{2.186174in}}{\pgfqpoint{1.437795in}{2.186174in}}%
\pgfpathcurveto{\pgfqpoint{1.426899in}{2.186174in}}{\pgfqpoint{1.416449in}{2.181845in}}{\pgfqpoint{1.408744in}{2.174140in}}%
\pgfpathcurveto{\pgfqpoint{1.401040in}{2.166436in}}{\pgfqpoint{1.396711in}{2.155985in}}{\pgfqpoint{1.396711in}{2.145090in}}%
\pgfpathcurveto{\pgfqpoint{1.396711in}{2.134194in}}{\pgfqpoint{1.401040in}{2.123743in}}{\pgfqpoint{1.408744in}{2.116039in}}%
\pgfpathcurveto{\pgfqpoint{1.416449in}{2.108335in}}{\pgfqpoint{1.426899in}{2.104006in}}{\pgfqpoint{1.437795in}{2.104006in}}%
\pgfpathlineto{\pgfqpoint{1.437795in}{2.104006in}}%
\pgfpathclose%
\pgfusepath{stroke}%
\end{pgfscope}%
\begin{pgfscope}%
\pgfpathrectangle{\pgfqpoint{0.688192in}{0.670138in}}{\pgfqpoint{7.111808in}{5.061530in}}%
\pgfusepath{clip}%
\pgfsetbuttcap%
\pgfsetroundjoin%
\pgfsetlinewidth{1.003750pt}%
\definecolor{currentstroke}{rgb}{0.000000,0.000000,0.000000}%
\pgfsetstrokecolor{currentstroke}%
\pgfsetdash{}{0pt}%
\pgfpathmoveto{\pgfqpoint{2.319819in}{1.096027in}}%
\pgfpathcurveto{\pgfqpoint{2.330715in}{1.096027in}}{\pgfqpoint{2.341166in}{1.100356in}}{\pgfqpoint{2.348870in}{1.108061in}}%
\pgfpathcurveto{\pgfqpoint{2.356574in}{1.115765in}}{\pgfqpoint{2.360903in}{1.126216in}}{\pgfqpoint{2.360903in}{1.137111in}}%
\pgfpathcurveto{\pgfqpoint{2.360903in}{1.148007in}}{\pgfqpoint{2.356574in}{1.158458in}}{\pgfqpoint{2.348870in}{1.166162in}}%
\pgfpathcurveto{\pgfqpoint{2.341166in}{1.173866in}}{\pgfqpoint{2.330715in}{1.178195in}}{\pgfqpoint{2.319819in}{1.178195in}}%
\pgfpathcurveto{\pgfqpoint{2.308924in}{1.178195in}}{\pgfqpoint{2.298473in}{1.173866in}}{\pgfqpoint{2.290769in}{1.166162in}}%
\pgfpathcurveto{\pgfqpoint{2.283064in}{1.158458in}}{\pgfqpoint{2.278735in}{1.148007in}}{\pgfqpoint{2.278735in}{1.137111in}}%
\pgfpathcurveto{\pgfqpoint{2.278735in}{1.126216in}}{\pgfqpoint{2.283064in}{1.115765in}}{\pgfqpoint{2.290769in}{1.108061in}}%
\pgfpathcurveto{\pgfqpoint{2.298473in}{1.100356in}}{\pgfqpoint{2.308924in}{1.096027in}}{\pgfqpoint{2.319819in}{1.096027in}}%
\pgfpathlineto{\pgfqpoint{2.319819in}{1.096027in}}%
\pgfpathclose%
\pgfusepath{stroke}%
\end{pgfscope}%
\begin{pgfscope}%
\pgfpathrectangle{\pgfqpoint{0.688192in}{0.670138in}}{\pgfqpoint{7.111808in}{5.061530in}}%
\pgfusepath{clip}%
\pgfsetbuttcap%
\pgfsetroundjoin%
\pgfsetlinewidth{1.003750pt}%
\definecolor{currentstroke}{rgb}{0.000000,0.000000,0.000000}%
\pgfsetstrokecolor{currentstroke}%
\pgfsetdash{}{0pt}%
\pgfpathmoveto{\pgfqpoint{2.244939in}{0.638531in}}%
\pgfpathcurveto{\pgfqpoint{2.255835in}{0.638531in}}{\pgfqpoint{2.266286in}{0.642860in}}{\pgfqpoint{2.273990in}{0.650564in}}%
\pgfpathcurveto{\pgfqpoint{2.281695in}{0.658269in}}{\pgfqpoint{2.286023in}{0.668719in}}{\pgfqpoint{2.286023in}{0.679615in}}%
\pgfpathcurveto{\pgfqpoint{2.286023in}{0.690511in}}{\pgfqpoint{2.281695in}{0.700961in}}{\pgfqpoint{2.273990in}{0.708666in}}%
\pgfpathcurveto{\pgfqpoint{2.266286in}{0.716370in}}{\pgfqpoint{2.255835in}{0.720699in}}{\pgfqpoint{2.244939in}{0.720699in}}%
\pgfpathcurveto{\pgfqpoint{2.234044in}{0.720699in}}{\pgfqpoint{2.223593in}{0.716370in}}{\pgfqpoint{2.215889in}{0.708666in}}%
\pgfpathcurveto{\pgfqpoint{2.208184in}{0.700961in}}{\pgfqpoint{2.203856in}{0.690511in}}{\pgfqpoint{2.203856in}{0.679615in}}%
\pgfpathcurveto{\pgfqpoint{2.203856in}{0.668719in}}{\pgfqpoint{2.208184in}{0.658269in}}{\pgfqpoint{2.215889in}{0.650564in}}%
\pgfpathcurveto{\pgfqpoint{2.223593in}{0.642860in}}{\pgfqpoint{2.234044in}{0.638531in}}{\pgfqpoint{2.244939in}{0.638531in}}%
\pgfusepath{stroke}%
\end{pgfscope}%
\begin{pgfscope}%
\pgfpathrectangle{\pgfqpoint{0.688192in}{0.670138in}}{\pgfqpoint{7.111808in}{5.061530in}}%
\pgfusepath{clip}%
\pgfsetbuttcap%
\pgfsetroundjoin%
\pgfsetlinewidth{1.003750pt}%
\definecolor{currentstroke}{rgb}{0.000000,0.000000,0.000000}%
\pgfsetstrokecolor{currentstroke}%
\pgfsetdash{}{0pt}%
\pgfpathmoveto{\pgfqpoint{5.283066in}{0.629702in}}%
\pgfpathcurveto{\pgfqpoint{5.293961in}{0.629702in}}{\pgfqpoint{5.304412in}{0.634031in}}{\pgfqpoint{5.312116in}{0.641735in}}%
\pgfpathcurveto{\pgfqpoint{5.319821in}{0.649439in}}{\pgfqpoint{5.324149in}{0.659890in}}{\pgfqpoint{5.324149in}{0.670786in}}%
\pgfpathcurveto{\pgfqpoint{5.324149in}{0.681681in}}{\pgfqpoint{5.319821in}{0.692132in}}{\pgfqpoint{5.312116in}{0.699837in}}%
\pgfpathcurveto{\pgfqpoint{5.304412in}{0.707541in}}{\pgfqpoint{5.293961in}{0.711870in}}{\pgfqpoint{5.283066in}{0.711870in}}%
\pgfpathcurveto{\pgfqpoint{5.272170in}{0.711870in}}{\pgfqpoint{5.261719in}{0.707541in}}{\pgfqpoint{5.254015in}{0.699837in}}%
\pgfpathcurveto{\pgfqpoint{5.246311in}{0.692132in}}{\pgfqpoint{5.241982in}{0.681681in}}{\pgfqpoint{5.241982in}{0.670786in}}%
\pgfpathcurveto{\pgfqpoint{5.241982in}{0.659890in}}{\pgfqpoint{5.246311in}{0.649439in}}{\pgfqpoint{5.254015in}{0.641735in}}%
\pgfpathcurveto{\pgfqpoint{5.261719in}{0.634031in}}{\pgfqpoint{5.272170in}{0.629702in}}{\pgfqpoint{5.283066in}{0.629702in}}%
\pgfusepath{stroke}%
\end{pgfscope}%
\begin{pgfscope}%
\pgfpathrectangle{\pgfqpoint{0.688192in}{0.670138in}}{\pgfqpoint{7.111808in}{5.061530in}}%
\pgfusepath{clip}%
\pgfsetbuttcap%
\pgfsetmiterjoin%
\definecolor{currentfill}{rgb}{0.839216,0.152941,0.156863}%
\pgfsetfillcolor{currentfill}%
\pgfsetfillopacity{0.200000}%
\pgfsetlinewidth{1.003750pt}%
\definecolor{currentstroke}{rgb}{0.839216,0.152941,0.156863}%
\pgfsetstrokecolor{currentstroke}%
\pgfsetstrokeopacity{0.200000}%
\pgfsetdash{}{0pt}%
\pgfpathmoveto{\pgfqpoint{0.688192in}{0.670138in}}%
\pgfpathlineto{\pgfqpoint{1.952178in}{0.670138in}}%
\pgfpathlineto{\pgfqpoint{1.952178in}{5.731668in}}%
\pgfpathlineto{\pgfqpoint{0.688192in}{5.731668in}}%
\pgfpathlineto{\pgfqpoint{0.688192in}{0.670138in}}%
\pgfpathclose%
\pgfusepath{stroke,fill}%
\end{pgfscope}%
\begin{pgfscope}%
\pgfsetbuttcap%
\pgfsetmiterjoin%
\definecolor{currentfill}{rgb}{0.839216,0.152941,0.156863}%
\pgfsetfillcolor{currentfill}%
\pgfsetfillopacity{0.200000}%
\pgfsetlinewidth{1.003750pt}%
\definecolor{currentstroke}{rgb}{0.839216,0.152941,0.156863}%
\pgfsetstrokecolor{currentstroke}%
\pgfsetstrokeopacity{0.200000}%
\pgfsetdash{}{0pt}%
\pgfpathrectangle{\pgfqpoint{0.688192in}{0.670138in}}{\pgfqpoint{7.111808in}{5.061530in}}%
\pgfusepath{clip}%
\pgfpathmoveto{\pgfqpoint{0.688192in}{0.670138in}}%
\pgfpathlineto{\pgfqpoint{1.952178in}{0.670138in}}%
\pgfpathlineto{\pgfqpoint{1.952178in}{5.731668in}}%
\pgfpathlineto{\pgfqpoint{0.688192in}{5.731668in}}%
\pgfpathlineto{\pgfqpoint{0.688192in}{0.670138in}}%
\pgfpathclose%
\pgfusepath{clip}%
\pgfsys@defobject{currentpattern}{\pgfqpoint{0in}{0in}}{\pgfqpoint{1in}{1in}}{%
\begin{pgfscope}%
\pgfpathrectangle{\pgfqpoint{0in}{0in}}{\pgfqpoint{1in}{1in}}%
\pgfusepath{clip}%
\pgfpathmoveto{\pgfqpoint{-0.500000in}{0.500000in}}%
\pgfpathlineto{\pgfqpoint{0.500000in}{1.500000in}}%
\pgfpathmoveto{\pgfqpoint{-0.333333in}{0.333333in}}%
\pgfpathlineto{\pgfqpoint{0.666667in}{1.333333in}}%
\pgfpathmoveto{\pgfqpoint{-0.166667in}{0.166667in}}%
\pgfpathlineto{\pgfqpoint{0.833333in}{1.166667in}}%
\pgfpathmoveto{\pgfqpoint{0.000000in}{0.000000in}}%
\pgfpathlineto{\pgfqpoint{1.000000in}{1.000000in}}%
\pgfpathmoveto{\pgfqpoint{0.166667in}{-0.166667in}}%
\pgfpathlineto{\pgfqpoint{1.166667in}{0.833333in}}%
\pgfpathmoveto{\pgfqpoint{0.333333in}{-0.333333in}}%
\pgfpathlineto{\pgfqpoint{1.333333in}{0.666667in}}%
\pgfpathmoveto{\pgfqpoint{0.500000in}{-0.500000in}}%
\pgfpathlineto{\pgfqpoint{1.500000in}{0.500000in}}%
\pgfusepath{stroke}%
\end{pgfscope}%
}%
\pgfsys@transformshift{0.688192in}{0.670138in}%
\pgfsys@useobject{currentpattern}{}%
\pgfsys@transformshift{1in}{0in}%
\pgfsys@useobject{currentpattern}{}%
\pgfsys@transformshift{1in}{0in}%
\pgfsys@transformshift{-2in}{0in}%
\pgfsys@transformshift{0in}{1in}%
\pgfsys@useobject{currentpattern}{}%
\pgfsys@transformshift{1in}{0in}%
\pgfsys@useobject{currentpattern}{}%
\pgfsys@transformshift{1in}{0in}%
\pgfsys@transformshift{-2in}{0in}%
\pgfsys@transformshift{0in}{1in}%
\pgfsys@useobject{currentpattern}{}%
\pgfsys@transformshift{1in}{0in}%
\pgfsys@useobject{currentpattern}{}%
\pgfsys@transformshift{1in}{0in}%
\pgfsys@transformshift{-2in}{0in}%
\pgfsys@transformshift{0in}{1in}%
\pgfsys@useobject{currentpattern}{}%
\pgfsys@transformshift{1in}{0in}%
\pgfsys@useobject{currentpattern}{}%
\pgfsys@transformshift{1in}{0in}%
\pgfsys@transformshift{-2in}{0in}%
\pgfsys@transformshift{0in}{1in}%
\pgfsys@useobject{currentpattern}{}%
\pgfsys@transformshift{1in}{0in}%
\pgfsys@useobject{currentpattern}{}%
\pgfsys@transformshift{1in}{0in}%
\pgfsys@transformshift{-2in}{0in}%
\pgfsys@transformshift{0in}{1in}%
\pgfsys@useobject{currentpattern}{}%
\pgfsys@transformshift{1in}{0in}%
\pgfsys@useobject{currentpattern}{}%
\pgfsys@transformshift{1in}{0in}%
\pgfsys@transformshift{-2in}{0in}%
\pgfsys@transformshift{0in}{1in}%
\end{pgfscope}%
\begin{pgfscope}%
\pgfpathrectangle{\pgfqpoint{0.688192in}{0.670138in}}{\pgfqpoint{7.111808in}{5.061530in}}%
\pgfusepath{clip}%
\pgfsetrectcap%
\pgfsetroundjoin%
\pgfsetlinewidth{0.803000pt}%
\definecolor{currentstroke}{rgb}{0.690196,0.690196,0.690196}%
\pgfsetstrokecolor{currentstroke}%
\pgfsetdash{}{0pt}%
\pgfpathmoveto{\pgfqpoint{1.186342in}{0.670138in}}%
\pgfpathlineto{\pgfqpoint{1.186342in}{5.731668in}}%
\pgfusepath{stroke}%
\end{pgfscope}%
\begin{pgfscope}%
\pgfsetbuttcap%
\pgfsetroundjoin%
\definecolor{currentfill}{rgb}{0.000000,0.000000,0.000000}%
\pgfsetfillcolor{currentfill}%
\pgfsetlinewidth{0.803000pt}%
\definecolor{currentstroke}{rgb}{0.000000,0.000000,0.000000}%
\pgfsetstrokecolor{currentstroke}%
\pgfsetdash{}{0pt}%
\pgfsys@defobject{currentmarker}{\pgfqpoint{0.000000in}{-0.048611in}}{\pgfqpoint{0.000000in}{0.000000in}}{%
\pgfpathmoveto{\pgfqpoint{0.000000in}{0.000000in}}%
\pgfpathlineto{\pgfqpoint{0.000000in}{-0.048611in}}%
\pgfusepath{stroke,fill}%
}%
\begin{pgfscope}%
\pgfsys@transformshift{1.186342in}{0.670138in}%
\pgfsys@useobject{currentmarker}{}%
\end{pgfscope}%
\end{pgfscope}%
\begin{pgfscope}%
\definecolor{textcolor}{rgb}{0.000000,0.000000,0.000000}%
\pgfsetstrokecolor{textcolor}%
\pgfsetfillcolor{textcolor}%
\pgftext[x=1.186342in,y=0.572916in,,top]{\color{textcolor}{\rmfamily\fontsize{14.000000}{16.800000}\selectfont\catcode`\^=\active\def^{\ifmmode\sp\else\^{}\fi}\catcode`\%=\active\def%{\%}$\mathdefault{5500}$}}%
\end{pgfscope}%
\begin{pgfscope}%
\pgfpathrectangle{\pgfqpoint{0.688192in}{0.670138in}}{\pgfqpoint{7.111808in}{5.061530in}}%
\pgfusepath{clip}%
\pgfsetrectcap%
\pgfsetroundjoin%
\pgfsetlinewidth{0.803000pt}%
\definecolor{currentstroke}{rgb}{0.690196,0.690196,0.690196}%
\pgfsetstrokecolor{currentstroke}%
\pgfsetdash{}{0pt}%
\pgfpathmoveto{\pgfqpoint{2.380707in}{0.670138in}}%
\pgfpathlineto{\pgfqpoint{2.380707in}{5.731668in}}%
\pgfusepath{stroke}%
\end{pgfscope}%
\begin{pgfscope}%
\pgfsetbuttcap%
\pgfsetroundjoin%
\definecolor{currentfill}{rgb}{0.000000,0.000000,0.000000}%
\pgfsetfillcolor{currentfill}%
\pgfsetlinewidth{0.803000pt}%
\definecolor{currentstroke}{rgb}{0.000000,0.000000,0.000000}%
\pgfsetstrokecolor{currentstroke}%
\pgfsetdash{}{0pt}%
\pgfsys@defobject{currentmarker}{\pgfqpoint{0.000000in}{-0.048611in}}{\pgfqpoint{0.000000in}{0.000000in}}{%
\pgfpathmoveto{\pgfqpoint{0.000000in}{0.000000in}}%
\pgfpathlineto{\pgfqpoint{0.000000in}{-0.048611in}}%
\pgfusepath{stroke,fill}%
}%
\begin{pgfscope}%
\pgfsys@transformshift{2.380707in}{0.670138in}%
\pgfsys@useobject{currentmarker}{}%
\end{pgfscope}%
\end{pgfscope}%
\begin{pgfscope}%
\definecolor{textcolor}{rgb}{0.000000,0.000000,0.000000}%
\pgfsetstrokecolor{textcolor}%
\pgfsetfillcolor{textcolor}%
\pgftext[x=2.380707in,y=0.572916in,,top]{\color{textcolor}{\rmfamily\fontsize{14.000000}{16.800000}\selectfont\catcode`\^=\active\def^{\ifmmode\sp\else\^{}\fi}\catcode`\%=\active\def%{\%}$\mathdefault{6000}$}}%
\end{pgfscope}%
\begin{pgfscope}%
\pgfpathrectangle{\pgfqpoint{0.688192in}{0.670138in}}{\pgfqpoint{7.111808in}{5.061530in}}%
\pgfusepath{clip}%
\pgfsetrectcap%
\pgfsetroundjoin%
\pgfsetlinewidth{0.803000pt}%
\definecolor{currentstroke}{rgb}{0.690196,0.690196,0.690196}%
\pgfsetstrokecolor{currentstroke}%
\pgfsetdash{}{0pt}%
\pgfpathmoveto{\pgfqpoint{3.575072in}{0.670138in}}%
\pgfpathlineto{\pgfqpoint{3.575072in}{5.731668in}}%
\pgfusepath{stroke}%
\end{pgfscope}%
\begin{pgfscope}%
\pgfsetbuttcap%
\pgfsetroundjoin%
\definecolor{currentfill}{rgb}{0.000000,0.000000,0.000000}%
\pgfsetfillcolor{currentfill}%
\pgfsetlinewidth{0.803000pt}%
\definecolor{currentstroke}{rgb}{0.000000,0.000000,0.000000}%
\pgfsetstrokecolor{currentstroke}%
\pgfsetdash{}{0pt}%
\pgfsys@defobject{currentmarker}{\pgfqpoint{0.000000in}{-0.048611in}}{\pgfqpoint{0.000000in}{0.000000in}}{%
\pgfpathmoveto{\pgfqpoint{0.000000in}{0.000000in}}%
\pgfpathlineto{\pgfqpoint{0.000000in}{-0.048611in}}%
\pgfusepath{stroke,fill}%
}%
\begin{pgfscope}%
\pgfsys@transformshift{3.575072in}{0.670138in}%
\pgfsys@useobject{currentmarker}{}%
\end{pgfscope}%
\end{pgfscope}%
\begin{pgfscope}%
\definecolor{textcolor}{rgb}{0.000000,0.000000,0.000000}%
\pgfsetstrokecolor{textcolor}%
\pgfsetfillcolor{textcolor}%
\pgftext[x=3.575072in,y=0.572916in,,top]{\color{textcolor}{\rmfamily\fontsize{14.000000}{16.800000}\selectfont\catcode`\^=\active\def^{\ifmmode\sp\else\^{}\fi}\catcode`\%=\active\def%{\%}$\mathdefault{6500}$}}%
\end{pgfscope}%
\begin{pgfscope}%
\pgfpathrectangle{\pgfqpoint{0.688192in}{0.670138in}}{\pgfqpoint{7.111808in}{5.061530in}}%
\pgfusepath{clip}%
\pgfsetrectcap%
\pgfsetroundjoin%
\pgfsetlinewidth{0.803000pt}%
\definecolor{currentstroke}{rgb}{0.690196,0.690196,0.690196}%
\pgfsetstrokecolor{currentstroke}%
\pgfsetdash{}{0pt}%
\pgfpathmoveto{\pgfqpoint{4.769438in}{0.670138in}}%
\pgfpathlineto{\pgfqpoint{4.769438in}{5.731668in}}%
\pgfusepath{stroke}%
\end{pgfscope}%
\begin{pgfscope}%
\pgfsetbuttcap%
\pgfsetroundjoin%
\definecolor{currentfill}{rgb}{0.000000,0.000000,0.000000}%
\pgfsetfillcolor{currentfill}%
\pgfsetlinewidth{0.803000pt}%
\definecolor{currentstroke}{rgb}{0.000000,0.000000,0.000000}%
\pgfsetstrokecolor{currentstroke}%
\pgfsetdash{}{0pt}%
\pgfsys@defobject{currentmarker}{\pgfqpoint{0.000000in}{-0.048611in}}{\pgfqpoint{0.000000in}{0.000000in}}{%
\pgfpathmoveto{\pgfqpoint{0.000000in}{0.000000in}}%
\pgfpathlineto{\pgfqpoint{0.000000in}{-0.048611in}}%
\pgfusepath{stroke,fill}%
}%
\begin{pgfscope}%
\pgfsys@transformshift{4.769438in}{0.670138in}%
\pgfsys@useobject{currentmarker}{}%
\end{pgfscope}%
\end{pgfscope}%
\begin{pgfscope}%
\definecolor{textcolor}{rgb}{0.000000,0.000000,0.000000}%
\pgfsetstrokecolor{textcolor}%
\pgfsetfillcolor{textcolor}%
\pgftext[x=4.769438in,y=0.572916in,,top]{\color{textcolor}{\rmfamily\fontsize{14.000000}{16.800000}\selectfont\catcode`\^=\active\def^{\ifmmode\sp\else\^{}\fi}\catcode`\%=\active\def%{\%}$\mathdefault{7000}$}}%
\end{pgfscope}%
\begin{pgfscope}%
\pgfpathrectangle{\pgfqpoint{0.688192in}{0.670138in}}{\pgfqpoint{7.111808in}{5.061530in}}%
\pgfusepath{clip}%
\pgfsetrectcap%
\pgfsetroundjoin%
\pgfsetlinewidth{0.803000pt}%
\definecolor{currentstroke}{rgb}{0.690196,0.690196,0.690196}%
\pgfsetstrokecolor{currentstroke}%
\pgfsetdash{}{0pt}%
\pgfpathmoveto{\pgfqpoint{5.963803in}{0.670138in}}%
\pgfpathlineto{\pgfqpoint{5.963803in}{5.731668in}}%
\pgfusepath{stroke}%
\end{pgfscope}%
\begin{pgfscope}%
\pgfsetbuttcap%
\pgfsetroundjoin%
\definecolor{currentfill}{rgb}{0.000000,0.000000,0.000000}%
\pgfsetfillcolor{currentfill}%
\pgfsetlinewidth{0.803000pt}%
\definecolor{currentstroke}{rgb}{0.000000,0.000000,0.000000}%
\pgfsetstrokecolor{currentstroke}%
\pgfsetdash{}{0pt}%
\pgfsys@defobject{currentmarker}{\pgfqpoint{0.000000in}{-0.048611in}}{\pgfqpoint{0.000000in}{0.000000in}}{%
\pgfpathmoveto{\pgfqpoint{0.000000in}{0.000000in}}%
\pgfpathlineto{\pgfqpoint{0.000000in}{-0.048611in}}%
\pgfusepath{stroke,fill}%
}%
\begin{pgfscope}%
\pgfsys@transformshift{5.963803in}{0.670138in}%
\pgfsys@useobject{currentmarker}{}%
\end{pgfscope}%
\end{pgfscope}%
\begin{pgfscope}%
\definecolor{textcolor}{rgb}{0.000000,0.000000,0.000000}%
\pgfsetstrokecolor{textcolor}%
\pgfsetfillcolor{textcolor}%
\pgftext[x=5.963803in,y=0.572916in,,top]{\color{textcolor}{\rmfamily\fontsize{14.000000}{16.800000}\selectfont\catcode`\^=\active\def^{\ifmmode\sp\else\^{}\fi}\catcode`\%=\active\def%{\%}$\mathdefault{7500}$}}%
\end{pgfscope}%
\begin{pgfscope}%
\pgfpathrectangle{\pgfqpoint{0.688192in}{0.670138in}}{\pgfqpoint{7.111808in}{5.061530in}}%
\pgfusepath{clip}%
\pgfsetrectcap%
\pgfsetroundjoin%
\pgfsetlinewidth{0.803000pt}%
\definecolor{currentstroke}{rgb}{0.690196,0.690196,0.690196}%
\pgfsetstrokecolor{currentstroke}%
\pgfsetdash{}{0pt}%
\pgfpathmoveto{\pgfqpoint{7.158168in}{0.670138in}}%
\pgfpathlineto{\pgfqpoint{7.158168in}{5.731668in}}%
\pgfusepath{stroke}%
\end{pgfscope}%
\begin{pgfscope}%
\pgfsetbuttcap%
\pgfsetroundjoin%
\definecolor{currentfill}{rgb}{0.000000,0.000000,0.000000}%
\pgfsetfillcolor{currentfill}%
\pgfsetlinewidth{0.803000pt}%
\definecolor{currentstroke}{rgb}{0.000000,0.000000,0.000000}%
\pgfsetstrokecolor{currentstroke}%
\pgfsetdash{}{0pt}%
\pgfsys@defobject{currentmarker}{\pgfqpoint{0.000000in}{-0.048611in}}{\pgfqpoint{0.000000in}{0.000000in}}{%
\pgfpathmoveto{\pgfqpoint{0.000000in}{0.000000in}}%
\pgfpathlineto{\pgfqpoint{0.000000in}{-0.048611in}}%
\pgfusepath{stroke,fill}%
}%
\begin{pgfscope}%
\pgfsys@transformshift{7.158168in}{0.670138in}%
\pgfsys@useobject{currentmarker}{}%
\end{pgfscope}%
\end{pgfscope}%
\begin{pgfscope}%
\definecolor{textcolor}{rgb}{0.000000,0.000000,0.000000}%
\pgfsetstrokecolor{textcolor}%
\pgfsetfillcolor{textcolor}%
\pgftext[x=7.158168in,y=0.572916in,,top]{\color{textcolor}{\rmfamily\fontsize{14.000000}{16.800000}\selectfont\catcode`\^=\active\def^{\ifmmode\sp\else\^{}\fi}\catcode`\%=\active\def%{\%}$\mathdefault{8000}$}}%
\end{pgfscope}%
\begin{pgfscope}%
\definecolor{textcolor}{rgb}{0.000000,0.000000,0.000000}%
\pgfsetstrokecolor{textcolor}%
\pgfsetfillcolor{textcolor}%
\pgftext[x=4.244096in,y=0.339583in,,top]{\color{textcolor}{\rmfamily\fontsize{18.000000}{21.600000}\selectfont\catcode`\^=\active\def^{\ifmmode\sp\else\^{}\fi}\catcode`\%=\active\def%{\%}Total Cost [M\$]}}%
\end{pgfscope}%
\begin{pgfscope}%
\pgfpathrectangle{\pgfqpoint{0.688192in}{0.670138in}}{\pgfqpoint{7.111808in}{5.061530in}}%
\pgfusepath{clip}%
\pgfsetrectcap%
\pgfsetroundjoin%
\pgfsetlinewidth{0.803000pt}%
\definecolor{currentstroke}{rgb}{0.690196,0.690196,0.690196}%
\pgfsetstrokecolor{currentstroke}%
\pgfsetdash{}{0pt}%
\pgfpathmoveto{\pgfqpoint{0.688192in}{1.174861in}}%
\pgfpathlineto{\pgfqpoint{7.800000in}{1.174861in}}%
\pgfusepath{stroke}%
\end{pgfscope}%
\begin{pgfscope}%
\pgfsetbuttcap%
\pgfsetroundjoin%
\definecolor{currentfill}{rgb}{0.000000,0.000000,0.000000}%
\pgfsetfillcolor{currentfill}%
\pgfsetlinewidth{0.803000pt}%
\definecolor{currentstroke}{rgb}{0.000000,0.000000,0.000000}%
\pgfsetstrokecolor{currentstroke}%
\pgfsetdash{}{0pt}%
\pgfsys@defobject{currentmarker}{\pgfqpoint{-0.048611in}{0.000000in}}{\pgfqpoint{-0.000000in}{0.000000in}}{%
\pgfpathmoveto{\pgfqpoint{-0.000000in}{0.000000in}}%
\pgfpathlineto{\pgfqpoint{-0.048611in}{0.000000in}}%
\pgfusepath{stroke,fill}%
}%
\begin{pgfscope}%
\pgfsys@transformshift{0.688192in}{1.174861in}%
\pgfsys@useobject{currentmarker}{}%
\end{pgfscope}%
\end{pgfscope}%
\begin{pgfscope}%
\definecolor{textcolor}{rgb}{0.000000,0.000000,0.000000}%
\pgfsetstrokecolor{textcolor}%
\pgfsetfillcolor{textcolor}%
\pgftext[x=0.395138in, y=1.105417in, left, base]{\color{textcolor}{\rmfamily\fontsize{14.000000}{16.800000}\selectfont\catcode`\^=\active\def^{\ifmmode\sp\else\^{}\fi}\catcode`\%=\active\def%{\%}$\mathdefault{10}$}}%
\end{pgfscope}%
\begin{pgfscope}%
\pgfpathrectangle{\pgfqpoint{0.688192in}{0.670138in}}{\pgfqpoint{7.111808in}{5.061530in}}%
\pgfusepath{clip}%
\pgfsetrectcap%
\pgfsetroundjoin%
\pgfsetlinewidth{0.803000pt}%
\definecolor{currentstroke}{rgb}{0.690196,0.690196,0.690196}%
\pgfsetstrokecolor{currentstroke}%
\pgfsetdash{}{0pt}%
\pgfpathmoveto{\pgfqpoint{0.688192in}{1.825660in}}%
\pgfpathlineto{\pgfqpoint{7.800000in}{1.825660in}}%
\pgfusepath{stroke}%
\end{pgfscope}%
\begin{pgfscope}%
\pgfsetbuttcap%
\pgfsetroundjoin%
\definecolor{currentfill}{rgb}{0.000000,0.000000,0.000000}%
\pgfsetfillcolor{currentfill}%
\pgfsetlinewidth{0.803000pt}%
\definecolor{currentstroke}{rgb}{0.000000,0.000000,0.000000}%
\pgfsetstrokecolor{currentstroke}%
\pgfsetdash{}{0pt}%
\pgfsys@defobject{currentmarker}{\pgfqpoint{-0.048611in}{0.000000in}}{\pgfqpoint{-0.000000in}{0.000000in}}{%
\pgfpathmoveto{\pgfqpoint{-0.000000in}{0.000000in}}%
\pgfpathlineto{\pgfqpoint{-0.048611in}{0.000000in}}%
\pgfusepath{stroke,fill}%
}%
\begin{pgfscope}%
\pgfsys@transformshift{0.688192in}{1.825660in}%
\pgfsys@useobject{currentmarker}{}%
\end{pgfscope}%
\end{pgfscope}%
\begin{pgfscope}%
\definecolor{textcolor}{rgb}{0.000000,0.000000,0.000000}%
\pgfsetstrokecolor{textcolor}%
\pgfsetfillcolor{textcolor}%
\pgftext[x=0.395138in, y=1.756215in, left, base]{\color{textcolor}{\rmfamily\fontsize{14.000000}{16.800000}\selectfont\catcode`\^=\active\def^{\ifmmode\sp\else\^{}\fi}\catcode`\%=\active\def%{\%}$\mathdefault{20}$}}%
\end{pgfscope}%
\begin{pgfscope}%
\pgfpathrectangle{\pgfqpoint{0.688192in}{0.670138in}}{\pgfqpoint{7.111808in}{5.061530in}}%
\pgfusepath{clip}%
\pgfsetrectcap%
\pgfsetroundjoin%
\pgfsetlinewidth{0.803000pt}%
\definecolor{currentstroke}{rgb}{0.690196,0.690196,0.690196}%
\pgfsetstrokecolor{currentstroke}%
\pgfsetdash{}{0pt}%
\pgfpathmoveto{\pgfqpoint{0.688192in}{2.476458in}}%
\pgfpathlineto{\pgfqpoint{7.800000in}{2.476458in}}%
\pgfusepath{stroke}%
\end{pgfscope}%
\begin{pgfscope}%
\pgfsetbuttcap%
\pgfsetroundjoin%
\definecolor{currentfill}{rgb}{0.000000,0.000000,0.000000}%
\pgfsetfillcolor{currentfill}%
\pgfsetlinewidth{0.803000pt}%
\definecolor{currentstroke}{rgb}{0.000000,0.000000,0.000000}%
\pgfsetstrokecolor{currentstroke}%
\pgfsetdash{}{0pt}%
\pgfsys@defobject{currentmarker}{\pgfqpoint{-0.048611in}{0.000000in}}{\pgfqpoint{-0.000000in}{0.000000in}}{%
\pgfpathmoveto{\pgfqpoint{-0.000000in}{0.000000in}}%
\pgfpathlineto{\pgfqpoint{-0.048611in}{0.000000in}}%
\pgfusepath{stroke,fill}%
}%
\begin{pgfscope}%
\pgfsys@transformshift{0.688192in}{2.476458in}%
\pgfsys@useobject{currentmarker}{}%
\end{pgfscope}%
\end{pgfscope}%
\begin{pgfscope}%
\definecolor{textcolor}{rgb}{0.000000,0.000000,0.000000}%
\pgfsetstrokecolor{textcolor}%
\pgfsetfillcolor{textcolor}%
\pgftext[x=0.395138in, y=2.407014in, left, base]{\color{textcolor}{\rmfamily\fontsize{14.000000}{16.800000}\selectfont\catcode`\^=\active\def^{\ifmmode\sp\else\^{}\fi}\catcode`\%=\active\def%{\%}$\mathdefault{30}$}}%
\end{pgfscope}%
\begin{pgfscope}%
\pgfpathrectangle{\pgfqpoint{0.688192in}{0.670138in}}{\pgfqpoint{7.111808in}{5.061530in}}%
\pgfusepath{clip}%
\pgfsetrectcap%
\pgfsetroundjoin%
\pgfsetlinewidth{0.803000pt}%
\definecolor{currentstroke}{rgb}{0.690196,0.690196,0.690196}%
\pgfsetstrokecolor{currentstroke}%
\pgfsetdash{}{0pt}%
\pgfpathmoveto{\pgfqpoint{0.688192in}{3.127256in}}%
\pgfpathlineto{\pgfqpoint{7.800000in}{3.127256in}}%
\pgfusepath{stroke}%
\end{pgfscope}%
\begin{pgfscope}%
\pgfsetbuttcap%
\pgfsetroundjoin%
\definecolor{currentfill}{rgb}{0.000000,0.000000,0.000000}%
\pgfsetfillcolor{currentfill}%
\pgfsetlinewidth{0.803000pt}%
\definecolor{currentstroke}{rgb}{0.000000,0.000000,0.000000}%
\pgfsetstrokecolor{currentstroke}%
\pgfsetdash{}{0pt}%
\pgfsys@defobject{currentmarker}{\pgfqpoint{-0.048611in}{0.000000in}}{\pgfqpoint{-0.000000in}{0.000000in}}{%
\pgfpathmoveto{\pgfqpoint{-0.000000in}{0.000000in}}%
\pgfpathlineto{\pgfqpoint{-0.048611in}{0.000000in}}%
\pgfusepath{stroke,fill}%
}%
\begin{pgfscope}%
\pgfsys@transformshift{0.688192in}{3.127256in}%
\pgfsys@useobject{currentmarker}{}%
\end{pgfscope}%
\end{pgfscope}%
\begin{pgfscope}%
\definecolor{textcolor}{rgb}{0.000000,0.000000,0.000000}%
\pgfsetstrokecolor{textcolor}%
\pgfsetfillcolor{textcolor}%
\pgftext[x=0.395138in, y=3.057812in, left, base]{\color{textcolor}{\rmfamily\fontsize{14.000000}{16.800000}\selectfont\catcode`\^=\active\def^{\ifmmode\sp\else\^{}\fi}\catcode`\%=\active\def%{\%}$\mathdefault{40}$}}%
\end{pgfscope}%
\begin{pgfscope}%
\pgfpathrectangle{\pgfqpoint{0.688192in}{0.670138in}}{\pgfqpoint{7.111808in}{5.061530in}}%
\pgfusepath{clip}%
\pgfsetrectcap%
\pgfsetroundjoin%
\pgfsetlinewidth{0.803000pt}%
\definecolor{currentstroke}{rgb}{0.690196,0.690196,0.690196}%
\pgfsetstrokecolor{currentstroke}%
\pgfsetdash{}{0pt}%
\pgfpathmoveto{\pgfqpoint{0.688192in}{3.778054in}}%
\pgfpathlineto{\pgfqpoint{7.800000in}{3.778054in}}%
\pgfusepath{stroke}%
\end{pgfscope}%
\begin{pgfscope}%
\pgfsetbuttcap%
\pgfsetroundjoin%
\definecolor{currentfill}{rgb}{0.000000,0.000000,0.000000}%
\pgfsetfillcolor{currentfill}%
\pgfsetlinewidth{0.803000pt}%
\definecolor{currentstroke}{rgb}{0.000000,0.000000,0.000000}%
\pgfsetstrokecolor{currentstroke}%
\pgfsetdash{}{0pt}%
\pgfsys@defobject{currentmarker}{\pgfqpoint{-0.048611in}{0.000000in}}{\pgfqpoint{-0.000000in}{0.000000in}}{%
\pgfpathmoveto{\pgfqpoint{-0.000000in}{0.000000in}}%
\pgfpathlineto{\pgfqpoint{-0.048611in}{0.000000in}}%
\pgfusepath{stroke,fill}%
}%
\begin{pgfscope}%
\pgfsys@transformshift{0.688192in}{3.778054in}%
\pgfsys@useobject{currentmarker}{}%
\end{pgfscope}%
\end{pgfscope}%
\begin{pgfscope}%
\definecolor{textcolor}{rgb}{0.000000,0.000000,0.000000}%
\pgfsetstrokecolor{textcolor}%
\pgfsetfillcolor{textcolor}%
\pgftext[x=0.395138in, y=3.708610in, left, base]{\color{textcolor}{\rmfamily\fontsize{14.000000}{16.800000}\selectfont\catcode`\^=\active\def^{\ifmmode\sp\else\^{}\fi}\catcode`\%=\active\def%{\%}$\mathdefault{50}$}}%
\end{pgfscope}%
\begin{pgfscope}%
\pgfpathrectangle{\pgfqpoint{0.688192in}{0.670138in}}{\pgfqpoint{7.111808in}{5.061530in}}%
\pgfusepath{clip}%
\pgfsetrectcap%
\pgfsetroundjoin%
\pgfsetlinewidth{0.803000pt}%
\definecolor{currentstroke}{rgb}{0.690196,0.690196,0.690196}%
\pgfsetstrokecolor{currentstroke}%
\pgfsetdash{}{0pt}%
\pgfpathmoveto{\pgfqpoint{0.688192in}{4.428853in}}%
\pgfpathlineto{\pgfqpoint{7.800000in}{4.428853in}}%
\pgfusepath{stroke}%
\end{pgfscope}%
\begin{pgfscope}%
\pgfsetbuttcap%
\pgfsetroundjoin%
\definecolor{currentfill}{rgb}{0.000000,0.000000,0.000000}%
\pgfsetfillcolor{currentfill}%
\pgfsetlinewidth{0.803000pt}%
\definecolor{currentstroke}{rgb}{0.000000,0.000000,0.000000}%
\pgfsetstrokecolor{currentstroke}%
\pgfsetdash{}{0pt}%
\pgfsys@defobject{currentmarker}{\pgfqpoint{-0.048611in}{0.000000in}}{\pgfqpoint{-0.000000in}{0.000000in}}{%
\pgfpathmoveto{\pgfqpoint{-0.000000in}{0.000000in}}%
\pgfpathlineto{\pgfqpoint{-0.048611in}{0.000000in}}%
\pgfusepath{stroke,fill}%
}%
\begin{pgfscope}%
\pgfsys@transformshift{0.688192in}{4.428853in}%
\pgfsys@useobject{currentmarker}{}%
\end{pgfscope}%
\end{pgfscope}%
\begin{pgfscope}%
\definecolor{textcolor}{rgb}{0.000000,0.000000,0.000000}%
\pgfsetstrokecolor{textcolor}%
\pgfsetfillcolor{textcolor}%
\pgftext[x=0.395138in, y=4.359408in, left, base]{\color{textcolor}{\rmfamily\fontsize{14.000000}{16.800000}\selectfont\catcode`\^=\active\def^{\ifmmode\sp\else\^{}\fi}\catcode`\%=\active\def%{\%}$\mathdefault{60}$}}%
\end{pgfscope}%
\begin{pgfscope}%
\pgfpathrectangle{\pgfqpoint{0.688192in}{0.670138in}}{\pgfqpoint{7.111808in}{5.061530in}}%
\pgfusepath{clip}%
\pgfsetrectcap%
\pgfsetroundjoin%
\pgfsetlinewidth{0.803000pt}%
\definecolor{currentstroke}{rgb}{0.690196,0.690196,0.690196}%
\pgfsetstrokecolor{currentstroke}%
\pgfsetdash{}{0pt}%
\pgfpathmoveto{\pgfqpoint{0.688192in}{5.079651in}}%
\pgfpathlineto{\pgfqpoint{7.800000in}{5.079651in}}%
\pgfusepath{stroke}%
\end{pgfscope}%
\begin{pgfscope}%
\pgfsetbuttcap%
\pgfsetroundjoin%
\definecolor{currentfill}{rgb}{0.000000,0.000000,0.000000}%
\pgfsetfillcolor{currentfill}%
\pgfsetlinewidth{0.803000pt}%
\definecolor{currentstroke}{rgb}{0.000000,0.000000,0.000000}%
\pgfsetstrokecolor{currentstroke}%
\pgfsetdash{}{0pt}%
\pgfsys@defobject{currentmarker}{\pgfqpoint{-0.048611in}{0.000000in}}{\pgfqpoint{-0.000000in}{0.000000in}}{%
\pgfpathmoveto{\pgfqpoint{-0.000000in}{0.000000in}}%
\pgfpathlineto{\pgfqpoint{-0.048611in}{0.000000in}}%
\pgfusepath{stroke,fill}%
}%
\begin{pgfscope}%
\pgfsys@transformshift{0.688192in}{5.079651in}%
\pgfsys@useobject{currentmarker}{}%
\end{pgfscope}%
\end{pgfscope}%
\begin{pgfscope}%
\definecolor{textcolor}{rgb}{0.000000,0.000000,0.000000}%
\pgfsetstrokecolor{textcolor}%
\pgfsetfillcolor{textcolor}%
\pgftext[x=0.395138in, y=5.010207in, left, base]{\color{textcolor}{\rmfamily\fontsize{14.000000}{16.800000}\selectfont\catcode`\^=\active\def^{\ifmmode\sp\else\^{}\fi}\catcode`\%=\active\def%{\%}$\mathdefault{70}$}}%
\end{pgfscope}%
\begin{pgfscope}%
\pgfpathrectangle{\pgfqpoint{0.688192in}{0.670138in}}{\pgfqpoint{7.111808in}{5.061530in}}%
\pgfusepath{clip}%
\pgfsetrectcap%
\pgfsetroundjoin%
\pgfsetlinewidth{0.803000pt}%
\definecolor{currentstroke}{rgb}{0.690196,0.690196,0.690196}%
\pgfsetstrokecolor{currentstroke}%
\pgfsetdash{}{0pt}%
\pgfpathmoveto{\pgfqpoint{0.688192in}{5.730449in}}%
\pgfpathlineto{\pgfqpoint{7.800000in}{5.730449in}}%
\pgfusepath{stroke}%
\end{pgfscope}%
\begin{pgfscope}%
\pgfsetbuttcap%
\pgfsetroundjoin%
\definecolor{currentfill}{rgb}{0.000000,0.000000,0.000000}%
\pgfsetfillcolor{currentfill}%
\pgfsetlinewidth{0.803000pt}%
\definecolor{currentstroke}{rgb}{0.000000,0.000000,0.000000}%
\pgfsetstrokecolor{currentstroke}%
\pgfsetdash{}{0pt}%
\pgfsys@defobject{currentmarker}{\pgfqpoint{-0.048611in}{0.000000in}}{\pgfqpoint{-0.000000in}{0.000000in}}{%
\pgfpathmoveto{\pgfqpoint{-0.000000in}{0.000000in}}%
\pgfpathlineto{\pgfqpoint{-0.048611in}{0.000000in}}%
\pgfusepath{stroke,fill}%
}%
\begin{pgfscope}%
\pgfsys@transformshift{0.688192in}{5.730449in}%
\pgfsys@useobject{currentmarker}{}%
\end{pgfscope}%
\end{pgfscope}%
\begin{pgfscope}%
\definecolor{textcolor}{rgb}{0.000000,0.000000,0.000000}%
\pgfsetstrokecolor{textcolor}%
\pgfsetfillcolor{textcolor}%
\pgftext[x=0.395138in, y=5.661005in, left, base]{\color{textcolor}{\rmfamily\fontsize{14.000000}{16.800000}\selectfont\catcode`\^=\active\def^{\ifmmode\sp\else\^{}\fi}\catcode`\%=\active\def%{\%}$\mathdefault{80}$}}%
\end{pgfscope}%
\begin{pgfscope}%
\definecolor{textcolor}{rgb}{0.000000,0.000000,0.000000}%
\pgfsetstrokecolor{textcolor}%
\pgfsetfillcolor{textcolor}%
\pgftext[x=0.339583in,y=3.200903in,,bottom,rotate=90.000000]{\color{textcolor}{\rmfamily\fontsize{18.000000}{21.600000}\selectfont\catcode`\^=\active\def^{\ifmmode\sp\else\^{}\fi}\catcode`\%=\active\def%{\%}Emissions [MT CO$_2$eq]}}%
\end{pgfscope}%
\begin{pgfscope}%
\pgfpathrectangle{\pgfqpoint{0.688192in}{0.670138in}}{\pgfqpoint{7.111808in}{5.061530in}}%
\pgfusepath{clip}%
\pgfsetrectcap%
\pgfsetroundjoin%
\pgfsetlinewidth{1.505625pt}%
\definecolor{currentstroke}{rgb}{0.000000,0.000000,0.000000}%
\pgfsetstrokecolor{currentstroke}%
\pgfsetdash{}{0pt}%
\pgfpathmoveto{\pgfqpoint{0.749254in}{1.445149in}}%
\pgfpathlineto{\pgfqpoint{0.752983in}{1.252761in}}%
\pgfpathlineto{\pgfqpoint{0.754090in}{1.154399in}}%
\pgfpathlineto{\pgfqpoint{0.760839in}{1.136279in}}%
\pgfpathlineto{\pgfqpoint{0.764625in}{1.015211in}}%
\pgfpathlineto{\pgfqpoint{0.769073in}{0.982259in}}%
\pgfpathlineto{\pgfqpoint{0.771295in}{0.982065in}}%
\pgfpathlineto{\pgfqpoint{0.776750in}{0.941782in}}%
\pgfpathlineto{\pgfqpoint{0.777016in}{0.919143in}}%
\pgfpathlineto{\pgfqpoint{0.779964in}{0.894614in}}%
\pgfpathlineto{\pgfqpoint{0.781884in}{0.888672in}}%
\pgfpathlineto{\pgfqpoint{0.782912in}{0.882838in}}%
\pgfpathlineto{\pgfqpoint{0.783279in}{0.849857in}}%
\pgfpathlineto{\pgfqpoint{0.787475in}{0.843942in}}%
\pgfpathlineto{\pgfqpoint{0.791486in}{0.837055in}}%
\pgfpathlineto{\pgfqpoint{0.793310in}{0.809534in}}%
\pgfpathlineto{\pgfqpoint{0.795978in}{0.801029in}}%
\pgfpathlineto{\pgfqpoint{0.796157in}{0.793846in}}%
\pgfpathlineto{\pgfqpoint{0.800976in}{0.790369in}}%
\pgfpathlineto{\pgfqpoint{0.807883in}{0.789563in}}%
\pgfpathlineto{\pgfqpoint{0.810327in}{0.766458in}}%
\pgfpathlineto{\pgfqpoint{0.816256in}{0.760843in}}%
\pgfpathlineto{\pgfqpoint{0.827191in}{0.759800in}}%
\pgfpathlineto{\pgfqpoint{0.830930in}{0.751031in}}%
\pgfpathlineto{\pgfqpoint{0.837359in}{0.748345in}}%
\pgfpathlineto{\pgfqpoint{0.842811in}{0.745008in}}%
\pgfpathlineto{\pgfqpoint{0.849436in}{0.740184in}}%
\pgfpathlineto{\pgfqpoint{0.850270in}{0.738398in}}%
\pgfpathlineto{\pgfqpoint{0.854347in}{0.736661in}}%
\pgfpathlineto{\pgfqpoint{0.872029in}{0.731700in}}%
\pgfpathlineto{\pgfqpoint{0.890835in}{0.724786in}}%
\pgfpathlineto{\pgfqpoint{0.916357in}{0.719853in}}%
\pgfpathlineto{\pgfqpoint{0.939729in}{0.718197in}}%
\pgfpathlineto{\pgfqpoint{0.942062in}{0.712194in}}%
\pgfpathlineto{\pgfqpoint{0.945316in}{0.711471in}}%
\pgfpathlineto{\pgfqpoint{0.945763in}{0.710834in}}%
\pgfpathlineto{\pgfqpoint{0.946217in}{0.710820in}}%
\pgfpathlineto{\pgfqpoint{0.954596in}{0.710219in}}%
\pgfpathlineto{\pgfqpoint{0.960596in}{0.708592in}}%
\pgfpathlineto{\pgfqpoint{0.987244in}{0.706065in}}%
\pgfpathlineto{\pgfqpoint{0.990735in}{0.704724in}}%
\pgfpathlineto{\pgfqpoint{0.990837in}{0.703282in}}%
\pgfpathlineto{\pgfqpoint{0.996565in}{0.703044in}}%
\pgfpathlineto{\pgfqpoint{1.011096in}{0.700386in}}%
\pgfpathlineto{\pgfqpoint{1.016505in}{0.700376in}}%
\pgfpathlineto{\pgfqpoint{1.017496in}{0.699473in}}%
\pgfpathlineto{\pgfqpoint{1.070825in}{0.697151in}}%
\pgfpathlineto{\pgfqpoint{1.075000in}{0.692618in}}%
\pgfpathlineto{\pgfqpoint{1.077372in}{0.692269in}}%
\pgfpathlineto{\pgfqpoint{1.085816in}{0.691662in}}%
\pgfpathlineto{\pgfqpoint{1.103303in}{0.689151in}}%
\pgfpathlineto{\pgfqpoint{1.108617in}{0.688763in}}%
\pgfpathlineto{\pgfqpoint{1.109998in}{0.688225in}}%
\pgfpathlineto{\pgfqpoint{1.126908in}{0.687169in}}%
\pgfpathlineto{\pgfqpoint{1.128867in}{0.686499in}}%
\pgfpathlineto{\pgfqpoint{1.147700in}{0.686452in}}%
\pgfpathlineto{\pgfqpoint{1.148995in}{0.685892in}}%
\pgfpathlineto{\pgfqpoint{1.159390in}{0.685625in}}%
\pgfpathlineto{\pgfqpoint{1.200157in}{0.685395in}}%
\pgfpathlineto{\pgfqpoint{1.220931in}{0.685389in}}%
\pgfpathlineto{\pgfqpoint{1.495917in}{0.683547in}}%
\pgfpathlineto{\pgfqpoint{1.503177in}{0.683449in}}%
\pgfpathlineto{\pgfqpoint{1.681839in}{0.682826in}}%
\pgfpathlineto{\pgfqpoint{1.684802in}{0.682783in}}%
\pgfpathlineto{\pgfqpoint{1.732115in}{0.682529in}}%
\pgfpathlineto{\pgfqpoint{1.738010in}{0.682319in}}%
\pgfpathlineto{\pgfqpoint{1.761288in}{0.681989in}}%
\pgfpathlineto{\pgfqpoint{1.765345in}{0.681943in}}%
\pgfpathlineto{\pgfqpoint{1.826952in}{0.681747in}}%
\pgfpathlineto{\pgfqpoint{2.063345in}{0.680516in}}%
\pgfpathlineto{\pgfqpoint{2.225009in}{0.680020in}}%
\pgfpathlineto{\pgfqpoint{2.244939in}{0.679615in}}%
\pgfpathlineto{\pgfqpoint{2.378269in}{0.679052in}}%
\pgfpathlineto{\pgfqpoint{2.429320in}{0.678525in}}%
\pgfpathlineto{\pgfqpoint{2.510252in}{0.678129in}}%
\pgfpathlineto{\pgfqpoint{2.542698in}{0.677943in}}%
\pgfpathlineto{\pgfqpoint{2.608500in}{0.677847in}}%
\pgfpathlineto{\pgfqpoint{2.677313in}{0.677350in}}%
\pgfpathlineto{\pgfqpoint{2.911373in}{0.676955in}}%
\pgfpathlineto{\pgfqpoint{2.932466in}{0.676438in}}%
\pgfpathlineto{\pgfqpoint{3.024552in}{0.675964in}}%
\pgfpathlineto{\pgfqpoint{3.291859in}{0.674921in}}%
\pgfpathlineto{\pgfqpoint{3.560021in}{0.673854in}}%
\pgfpathlineto{\pgfqpoint{3.863667in}{0.673125in}}%
\pgfpathlineto{\pgfqpoint{4.213238in}{0.672266in}}%
\pgfpathlineto{\pgfqpoint{4.613721in}{0.672131in}}%
\pgfpathlineto{\pgfqpoint{4.631124in}{0.671946in}}%
\pgfpathlineto{\pgfqpoint{4.726657in}{0.671660in}}%
\pgfpathlineto{\pgfqpoint{4.764231in}{0.671484in}}%
\pgfpathlineto{\pgfqpoint{5.135185in}{0.671116in}}%
\pgfpathlineto{\pgfqpoint{5.159789in}{0.670855in}}%
\pgfpathlineto{\pgfqpoint{5.614292in}{0.670505in}}%
\pgfpathlineto{\pgfqpoint{6.004393in}{0.670138in}}%
\pgfpathlineto{\pgfqpoint{6.004393in}{0.670138in}}%
\pgfusepath{stroke}%
\end{pgfscope}%
\begin{pgfscope}%
\pgfpathrectangle{\pgfqpoint{0.688192in}{0.670138in}}{\pgfqpoint{7.111808in}{5.061530in}}%
\pgfusepath{clip}%
\pgfsetbuttcap%
\pgfsetroundjoin%
\definecolor{currentfill}{rgb}{0.000000,0.000000,0.000000}%
\pgfsetfillcolor{currentfill}%
\pgfsetlinewidth{1.003750pt}%
\definecolor{currentstroke}{rgb}{0.000000,0.000000,0.000000}%
\pgfsetstrokecolor{currentstroke}%
\pgfsetdash{}{0pt}%
\pgfsys@defobject{currentmarker}{\pgfqpoint{-0.006944in}{-0.006944in}}{\pgfqpoint{0.006944in}{0.006944in}}{%
\pgfpathmoveto{\pgfqpoint{0.000000in}{-0.006944in}}%
\pgfpathcurveto{\pgfqpoint{0.001842in}{-0.006944in}}{\pgfqpoint{0.003608in}{-0.006213in}}{\pgfqpoint{0.004910in}{-0.004910in}}%
\pgfpathcurveto{\pgfqpoint{0.006213in}{-0.003608in}}{\pgfqpoint{0.006944in}{-0.001842in}}{\pgfqpoint{0.006944in}{0.000000in}}%
\pgfpathcurveto{\pgfqpoint{0.006944in}{0.001842in}}{\pgfqpoint{0.006213in}{0.003608in}}{\pgfqpoint{0.004910in}{0.004910in}}%
\pgfpathcurveto{\pgfqpoint{0.003608in}{0.006213in}}{\pgfqpoint{0.001842in}{0.006944in}}{\pgfqpoint{0.000000in}{0.006944in}}%
\pgfpathcurveto{\pgfqpoint{-0.001842in}{0.006944in}}{\pgfqpoint{-0.003608in}{0.006213in}}{\pgfqpoint{-0.004910in}{0.004910in}}%
\pgfpathcurveto{\pgfqpoint{-0.006213in}{0.003608in}}{\pgfqpoint{-0.006944in}{0.001842in}}{\pgfqpoint{-0.006944in}{0.000000in}}%
\pgfpathcurveto{\pgfqpoint{-0.006944in}{-0.001842in}}{\pgfqpoint{-0.006213in}{-0.003608in}}{\pgfqpoint{-0.004910in}{-0.004910in}}%
\pgfpathcurveto{\pgfqpoint{-0.003608in}{-0.006213in}}{\pgfqpoint{-0.001842in}{-0.006944in}}{\pgfqpoint{0.000000in}{-0.006944in}}%
\pgfpathlineto{\pgfqpoint{0.000000in}{-0.006944in}}%
\pgfpathclose%
\pgfusepath{stroke,fill}%
}%
\begin{pgfscope}%
\pgfsys@transformshift{0.749254in}{1.445149in}%
\pgfsys@useobject{currentmarker}{}%
\end{pgfscope}%
\begin{pgfscope}%
\pgfsys@transformshift{0.752983in}{1.252761in}%
\pgfsys@useobject{currentmarker}{}%
\end{pgfscope}%
\begin{pgfscope}%
\pgfsys@transformshift{0.754090in}{1.154399in}%
\pgfsys@useobject{currentmarker}{}%
\end{pgfscope}%
\begin{pgfscope}%
\pgfsys@transformshift{0.760839in}{1.136279in}%
\pgfsys@useobject{currentmarker}{}%
\end{pgfscope}%
\begin{pgfscope}%
\pgfsys@transformshift{0.764625in}{1.015211in}%
\pgfsys@useobject{currentmarker}{}%
\end{pgfscope}%
\begin{pgfscope}%
\pgfsys@transformshift{0.769073in}{0.982259in}%
\pgfsys@useobject{currentmarker}{}%
\end{pgfscope}%
\begin{pgfscope}%
\pgfsys@transformshift{0.771295in}{0.982065in}%
\pgfsys@useobject{currentmarker}{}%
\end{pgfscope}%
\begin{pgfscope}%
\pgfsys@transformshift{0.776750in}{0.941782in}%
\pgfsys@useobject{currentmarker}{}%
\end{pgfscope}%
\begin{pgfscope}%
\pgfsys@transformshift{0.777016in}{0.919143in}%
\pgfsys@useobject{currentmarker}{}%
\end{pgfscope}%
\begin{pgfscope}%
\pgfsys@transformshift{0.779964in}{0.894614in}%
\pgfsys@useobject{currentmarker}{}%
\end{pgfscope}%
\begin{pgfscope}%
\pgfsys@transformshift{0.781884in}{0.888672in}%
\pgfsys@useobject{currentmarker}{}%
\end{pgfscope}%
\begin{pgfscope}%
\pgfsys@transformshift{0.782833in}{0.883385in}%
\pgfsys@useobject{currentmarker}{}%
\end{pgfscope}%
\begin{pgfscope}%
\pgfsys@transformshift{0.782912in}{0.882838in}%
\pgfsys@useobject{currentmarker}{}%
\end{pgfscope}%
\begin{pgfscope}%
\pgfsys@transformshift{0.783279in}{0.849857in}%
\pgfsys@useobject{currentmarker}{}%
\end{pgfscope}%
\begin{pgfscope}%
\pgfsys@transformshift{0.783279in}{0.849857in}%
\pgfsys@useobject{currentmarker}{}%
\end{pgfscope}%
\begin{pgfscope}%
\pgfsys@transformshift{0.787475in}{0.843942in}%
\pgfsys@useobject{currentmarker}{}%
\end{pgfscope}%
\begin{pgfscope}%
\pgfsys@transformshift{0.791486in}{0.837055in}%
\pgfsys@useobject{currentmarker}{}%
\end{pgfscope}%
\begin{pgfscope}%
\pgfsys@transformshift{0.793302in}{0.809599in}%
\pgfsys@useobject{currentmarker}{}%
\end{pgfscope}%
\begin{pgfscope}%
\pgfsys@transformshift{0.793310in}{0.809534in}%
\pgfsys@useobject{currentmarker}{}%
\end{pgfscope}%
\begin{pgfscope}%
\pgfsys@transformshift{0.795978in}{0.801029in}%
\pgfsys@useobject{currentmarker}{}%
\end{pgfscope}%
\begin{pgfscope}%
\pgfsys@transformshift{0.796157in}{0.793846in}%
\pgfsys@useobject{currentmarker}{}%
\end{pgfscope}%
\begin{pgfscope}%
\pgfsys@transformshift{0.800976in}{0.790369in}%
\pgfsys@useobject{currentmarker}{}%
\end{pgfscope}%
\begin{pgfscope}%
\pgfsys@transformshift{0.807883in}{0.789563in}%
\pgfsys@useobject{currentmarker}{}%
\end{pgfscope}%
\begin{pgfscope}%
\pgfsys@transformshift{0.810327in}{0.766458in}%
\pgfsys@useobject{currentmarker}{}%
\end{pgfscope}%
\begin{pgfscope}%
\pgfsys@transformshift{0.816256in}{0.760843in}%
\pgfsys@useobject{currentmarker}{}%
\end{pgfscope}%
\begin{pgfscope}%
\pgfsys@transformshift{0.827191in}{0.759800in}%
\pgfsys@useobject{currentmarker}{}%
\end{pgfscope}%
\begin{pgfscope}%
\pgfsys@transformshift{0.830930in}{0.751031in}%
\pgfsys@useobject{currentmarker}{}%
\end{pgfscope}%
\begin{pgfscope}%
\pgfsys@transformshift{0.834764in}{0.749497in}%
\pgfsys@useobject{currentmarker}{}%
\end{pgfscope}%
\begin{pgfscope}%
\pgfsys@transformshift{0.837359in}{0.748345in}%
\pgfsys@useobject{currentmarker}{}%
\end{pgfscope}%
\begin{pgfscope}%
\pgfsys@transformshift{0.842310in}{0.745304in}%
\pgfsys@useobject{currentmarker}{}%
\end{pgfscope}%
\begin{pgfscope}%
\pgfsys@transformshift{0.842696in}{0.745162in}%
\pgfsys@useobject{currentmarker}{}%
\end{pgfscope}%
\begin{pgfscope}%
\pgfsys@transformshift{0.842811in}{0.745008in}%
\pgfsys@useobject{currentmarker}{}%
\end{pgfscope}%
\begin{pgfscope}%
\pgfsys@transformshift{0.849436in}{0.740184in}%
\pgfsys@useobject{currentmarker}{}%
\end{pgfscope}%
\begin{pgfscope}%
\pgfsys@transformshift{0.849953in}{0.739357in}%
\pgfsys@useobject{currentmarker}{}%
\end{pgfscope}%
\begin{pgfscope}%
\pgfsys@transformshift{0.850270in}{0.738398in}%
\pgfsys@useobject{currentmarker}{}%
\end{pgfscope}%
\begin{pgfscope}%
\pgfsys@transformshift{0.854347in}{0.736661in}%
\pgfsys@useobject{currentmarker}{}%
\end{pgfscope}%
\begin{pgfscope}%
\pgfsys@transformshift{0.872029in}{0.731700in}%
\pgfsys@useobject{currentmarker}{}%
\end{pgfscope}%
\begin{pgfscope}%
\pgfsys@transformshift{0.890835in}{0.724786in}%
\pgfsys@useobject{currentmarker}{}%
\end{pgfscope}%
\begin{pgfscope}%
\pgfsys@transformshift{0.916357in}{0.719853in}%
\pgfsys@useobject{currentmarker}{}%
\end{pgfscope}%
\begin{pgfscope}%
\pgfsys@transformshift{0.939729in}{0.718197in}%
\pgfsys@useobject{currentmarker}{}%
\end{pgfscope}%
\begin{pgfscope}%
\pgfsys@transformshift{0.940682in}{0.716009in}%
\pgfsys@useobject{currentmarker}{}%
\end{pgfscope}%
\begin{pgfscope}%
\pgfsys@transformshift{0.942062in}{0.712194in}%
\pgfsys@useobject{currentmarker}{}%
\end{pgfscope}%
\begin{pgfscope}%
\pgfsys@transformshift{0.945316in}{0.711471in}%
\pgfsys@useobject{currentmarker}{}%
\end{pgfscope}%
\begin{pgfscope}%
\pgfsys@transformshift{0.945763in}{0.710834in}%
\pgfsys@useobject{currentmarker}{}%
\end{pgfscope}%
\begin{pgfscope}%
\pgfsys@transformshift{0.946217in}{0.710820in}%
\pgfsys@useobject{currentmarker}{}%
\end{pgfscope}%
\begin{pgfscope}%
\pgfsys@transformshift{0.954596in}{0.710219in}%
\pgfsys@useobject{currentmarker}{}%
\end{pgfscope}%
\begin{pgfscope}%
\pgfsys@transformshift{0.958858in}{0.709014in}%
\pgfsys@useobject{currentmarker}{}%
\end{pgfscope}%
\begin{pgfscope}%
\pgfsys@transformshift{0.960596in}{0.708592in}%
\pgfsys@useobject{currentmarker}{}%
\end{pgfscope}%
\begin{pgfscope}%
\pgfsys@transformshift{0.987244in}{0.706065in}%
\pgfsys@useobject{currentmarker}{}%
\end{pgfscope}%
\begin{pgfscope}%
\pgfsys@transformshift{0.990735in}{0.704724in}%
\pgfsys@useobject{currentmarker}{}%
\end{pgfscope}%
\begin{pgfscope}%
\pgfsys@transformshift{0.990837in}{0.703282in}%
\pgfsys@useobject{currentmarker}{}%
\end{pgfscope}%
\begin{pgfscope}%
\pgfsys@transformshift{0.996565in}{0.703044in}%
\pgfsys@useobject{currentmarker}{}%
\end{pgfscope}%
\begin{pgfscope}%
\pgfsys@transformshift{1.010100in}{0.700622in}%
\pgfsys@useobject{currentmarker}{}%
\end{pgfscope}%
\begin{pgfscope}%
\pgfsys@transformshift{1.011096in}{0.700386in}%
\pgfsys@useobject{currentmarker}{}%
\end{pgfscope}%
\begin{pgfscope}%
\pgfsys@transformshift{1.016505in}{0.700376in}%
\pgfsys@useobject{currentmarker}{}%
\end{pgfscope}%
\begin{pgfscope}%
\pgfsys@transformshift{1.017496in}{0.699473in}%
\pgfsys@useobject{currentmarker}{}%
\end{pgfscope}%
\begin{pgfscope}%
\pgfsys@transformshift{1.070825in}{0.697151in}%
\pgfsys@useobject{currentmarker}{}%
\end{pgfscope}%
\begin{pgfscope}%
\pgfsys@transformshift{1.075000in}{0.692618in}%
\pgfsys@useobject{currentmarker}{}%
\end{pgfscope}%
\begin{pgfscope}%
\pgfsys@transformshift{1.077361in}{0.692269in}%
\pgfsys@useobject{currentmarker}{}%
\end{pgfscope}%
\begin{pgfscope}%
\pgfsys@transformshift{1.077372in}{0.692269in}%
\pgfsys@useobject{currentmarker}{}%
\end{pgfscope}%
\begin{pgfscope}%
\pgfsys@transformshift{1.084164in}{0.691950in}%
\pgfsys@useobject{currentmarker}{}%
\end{pgfscope}%
\begin{pgfscope}%
\pgfsys@transformshift{1.085816in}{0.691662in}%
\pgfsys@useobject{currentmarker}{}%
\end{pgfscope}%
\begin{pgfscope}%
\pgfsys@transformshift{1.102529in}{0.689352in}%
\pgfsys@useobject{currentmarker}{}%
\end{pgfscope}%
\begin{pgfscope}%
\pgfsys@transformshift{1.103303in}{0.689151in}%
\pgfsys@useobject{currentmarker}{}%
\end{pgfscope}%
\begin{pgfscope}%
\pgfsys@transformshift{1.108579in}{0.688860in}%
\pgfsys@useobject{currentmarker}{}%
\end{pgfscope}%
\begin{pgfscope}%
\pgfsys@transformshift{1.108617in}{0.688763in}%
\pgfsys@useobject{currentmarker}{}%
\end{pgfscope}%
\begin{pgfscope}%
\pgfsys@transformshift{1.109998in}{0.688225in}%
\pgfsys@useobject{currentmarker}{}%
\end{pgfscope}%
\begin{pgfscope}%
\pgfsys@transformshift{1.114241in}{0.688028in}%
\pgfsys@useobject{currentmarker}{}%
\end{pgfscope}%
\begin{pgfscope}%
\pgfsys@transformshift{1.116764in}{0.688006in}%
\pgfsys@useobject{currentmarker}{}%
\end{pgfscope}%
\begin{pgfscope}%
\pgfsys@transformshift{1.126908in}{0.687169in}%
\pgfsys@useobject{currentmarker}{}%
\end{pgfscope}%
\begin{pgfscope}%
\pgfsys@transformshift{1.128607in}{0.686506in}%
\pgfsys@useobject{currentmarker}{}%
\end{pgfscope}%
\begin{pgfscope}%
\pgfsys@transformshift{1.128867in}{0.686499in}%
\pgfsys@useobject{currentmarker}{}%
\end{pgfscope}%
\begin{pgfscope}%
\pgfsys@transformshift{1.147700in}{0.686452in}%
\pgfsys@useobject{currentmarker}{}%
\end{pgfscope}%
\begin{pgfscope}%
\pgfsys@transformshift{1.148995in}{0.685892in}%
\pgfsys@useobject{currentmarker}{}%
\end{pgfscope}%
\begin{pgfscope}%
\pgfsys@transformshift{1.151625in}{0.685778in}%
\pgfsys@useobject{currentmarker}{}%
\end{pgfscope}%
\begin{pgfscope}%
\pgfsys@transformshift{1.159390in}{0.685625in}%
\pgfsys@useobject{currentmarker}{}%
\end{pgfscope}%
\begin{pgfscope}%
\pgfsys@transformshift{1.164071in}{0.685586in}%
\pgfsys@useobject{currentmarker}{}%
\end{pgfscope}%
\begin{pgfscope}%
\pgfsys@transformshift{1.178009in}{0.685447in}%
\pgfsys@useobject{currentmarker}{}%
\end{pgfscope}%
\begin{pgfscope}%
\pgfsys@transformshift{1.197079in}{0.685403in}%
\pgfsys@useobject{currentmarker}{}%
\end{pgfscope}%
\begin{pgfscope}%
\pgfsys@transformshift{1.198965in}{0.685402in}%
\pgfsys@useobject{currentmarker}{}%
\end{pgfscope}%
\begin{pgfscope}%
\pgfsys@transformshift{1.200157in}{0.685395in}%
\pgfsys@useobject{currentmarker}{}%
\end{pgfscope}%
\begin{pgfscope}%
\pgfsys@transformshift{1.220931in}{0.685389in}%
\pgfsys@useobject{currentmarker}{}%
\end{pgfscope}%
\begin{pgfscope}%
\pgfsys@transformshift{1.315614in}{0.684844in}%
\pgfsys@useobject{currentmarker}{}%
\end{pgfscope}%
\begin{pgfscope}%
\pgfsys@transformshift{1.334128in}{0.684517in}%
\pgfsys@useobject{currentmarker}{}%
\end{pgfscope}%
\begin{pgfscope}%
\pgfsys@transformshift{1.363285in}{0.684419in}%
\pgfsys@useobject{currentmarker}{}%
\end{pgfscope}%
\begin{pgfscope}%
\pgfsys@transformshift{1.364139in}{0.684404in}%
\pgfsys@useobject{currentmarker}{}%
\end{pgfscope}%
\begin{pgfscope}%
\pgfsys@transformshift{1.376266in}{0.684291in}%
\pgfsys@useobject{currentmarker}{}%
\end{pgfscope}%
\begin{pgfscope}%
\pgfsys@transformshift{1.396594in}{0.684212in}%
\pgfsys@useobject{currentmarker}{}%
\end{pgfscope}%
\begin{pgfscope}%
\pgfsys@transformshift{1.399923in}{0.684150in}%
\pgfsys@useobject{currentmarker}{}%
\end{pgfscope}%
\begin{pgfscope}%
\pgfsys@transformshift{1.407033in}{0.684070in}%
\pgfsys@useobject{currentmarker}{}%
\end{pgfscope}%
\begin{pgfscope}%
\pgfsys@transformshift{1.444241in}{0.684035in}%
\pgfsys@useobject{currentmarker}{}%
\end{pgfscope}%
\begin{pgfscope}%
\pgfsys@transformshift{1.455660in}{0.683986in}%
\pgfsys@useobject{currentmarker}{}%
\end{pgfscope}%
\begin{pgfscope}%
\pgfsys@transformshift{1.469414in}{0.683962in}%
\pgfsys@useobject{currentmarker}{}%
\end{pgfscope}%
\begin{pgfscope}%
\pgfsys@transformshift{1.472622in}{0.683959in}%
\pgfsys@useobject{currentmarker}{}%
\end{pgfscope}%
\begin{pgfscope}%
\pgfsys@transformshift{1.473551in}{0.683948in}%
\pgfsys@useobject{currentmarker}{}%
\end{pgfscope}%
\begin{pgfscope}%
\pgfsys@transformshift{1.473926in}{0.683941in}%
\pgfsys@useobject{currentmarker}{}%
\end{pgfscope}%
\begin{pgfscope}%
\pgfsys@transformshift{1.474245in}{0.683935in}%
\pgfsys@useobject{currentmarker}{}%
\end{pgfscope}%
\begin{pgfscope}%
\pgfsys@transformshift{1.475910in}{0.683912in}%
\pgfsys@useobject{currentmarker}{}%
\end{pgfscope}%
\begin{pgfscope}%
\pgfsys@transformshift{1.477331in}{0.683875in}%
\pgfsys@useobject{currentmarker}{}%
\end{pgfscope}%
\begin{pgfscope}%
\pgfsys@transformshift{1.495917in}{0.683547in}%
\pgfsys@useobject{currentmarker}{}%
\end{pgfscope}%
\begin{pgfscope}%
\pgfsys@transformshift{1.503177in}{0.683449in}%
\pgfsys@useobject{currentmarker}{}%
\end{pgfscope}%
\begin{pgfscope}%
\pgfsys@transformshift{1.582513in}{0.683280in}%
\pgfsys@useobject{currentmarker}{}%
\end{pgfscope}%
\begin{pgfscope}%
\pgfsys@transformshift{1.658826in}{0.683277in}%
\pgfsys@useobject{currentmarker}{}%
\end{pgfscope}%
\begin{pgfscope}%
\pgfsys@transformshift{1.659921in}{0.683127in}%
\pgfsys@useobject{currentmarker}{}%
\end{pgfscope}%
\begin{pgfscope}%
\pgfsys@transformshift{1.663205in}{0.683065in}%
\pgfsys@useobject{currentmarker}{}%
\end{pgfscope}%
\begin{pgfscope}%
\pgfsys@transformshift{1.675058in}{0.682998in}%
\pgfsys@useobject{currentmarker}{}%
\end{pgfscope}%
\begin{pgfscope}%
\pgfsys@transformshift{1.678208in}{0.682876in}%
\pgfsys@useobject{currentmarker}{}%
\end{pgfscope}%
\begin{pgfscope}%
\pgfsys@transformshift{1.681839in}{0.682826in}%
\pgfsys@useobject{currentmarker}{}%
\end{pgfscope}%
\begin{pgfscope}%
\pgfsys@transformshift{1.684802in}{0.682783in}%
\pgfsys@useobject{currentmarker}{}%
\end{pgfscope}%
\begin{pgfscope}%
\pgfsys@transformshift{1.719213in}{0.682724in}%
\pgfsys@useobject{currentmarker}{}%
\end{pgfscope}%
\begin{pgfscope}%
\pgfsys@transformshift{1.731063in}{0.682549in}%
\pgfsys@useobject{currentmarker}{}%
\end{pgfscope}%
\begin{pgfscope}%
\pgfsys@transformshift{1.731531in}{0.682540in}%
\pgfsys@useobject{currentmarker}{}%
\end{pgfscope}%
\begin{pgfscope}%
\pgfsys@transformshift{1.731531in}{0.682540in}%
\pgfsys@useobject{currentmarker}{}%
\end{pgfscope}%
\begin{pgfscope}%
\pgfsys@transformshift{1.731710in}{0.682537in}%
\pgfsys@useobject{currentmarker}{}%
\end{pgfscope}%
\begin{pgfscope}%
\pgfsys@transformshift{1.732115in}{0.682529in}%
\pgfsys@useobject{currentmarker}{}%
\end{pgfscope}%
\begin{pgfscope}%
\pgfsys@transformshift{1.738010in}{0.682319in}%
\pgfsys@useobject{currentmarker}{}%
\end{pgfscope}%
\begin{pgfscope}%
\pgfsys@transformshift{1.738010in}{0.682319in}%
\pgfsys@useobject{currentmarker}{}%
\end{pgfscope}%
\begin{pgfscope}%
\pgfsys@transformshift{1.755173in}{0.682260in}%
\pgfsys@useobject{currentmarker}{}%
\end{pgfscope}%
\begin{pgfscope}%
\pgfsys@transformshift{1.759976in}{0.682015in}%
\pgfsys@useobject{currentmarker}{}%
\end{pgfscope}%
\begin{pgfscope}%
\pgfsys@transformshift{1.761288in}{0.681989in}%
\pgfsys@useobject{currentmarker}{}%
\end{pgfscope}%
\begin{pgfscope}%
\pgfsys@transformshift{1.765345in}{0.681943in}%
\pgfsys@useobject{currentmarker}{}%
\end{pgfscope}%
\begin{pgfscope}%
\pgfsys@transformshift{1.811432in}{0.681927in}%
\pgfsys@useobject{currentmarker}{}%
\end{pgfscope}%
\begin{pgfscope}%
\pgfsys@transformshift{1.811432in}{0.681927in}%
\pgfsys@useobject{currentmarker}{}%
\end{pgfscope}%
\begin{pgfscope}%
\pgfsys@transformshift{1.815579in}{0.681901in}%
\pgfsys@useobject{currentmarker}{}%
\end{pgfscope}%
\begin{pgfscope}%
\pgfsys@transformshift{1.826952in}{0.681747in}%
\pgfsys@useobject{currentmarker}{}%
\end{pgfscope}%
\begin{pgfscope}%
\pgfsys@transformshift{1.852021in}{0.681611in}%
\pgfsys@useobject{currentmarker}{}%
\end{pgfscope}%
\begin{pgfscope}%
\pgfsys@transformshift{1.892079in}{0.681466in}%
\pgfsys@useobject{currentmarker}{}%
\end{pgfscope}%
\begin{pgfscope}%
\pgfsys@transformshift{1.903233in}{0.681282in}%
\pgfsys@useobject{currentmarker}{}%
\end{pgfscope}%
\begin{pgfscope}%
\pgfsys@transformshift{1.928513in}{0.681173in}%
\pgfsys@useobject{currentmarker}{}%
\end{pgfscope}%
\begin{pgfscope}%
\pgfsys@transformshift{1.967310in}{0.681083in}%
\pgfsys@useobject{currentmarker}{}%
\end{pgfscope}%
\begin{pgfscope}%
\pgfsys@transformshift{1.970905in}{0.680890in}%
\pgfsys@useobject{currentmarker}{}%
\end{pgfscope}%
\begin{pgfscope}%
\pgfsys@transformshift{2.063345in}{0.680516in}%
\pgfsys@useobject{currentmarker}{}%
\end{pgfscope}%
\begin{pgfscope}%
\pgfsys@transformshift{2.225009in}{0.680020in}%
\pgfsys@useobject{currentmarker}{}%
\end{pgfscope}%
\begin{pgfscope}%
\pgfsys@transformshift{2.235916in}{0.679656in}%
\pgfsys@useobject{currentmarker}{}%
\end{pgfscope}%
\begin{pgfscope}%
\pgfsys@transformshift{2.237139in}{0.679645in}%
\pgfsys@useobject{currentmarker}{}%
\end{pgfscope}%
\begin{pgfscope}%
\pgfsys@transformshift{2.243733in}{0.679632in}%
\pgfsys@useobject{currentmarker}{}%
\end{pgfscope}%
\begin{pgfscope}%
\pgfsys@transformshift{2.244939in}{0.679615in}%
\pgfsys@useobject{currentmarker}{}%
\end{pgfscope}%
\begin{pgfscope}%
\pgfsys@transformshift{2.265105in}{0.679558in}%
\pgfsys@useobject{currentmarker}{}%
\end{pgfscope}%
\begin{pgfscope}%
\pgfsys@transformshift{2.378269in}{0.679052in}%
\pgfsys@useobject{currentmarker}{}%
\end{pgfscope}%
\begin{pgfscope}%
\pgfsys@transformshift{2.392882in}{0.678851in}%
\pgfsys@useobject{currentmarker}{}%
\end{pgfscope}%
\begin{pgfscope}%
\pgfsys@transformshift{2.429320in}{0.678525in}%
\pgfsys@useobject{currentmarker}{}%
\end{pgfscope}%
\begin{pgfscope}%
\pgfsys@transformshift{2.452500in}{0.678477in}%
\pgfsys@useobject{currentmarker}{}%
\end{pgfscope}%
\begin{pgfscope}%
\pgfsys@transformshift{2.489265in}{0.678468in}%
\pgfsys@useobject{currentmarker}{}%
\end{pgfscope}%
\begin{pgfscope}%
\pgfsys@transformshift{2.494145in}{0.678397in}%
\pgfsys@useobject{currentmarker}{}%
\end{pgfscope}%
\begin{pgfscope}%
\pgfsys@transformshift{2.509817in}{0.678139in}%
\pgfsys@useobject{currentmarker}{}%
\end{pgfscope}%
\begin{pgfscope}%
\pgfsys@transformshift{2.510252in}{0.678129in}%
\pgfsys@useobject{currentmarker}{}%
\end{pgfscope}%
\begin{pgfscope}%
\pgfsys@transformshift{2.542698in}{0.677943in}%
\pgfsys@useobject{currentmarker}{}%
\end{pgfscope}%
\begin{pgfscope}%
\pgfsys@transformshift{2.608500in}{0.677847in}%
\pgfsys@useobject{currentmarker}{}%
\end{pgfscope}%
\begin{pgfscope}%
\pgfsys@transformshift{2.672675in}{0.677398in}%
\pgfsys@useobject{currentmarker}{}%
\end{pgfscope}%
\begin{pgfscope}%
\pgfsys@transformshift{2.677313in}{0.677350in}%
\pgfsys@useobject{currentmarker}{}%
\end{pgfscope}%
\begin{pgfscope}%
\pgfsys@transformshift{2.911373in}{0.676955in}%
\pgfsys@useobject{currentmarker}{}%
\end{pgfscope}%
\begin{pgfscope}%
\pgfsys@transformshift{2.932466in}{0.676438in}%
\pgfsys@useobject{currentmarker}{}%
\end{pgfscope}%
\begin{pgfscope}%
\pgfsys@transformshift{2.976355in}{0.676171in}%
\pgfsys@useobject{currentmarker}{}%
\end{pgfscope}%
\begin{pgfscope}%
\pgfsys@transformshift{3.024552in}{0.675964in}%
\pgfsys@useobject{currentmarker}{}%
\end{pgfscope}%
\begin{pgfscope}%
\pgfsys@transformshift{3.280946in}{0.675128in}%
\pgfsys@useobject{currentmarker}{}%
\end{pgfscope}%
\begin{pgfscope}%
\pgfsys@transformshift{3.291859in}{0.674921in}%
\pgfsys@useobject{currentmarker}{}%
\end{pgfscope}%
\begin{pgfscope}%
\pgfsys@transformshift{3.560021in}{0.673854in}%
\pgfsys@useobject{currentmarker}{}%
\end{pgfscope}%
\begin{pgfscope}%
\pgfsys@transformshift{3.790533in}{0.673411in}%
\pgfsys@useobject{currentmarker}{}%
\end{pgfscope}%
\begin{pgfscope}%
\pgfsys@transformshift{3.825636in}{0.673260in}%
\pgfsys@useobject{currentmarker}{}%
\end{pgfscope}%
\begin{pgfscope}%
\pgfsys@transformshift{3.863667in}{0.673125in}%
\pgfsys@useobject{currentmarker}{}%
\end{pgfscope}%
\begin{pgfscope}%
\pgfsys@transformshift{3.986933in}{0.672748in}%
\pgfsys@useobject{currentmarker}{}%
\end{pgfscope}%
\begin{pgfscope}%
\pgfsys@transformshift{3.995987in}{0.672712in}%
\pgfsys@useobject{currentmarker}{}%
\end{pgfscope}%
\begin{pgfscope}%
\pgfsys@transformshift{4.155194in}{0.672354in}%
\pgfsys@useobject{currentmarker}{}%
\end{pgfscope}%
\begin{pgfscope}%
\pgfsys@transformshift{4.207141in}{0.672282in}%
\pgfsys@useobject{currentmarker}{}%
\end{pgfscope}%
\begin{pgfscope}%
\pgfsys@transformshift{4.213238in}{0.672266in}%
\pgfsys@useobject{currentmarker}{}%
\end{pgfscope}%
\begin{pgfscope}%
\pgfsys@transformshift{4.420707in}{0.672256in}%
\pgfsys@useobject{currentmarker}{}%
\end{pgfscope}%
\begin{pgfscope}%
\pgfsys@transformshift{4.421026in}{0.672255in}%
\pgfsys@useobject{currentmarker}{}%
\end{pgfscope}%
\begin{pgfscope}%
\pgfsys@transformshift{4.491507in}{0.672222in}%
\pgfsys@useobject{currentmarker}{}%
\end{pgfscope}%
\begin{pgfscope}%
\pgfsys@transformshift{4.576929in}{0.672144in}%
\pgfsys@useobject{currentmarker}{}%
\end{pgfscope}%
\begin{pgfscope}%
\pgfsys@transformshift{4.613721in}{0.672131in}%
\pgfsys@useobject{currentmarker}{}%
\end{pgfscope}%
\begin{pgfscope}%
\pgfsys@transformshift{4.631124in}{0.671946in}%
\pgfsys@useobject{currentmarker}{}%
\end{pgfscope}%
\begin{pgfscope}%
\pgfsys@transformshift{4.661935in}{0.671930in}%
\pgfsys@useobject{currentmarker}{}%
\end{pgfscope}%
\begin{pgfscope}%
\pgfsys@transformshift{4.688548in}{0.671723in}%
\pgfsys@useobject{currentmarker}{}%
\end{pgfscope}%
\begin{pgfscope}%
\pgfsys@transformshift{4.726657in}{0.671660in}%
\pgfsys@useobject{currentmarker}{}%
\end{pgfscope}%
\begin{pgfscope}%
\pgfsys@transformshift{4.735203in}{0.671563in}%
\pgfsys@useobject{currentmarker}{}%
\end{pgfscope}%
\begin{pgfscope}%
\pgfsys@transformshift{4.750941in}{0.671524in}%
\pgfsys@useobject{currentmarker}{}%
\end{pgfscope}%
\begin{pgfscope}%
\pgfsys@transformshift{4.764231in}{0.671484in}%
\pgfsys@useobject{currentmarker}{}%
\end{pgfscope}%
\begin{pgfscope}%
\pgfsys@transformshift{4.914508in}{0.671444in}%
\pgfsys@useobject{currentmarker}{}%
\end{pgfscope}%
\begin{pgfscope}%
\pgfsys@transformshift{4.924693in}{0.671413in}%
\pgfsys@useobject{currentmarker}{}%
\end{pgfscope}%
\begin{pgfscope}%
\pgfsys@transformshift{4.939656in}{0.671407in}%
\pgfsys@useobject{currentmarker}{}%
\end{pgfscope}%
\begin{pgfscope}%
\pgfsys@transformshift{4.965126in}{0.671366in}%
\pgfsys@useobject{currentmarker}{}%
\end{pgfscope}%
\begin{pgfscope}%
\pgfsys@transformshift{4.979341in}{0.671320in}%
\pgfsys@useobject{currentmarker}{}%
\end{pgfscope}%
\begin{pgfscope}%
\pgfsys@transformshift{4.994709in}{0.671264in}%
\pgfsys@useobject{currentmarker}{}%
\end{pgfscope}%
\begin{pgfscope}%
\pgfsys@transformshift{5.085461in}{0.671155in}%
\pgfsys@useobject{currentmarker}{}%
\end{pgfscope}%
\begin{pgfscope}%
\pgfsys@transformshift{5.135185in}{0.671116in}%
\pgfsys@useobject{currentmarker}{}%
\end{pgfscope}%
\begin{pgfscope}%
\pgfsys@transformshift{5.140992in}{0.670993in}%
\pgfsys@useobject{currentmarker}{}%
\end{pgfscope}%
\begin{pgfscope}%
\pgfsys@transformshift{5.159789in}{0.670855in}%
\pgfsys@useobject{currentmarker}{}%
\end{pgfscope}%
\begin{pgfscope}%
\pgfsys@transformshift{5.274608in}{0.670832in}%
\pgfsys@useobject{currentmarker}{}%
\end{pgfscope}%
\begin{pgfscope}%
\pgfsys@transformshift{5.274608in}{0.670832in}%
\pgfsys@useobject{currentmarker}{}%
\end{pgfscope}%
\begin{pgfscope}%
\pgfsys@transformshift{5.283066in}{0.670786in}%
\pgfsys@useobject{currentmarker}{}%
\end{pgfscope}%
\begin{pgfscope}%
\pgfsys@transformshift{5.359877in}{0.670710in}%
\pgfsys@useobject{currentmarker}{}%
\end{pgfscope}%
\begin{pgfscope}%
\pgfsys@transformshift{5.370884in}{0.670626in}%
\pgfsys@useobject{currentmarker}{}%
\end{pgfscope}%
\begin{pgfscope}%
\pgfsys@transformshift{5.527297in}{0.670601in}%
\pgfsys@useobject{currentmarker}{}%
\end{pgfscope}%
\begin{pgfscope}%
\pgfsys@transformshift{5.557238in}{0.670526in}%
\pgfsys@useobject{currentmarker}{}%
\end{pgfscope}%
\begin{pgfscope}%
\pgfsys@transformshift{5.614292in}{0.670505in}%
\pgfsys@useobject{currentmarker}{}%
\end{pgfscope}%
\begin{pgfscope}%
\pgfsys@transformshift{5.657223in}{0.670449in}%
\pgfsys@useobject{currentmarker}{}%
\end{pgfscope}%
\begin{pgfscope}%
\pgfsys@transformshift{5.714356in}{0.670428in}%
\pgfsys@useobject{currentmarker}{}%
\end{pgfscope}%
\begin{pgfscope}%
\pgfsys@transformshift{5.840166in}{0.670247in}%
\pgfsys@useobject{currentmarker}{}%
\end{pgfscope}%
\begin{pgfscope}%
\pgfsys@transformshift{6.004393in}{0.670138in}%
\pgfsys@useobject{currentmarker}{}%
\end{pgfscope}%
\end{pgfscope}%
\begin{pgfscope}%
\pgfpathrectangle{\pgfqpoint{0.688192in}{0.670138in}}{\pgfqpoint{7.111808in}{5.061530in}}%
\pgfusepath{clip}%
\pgfsetrectcap%
\pgfsetroundjoin%
\pgfsetlinewidth{1.505625pt}%
\definecolor{currentstroke}{rgb}{0.501961,0.501961,0.501961}%
\pgfsetstrokecolor{currentstroke}%
\pgfsetstrokeopacity{0.500000}%
\pgfsetdash{}{0pt}%
\pgfpathmoveto{\pgfqpoint{2.019347in}{1.537258in}}%
\pgfpathlineto{\pgfqpoint{2.023448in}{1.325631in}}%
\pgfpathlineto{\pgfqpoint{2.024666in}{1.217432in}}%
\pgfpathlineto{\pgfqpoint{2.032091in}{1.197500in}}%
\pgfpathlineto{\pgfqpoint{2.036255in}{1.064326in}}%
\pgfpathlineto{\pgfqpoint{2.041148in}{1.028078in}}%
\pgfpathlineto{\pgfqpoint{2.043592in}{1.027865in}}%
\pgfpathlineto{\pgfqpoint{2.049592in}{0.983554in}}%
\pgfpathlineto{\pgfqpoint{2.049885in}{0.958651in}}%
\pgfpathlineto{\pgfqpoint{2.053128in}{0.931670in}}%
\pgfpathlineto{\pgfqpoint{2.055240in}{0.925133in}}%
\pgfpathlineto{\pgfqpoint{2.056370in}{0.918715in}}%
\pgfpathlineto{\pgfqpoint{2.056775in}{0.882436in}}%
\pgfpathlineto{\pgfqpoint{2.061390in}{0.875930in}}%
\pgfpathlineto{\pgfqpoint{2.065802in}{0.868354in}}%
\pgfpathlineto{\pgfqpoint{2.067808in}{0.838081in}}%
\pgfpathlineto{\pgfqpoint{2.070744in}{0.828726in}}%
\pgfpathlineto{\pgfqpoint{2.070940in}{0.820824in}}%
\pgfpathlineto{\pgfqpoint{2.076241in}{0.816999in}}%
\pgfpathlineto{\pgfqpoint{2.083839in}{0.816113in}}%
\pgfpathlineto{\pgfqpoint{2.086528in}{0.790698in}}%
\pgfpathlineto{\pgfqpoint{2.093049in}{0.784521in}}%
\pgfpathlineto{\pgfqpoint{2.105078in}{0.783373in}}%
\pgfpathlineto{\pgfqpoint{2.109190in}{0.773728in}}%
\pgfpathlineto{\pgfqpoint{2.116263in}{0.770773in}}%
\pgfpathlineto{\pgfqpoint{2.122260in}{0.767103in}}%
\pgfpathlineto{\pgfqpoint{2.129547in}{0.761796in}}%
\pgfpathlineto{\pgfqpoint{2.130465in}{0.759831in}}%
\pgfpathlineto{\pgfqpoint{2.134949in}{0.757921in}}%
\pgfpathlineto{\pgfqpoint{2.154400in}{0.752463in}}%
\pgfpathlineto{\pgfqpoint{2.175086in}{0.744858in}}%
\pgfpathlineto{\pgfqpoint{2.203161in}{0.739432in}}%
\pgfpathlineto{\pgfqpoint{2.228869in}{0.737610in}}%
\pgfpathlineto{\pgfqpoint{2.229917in}{0.735204in}}%
\pgfpathlineto{\pgfqpoint{2.231436in}{0.731007in}}%
\pgfpathlineto{\pgfqpoint{2.235016in}{0.730212in}}%
\pgfpathlineto{\pgfqpoint{2.235507in}{0.729511in}}%
\pgfpathlineto{\pgfqpoint{2.236007in}{0.729495in}}%
\pgfpathlineto{\pgfqpoint{2.245223in}{0.728835in}}%
\pgfpathlineto{\pgfqpoint{2.251823in}{0.727045in}}%
\pgfpathlineto{\pgfqpoint{2.281136in}{0.724265in}}%
\pgfpathlineto{\pgfqpoint{2.284976in}{0.722790in}}%
\pgfpathlineto{\pgfqpoint{2.285088in}{0.721204in}}%
\pgfpathlineto{\pgfqpoint{2.291389in}{0.720942in}}%
\pgfpathlineto{\pgfqpoint{2.307373in}{0.718018in}}%
\pgfpathlineto{\pgfqpoint{2.313323in}{0.718007in}}%
\pgfpathlineto{\pgfqpoint{2.314413in}{0.717013in}}%
\pgfpathlineto{\pgfqpoint{2.373075in}{0.714460in}}%
\pgfpathlineto{\pgfqpoint{2.377668in}{0.709474in}}%
\pgfpathlineto{\pgfqpoint{2.380277in}{0.709090in}}%
\pgfpathlineto{\pgfqpoint{2.389566in}{0.708422in}}%
\pgfpathlineto{\pgfqpoint{2.408800in}{0.705660in}}%
\pgfpathlineto{\pgfqpoint{2.414647in}{0.705233in}}%
\pgfpathlineto{\pgfqpoint{2.416166in}{0.704641in}}%
\pgfpathlineto{\pgfqpoint{2.423608in}{0.704401in}}%
\pgfpathlineto{\pgfqpoint{2.434766in}{0.703479in}}%
\pgfpathlineto{\pgfqpoint{2.436921in}{0.702743in}}%
\pgfpathlineto{\pgfqpoint{2.457637in}{0.702691in}}%
\pgfpathlineto{\pgfqpoint{2.459062in}{0.702074in}}%
\pgfpathlineto{\pgfqpoint{2.470497in}{0.701782in}}%
\pgfpathlineto{\pgfqpoint{2.515340in}{0.701528in}}%
\pgfpathlineto{\pgfqpoint{2.538192in}{0.701522in}}%
\pgfpathlineto{\pgfqpoint{2.820232in}{0.699856in}}%
\pgfpathlineto{\pgfqpoint{2.848662in}{0.699388in}}%
\pgfpathlineto{\pgfqpoint{3.045190in}{0.698703in}}%
\pgfpathlineto{\pgfqpoint{3.048450in}{0.698655in}}%
\pgfpathlineto{\pgfqpoint{3.100494in}{0.698376in}}%
\pgfpathlineto{\pgfqpoint{3.106979in}{0.698145in}}%
\pgfpathlineto{\pgfqpoint{3.132584in}{0.697781in}}%
\pgfpathlineto{\pgfqpoint{3.137047in}{0.697731in}}%
\pgfpathlineto{\pgfqpoint{3.204814in}{0.697516in}}%
\pgfpathlineto{\pgfqpoint{3.464847in}{0.696161in}}%
\pgfpathlineto{\pgfqpoint{3.642678in}{0.695616in}}%
\pgfpathlineto{\pgfqpoint{3.663274in}{0.695189in}}%
\pgfpathlineto{\pgfqpoint{3.686783in}{0.695107in}}%
\pgfpathlineto{\pgfqpoint{3.827337in}{0.694330in}}%
\pgfpathlineto{\pgfqpoint{3.892918in}{0.693918in}}%
\pgfpathlineto{\pgfqpoint{3.938727in}{0.693830in}}%
\pgfpathlineto{\pgfqpoint{3.956444in}{0.693536in}}%
\pgfpathlineto{\pgfqpoint{3.992135in}{0.693331in}}%
\pgfpathlineto{\pgfqpoint{4.064518in}{0.693225in}}%
\pgfpathlineto{\pgfqpoint{4.140212in}{0.692679in}}%
\pgfpathlineto{\pgfqpoint{4.397678in}{0.692245in}}%
\pgfpathlineto{\pgfqpoint{4.420880in}{0.691675in}}%
\pgfpathlineto{\pgfqpoint{4.522174in}{0.691154in}}%
\pgfpathlineto{\pgfqpoint{4.816213in}{0.690007in}}%
\pgfpathlineto{\pgfqpoint{5.111190in}{0.688834in}}%
\pgfpathlineto{\pgfqpoint{5.445201in}{0.688032in}}%
\pgfpathlineto{\pgfqpoint{5.829729in}{0.687086in}}%
\pgfpathlineto{\pgfqpoint{6.270261in}{0.686938in}}%
\pgfpathlineto{\pgfqpoint{6.289404in}{0.686734in}}%
\pgfpathlineto{\pgfqpoint{6.394490in}{0.686420in}}%
\pgfpathlineto{\pgfqpoint{6.421203in}{0.686270in}}%
\pgfpathlineto{\pgfqpoint{6.435822in}{0.686226in}}%
\pgfpathlineto{\pgfqpoint{6.789175in}{0.685864in}}%
\pgfpathlineto{\pgfqpoint{7.091032in}{0.685375in}}%
\pgfpathlineto{\pgfqpoint{7.103140in}{0.685283in}}%
\pgfpathlineto{\pgfqpoint{7.480959in}{0.685064in}}%
\pgfpathlineto{\pgfqpoint{7.800000in}{0.684746in}}%
\pgfpathlineto{\pgfqpoint{7.800000in}{0.684746in}}%
\pgfusepath{stroke}%
\end{pgfscope}%
\begin{pgfscope}%
\pgfsetrectcap%
\pgfsetmiterjoin%
\pgfsetlinewidth{0.803000pt}%
\definecolor{currentstroke}{rgb}{0.000000,0.000000,0.000000}%
\pgfsetstrokecolor{currentstroke}%
\pgfsetdash{}{0pt}%
\pgfpathmoveto{\pgfqpoint{0.688192in}{0.670138in}}%
\pgfpathlineto{\pgfqpoint{0.688192in}{5.731668in}}%
\pgfusepath{stroke}%
\end{pgfscope}%
\begin{pgfscope}%
\pgfsetrectcap%
\pgfsetmiterjoin%
\pgfsetlinewidth{0.803000pt}%
\definecolor{currentstroke}{rgb}{0.000000,0.000000,0.000000}%
\pgfsetstrokecolor{currentstroke}%
\pgfsetdash{}{0pt}%
\pgfpathmoveto{\pgfqpoint{7.800000in}{0.670138in}}%
\pgfpathlineto{\pgfqpoint{7.800000in}{5.731668in}}%
\pgfusepath{stroke}%
\end{pgfscope}%
\begin{pgfscope}%
\pgfsetrectcap%
\pgfsetmiterjoin%
\pgfsetlinewidth{0.803000pt}%
\definecolor{currentstroke}{rgb}{0.000000,0.000000,0.000000}%
\pgfsetstrokecolor{currentstroke}%
\pgfsetdash{}{0pt}%
\pgfpathmoveto{\pgfqpoint{0.688192in}{0.670138in}}%
\pgfpathlineto{\pgfqpoint{7.800000in}{0.670138in}}%
\pgfusepath{stroke}%
\end{pgfscope}%
\begin{pgfscope}%
\pgfsetrectcap%
\pgfsetmiterjoin%
\pgfsetlinewidth{0.803000pt}%
\definecolor{currentstroke}{rgb}{0.000000,0.000000,0.000000}%
\pgfsetstrokecolor{currentstroke}%
\pgfsetdash{}{0pt}%
\pgfpathmoveto{\pgfqpoint{0.688192in}{5.731668in}}%
\pgfpathlineto{\pgfqpoint{7.800000in}{5.731668in}}%
\pgfusepath{stroke}%
\end{pgfscope}%
\begin{pgfscope}%
\pgfsetbuttcap%
\pgfsetmiterjoin%
\pgfsetlinewidth{1.003750pt}%
\definecolor{currentstroke}{rgb}{0.000000,0.000000,0.000000}%
\pgfsetstrokecolor{currentstroke}%
\pgfsetstrokeopacity{0.500000}%
\pgfsetdash{}{0pt}%
\pgfpathmoveto{\pgfqpoint{0.640417in}{1.109781in}}%
\pgfpathlineto{\pgfqpoint{0.828032in}{1.109781in}}%
\pgfpathlineto{\pgfqpoint{0.828032in}{1.508065in}}%
\pgfpathlineto{\pgfqpoint{0.640417in}{1.508065in}}%
\pgfpathlineto{\pgfqpoint{0.640417in}{1.109781in}}%
\pgfpathclose%
\pgfpathmoveto{\pgfqpoint{4.244096in}{5.579823in}}%
\pgfpathquadraticcurveto{\pgfqpoint{2.442256in}{3.543944in}}{\pgfqpoint{0.640417in}{1.508065in}}%
\pgfpathmoveto{\pgfqpoint{7.586646in}{3.200903in}}%
\pgfpathquadraticcurveto{\pgfqpoint{4.207339in}{2.155342in}}{\pgfqpoint{0.828032in}{1.109781in}}%
\pgfusepath{stroke}%
\end{pgfscope}%
\begin{pgfscope}%
\pgfsetbuttcap%
\pgfsetmiterjoin%
\definecolor{currentfill}{rgb}{1.000000,1.000000,1.000000}%
\pgfsetfillcolor{currentfill}%
\pgfsetlinewidth{0.000000pt}%
\definecolor{currentstroke}{rgb}{0.000000,0.000000,0.000000}%
\pgfsetstrokecolor{currentstroke}%
\pgfsetstrokeopacity{0.000000}%
\pgfsetdash{}{0pt}%
\pgfpathmoveto{\pgfqpoint{4.244096in}{3.200903in}}%
\pgfpathlineto{\pgfqpoint{7.586646in}{3.200903in}}%
\pgfpathlineto{\pgfqpoint{7.586646in}{5.579823in}}%
\pgfpathlineto{\pgfqpoint{4.244096in}{5.579823in}}%
\pgfpathlineto{\pgfqpoint{4.244096in}{3.200903in}}%
\pgfpathclose%
\pgfusepath{fill}%
\end{pgfscope}%
\begin{pgfscope}%
\pgfpathrectangle{\pgfqpoint{4.244096in}{3.200903in}}{\pgfqpoint{3.342550in}{2.378919in}}%
\pgfusepath{clip}%
\pgfsetbuttcap%
\pgfsetmiterjoin%
\definecolor{currentfill}{rgb}{0.501961,0.501961,0.501961}%
\pgfsetfillcolor{currentfill}%
\pgfsetfillopacity{0.500000}%
\pgfsetlinewidth{1.003750pt}%
\definecolor{currentstroke}{rgb}{0.501961,0.501961,0.501961}%
\pgfsetstrokecolor{currentstroke}%
\pgfsetstrokeopacity{0.500000}%
\pgfsetdash{}{0pt}%
\pgfpathmoveto{\pgfqpoint{6.183131in}{5.204032in}}%
\pgfpathlineto{\pgfqpoint{6.249561in}{4.054913in}}%
\pgfpathlineto{\pgfqpoint{6.269290in}{3.467400in}}%
\pgfpathlineto{\pgfqpoint{6.389537in}{3.359169in}}%
\pgfpathlineto{\pgfqpoint{6.456979in}{2.636044in}}%
\pgfpathlineto{\pgfqpoint{6.536223in}{2.439219in}}%
\pgfpathlineto{\pgfqpoint{6.575820in}{2.438061in}}%
\pgfpathlineto{\pgfqpoint{6.672999in}{2.197456in}}%
\pgfpathlineto{\pgfqpoint{6.677744in}{2.062234in}}%
\pgfpathlineto{\pgfqpoint{6.730258in}{1.915726in}}%
\pgfpathlineto{\pgfqpoint{6.764468in}{1.880232in}}%
\pgfpathlineto{\pgfqpoint{6.781373in}{1.848654in}}%
\pgfpathlineto{\pgfqpoint{6.782778in}{1.845385in}}%
\pgfpathlineto{\pgfqpoint{6.789324in}{1.648392in}}%
\pgfpathlineto{\pgfqpoint{6.789324in}{1.648392in}}%
\pgfpathlineto{\pgfqpoint{6.864070in}{1.613063in}}%
\pgfpathlineto{\pgfqpoint{6.935536in}{1.571928in}}%
\pgfpathlineto{\pgfqpoint{6.967891in}{1.407934in}}%
\pgfpathlineto{\pgfqpoint{6.968024in}{1.407547in}}%
\pgfpathlineto{\pgfqpoint{7.015571in}{1.356747in}}%
\pgfpathlineto{\pgfqpoint{7.018753in}{1.313843in}}%
\pgfpathlineto{\pgfqpoint{7.104609in}{1.293074in}}%
\pgfpathlineto{\pgfqpoint{7.227669in}{1.288262in}}%
\pgfpathlineto{\pgfqpoint{7.271213in}{1.150259in}}%
\pgfpathlineto{\pgfqpoint{7.376840in}{1.116716in}}%
\pgfpathlineto{\pgfqpoint{7.571657in}{1.110486in}}%
\pgfpathlineto{\pgfqpoint{7.638262in}{1.058114in}}%
\pgfpathlineto{\pgfqpoint{7.706582in}{1.048952in}}%
\pgfpathlineto{\pgfqpoint{7.752811in}{1.042069in}}%
\pgfpathlineto{\pgfqpoint{7.841011in}{1.023902in}}%
\pgfpathlineto{\pgfqpoint{7.847894in}{1.023059in}}%
\pgfpathlineto{\pgfqpoint{7.849942in}{1.022137in}}%
\pgfpathlineto{\pgfqpoint{7.967962in}{0.993325in}}%
\pgfpathlineto{\pgfqpoint{7.977189in}{0.988385in}}%
\pgfpathlineto{\pgfqpoint{7.982834in}{0.982653in}}%
\pgfpathlineto{\pgfqpoint{8.055457in}{0.972283in}}%
\pgfpathlineto{\pgfqpoint{8.370492in}{0.942646in}}%
\pgfpathlineto{\pgfqpoint{8.705529in}{0.901352in}}%
\pgfpathlineto{\pgfqpoint{9.160238in}{0.871890in}}%
\pgfpathlineto{\pgfqpoint{9.576627in}{0.861995in}}%
\pgfpathlineto{\pgfqpoint{9.593600in}{0.848929in}}%
\pgfpathlineto{\pgfqpoint{9.618189in}{0.826140in}}%
\pgfpathlineto{\pgfqpoint{9.676172in}{0.821823in}}%
\pgfpathlineto{\pgfqpoint{9.684133in}{0.818018in}}%
\pgfpathlineto{\pgfqpoint{9.692223in}{0.817932in}}%
\pgfpathlineto{\pgfqpoint{9.841496in}{0.814346in}}%
\pgfpathlineto{\pgfqpoint{9.917418in}{0.807146in}}%
\pgfpathlineto{\pgfqpoint{9.948382in}{0.804628in}}%
\pgfpathlineto{\pgfqpoint{10.423157in}{0.789533in}}%
\pgfpathlineto{\pgfqpoint{10.485339in}{0.781523in}}%
\pgfpathlineto{\pgfqpoint{10.487167in}{0.772911in}}%
\pgfpathlineto{\pgfqpoint{10.589219in}{0.771488in}}%
\pgfpathlineto{\pgfqpoint{10.830360in}{0.757023in}}%
\pgfpathlineto{\pgfqpoint{10.848103in}{0.755610in}}%
\pgfpathlineto{\pgfqpoint{10.944464in}{0.755552in}}%
\pgfpathlineto{\pgfqpoint{10.962113in}{0.750156in}}%
\pgfpathlineto{\pgfqpoint{11.912222in}{0.736291in}}%
\pgfpathlineto{\pgfqpoint{11.986609in}{0.709217in}}%
\pgfpathlineto{\pgfqpoint{12.028667in}{0.707130in}}%
\pgfpathlineto{\pgfqpoint{12.028870in}{0.707130in}}%
\pgfpathlineto{\pgfqpoint{12.149871in}{0.705225in}}%
\pgfpathlineto{\pgfqpoint{12.179313in}{0.703504in}}%
\pgfpathlineto{\pgfqpoint{12.477058in}{0.689705in}}%
\pgfpathlineto{\pgfqpoint{12.490844in}{0.688509in}}%
\pgfpathlineto{\pgfqpoint{12.584840in}{0.686767in}}%
\pgfpathlineto{\pgfqpoint{12.585533in}{0.686190in}}%
\pgfpathlineto{\pgfqpoint{12.610134in}{0.682975in}}%
\pgfpathlineto{\pgfqpoint{12.685729in}{0.681799in}}%
\pgfpathlineto{\pgfqpoint{12.730674in}{0.681670in}}%
\pgfpathlineto{\pgfqpoint{12.911389in}{0.676667in}}%
\pgfpathlineto{\pgfqpoint{12.941669in}{0.672708in}}%
\pgfpathlineto{\pgfqpoint{12.946295in}{0.672668in}}%
\pgfpathlineto{\pgfqpoint{13.281821in}{0.672388in}}%
\pgfpathlineto{\pgfqpoint{13.304894in}{0.669039in}}%
\pgfpathlineto{\pgfqpoint{13.351761in}{0.668359in}}%
\pgfpathlineto{\pgfqpoint{13.490100in}{0.667448in}}%
\pgfpathlineto{\pgfqpoint{13.573499in}{0.667212in}}%
\pgfpathlineto{\pgfqpoint{13.821810in}{0.666382in}}%
\pgfpathlineto{\pgfqpoint{14.161557in}{0.666117in}}%
\pgfpathlineto{\pgfqpoint{14.195159in}{0.666116in}}%
\pgfpathlineto{\pgfqpoint{14.216394in}{0.666071in}}%
\pgfpathlineto{\pgfqpoint{14.586512in}{0.666039in}}%
\pgfpathlineto{\pgfqpoint{16.273377in}{0.662779in}}%
\pgfpathlineto{\pgfqpoint{16.603212in}{0.660831in}}%
\pgfpathlineto{\pgfqpoint{17.122676in}{0.660240in}}%
\pgfpathlineto{\pgfqpoint{17.137897in}{0.660151in}}%
\pgfpathlineto{\pgfqpoint{17.353940in}{0.659476in}}%
\pgfpathlineto{\pgfqpoint{17.716115in}{0.659004in}}%
\pgfpathlineto{\pgfqpoint{17.775411in}{0.658636in}}%
\pgfpathlineto{\pgfqpoint{17.902085in}{0.658159in}}%
\pgfpathlineto{\pgfqpoint{18.564983in}{0.657947in}}%
\pgfpathlineto{\pgfqpoint{18.768429in}{0.657657in}}%
\pgfpathlineto{\pgfqpoint{19.013470in}{0.657511in}}%
\pgfpathlineto{\pgfqpoint{19.070619in}{0.657497in}}%
\pgfpathlineto{\pgfqpoint{19.087175in}{0.657429in}}%
\pgfpathlineto{\pgfqpoint{19.093845in}{0.657388in}}%
\pgfpathlineto{\pgfqpoint{19.099542in}{0.657351in}}%
\pgfpathlineto{\pgfqpoint{19.129200in}{0.657212in}}%
\pgfpathlineto{\pgfqpoint{19.154518in}{0.656992in}}%
\pgfpathlineto{\pgfqpoint{19.485643in}{0.655036in}}%
\pgfpathlineto{\pgfqpoint{19.614987in}{0.654449in}}%
\pgfpathlineto{\pgfqpoint{21.028438in}{0.653442in}}%
\pgfpathlineto{\pgfqpoint{22.388020in}{0.653421in}}%
\pgfpathlineto{\pgfqpoint{22.407524in}{0.652526in}}%
\pgfpathlineto{\pgfqpoint{22.466045in}{0.652155in}}%
\pgfpathlineto{\pgfqpoint{22.677220in}{0.651758in}}%
\pgfpathlineto{\pgfqpoint{22.733327in}{0.651029in}}%
\pgfpathlineto{\pgfqpoint{22.798021in}{0.650730in}}%
\pgfpathlineto{\pgfqpoint{22.850812in}{0.650473in}}%
\pgfpathlineto{\pgfqpoint{23.463873in}{0.650116in}}%
\pgfpathlineto{\pgfqpoint{23.674988in}{0.649071in}}%
\pgfpathlineto{\pgfqpoint{23.683338in}{0.649020in}}%
\pgfpathlineto{\pgfqpoint{23.683338in}{0.649020in}}%
\pgfpathlineto{\pgfqpoint{23.686514in}{0.649003in}}%
\pgfpathlineto{\pgfqpoint{23.693730in}{0.648957in}}%
\pgfpathlineto{\pgfqpoint{23.798767in}{0.647701in}}%
\pgfpathlineto{\pgfqpoint{23.798767in}{0.647701in}}%
\pgfpathlineto{\pgfqpoint{24.104536in}{0.647350in}}%
\pgfpathlineto{\pgfqpoint{24.190103in}{0.645882in}}%
\pgfpathlineto{\pgfqpoint{24.213483in}{0.645727in}}%
\pgfpathlineto{\pgfqpoint{24.285755in}{0.645454in}}%
\pgfpathlineto{\pgfqpoint{25.106846in}{0.645357in}}%
\pgfpathlineto{\pgfqpoint{25.106846in}{0.645357in}}%
\pgfpathlineto{\pgfqpoint{25.180730in}{0.645205in}}%
\pgfpathlineto{\pgfqpoint{25.383339in}{0.644286in}}%
\pgfpathlineto{\pgfqpoint{25.829983in}{0.643473in}}%
\pgfpathlineto{\pgfqpoint{26.543641in}{0.642604in}}%
\pgfpathlineto{\pgfqpoint{26.742367in}{0.641508in}}%
\pgfpathlineto{\pgfqpoint{27.192755in}{0.640852in}}%
\pgfpathlineto{\pgfqpoint{27.883961in}{0.640314in}}%
\pgfpathlineto{\pgfqpoint{27.948008in}{0.639163in}}%
\pgfpathlineto{\pgfqpoint{29.594915in}{0.636927in}}%
\pgfpathlineto{\pgfqpoint{32.475120in}{0.633968in}}%
\pgfpathlineto{\pgfqpoint{32.669427in}{0.631796in}}%
\pgfpathlineto{\pgfqpoint{32.691227in}{0.631731in}}%
\pgfpathlineto{\pgfqpoint{32.808700in}{0.631651in}}%
\pgfpathlineto{\pgfqpoint{32.830192in}{0.631549in}}%
\pgfpathlineto{\pgfqpoint{33.189460in}{0.631207in}}%
\pgfpathlineto{\pgfqpoint{35.205583in}{0.628188in}}%
\pgfpathlineto{\pgfqpoint{35.465921in}{0.626984in}}%
\pgfpathlineto{\pgfqpoint{36.115103in}{0.625037in}}%
\pgfpathlineto{\pgfqpoint{36.528089in}{0.624750in}}%
\pgfpathlineto{\pgfqpoint{37.183078in}{0.624698in}}%
\pgfpathlineto{\pgfqpoint{37.270027in}{0.624274in}}%
\pgfpathlineto{\pgfqpoint{37.549244in}{0.622731in}}%
\pgfpathlineto{\pgfqpoint{37.556982in}{0.622673in}}%
\pgfpathlineto{\pgfqpoint{38.135042in}{0.621562in}}%
\pgfpathlineto{\pgfqpoint{39.307371in}{0.620988in}}%
\pgfpathlineto{\pgfqpoint{40.450708in}{0.618305in}}%
\pgfpathlineto{\pgfqpoint{40.533336in}{0.618020in}}%
\pgfpathlineto{\pgfqpoint{44.703339in}{0.615663in}}%
\pgfpathlineto{\pgfqpoint{45.079133in}{0.612571in}}%
\pgfpathlineto{\pgfqpoint{45.861065in}{0.610979in}}%
\pgfpathlineto{\pgfqpoint{46.719722in}{0.609743in}}%
\pgfpathlineto{\pgfqpoint{51.287641in}{0.604750in}}%
\pgfpathlineto{\pgfqpoint{51.482064in}{0.603514in}}%
\pgfpathlineto{\pgfqpoint{56.259615in}{0.597141in}}%
\pgfpathlineto{\pgfqpoint{60.366413in}{0.594494in}}%
\pgfpathlineto{\pgfqpoint{60.991810in}{0.593591in}}%
\pgfpathlineto{\pgfqpoint{61.669359in}{0.592787in}}%
\pgfpathlineto{\pgfqpoint{63.865471in}{0.590530in}}%
\pgfpathlineto{\pgfqpoint{64.026773in}{0.590319in}}%
\pgfpathlineto{\pgfqpoint{66.863188in}{0.588176in}}%
\pgfpathlineto{\pgfqpoint{67.788674in}{0.587750in}}%
\pgfpathlineto{\pgfqpoint{67.897307in}{0.587653in}}%
\pgfpathlineto{\pgfqpoint{71.593556in}{0.587596in}}%
\pgfpathlineto{\pgfqpoint{71.599242in}{0.587587in}}%
\pgfpathlineto{\pgfqpoint{72.854923in}{0.587388in}}%
\pgfpathlineto{\pgfqpoint{74.376808in}{0.586926in}}%
\pgfpathlineto{\pgfqpoint{75.032298in}{0.586847in}}%
\pgfpathlineto{\pgfqpoint{75.342338in}{0.585741in}}%
\pgfpathlineto{\pgfqpoint{75.891260in}{0.585645in}}%
\pgfpathlineto{\pgfqpoint{76.365395in}{0.584411in}}%
\pgfpathlineto{\pgfqpoint{77.044347in}{0.584036in}}%
\pgfpathlineto{\pgfqpoint{77.196610in}{0.583453in}}%
\pgfpathlineto{\pgfqpoint{77.477002in}{0.583220in}}%
\pgfpathlineto{\pgfqpoint{77.713773in}{0.582984in}}%
\pgfpathlineto{\pgfqpoint{80.391093in}{0.582743in}}%
\pgfpathlineto{\pgfqpoint{80.572547in}{0.582560in}}%
\pgfpathlineto{\pgfqpoint{80.839129in}{0.582522in}}%
\pgfpathlineto{\pgfqpoint{81.292905in}{0.582281in}}%
\pgfpathlineto{\pgfqpoint{81.546160in}{0.582001in}}%
\pgfpathlineto{\pgfqpoint{81.819957in}{0.581669in}}%
\pgfpathlineto{\pgfqpoint{83.436786in}{0.581017in}}%
\pgfpathlineto{\pgfqpoint{84.322673in}{0.580784in}}%
\pgfpathlineto{\pgfqpoint{84.426126in}{0.580047in}}%
\pgfpathlineto{\pgfqpoint{84.761016in}{0.579228in}}%
\pgfpathlineto{\pgfqpoint{86.806617in}{0.579090in}}%
\pgfpathlineto{\pgfqpoint{86.806617in}{0.579090in}}%
\pgfpathlineto{\pgfqpoint{86.957300in}{0.578813in}}%
\pgfpathlineto{\pgfqpoint{88.325766in}{0.578362in}}%
\pgfpathlineto{\pgfqpoint{88.521870in}{0.577860in}}%
\pgfpathlineto{\pgfqpoint{91.308507in}{0.577711in}}%
\pgfpathlineto{\pgfqpoint{91.841938in}{0.577258in}}%
\pgfpathlineto{\pgfqpoint{92.858407in}{0.577137in}}%
\pgfpathlineto{\pgfqpoint{93.623279in}{0.576803in}}%
\pgfpathlineto{\pgfqpoint{94.641142in}{0.576675in}}%
\pgfpathlineto{\pgfqpoint{96.882572in}{0.575594in}}%
\pgfpathlineto{\pgfqpoint{99.808438in}{0.574944in}}%
\pgfpathlineto{\pgfqpoint{131.798882in}{0.662193in}}%
\pgfpathlineto{\pgfqpoint{128.580430in}{0.662909in}}%
\pgfpathlineto{\pgfqpoint{126.114856in}{0.664098in}}%
\pgfpathlineto{\pgfqpoint{124.995207in}{0.664239in}}%
\pgfpathlineto{\pgfqpoint{124.153848in}{0.664606in}}%
\pgfpathlineto{\pgfqpoint{123.035732in}{0.664740in}}%
\pgfpathlineto{\pgfqpoint{122.448957in}{0.665238in}}%
\pgfpathlineto{\pgfqpoint{119.383657in}{0.665401in}}%
\pgfpathlineto{\pgfqpoint{119.167943in}{0.665954in}}%
\pgfpathlineto{\pgfqpoint{117.662630in}{0.666449in}}%
\pgfpathlineto{\pgfqpoint{117.496879in}{0.666754in}}%
\pgfpathlineto{\pgfqpoint{117.496879in}{0.666754in}}%
\pgfpathlineto{\pgfqpoint{115.246717in}{0.666906in}}%
\pgfpathlineto{\pgfqpoint{114.878338in}{0.667807in}}%
\pgfpathlineto{\pgfqpoint{114.764540in}{0.668618in}}%
\pgfpathlineto{\pgfqpoint{113.790065in}{0.668874in}}%
\pgfpathlineto{\pgfqpoint{112.011552in}{0.669591in}}%
\pgfpathlineto{\pgfqpoint{111.710376in}{0.669956in}}%
\pgfpathlineto{\pgfqpoint{111.431796in}{0.670264in}}%
\pgfpathlineto{\pgfqpoint{110.932643in}{0.670529in}}%
\pgfpathlineto{\pgfqpoint{110.639402in}{0.670571in}}%
\pgfpathlineto{\pgfqpoint{110.439803in}{0.670772in}}%
\pgfpathlineto{\pgfqpoint{107.494751in}{0.671038in}}%
\pgfpathlineto{\pgfqpoint{107.234302in}{0.671298in}}%
\pgfpathlineto{\pgfqpoint{106.925871in}{0.671553in}}%
\pgfpathlineto{\pgfqpoint{106.758382in}{0.672195in}}%
\pgfpathlineto{\pgfqpoint{106.011535in}{0.672607in}}%
\pgfpathlineto{\pgfqpoint{105.489986in}{0.673965in}}%
\pgfpathlineto{\pgfqpoint{104.886172in}{0.674070in}}%
\pgfpathlineto{\pgfqpoint{104.545128in}{0.675287in}}%
\pgfpathlineto{\pgfqpoint{103.824089in}{0.675374in}}%
\pgfpathlineto{\pgfqpoint{102.150016in}{0.675883in}}%
\pgfpathlineto{\pgfqpoint{100.768767in}{0.676101in}}%
\pgfpathlineto{\pgfqpoint{100.762512in}{0.676111in}}%
\pgfpathlineto{\pgfqpoint{96.696638in}{0.676173in}}%
\pgfpathlineto{\pgfqpoint{96.577141in}{0.676281in}}%
\pgfpathlineto{\pgfqpoint{95.559107in}{0.676749in}}%
\pgfpathlineto{\pgfqpoint{92.439051in}{0.679106in}}%
\pgfpathlineto{\pgfqpoint{92.261618in}{0.679338in}}%
\pgfpathlineto{\pgfqpoint{89.845895in}{0.681821in}}%
\pgfpathlineto{\pgfqpoint{89.100591in}{0.682705in}}%
\pgfpathlineto{\pgfqpoint{88.412654in}{0.683699in}}%
\pgfpathlineto{\pgfqpoint{83.895177in}{0.686611in}}%
\pgfpathlineto{\pgfqpoint{78.639871in}{0.693620in}}%
\pgfpathlineto{\pgfqpoint{78.426006in}{0.694980in}}%
\pgfpathlineto{\pgfqpoint{73.401295in}{0.700472in}}%
\pgfpathlineto{\pgfqpoint{72.456771in}{0.701832in}}%
\pgfpathlineto{\pgfqpoint{71.596647in}{0.703583in}}%
\pgfpathlineto{\pgfqpoint{71.183273in}{0.706985in}}%
\pgfpathlineto{\pgfqpoint{66.596270in}{0.709577in}}%
\pgfpathlineto{\pgfqpoint{66.505379in}{0.709891in}}%
\pgfpathlineto{\pgfqpoint{65.247709in}{0.712842in}}%
\pgfpathlineto{\pgfqpoint{63.958146in}{0.713474in}}%
\pgfpathlineto{\pgfqpoint{63.322280in}{0.714695in}}%
\pgfpathlineto{\pgfqpoint{63.313768in}{0.714760in}}%
\pgfpathlineto{\pgfqpoint{63.006630in}{0.716456in}}%
\pgfpathlineto{\pgfqpoint{62.910986in}{0.716923in}}%
\pgfpathlineto{\pgfqpoint{62.190498in}{0.716981in}}%
\pgfpathlineto{\pgfqpoint{61.736213in}{0.717296in}}%
\pgfpathlineto{\pgfqpoint{61.022113in}{0.719438in}}%
\pgfpathlineto{\pgfqpoint{60.735741in}{0.720762in}}%
\pgfpathlineto{\pgfqpoint{58.518006in}{0.724082in}}%
\pgfpathlineto{\pgfqpoint{58.122812in}{0.724459in}}%
\pgfpathlineto{\pgfqpoint{58.099171in}{0.724572in}}%
\pgfpathlineto{\pgfqpoint{57.969950in}{0.724659in}}%
\pgfpathlineto{\pgfqpoint{57.945970in}{0.724731in}}%
\pgfpathlineto{\pgfqpoint{57.732232in}{0.727120in}}%
\pgfpathlineto{\pgfqpoint{54.564007in}{0.730375in}}%
\pgfpathlineto{\pgfqpoint{52.752409in}{0.732835in}}%
\pgfpathlineto{\pgfqpoint{52.681958in}{0.734101in}}%
\pgfpathlineto{\pgfqpoint{51.921631in}{0.734692in}}%
\pgfpathlineto{\pgfqpoint{51.426204in}{0.735415in}}%
\pgfpathlineto{\pgfqpoint{51.207605in}{0.736620in}}%
\pgfpathlineto{\pgfqpoint{50.422582in}{0.737576in}}%
\pgfpathlineto{\pgfqpoint{49.931274in}{0.738470in}}%
\pgfpathlineto{\pgfqpoint{49.708403in}{0.739480in}}%
\pgfpathlineto{\pgfqpoint{49.627131in}{0.739648in}}%
\pgfpathlineto{\pgfqpoint{49.627131in}{0.739648in}}%
\pgfpathlineto{\pgfqpoint{48.723931in}{0.739755in}}%
\pgfpathlineto{\pgfqpoint{48.644432in}{0.740055in}}%
\pgfpathlineto{\pgfqpoint{48.618714in}{0.740226in}}%
\pgfpathlineto{\pgfqpoint{48.524590in}{0.741840in}}%
\pgfpathlineto{\pgfqpoint{48.188244in}{0.742226in}}%
\pgfpathlineto{\pgfqpoint{48.188244in}{0.742226in}}%
\pgfpathlineto{\pgfqpoint{48.072703in}{0.743608in}}%
\pgfpathlineto{\pgfqpoint{48.064766in}{0.743658in}}%
\pgfpathlineto{\pgfqpoint{48.061272in}{0.743678in}}%
\pgfpathlineto{\pgfqpoint{48.061272in}{0.743678in}}%
\pgfpathlineto{\pgfqpoint{48.052087in}{0.743734in}}%
\pgfpathlineto{\pgfqpoint{47.819861in}{0.744883in}}%
\pgfpathlineto{\pgfqpoint{47.145493in}{0.745276in}}%
\pgfpathlineto{\pgfqpoint{47.087424in}{0.745559in}}%
\pgfpathlineto{\pgfqpoint{47.016260in}{0.745887in}}%
\pgfpathlineto{\pgfqpoint{46.954542in}{0.746689in}}%
\pgfpathlineto{\pgfqpoint{46.722249in}{0.747126in}}%
\pgfpathlineto{\pgfqpoint{46.657877in}{0.747534in}}%
\pgfpathlineto{\pgfqpoint{46.636422in}{0.748518in}}%
\pgfpathlineto{\pgfqpoint{45.140882in}{0.748541in}}%
\pgfpathlineto{\pgfqpoint{43.586085in}{0.749650in}}%
\pgfpathlineto{\pgfqpoint{43.443808in}{0.750294in}}%
\pgfpathlineto{\pgfqpoint{43.079570in}{0.752447in}}%
\pgfpathlineto{\pgfqpoint{43.051721in}{0.752689in}}%
\pgfpathlineto{\pgfqpoint{43.019096in}{0.752842in}}%
\pgfpathlineto{\pgfqpoint{43.012830in}{0.752882in}}%
\pgfpathlineto{\pgfqpoint{43.005493in}{0.752928in}}%
\pgfpathlineto{\pgfqpoint{42.987281in}{0.753002in}}%
\pgfpathlineto{\pgfqpoint{42.924417in}{0.753017in}}%
\pgfpathlineto{\pgfqpoint{42.654872in}{0.753178in}}%
\pgfpathlineto{\pgfqpoint{42.431081in}{0.753497in}}%
\pgfpathlineto{\pgfqpoint{41.701894in}{0.753731in}}%
\pgfpathlineto{\pgfqpoint{41.562553in}{0.754255in}}%
\pgfpathlineto{\pgfqpoint{41.497327in}{0.754660in}}%
\pgfpathlineto{\pgfqpoint{41.098935in}{0.755179in}}%
\pgfpathlineto{\pgfqpoint{40.861287in}{0.755921in}}%
\pgfpathlineto{\pgfqpoint{40.844544in}{0.756019in}}%
\pgfpathlineto{\pgfqpoint{40.273133in}{0.756669in}}%
\pgfpathlineto{\pgfqpoint{39.910315in}{0.758812in}}%
\pgfpathlineto{\pgfqpoint{38.054764in}{0.762399in}}%
\pgfpathlineto{\pgfqpoint{37.647634in}{0.762433in}}%
\pgfpathlineto{\pgfqpoint{37.624276in}{0.762483in}}%
\pgfpathlineto{\pgfqpoint{37.587313in}{0.762484in}}%
\pgfpathlineto{\pgfqpoint{37.213591in}{0.762775in}}%
\pgfpathlineto{\pgfqpoint{36.940449in}{0.763689in}}%
\pgfpathlineto{\pgfqpoint{36.848710in}{0.763949in}}%
\pgfpathlineto{\pgfqpoint{36.696537in}{0.764950in}}%
\pgfpathlineto{\pgfqpoint{36.644984in}{0.765698in}}%
\pgfpathlineto{\pgfqpoint{36.619604in}{0.769382in}}%
\pgfpathlineto{\pgfqpoint{36.250524in}{0.769690in}}%
\pgfpathlineto{\pgfqpoint{36.245437in}{0.769734in}}%
\pgfpathlineto{\pgfqpoint{36.212129in}{0.774089in}}%
\pgfpathlineto{\pgfqpoint{36.013342in}{0.779592in}}%
\pgfpathlineto{\pgfqpoint{35.963902in}{0.779734in}}%
\pgfpathlineto{\pgfqpoint{35.880747in}{0.781028in}}%
\pgfpathlineto{\pgfqpoint{35.853687in}{0.784564in}}%
\pgfpathlineto{\pgfqpoint{35.852925in}{0.785199in}}%
\pgfpathlineto{\pgfqpoint{35.749529in}{0.787115in}}%
\pgfpathlineto{\pgfqpoint{35.734364in}{0.788431in}}%
\pgfpathlineto{\pgfqpoint{35.406844in}{0.803609in}}%
\pgfpathlineto{\pgfqpoint{35.374458in}{0.805503in}}%
\pgfpathlineto{\pgfqpoint{35.241357in}{0.807599in}}%
\pgfpathlineto{\pgfqpoint{35.241134in}{0.807599in}}%
\pgfpathlineto{\pgfqpoint{35.194870in}{0.809894in}}%
\pgfpathlineto{\pgfqpoint{35.113045in}{0.839675in}}%
\pgfpathlineto{\pgfqpoint{34.067924in}{0.854927in}}%
\pgfpathlineto{\pgfqpoint{34.048511in}{0.860862in}}%
\pgfpathlineto{\pgfqpoint{33.942513in}{0.860926in}}%
\pgfpathlineto{\pgfqpoint{33.922997in}{0.862481in}}%
\pgfpathlineto{\pgfqpoint{33.657741in}{0.878392in}}%
\pgfpathlineto{\pgfqpoint{33.545484in}{0.879957in}}%
\pgfpathlineto{\pgfqpoint{33.543473in}{0.889430in}}%
\pgfpathlineto{\pgfqpoint{33.475073in}{0.898242in}}%
\pgfpathlineto{\pgfqpoint{32.952821in}{0.914846in}}%
\pgfpathlineto{\pgfqpoint{32.918760in}{0.917616in}}%
\pgfpathlineto{\pgfqpoint{32.835245in}{0.925536in}}%
\pgfpathlineto{\pgfqpoint{32.671046in}{0.929481in}}%
\pgfpathlineto{\pgfqpoint{32.662146in}{0.929575in}}%
\pgfpathlineto{\pgfqpoint{32.653389in}{0.933761in}}%
\pgfpathlineto{\pgfqpoint{32.589609in}{0.938509in}}%
\pgfpathlineto{\pgfqpoint{32.562560in}{0.963578in}}%
\pgfpathlineto{\pgfqpoint{32.543890in}{0.977950in}}%
\pgfpathlineto{\pgfqpoint{32.085862in}{0.988834in}}%
\pgfpathlineto{\pgfqpoint{31.585682in}{1.021242in}}%
\pgfpathlineto{\pgfqpoint{31.217142in}{1.066666in}}%
\pgfpathlineto{\pgfqpoint{30.870603in}{1.099266in}}%
\pgfpathlineto{\pgfqpoint{30.790717in}{1.110673in}}%
\pgfpathlineto{\pgfqpoint{30.784508in}{1.116979in}}%
\pgfpathlineto{\pgfqpoint{30.774359in}{1.122412in}}%
\pgfpathlineto{\pgfqpoint{30.644536in}{1.154106in}}%
\pgfpathlineto{\pgfqpoint{30.642284in}{1.155120in}}%
\pgfpathlineto{\pgfqpoint{30.634713in}{1.156048in}}%
\pgfpathlineto{\pgfqpoint{30.537693in}{1.176031in}}%
\pgfpathlineto{\pgfqpoint{30.486840in}{1.183602in}}%
\pgfpathlineto{\pgfqpoint{30.411688in}{1.193680in}}%
\pgfpathlineto{\pgfqpoint{30.338423in}{1.251290in}}%
\pgfpathlineto{\pgfqpoint{30.124124in}{1.258142in}}%
\pgfpathlineto{\pgfqpoint{30.007935in}{1.295040in}}%
\pgfpathlineto{\pgfqpoint{29.960037in}{1.446843in}}%
\pgfpathlineto{\pgfqpoint{29.824670in}{1.452137in}}%
\pgfpathlineto{\pgfqpoint{29.730229in}{1.474982in}}%
\pgfpathlineto{\pgfqpoint{29.726728in}{1.522177in}}%
\pgfpathlineto{\pgfqpoint{29.674427in}{1.578058in}}%
\pgfpathlineto{\pgfqpoint{29.674280in}{1.578482in}}%
\pgfpathlineto{\pgfqpoint{29.638690in}{1.758877in}}%
\pgfpathlineto{\pgfqpoint{29.560077in}{1.804124in}}%
\pgfpathlineto{\pgfqpoint{29.477857in}{1.842987in}}%
\pgfpathlineto{\pgfqpoint{29.477857in}{1.842987in}}%
\pgfpathlineto{\pgfqpoint{29.470656in}{2.059679in}}%
\pgfpathlineto{\pgfqpoint{29.469111in}{2.063275in}}%
\pgfpathlineto{\pgfqpoint{29.450516in}{2.098010in}}%
\pgfpathlineto{\pgfqpoint{29.412884in}{2.137054in}}%
\pgfpathlineto{\pgfqpoint{29.355119in}{2.298213in}}%
\pgfpathlineto{\pgfqpoint{29.349899in}{2.446957in}}%
\pgfpathlineto{\pgfqpoint{29.243003in}{2.711622in}}%
\pgfpathlineto{\pgfqpoint{29.199446in}{2.712896in}}%
\pgfpathlineto{\pgfqpoint{29.112277in}{2.929403in}}%
\pgfpathlineto{\pgfqpoint{29.038091in}{3.724842in}}%
\pgfpathlineto{\pgfqpoint{28.905819in}{3.843895in}}%
\pgfpathlineto{\pgfqpoint{28.884118in}{4.490159in}}%
\pgfpathlineto{\pgfqpoint{28.811045in}{5.754191in}}%
\pgfpathlineto{\pgfqpoint{6.183131in}{5.204032in}}%
\pgfpathclose%
\pgfusepath{stroke,fill}%
\end{pgfscope}%
\begin{pgfscope}%
\pgfpathrectangle{\pgfqpoint{4.244096in}{3.200903in}}{\pgfqpoint{3.342550in}{2.378919in}}%
\pgfusepath{clip}%
\pgfsetbuttcap%
\pgfsetroundjoin%
\pgfsetlinewidth{1.003750pt}%
\definecolor{currentstroke}{rgb}{1.000000,0.000000,0.000000}%
\pgfsetstrokecolor{currentstroke}%
\pgfsetdash{}{0pt}%
\pgfpathmoveto{\pgfqpoint{19.239472in}{27.948364in}}%
\pgfpathcurveto{\pgfqpoint{19.247709in}{27.948364in}}{\pgfqpoint{19.255609in}{27.951636in}}{\pgfqpoint{19.261432in}{27.957460in}}%
\pgfpathcurveto{\pgfqpoint{19.267256in}{27.963284in}}{\pgfqpoint{19.270529in}{27.971184in}}{\pgfqpoint{19.270529in}{27.979420in}}%
\pgfpathcurveto{\pgfqpoint{19.270529in}{27.987657in}}{\pgfqpoint{19.267256in}{27.995557in}}{\pgfqpoint{19.261432in}{28.001381in}}%
\pgfpathcurveto{\pgfqpoint{19.255609in}{28.007205in}}{\pgfqpoint{19.247709in}{28.010477in}}{\pgfqpoint{19.239472in}{28.010477in}}%
\pgfpathcurveto{\pgfqpoint{19.231236in}{28.010477in}}{\pgfqpoint{19.223336in}{28.007205in}}{\pgfqpoint{19.217512in}{28.001381in}}%
\pgfpathcurveto{\pgfqpoint{19.211688in}{27.995557in}}{\pgfqpoint{19.208416in}{27.987657in}}{\pgfqpoint{19.208416in}{27.979420in}}%
\pgfpathcurveto{\pgfqpoint{19.208416in}{27.971184in}}{\pgfqpoint{19.211688in}{27.963284in}}{\pgfqpoint{19.217512in}{27.957460in}}%
\pgfpathcurveto{\pgfqpoint{19.223336in}{27.951636in}}{\pgfqpoint{19.231236in}{27.948364in}}{\pgfqpoint{19.239472in}{27.948364in}}%
\pgfusepath{stroke}%
\end{pgfscope}%
\begin{pgfscope}%
\pgfpathrectangle{\pgfqpoint{4.244096in}{3.200903in}}{\pgfqpoint{3.342550in}{2.378919in}}%
\pgfusepath{clip}%
\pgfsetbuttcap%
\pgfsetroundjoin%
\pgfsetlinewidth{1.003750pt}%
\definecolor{currentstroke}{rgb}{1.000000,0.000000,0.000000}%
\pgfsetstrokecolor{currentstroke}%
\pgfsetdash{}{0pt}%
\pgfpathmoveto{\pgfqpoint{13.588165in}{11.016463in}}%
\pgfpathcurveto{\pgfqpoint{13.596401in}{11.016463in}}{\pgfqpoint{13.604301in}{11.019735in}}{\pgfqpoint{13.610125in}{11.025559in}}%
\pgfpathcurveto{\pgfqpoint{13.615949in}{11.031383in}}{\pgfqpoint{13.619221in}{11.039283in}}{\pgfqpoint{13.619221in}{11.047520in}}%
\pgfpathcurveto{\pgfqpoint{13.619221in}{11.055756in}}{\pgfqpoint{13.615949in}{11.063656in}}{\pgfqpoint{13.610125in}{11.069480in}}%
\pgfpathcurveto{\pgfqpoint{13.604301in}{11.075304in}}{\pgfqpoint{13.596401in}{11.078576in}}{\pgfqpoint{13.588165in}{11.078576in}}%
\pgfpathcurveto{\pgfqpoint{13.579928in}{11.078576in}}{\pgfqpoint{13.572028in}{11.075304in}}{\pgfqpoint{13.566204in}{11.069480in}}%
\pgfpathcurveto{\pgfqpoint{13.560380in}{11.063656in}}{\pgfqpoint{13.557108in}{11.055756in}}{\pgfqpoint{13.557108in}{11.047520in}}%
\pgfpathcurveto{\pgfqpoint{13.557108in}{11.039283in}}{\pgfqpoint{13.560380in}{11.031383in}}{\pgfqpoint{13.566204in}{11.025559in}}%
\pgfpathcurveto{\pgfqpoint{13.572028in}{11.019735in}}{\pgfqpoint{13.579928in}{11.016463in}}{\pgfqpoint{13.588165in}{11.016463in}}%
\pgfusepath{stroke}%
\end{pgfscope}%
\begin{pgfscope}%
\pgfpathrectangle{\pgfqpoint{4.244096in}{3.200903in}}{\pgfqpoint{3.342550in}{2.378919in}}%
\pgfusepath{clip}%
\pgfsetbuttcap%
\pgfsetroundjoin%
\pgfsetlinewidth{1.003750pt}%
\definecolor{currentstroke}{rgb}{1.000000,0.000000,0.000000}%
\pgfsetstrokecolor{currentstroke}%
\pgfsetdash{}{0pt}%
\pgfpathmoveto{\pgfqpoint{14.457083in}{11.475748in}}%
\pgfpathcurveto{\pgfqpoint{14.465319in}{11.475748in}}{\pgfqpoint{14.473219in}{11.479020in}}{\pgfqpoint{14.479043in}{11.484844in}}%
\pgfpathcurveto{\pgfqpoint{14.484867in}{11.490668in}}{\pgfqpoint{14.488140in}{11.498568in}}{\pgfqpoint{14.488140in}{11.506805in}}%
\pgfpathcurveto{\pgfqpoint{14.488140in}{11.515041in}}{\pgfqpoint{14.484867in}{11.522941in}}{\pgfqpoint{14.479043in}{11.528765in}}%
\pgfpathcurveto{\pgfqpoint{14.473219in}{11.534589in}}{\pgfqpoint{14.465319in}{11.537861in}}{\pgfqpoint{14.457083in}{11.537861in}}%
\pgfpathcurveto{\pgfqpoint{14.448847in}{11.537861in}}{\pgfqpoint{14.440947in}{11.534589in}}{\pgfqpoint{14.435123in}{11.528765in}}%
\pgfpathcurveto{\pgfqpoint{14.429299in}{11.522941in}}{\pgfqpoint{14.426027in}{11.515041in}}{\pgfqpoint{14.426027in}{11.506805in}}%
\pgfpathcurveto{\pgfqpoint{14.426027in}{11.498568in}}{\pgfqpoint{14.429299in}{11.490668in}}{\pgfqpoint{14.435123in}{11.484844in}}%
\pgfpathcurveto{\pgfqpoint{14.440947in}{11.479020in}}{\pgfqpoint{14.448847in}{11.475748in}}{\pgfqpoint{14.457083in}{11.475748in}}%
\pgfusepath{stroke}%
\end{pgfscope}%
\begin{pgfscope}%
\pgfpathrectangle{\pgfqpoint{4.244096in}{3.200903in}}{\pgfqpoint{3.342550in}{2.378919in}}%
\pgfusepath{clip}%
\pgfsetbuttcap%
\pgfsetroundjoin%
\pgfsetlinewidth{1.003750pt}%
\definecolor{currentstroke}{rgb}{1.000000,0.000000,0.000000}%
\pgfsetstrokecolor{currentstroke}%
\pgfsetdash{}{0pt}%
\pgfpathmoveto{\pgfqpoint{17.270649in}{11.656531in}}%
\pgfpathcurveto{\pgfqpoint{17.278886in}{11.656531in}}{\pgfqpoint{17.286786in}{11.659804in}}{\pgfqpoint{17.292610in}{11.665627in}}%
\pgfpathcurveto{\pgfqpoint{17.298434in}{11.671451in}}{\pgfqpoint{17.301706in}{11.679351in}}{\pgfqpoint{17.301706in}{11.687588in}}%
\pgfpathcurveto{\pgfqpoint{17.301706in}{11.695824in}}{\pgfqpoint{17.298434in}{11.703724in}}{\pgfqpoint{17.292610in}{11.709548in}}%
\pgfpathcurveto{\pgfqpoint{17.286786in}{11.715372in}}{\pgfqpoint{17.278886in}{11.718644in}}{\pgfqpoint{17.270649in}{11.718644in}}%
\pgfpathcurveto{\pgfqpoint{17.262413in}{11.718644in}}{\pgfqpoint{17.254513in}{11.715372in}}{\pgfqpoint{17.248689in}{11.709548in}}%
\pgfpathcurveto{\pgfqpoint{17.242865in}{11.703724in}}{\pgfqpoint{17.239593in}{11.695824in}}{\pgfqpoint{17.239593in}{11.687588in}}%
\pgfpathcurveto{\pgfqpoint{17.239593in}{11.679351in}}{\pgfqpoint{17.242865in}{11.671451in}}{\pgfqpoint{17.248689in}{11.665627in}}%
\pgfpathcurveto{\pgfqpoint{17.254513in}{11.659804in}}{\pgfqpoint{17.262413in}{11.656531in}}{\pgfqpoint{17.270649in}{11.656531in}}%
\pgfusepath{stroke}%
\end{pgfscope}%
\begin{pgfscope}%
\pgfpathrectangle{\pgfqpoint{4.244096in}{3.200903in}}{\pgfqpoint{3.342550in}{2.378919in}}%
\pgfusepath{clip}%
\pgfsetbuttcap%
\pgfsetroundjoin%
\pgfsetlinewidth{1.003750pt}%
\definecolor{currentstroke}{rgb}{1.000000,0.000000,0.000000}%
\pgfsetstrokecolor{currentstroke}%
\pgfsetdash{}{0pt}%
\pgfpathmoveto{\pgfqpoint{15.432888in}{12.453822in}}%
\pgfpathcurveto{\pgfqpoint{15.441124in}{12.453822in}}{\pgfqpoint{15.449024in}{12.457094in}}{\pgfqpoint{15.454848in}{12.462918in}}%
\pgfpathcurveto{\pgfqpoint{15.460672in}{12.468742in}}{\pgfqpoint{15.463944in}{12.476642in}}{\pgfqpoint{15.463944in}{12.484878in}}%
\pgfpathcurveto{\pgfqpoint{15.463944in}{12.493115in}}{\pgfqpoint{15.460672in}{12.501015in}}{\pgfqpoint{15.454848in}{12.506839in}}%
\pgfpathcurveto{\pgfqpoint{15.449024in}{12.512662in}}{\pgfqpoint{15.441124in}{12.515935in}}{\pgfqpoint{15.432888in}{12.515935in}}%
\pgfpathcurveto{\pgfqpoint{15.424651in}{12.515935in}}{\pgfqpoint{15.416751in}{12.512662in}}{\pgfqpoint{15.410927in}{12.506839in}}%
\pgfpathcurveto{\pgfqpoint{15.405104in}{12.501015in}}{\pgfqpoint{15.401831in}{12.493115in}}{\pgfqpoint{15.401831in}{12.484878in}}%
\pgfpathcurveto{\pgfqpoint{15.401831in}{12.476642in}}{\pgfqpoint{15.405104in}{12.468742in}}{\pgfqpoint{15.410927in}{12.462918in}}%
\pgfpathcurveto{\pgfqpoint{15.416751in}{12.457094in}}{\pgfqpoint{15.424651in}{12.453822in}}{\pgfqpoint{15.432888in}{12.453822in}}%
\pgfusepath{stroke}%
\end{pgfscope}%
\begin{pgfscope}%
\pgfpathrectangle{\pgfqpoint{4.244096in}{3.200903in}}{\pgfqpoint{3.342550in}{2.378919in}}%
\pgfusepath{clip}%
\pgfsetbuttcap%
\pgfsetroundjoin%
\pgfsetlinewidth{1.003750pt}%
\definecolor{currentstroke}{rgb}{1.000000,0.000000,0.000000}%
\pgfsetstrokecolor{currentstroke}%
\pgfsetdash{}{0pt}%
\pgfpathmoveto{\pgfqpoint{14.045517in}{9.708619in}}%
\pgfpathcurveto{\pgfqpoint{14.053754in}{9.708619in}}{\pgfqpoint{14.061654in}{9.711892in}}{\pgfqpoint{14.067478in}{9.717716in}}%
\pgfpathcurveto{\pgfqpoint{14.073302in}{9.723540in}}{\pgfqpoint{14.076574in}{9.731440in}}{\pgfqpoint{14.076574in}{9.739676in}}%
\pgfpathcurveto{\pgfqpoint{14.076574in}{9.747912in}}{\pgfqpoint{14.073302in}{9.755812in}}{\pgfqpoint{14.067478in}{9.761636in}}%
\pgfpathcurveto{\pgfqpoint{14.061654in}{9.767460in}}{\pgfqpoint{14.053754in}{9.770732in}}{\pgfqpoint{14.045517in}{9.770732in}}%
\pgfpathcurveto{\pgfqpoint{14.037281in}{9.770732in}}{\pgfqpoint{14.029381in}{9.767460in}}{\pgfqpoint{14.023557in}{9.761636in}}%
\pgfpathcurveto{\pgfqpoint{14.017733in}{9.755812in}}{\pgfqpoint{14.014461in}{9.747912in}}{\pgfqpoint{14.014461in}{9.739676in}}%
\pgfpathcurveto{\pgfqpoint{14.014461in}{9.731440in}}{\pgfqpoint{14.017733in}{9.723540in}}{\pgfqpoint{14.023557in}{9.717716in}}%
\pgfpathcurveto{\pgfqpoint{14.029381in}{9.711892in}}{\pgfqpoint{14.037281in}{9.708619in}}{\pgfqpoint{14.045517in}{9.708619in}}%
\pgfusepath{stroke}%
\end{pgfscope}%
\begin{pgfscope}%
\pgfpathrectangle{\pgfqpoint{4.244096in}{3.200903in}}{\pgfqpoint{3.342550in}{2.378919in}}%
\pgfusepath{clip}%
\pgfsetbuttcap%
\pgfsetroundjoin%
\pgfsetlinewidth{1.003750pt}%
\definecolor{currentstroke}{rgb}{1.000000,0.000000,0.000000}%
\pgfsetstrokecolor{currentstroke}%
\pgfsetdash{}{0pt}%
\pgfpathmoveto{\pgfqpoint{12.975229in}{7.724824in}}%
\pgfpathcurveto{\pgfqpoint{12.983466in}{7.724824in}}{\pgfqpoint{12.991366in}{7.728096in}}{\pgfqpoint{12.997190in}{7.733920in}}%
\pgfpathcurveto{\pgfqpoint{13.003014in}{7.739744in}}{\pgfqpoint{13.006286in}{7.747644in}}{\pgfqpoint{13.006286in}{7.755880in}}%
\pgfpathcurveto{\pgfqpoint{13.006286in}{7.764117in}}{\pgfqpoint{13.003014in}{7.772017in}}{\pgfqpoint{12.997190in}{7.777841in}}%
\pgfpathcurveto{\pgfqpoint{12.991366in}{7.783664in}}{\pgfqpoint{12.983466in}{7.786937in}}{\pgfqpoint{12.975229in}{7.786937in}}%
\pgfpathcurveto{\pgfqpoint{12.966993in}{7.786937in}}{\pgfqpoint{12.959093in}{7.783664in}}{\pgfqpoint{12.953269in}{7.777841in}}%
\pgfpathcurveto{\pgfqpoint{12.947445in}{7.772017in}}{\pgfqpoint{12.944173in}{7.764117in}}{\pgfqpoint{12.944173in}{7.755880in}}%
\pgfpathcurveto{\pgfqpoint{12.944173in}{7.747644in}}{\pgfqpoint{12.947445in}{7.739744in}}{\pgfqpoint{12.953269in}{7.733920in}}%
\pgfpathcurveto{\pgfqpoint{12.959093in}{7.728096in}}{\pgfqpoint{12.966993in}{7.724824in}}{\pgfqpoint{12.975229in}{7.724824in}}%
\pgfusepath{stroke}%
\end{pgfscope}%
\begin{pgfscope}%
\pgfpathrectangle{\pgfqpoint{4.244096in}{3.200903in}}{\pgfqpoint{3.342550in}{2.378919in}}%
\pgfusepath{clip}%
\pgfsetbuttcap%
\pgfsetroundjoin%
\pgfsetlinewidth{1.003750pt}%
\definecolor{currentstroke}{rgb}{1.000000,0.000000,0.000000}%
\pgfsetstrokecolor{currentstroke}%
\pgfsetdash{}{0pt}%
\pgfpathmoveto{\pgfqpoint{12.925500in}{7.640475in}}%
\pgfpathcurveto{\pgfqpoint{12.933736in}{7.640475in}}{\pgfqpoint{12.941637in}{7.643748in}}{\pgfqpoint{12.947460in}{7.649572in}}%
\pgfpathcurveto{\pgfqpoint{12.953284in}{7.655395in}}{\pgfqpoint{12.956557in}{7.663295in}}{\pgfqpoint{12.956557in}{7.671532in}}%
\pgfpathcurveto{\pgfqpoint{12.956557in}{7.679768in}}{\pgfqpoint{12.953284in}{7.687668in}}{\pgfqpoint{12.947460in}{7.693492in}}%
\pgfpathcurveto{\pgfqpoint{12.941637in}{7.699316in}}{\pgfqpoint{12.933736in}{7.702588in}}{\pgfqpoint{12.925500in}{7.702588in}}%
\pgfpathcurveto{\pgfqpoint{12.917264in}{7.702588in}}{\pgfqpoint{12.909364in}{7.699316in}}{\pgfqpoint{12.903540in}{7.693492in}}%
\pgfpathcurveto{\pgfqpoint{12.897716in}{7.687668in}}{\pgfqpoint{12.894444in}{7.679768in}}{\pgfqpoint{12.894444in}{7.671532in}}%
\pgfpathcurveto{\pgfqpoint{12.894444in}{7.663295in}}{\pgfqpoint{12.897716in}{7.655395in}}{\pgfqpoint{12.903540in}{7.649572in}}%
\pgfpathcurveto{\pgfqpoint{12.909364in}{7.643748in}}{\pgfqpoint{12.917264in}{7.640475in}}{\pgfqpoint{12.925500in}{7.640475in}}%
\pgfusepath{stroke}%
\end{pgfscope}%
\begin{pgfscope}%
\pgfpathrectangle{\pgfqpoint{4.244096in}{3.200903in}}{\pgfqpoint{3.342550in}{2.378919in}}%
\pgfusepath{clip}%
\pgfsetbuttcap%
\pgfsetroundjoin%
\pgfsetlinewidth{1.003750pt}%
\definecolor{currentstroke}{rgb}{1.000000,0.000000,0.000000}%
\pgfsetstrokecolor{currentstroke}%
\pgfsetdash{}{0pt}%
\pgfpathmoveto{\pgfqpoint{15.077163in}{10.679539in}}%
\pgfpathcurveto{\pgfqpoint{15.085399in}{10.679539in}}{\pgfqpoint{15.093299in}{10.682811in}}{\pgfqpoint{15.099123in}{10.688635in}}%
\pgfpathcurveto{\pgfqpoint{15.104947in}{10.694459in}}{\pgfqpoint{15.108220in}{10.702359in}}{\pgfqpoint{15.108220in}{10.710595in}}%
\pgfpathcurveto{\pgfqpoint{15.108220in}{10.718831in}}{\pgfqpoint{15.104947in}{10.726731in}}{\pgfqpoint{15.099123in}{10.732555in}}%
\pgfpathcurveto{\pgfqpoint{15.093299in}{10.738379in}}{\pgfqpoint{15.085399in}{10.741652in}}{\pgfqpoint{15.077163in}{10.741652in}}%
\pgfpathcurveto{\pgfqpoint{15.068927in}{10.741652in}}{\pgfqpoint{15.061027in}{10.738379in}}{\pgfqpoint{15.055203in}{10.732555in}}%
\pgfpathcurveto{\pgfqpoint{15.049379in}{10.726731in}}{\pgfqpoint{15.046107in}{10.718831in}}{\pgfqpoint{15.046107in}{10.710595in}}%
\pgfpathcurveto{\pgfqpoint{15.046107in}{10.702359in}}{\pgfqpoint{15.049379in}{10.694459in}}{\pgfqpoint{15.055203in}{10.688635in}}%
\pgfpathcurveto{\pgfqpoint{15.061027in}{10.682811in}}{\pgfqpoint{15.068927in}{10.679539in}}{\pgfqpoint{15.077163in}{10.679539in}}%
\pgfusepath{stroke}%
\end{pgfscope}%
\begin{pgfscope}%
\pgfpathrectangle{\pgfqpoint{4.244096in}{3.200903in}}{\pgfqpoint{3.342550in}{2.378919in}}%
\pgfusepath{clip}%
\pgfsetbuttcap%
\pgfsetroundjoin%
\pgfsetlinewidth{1.003750pt}%
\definecolor{currentstroke}{rgb}{1.000000,0.000000,0.000000}%
\pgfsetstrokecolor{currentstroke}%
\pgfsetdash{}{0pt}%
\pgfpathmoveto{\pgfqpoint{14.742570in}{9.052657in}}%
\pgfpathcurveto{\pgfqpoint{14.750807in}{9.052657in}}{\pgfqpoint{14.758707in}{9.055930in}}{\pgfqpoint{14.764531in}{9.061754in}}%
\pgfpathcurveto{\pgfqpoint{14.770355in}{9.067578in}}{\pgfqpoint{14.773627in}{9.075478in}}{\pgfqpoint{14.773627in}{9.083714in}}%
\pgfpathcurveto{\pgfqpoint{14.773627in}{9.091950in}}{\pgfqpoint{14.770355in}{9.099850in}}{\pgfqpoint{14.764531in}{9.105674in}}%
\pgfpathcurveto{\pgfqpoint{14.758707in}{9.111498in}}{\pgfqpoint{14.750807in}{9.114770in}}{\pgfqpoint{14.742570in}{9.114770in}}%
\pgfpathcurveto{\pgfqpoint{14.734334in}{9.114770in}}{\pgfqpoint{14.726434in}{9.111498in}}{\pgfqpoint{14.720610in}{9.105674in}}%
\pgfpathcurveto{\pgfqpoint{14.714786in}{9.099850in}}{\pgfqpoint{14.711514in}{9.091950in}}{\pgfqpoint{14.711514in}{9.083714in}}%
\pgfpathcurveto{\pgfqpoint{14.711514in}{9.075478in}}{\pgfqpoint{14.714786in}{9.067578in}}{\pgfqpoint{14.720610in}{9.061754in}}%
\pgfpathcurveto{\pgfqpoint{14.726434in}{9.055930in}}{\pgfqpoint{14.734334in}{9.052657in}}{\pgfqpoint{14.742570in}{9.052657in}}%
\pgfusepath{stroke}%
\end{pgfscope}%
\begin{pgfscope}%
\pgfpathrectangle{\pgfqpoint{4.244096in}{3.200903in}}{\pgfqpoint{3.342550in}{2.378919in}}%
\pgfusepath{clip}%
\pgfsetbuttcap%
\pgfsetroundjoin%
\pgfsetlinewidth{1.003750pt}%
\definecolor{currentstroke}{rgb}{1.000000,0.000000,0.000000}%
\pgfsetstrokecolor{currentstroke}%
\pgfsetdash{}{0pt}%
\pgfpathmoveto{\pgfqpoint{14.738236in}{9.029971in}}%
\pgfpathcurveto{\pgfqpoint{14.746472in}{9.029971in}}{\pgfqpoint{14.754372in}{9.033243in}}{\pgfqpoint{14.760196in}{9.039067in}}%
\pgfpathcurveto{\pgfqpoint{14.766020in}{9.044891in}}{\pgfqpoint{14.769293in}{9.052791in}}{\pgfqpoint{14.769293in}{9.061027in}}%
\pgfpathcurveto{\pgfqpoint{14.769293in}{9.069264in}}{\pgfqpoint{14.766020in}{9.077164in}}{\pgfqpoint{14.760196in}{9.082988in}}%
\pgfpathcurveto{\pgfqpoint{14.754372in}{9.088811in}}{\pgfqpoint{14.746472in}{9.092084in}}{\pgfqpoint{14.738236in}{9.092084in}}%
\pgfpathcurveto{\pgfqpoint{14.730000in}{9.092084in}}{\pgfqpoint{14.722100in}{9.088811in}}{\pgfqpoint{14.716276in}{9.082988in}}%
\pgfpathcurveto{\pgfqpoint{14.710452in}{9.077164in}}{\pgfqpoint{14.707180in}{9.069264in}}{\pgfqpoint{14.707180in}{9.061027in}}%
\pgfpathcurveto{\pgfqpoint{14.707180in}{9.052791in}}{\pgfqpoint{14.710452in}{9.044891in}}{\pgfqpoint{14.716276in}{9.039067in}}%
\pgfpathcurveto{\pgfqpoint{14.722100in}{9.033243in}}{\pgfqpoint{14.730000in}{9.029971in}}{\pgfqpoint{14.738236in}{9.029971in}}%
\pgfusepath{stroke}%
\end{pgfscope}%
\begin{pgfscope}%
\pgfpathrectangle{\pgfqpoint{4.244096in}{3.200903in}}{\pgfqpoint{3.342550in}{2.378919in}}%
\pgfusepath{clip}%
\pgfsetbuttcap%
\pgfsetroundjoin%
\pgfsetlinewidth{1.003750pt}%
\definecolor{currentstroke}{rgb}{1.000000,0.000000,0.000000}%
\pgfsetstrokecolor{currentstroke}%
\pgfsetdash{}{0pt}%
\pgfpathmoveto{\pgfqpoint{14.696009in}{6.898639in}}%
\pgfpathcurveto{\pgfqpoint{14.704245in}{6.898639in}}{\pgfqpoint{14.712145in}{6.901911in}}{\pgfqpoint{14.717969in}{6.907735in}}%
\pgfpathcurveto{\pgfqpoint{14.723793in}{6.913559in}}{\pgfqpoint{14.727065in}{6.921459in}}{\pgfqpoint{14.727065in}{6.929696in}}%
\pgfpathcurveto{\pgfqpoint{14.727065in}{6.937932in}}{\pgfqpoint{14.723793in}{6.945832in}}{\pgfqpoint{14.717969in}{6.951656in}}%
\pgfpathcurveto{\pgfqpoint{14.712145in}{6.957480in}}{\pgfqpoint{14.704245in}{6.960752in}}{\pgfqpoint{14.696009in}{6.960752in}}%
\pgfpathcurveto{\pgfqpoint{14.687773in}{6.960752in}}{\pgfqpoint{14.679872in}{6.957480in}}{\pgfqpoint{14.674049in}{6.951656in}}%
\pgfpathcurveto{\pgfqpoint{14.668225in}{6.945832in}}{\pgfqpoint{14.664952in}{6.937932in}}{\pgfqpoint{14.664952in}{6.929696in}}%
\pgfpathcurveto{\pgfqpoint{14.664952in}{6.921459in}}{\pgfqpoint{14.668225in}{6.913559in}}{\pgfqpoint{14.674049in}{6.907735in}}%
\pgfpathcurveto{\pgfqpoint{14.679872in}{6.901911in}}{\pgfqpoint{14.687773in}{6.898639in}}{\pgfqpoint{14.696009in}{6.898639in}}%
\pgfusepath{stroke}%
\end{pgfscope}%
\begin{pgfscope}%
\pgfpathrectangle{\pgfqpoint{4.244096in}{3.200903in}}{\pgfqpoint{3.342550in}{2.378919in}}%
\pgfusepath{clip}%
\pgfsetbuttcap%
\pgfsetroundjoin%
\pgfsetlinewidth{1.003750pt}%
\definecolor{currentstroke}{rgb}{1.000000,0.000000,0.000000}%
\pgfsetstrokecolor{currentstroke}%
\pgfsetdash{}{0pt}%
\pgfpathmoveto{\pgfqpoint{14.651290in}{6.915351in}}%
\pgfpathcurveto{\pgfqpoint{14.659527in}{6.915351in}}{\pgfqpoint{14.667427in}{6.918624in}}{\pgfqpoint{14.673251in}{6.924448in}}%
\pgfpathcurveto{\pgfqpoint{14.679075in}{6.930272in}}{\pgfqpoint{14.682347in}{6.938172in}}{\pgfqpoint{14.682347in}{6.946408in}}%
\pgfpathcurveto{\pgfqpoint{14.682347in}{6.954644in}}{\pgfqpoint{14.679075in}{6.962544in}}{\pgfqpoint{14.673251in}{6.968368in}}%
\pgfpathcurveto{\pgfqpoint{14.667427in}{6.974192in}}{\pgfqpoint{14.659527in}{6.977464in}}{\pgfqpoint{14.651290in}{6.977464in}}%
\pgfpathcurveto{\pgfqpoint{14.643054in}{6.977464in}}{\pgfqpoint{14.635154in}{6.974192in}}{\pgfqpoint{14.629330in}{6.968368in}}%
\pgfpathcurveto{\pgfqpoint{14.623506in}{6.962544in}}{\pgfqpoint{14.620234in}{6.954644in}}{\pgfqpoint{14.620234in}{6.946408in}}%
\pgfpathcurveto{\pgfqpoint{14.620234in}{6.938172in}}{\pgfqpoint{14.623506in}{6.930272in}}{\pgfqpoint{14.629330in}{6.924448in}}%
\pgfpathcurveto{\pgfqpoint{14.635154in}{6.918624in}}{\pgfqpoint{14.643054in}{6.915351in}}{\pgfqpoint{14.651290in}{6.915351in}}%
\pgfusepath{stroke}%
\end{pgfscope}%
\begin{pgfscope}%
\pgfpathrectangle{\pgfqpoint{4.244096in}{3.200903in}}{\pgfqpoint{3.342550in}{2.378919in}}%
\pgfusepath{clip}%
\pgfsetbuttcap%
\pgfsetroundjoin%
\pgfsetlinewidth{1.003750pt}%
\definecolor{currentstroke}{rgb}{1.000000,0.000000,0.000000}%
\pgfsetstrokecolor{currentstroke}%
\pgfsetdash{}{0pt}%
\pgfpathmoveto{\pgfqpoint{17.503116in}{8.763473in}}%
\pgfpathcurveto{\pgfqpoint{17.511352in}{8.763473in}}{\pgfqpoint{17.519252in}{8.766746in}}{\pgfqpoint{17.525076in}{8.772570in}}%
\pgfpathcurveto{\pgfqpoint{17.530900in}{8.778393in}}{\pgfqpoint{17.534172in}{8.786293in}}{\pgfqpoint{17.534172in}{8.794530in}}%
\pgfpathcurveto{\pgfqpoint{17.534172in}{8.802766in}}{\pgfqpoint{17.530900in}{8.810666in}}{\pgfqpoint{17.525076in}{8.816490in}}%
\pgfpathcurveto{\pgfqpoint{17.519252in}{8.822314in}}{\pgfqpoint{17.511352in}{8.825586in}}{\pgfqpoint{17.503116in}{8.825586in}}%
\pgfpathcurveto{\pgfqpoint{17.494880in}{8.825586in}}{\pgfqpoint{17.486980in}{8.822314in}}{\pgfqpoint{17.481156in}{8.816490in}}%
\pgfpathcurveto{\pgfqpoint{17.475332in}{8.810666in}}{\pgfqpoint{17.472059in}{8.802766in}}{\pgfqpoint{17.472059in}{8.794530in}}%
\pgfpathcurveto{\pgfqpoint{17.472059in}{8.786293in}}{\pgfqpoint{17.475332in}{8.778393in}}{\pgfqpoint{17.481156in}{8.772570in}}%
\pgfpathcurveto{\pgfqpoint{17.486980in}{8.766746in}}{\pgfqpoint{17.494880in}{8.763473in}}{\pgfqpoint{17.503116in}{8.763473in}}%
\pgfusepath{stroke}%
\end{pgfscope}%
\begin{pgfscope}%
\pgfpathrectangle{\pgfqpoint{4.244096in}{3.200903in}}{\pgfqpoint{3.342550in}{2.378919in}}%
\pgfusepath{clip}%
\pgfsetbuttcap%
\pgfsetroundjoin%
\pgfsetlinewidth{1.003750pt}%
\definecolor{currentstroke}{rgb}{1.000000,0.000000,0.000000}%
\pgfsetstrokecolor{currentstroke}%
\pgfsetdash{}{0pt}%
\pgfpathmoveto{\pgfqpoint{14.748100in}{6.801868in}}%
\pgfpathcurveto{\pgfqpoint{14.756336in}{6.801868in}}{\pgfqpoint{14.764236in}{6.805141in}}{\pgfqpoint{14.770060in}{6.810965in}}%
\pgfpathcurveto{\pgfqpoint{14.775884in}{6.816788in}}{\pgfqpoint{14.779156in}{6.824688in}}{\pgfqpoint{14.779156in}{6.832925in}}%
\pgfpathcurveto{\pgfqpoint{14.779156in}{6.841161in}}{\pgfqpoint{14.775884in}{6.849061in}}{\pgfqpoint{14.770060in}{6.854885in}}%
\pgfpathcurveto{\pgfqpoint{14.764236in}{6.860709in}}{\pgfqpoint{14.756336in}{6.863981in}}{\pgfqpoint{14.748100in}{6.863981in}}%
\pgfpathcurveto{\pgfqpoint{14.739863in}{6.863981in}}{\pgfqpoint{14.731963in}{6.860709in}}{\pgfqpoint{14.726139in}{6.854885in}}%
\pgfpathcurveto{\pgfqpoint{14.720316in}{6.849061in}}{\pgfqpoint{14.717043in}{6.841161in}}{\pgfqpoint{14.717043in}{6.832925in}}%
\pgfpathcurveto{\pgfqpoint{14.717043in}{6.824688in}}{\pgfqpoint{14.720316in}{6.816788in}}{\pgfqpoint{14.726139in}{6.810965in}}%
\pgfpathcurveto{\pgfqpoint{14.731963in}{6.805141in}}{\pgfqpoint{14.739863in}{6.801868in}}{\pgfqpoint{14.748100in}{6.801868in}}%
\pgfusepath{stroke}%
\end{pgfscope}%
\begin{pgfscope}%
\pgfpathrectangle{\pgfqpoint{4.244096in}{3.200903in}}{\pgfqpoint{3.342550in}{2.378919in}}%
\pgfusepath{clip}%
\pgfsetbuttcap%
\pgfsetroundjoin%
\pgfsetlinewidth{1.003750pt}%
\definecolor{currentstroke}{rgb}{1.000000,0.000000,0.000000}%
\pgfsetstrokecolor{currentstroke}%
\pgfsetdash{}{0pt}%
\pgfpathmoveto{\pgfqpoint{16.894613in}{7.429349in}}%
\pgfpathcurveto{\pgfqpoint{16.902849in}{7.429349in}}{\pgfqpoint{16.910749in}{7.432621in}}{\pgfqpoint{16.916573in}{7.438445in}}%
\pgfpathcurveto{\pgfqpoint{16.922397in}{7.444269in}}{\pgfqpoint{16.925669in}{7.452169in}}{\pgfqpoint{16.925669in}{7.460405in}}%
\pgfpathcurveto{\pgfqpoint{16.925669in}{7.468641in}}{\pgfqpoint{16.922397in}{7.476541in}}{\pgfqpoint{16.916573in}{7.482365in}}%
\pgfpathcurveto{\pgfqpoint{16.910749in}{7.488189in}}{\pgfqpoint{16.902849in}{7.491462in}}{\pgfqpoint{16.894613in}{7.491462in}}%
\pgfpathcurveto{\pgfqpoint{16.886376in}{7.491462in}}{\pgfqpoint{16.878476in}{7.488189in}}{\pgfqpoint{16.872653in}{7.482365in}}%
\pgfpathcurveto{\pgfqpoint{16.866829in}{7.476541in}}{\pgfqpoint{16.863556in}{7.468641in}}{\pgfqpoint{16.863556in}{7.460405in}}%
\pgfpathcurveto{\pgfqpoint{16.863556in}{7.452169in}}{\pgfqpoint{16.866829in}{7.444269in}}{\pgfqpoint{16.872653in}{7.438445in}}%
\pgfpathcurveto{\pgfqpoint{16.878476in}{7.432621in}}{\pgfqpoint{16.886376in}{7.429349in}}{\pgfqpoint{16.894613in}{7.429349in}}%
\pgfusepath{stroke}%
\end{pgfscope}%
\begin{pgfscope}%
\pgfpathrectangle{\pgfqpoint{4.244096in}{3.200903in}}{\pgfqpoint{3.342550in}{2.378919in}}%
\pgfusepath{clip}%
\pgfsetbuttcap%
\pgfsetroundjoin%
\pgfsetlinewidth{1.003750pt}%
\definecolor{currentstroke}{rgb}{1.000000,0.000000,0.000000}%
\pgfsetstrokecolor{currentstroke}%
\pgfsetdash{}{0pt}%
\pgfpathmoveto{\pgfqpoint{17.551769in}{7.073694in}}%
\pgfpathcurveto{\pgfqpoint{17.560005in}{7.073694in}}{\pgfqpoint{17.567905in}{7.076967in}}{\pgfqpoint{17.573729in}{7.082791in}}%
\pgfpathcurveto{\pgfqpoint{17.579553in}{7.088615in}}{\pgfqpoint{17.582825in}{7.096515in}}{\pgfqpoint{17.582825in}{7.104751in}}%
\pgfpathcurveto{\pgfqpoint{17.582825in}{7.112987in}}{\pgfqpoint{17.579553in}{7.120887in}}{\pgfqpoint{17.573729in}{7.126711in}}%
\pgfpathcurveto{\pgfqpoint{17.567905in}{7.132535in}}{\pgfqpoint{17.560005in}{7.135807in}}{\pgfqpoint{17.551769in}{7.135807in}}%
\pgfpathcurveto{\pgfqpoint{17.543532in}{7.135807in}}{\pgfqpoint{17.535632in}{7.132535in}}{\pgfqpoint{17.529808in}{7.126711in}}%
\pgfpathcurveto{\pgfqpoint{17.523984in}{7.120887in}}{\pgfqpoint{17.520712in}{7.112987in}}{\pgfqpoint{17.520712in}{7.104751in}}%
\pgfpathcurveto{\pgfqpoint{17.520712in}{7.096515in}}{\pgfqpoint{17.523984in}{7.088615in}}{\pgfqpoint{17.529808in}{7.082791in}}%
\pgfpathcurveto{\pgfqpoint{17.535632in}{7.076967in}}{\pgfqpoint{17.543532in}{7.073694in}}{\pgfqpoint{17.551769in}{7.073694in}}%
\pgfusepath{stroke}%
\end{pgfscope}%
\begin{pgfscope}%
\pgfpathrectangle{\pgfqpoint{4.244096in}{3.200903in}}{\pgfqpoint{3.342550in}{2.378919in}}%
\pgfusepath{clip}%
\pgfsetbuttcap%
\pgfsetroundjoin%
\pgfsetlinewidth{1.003750pt}%
\definecolor{currentstroke}{rgb}{1.000000,0.000000,0.000000}%
\pgfsetstrokecolor{currentstroke}%
\pgfsetdash{}{0pt}%
\pgfpathmoveto{\pgfqpoint{14.475131in}{7.712408in}}%
\pgfpathcurveto{\pgfqpoint{14.483368in}{7.712408in}}{\pgfqpoint{14.491268in}{7.715680in}}{\pgfqpoint{14.497092in}{7.721504in}}%
\pgfpathcurveto{\pgfqpoint{14.502915in}{7.727328in}}{\pgfqpoint{14.506188in}{7.735228in}}{\pgfqpoint{14.506188in}{7.743464in}}%
\pgfpathcurveto{\pgfqpoint{14.506188in}{7.751701in}}{\pgfqpoint{14.502915in}{7.759601in}}{\pgfqpoint{14.497092in}{7.765425in}}%
\pgfpathcurveto{\pgfqpoint{14.491268in}{7.771248in}}{\pgfqpoint{14.483368in}{7.774521in}}{\pgfqpoint{14.475131in}{7.774521in}}%
\pgfpathcurveto{\pgfqpoint{14.466895in}{7.774521in}}{\pgfqpoint{14.458995in}{7.771248in}}{\pgfqpoint{14.453171in}{7.765425in}}%
\pgfpathcurveto{\pgfqpoint{14.447347in}{7.759601in}}{\pgfqpoint{14.444075in}{7.751701in}}{\pgfqpoint{14.444075in}{7.743464in}}%
\pgfpathcurveto{\pgfqpoint{14.444075in}{7.735228in}}{\pgfqpoint{14.447347in}{7.727328in}}{\pgfqpoint{14.453171in}{7.721504in}}%
\pgfpathcurveto{\pgfqpoint{14.458995in}{7.715680in}}{\pgfqpoint{14.466895in}{7.712408in}}{\pgfqpoint{14.475131in}{7.712408in}}%
\pgfusepath{stroke}%
\end{pgfscope}%
\begin{pgfscope}%
\pgfpathrectangle{\pgfqpoint{4.244096in}{3.200903in}}{\pgfqpoint{3.342550in}{2.378919in}}%
\pgfusepath{clip}%
\pgfsetbuttcap%
\pgfsetroundjoin%
\pgfsetlinewidth{1.003750pt}%
\definecolor{currentstroke}{rgb}{1.000000,0.000000,0.000000}%
\pgfsetstrokecolor{currentstroke}%
\pgfsetdash{}{0pt}%
\pgfpathmoveto{\pgfqpoint{16.374031in}{6.502004in}}%
\pgfpathcurveto{\pgfqpoint{16.382267in}{6.502004in}}{\pgfqpoint{16.390167in}{6.505276in}}{\pgfqpoint{16.395991in}{6.511100in}}%
\pgfpathcurveto{\pgfqpoint{16.401815in}{6.516924in}}{\pgfqpoint{16.405087in}{6.524824in}}{\pgfqpoint{16.405087in}{6.533060in}}%
\pgfpathcurveto{\pgfqpoint{16.405087in}{6.541297in}}{\pgfqpoint{16.401815in}{6.549197in}}{\pgfqpoint{16.395991in}{6.555021in}}%
\pgfpathcurveto{\pgfqpoint{16.390167in}{6.560845in}}{\pgfqpoint{16.382267in}{6.564117in}}{\pgfqpoint{16.374031in}{6.564117in}}%
\pgfpathcurveto{\pgfqpoint{16.365795in}{6.564117in}}{\pgfqpoint{16.357895in}{6.560845in}}{\pgfqpoint{16.352071in}{6.555021in}}%
\pgfpathcurveto{\pgfqpoint{16.346247in}{6.549197in}}{\pgfqpoint{16.342974in}{6.541297in}}{\pgfqpoint{16.342974in}{6.533060in}}%
\pgfpathcurveto{\pgfqpoint{16.342974in}{6.524824in}}{\pgfqpoint{16.346247in}{6.516924in}}{\pgfqpoint{16.352071in}{6.511100in}}%
\pgfpathcurveto{\pgfqpoint{16.357895in}{6.505276in}}{\pgfqpoint{16.365795in}{6.502004in}}{\pgfqpoint{16.374031in}{6.502004in}}%
\pgfusepath{stroke}%
\end{pgfscope}%
\begin{pgfscope}%
\pgfpathrectangle{\pgfqpoint{4.244096in}{3.200903in}}{\pgfqpoint{3.342550in}{2.378919in}}%
\pgfusepath{clip}%
\pgfsetbuttcap%
\pgfsetroundjoin%
\pgfsetlinewidth{1.003750pt}%
\definecolor{currentstroke}{rgb}{1.000000,0.000000,0.000000}%
\pgfsetstrokecolor{currentstroke}%
\pgfsetdash{}{0pt}%
\pgfpathmoveto{\pgfqpoint{17.087643in}{6.009794in}}%
\pgfpathcurveto{\pgfqpoint{17.095879in}{6.009794in}}{\pgfqpoint{17.103779in}{6.013067in}}{\pgfqpoint{17.109603in}{6.018891in}}%
\pgfpathcurveto{\pgfqpoint{17.115427in}{6.024715in}}{\pgfqpoint{17.118699in}{6.032615in}}{\pgfqpoint{17.118699in}{6.040851in}}%
\pgfpathcurveto{\pgfqpoint{17.118699in}{6.049087in}}{\pgfqpoint{17.115427in}{6.056987in}}{\pgfqpoint{17.109603in}{6.062811in}}%
\pgfpathcurveto{\pgfqpoint{17.103779in}{6.068635in}}{\pgfqpoint{17.095879in}{6.071907in}}{\pgfqpoint{17.087643in}{6.071907in}}%
\pgfpathcurveto{\pgfqpoint{17.079407in}{6.071907in}}{\pgfqpoint{17.071507in}{6.068635in}}{\pgfqpoint{17.065683in}{6.062811in}}%
\pgfpathcurveto{\pgfqpoint{17.059859in}{6.056987in}}{\pgfqpoint{17.056586in}{6.049087in}}{\pgfqpoint{17.056586in}{6.040851in}}%
\pgfpathcurveto{\pgfqpoint{17.056586in}{6.032615in}}{\pgfqpoint{17.059859in}{6.024715in}}{\pgfqpoint{17.065683in}{6.018891in}}%
\pgfpathcurveto{\pgfqpoint{17.071507in}{6.013067in}}{\pgfqpoint{17.079407in}{6.009794in}}{\pgfqpoint{17.087643in}{6.009794in}}%
\pgfusepath{stroke}%
\end{pgfscope}%
\begin{pgfscope}%
\pgfpathrectangle{\pgfqpoint{4.244096in}{3.200903in}}{\pgfqpoint{3.342550in}{2.378919in}}%
\pgfusepath{clip}%
\pgfsetbuttcap%
\pgfsetroundjoin%
\pgfsetlinewidth{1.003750pt}%
\definecolor{currentstroke}{rgb}{1.000000,0.000000,0.000000}%
\pgfsetstrokecolor{currentstroke}%
\pgfsetdash{}{0pt}%
\pgfpathmoveto{\pgfqpoint{17.068290in}{6.220276in}}%
\pgfpathcurveto{\pgfqpoint{17.076527in}{6.220276in}}{\pgfqpoint{17.084427in}{6.223548in}}{\pgfqpoint{17.090251in}{6.229372in}}%
\pgfpathcurveto{\pgfqpoint{17.096075in}{6.235196in}}{\pgfqpoint{17.099347in}{6.243096in}}{\pgfqpoint{17.099347in}{6.251332in}}%
\pgfpathcurveto{\pgfqpoint{17.099347in}{6.259568in}}{\pgfqpoint{17.096075in}{6.267468in}}{\pgfqpoint{17.090251in}{6.273292in}}%
\pgfpathcurveto{\pgfqpoint{17.084427in}{6.279116in}}{\pgfqpoint{17.076527in}{6.282389in}}{\pgfqpoint{17.068290in}{6.282389in}}%
\pgfpathcurveto{\pgfqpoint{17.060054in}{6.282389in}}{\pgfqpoint{17.052154in}{6.279116in}}{\pgfqpoint{17.046330in}{6.273292in}}%
\pgfpathcurveto{\pgfqpoint{17.040506in}{6.267468in}}{\pgfqpoint{17.037234in}{6.259568in}}{\pgfqpoint{17.037234in}{6.251332in}}%
\pgfpathcurveto{\pgfqpoint{17.037234in}{6.243096in}}{\pgfqpoint{17.040506in}{6.235196in}}{\pgfqpoint{17.046330in}{6.229372in}}%
\pgfpathcurveto{\pgfqpoint{17.052154in}{6.223548in}}{\pgfqpoint{17.060054in}{6.220276in}}{\pgfqpoint{17.068290in}{6.220276in}}%
\pgfusepath{stroke}%
\end{pgfscope}%
\begin{pgfscope}%
\pgfpathrectangle{\pgfqpoint{4.244096in}{3.200903in}}{\pgfqpoint{3.342550in}{2.378919in}}%
\pgfusepath{clip}%
\pgfsetbuttcap%
\pgfsetroundjoin%
\pgfsetlinewidth{1.003750pt}%
\definecolor{currentstroke}{rgb}{1.000000,0.000000,0.000000}%
\pgfsetstrokecolor{currentstroke}%
\pgfsetdash{}{0pt}%
\pgfpathmoveto{\pgfqpoint{14.482491in}{7.636590in}}%
\pgfpathcurveto{\pgfqpoint{14.490728in}{7.636590in}}{\pgfqpoint{14.498628in}{7.639862in}}{\pgfqpoint{14.504451in}{7.645686in}}%
\pgfpathcurveto{\pgfqpoint{14.510275in}{7.651510in}}{\pgfqpoint{14.513548in}{7.659410in}}{\pgfqpoint{14.513548in}{7.667646in}}%
\pgfpathcurveto{\pgfqpoint{14.513548in}{7.675882in}}{\pgfqpoint{14.510275in}{7.683782in}}{\pgfqpoint{14.504451in}{7.689606in}}%
\pgfpathcurveto{\pgfqpoint{14.498628in}{7.695430in}}{\pgfqpoint{14.490728in}{7.698703in}}{\pgfqpoint{14.482491in}{7.698703in}}%
\pgfpathcurveto{\pgfqpoint{14.474255in}{7.698703in}}{\pgfqpoint{14.466355in}{7.695430in}}{\pgfqpoint{14.460531in}{7.689606in}}%
\pgfpathcurveto{\pgfqpoint{14.454707in}{7.683782in}}{\pgfqpoint{14.451435in}{7.675882in}}{\pgfqpoint{14.451435in}{7.667646in}}%
\pgfpathcurveto{\pgfqpoint{14.451435in}{7.659410in}}{\pgfqpoint{14.454707in}{7.651510in}}{\pgfqpoint{14.460531in}{7.645686in}}%
\pgfpathcurveto{\pgfqpoint{14.466355in}{7.639862in}}{\pgfqpoint{14.474255in}{7.636590in}}{\pgfqpoint{14.482491in}{7.636590in}}%
\pgfusepath{stroke}%
\end{pgfscope}%
\begin{pgfscope}%
\pgfpathrectangle{\pgfqpoint{4.244096in}{3.200903in}}{\pgfqpoint{3.342550in}{2.378919in}}%
\pgfusepath{clip}%
\pgfsetbuttcap%
\pgfsetroundjoin%
\pgfsetlinewidth{1.003750pt}%
\definecolor{currentstroke}{rgb}{1.000000,0.000000,0.000000}%
\pgfsetstrokecolor{currentstroke}%
\pgfsetdash{}{0pt}%
\pgfpathmoveto{\pgfqpoint{15.451609in}{8.762695in}}%
\pgfpathcurveto{\pgfqpoint{15.459845in}{8.762695in}}{\pgfqpoint{15.467745in}{8.765967in}}{\pgfqpoint{15.473569in}{8.771791in}}%
\pgfpathcurveto{\pgfqpoint{15.479393in}{8.777615in}}{\pgfqpoint{15.482665in}{8.785515in}}{\pgfqpoint{15.482665in}{8.793752in}}%
\pgfpathcurveto{\pgfqpoint{15.482665in}{8.801988in}}{\pgfqpoint{15.479393in}{8.809888in}}{\pgfqpoint{15.473569in}{8.815712in}}%
\pgfpathcurveto{\pgfqpoint{15.467745in}{8.821536in}}{\pgfqpoint{15.459845in}{8.824808in}}{\pgfqpoint{15.451609in}{8.824808in}}%
\pgfpathcurveto{\pgfqpoint{15.443373in}{8.824808in}}{\pgfqpoint{15.435473in}{8.821536in}}{\pgfqpoint{15.429649in}{8.815712in}}%
\pgfpathcurveto{\pgfqpoint{15.423825in}{8.809888in}}{\pgfqpoint{15.420552in}{8.801988in}}{\pgfqpoint{15.420552in}{8.793752in}}%
\pgfpathcurveto{\pgfqpoint{15.420552in}{8.785515in}}{\pgfqpoint{15.423825in}{8.777615in}}{\pgfqpoint{15.429649in}{8.771791in}}%
\pgfpathcurveto{\pgfqpoint{15.435473in}{8.765967in}}{\pgfqpoint{15.443373in}{8.762695in}}{\pgfqpoint{15.451609in}{8.762695in}}%
\pgfusepath{stroke}%
\end{pgfscope}%
\begin{pgfscope}%
\pgfpathrectangle{\pgfqpoint{4.244096in}{3.200903in}}{\pgfqpoint{3.342550in}{2.378919in}}%
\pgfusepath{clip}%
\pgfsetbuttcap%
\pgfsetroundjoin%
\pgfsetlinewidth{1.003750pt}%
\definecolor{currentstroke}{rgb}{1.000000,0.000000,0.000000}%
\pgfsetstrokecolor{currentstroke}%
\pgfsetdash{}{0pt}%
\pgfpathmoveto{\pgfqpoint{14.834805in}{8.834137in}}%
\pgfpathcurveto{\pgfqpoint{14.843042in}{8.834137in}}{\pgfqpoint{14.850942in}{8.837409in}}{\pgfqpoint{14.856766in}{8.843233in}}%
\pgfpathcurveto{\pgfqpoint{14.862589in}{8.849057in}}{\pgfqpoint{14.865862in}{8.856957in}}{\pgfqpoint{14.865862in}{8.865193in}}%
\pgfpathcurveto{\pgfqpoint{14.865862in}{8.873430in}}{\pgfqpoint{14.862589in}{8.881330in}}{\pgfqpoint{14.856766in}{8.887154in}}%
\pgfpathcurveto{\pgfqpoint{14.850942in}{8.892978in}}{\pgfqpoint{14.843042in}{8.896250in}}{\pgfqpoint{14.834805in}{8.896250in}}%
\pgfpathcurveto{\pgfqpoint{14.826569in}{8.896250in}}{\pgfqpoint{14.818669in}{8.892978in}}{\pgfqpoint{14.812845in}{8.887154in}}%
\pgfpathcurveto{\pgfqpoint{14.807021in}{8.881330in}}{\pgfqpoint{14.803749in}{8.873430in}}{\pgfqpoint{14.803749in}{8.865193in}}%
\pgfpathcurveto{\pgfqpoint{14.803749in}{8.856957in}}{\pgfqpoint{14.807021in}{8.849057in}}{\pgfqpoint{14.812845in}{8.843233in}}%
\pgfpathcurveto{\pgfqpoint{14.818669in}{8.837409in}}{\pgfqpoint{14.826569in}{8.834137in}}{\pgfqpoint{14.834805in}{8.834137in}}%
\pgfusepath{stroke}%
\end{pgfscope}%
\begin{pgfscope}%
\pgfpathrectangle{\pgfqpoint{4.244096in}{3.200903in}}{\pgfqpoint{3.342550in}{2.378919in}}%
\pgfusepath{clip}%
\pgfsetbuttcap%
\pgfsetroundjoin%
\pgfsetlinewidth{1.003750pt}%
\definecolor{currentstroke}{rgb}{1.000000,0.000000,0.000000}%
\pgfsetstrokecolor{currentstroke}%
\pgfsetdash{}{0pt}%
\pgfpathmoveto{\pgfqpoint{14.408118in}{7.694337in}}%
\pgfpathcurveto{\pgfqpoint{14.416354in}{7.694337in}}{\pgfqpoint{14.424254in}{7.697610in}}{\pgfqpoint{14.430078in}{7.703434in}}%
\pgfpathcurveto{\pgfqpoint{14.435902in}{7.709258in}}{\pgfqpoint{14.439175in}{7.717158in}}{\pgfqpoint{14.439175in}{7.725394in}}%
\pgfpathcurveto{\pgfqpoint{14.439175in}{7.733630in}}{\pgfqpoint{14.435902in}{7.741530in}}{\pgfqpoint{14.430078in}{7.747354in}}%
\pgfpathcurveto{\pgfqpoint{14.424254in}{7.753178in}}{\pgfqpoint{14.416354in}{7.756450in}}{\pgfqpoint{14.408118in}{7.756450in}}%
\pgfpathcurveto{\pgfqpoint{14.399882in}{7.756450in}}{\pgfqpoint{14.391982in}{7.753178in}}{\pgfqpoint{14.386158in}{7.747354in}}%
\pgfpathcurveto{\pgfqpoint{14.380334in}{7.741530in}}{\pgfqpoint{14.377062in}{7.733630in}}{\pgfqpoint{14.377062in}{7.725394in}}%
\pgfpathcurveto{\pgfqpoint{14.377062in}{7.717158in}}{\pgfqpoint{14.380334in}{7.709258in}}{\pgfqpoint{14.386158in}{7.703434in}}%
\pgfpathcurveto{\pgfqpoint{14.391982in}{7.697610in}}{\pgfqpoint{14.399882in}{7.694337in}}{\pgfqpoint{14.408118in}{7.694337in}}%
\pgfusepath{stroke}%
\end{pgfscope}%
\begin{pgfscope}%
\pgfpathrectangle{\pgfqpoint{4.244096in}{3.200903in}}{\pgfqpoint{3.342550in}{2.378919in}}%
\pgfusepath{clip}%
\pgfsetbuttcap%
\pgfsetroundjoin%
\pgfsetlinewidth{1.003750pt}%
\definecolor{currentstroke}{rgb}{1.000000,0.000000,0.000000}%
\pgfsetstrokecolor{currentstroke}%
\pgfsetdash{}{0pt}%
\pgfpathmoveto{\pgfqpoint{5.095246in}{4.382614in}}%
\pgfpathcurveto{\pgfqpoint{5.103482in}{4.382614in}}{\pgfqpoint{5.111382in}{4.385886in}}{\pgfqpoint{5.117206in}{4.391710in}}%
\pgfpathcurveto{\pgfqpoint{5.123030in}{4.397534in}}{\pgfqpoint{5.126302in}{4.405434in}}{\pgfqpoint{5.126302in}{4.413670in}}%
\pgfpathcurveto{\pgfqpoint{5.126302in}{4.421907in}}{\pgfqpoint{5.123030in}{4.429807in}}{\pgfqpoint{5.117206in}{4.435631in}}%
\pgfpathcurveto{\pgfqpoint{5.111382in}{4.441455in}}{\pgfqpoint{5.103482in}{4.444727in}}{\pgfqpoint{5.095246in}{4.444727in}}%
\pgfpathcurveto{\pgfqpoint{5.087010in}{4.444727in}}{\pgfqpoint{5.079110in}{4.441455in}}{\pgfqpoint{5.073286in}{4.435631in}}%
\pgfpathcurveto{\pgfqpoint{5.067462in}{4.429807in}}{\pgfqpoint{5.064189in}{4.421907in}}{\pgfqpoint{5.064189in}{4.413670in}}%
\pgfpathcurveto{\pgfqpoint{5.064189in}{4.405434in}}{\pgfqpoint{5.067462in}{4.397534in}}{\pgfqpoint{5.073286in}{4.391710in}}%
\pgfpathcurveto{\pgfqpoint{5.079110in}{4.385886in}}{\pgfqpoint{5.087010in}{4.382614in}}{\pgfqpoint{5.095246in}{4.382614in}}%
\pgfpathlineto{\pgfqpoint{5.095246in}{4.382614in}}%
\pgfpathclose%
\pgfusepath{stroke}%
\end{pgfscope}%
\begin{pgfscope}%
\pgfpathrectangle{\pgfqpoint{4.244096in}{3.200903in}}{\pgfqpoint{3.342550in}{2.378919in}}%
\pgfusepath{clip}%
\pgfsetbuttcap%
\pgfsetmiterjoin%
\definecolor{currentfill}{rgb}{0.839216,0.152941,0.156863}%
\pgfsetfillcolor{currentfill}%
\pgfsetfillopacity{0.200000}%
\pgfsetlinewidth{1.003750pt}%
\definecolor{currentstroke}{rgb}{0.839216,0.152941,0.156863}%
\pgfsetstrokecolor{currentstroke}%
\pgfsetstrokeopacity{0.200000}%
\pgfsetdash{}{0pt}%
\pgfpathmoveto{\pgfqpoint{5.095246in}{3.200903in}}%
\pgfpathlineto{\pgfqpoint{27.614371in}{3.200903in}}%
\pgfpathlineto{\pgfqpoint{27.614371in}{5.579823in}}%
\pgfpathlineto{\pgfqpoint{5.095246in}{5.579823in}}%
\pgfpathlineto{\pgfqpoint{5.095246in}{3.200903in}}%
\pgfpathclose%
\pgfusepath{stroke,fill}%
\end{pgfscope}%
\begin{pgfscope}%
\pgfsetbuttcap%
\pgfsetmiterjoin%
\definecolor{currentfill}{rgb}{0.839216,0.152941,0.156863}%
\pgfsetfillcolor{currentfill}%
\pgfsetfillopacity{0.200000}%
\pgfsetlinewidth{1.003750pt}%
\definecolor{currentstroke}{rgb}{0.839216,0.152941,0.156863}%
\pgfsetstrokecolor{currentstroke}%
\pgfsetstrokeopacity{0.200000}%
\pgfsetdash{}{0pt}%
\pgfpathrectangle{\pgfqpoint{4.244096in}{3.200903in}}{\pgfqpoint{3.342550in}{2.378919in}}%
\pgfusepath{clip}%
\pgfpathmoveto{\pgfqpoint{5.095246in}{3.200903in}}%
\pgfpathlineto{\pgfqpoint{27.614371in}{3.200903in}}%
\pgfpathlineto{\pgfqpoint{27.614371in}{5.579823in}}%
\pgfpathlineto{\pgfqpoint{5.095246in}{5.579823in}}%
\pgfpathlineto{\pgfqpoint{5.095246in}{3.200903in}}%
\pgfpathclose%
\pgfusepath{clip}%
\pgfsys@defobject{currentpattern}{\pgfqpoint{0in}{0in}}{\pgfqpoint{1in}{1in}}{%
\begin{pgfscope}%
\pgfpathrectangle{\pgfqpoint{0in}{0in}}{\pgfqpoint{1in}{1in}}%
\pgfusepath{clip}%
\pgfpathmoveto{\pgfqpoint{-0.500000in}{0.500000in}}%
\pgfpathlineto{\pgfqpoint{0.500000in}{1.500000in}}%
\pgfpathmoveto{\pgfqpoint{-0.333333in}{0.333333in}}%
\pgfpathlineto{\pgfqpoint{0.666667in}{1.333333in}}%
\pgfpathmoveto{\pgfqpoint{-0.166667in}{0.166667in}}%
\pgfpathlineto{\pgfqpoint{0.833333in}{1.166667in}}%
\pgfpathmoveto{\pgfqpoint{0.000000in}{0.000000in}}%
\pgfpathlineto{\pgfqpoint{1.000000in}{1.000000in}}%
\pgfpathmoveto{\pgfqpoint{0.166667in}{-0.166667in}}%
\pgfpathlineto{\pgfqpoint{1.166667in}{0.833333in}}%
\pgfpathmoveto{\pgfqpoint{0.333333in}{-0.333333in}}%
\pgfpathlineto{\pgfqpoint{1.333333in}{0.666667in}}%
\pgfpathmoveto{\pgfqpoint{0.500000in}{-0.500000in}}%
\pgfpathlineto{\pgfqpoint{1.500000in}{0.500000in}}%
\pgfusepath{stroke}%
\end{pgfscope}%
}%
\pgfsys@transformshift{5.095246in}{3.200903in}%
\pgfsys@useobject{currentpattern}{}%
\pgfsys@transformshift{1in}{0in}%
\pgfsys@useobject{currentpattern}{}%
\pgfsys@transformshift{1in}{0in}%
\pgfsys@useobject{currentpattern}{}%
\pgfsys@transformshift{1in}{0in}%
\pgfsys@useobject{currentpattern}{}%
\pgfsys@transformshift{1in}{0in}%
\pgfsys@useobject{currentpattern}{}%
\pgfsys@transformshift{1in}{0in}%
\pgfsys@useobject{currentpattern}{}%
\pgfsys@transformshift{1in}{0in}%
\pgfsys@useobject{currentpattern}{}%
\pgfsys@transformshift{1in}{0in}%
\pgfsys@useobject{currentpattern}{}%
\pgfsys@transformshift{1in}{0in}%
\pgfsys@useobject{currentpattern}{}%
\pgfsys@transformshift{1in}{0in}%
\pgfsys@useobject{currentpattern}{}%
\pgfsys@transformshift{1in}{0in}%
\pgfsys@useobject{currentpattern}{}%
\pgfsys@transformshift{1in}{0in}%
\pgfsys@useobject{currentpattern}{}%
\pgfsys@transformshift{1in}{0in}%
\pgfsys@useobject{currentpattern}{}%
\pgfsys@transformshift{1in}{0in}%
\pgfsys@useobject{currentpattern}{}%
\pgfsys@transformshift{1in}{0in}%
\pgfsys@useobject{currentpattern}{}%
\pgfsys@transformshift{1in}{0in}%
\pgfsys@useobject{currentpattern}{}%
\pgfsys@transformshift{1in}{0in}%
\pgfsys@useobject{currentpattern}{}%
\pgfsys@transformshift{1in}{0in}%
\pgfsys@useobject{currentpattern}{}%
\pgfsys@transformshift{1in}{0in}%
\pgfsys@useobject{currentpattern}{}%
\pgfsys@transformshift{1in}{0in}%
\pgfsys@useobject{currentpattern}{}%
\pgfsys@transformshift{1in}{0in}%
\pgfsys@useobject{currentpattern}{}%
\pgfsys@transformshift{1in}{0in}%
\pgfsys@useobject{currentpattern}{}%
\pgfsys@transformshift{1in}{0in}%
\pgfsys@useobject{currentpattern}{}%
\pgfsys@transformshift{1in}{0in}%
\pgfsys@transformshift{-23in}{0in}%
\pgfsys@transformshift{0in}{1in}%
\pgfsys@useobject{currentpattern}{}%
\pgfsys@transformshift{1in}{0in}%
\pgfsys@useobject{currentpattern}{}%
\pgfsys@transformshift{1in}{0in}%
\pgfsys@useobject{currentpattern}{}%
\pgfsys@transformshift{1in}{0in}%
\pgfsys@useobject{currentpattern}{}%
\pgfsys@transformshift{1in}{0in}%
\pgfsys@useobject{currentpattern}{}%
\pgfsys@transformshift{1in}{0in}%
\pgfsys@useobject{currentpattern}{}%
\pgfsys@transformshift{1in}{0in}%
\pgfsys@useobject{currentpattern}{}%
\pgfsys@transformshift{1in}{0in}%
\pgfsys@useobject{currentpattern}{}%
\pgfsys@transformshift{1in}{0in}%
\pgfsys@useobject{currentpattern}{}%
\pgfsys@transformshift{1in}{0in}%
\pgfsys@useobject{currentpattern}{}%
\pgfsys@transformshift{1in}{0in}%
\pgfsys@useobject{currentpattern}{}%
\pgfsys@transformshift{1in}{0in}%
\pgfsys@useobject{currentpattern}{}%
\pgfsys@transformshift{1in}{0in}%
\pgfsys@useobject{currentpattern}{}%
\pgfsys@transformshift{1in}{0in}%
\pgfsys@useobject{currentpattern}{}%
\pgfsys@transformshift{1in}{0in}%
\pgfsys@useobject{currentpattern}{}%
\pgfsys@transformshift{1in}{0in}%
\pgfsys@useobject{currentpattern}{}%
\pgfsys@transformshift{1in}{0in}%
\pgfsys@useobject{currentpattern}{}%
\pgfsys@transformshift{1in}{0in}%
\pgfsys@useobject{currentpattern}{}%
\pgfsys@transformshift{1in}{0in}%
\pgfsys@useobject{currentpattern}{}%
\pgfsys@transformshift{1in}{0in}%
\pgfsys@useobject{currentpattern}{}%
\pgfsys@transformshift{1in}{0in}%
\pgfsys@useobject{currentpattern}{}%
\pgfsys@transformshift{1in}{0in}%
\pgfsys@useobject{currentpattern}{}%
\pgfsys@transformshift{1in}{0in}%
\pgfsys@useobject{currentpattern}{}%
\pgfsys@transformshift{1in}{0in}%
\pgfsys@transformshift{-23in}{0in}%
\pgfsys@transformshift{0in}{1in}%
\pgfsys@useobject{currentpattern}{}%
\pgfsys@transformshift{1in}{0in}%
\pgfsys@useobject{currentpattern}{}%
\pgfsys@transformshift{1in}{0in}%
\pgfsys@useobject{currentpattern}{}%
\pgfsys@transformshift{1in}{0in}%
\pgfsys@useobject{currentpattern}{}%
\pgfsys@transformshift{1in}{0in}%
\pgfsys@useobject{currentpattern}{}%
\pgfsys@transformshift{1in}{0in}%
\pgfsys@useobject{currentpattern}{}%
\pgfsys@transformshift{1in}{0in}%
\pgfsys@useobject{currentpattern}{}%
\pgfsys@transformshift{1in}{0in}%
\pgfsys@useobject{currentpattern}{}%
\pgfsys@transformshift{1in}{0in}%
\pgfsys@useobject{currentpattern}{}%
\pgfsys@transformshift{1in}{0in}%
\pgfsys@useobject{currentpattern}{}%
\pgfsys@transformshift{1in}{0in}%
\pgfsys@useobject{currentpattern}{}%
\pgfsys@transformshift{1in}{0in}%
\pgfsys@useobject{currentpattern}{}%
\pgfsys@transformshift{1in}{0in}%
\pgfsys@useobject{currentpattern}{}%
\pgfsys@transformshift{1in}{0in}%
\pgfsys@useobject{currentpattern}{}%
\pgfsys@transformshift{1in}{0in}%
\pgfsys@useobject{currentpattern}{}%
\pgfsys@transformshift{1in}{0in}%
\pgfsys@useobject{currentpattern}{}%
\pgfsys@transformshift{1in}{0in}%
\pgfsys@useobject{currentpattern}{}%
\pgfsys@transformshift{1in}{0in}%
\pgfsys@useobject{currentpattern}{}%
\pgfsys@transformshift{1in}{0in}%
\pgfsys@useobject{currentpattern}{}%
\pgfsys@transformshift{1in}{0in}%
\pgfsys@useobject{currentpattern}{}%
\pgfsys@transformshift{1in}{0in}%
\pgfsys@useobject{currentpattern}{}%
\pgfsys@transformshift{1in}{0in}%
\pgfsys@useobject{currentpattern}{}%
\pgfsys@transformshift{1in}{0in}%
\pgfsys@useobject{currentpattern}{}%
\pgfsys@transformshift{1in}{0in}%
\pgfsys@transformshift{-23in}{0in}%
\pgfsys@transformshift{0in}{1in}%
\end{pgfscope}%
\begin{pgfscope}%
\pgfpathrectangle{\pgfqpoint{4.244096in}{3.200903in}}{\pgfqpoint{3.342550in}{2.378919in}}%
\pgfusepath{clip}%
\pgfsetrectcap%
\pgfsetroundjoin%
\pgfsetlinewidth{0.803000pt}%
\definecolor{currentstroke}{rgb}{0.690196,0.690196,0.690196}%
\pgfsetstrokecolor{currentstroke}%
\pgfsetdash{}{0pt}%
\pgfpathmoveto{\pgfqpoint{4.607620in}{3.200903in}}%
\pgfpathlineto{\pgfqpoint{4.607620in}{5.579823in}}%
\pgfusepath{stroke}%
\end{pgfscope}%
\begin{pgfscope}%
\pgfsetbuttcap%
\pgfsetroundjoin%
\definecolor{currentfill}{rgb}{0.000000,0.000000,0.000000}%
\pgfsetfillcolor{currentfill}%
\pgfsetlinewidth{0.803000pt}%
\definecolor{currentstroke}{rgb}{0.000000,0.000000,0.000000}%
\pgfsetstrokecolor{currentstroke}%
\pgfsetdash{}{0pt}%
\pgfsys@defobject{currentmarker}{\pgfqpoint{0.000000in}{-0.048611in}}{\pgfqpoint{0.000000in}{0.000000in}}{%
\pgfpathmoveto{\pgfqpoint{0.000000in}{0.000000in}}%
\pgfpathlineto{\pgfqpoint{0.000000in}{-0.048611in}}%
\pgfusepath{stroke,fill}%
}%
\begin{pgfscope}%
\pgfsys@transformshift{4.607620in}{3.200903in}%
\pgfsys@useobject{currentmarker}{}%
\end{pgfscope}%
\end{pgfscope}%
\begin{pgfscope}%
\definecolor{textcolor}{rgb}{0.000000,0.000000,0.000000}%
\pgfsetstrokecolor{textcolor}%
\pgfsetfillcolor{textcolor}%
\pgftext[x=4.607620in,y=3.103681in,,top]{\color{textcolor}{\rmfamily\fontsize{14.000000}{16.800000}\selectfont\catcode`\^=\active\def^{\ifmmode\sp\else\^{}\fi}\catcode`\%=\active\def%{\%}$\mathdefault{5280}$}}%
\end{pgfscope}%
\begin{pgfscope}%
\pgfpathrectangle{\pgfqpoint{4.244096in}{3.200903in}}{\pgfqpoint{3.342550in}{2.378919in}}%
\pgfusepath{clip}%
\pgfsetrectcap%
\pgfsetroundjoin%
\pgfsetlinewidth{0.803000pt}%
\definecolor{currentstroke}{rgb}{0.690196,0.690196,0.690196}%
\pgfsetstrokecolor{currentstroke}%
\pgfsetdash{}{0pt}%
\pgfpathmoveto{\pgfqpoint{5.458771in}{3.200903in}}%
\pgfpathlineto{\pgfqpoint{5.458771in}{5.579823in}}%
\pgfusepath{stroke}%
\end{pgfscope}%
\begin{pgfscope}%
\pgfsetbuttcap%
\pgfsetroundjoin%
\definecolor{currentfill}{rgb}{0.000000,0.000000,0.000000}%
\pgfsetfillcolor{currentfill}%
\pgfsetlinewidth{0.803000pt}%
\definecolor{currentstroke}{rgb}{0.000000,0.000000,0.000000}%
\pgfsetstrokecolor{currentstroke}%
\pgfsetdash{}{0pt}%
\pgfsys@defobject{currentmarker}{\pgfqpoint{0.000000in}{-0.048611in}}{\pgfqpoint{0.000000in}{0.000000in}}{%
\pgfpathmoveto{\pgfqpoint{0.000000in}{0.000000in}}%
\pgfpathlineto{\pgfqpoint{0.000000in}{-0.048611in}}%
\pgfusepath{stroke,fill}%
}%
\begin{pgfscope}%
\pgfsys@transformshift{5.458771in}{3.200903in}%
\pgfsys@useobject{currentmarker}{}%
\end{pgfscope}%
\end{pgfscope}%
\begin{pgfscope}%
\definecolor{textcolor}{rgb}{0.000000,0.000000,0.000000}%
\pgfsetstrokecolor{textcolor}%
\pgfsetfillcolor{textcolor}%
\pgftext[x=5.458771in,y=3.103681in,,top]{\color{textcolor}{\rmfamily\fontsize{14.000000}{16.800000}\selectfont\catcode`\^=\active\def^{\ifmmode\sp\else\^{}\fi}\catcode`\%=\active\def%{\%}$\mathdefault{5300}$}}%
\end{pgfscope}%
\begin{pgfscope}%
\pgfpathrectangle{\pgfqpoint{4.244096in}{3.200903in}}{\pgfqpoint{3.342550in}{2.378919in}}%
\pgfusepath{clip}%
\pgfsetrectcap%
\pgfsetroundjoin%
\pgfsetlinewidth{0.803000pt}%
\definecolor{currentstroke}{rgb}{0.690196,0.690196,0.690196}%
\pgfsetstrokecolor{currentstroke}%
\pgfsetdash{}{0pt}%
\pgfpathmoveto{\pgfqpoint{6.309921in}{3.200903in}}%
\pgfpathlineto{\pgfqpoint{6.309921in}{5.579823in}}%
\pgfusepath{stroke}%
\end{pgfscope}%
\begin{pgfscope}%
\pgfsetbuttcap%
\pgfsetroundjoin%
\definecolor{currentfill}{rgb}{0.000000,0.000000,0.000000}%
\pgfsetfillcolor{currentfill}%
\pgfsetlinewidth{0.803000pt}%
\definecolor{currentstroke}{rgb}{0.000000,0.000000,0.000000}%
\pgfsetstrokecolor{currentstroke}%
\pgfsetdash{}{0pt}%
\pgfsys@defobject{currentmarker}{\pgfqpoint{0.000000in}{-0.048611in}}{\pgfqpoint{0.000000in}{0.000000in}}{%
\pgfpathmoveto{\pgfqpoint{0.000000in}{0.000000in}}%
\pgfpathlineto{\pgfqpoint{0.000000in}{-0.048611in}}%
\pgfusepath{stroke,fill}%
}%
\begin{pgfscope}%
\pgfsys@transformshift{6.309921in}{3.200903in}%
\pgfsys@useobject{currentmarker}{}%
\end{pgfscope}%
\end{pgfscope}%
\begin{pgfscope}%
\definecolor{textcolor}{rgb}{0.000000,0.000000,0.000000}%
\pgfsetstrokecolor{textcolor}%
\pgfsetfillcolor{textcolor}%
\pgftext[x=6.309921in,y=3.103681in,,top]{\color{textcolor}{\rmfamily\fontsize{14.000000}{16.800000}\selectfont\catcode`\^=\active\def^{\ifmmode\sp\else\^{}\fi}\catcode`\%=\active\def%{\%}$\mathdefault{5320}$}}%
\end{pgfscope}%
\begin{pgfscope}%
\pgfpathrectangle{\pgfqpoint{4.244096in}{3.200903in}}{\pgfqpoint{3.342550in}{2.378919in}}%
\pgfusepath{clip}%
\pgfsetrectcap%
\pgfsetroundjoin%
\pgfsetlinewidth{0.803000pt}%
\definecolor{currentstroke}{rgb}{0.690196,0.690196,0.690196}%
\pgfsetstrokecolor{currentstroke}%
\pgfsetdash{}{0pt}%
\pgfpathmoveto{\pgfqpoint{7.161071in}{3.200903in}}%
\pgfpathlineto{\pgfqpoint{7.161071in}{5.579823in}}%
\pgfusepath{stroke}%
\end{pgfscope}%
\begin{pgfscope}%
\pgfsetbuttcap%
\pgfsetroundjoin%
\definecolor{currentfill}{rgb}{0.000000,0.000000,0.000000}%
\pgfsetfillcolor{currentfill}%
\pgfsetlinewidth{0.803000pt}%
\definecolor{currentstroke}{rgb}{0.000000,0.000000,0.000000}%
\pgfsetstrokecolor{currentstroke}%
\pgfsetdash{}{0pt}%
\pgfsys@defobject{currentmarker}{\pgfqpoint{0.000000in}{-0.048611in}}{\pgfqpoint{0.000000in}{0.000000in}}{%
\pgfpathmoveto{\pgfqpoint{0.000000in}{0.000000in}}%
\pgfpathlineto{\pgfqpoint{0.000000in}{-0.048611in}}%
\pgfusepath{stroke,fill}%
}%
\begin{pgfscope}%
\pgfsys@transformshift{7.161071in}{3.200903in}%
\pgfsys@useobject{currentmarker}{}%
\end{pgfscope}%
\end{pgfscope}%
\begin{pgfscope}%
\definecolor{textcolor}{rgb}{0.000000,0.000000,0.000000}%
\pgfsetstrokecolor{textcolor}%
\pgfsetfillcolor{textcolor}%
\pgftext[x=7.161071in,y=3.103681in,,top]{\color{textcolor}{\rmfamily\fontsize{14.000000}{16.800000}\selectfont\catcode`\^=\active\def^{\ifmmode\sp\else\^{}\fi}\catcode`\%=\active\def%{\%}$\mathdefault{5340}$}}%
\end{pgfscope}%
\begin{pgfscope}%
\pgfpathrectangle{\pgfqpoint{4.244096in}{3.200903in}}{\pgfqpoint{3.342550in}{2.378919in}}%
\pgfusepath{clip}%
\pgfsetrectcap%
\pgfsetroundjoin%
\pgfsetlinewidth{0.803000pt}%
\definecolor{currentstroke}{rgb}{0.690196,0.690196,0.690196}%
\pgfsetstrokecolor{currentstroke}%
\pgfsetdash{}{0pt}%
\pgfpathmoveto{\pgfqpoint{4.244096in}{3.589621in}}%
\pgfpathlineto{\pgfqpoint{7.586646in}{3.589621in}}%
\pgfusepath{stroke}%
\end{pgfscope}%
\begin{pgfscope}%
\pgfsetbuttcap%
\pgfsetroundjoin%
\definecolor{currentfill}{rgb}{0.000000,0.000000,0.000000}%
\pgfsetfillcolor{currentfill}%
\pgfsetlinewidth{0.803000pt}%
\definecolor{currentstroke}{rgb}{0.000000,0.000000,0.000000}%
\pgfsetstrokecolor{currentstroke}%
\pgfsetdash{}{0pt}%
\pgfsys@defobject{currentmarker}{\pgfqpoint{-0.048611in}{0.000000in}}{\pgfqpoint{-0.000000in}{0.000000in}}{%
\pgfpathmoveto{\pgfqpoint{-0.000000in}{0.000000in}}%
\pgfpathlineto{\pgfqpoint{-0.048611in}{0.000000in}}%
\pgfusepath{stroke,fill}%
}%
\begin{pgfscope}%
\pgfsys@transformshift{4.244096in}{3.589621in}%
\pgfsys@useobject{currentmarker}{}%
\end{pgfscope}%
\end{pgfscope}%
\begin{pgfscope}%
\definecolor{textcolor}{rgb}{0.000000,0.000000,0.000000}%
\pgfsetstrokecolor{textcolor}%
\pgfsetfillcolor{textcolor}%
\pgftext[x=3.951043in, y=3.520176in, left, base]{\color{textcolor}{\rmfamily\fontsize{14.000000}{16.800000}\selectfont\catcode`\^=\active\def^{\ifmmode\sp\else\^{}\fi}\catcode`\%=\active\def%{\%}$\mathdefault{10}$}}%
\end{pgfscope}%
\begin{pgfscope}%
\pgfpathrectangle{\pgfqpoint{4.244096in}{3.200903in}}{\pgfqpoint{3.342550in}{2.378919in}}%
\pgfusepath{clip}%
\pgfsetrectcap%
\pgfsetroundjoin%
\pgfsetlinewidth{0.803000pt}%
\definecolor{currentstroke}{rgb}{0.690196,0.690196,0.690196}%
\pgfsetstrokecolor{currentstroke}%
\pgfsetdash{}{0pt}%
\pgfpathmoveto{\pgfqpoint{4.244096in}{4.367055in}}%
\pgfpathlineto{\pgfqpoint{7.586646in}{4.367055in}}%
\pgfusepath{stroke}%
\end{pgfscope}%
\begin{pgfscope}%
\pgfsetbuttcap%
\pgfsetroundjoin%
\definecolor{currentfill}{rgb}{0.000000,0.000000,0.000000}%
\pgfsetfillcolor{currentfill}%
\pgfsetlinewidth{0.803000pt}%
\definecolor{currentstroke}{rgb}{0.000000,0.000000,0.000000}%
\pgfsetstrokecolor{currentstroke}%
\pgfsetdash{}{0pt}%
\pgfsys@defobject{currentmarker}{\pgfqpoint{-0.048611in}{0.000000in}}{\pgfqpoint{-0.000000in}{0.000000in}}{%
\pgfpathmoveto{\pgfqpoint{-0.000000in}{0.000000in}}%
\pgfpathlineto{\pgfqpoint{-0.048611in}{0.000000in}}%
\pgfusepath{stroke,fill}%
}%
\begin{pgfscope}%
\pgfsys@transformshift{4.244096in}{4.367055in}%
\pgfsys@useobject{currentmarker}{}%
\end{pgfscope}%
\end{pgfscope}%
\begin{pgfscope}%
\definecolor{textcolor}{rgb}{0.000000,0.000000,0.000000}%
\pgfsetstrokecolor{textcolor}%
\pgfsetfillcolor{textcolor}%
\pgftext[x=3.951043in, y=4.297611in, left, base]{\color{textcolor}{\rmfamily\fontsize{14.000000}{16.800000}\selectfont\catcode`\^=\active\def^{\ifmmode\sp\else\^{}\fi}\catcode`\%=\active\def%{\%}$\mathdefault{12}$}}%
\end{pgfscope}%
\begin{pgfscope}%
\pgfpathrectangle{\pgfqpoint{4.244096in}{3.200903in}}{\pgfqpoint{3.342550in}{2.378919in}}%
\pgfusepath{clip}%
\pgfsetrectcap%
\pgfsetroundjoin%
\pgfsetlinewidth{0.803000pt}%
\definecolor{currentstroke}{rgb}{0.690196,0.690196,0.690196}%
\pgfsetstrokecolor{currentstroke}%
\pgfsetdash{}{0pt}%
\pgfpathmoveto{\pgfqpoint{4.244096in}{5.144490in}}%
\pgfpathlineto{\pgfqpoint{7.586646in}{5.144490in}}%
\pgfusepath{stroke}%
\end{pgfscope}%
\begin{pgfscope}%
\pgfsetbuttcap%
\pgfsetroundjoin%
\definecolor{currentfill}{rgb}{0.000000,0.000000,0.000000}%
\pgfsetfillcolor{currentfill}%
\pgfsetlinewidth{0.803000pt}%
\definecolor{currentstroke}{rgb}{0.000000,0.000000,0.000000}%
\pgfsetstrokecolor{currentstroke}%
\pgfsetdash{}{0pt}%
\pgfsys@defobject{currentmarker}{\pgfqpoint{-0.048611in}{0.000000in}}{\pgfqpoint{-0.000000in}{0.000000in}}{%
\pgfpathmoveto{\pgfqpoint{-0.000000in}{0.000000in}}%
\pgfpathlineto{\pgfqpoint{-0.048611in}{0.000000in}}%
\pgfusepath{stroke,fill}%
}%
\begin{pgfscope}%
\pgfsys@transformshift{4.244096in}{5.144490in}%
\pgfsys@useobject{currentmarker}{}%
\end{pgfscope}%
\end{pgfscope}%
\begin{pgfscope}%
\definecolor{textcolor}{rgb}{0.000000,0.000000,0.000000}%
\pgfsetstrokecolor{textcolor}%
\pgfsetfillcolor{textcolor}%
\pgftext[x=3.951043in, y=5.075046in, left, base]{\color{textcolor}{\rmfamily\fontsize{14.000000}{16.800000}\selectfont\catcode`\^=\active\def^{\ifmmode\sp\else\^{}\fi}\catcode`\%=\active\def%{\%}$\mathdefault{14}$}}%
\end{pgfscope}%
\begin{pgfscope}%
\pgfpathrectangle{\pgfqpoint{4.244096in}{3.200903in}}{\pgfqpoint{3.342550in}{2.378919in}}%
\pgfusepath{clip}%
\pgfsetrectcap%
\pgfsetroundjoin%
\pgfsetlinewidth{1.505625pt}%
\definecolor{currentstroke}{rgb}{0.000000,0.000000,0.000000}%
\pgfsetstrokecolor{currentstroke}%
\pgfsetdash{}{0pt}%
\pgfpathmoveto{\pgfqpoint{6.183131in}{5.204032in}}%
\pgfpathlineto{\pgfqpoint{6.249561in}{4.054913in}}%
\pgfpathlineto{\pgfqpoint{6.269290in}{3.467400in}}%
\pgfpathlineto{\pgfqpoint{6.389537in}{3.359169in}}%
\pgfpathlineto{\pgfqpoint{6.404531in}{3.198403in}}%
\pgfpathlineto{\pgfqpoint{6.404531in}{3.198403in}}%
\pgfusepath{stroke}%
\end{pgfscope}%
\begin{pgfscope}%
\pgfpathrectangle{\pgfqpoint{4.244096in}{3.200903in}}{\pgfqpoint{3.342550in}{2.378919in}}%
\pgfusepath{clip}%
\pgfsetbuttcap%
\pgfsetroundjoin%
\definecolor{currentfill}{rgb}{0.000000,0.000000,0.000000}%
\pgfsetfillcolor{currentfill}%
\pgfsetlinewidth{1.003750pt}%
\definecolor{currentstroke}{rgb}{0.000000,0.000000,0.000000}%
\pgfsetstrokecolor{currentstroke}%
\pgfsetdash{}{0pt}%
\pgfsys@defobject{currentmarker}{\pgfqpoint{-0.041667in}{-0.041667in}}{\pgfqpoint{0.041667in}{0.041667in}}{%
\pgfpathmoveto{\pgfqpoint{0.000000in}{-0.041667in}}%
\pgfpathcurveto{\pgfqpoint{0.011050in}{-0.041667in}}{\pgfqpoint{0.021649in}{-0.037276in}}{\pgfqpoint{0.029463in}{-0.029463in}}%
\pgfpathcurveto{\pgfqpoint{0.037276in}{-0.021649in}}{\pgfqpoint{0.041667in}{-0.011050in}}{\pgfqpoint{0.041667in}{0.000000in}}%
\pgfpathcurveto{\pgfqpoint{0.041667in}{0.011050in}}{\pgfqpoint{0.037276in}{0.021649in}}{\pgfqpoint{0.029463in}{0.029463in}}%
\pgfpathcurveto{\pgfqpoint{0.021649in}{0.037276in}}{\pgfqpoint{0.011050in}{0.041667in}}{\pgfqpoint{0.000000in}{0.041667in}}%
\pgfpathcurveto{\pgfqpoint{-0.011050in}{0.041667in}}{\pgfqpoint{-0.021649in}{0.037276in}}{\pgfqpoint{-0.029463in}{0.029463in}}%
\pgfpathcurveto{\pgfqpoint{-0.037276in}{0.021649in}}{\pgfqpoint{-0.041667in}{0.011050in}}{\pgfqpoint{-0.041667in}{0.000000in}}%
\pgfpathcurveto{\pgfqpoint{-0.041667in}{-0.011050in}}{\pgfqpoint{-0.037276in}{-0.021649in}}{\pgfqpoint{-0.029463in}{-0.029463in}}%
\pgfpathcurveto{\pgfqpoint{-0.021649in}{-0.037276in}}{\pgfqpoint{-0.011050in}{-0.041667in}}{\pgfqpoint{0.000000in}{-0.041667in}}%
\pgfpathlineto{\pgfqpoint{0.000000in}{-0.041667in}}%
\pgfpathclose%
\pgfusepath{stroke,fill}%
}%
\begin{pgfscope}%
\pgfsys@transformshift{6.183131in}{5.204032in}%
\pgfsys@useobject{currentmarker}{}%
\end{pgfscope}%
\begin{pgfscope}%
\pgfsys@transformshift{6.249561in}{4.054913in}%
\pgfsys@useobject{currentmarker}{}%
\end{pgfscope}%
\begin{pgfscope}%
\pgfsys@transformshift{6.269290in}{3.467400in}%
\pgfsys@useobject{currentmarker}{}%
\end{pgfscope}%
\begin{pgfscope}%
\pgfsys@transformshift{6.389537in}{3.359169in}%
\pgfsys@useobject{currentmarker}{}%
\end{pgfscope}%
\begin{pgfscope}%
\pgfsys@transformshift{6.456979in}{2.636044in}%
\pgfsys@useobject{currentmarker}{}%
\end{pgfscope}%
\begin{pgfscope}%
\pgfsys@transformshift{6.536223in}{2.439219in}%
\pgfsys@useobject{currentmarker}{}%
\end{pgfscope}%
\begin{pgfscope}%
\pgfsys@transformshift{6.575820in}{2.438061in}%
\pgfsys@useobject{currentmarker}{}%
\end{pgfscope}%
\begin{pgfscope}%
\pgfsys@transformshift{6.672999in}{2.197456in}%
\pgfsys@useobject{currentmarker}{}%
\end{pgfscope}%
\begin{pgfscope}%
\pgfsys@transformshift{6.677744in}{2.062234in}%
\pgfsys@useobject{currentmarker}{}%
\end{pgfscope}%
\begin{pgfscope}%
\pgfsys@transformshift{6.730258in}{1.915726in}%
\pgfsys@useobject{currentmarker}{}%
\end{pgfscope}%
\begin{pgfscope}%
\pgfsys@transformshift{6.764468in}{1.880232in}%
\pgfsys@useobject{currentmarker}{}%
\end{pgfscope}%
\begin{pgfscope}%
\pgfsys@transformshift{6.781373in}{1.848654in}%
\pgfsys@useobject{currentmarker}{}%
\end{pgfscope}%
\begin{pgfscope}%
\pgfsys@transformshift{6.782778in}{1.845385in}%
\pgfsys@useobject{currentmarker}{}%
\end{pgfscope}%
\begin{pgfscope}%
\pgfsys@transformshift{6.789324in}{1.648392in}%
\pgfsys@useobject{currentmarker}{}%
\end{pgfscope}%
\begin{pgfscope}%
\pgfsys@transformshift{6.789324in}{1.648392in}%
\pgfsys@useobject{currentmarker}{}%
\end{pgfscope}%
\begin{pgfscope}%
\pgfsys@transformshift{6.864070in}{1.613063in}%
\pgfsys@useobject{currentmarker}{}%
\end{pgfscope}%
\begin{pgfscope}%
\pgfsys@transformshift{6.935536in}{1.571928in}%
\pgfsys@useobject{currentmarker}{}%
\end{pgfscope}%
\begin{pgfscope}%
\pgfsys@transformshift{6.967891in}{1.407934in}%
\pgfsys@useobject{currentmarker}{}%
\end{pgfscope}%
\begin{pgfscope}%
\pgfsys@transformshift{6.968024in}{1.407547in}%
\pgfsys@useobject{currentmarker}{}%
\end{pgfscope}%
\begin{pgfscope}%
\pgfsys@transformshift{7.015571in}{1.356747in}%
\pgfsys@useobject{currentmarker}{}%
\end{pgfscope}%
\begin{pgfscope}%
\pgfsys@transformshift{7.018753in}{1.313843in}%
\pgfsys@useobject{currentmarker}{}%
\end{pgfscope}%
\begin{pgfscope}%
\pgfsys@transformshift{7.104609in}{1.293074in}%
\pgfsys@useobject{currentmarker}{}%
\end{pgfscope}%
\begin{pgfscope}%
\pgfsys@transformshift{7.227669in}{1.288262in}%
\pgfsys@useobject{currentmarker}{}%
\end{pgfscope}%
\begin{pgfscope}%
\pgfsys@transformshift{7.271213in}{1.150259in}%
\pgfsys@useobject{currentmarker}{}%
\end{pgfscope}%
\begin{pgfscope}%
\pgfsys@transformshift{7.376840in}{1.116716in}%
\pgfsys@useobject{currentmarker}{}%
\end{pgfscope}%
\begin{pgfscope}%
\pgfsys@transformshift{7.571657in}{1.110486in}%
\pgfsys@useobject{currentmarker}{}%
\end{pgfscope}%
\begin{pgfscope}%
\pgfsys@transformshift{7.638262in}{1.058114in}%
\pgfsys@useobject{currentmarker}{}%
\end{pgfscope}%
\begin{pgfscope}%
\pgfsys@transformshift{7.706582in}{1.048952in}%
\pgfsys@useobject{currentmarker}{}%
\end{pgfscope}%
\begin{pgfscope}%
\pgfsys@transformshift{7.752811in}{1.042069in}%
\pgfsys@useobject{currentmarker}{}%
\end{pgfscope}%
\begin{pgfscope}%
\pgfsys@transformshift{7.841011in}{1.023902in}%
\pgfsys@useobject{currentmarker}{}%
\end{pgfscope}%
\begin{pgfscope}%
\pgfsys@transformshift{7.847894in}{1.023059in}%
\pgfsys@useobject{currentmarker}{}%
\end{pgfscope}%
\begin{pgfscope}%
\pgfsys@transformshift{7.849942in}{1.022137in}%
\pgfsys@useobject{currentmarker}{}%
\end{pgfscope}%
\begin{pgfscope}%
\pgfsys@transformshift{7.967962in}{0.993325in}%
\pgfsys@useobject{currentmarker}{}%
\end{pgfscope}%
\begin{pgfscope}%
\pgfsys@transformshift{7.977189in}{0.988385in}%
\pgfsys@useobject{currentmarker}{}%
\end{pgfscope}%
\begin{pgfscope}%
\pgfsys@transformshift{7.982834in}{0.982653in}%
\pgfsys@useobject{currentmarker}{}%
\end{pgfscope}%
\begin{pgfscope}%
\pgfsys@transformshift{8.055457in}{0.972283in}%
\pgfsys@useobject{currentmarker}{}%
\end{pgfscope}%
\begin{pgfscope}%
\pgfsys@transformshift{8.370492in}{0.942646in}%
\pgfsys@useobject{currentmarker}{}%
\end{pgfscope}%
\begin{pgfscope}%
\pgfsys@transformshift{8.705529in}{0.901352in}%
\pgfsys@useobject{currentmarker}{}%
\end{pgfscope}%
\begin{pgfscope}%
\pgfsys@transformshift{9.160238in}{0.871890in}%
\pgfsys@useobject{currentmarker}{}%
\end{pgfscope}%
\begin{pgfscope}%
\pgfsys@transformshift{9.576627in}{0.861995in}%
\pgfsys@useobject{currentmarker}{}%
\end{pgfscope}%
\begin{pgfscope}%
\pgfsys@transformshift{9.593600in}{0.848929in}%
\pgfsys@useobject{currentmarker}{}%
\end{pgfscope}%
\begin{pgfscope}%
\pgfsys@transformshift{9.618189in}{0.826140in}%
\pgfsys@useobject{currentmarker}{}%
\end{pgfscope}%
\begin{pgfscope}%
\pgfsys@transformshift{9.676172in}{0.821823in}%
\pgfsys@useobject{currentmarker}{}%
\end{pgfscope}%
\begin{pgfscope}%
\pgfsys@transformshift{9.684133in}{0.818018in}%
\pgfsys@useobject{currentmarker}{}%
\end{pgfscope}%
\begin{pgfscope}%
\pgfsys@transformshift{9.692223in}{0.817932in}%
\pgfsys@useobject{currentmarker}{}%
\end{pgfscope}%
\begin{pgfscope}%
\pgfsys@transformshift{9.841496in}{0.814346in}%
\pgfsys@useobject{currentmarker}{}%
\end{pgfscope}%
\begin{pgfscope}%
\pgfsys@transformshift{9.917418in}{0.807146in}%
\pgfsys@useobject{currentmarker}{}%
\end{pgfscope}%
\begin{pgfscope}%
\pgfsys@transformshift{9.948382in}{0.804628in}%
\pgfsys@useobject{currentmarker}{}%
\end{pgfscope}%
\begin{pgfscope}%
\pgfsys@transformshift{10.423157in}{0.789533in}%
\pgfsys@useobject{currentmarker}{}%
\end{pgfscope}%
\begin{pgfscope}%
\pgfsys@transformshift{10.485339in}{0.781523in}%
\pgfsys@useobject{currentmarker}{}%
\end{pgfscope}%
\begin{pgfscope}%
\pgfsys@transformshift{10.487167in}{0.772911in}%
\pgfsys@useobject{currentmarker}{}%
\end{pgfscope}%
\begin{pgfscope}%
\pgfsys@transformshift{10.589219in}{0.771488in}%
\pgfsys@useobject{currentmarker}{}%
\end{pgfscope}%
\begin{pgfscope}%
\pgfsys@transformshift{10.830360in}{0.757023in}%
\pgfsys@useobject{currentmarker}{}%
\end{pgfscope}%
\begin{pgfscope}%
\pgfsys@transformshift{10.848103in}{0.755610in}%
\pgfsys@useobject{currentmarker}{}%
\end{pgfscope}%
\begin{pgfscope}%
\pgfsys@transformshift{10.944464in}{0.755552in}%
\pgfsys@useobject{currentmarker}{}%
\end{pgfscope}%
\begin{pgfscope}%
\pgfsys@transformshift{10.962113in}{0.750156in}%
\pgfsys@useobject{currentmarker}{}%
\end{pgfscope}%
\begin{pgfscope}%
\pgfsys@transformshift{11.912222in}{0.736291in}%
\pgfsys@useobject{currentmarker}{}%
\end{pgfscope}%
\begin{pgfscope}%
\pgfsys@transformshift{11.986609in}{0.709217in}%
\pgfsys@useobject{currentmarker}{}%
\end{pgfscope}%
\begin{pgfscope}%
\pgfsys@transformshift{12.028667in}{0.707130in}%
\pgfsys@useobject{currentmarker}{}%
\end{pgfscope}%
\begin{pgfscope}%
\pgfsys@transformshift{12.028870in}{0.707130in}%
\pgfsys@useobject{currentmarker}{}%
\end{pgfscope}%
\begin{pgfscope}%
\pgfsys@transformshift{12.149871in}{0.705225in}%
\pgfsys@useobject{currentmarker}{}%
\end{pgfscope}%
\begin{pgfscope}%
\pgfsys@transformshift{12.179313in}{0.703504in}%
\pgfsys@useobject{currentmarker}{}%
\end{pgfscope}%
\begin{pgfscope}%
\pgfsys@transformshift{12.477058in}{0.689705in}%
\pgfsys@useobject{currentmarker}{}%
\end{pgfscope}%
\begin{pgfscope}%
\pgfsys@transformshift{12.490844in}{0.688509in}%
\pgfsys@useobject{currentmarker}{}%
\end{pgfscope}%
\begin{pgfscope}%
\pgfsys@transformshift{12.584840in}{0.686767in}%
\pgfsys@useobject{currentmarker}{}%
\end{pgfscope}%
\begin{pgfscope}%
\pgfsys@transformshift{12.585533in}{0.686190in}%
\pgfsys@useobject{currentmarker}{}%
\end{pgfscope}%
\begin{pgfscope}%
\pgfsys@transformshift{12.610134in}{0.682975in}%
\pgfsys@useobject{currentmarker}{}%
\end{pgfscope}%
\begin{pgfscope}%
\pgfsys@transformshift{12.685729in}{0.681799in}%
\pgfsys@useobject{currentmarker}{}%
\end{pgfscope}%
\begin{pgfscope}%
\pgfsys@transformshift{12.730674in}{0.681670in}%
\pgfsys@useobject{currentmarker}{}%
\end{pgfscope}%
\begin{pgfscope}%
\pgfsys@transformshift{12.911389in}{0.676667in}%
\pgfsys@useobject{currentmarker}{}%
\end{pgfscope}%
\begin{pgfscope}%
\pgfsys@transformshift{12.941669in}{0.672708in}%
\pgfsys@useobject{currentmarker}{}%
\end{pgfscope}%
\begin{pgfscope}%
\pgfsys@transformshift{12.946295in}{0.672668in}%
\pgfsys@useobject{currentmarker}{}%
\end{pgfscope}%
\begin{pgfscope}%
\pgfsys@transformshift{13.281821in}{0.672388in}%
\pgfsys@useobject{currentmarker}{}%
\end{pgfscope}%
\begin{pgfscope}%
\pgfsys@transformshift{13.304894in}{0.669039in}%
\pgfsys@useobject{currentmarker}{}%
\end{pgfscope}%
\begin{pgfscope}%
\pgfsys@transformshift{13.351761in}{0.668359in}%
\pgfsys@useobject{currentmarker}{}%
\end{pgfscope}%
\begin{pgfscope}%
\pgfsys@transformshift{13.490100in}{0.667448in}%
\pgfsys@useobject{currentmarker}{}%
\end{pgfscope}%
\begin{pgfscope}%
\pgfsys@transformshift{13.573499in}{0.667212in}%
\pgfsys@useobject{currentmarker}{}%
\end{pgfscope}%
\begin{pgfscope}%
\pgfsys@transformshift{13.821810in}{0.666382in}%
\pgfsys@useobject{currentmarker}{}%
\end{pgfscope}%
\begin{pgfscope}%
\pgfsys@transformshift{14.161557in}{0.666117in}%
\pgfsys@useobject{currentmarker}{}%
\end{pgfscope}%
\begin{pgfscope}%
\pgfsys@transformshift{14.195159in}{0.666116in}%
\pgfsys@useobject{currentmarker}{}%
\end{pgfscope}%
\begin{pgfscope}%
\pgfsys@transformshift{14.216394in}{0.666071in}%
\pgfsys@useobject{currentmarker}{}%
\end{pgfscope}%
\begin{pgfscope}%
\pgfsys@transformshift{14.586512in}{0.666039in}%
\pgfsys@useobject{currentmarker}{}%
\end{pgfscope}%
\begin{pgfscope}%
\pgfsys@transformshift{16.273377in}{0.662779in}%
\pgfsys@useobject{currentmarker}{}%
\end{pgfscope}%
\begin{pgfscope}%
\pgfsys@transformshift{16.603212in}{0.660831in}%
\pgfsys@useobject{currentmarker}{}%
\end{pgfscope}%
\begin{pgfscope}%
\pgfsys@transformshift{17.122676in}{0.660240in}%
\pgfsys@useobject{currentmarker}{}%
\end{pgfscope}%
\begin{pgfscope}%
\pgfsys@transformshift{17.137897in}{0.660151in}%
\pgfsys@useobject{currentmarker}{}%
\end{pgfscope}%
\begin{pgfscope}%
\pgfsys@transformshift{17.353940in}{0.659476in}%
\pgfsys@useobject{currentmarker}{}%
\end{pgfscope}%
\begin{pgfscope}%
\pgfsys@transformshift{17.716115in}{0.659004in}%
\pgfsys@useobject{currentmarker}{}%
\end{pgfscope}%
\begin{pgfscope}%
\pgfsys@transformshift{17.775411in}{0.658636in}%
\pgfsys@useobject{currentmarker}{}%
\end{pgfscope}%
\begin{pgfscope}%
\pgfsys@transformshift{17.902085in}{0.658159in}%
\pgfsys@useobject{currentmarker}{}%
\end{pgfscope}%
\begin{pgfscope}%
\pgfsys@transformshift{18.564983in}{0.657947in}%
\pgfsys@useobject{currentmarker}{}%
\end{pgfscope}%
\begin{pgfscope}%
\pgfsys@transformshift{18.768429in}{0.657657in}%
\pgfsys@useobject{currentmarker}{}%
\end{pgfscope}%
\begin{pgfscope}%
\pgfsys@transformshift{19.013470in}{0.657511in}%
\pgfsys@useobject{currentmarker}{}%
\end{pgfscope}%
\begin{pgfscope}%
\pgfsys@transformshift{19.070619in}{0.657497in}%
\pgfsys@useobject{currentmarker}{}%
\end{pgfscope}%
\begin{pgfscope}%
\pgfsys@transformshift{19.087175in}{0.657429in}%
\pgfsys@useobject{currentmarker}{}%
\end{pgfscope}%
\begin{pgfscope}%
\pgfsys@transformshift{19.093845in}{0.657388in}%
\pgfsys@useobject{currentmarker}{}%
\end{pgfscope}%
\begin{pgfscope}%
\pgfsys@transformshift{19.099542in}{0.657351in}%
\pgfsys@useobject{currentmarker}{}%
\end{pgfscope}%
\begin{pgfscope}%
\pgfsys@transformshift{19.129200in}{0.657212in}%
\pgfsys@useobject{currentmarker}{}%
\end{pgfscope}%
\begin{pgfscope}%
\pgfsys@transformshift{19.154518in}{0.656992in}%
\pgfsys@useobject{currentmarker}{}%
\end{pgfscope}%
\begin{pgfscope}%
\pgfsys@transformshift{19.485643in}{0.655036in}%
\pgfsys@useobject{currentmarker}{}%
\end{pgfscope}%
\begin{pgfscope}%
\pgfsys@transformshift{19.614987in}{0.654449in}%
\pgfsys@useobject{currentmarker}{}%
\end{pgfscope}%
\begin{pgfscope}%
\pgfsys@transformshift{21.028438in}{0.653442in}%
\pgfsys@useobject{currentmarker}{}%
\end{pgfscope}%
\begin{pgfscope}%
\pgfsys@transformshift{22.388020in}{0.653421in}%
\pgfsys@useobject{currentmarker}{}%
\end{pgfscope}%
\begin{pgfscope}%
\pgfsys@transformshift{22.407524in}{0.652526in}%
\pgfsys@useobject{currentmarker}{}%
\end{pgfscope}%
\begin{pgfscope}%
\pgfsys@transformshift{22.466045in}{0.652155in}%
\pgfsys@useobject{currentmarker}{}%
\end{pgfscope}%
\begin{pgfscope}%
\pgfsys@transformshift{22.677220in}{0.651758in}%
\pgfsys@useobject{currentmarker}{}%
\end{pgfscope}%
\begin{pgfscope}%
\pgfsys@transformshift{22.733327in}{0.651029in}%
\pgfsys@useobject{currentmarker}{}%
\end{pgfscope}%
\begin{pgfscope}%
\pgfsys@transformshift{22.798021in}{0.650730in}%
\pgfsys@useobject{currentmarker}{}%
\end{pgfscope}%
\begin{pgfscope}%
\pgfsys@transformshift{22.850812in}{0.650473in}%
\pgfsys@useobject{currentmarker}{}%
\end{pgfscope}%
\begin{pgfscope}%
\pgfsys@transformshift{23.463873in}{0.650116in}%
\pgfsys@useobject{currentmarker}{}%
\end{pgfscope}%
\begin{pgfscope}%
\pgfsys@transformshift{23.674988in}{0.649071in}%
\pgfsys@useobject{currentmarker}{}%
\end{pgfscope}%
\begin{pgfscope}%
\pgfsys@transformshift{23.683338in}{0.649020in}%
\pgfsys@useobject{currentmarker}{}%
\end{pgfscope}%
\begin{pgfscope}%
\pgfsys@transformshift{23.683338in}{0.649020in}%
\pgfsys@useobject{currentmarker}{}%
\end{pgfscope}%
\begin{pgfscope}%
\pgfsys@transformshift{23.686514in}{0.649003in}%
\pgfsys@useobject{currentmarker}{}%
\end{pgfscope}%
\begin{pgfscope}%
\pgfsys@transformshift{23.693730in}{0.648957in}%
\pgfsys@useobject{currentmarker}{}%
\end{pgfscope}%
\begin{pgfscope}%
\pgfsys@transformshift{23.798767in}{0.647701in}%
\pgfsys@useobject{currentmarker}{}%
\end{pgfscope}%
\begin{pgfscope}%
\pgfsys@transformshift{23.798767in}{0.647701in}%
\pgfsys@useobject{currentmarker}{}%
\end{pgfscope}%
\begin{pgfscope}%
\pgfsys@transformshift{24.104536in}{0.647350in}%
\pgfsys@useobject{currentmarker}{}%
\end{pgfscope}%
\begin{pgfscope}%
\pgfsys@transformshift{24.190103in}{0.645882in}%
\pgfsys@useobject{currentmarker}{}%
\end{pgfscope}%
\begin{pgfscope}%
\pgfsys@transformshift{24.213483in}{0.645727in}%
\pgfsys@useobject{currentmarker}{}%
\end{pgfscope}%
\begin{pgfscope}%
\pgfsys@transformshift{24.285755in}{0.645454in}%
\pgfsys@useobject{currentmarker}{}%
\end{pgfscope}%
\begin{pgfscope}%
\pgfsys@transformshift{25.106846in}{0.645357in}%
\pgfsys@useobject{currentmarker}{}%
\end{pgfscope}%
\begin{pgfscope}%
\pgfsys@transformshift{25.106846in}{0.645357in}%
\pgfsys@useobject{currentmarker}{}%
\end{pgfscope}%
\begin{pgfscope}%
\pgfsys@transformshift{25.180730in}{0.645205in}%
\pgfsys@useobject{currentmarker}{}%
\end{pgfscope}%
\begin{pgfscope}%
\pgfsys@transformshift{25.383339in}{0.644286in}%
\pgfsys@useobject{currentmarker}{}%
\end{pgfscope}%
\begin{pgfscope}%
\pgfsys@transformshift{25.829983in}{0.643473in}%
\pgfsys@useobject{currentmarker}{}%
\end{pgfscope}%
\begin{pgfscope}%
\pgfsys@transformshift{26.543641in}{0.642604in}%
\pgfsys@useobject{currentmarker}{}%
\end{pgfscope}%
\begin{pgfscope}%
\pgfsys@transformshift{26.742367in}{0.641508in}%
\pgfsys@useobject{currentmarker}{}%
\end{pgfscope}%
\begin{pgfscope}%
\pgfsys@transformshift{27.192755in}{0.640852in}%
\pgfsys@useobject{currentmarker}{}%
\end{pgfscope}%
\begin{pgfscope}%
\pgfsys@transformshift{27.883961in}{0.640314in}%
\pgfsys@useobject{currentmarker}{}%
\end{pgfscope}%
\begin{pgfscope}%
\pgfsys@transformshift{27.948008in}{0.639163in}%
\pgfsys@useobject{currentmarker}{}%
\end{pgfscope}%
\begin{pgfscope}%
\pgfsys@transformshift{29.594915in}{0.636927in}%
\pgfsys@useobject{currentmarker}{}%
\end{pgfscope}%
\begin{pgfscope}%
\pgfsys@transformshift{32.475120in}{0.633968in}%
\pgfsys@useobject{currentmarker}{}%
\end{pgfscope}%
\begin{pgfscope}%
\pgfsys@transformshift{32.669427in}{0.631796in}%
\pgfsys@useobject{currentmarker}{}%
\end{pgfscope}%
\begin{pgfscope}%
\pgfsys@transformshift{32.691227in}{0.631731in}%
\pgfsys@useobject{currentmarker}{}%
\end{pgfscope}%
\begin{pgfscope}%
\pgfsys@transformshift{32.808700in}{0.631651in}%
\pgfsys@useobject{currentmarker}{}%
\end{pgfscope}%
\begin{pgfscope}%
\pgfsys@transformshift{32.830192in}{0.631549in}%
\pgfsys@useobject{currentmarker}{}%
\end{pgfscope}%
\begin{pgfscope}%
\pgfsys@transformshift{33.189460in}{0.631207in}%
\pgfsys@useobject{currentmarker}{}%
\end{pgfscope}%
\begin{pgfscope}%
\pgfsys@transformshift{35.205583in}{0.628188in}%
\pgfsys@useobject{currentmarker}{}%
\end{pgfscope}%
\begin{pgfscope}%
\pgfsys@transformshift{35.465921in}{0.626984in}%
\pgfsys@useobject{currentmarker}{}%
\end{pgfscope}%
\begin{pgfscope}%
\pgfsys@transformshift{36.115103in}{0.625037in}%
\pgfsys@useobject{currentmarker}{}%
\end{pgfscope}%
\begin{pgfscope}%
\pgfsys@transformshift{36.528089in}{0.624750in}%
\pgfsys@useobject{currentmarker}{}%
\end{pgfscope}%
\begin{pgfscope}%
\pgfsys@transformshift{37.183078in}{0.624698in}%
\pgfsys@useobject{currentmarker}{}%
\end{pgfscope}%
\begin{pgfscope}%
\pgfsys@transformshift{37.270027in}{0.624274in}%
\pgfsys@useobject{currentmarker}{}%
\end{pgfscope}%
\begin{pgfscope}%
\pgfsys@transformshift{37.549244in}{0.622731in}%
\pgfsys@useobject{currentmarker}{}%
\end{pgfscope}%
\begin{pgfscope}%
\pgfsys@transformshift{37.556982in}{0.622673in}%
\pgfsys@useobject{currentmarker}{}%
\end{pgfscope}%
\begin{pgfscope}%
\pgfsys@transformshift{38.135042in}{0.621562in}%
\pgfsys@useobject{currentmarker}{}%
\end{pgfscope}%
\begin{pgfscope}%
\pgfsys@transformshift{39.307371in}{0.620988in}%
\pgfsys@useobject{currentmarker}{}%
\end{pgfscope}%
\begin{pgfscope}%
\pgfsys@transformshift{40.450708in}{0.618305in}%
\pgfsys@useobject{currentmarker}{}%
\end{pgfscope}%
\begin{pgfscope}%
\pgfsys@transformshift{40.533336in}{0.618020in}%
\pgfsys@useobject{currentmarker}{}%
\end{pgfscope}%
\begin{pgfscope}%
\pgfsys@transformshift{44.703339in}{0.615663in}%
\pgfsys@useobject{currentmarker}{}%
\end{pgfscope}%
\begin{pgfscope}%
\pgfsys@transformshift{45.079133in}{0.612571in}%
\pgfsys@useobject{currentmarker}{}%
\end{pgfscope}%
\begin{pgfscope}%
\pgfsys@transformshift{45.861065in}{0.610979in}%
\pgfsys@useobject{currentmarker}{}%
\end{pgfscope}%
\begin{pgfscope}%
\pgfsys@transformshift{46.719722in}{0.609743in}%
\pgfsys@useobject{currentmarker}{}%
\end{pgfscope}%
\begin{pgfscope}%
\pgfsys@transformshift{51.287641in}{0.604750in}%
\pgfsys@useobject{currentmarker}{}%
\end{pgfscope}%
\begin{pgfscope}%
\pgfsys@transformshift{51.482064in}{0.603514in}%
\pgfsys@useobject{currentmarker}{}%
\end{pgfscope}%
\begin{pgfscope}%
\pgfsys@transformshift{56.259615in}{0.597141in}%
\pgfsys@useobject{currentmarker}{}%
\end{pgfscope}%
\begin{pgfscope}%
\pgfsys@transformshift{60.366413in}{0.594494in}%
\pgfsys@useobject{currentmarker}{}%
\end{pgfscope}%
\begin{pgfscope}%
\pgfsys@transformshift{60.991810in}{0.593591in}%
\pgfsys@useobject{currentmarker}{}%
\end{pgfscope}%
\begin{pgfscope}%
\pgfsys@transformshift{61.669359in}{0.592787in}%
\pgfsys@useobject{currentmarker}{}%
\end{pgfscope}%
\begin{pgfscope}%
\pgfsys@transformshift{63.865471in}{0.590530in}%
\pgfsys@useobject{currentmarker}{}%
\end{pgfscope}%
\begin{pgfscope}%
\pgfsys@transformshift{64.026773in}{0.590319in}%
\pgfsys@useobject{currentmarker}{}%
\end{pgfscope}%
\begin{pgfscope}%
\pgfsys@transformshift{66.863188in}{0.588176in}%
\pgfsys@useobject{currentmarker}{}%
\end{pgfscope}%
\begin{pgfscope}%
\pgfsys@transformshift{67.788674in}{0.587750in}%
\pgfsys@useobject{currentmarker}{}%
\end{pgfscope}%
\begin{pgfscope}%
\pgfsys@transformshift{67.897307in}{0.587653in}%
\pgfsys@useobject{currentmarker}{}%
\end{pgfscope}%
\begin{pgfscope}%
\pgfsys@transformshift{71.593556in}{0.587596in}%
\pgfsys@useobject{currentmarker}{}%
\end{pgfscope}%
\begin{pgfscope}%
\pgfsys@transformshift{71.599242in}{0.587587in}%
\pgfsys@useobject{currentmarker}{}%
\end{pgfscope}%
\begin{pgfscope}%
\pgfsys@transformshift{72.854923in}{0.587388in}%
\pgfsys@useobject{currentmarker}{}%
\end{pgfscope}%
\begin{pgfscope}%
\pgfsys@transformshift{74.376808in}{0.586926in}%
\pgfsys@useobject{currentmarker}{}%
\end{pgfscope}%
\begin{pgfscope}%
\pgfsys@transformshift{75.032298in}{0.586847in}%
\pgfsys@useobject{currentmarker}{}%
\end{pgfscope}%
\begin{pgfscope}%
\pgfsys@transformshift{75.342338in}{0.585741in}%
\pgfsys@useobject{currentmarker}{}%
\end{pgfscope}%
\begin{pgfscope}%
\pgfsys@transformshift{75.891260in}{0.585645in}%
\pgfsys@useobject{currentmarker}{}%
\end{pgfscope}%
\begin{pgfscope}%
\pgfsys@transformshift{76.365395in}{0.584411in}%
\pgfsys@useobject{currentmarker}{}%
\end{pgfscope}%
\begin{pgfscope}%
\pgfsys@transformshift{77.044347in}{0.584036in}%
\pgfsys@useobject{currentmarker}{}%
\end{pgfscope}%
\begin{pgfscope}%
\pgfsys@transformshift{77.196610in}{0.583453in}%
\pgfsys@useobject{currentmarker}{}%
\end{pgfscope}%
\begin{pgfscope}%
\pgfsys@transformshift{77.477002in}{0.583220in}%
\pgfsys@useobject{currentmarker}{}%
\end{pgfscope}%
\begin{pgfscope}%
\pgfsys@transformshift{77.713773in}{0.582984in}%
\pgfsys@useobject{currentmarker}{}%
\end{pgfscope}%
\begin{pgfscope}%
\pgfsys@transformshift{80.391093in}{0.582743in}%
\pgfsys@useobject{currentmarker}{}%
\end{pgfscope}%
\begin{pgfscope}%
\pgfsys@transformshift{80.572547in}{0.582560in}%
\pgfsys@useobject{currentmarker}{}%
\end{pgfscope}%
\begin{pgfscope}%
\pgfsys@transformshift{80.839129in}{0.582522in}%
\pgfsys@useobject{currentmarker}{}%
\end{pgfscope}%
\begin{pgfscope}%
\pgfsys@transformshift{81.292905in}{0.582281in}%
\pgfsys@useobject{currentmarker}{}%
\end{pgfscope}%
\begin{pgfscope}%
\pgfsys@transformshift{81.546160in}{0.582001in}%
\pgfsys@useobject{currentmarker}{}%
\end{pgfscope}%
\begin{pgfscope}%
\pgfsys@transformshift{81.819957in}{0.581669in}%
\pgfsys@useobject{currentmarker}{}%
\end{pgfscope}%
\begin{pgfscope}%
\pgfsys@transformshift{83.436786in}{0.581017in}%
\pgfsys@useobject{currentmarker}{}%
\end{pgfscope}%
\begin{pgfscope}%
\pgfsys@transformshift{84.322673in}{0.580784in}%
\pgfsys@useobject{currentmarker}{}%
\end{pgfscope}%
\begin{pgfscope}%
\pgfsys@transformshift{84.426126in}{0.580047in}%
\pgfsys@useobject{currentmarker}{}%
\end{pgfscope}%
\begin{pgfscope}%
\pgfsys@transformshift{84.761016in}{0.579228in}%
\pgfsys@useobject{currentmarker}{}%
\end{pgfscope}%
\begin{pgfscope}%
\pgfsys@transformshift{86.806617in}{0.579090in}%
\pgfsys@useobject{currentmarker}{}%
\end{pgfscope}%
\begin{pgfscope}%
\pgfsys@transformshift{86.806617in}{0.579090in}%
\pgfsys@useobject{currentmarker}{}%
\end{pgfscope}%
\begin{pgfscope}%
\pgfsys@transformshift{86.957300in}{0.578813in}%
\pgfsys@useobject{currentmarker}{}%
\end{pgfscope}%
\begin{pgfscope}%
\pgfsys@transformshift{88.325766in}{0.578362in}%
\pgfsys@useobject{currentmarker}{}%
\end{pgfscope}%
\begin{pgfscope}%
\pgfsys@transformshift{88.521870in}{0.577860in}%
\pgfsys@useobject{currentmarker}{}%
\end{pgfscope}%
\begin{pgfscope}%
\pgfsys@transformshift{91.308507in}{0.577711in}%
\pgfsys@useobject{currentmarker}{}%
\end{pgfscope}%
\begin{pgfscope}%
\pgfsys@transformshift{91.841938in}{0.577258in}%
\pgfsys@useobject{currentmarker}{}%
\end{pgfscope}%
\begin{pgfscope}%
\pgfsys@transformshift{92.858407in}{0.577137in}%
\pgfsys@useobject{currentmarker}{}%
\end{pgfscope}%
\begin{pgfscope}%
\pgfsys@transformshift{93.623279in}{0.576803in}%
\pgfsys@useobject{currentmarker}{}%
\end{pgfscope}%
\begin{pgfscope}%
\pgfsys@transformshift{94.641142in}{0.576675in}%
\pgfsys@useobject{currentmarker}{}%
\end{pgfscope}%
\begin{pgfscope}%
\pgfsys@transformshift{96.882572in}{0.575594in}%
\pgfsys@useobject{currentmarker}{}%
\end{pgfscope}%
\begin{pgfscope}%
\pgfsys@transformshift{99.808438in}{0.574944in}%
\pgfsys@useobject{currentmarker}{}%
\end{pgfscope}%
\end{pgfscope}%
\begin{pgfscope}%
\pgfpathrectangle{\pgfqpoint{4.244096in}{3.200903in}}{\pgfqpoint{3.342550in}{2.378919in}}%
\pgfusepath{clip}%
\pgfsetrectcap%
\pgfsetroundjoin%
\pgfsetlinewidth{1.505625pt}%
\definecolor{currentstroke}{rgb}{0.501961,0.501961,0.501961}%
\pgfsetstrokecolor{currentstroke}%
\pgfsetstrokeopacity{0.500000}%
\pgfsetdash{}{0pt}%
\pgfpathmoveto{\pgfqpoint{0.000000in}{0.000000in}}%
\pgfusepath{stroke}%
\end{pgfscope}%
\begin{pgfscope}%
\pgfsetrectcap%
\pgfsetmiterjoin%
\pgfsetlinewidth{0.803000pt}%
\definecolor{currentstroke}{rgb}{0.000000,0.000000,0.000000}%
\pgfsetstrokecolor{currentstroke}%
\pgfsetdash{}{0pt}%
\pgfpathmoveto{\pgfqpoint{4.244096in}{3.200903in}}%
\pgfpathlineto{\pgfqpoint{4.244096in}{5.579823in}}%
\pgfusepath{stroke}%
\end{pgfscope}%
\begin{pgfscope}%
\pgfsetrectcap%
\pgfsetmiterjoin%
\pgfsetlinewidth{0.803000pt}%
\definecolor{currentstroke}{rgb}{0.000000,0.000000,0.000000}%
\pgfsetstrokecolor{currentstroke}%
\pgfsetdash{}{0pt}%
\pgfpathmoveto{\pgfqpoint{7.586646in}{3.200903in}}%
\pgfpathlineto{\pgfqpoint{7.586646in}{5.579823in}}%
\pgfusepath{stroke}%
\end{pgfscope}%
\begin{pgfscope}%
\pgfsetrectcap%
\pgfsetmiterjoin%
\pgfsetlinewidth{0.803000pt}%
\definecolor{currentstroke}{rgb}{0.000000,0.000000,0.000000}%
\pgfsetstrokecolor{currentstroke}%
\pgfsetdash{}{0pt}%
\pgfpathmoveto{\pgfqpoint{4.244096in}{3.200903in}}%
\pgfpathlineto{\pgfqpoint{7.586646in}{3.200903in}}%
\pgfusepath{stroke}%
\end{pgfscope}%
\begin{pgfscope}%
\pgfsetrectcap%
\pgfsetmiterjoin%
\pgfsetlinewidth{0.803000pt}%
\definecolor{currentstroke}{rgb}{0.000000,0.000000,0.000000}%
\pgfsetstrokecolor{currentstroke}%
\pgfsetdash{}{0pt}%
\pgfpathmoveto{\pgfqpoint{4.244096in}{5.579823in}}%
\pgfpathlineto{\pgfqpoint{7.586646in}{5.579823in}}%
\pgfusepath{stroke}%
\end{pgfscope}%
\begin{pgfscope}%
\pgfsetbuttcap%
\pgfsetmiterjoin%
\definecolor{currentfill}{rgb}{1.000000,1.000000,1.000000}%
\pgfsetfillcolor{currentfill}%
\pgfsetlinewidth{1.003750pt}%
\definecolor{currentstroke}{rgb}{0.800000,0.800000,0.800000}%
\pgfsetstrokecolor{currentstroke}%
\pgfsetdash{}{0pt}%
\pgfpathmoveto{\pgfqpoint{4.239573in}{0.781249in}}%
\pgfpathlineto{\pgfqpoint{7.644444in}{0.781249in}}%
\pgfpathquadraticcurveto{\pgfqpoint{7.688889in}{0.781249in}}{\pgfqpoint{7.688889in}{0.825694in}}%
\pgfpathlineto{\pgfqpoint{7.688889in}{2.504473in}}%
\pgfpathquadraticcurveto{\pgfqpoint{7.688889in}{2.548917in}}{\pgfqpoint{7.644444in}{2.548917in}}%
\pgfpathlineto{\pgfqpoint{4.239573in}{2.548917in}}%
\pgfpathquadraticcurveto{\pgfqpoint{4.195129in}{2.548917in}}{\pgfqpoint{4.195129in}{2.504473in}}%
\pgfpathlineto{\pgfqpoint{4.195129in}{0.825694in}}%
\pgfpathquadraticcurveto{\pgfqpoint{4.195129in}{0.781249in}}{\pgfqpoint{4.239573in}{0.781249in}}%
\pgfpathlineto{\pgfqpoint{4.239573in}{0.781249in}}%
\pgfpathclose%
\pgfusepath{stroke,fill}%
\end{pgfscope}%
\begin{pgfscope}%
\pgfsetrectcap%
\pgfsetroundjoin%
\pgfsetlinewidth{1.505625pt}%
\definecolor{currentstroke}{rgb}{0.000000,0.000000,0.000000}%
\pgfsetstrokecolor{currentstroke}%
\pgfsetdash{}{0pt}%
\pgfpathmoveto{\pgfqpoint{4.284018in}{2.371140in}}%
\pgfpathlineto{\pgfqpoint{4.506240in}{2.371140in}}%
\pgfpathlineto{\pgfqpoint{4.728462in}{2.371140in}}%
\pgfusepath{stroke}%
\end{pgfscope}%
\begin{pgfscope}%
\pgfsetbuttcap%
\pgfsetroundjoin%
\definecolor{currentfill}{rgb}{0.000000,0.000000,0.000000}%
\pgfsetfillcolor{currentfill}%
\pgfsetlinewidth{1.003750pt}%
\definecolor{currentstroke}{rgb}{0.000000,0.000000,0.000000}%
\pgfsetstrokecolor{currentstroke}%
\pgfsetdash{}{0pt}%
\pgfsys@defobject{currentmarker}{\pgfqpoint{-0.006944in}{-0.006944in}}{\pgfqpoint{0.006944in}{0.006944in}}{%
\pgfpathmoveto{\pgfqpoint{0.000000in}{-0.006944in}}%
\pgfpathcurveto{\pgfqpoint{0.001842in}{-0.006944in}}{\pgfqpoint{0.003608in}{-0.006213in}}{\pgfqpoint{0.004910in}{-0.004910in}}%
\pgfpathcurveto{\pgfqpoint{0.006213in}{-0.003608in}}{\pgfqpoint{0.006944in}{-0.001842in}}{\pgfqpoint{0.006944in}{0.000000in}}%
\pgfpathcurveto{\pgfqpoint{0.006944in}{0.001842in}}{\pgfqpoint{0.006213in}{0.003608in}}{\pgfqpoint{0.004910in}{0.004910in}}%
\pgfpathcurveto{\pgfqpoint{0.003608in}{0.006213in}}{\pgfqpoint{0.001842in}{0.006944in}}{\pgfqpoint{0.000000in}{0.006944in}}%
\pgfpathcurveto{\pgfqpoint{-0.001842in}{0.006944in}}{\pgfqpoint{-0.003608in}{0.006213in}}{\pgfqpoint{-0.004910in}{0.004910in}}%
\pgfpathcurveto{\pgfqpoint{-0.006213in}{0.003608in}}{\pgfqpoint{-0.006944in}{0.001842in}}{\pgfqpoint{-0.006944in}{0.000000in}}%
\pgfpathcurveto{\pgfqpoint{-0.006944in}{-0.001842in}}{\pgfqpoint{-0.006213in}{-0.003608in}}{\pgfqpoint{-0.004910in}{-0.004910in}}%
\pgfpathcurveto{\pgfqpoint{-0.003608in}{-0.006213in}}{\pgfqpoint{-0.001842in}{-0.006944in}}{\pgfqpoint{0.000000in}{-0.006944in}}%
\pgfpathlineto{\pgfqpoint{0.000000in}{-0.006944in}}%
\pgfpathclose%
\pgfusepath{stroke,fill}%
}%
\begin{pgfscope}%
\pgfsys@transformshift{4.506240in}{2.371140in}%
\pgfsys@useobject{currentmarker}{}%
\end{pgfscope}%
\end{pgfscope}%
\begin{pgfscope}%
\definecolor{textcolor}{rgb}{0.000000,0.000000,0.000000}%
\pgfsetstrokecolor{textcolor}%
\pgfsetfillcolor{textcolor}%
\pgftext[x=4.906240in,y=2.293362in,left,base]{\color{textcolor}{\rmfamily\fontsize{16.000000}{19.200000}\selectfont\catcode`\^=\active\def^{\ifmmode\sp\else\^{}\fi}\catcode`\%=\active\def%{\%}osier}}%
\end{pgfscope}%
\begin{pgfscope}%
\pgfsetrectcap%
\pgfsetroundjoin%
\pgfsetlinewidth{1.505625pt}%
\definecolor{currentstroke}{rgb}{0.501961,0.501961,0.501961}%
\pgfsetstrokecolor{currentstroke}%
\pgfsetstrokeopacity{0.500000}%
\pgfsetdash{}{0pt}%
\pgfpathmoveto{\pgfqpoint{4.284018in}{2.033563in}}%
\pgfpathlineto{\pgfqpoint{4.506240in}{2.033563in}}%
\pgfpathlineto{\pgfqpoint{4.728462in}{2.033563in}}%
\pgfusepath{stroke}%
\end{pgfscope}%
\begin{pgfscope}%
\definecolor{textcolor}{rgb}{0.000000,0.000000,0.000000}%
\pgfsetstrokecolor{textcolor}%
\pgfsetfillcolor{textcolor}%
\pgftext[x=4.906240in,y=1.955785in,left,base]{\color{textcolor}{\rmfamily\fontsize{16.000000}{19.200000}\selectfont\catcode`\^=\active\def^{\ifmmode\sp\else\^{}\fi}\catcode`\%=\active\def%{\%}near-optimal space (osier)}}%
\end{pgfscope}%
\begin{pgfscope}%
\pgfsetbuttcap%
\pgfsetroundjoin%
\pgfsetlinewidth{1.003750pt}%
\definecolor{currentstroke}{rgb}{1.000000,0.000000,0.000000}%
\pgfsetstrokecolor{currentstroke}%
\pgfsetdash{}{0pt}%
\pgfpathmoveto{\pgfqpoint{4.506240in}{1.635458in}}%
\pgfpathcurveto{\pgfqpoint{4.517135in}{1.635458in}}{\pgfqpoint{4.527586in}{1.639786in}}{\pgfqpoint{4.535291in}{1.647491in}}%
\pgfpathcurveto{\pgfqpoint{4.542995in}{1.655195in}}{\pgfqpoint{4.547324in}{1.665646in}}{\pgfqpoint{4.547324in}{1.676541in}}%
\pgfpathcurveto{\pgfqpoint{4.547324in}{1.687437in}}{\pgfqpoint{4.542995in}{1.697888in}}{\pgfqpoint{4.535291in}{1.705592in}}%
\pgfpathcurveto{\pgfqpoint{4.527586in}{1.713296in}}{\pgfqpoint{4.517135in}{1.717625in}}{\pgfqpoint{4.506240in}{1.717625in}}%
\pgfpathcurveto{\pgfqpoint{4.495344in}{1.717625in}}{\pgfqpoint{4.484894in}{1.713296in}}{\pgfqpoint{4.477189in}{1.705592in}}%
\pgfpathcurveto{\pgfqpoint{4.469485in}{1.697888in}}{\pgfqpoint{4.465156in}{1.687437in}}{\pgfqpoint{4.465156in}{1.676541in}}%
\pgfpathcurveto{\pgfqpoint{4.465156in}{1.665646in}}{\pgfqpoint{4.469485in}{1.655195in}}{\pgfqpoint{4.477189in}{1.647491in}}%
\pgfpathcurveto{\pgfqpoint{4.484894in}{1.639786in}}{\pgfqpoint{4.495344in}{1.635458in}}{\pgfqpoint{4.506240in}{1.635458in}}%
\pgfpathlineto{\pgfqpoint{4.506240in}{1.635458in}}%
\pgfpathclose%
\pgfusepath{stroke}%
\end{pgfscope}%
\begin{pgfscope}%
\definecolor{textcolor}{rgb}{0.000000,0.000000,0.000000}%
\pgfsetstrokecolor{textcolor}%
\pgfsetfillcolor{textcolor}%
\pgftext[x=4.906240in,y=1.618208in,left,base]{\color{textcolor}{\rmfamily\fontsize{16.000000}{19.200000}\selectfont\catcode`\^=\active\def^{\ifmmode\sp\else\^{}\fi}\catcode`\%=\active\def%{\%}temoa+mga}}%
\end{pgfscope}%
\begin{pgfscope}%
\pgfsetbuttcap%
\pgfsetroundjoin%
\pgfsetlinewidth{1.003750pt}%
\definecolor{currentstroke}{rgb}{0.000000,0.000000,0.000000}%
\pgfsetstrokecolor{currentstroke}%
\pgfsetdash{}{0pt}%
\pgfpathmoveto{\pgfqpoint{4.506240in}{1.297881in}}%
\pgfpathcurveto{\pgfqpoint{4.517135in}{1.297881in}}{\pgfqpoint{4.527586in}{1.302210in}}{\pgfqpoint{4.535291in}{1.309914in}}%
\pgfpathcurveto{\pgfqpoint{4.542995in}{1.317618in}}{\pgfqpoint{4.547324in}{1.328069in}}{\pgfqpoint{4.547324in}{1.338965in}}%
\pgfpathcurveto{\pgfqpoint{4.547324in}{1.349860in}}{\pgfqpoint{4.542995in}{1.360311in}}{\pgfqpoint{4.535291in}{1.368015in}}%
\pgfpathcurveto{\pgfqpoint{4.527586in}{1.375720in}}{\pgfqpoint{4.517135in}{1.380048in}}{\pgfqpoint{4.506240in}{1.380048in}}%
\pgfpathcurveto{\pgfqpoint{4.495344in}{1.380048in}}{\pgfqpoint{4.484894in}{1.375720in}}{\pgfqpoint{4.477189in}{1.368015in}}%
\pgfpathcurveto{\pgfqpoint{4.469485in}{1.360311in}}{\pgfqpoint{4.465156in}{1.349860in}}{\pgfqpoint{4.465156in}{1.338965in}}%
\pgfpathcurveto{\pgfqpoint{4.465156in}{1.328069in}}{\pgfqpoint{4.469485in}{1.317618in}}{\pgfqpoint{4.477189in}{1.309914in}}%
\pgfpathcurveto{\pgfqpoint{4.484894in}{1.302210in}}{\pgfqpoint{4.495344in}{1.297881in}}{\pgfqpoint{4.506240in}{1.297881in}}%
\pgfpathlineto{\pgfqpoint{4.506240in}{1.297881in}}%
\pgfpathclose%
\pgfusepath{stroke}%
\end{pgfscope}%
\begin{pgfscope}%
\definecolor{textcolor}{rgb}{0.000000,0.000000,0.000000}%
\pgfsetstrokecolor{textcolor}%
\pgfsetfillcolor{textcolor}%
\pgftext[x=4.906240in,y=1.280631in,left,base]{\color{textcolor}{\rmfamily\fontsize{16.000000}{19.200000}\selectfont\catcode`\^=\active\def^{\ifmmode\sp\else\^{}\fi}\catcode`\%=\active\def%{\%}osier (tested points)}}%
\end{pgfscope}%
\begin{pgfscope}%
\pgfsetbuttcap%
\pgfsetmiterjoin%
\definecolor{currentfill}{rgb}{0.839216,0.152941,0.156863}%
\pgfsetfillcolor{currentfill}%
\pgfsetfillopacity{0.200000}%
\pgfsetlinewidth{1.003750pt}%
\definecolor{currentstroke}{rgb}{0.839216,0.152941,0.156863}%
\pgfsetstrokecolor{currentstroke}%
\pgfsetstrokeopacity{0.200000}%
\pgfsetdash{}{0pt}%
\pgfpathmoveto{\pgfqpoint{4.284018in}{0.929937in}}%
\pgfpathlineto{\pgfqpoint{4.728462in}{0.929937in}}%
\pgfpathlineto{\pgfqpoint{4.728462in}{1.085493in}}%
\pgfpathlineto{\pgfqpoint{4.284018in}{1.085493in}}%
\pgfpathlineto{\pgfqpoint{4.284018in}{0.929937in}}%
\pgfpathclose%
\pgfusepath{stroke,fill}%
\end{pgfscope}%
\begin{pgfscope}%
\pgfsetbuttcap%
\pgfsetmiterjoin%
\definecolor{currentfill}{rgb}{0.839216,0.152941,0.156863}%
\pgfsetfillcolor{currentfill}%
\pgfsetfillopacity{0.200000}%
\pgfsetlinewidth{1.003750pt}%
\definecolor{currentstroke}{rgb}{0.839216,0.152941,0.156863}%
\pgfsetstrokecolor{currentstroke}%
\pgfsetstrokeopacity{0.200000}%
\pgfsetdash{}{0pt}%
\pgfpathmoveto{\pgfqpoint{4.284018in}{0.929937in}}%
\pgfpathlineto{\pgfqpoint{4.728462in}{0.929937in}}%
\pgfpathlineto{\pgfqpoint{4.728462in}{1.085493in}}%
\pgfpathlineto{\pgfqpoint{4.284018in}{1.085493in}}%
\pgfpathlineto{\pgfqpoint{4.284018in}{0.929937in}}%
\pgfpathclose%
\pgfusepath{clip}%
\pgfsys@defobject{currentpattern}{\pgfqpoint{0in}{0in}}{\pgfqpoint{1in}{1in}}{%
\begin{pgfscope}%
\pgfpathrectangle{\pgfqpoint{0in}{0in}}{\pgfqpoint{1in}{1in}}%
\pgfusepath{clip}%
\pgfpathmoveto{\pgfqpoint{-0.500000in}{0.500000in}}%
\pgfpathlineto{\pgfqpoint{0.500000in}{1.500000in}}%
\pgfpathmoveto{\pgfqpoint{-0.333333in}{0.333333in}}%
\pgfpathlineto{\pgfqpoint{0.666667in}{1.333333in}}%
\pgfpathmoveto{\pgfqpoint{-0.166667in}{0.166667in}}%
\pgfpathlineto{\pgfqpoint{0.833333in}{1.166667in}}%
\pgfpathmoveto{\pgfqpoint{0.000000in}{0.000000in}}%
\pgfpathlineto{\pgfqpoint{1.000000in}{1.000000in}}%
\pgfpathmoveto{\pgfqpoint{0.166667in}{-0.166667in}}%
\pgfpathlineto{\pgfqpoint{1.166667in}{0.833333in}}%
\pgfpathmoveto{\pgfqpoint{0.333333in}{-0.333333in}}%
\pgfpathlineto{\pgfqpoint{1.333333in}{0.666667in}}%
\pgfpathmoveto{\pgfqpoint{0.500000in}{-0.500000in}}%
\pgfpathlineto{\pgfqpoint{1.500000in}{0.500000in}}%
\pgfusepath{stroke}%
\end{pgfscope}%
}%
\pgfsys@transformshift{4.284018in}{0.929937in}%
\pgfsys@useobject{currentpattern}{}%
\pgfsys@transformshift{1in}{0in}%
\pgfsys@transformshift{-1in}{0in}%
\pgfsys@transformshift{0in}{1in}%
\end{pgfscope}%
\begin{pgfscope}%
\definecolor{textcolor}{rgb}{0.000000,0.000000,0.000000}%
\pgfsetstrokecolor{textcolor}%
\pgfsetfillcolor{textcolor}%
\pgftext[x=4.906240in,y=0.929937in,left,base]{\color{textcolor}{\rmfamily\fontsize{16.000000}{19.200000}\selectfont\catcode`\^=\active\def^{\ifmmode\sp\else\^{}\fi}\catcode`\%=\active\def%{\%}near-optimal space (Temoa)}}%
\end{pgfscope}%
\end{pgfpicture}%
\makeatother%
\endgroup%
}
        \caption{Comparison between the near-optimal objective spaces for \gls{osier} and
        \gls{temoa} for two objectives. \gls{osier}'s least-cost solution is
        within 0.5\% of \gls{temoa}'s solution.}
        \label{fig:osier-temoa-benchmark}
    \end{center}
\end{figure}

The red-shaded area shows the sub-optimal space sampled with \gls{temoa}'s
\gls{mga} algorithm, while the gray area shows \gls{osier}'s sub-optimal space.
None of \gls{temoa}'s \gls{mga} solutions overlap with \gls{osier}'s sub-optimal
space, which demonstrates that traditional \gls{mga} does not guarantee improved
results along any other objective axis. 

Figure \ref{fig:osier-near-optimal} zooms in on \gls{osier}'s near-optimal
space, showing all of the solutions therein in green. Points in red correspond to
a subset of 25 solutions selected via farthest-first-traversal.

\begin{figure}[ht!]
    \begin{center}
        \resizebox{\columnwidth}{!}{%% Creator: Matplotlib, PGF backend
%%
%% To include the figure in your LaTeX document, write
%%   \input{<filename>.pgf}
%%
%% Make sure the required packages are loaded in your preamble
%%   \usepackage{pgf}
%%
%% Also ensure that all the required font packages are loaded; for instance,
%% the lmodern package is sometimes necessary when using math font.
%%   \usepackage{lmodern}
%%
%% Figures using additional raster images can only be included by \input if
%% they are in the same directory as the main LaTeX file. For loading figures
%% from other directories you can use the `import` package
%%   \usepackage{import}
%%
%% and then include the figures with
%%   \import{<path to file>}{<filename>.pgf}
%%
%% Matplotlib used the following preamble
%%   \def\mathdefault#1{#1}
%%   \everymath=\expandafter{\the\everymath\displaystyle}
%%   \IfFileExists{scrextend.sty}{
%%     \usepackage[fontsize=10.000000pt]{scrextend}
%%   }{
%%     \renewcommand{\normalsize}{\fontsize{10.000000}{12.000000}\selectfont}
%%     \normalsize
%%   }
%%   
%%   \makeatletter\@ifpackageloaded{underscore}{}{\usepackage[strings]{underscore}}\makeatother
%%
\begingroup%
\makeatletter%
\begin{pgfpicture}%
\pgfpathrectangle{\pgfpointorigin}{\pgfqpoint{7.900000in}{5.900000in}}%
\pgfusepath{use as bounding box, clip}%
\begin{pgfscope}%
\pgfsetbuttcap%
\pgfsetmiterjoin%
\definecolor{currentfill}{rgb}{1.000000,1.000000,1.000000}%
\pgfsetfillcolor{currentfill}%
\pgfsetlinewidth{0.000000pt}%
\definecolor{currentstroke}{rgb}{0.000000,0.000000,0.000000}%
\pgfsetstrokecolor{currentstroke}%
\pgfsetdash{}{0pt}%
\pgfpathmoveto{\pgfqpoint{0.000000in}{0.000000in}}%
\pgfpathlineto{\pgfqpoint{7.900000in}{0.000000in}}%
\pgfpathlineto{\pgfqpoint{7.900000in}{5.900000in}}%
\pgfpathlineto{\pgfqpoint{0.000000in}{5.900000in}}%
\pgfpathlineto{\pgfqpoint{0.000000in}{0.000000in}}%
\pgfpathclose%
\pgfusepath{fill}%
\end{pgfscope}%
\begin{pgfscope}%
\pgfsetbuttcap%
\pgfsetmiterjoin%
\definecolor{currentfill}{rgb}{1.000000,1.000000,1.000000}%
\pgfsetfillcolor{currentfill}%
\pgfsetlinewidth{0.000000pt}%
\definecolor{currentstroke}{rgb}{0.000000,0.000000,0.000000}%
\pgfsetstrokecolor{currentstroke}%
\pgfsetstrokeopacity{0.000000}%
\pgfsetdash{}{0pt}%
\pgfpathmoveto{\pgfqpoint{0.688192in}{0.670138in}}%
\pgfpathlineto{\pgfqpoint{7.800000in}{0.670138in}}%
\pgfpathlineto{\pgfqpoint{7.800000in}{5.800000in}}%
\pgfpathlineto{\pgfqpoint{0.688192in}{5.800000in}}%
\pgfpathlineto{\pgfqpoint{0.688192in}{0.670138in}}%
\pgfpathclose%
\pgfusepath{fill}%
\end{pgfscope}%
\begin{pgfscope}%
\pgfpathrectangle{\pgfqpoint{0.688192in}{0.670138in}}{\pgfqpoint{7.111808in}{5.129862in}}%
\pgfusepath{clip}%
\pgfsetbuttcap%
\pgfsetroundjoin%
\pgfsetlinewidth{1.003750pt}%
\definecolor{currentstroke}{rgb}{0.000000,0.000000,0.000000}%
\pgfsetstrokecolor{currentstroke}%
\pgfsetdash{}{0pt}%
\pgfpathmoveto{\pgfqpoint{5.329102in}{5.309073in}}%
\pgfpathcurveto{\pgfqpoint{5.339998in}{5.309073in}}{\pgfqpoint{5.350449in}{5.313402in}}{\pgfqpoint{5.358153in}{5.321106in}}%
\pgfpathcurveto{\pgfqpoint{5.365857in}{5.328811in}}{\pgfqpoint{5.370186in}{5.339261in}}{\pgfqpoint{5.370186in}{5.350157in}}%
\pgfpathcurveto{\pgfqpoint{5.370186in}{5.361052in}}{\pgfqpoint{5.365857in}{5.371503in}}{\pgfqpoint{5.358153in}{5.379208in}}%
\pgfpathcurveto{\pgfqpoint{5.350449in}{5.386912in}}{\pgfqpoint{5.339998in}{5.391241in}}{\pgfqpoint{5.329102in}{5.391241in}}%
\pgfpathcurveto{\pgfqpoint{5.318207in}{5.391241in}}{\pgfqpoint{5.307756in}{5.386912in}}{\pgfqpoint{5.300052in}{5.379208in}}%
\pgfpathcurveto{\pgfqpoint{5.292347in}{5.371503in}}{\pgfqpoint{5.288018in}{5.361052in}}{\pgfqpoint{5.288018in}{5.350157in}}%
\pgfpathcurveto{\pgfqpoint{5.288018in}{5.339261in}}{\pgfqpoint{5.292347in}{5.328811in}}{\pgfqpoint{5.300052in}{5.321106in}}%
\pgfpathcurveto{\pgfqpoint{5.307756in}{5.313402in}}{\pgfqpoint{5.318207in}{5.309073in}}{\pgfqpoint{5.329102in}{5.309073in}}%
\pgfpathlineto{\pgfqpoint{5.329102in}{5.309073in}}%
\pgfpathclose%
\pgfusepath{stroke}%
\end{pgfscope}%
\begin{pgfscope}%
\pgfpathrectangle{\pgfqpoint{0.688192in}{0.670138in}}{\pgfqpoint{7.111808in}{5.129862in}}%
\pgfusepath{clip}%
\pgfsetbuttcap%
\pgfsetroundjoin%
\pgfsetlinewidth{1.003750pt}%
\definecolor{currentstroke}{rgb}{0.000000,0.000000,0.000000}%
\pgfsetstrokecolor{currentstroke}%
\pgfsetdash{}{0pt}%
\pgfpathmoveto{\pgfqpoint{1.840914in}{0.696069in}}%
\pgfpathcurveto{\pgfqpoint{1.851809in}{0.696069in}}{\pgfqpoint{1.862260in}{0.700398in}}{\pgfqpoint{1.869964in}{0.708103in}}%
\pgfpathcurveto{\pgfqpoint{1.877669in}{0.715807in}}{\pgfqpoint{1.881997in}{0.726258in}}{\pgfqpoint{1.881997in}{0.737153in}}%
\pgfpathcurveto{\pgfqpoint{1.881997in}{0.748049in}}{\pgfqpoint{1.877669in}{0.758500in}}{\pgfqpoint{1.869964in}{0.766204in}}%
\pgfpathcurveto{\pgfqpoint{1.862260in}{0.773908in}}{\pgfqpoint{1.851809in}{0.778237in}}{\pgfqpoint{1.840914in}{0.778237in}}%
\pgfpathcurveto{\pgfqpoint{1.830018in}{0.778237in}}{\pgfqpoint{1.819567in}{0.773908in}}{\pgfqpoint{1.811863in}{0.766204in}}%
\pgfpathcurveto{\pgfqpoint{1.804159in}{0.758500in}}{\pgfqpoint{1.799830in}{0.748049in}}{\pgfqpoint{1.799830in}{0.737153in}}%
\pgfpathcurveto{\pgfqpoint{1.799830in}{0.726258in}}{\pgfqpoint{1.804159in}{0.715807in}}{\pgfqpoint{1.811863in}{0.708103in}}%
\pgfpathcurveto{\pgfqpoint{1.819567in}{0.700398in}}{\pgfqpoint{1.830018in}{0.696069in}}{\pgfqpoint{1.840914in}{0.696069in}}%
\pgfpathlineto{\pgfqpoint{1.840914in}{0.696069in}}%
\pgfpathclose%
\pgfusepath{stroke}%
\end{pgfscope}%
\begin{pgfscope}%
\pgfpathrectangle{\pgfqpoint{0.688192in}{0.670138in}}{\pgfqpoint{7.111808in}{5.129862in}}%
\pgfusepath{clip}%
\pgfsetbuttcap%
\pgfsetroundjoin%
\pgfsetlinewidth{1.003750pt}%
\definecolor{currentstroke}{rgb}{0.000000,0.000000,0.000000}%
\pgfsetstrokecolor{currentstroke}%
\pgfsetdash{}{0pt}%
\pgfpathmoveto{\pgfqpoint{1.101880in}{0.720677in}}%
\pgfpathcurveto{\pgfqpoint{1.112775in}{0.720677in}}{\pgfqpoint{1.123226in}{0.725006in}}{\pgfqpoint{1.130930in}{0.732710in}}%
\pgfpathcurveto{\pgfqpoint{1.138635in}{0.740415in}}{\pgfqpoint{1.142964in}{0.750865in}}{\pgfqpoint{1.142964in}{0.761761in}}%
\pgfpathcurveto{\pgfqpoint{1.142964in}{0.772656in}}{\pgfqpoint{1.138635in}{0.783107in}}{\pgfqpoint{1.130930in}{0.790812in}}%
\pgfpathcurveto{\pgfqpoint{1.123226in}{0.798516in}}{\pgfqpoint{1.112775in}{0.802845in}}{\pgfqpoint{1.101880in}{0.802845in}}%
\pgfpathcurveto{\pgfqpoint{1.090984in}{0.802845in}}{\pgfqpoint{1.080533in}{0.798516in}}{\pgfqpoint{1.072829in}{0.790812in}}%
\pgfpathcurveto{\pgfqpoint{1.065125in}{0.783107in}}{\pgfqpoint{1.060796in}{0.772656in}}{\pgfqpoint{1.060796in}{0.761761in}}%
\pgfpathcurveto{\pgfqpoint{1.060796in}{0.750865in}}{\pgfqpoint{1.065125in}{0.740415in}}{\pgfqpoint{1.072829in}{0.732710in}}%
\pgfpathcurveto{\pgfqpoint{1.080533in}{0.725006in}}{\pgfqpoint{1.090984in}{0.720677in}}{\pgfqpoint{1.101880in}{0.720677in}}%
\pgfpathlineto{\pgfqpoint{1.101880in}{0.720677in}}%
\pgfpathclose%
\pgfusepath{stroke}%
\end{pgfscope}%
\begin{pgfscope}%
\pgfpathrectangle{\pgfqpoint{0.688192in}{0.670138in}}{\pgfqpoint{7.111808in}{5.129862in}}%
\pgfusepath{clip}%
\pgfsetbuttcap%
\pgfsetroundjoin%
\pgfsetlinewidth{1.003750pt}%
\definecolor{currentstroke}{rgb}{0.000000,0.000000,0.000000}%
\pgfsetstrokecolor{currentstroke}%
\pgfsetdash{}{0pt}%
\pgfpathmoveto{\pgfqpoint{1.840914in}{0.696069in}}%
\pgfpathcurveto{\pgfqpoint{1.851809in}{0.696069in}}{\pgfqpoint{1.862260in}{0.700398in}}{\pgfqpoint{1.869964in}{0.708103in}}%
\pgfpathcurveto{\pgfqpoint{1.877669in}{0.715807in}}{\pgfqpoint{1.881997in}{0.726258in}}{\pgfqpoint{1.881997in}{0.737153in}}%
\pgfpathcurveto{\pgfqpoint{1.881997in}{0.748049in}}{\pgfqpoint{1.877669in}{0.758500in}}{\pgfqpoint{1.869964in}{0.766204in}}%
\pgfpathcurveto{\pgfqpoint{1.862260in}{0.773908in}}{\pgfqpoint{1.851809in}{0.778237in}}{\pgfqpoint{1.840914in}{0.778237in}}%
\pgfpathcurveto{\pgfqpoint{1.830018in}{0.778237in}}{\pgfqpoint{1.819567in}{0.773908in}}{\pgfqpoint{1.811863in}{0.766204in}}%
\pgfpathcurveto{\pgfqpoint{1.804159in}{0.758500in}}{\pgfqpoint{1.799830in}{0.748049in}}{\pgfqpoint{1.799830in}{0.737153in}}%
\pgfpathcurveto{\pgfqpoint{1.799830in}{0.726258in}}{\pgfqpoint{1.804159in}{0.715807in}}{\pgfqpoint{1.811863in}{0.708103in}}%
\pgfpathcurveto{\pgfqpoint{1.819567in}{0.700398in}}{\pgfqpoint{1.830018in}{0.696069in}}{\pgfqpoint{1.840914in}{0.696069in}}%
\pgfpathlineto{\pgfqpoint{1.840914in}{0.696069in}}%
\pgfpathclose%
\pgfusepath{stroke}%
\end{pgfscope}%
\begin{pgfscope}%
\pgfpathrectangle{\pgfqpoint{0.688192in}{0.670138in}}{\pgfqpoint{7.111808in}{5.129862in}}%
\pgfusepath{clip}%
\pgfsetbuttcap%
\pgfsetroundjoin%
\pgfsetlinewidth{1.003750pt}%
\definecolor{currentstroke}{rgb}{0.000000,0.000000,0.000000}%
\pgfsetstrokecolor{currentstroke}%
\pgfsetdash{}{0pt}%
\pgfpathmoveto{\pgfqpoint{2.869035in}{0.669385in}}%
\pgfpathcurveto{\pgfqpoint{2.879931in}{0.669385in}}{\pgfqpoint{2.890382in}{0.673714in}}{\pgfqpoint{2.898086in}{0.681419in}}%
\pgfpathcurveto{\pgfqpoint{2.905790in}{0.689123in}}{\pgfqpoint{2.910119in}{0.699574in}}{\pgfqpoint{2.910119in}{0.710469in}}%
\pgfpathcurveto{\pgfqpoint{2.910119in}{0.721365in}}{\pgfqpoint{2.905790in}{0.731816in}}{\pgfqpoint{2.898086in}{0.739520in}}%
\pgfpathcurveto{\pgfqpoint{2.890382in}{0.747224in}}{\pgfqpoint{2.879931in}{0.751553in}}{\pgfqpoint{2.869035in}{0.751553in}}%
\pgfpathcurveto{\pgfqpoint{2.858140in}{0.751553in}}{\pgfqpoint{2.847689in}{0.747224in}}{\pgfqpoint{2.839985in}{0.739520in}}%
\pgfpathcurveto{\pgfqpoint{2.832280in}{0.731816in}}{\pgfqpoint{2.827952in}{0.721365in}}{\pgfqpoint{2.827952in}{0.710469in}}%
\pgfpathcurveto{\pgfqpoint{2.827952in}{0.699574in}}{\pgfqpoint{2.832280in}{0.689123in}}{\pgfqpoint{2.839985in}{0.681419in}}%
\pgfpathcurveto{\pgfqpoint{2.847689in}{0.673714in}}{\pgfqpoint{2.858140in}{0.669385in}}{\pgfqpoint{2.869035in}{0.669385in}}%
\pgfpathlineto{\pgfqpoint{2.869035in}{0.669385in}}%
\pgfpathclose%
\pgfusepath{stroke}%
\end{pgfscope}%
\begin{pgfscope}%
\pgfpathrectangle{\pgfqpoint{0.688192in}{0.670138in}}{\pgfqpoint{7.111808in}{5.129862in}}%
\pgfusepath{clip}%
\pgfsetbuttcap%
\pgfsetroundjoin%
\pgfsetlinewidth{1.003750pt}%
\definecolor{currentstroke}{rgb}{0.000000,0.000000,0.000000}%
\pgfsetstrokecolor{currentstroke}%
\pgfsetdash{}{0pt}%
\pgfpathmoveto{\pgfqpoint{0.789763in}{1.038554in}}%
\pgfpathcurveto{\pgfqpoint{0.800659in}{1.038554in}}{\pgfqpoint{0.811109in}{1.042883in}}{\pgfqpoint{0.818814in}{1.050587in}}%
\pgfpathcurveto{\pgfqpoint{0.826518in}{1.058291in}}{\pgfqpoint{0.830847in}{1.068742in}}{\pgfqpoint{0.830847in}{1.079638in}}%
\pgfpathcurveto{\pgfqpoint{0.830847in}{1.090533in}}{\pgfqpoint{0.826518in}{1.100984in}}{\pgfqpoint{0.818814in}{1.108688in}}%
\pgfpathcurveto{\pgfqpoint{0.811109in}{1.116393in}}{\pgfqpoint{0.800659in}{1.120722in}}{\pgfqpoint{0.789763in}{1.120722in}}%
\pgfpathcurveto{\pgfqpoint{0.778868in}{1.120722in}}{\pgfqpoint{0.768417in}{1.116393in}}{\pgfqpoint{0.760712in}{1.108688in}}%
\pgfpathcurveto{\pgfqpoint{0.753008in}{1.100984in}}{\pgfqpoint{0.748679in}{1.090533in}}{\pgfqpoint{0.748679in}{1.079638in}}%
\pgfpathcurveto{\pgfqpoint{0.748679in}{1.068742in}}{\pgfqpoint{0.753008in}{1.058291in}}{\pgfqpoint{0.760712in}{1.050587in}}%
\pgfpathcurveto{\pgfqpoint{0.768417in}{1.042883in}}{\pgfqpoint{0.778868in}{1.038554in}}{\pgfqpoint{0.789763in}{1.038554in}}%
\pgfpathlineto{\pgfqpoint{0.789763in}{1.038554in}}%
\pgfpathclose%
\pgfusepath{stroke}%
\end{pgfscope}%
\begin{pgfscope}%
\pgfpathrectangle{\pgfqpoint{0.688192in}{0.670138in}}{\pgfqpoint{7.111808in}{5.129862in}}%
\pgfusepath{clip}%
\pgfsetbuttcap%
\pgfsetroundjoin%
\pgfsetlinewidth{1.003750pt}%
\definecolor{currentstroke}{rgb}{0.000000,0.000000,0.000000}%
\pgfsetstrokecolor{currentstroke}%
\pgfsetdash{}{0pt}%
\pgfpathmoveto{\pgfqpoint{1.679564in}{0.702362in}}%
\pgfpathcurveto{\pgfqpoint{1.690460in}{0.702362in}}{\pgfqpoint{1.700911in}{0.706691in}}{\pgfqpoint{1.708615in}{0.714395in}}%
\pgfpathcurveto{\pgfqpoint{1.716319in}{0.722099in}}{\pgfqpoint{1.720648in}{0.732550in}}{\pgfqpoint{1.720648in}{0.743446in}}%
\pgfpathcurveto{\pgfqpoint{1.720648in}{0.754341in}}{\pgfqpoint{1.716319in}{0.764792in}}{\pgfqpoint{1.708615in}{0.772496in}}%
\pgfpathcurveto{\pgfqpoint{1.700911in}{0.780201in}}{\pgfqpoint{1.690460in}{0.784529in}}{\pgfqpoint{1.679564in}{0.784529in}}%
\pgfpathcurveto{\pgfqpoint{1.668669in}{0.784529in}}{\pgfqpoint{1.658218in}{0.780201in}}{\pgfqpoint{1.650514in}{0.772496in}}%
\pgfpathcurveto{\pgfqpoint{1.642809in}{0.764792in}}{\pgfqpoint{1.638480in}{0.754341in}}{\pgfqpoint{1.638480in}{0.743446in}}%
\pgfpathcurveto{\pgfqpoint{1.638480in}{0.732550in}}{\pgfqpoint{1.642809in}{0.722099in}}{\pgfqpoint{1.650514in}{0.714395in}}%
\pgfpathcurveto{\pgfqpoint{1.658218in}{0.706691in}}{\pgfqpoint{1.668669in}{0.702362in}}{\pgfqpoint{1.679564in}{0.702362in}}%
\pgfpathlineto{\pgfqpoint{1.679564in}{0.702362in}}%
\pgfpathclose%
\pgfusepath{stroke}%
\end{pgfscope}%
\begin{pgfscope}%
\pgfpathrectangle{\pgfqpoint{0.688192in}{0.670138in}}{\pgfqpoint{7.111808in}{5.129862in}}%
\pgfusepath{clip}%
\pgfsetbuttcap%
\pgfsetroundjoin%
\pgfsetlinewidth{1.003750pt}%
\definecolor{currentstroke}{rgb}{0.000000,0.000000,0.000000}%
\pgfsetstrokecolor{currentstroke}%
\pgfsetdash{}{0pt}%
\pgfpathmoveto{\pgfqpoint{1.045306in}{0.741536in}}%
\pgfpathcurveto{\pgfqpoint{1.056202in}{0.741536in}}{\pgfqpoint{1.066653in}{0.745865in}}{\pgfqpoint{1.074357in}{0.753569in}}%
\pgfpathcurveto{\pgfqpoint{1.082061in}{0.761274in}}{\pgfqpoint{1.086390in}{0.771725in}}{\pgfqpoint{1.086390in}{0.782620in}}%
\pgfpathcurveto{\pgfqpoint{1.086390in}{0.793516in}}{\pgfqpoint{1.082061in}{0.803967in}}{\pgfqpoint{1.074357in}{0.811671in}}%
\pgfpathcurveto{\pgfqpoint{1.066653in}{0.819375in}}{\pgfqpoint{1.056202in}{0.823704in}}{\pgfqpoint{1.045306in}{0.823704in}}%
\pgfpathcurveto{\pgfqpoint{1.034411in}{0.823704in}}{\pgfqpoint{1.023960in}{0.819375in}}{\pgfqpoint{1.016256in}{0.811671in}}%
\pgfpathcurveto{\pgfqpoint{1.008551in}{0.803967in}}{\pgfqpoint{1.004222in}{0.793516in}}{\pgfqpoint{1.004222in}{0.782620in}}%
\pgfpathcurveto{\pgfqpoint{1.004222in}{0.771725in}}{\pgfqpoint{1.008551in}{0.761274in}}{\pgfqpoint{1.016256in}{0.753569in}}%
\pgfpathcurveto{\pgfqpoint{1.023960in}{0.745865in}}{\pgfqpoint{1.034411in}{0.741536in}}{\pgfqpoint{1.045306in}{0.741536in}}%
\pgfpathlineto{\pgfqpoint{1.045306in}{0.741536in}}%
\pgfpathclose%
\pgfusepath{stroke}%
\end{pgfscope}%
\begin{pgfscope}%
\pgfpathrectangle{\pgfqpoint{0.688192in}{0.670138in}}{\pgfqpoint{7.111808in}{5.129862in}}%
\pgfusepath{clip}%
\pgfsetbuttcap%
\pgfsetroundjoin%
\pgfsetlinewidth{1.003750pt}%
\definecolor{currentstroke}{rgb}{0.000000,0.000000,0.000000}%
\pgfsetstrokecolor{currentstroke}%
\pgfsetdash{}{0pt}%
\pgfpathmoveto{\pgfqpoint{2.013663in}{0.690447in}}%
\pgfpathcurveto{\pgfqpoint{2.024559in}{0.690447in}}{\pgfqpoint{2.035010in}{0.694776in}}{\pgfqpoint{2.042714in}{0.702480in}}%
\pgfpathcurveto{\pgfqpoint{2.050418in}{0.710184in}}{\pgfqpoint{2.054747in}{0.720635in}}{\pgfqpoint{2.054747in}{0.731531in}}%
\pgfpathcurveto{\pgfqpoint{2.054747in}{0.742426in}}{\pgfqpoint{2.050418in}{0.752877in}}{\pgfqpoint{2.042714in}{0.760582in}}%
\pgfpathcurveto{\pgfqpoint{2.035010in}{0.768286in}}{\pgfqpoint{2.024559in}{0.772615in}}{\pgfqpoint{2.013663in}{0.772615in}}%
\pgfpathcurveto{\pgfqpoint{2.002768in}{0.772615in}}{\pgfqpoint{1.992317in}{0.768286in}}{\pgfqpoint{1.984613in}{0.760582in}}%
\pgfpathcurveto{\pgfqpoint{1.976908in}{0.752877in}}{\pgfqpoint{1.972579in}{0.742426in}}{\pgfqpoint{1.972579in}{0.731531in}}%
\pgfpathcurveto{\pgfqpoint{1.972579in}{0.720635in}}{\pgfqpoint{1.976908in}{0.710184in}}{\pgfqpoint{1.984613in}{0.702480in}}%
\pgfpathcurveto{\pgfqpoint{1.992317in}{0.694776in}}{\pgfqpoint{2.002768in}{0.690447in}}{\pgfqpoint{2.013663in}{0.690447in}}%
\pgfpathlineto{\pgfqpoint{2.013663in}{0.690447in}}%
\pgfpathclose%
\pgfusepath{stroke}%
\end{pgfscope}%
\begin{pgfscope}%
\pgfpathrectangle{\pgfqpoint{0.688192in}{0.670138in}}{\pgfqpoint{7.111808in}{5.129862in}}%
\pgfusepath{clip}%
\pgfsetbuttcap%
\pgfsetroundjoin%
\pgfsetlinewidth{1.003750pt}%
\definecolor{currentstroke}{rgb}{0.000000,0.000000,0.000000}%
\pgfsetstrokecolor{currentstroke}%
\pgfsetdash{}{0pt}%
\pgfpathmoveto{\pgfqpoint{0.895312in}{0.866174in}}%
\pgfpathcurveto{\pgfqpoint{0.906208in}{0.866174in}}{\pgfqpoint{0.916658in}{0.870503in}}{\pgfqpoint{0.924363in}{0.878207in}}%
\pgfpathcurveto{\pgfqpoint{0.932067in}{0.885911in}}{\pgfqpoint{0.936396in}{0.896362in}}{\pgfqpoint{0.936396in}{0.907258in}}%
\pgfpathcurveto{\pgfqpoint{0.936396in}{0.918153in}}{\pgfqpoint{0.932067in}{0.928604in}}{\pgfqpoint{0.924363in}{0.936308in}}%
\pgfpathcurveto{\pgfqpoint{0.916658in}{0.944013in}}{\pgfqpoint{0.906208in}{0.948341in}}{\pgfqpoint{0.895312in}{0.948341in}}%
\pgfpathcurveto{\pgfqpoint{0.884416in}{0.948341in}}{\pgfqpoint{0.873966in}{0.944013in}}{\pgfqpoint{0.866261in}{0.936308in}}%
\pgfpathcurveto{\pgfqpoint{0.858557in}{0.928604in}}{\pgfqpoint{0.854228in}{0.918153in}}{\pgfqpoint{0.854228in}{0.907258in}}%
\pgfpathcurveto{\pgfqpoint{0.854228in}{0.896362in}}{\pgfqpoint{0.858557in}{0.885911in}}{\pgfqpoint{0.866261in}{0.878207in}}%
\pgfpathcurveto{\pgfqpoint{0.873966in}{0.870503in}}{\pgfqpoint{0.884416in}{0.866174in}}{\pgfqpoint{0.895312in}{0.866174in}}%
\pgfpathlineto{\pgfqpoint{0.895312in}{0.866174in}}%
\pgfpathclose%
\pgfusepath{stroke}%
\end{pgfscope}%
\begin{pgfscope}%
\pgfpathrectangle{\pgfqpoint{0.688192in}{0.670138in}}{\pgfqpoint{7.111808in}{5.129862in}}%
\pgfusepath{clip}%
\pgfsetbuttcap%
\pgfsetroundjoin%
\pgfsetlinewidth{1.003750pt}%
\definecolor{currentstroke}{rgb}{0.000000,0.000000,0.000000}%
\pgfsetstrokecolor{currentstroke}%
\pgfsetdash{}{0pt}%
\pgfpathmoveto{\pgfqpoint{0.794194in}{1.022605in}}%
\pgfpathcurveto{\pgfqpoint{0.805090in}{1.022605in}}{\pgfqpoint{0.815541in}{1.026933in}}{\pgfqpoint{0.823245in}{1.034638in}}%
\pgfpathcurveto{\pgfqpoint{0.830949in}{1.042342in}}{\pgfqpoint{0.835278in}{1.052793in}}{\pgfqpoint{0.835278in}{1.063688in}}%
\pgfpathcurveto{\pgfqpoint{0.835278in}{1.074584in}}{\pgfqpoint{0.830949in}{1.085035in}}{\pgfqpoint{0.823245in}{1.092739in}}%
\pgfpathcurveto{\pgfqpoint{0.815541in}{1.100443in}}{\pgfqpoint{0.805090in}{1.104772in}}{\pgfqpoint{0.794194in}{1.104772in}}%
\pgfpathcurveto{\pgfqpoint{0.783299in}{1.104772in}}{\pgfqpoint{0.772848in}{1.100443in}}{\pgfqpoint{0.765144in}{1.092739in}}%
\pgfpathcurveto{\pgfqpoint{0.757439in}{1.085035in}}{\pgfqpoint{0.753110in}{1.074584in}}{\pgfqpoint{0.753110in}{1.063688in}}%
\pgfpathcurveto{\pgfqpoint{0.753110in}{1.052793in}}{\pgfqpoint{0.757439in}{1.042342in}}{\pgfqpoint{0.765144in}{1.034638in}}%
\pgfpathcurveto{\pgfqpoint{0.772848in}{1.026933in}}{\pgfqpoint{0.783299in}{1.022605in}}{\pgfqpoint{0.794194in}{1.022605in}}%
\pgfpathlineto{\pgfqpoint{0.794194in}{1.022605in}}%
\pgfpathclose%
\pgfusepath{stroke}%
\end{pgfscope}%
\begin{pgfscope}%
\pgfpathrectangle{\pgfqpoint{0.688192in}{0.670138in}}{\pgfqpoint{7.111808in}{5.129862in}}%
\pgfusepath{clip}%
\pgfsetbuttcap%
\pgfsetroundjoin%
\pgfsetlinewidth{1.003750pt}%
\definecolor{currentstroke}{rgb}{0.000000,0.000000,0.000000}%
\pgfsetstrokecolor{currentstroke}%
\pgfsetdash{}{0pt}%
\pgfpathmoveto{\pgfqpoint{1.019140in}{0.759981in}}%
\pgfpathcurveto{\pgfqpoint{1.030035in}{0.759981in}}{\pgfqpoint{1.040486in}{0.764309in}}{\pgfqpoint{1.048191in}{0.772014in}}%
\pgfpathcurveto{\pgfqpoint{1.055895in}{0.779718in}}{\pgfqpoint{1.060224in}{0.790169in}}{\pgfqpoint{1.060224in}{0.801064in}}%
\pgfpathcurveto{\pgfqpoint{1.060224in}{0.811960in}}{\pgfqpoint{1.055895in}{0.822411in}}{\pgfqpoint{1.048191in}{0.830115in}}%
\pgfpathcurveto{\pgfqpoint{1.040486in}{0.837820in}}{\pgfqpoint{1.030035in}{0.842148in}}{\pgfqpoint{1.019140in}{0.842148in}}%
\pgfpathcurveto{\pgfqpoint{1.008244in}{0.842148in}}{\pgfqpoint{0.997794in}{0.837820in}}{\pgfqpoint{0.990089in}{0.830115in}}%
\pgfpathcurveto{\pgfqpoint{0.982385in}{0.822411in}}{\pgfqpoint{0.978056in}{0.811960in}}{\pgfqpoint{0.978056in}{0.801064in}}%
\pgfpathcurveto{\pgfqpoint{0.978056in}{0.790169in}}{\pgfqpoint{0.982385in}{0.779718in}}{\pgfqpoint{0.990089in}{0.772014in}}%
\pgfpathcurveto{\pgfqpoint{0.997794in}{0.764309in}}{\pgfqpoint{1.008244in}{0.759981in}}{\pgfqpoint{1.019140in}{0.759981in}}%
\pgfpathlineto{\pgfqpoint{1.019140in}{0.759981in}}%
\pgfpathclose%
\pgfusepath{stroke}%
\end{pgfscope}%
\begin{pgfscope}%
\pgfpathrectangle{\pgfqpoint{0.688192in}{0.670138in}}{\pgfqpoint{7.111808in}{5.129862in}}%
\pgfusepath{clip}%
\pgfsetbuttcap%
\pgfsetroundjoin%
\pgfsetlinewidth{1.003750pt}%
\definecolor{currentstroke}{rgb}{0.000000,0.000000,0.000000}%
\pgfsetstrokecolor{currentstroke}%
\pgfsetdash{}{0pt}%
\pgfpathmoveto{\pgfqpoint{1.050667in}{0.739240in}}%
\pgfpathcurveto{\pgfqpoint{1.061563in}{0.739240in}}{\pgfqpoint{1.072014in}{0.743568in}}{\pgfqpoint{1.079718in}{0.751273in}}%
\pgfpathcurveto{\pgfqpoint{1.087422in}{0.758977in}}{\pgfqpoint{1.091751in}{0.769428in}}{\pgfqpoint{1.091751in}{0.780323in}}%
\pgfpathcurveto{\pgfqpoint{1.091751in}{0.791219in}}{\pgfqpoint{1.087422in}{0.801670in}}{\pgfqpoint{1.079718in}{0.809374in}}%
\pgfpathcurveto{\pgfqpoint{1.072014in}{0.817078in}}{\pgfqpoint{1.061563in}{0.821407in}}{\pgfqpoint{1.050667in}{0.821407in}}%
\pgfpathcurveto{\pgfqpoint{1.039772in}{0.821407in}}{\pgfqpoint{1.029321in}{0.817078in}}{\pgfqpoint{1.021617in}{0.809374in}}%
\pgfpathcurveto{\pgfqpoint{1.013912in}{0.801670in}}{\pgfqpoint{1.009583in}{0.791219in}}{\pgfqpoint{1.009583in}{0.780323in}}%
\pgfpathcurveto{\pgfqpoint{1.009583in}{0.769428in}}{\pgfqpoint{1.013912in}{0.758977in}}{\pgfqpoint{1.021617in}{0.751273in}}%
\pgfpathcurveto{\pgfqpoint{1.029321in}{0.743568in}}{\pgfqpoint{1.039772in}{0.739240in}}{\pgfqpoint{1.050667in}{0.739240in}}%
\pgfpathlineto{\pgfqpoint{1.050667in}{0.739240in}}%
\pgfpathclose%
\pgfusepath{stroke}%
\end{pgfscope}%
\begin{pgfscope}%
\pgfpathrectangle{\pgfqpoint{0.688192in}{0.670138in}}{\pgfqpoint{7.111808in}{5.129862in}}%
\pgfusepath{clip}%
\pgfsetbuttcap%
\pgfsetroundjoin%
\pgfsetlinewidth{1.003750pt}%
\definecolor{currentstroke}{rgb}{0.000000,0.000000,0.000000}%
\pgfsetstrokecolor{currentstroke}%
\pgfsetdash{}{0pt}%
\pgfpathmoveto{\pgfqpoint{0.881277in}{0.900428in}}%
\pgfpathcurveto{\pgfqpoint{0.892173in}{0.900428in}}{\pgfqpoint{0.902624in}{0.904756in}}{\pgfqpoint{0.910328in}{0.912461in}}%
\pgfpathcurveto{\pgfqpoint{0.918032in}{0.920165in}}{\pgfqpoint{0.922361in}{0.930616in}}{\pgfqpoint{0.922361in}{0.941511in}}%
\pgfpathcurveto{\pgfqpoint{0.922361in}{0.952407in}}{\pgfqpoint{0.918032in}{0.962858in}}{\pgfqpoint{0.910328in}{0.970562in}}%
\pgfpathcurveto{\pgfqpoint{0.902624in}{0.978266in}}{\pgfqpoint{0.892173in}{0.982595in}}{\pgfqpoint{0.881277in}{0.982595in}}%
\pgfpathcurveto{\pgfqpoint{0.870382in}{0.982595in}}{\pgfqpoint{0.859931in}{0.978266in}}{\pgfqpoint{0.852226in}{0.970562in}}%
\pgfpathcurveto{\pgfqpoint{0.844522in}{0.962858in}}{\pgfqpoint{0.840193in}{0.952407in}}{\pgfqpoint{0.840193in}{0.941511in}}%
\pgfpathcurveto{\pgfqpoint{0.840193in}{0.930616in}}{\pgfqpoint{0.844522in}{0.920165in}}{\pgfqpoint{0.852226in}{0.912461in}}%
\pgfpathcurveto{\pgfqpoint{0.859931in}{0.904756in}}{\pgfqpoint{0.870382in}{0.900428in}}{\pgfqpoint{0.881277in}{0.900428in}}%
\pgfpathlineto{\pgfqpoint{0.881277in}{0.900428in}}%
\pgfpathclose%
\pgfusepath{stroke}%
\end{pgfscope}%
\begin{pgfscope}%
\pgfpathrectangle{\pgfqpoint{0.688192in}{0.670138in}}{\pgfqpoint{7.111808in}{5.129862in}}%
\pgfusepath{clip}%
\pgfsetbuttcap%
\pgfsetroundjoin%
\pgfsetlinewidth{1.003750pt}%
\definecolor{currentstroke}{rgb}{0.000000,0.000000,0.000000}%
\pgfsetstrokecolor{currentstroke}%
\pgfsetdash{}{0pt}%
\pgfpathmoveto{\pgfqpoint{0.721104in}{1.921896in}}%
\pgfpathcurveto{\pgfqpoint{0.732000in}{1.921896in}}{\pgfqpoint{0.742451in}{1.926225in}}{\pgfqpoint{0.750155in}{1.933929in}}%
\pgfpathcurveto{\pgfqpoint{0.757859in}{1.941634in}}{\pgfqpoint{0.762188in}{1.952084in}}{\pgfqpoint{0.762188in}{1.962980in}}%
\pgfpathcurveto{\pgfqpoint{0.762188in}{1.973876in}}{\pgfqpoint{0.757859in}{1.984326in}}{\pgfqpoint{0.750155in}{1.992031in}}%
\pgfpathcurveto{\pgfqpoint{0.742451in}{1.999735in}}{\pgfqpoint{0.732000in}{2.004064in}}{\pgfqpoint{0.721104in}{2.004064in}}%
\pgfpathcurveto{\pgfqpoint{0.710209in}{2.004064in}}{\pgfqpoint{0.699758in}{1.999735in}}{\pgfqpoint{0.692054in}{1.992031in}}%
\pgfpathcurveto{\pgfqpoint{0.684349in}{1.984326in}}{\pgfqpoint{0.680020in}{1.973876in}}{\pgfqpoint{0.680020in}{1.962980in}}%
\pgfpathcurveto{\pgfqpoint{0.680020in}{1.952084in}}{\pgfqpoint{0.684349in}{1.941634in}}{\pgfqpoint{0.692054in}{1.933929in}}%
\pgfpathcurveto{\pgfqpoint{0.699758in}{1.926225in}}{\pgfqpoint{0.710209in}{1.921896in}}{\pgfqpoint{0.721104in}{1.921896in}}%
\pgfpathlineto{\pgfqpoint{0.721104in}{1.921896in}}%
\pgfpathclose%
\pgfusepath{stroke}%
\end{pgfscope}%
\begin{pgfscope}%
\pgfpathrectangle{\pgfqpoint{0.688192in}{0.670138in}}{\pgfqpoint{7.111808in}{5.129862in}}%
\pgfusepath{clip}%
\pgfsetbuttcap%
\pgfsetroundjoin%
\pgfsetlinewidth{1.003750pt}%
\definecolor{currentstroke}{rgb}{0.000000,0.000000,0.000000}%
\pgfsetstrokecolor{currentstroke}%
\pgfsetdash{}{0pt}%
\pgfpathmoveto{\pgfqpoint{1.351667in}{0.711476in}}%
\pgfpathcurveto{\pgfqpoint{1.362562in}{0.711476in}}{\pgfqpoint{1.373013in}{0.715805in}}{\pgfqpoint{1.380718in}{0.723510in}}%
\pgfpathcurveto{\pgfqpoint{1.388422in}{0.731214in}}{\pgfqpoint{1.392751in}{0.741665in}}{\pgfqpoint{1.392751in}{0.752560in}}%
\pgfpathcurveto{\pgfqpoint{1.392751in}{0.763456in}}{\pgfqpoint{1.388422in}{0.773907in}}{\pgfqpoint{1.380718in}{0.781611in}}%
\pgfpathcurveto{\pgfqpoint{1.373013in}{0.789315in}}{\pgfqpoint{1.362562in}{0.793644in}}{\pgfqpoint{1.351667in}{0.793644in}}%
\pgfpathcurveto{\pgfqpoint{1.340771in}{0.793644in}}{\pgfqpoint{1.330321in}{0.789315in}}{\pgfqpoint{1.322616in}{0.781611in}}%
\pgfpathcurveto{\pgfqpoint{1.314912in}{0.773907in}}{\pgfqpoint{1.310583in}{0.763456in}}{\pgfqpoint{1.310583in}{0.752560in}}%
\pgfpathcurveto{\pgfqpoint{1.310583in}{0.741665in}}{\pgfqpoint{1.314912in}{0.731214in}}{\pgfqpoint{1.322616in}{0.723510in}}%
\pgfpathcurveto{\pgfqpoint{1.330321in}{0.715805in}}{\pgfqpoint{1.340771in}{0.711476in}}{\pgfqpoint{1.351667in}{0.711476in}}%
\pgfpathlineto{\pgfqpoint{1.351667in}{0.711476in}}%
\pgfpathclose%
\pgfusepath{stroke}%
\end{pgfscope}%
\begin{pgfscope}%
\pgfpathrectangle{\pgfqpoint{0.688192in}{0.670138in}}{\pgfqpoint{7.111808in}{5.129862in}}%
\pgfusepath{clip}%
\pgfsetbuttcap%
\pgfsetroundjoin%
\pgfsetlinewidth{1.003750pt}%
\definecolor{currentstroke}{rgb}{0.000000,0.000000,0.000000}%
\pgfsetstrokecolor{currentstroke}%
\pgfsetdash{}{0pt}%
\pgfpathmoveto{\pgfqpoint{0.928243in}{0.841597in}}%
\pgfpathcurveto{\pgfqpoint{0.939139in}{0.841597in}}{\pgfqpoint{0.949589in}{0.845926in}}{\pgfqpoint{0.957294in}{0.853631in}}%
\pgfpathcurveto{\pgfqpoint{0.964998in}{0.861335in}}{\pgfqpoint{0.969327in}{0.871786in}}{\pgfqpoint{0.969327in}{0.882681in}}%
\pgfpathcurveto{\pgfqpoint{0.969327in}{0.893577in}}{\pgfqpoint{0.964998in}{0.904028in}}{\pgfqpoint{0.957294in}{0.911732in}}%
\pgfpathcurveto{\pgfqpoint{0.949589in}{0.919436in}}{\pgfqpoint{0.939139in}{0.923765in}}{\pgfqpoint{0.928243in}{0.923765in}}%
\pgfpathcurveto{\pgfqpoint{0.917347in}{0.923765in}}{\pgfqpoint{0.906897in}{0.919436in}}{\pgfqpoint{0.899192in}{0.911732in}}%
\pgfpathcurveto{\pgfqpoint{0.891488in}{0.904028in}}{\pgfqpoint{0.887159in}{0.893577in}}{\pgfqpoint{0.887159in}{0.882681in}}%
\pgfpathcurveto{\pgfqpoint{0.887159in}{0.871786in}}{\pgfqpoint{0.891488in}{0.861335in}}{\pgfqpoint{0.899192in}{0.853631in}}%
\pgfpathcurveto{\pgfqpoint{0.906897in}{0.845926in}}{\pgfqpoint{0.917347in}{0.841597in}}{\pgfqpoint{0.928243in}{0.841597in}}%
\pgfpathlineto{\pgfqpoint{0.928243in}{0.841597in}}%
\pgfpathclose%
\pgfusepath{stroke}%
\end{pgfscope}%
\begin{pgfscope}%
\pgfpathrectangle{\pgfqpoint{0.688192in}{0.670138in}}{\pgfqpoint{7.111808in}{5.129862in}}%
\pgfusepath{clip}%
\pgfsetbuttcap%
\pgfsetroundjoin%
\pgfsetlinewidth{1.003750pt}%
\definecolor{currentstroke}{rgb}{0.000000,0.000000,0.000000}%
\pgfsetstrokecolor{currentstroke}%
\pgfsetdash{}{0pt}%
\pgfpathmoveto{\pgfqpoint{6.014792in}{1.437812in}}%
\pgfpathcurveto{\pgfqpoint{6.025688in}{1.437812in}}{\pgfqpoint{6.036139in}{1.442141in}}{\pgfqpoint{6.043843in}{1.449845in}}%
\pgfpathcurveto{\pgfqpoint{6.051547in}{1.457549in}}{\pgfqpoint{6.055876in}{1.468000in}}{\pgfqpoint{6.055876in}{1.478896in}}%
\pgfpathcurveto{\pgfqpoint{6.055876in}{1.489791in}}{\pgfqpoint{6.051547in}{1.500242in}}{\pgfqpoint{6.043843in}{1.507947in}}%
\pgfpathcurveto{\pgfqpoint{6.036139in}{1.515651in}}{\pgfqpoint{6.025688in}{1.519980in}}{\pgfqpoint{6.014792in}{1.519980in}}%
\pgfpathcurveto{\pgfqpoint{6.003897in}{1.519980in}}{\pgfqpoint{5.993446in}{1.515651in}}{\pgfqpoint{5.985741in}{1.507947in}}%
\pgfpathcurveto{\pgfqpoint{5.978037in}{1.500242in}}{\pgfqpoint{5.973708in}{1.489791in}}{\pgfqpoint{5.973708in}{1.478896in}}%
\pgfpathcurveto{\pgfqpoint{5.973708in}{1.468000in}}{\pgfqpoint{5.978037in}{1.457549in}}{\pgfqpoint{5.985741in}{1.449845in}}%
\pgfpathcurveto{\pgfqpoint{5.993446in}{1.442141in}}{\pgfqpoint{6.003897in}{1.437812in}}{\pgfqpoint{6.014792in}{1.437812in}}%
\pgfpathlineto{\pgfqpoint{6.014792in}{1.437812in}}%
\pgfpathclose%
\pgfusepath{stroke}%
\end{pgfscope}%
\begin{pgfscope}%
\pgfpathrectangle{\pgfqpoint{0.688192in}{0.670138in}}{\pgfqpoint{7.111808in}{5.129862in}}%
\pgfusepath{clip}%
\pgfsetbuttcap%
\pgfsetroundjoin%
\pgfsetlinewidth{1.003750pt}%
\definecolor{currentstroke}{rgb}{0.000000,0.000000,0.000000}%
\pgfsetstrokecolor{currentstroke}%
\pgfsetdash{}{0pt}%
\pgfpathmoveto{\pgfqpoint{0.895312in}{0.866174in}}%
\pgfpathcurveto{\pgfqpoint{0.906208in}{0.866174in}}{\pgfqpoint{0.916658in}{0.870503in}}{\pgfqpoint{0.924363in}{0.878207in}}%
\pgfpathcurveto{\pgfqpoint{0.932067in}{0.885911in}}{\pgfqpoint{0.936396in}{0.896362in}}{\pgfqpoint{0.936396in}{0.907258in}}%
\pgfpathcurveto{\pgfqpoint{0.936396in}{0.918153in}}{\pgfqpoint{0.932067in}{0.928604in}}{\pgfqpoint{0.924363in}{0.936308in}}%
\pgfpathcurveto{\pgfqpoint{0.916658in}{0.944013in}}{\pgfqpoint{0.906208in}{0.948341in}}{\pgfqpoint{0.895312in}{0.948341in}}%
\pgfpathcurveto{\pgfqpoint{0.884416in}{0.948341in}}{\pgfqpoint{0.873966in}{0.944013in}}{\pgfqpoint{0.866261in}{0.936308in}}%
\pgfpathcurveto{\pgfqpoint{0.858557in}{0.928604in}}{\pgfqpoint{0.854228in}{0.918153in}}{\pgfqpoint{0.854228in}{0.907258in}}%
\pgfpathcurveto{\pgfqpoint{0.854228in}{0.896362in}}{\pgfqpoint{0.858557in}{0.885911in}}{\pgfqpoint{0.866261in}{0.878207in}}%
\pgfpathcurveto{\pgfqpoint{0.873966in}{0.870503in}}{\pgfqpoint{0.884416in}{0.866174in}}{\pgfqpoint{0.895312in}{0.866174in}}%
\pgfpathlineto{\pgfqpoint{0.895312in}{0.866174in}}%
\pgfpathclose%
\pgfusepath{stroke}%
\end{pgfscope}%
\begin{pgfscope}%
\pgfpathrectangle{\pgfqpoint{0.688192in}{0.670138in}}{\pgfqpoint{7.111808in}{5.129862in}}%
\pgfusepath{clip}%
\pgfsetbuttcap%
\pgfsetroundjoin%
\pgfsetlinewidth{1.003750pt}%
\definecolor{currentstroke}{rgb}{0.000000,0.000000,0.000000}%
\pgfsetstrokecolor{currentstroke}%
\pgfsetdash{}{0pt}%
\pgfpathmoveto{\pgfqpoint{1.759569in}{0.698796in}}%
\pgfpathcurveto{\pgfqpoint{1.770464in}{0.698796in}}{\pgfqpoint{1.780915in}{0.703125in}}{\pgfqpoint{1.788619in}{0.710830in}}%
\pgfpathcurveto{\pgfqpoint{1.796324in}{0.718534in}}{\pgfqpoint{1.800652in}{0.728985in}}{\pgfqpoint{1.800652in}{0.739880in}}%
\pgfpathcurveto{\pgfqpoint{1.800652in}{0.750776in}}{\pgfqpoint{1.796324in}{0.761227in}}{\pgfqpoint{1.788619in}{0.768931in}}%
\pgfpathcurveto{\pgfqpoint{1.780915in}{0.776635in}}{\pgfqpoint{1.770464in}{0.780964in}}{\pgfqpoint{1.759569in}{0.780964in}}%
\pgfpathcurveto{\pgfqpoint{1.748673in}{0.780964in}}{\pgfqpoint{1.738222in}{0.776635in}}{\pgfqpoint{1.730518in}{0.768931in}}%
\pgfpathcurveto{\pgfqpoint{1.722814in}{0.761227in}}{\pgfqpoint{1.718485in}{0.750776in}}{\pgfqpoint{1.718485in}{0.739880in}}%
\pgfpathcurveto{\pgfqpoint{1.718485in}{0.728985in}}{\pgfqpoint{1.722814in}{0.718534in}}{\pgfqpoint{1.730518in}{0.710830in}}%
\pgfpathcurveto{\pgfqpoint{1.738222in}{0.703125in}}{\pgfqpoint{1.748673in}{0.698796in}}{\pgfqpoint{1.759569in}{0.698796in}}%
\pgfpathlineto{\pgfqpoint{1.759569in}{0.698796in}}%
\pgfpathclose%
\pgfusepath{stroke}%
\end{pgfscope}%
\begin{pgfscope}%
\pgfpathrectangle{\pgfqpoint{0.688192in}{0.670138in}}{\pgfqpoint{7.111808in}{5.129862in}}%
\pgfusepath{clip}%
\pgfsetbuttcap%
\pgfsetroundjoin%
\pgfsetlinewidth{1.003750pt}%
\definecolor{currentstroke}{rgb}{0.000000,0.000000,0.000000}%
\pgfsetstrokecolor{currentstroke}%
\pgfsetdash{}{0pt}%
\pgfpathmoveto{\pgfqpoint{1.351667in}{0.711476in}}%
\pgfpathcurveto{\pgfqpoint{1.362562in}{0.711476in}}{\pgfqpoint{1.373013in}{0.715805in}}{\pgfqpoint{1.380718in}{0.723510in}}%
\pgfpathcurveto{\pgfqpoint{1.388422in}{0.731214in}}{\pgfqpoint{1.392751in}{0.741665in}}{\pgfqpoint{1.392751in}{0.752560in}}%
\pgfpathcurveto{\pgfqpoint{1.392751in}{0.763456in}}{\pgfqpoint{1.388422in}{0.773907in}}{\pgfqpoint{1.380718in}{0.781611in}}%
\pgfpathcurveto{\pgfqpoint{1.373013in}{0.789315in}}{\pgfqpoint{1.362562in}{0.793644in}}{\pgfqpoint{1.351667in}{0.793644in}}%
\pgfpathcurveto{\pgfqpoint{1.340771in}{0.793644in}}{\pgfqpoint{1.330321in}{0.789315in}}{\pgfqpoint{1.322616in}{0.781611in}}%
\pgfpathcurveto{\pgfqpoint{1.314912in}{0.773907in}}{\pgfqpoint{1.310583in}{0.763456in}}{\pgfqpoint{1.310583in}{0.752560in}}%
\pgfpathcurveto{\pgfqpoint{1.310583in}{0.741665in}}{\pgfqpoint{1.314912in}{0.731214in}}{\pgfqpoint{1.322616in}{0.723510in}}%
\pgfpathcurveto{\pgfqpoint{1.330321in}{0.715805in}}{\pgfqpoint{1.340771in}{0.711476in}}{\pgfqpoint{1.351667in}{0.711476in}}%
\pgfpathlineto{\pgfqpoint{1.351667in}{0.711476in}}%
\pgfpathclose%
\pgfusepath{stroke}%
\end{pgfscope}%
\begin{pgfscope}%
\pgfpathrectangle{\pgfqpoint{0.688192in}{0.670138in}}{\pgfqpoint{7.111808in}{5.129862in}}%
\pgfusepath{clip}%
\pgfsetbuttcap%
\pgfsetroundjoin%
\pgfsetlinewidth{1.003750pt}%
\definecolor{currentstroke}{rgb}{0.000000,0.000000,0.000000}%
\pgfsetstrokecolor{currentstroke}%
\pgfsetdash{}{0pt}%
\pgfpathmoveto{\pgfqpoint{1.631842in}{0.703864in}}%
\pgfpathcurveto{\pgfqpoint{1.642737in}{0.703864in}}{\pgfqpoint{1.653188in}{0.708193in}}{\pgfqpoint{1.660893in}{0.715897in}}%
\pgfpathcurveto{\pgfqpoint{1.668597in}{0.723601in}}{\pgfqpoint{1.672926in}{0.734052in}}{\pgfqpoint{1.672926in}{0.744948in}}%
\pgfpathcurveto{\pgfqpoint{1.672926in}{0.755843in}}{\pgfqpoint{1.668597in}{0.766294in}}{\pgfqpoint{1.660893in}{0.773998in}}%
\pgfpathcurveto{\pgfqpoint{1.653188in}{0.781703in}}{\pgfqpoint{1.642737in}{0.786032in}}{\pgfqpoint{1.631842in}{0.786032in}}%
\pgfpathcurveto{\pgfqpoint{1.620946in}{0.786032in}}{\pgfqpoint{1.610496in}{0.781703in}}{\pgfqpoint{1.602791in}{0.773998in}}%
\pgfpathcurveto{\pgfqpoint{1.595087in}{0.766294in}}{\pgfqpoint{1.590758in}{0.755843in}}{\pgfqpoint{1.590758in}{0.744948in}}%
\pgfpathcurveto{\pgfqpoint{1.590758in}{0.734052in}}{\pgfqpoint{1.595087in}{0.723601in}}{\pgfqpoint{1.602791in}{0.715897in}}%
\pgfpathcurveto{\pgfqpoint{1.610496in}{0.708193in}}{\pgfqpoint{1.620946in}{0.703864in}}{\pgfqpoint{1.631842in}{0.703864in}}%
\pgfpathlineto{\pgfqpoint{1.631842in}{0.703864in}}%
\pgfpathclose%
\pgfusepath{stroke}%
\end{pgfscope}%
\begin{pgfscope}%
\pgfpathrectangle{\pgfqpoint{0.688192in}{0.670138in}}{\pgfqpoint{7.111808in}{5.129862in}}%
\pgfusepath{clip}%
\pgfsetbuttcap%
\pgfsetroundjoin%
\pgfsetlinewidth{1.003750pt}%
\definecolor{currentstroke}{rgb}{0.000000,0.000000,0.000000}%
\pgfsetstrokecolor{currentstroke}%
\pgfsetdash{}{0pt}%
\pgfpathmoveto{\pgfqpoint{6.024600in}{3.448401in}}%
\pgfpathcurveto{\pgfqpoint{6.035496in}{3.448401in}}{\pgfqpoint{6.045947in}{3.452730in}}{\pgfqpoint{6.053651in}{3.460434in}}%
\pgfpathcurveto{\pgfqpoint{6.061355in}{3.468139in}}{\pgfqpoint{6.065684in}{3.478589in}}{\pgfqpoint{6.065684in}{3.489485in}}%
\pgfpathcurveto{\pgfqpoint{6.065684in}{3.500381in}}{\pgfqpoint{6.061355in}{3.510831in}}{\pgfqpoint{6.053651in}{3.518536in}}%
\pgfpathcurveto{\pgfqpoint{6.045947in}{3.526240in}}{\pgfqpoint{6.035496in}{3.530569in}}{\pgfqpoint{6.024600in}{3.530569in}}%
\pgfpathcurveto{\pgfqpoint{6.013705in}{3.530569in}}{\pgfqpoint{6.003254in}{3.526240in}}{\pgfqpoint{5.995550in}{3.518536in}}%
\pgfpathcurveto{\pgfqpoint{5.987845in}{3.510831in}}{\pgfqpoint{5.983517in}{3.500381in}}{\pgfqpoint{5.983517in}{3.489485in}}%
\pgfpathcurveto{\pgfqpoint{5.983517in}{3.478589in}}{\pgfqpoint{5.987845in}{3.468139in}}{\pgfqpoint{5.995550in}{3.460434in}}%
\pgfpathcurveto{\pgfqpoint{6.003254in}{3.452730in}}{\pgfqpoint{6.013705in}{3.448401in}}{\pgfqpoint{6.024600in}{3.448401in}}%
\pgfpathlineto{\pgfqpoint{6.024600in}{3.448401in}}%
\pgfpathclose%
\pgfusepath{stroke}%
\end{pgfscope}%
\begin{pgfscope}%
\pgfpathrectangle{\pgfqpoint{0.688192in}{0.670138in}}{\pgfqpoint{7.111808in}{5.129862in}}%
\pgfusepath{clip}%
\pgfsetbuttcap%
\pgfsetroundjoin%
\pgfsetlinewidth{1.003750pt}%
\definecolor{currentstroke}{rgb}{0.000000,0.000000,0.000000}%
\pgfsetstrokecolor{currentstroke}%
\pgfsetdash{}{0pt}%
\pgfpathmoveto{\pgfqpoint{0.721104in}{1.921896in}}%
\pgfpathcurveto{\pgfqpoint{0.732000in}{1.921896in}}{\pgfqpoint{0.742451in}{1.926225in}}{\pgfqpoint{0.750155in}{1.933929in}}%
\pgfpathcurveto{\pgfqpoint{0.757859in}{1.941634in}}{\pgfqpoint{0.762188in}{1.952084in}}{\pgfqpoint{0.762188in}{1.962980in}}%
\pgfpathcurveto{\pgfqpoint{0.762188in}{1.973876in}}{\pgfqpoint{0.757859in}{1.984326in}}{\pgfqpoint{0.750155in}{1.992031in}}%
\pgfpathcurveto{\pgfqpoint{0.742451in}{1.999735in}}{\pgfqpoint{0.732000in}{2.004064in}}{\pgfqpoint{0.721104in}{2.004064in}}%
\pgfpathcurveto{\pgfqpoint{0.710209in}{2.004064in}}{\pgfqpoint{0.699758in}{1.999735in}}{\pgfqpoint{0.692054in}{1.992031in}}%
\pgfpathcurveto{\pgfqpoint{0.684349in}{1.984326in}}{\pgfqpoint{0.680020in}{1.973876in}}{\pgfqpoint{0.680020in}{1.962980in}}%
\pgfpathcurveto{\pgfqpoint{0.680020in}{1.952084in}}{\pgfqpoint{0.684349in}{1.941634in}}{\pgfqpoint{0.692054in}{1.933929in}}%
\pgfpathcurveto{\pgfqpoint{0.699758in}{1.926225in}}{\pgfqpoint{0.710209in}{1.921896in}}{\pgfqpoint{0.721104in}{1.921896in}}%
\pgfpathlineto{\pgfqpoint{0.721104in}{1.921896in}}%
\pgfpathclose%
\pgfusepath{stroke}%
\end{pgfscope}%
\begin{pgfscope}%
\pgfpathrectangle{\pgfqpoint{0.688192in}{0.670138in}}{\pgfqpoint{7.111808in}{5.129862in}}%
\pgfusepath{clip}%
\pgfsetbuttcap%
\pgfsetroundjoin%
\pgfsetlinewidth{1.003750pt}%
\definecolor{currentstroke}{rgb}{0.000000,0.000000,0.000000}%
\pgfsetstrokecolor{currentstroke}%
\pgfsetdash{}{0pt}%
\pgfpathmoveto{\pgfqpoint{2.751273in}{1.008701in}}%
\pgfpathcurveto{\pgfqpoint{2.762169in}{1.008701in}}{\pgfqpoint{2.772620in}{1.013030in}}{\pgfqpoint{2.780324in}{1.020734in}}%
\pgfpathcurveto{\pgfqpoint{2.788028in}{1.028439in}}{\pgfqpoint{2.792357in}{1.038890in}}{\pgfqpoint{2.792357in}{1.049785in}}%
\pgfpathcurveto{\pgfqpoint{2.792357in}{1.060681in}}{\pgfqpoint{2.788028in}{1.071131in}}{\pgfqpoint{2.780324in}{1.078836in}}%
\pgfpathcurveto{\pgfqpoint{2.772620in}{1.086540in}}{\pgfqpoint{2.762169in}{1.090869in}}{\pgfqpoint{2.751273in}{1.090869in}}%
\pgfpathcurveto{\pgfqpoint{2.740378in}{1.090869in}}{\pgfqpoint{2.729927in}{1.086540in}}{\pgfqpoint{2.722223in}{1.078836in}}%
\pgfpathcurveto{\pgfqpoint{2.714518in}{1.071131in}}{\pgfqpoint{2.710189in}{1.060681in}}{\pgfqpoint{2.710189in}{1.049785in}}%
\pgfpathcurveto{\pgfqpoint{2.710189in}{1.038890in}}{\pgfqpoint{2.714518in}{1.028439in}}{\pgfqpoint{2.722223in}{1.020734in}}%
\pgfpathcurveto{\pgfqpoint{2.729927in}{1.013030in}}{\pgfqpoint{2.740378in}{1.008701in}}{\pgfqpoint{2.751273in}{1.008701in}}%
\pgfpathlineto{\pgfqpoint{2.751273in}{1.008701in}}%
\pgfpathclose%
\pgfusepath{stroke}%
\end{pgfscope}%
\begin{pgfscope}%
\pgfpathrectangle{\pgfqpoint{0.688192in}{0.670138in}}{\pgfqpoint{7.111808in}{5.129862in}}%
\pgfusepath{clip}%
\pgfsetbuttcap%
\pgfsetroundjoin%
\pgfsetlinewidth{1.003750pt}%
\definecolor{currentstroke}{rgb}{0.000000,0.000000,0.000000}%
\pgfsetstrokecolor{currentstroke}%
\pgfsetdash{}{0pt}%
\pgfpathmoveto{\pgfqpoint{1.625190in}{0.704414in}}%
\pgfpathcurveto{\pgfqpoint{1.636086in}{0.704414in}}{\pgfqpoint{1.646537in}{0.708742in}}{\pgfqpoint{1.654241in}{0.716447in}}%
\pgfpathcurveto{\pgfqpoint{1.661945in}{0.724151in}}{\pgfqpoint{1.666274in}{0.734602in}}{\pgfqpoint{1.666274in}{0.745498in}}%
\pgfpathcurveto{\pgfqpoint{1.666274in}{0.756393in}}{\pgfqpoint{1.661945in}{0.766844in}}{\pgfqpoint{1.654241in}{0.774548in}}%
\pgfpathcurveto{\pgfqpoint{1.646537in}{0.782253in}}{\pgfqpoint{1.636086in}{0.786581in}}{\pgfqpoint{1.625190in}{0.786581in}}%
\pgfpathcurveto{\pgfqpoint{1.614295in}{0.786581in}}{\pgfqpoint{1.603844in}{0.782253in}}{\pgfqpoint{1.596140in}{0.774548in}}%
\pgfpathcurveto{\pgfqpoint{1.588435in}{0.766844in}}{\pgfqpoint{1.584107in}{0.756393in}}{\pgfqpoint{1.584107in}{0.745498in}}%
\pgfpathcurveto{\pgfqpoint{1.584107in}{0.734602in}}{\pgfqpoint{1.588435in}{0.724151in}}{\pgfqpoint{1.596140in}{0.716447in}}%
\pgfpathcurveto{\pgfqpoint{1.603844in}{0.708742in}}{\pgfqpoint{1.614295in}{0.704414in}}{\pgfqpoint{1.625190in}{0.704414in}}%
\pgfpathlineto{\pgfqpoint{1.625190in}{0.704414in}}%
\pgfpathclose%
\pgfusepath{stroke}%
\end{pgfscope}%
\begin{pgfscope}%
\pgfpathrectangle{\pgfqpoint{0.688192in}{0.670138in}}{\pgfqpoint{7.111808in}{5.129862in}}%
\pgfusepath{clip}%
\pgfsetbuttcap%
\pgfsetroundjoin%
\pgfsetlinewidth{1.003750pt}%
\definecolor{currentstroke}{rgb}{0.000000,0.000000,0.000000}%
\pgfsetstrokecolor{currentstroke}%
\pgfsetdash{}{0pt}%
\pgfpathmoveto{\pgfqpoint{0.880316in}{0.913368in}}%
\pgfpathcurveto{\pgfqpoint{0.891212in}{0.913368in}}{\pgfqpoint{0.901663in}{0.917697in}}{\pgfqpoint{0.909367in}{0.925401in}}%
\pgfpathcurveto{\pgfqpoint{0.917071in}{0.933106in}}{\pgfqpoint{0.921400in}{0.943557in}}{\pgfqpoint{0.921400in}{0.954452in}}%
\pgfpathcurveto{\pgfqpoint{0.921400in}{0.965348in}}{\pgfqpoint{0.917071in}{0.975799in}}{\pgfqpoint{0.909367in}{0.983503in}}%
\pgfpathcurveto{\pgfqpoint{0.901663in}{0.991207in}}{\pgfqpoint{0.891212in}{0.995536in}}{\pgfqpoint{0.880316in}{0.995536in}}%
\pgfpathcurveto{\pgfqpoint{0.869421in}{0.995536in}}{\pgfqpoint{0.858970in}{0.991207in}}{\pgfqpoint{0.851266in}{0.983503in}}%
\pgfpathcurveto{\pgfqpoint{0.843561in}{0.975799in}}{\pgfqpoint{0.839232in}{0.965348in}}{\pgfqpoint{0.839232in}{0.954452in}}%
\pgfpathcurveto{\pgfqpoint{0.839232in}{0.943557in}}{\pgfqpoint{0.843561in}{0.933106in}}{\pgfqpoint{0.851266in}{0.925401in}}%
\pgfpathcurveto{\pgfqpoint{0.858970in}{0.917697in}}{\pgfqpoint{0.869421in}{0.913368in}}{\pgfqpoint{0.880316in}{0.913368in}}%
\pgfpathlineto{\pgfqpoint{0.880316in}{0.913368in}}%
\pgfpathclose%
\pgfusepath{stroke}%
\end{pgfscope}%
\begin{pgfscope}%
\pgfpathrectangle{\pgfqpoint{0.688192in}{0.670138in}}{\pgfqpoint{7.111808in}{5.129862in}}%
\pgfusepath{clip}%
\pgfsetbuttcap%
\pgfsetroundjoin%
\pgfsetlinewidth{1.003750pt}%
\definecolor{currentstroke}{rgb}{0.000000,0.000000,0.000000}%
\pgfsetstrokecolor{currentstroke}%
\pgfsetdash{}{0pt}%
\pgfpathmoveto{\pgfqpoint{1.631842in}{0.703864in}}%
\pgfpathcurveto{\pgfqpoint{1.642737in}{0.703864in}}{\pgfqpoint{1.653188in}{0.708193in}}{\pgfqpoint{1.660893in}{0.715897in}}%
\pgfpathcurveto{\pgfqpoint{1.668597in}{0.723601in}}{\pgfqpoint{1.672926in}{0.734052in}}{\pgfqpoint{1.672926in}{0.744948in}}%
\pgfpathcurveto{\pgfqpoint{1.672926in}{0.755843in}}{\pgfqpoint{1.668597in}{0.766294in}}{\pgfqpoint{1.660893in}{0.773998in}}%
\pgfpathcurveto{\pgfqpoint{1.653188in}{0.781703in}}{\pgfqpoint{1.642737in}{0.786032in}}{\pgfqpoint{1.631842in}{0.786032in}}%
\pgfpathcurveto{\pgfqpoint{1.620946in}{0.786032in}}{\pgfqpoint{1.610496in}{0.781703in}}{\pgfqpoint{1.602791in}{0.773998in}}%
\pgfpathcurveto{\pgfqpoint{1.595087in}{0.766294in}}{\pgfqpoint{1.590758in}{0.755843in}}{\pgfqpoint{1.590758in}{0.744948in}}%
\pgfpathcurveto{\pgfqpoint{1.590758in}{0.734052in}}{\pgfqpoint{1.595087in}{0.723601in}}{\pgfqpoint{1.602791in}{0.715897in}}%
\pgfpathcurveto{\pgfqpoint{1.610496in}{0.708193in}}{\pgfqpoint{1.620946in}{0.703864in}}{\pgfqpoint{1.631842in}{0.703864in}}%
\pgfpathlineto{\pgfqpoint{1.631842in}{0.703864in}}%
\pgfpathclose%
\pgfusepath{stroke}%
\end{pgfscope}%
\begin{pgfscope}%
\pgfpathrectangle{\pgfqpoint{0.688192in}{0.670138in}}{\pgfqpoint{7.111808in}{5.129862in}}%
\pgfusepath{clip}%
\pgfsetbuttcap%
\pgfsetroundjoin%
\pgfsetlinewidth{1.003750pt}%
\definecolor{currentstroke}{rgb}{0.000000,0.000000,0.000000}%
\pgfsetstrokecolor{currentstroke}%
\pgfsetdash{}{0pt}%
\pgfpathmoveto{\pgfqpoint{1.044526in}{0.742721in}}%
\pgfpathcurveto{\pgfqpoint{1.055421in}{0.742721in}}{\pgfqpoint{1.065872in}{0.747050in}}{\pgfqpoint{1.073577in}{0.754754in}}%
\pgfpathcurveto{\pgfqpoint{1.081281in}{0.762459in}}{\pgfqpoint{1.085610in}{0.772909in}}{\pgfqpoint{1.085610in}{0.783805in}}%
\pgfpathcurveto{\pgfqpoint{1.085610in}{0.794701in}}{\pgfqpoint{1.081281in}{0.805151in}}{\pgfqpoint{1.073577in}{0.812856in}}%
\pgfpathcurveto{\pgfqpoint{1.065872in}{0.820560in}}{\pgfqpoint{1.055421in}{0.824889in}}{\pgfqpoint{1.044526in}{0.824889in}}%
\pgfpathcurveto{\pgfqpoint{1.033630in}{0.824889in}}{\pgfqpoint{1.023179in}{0.820560in}}{\pgfqpoint{1.015475in}{0.812856in}}%
\pgfpathcurveto{\pgfqpoint{1.007771in}{0.805151in}}{\pgfqpoint{1.003442in}{0.794701in}}{\pgfqpoint{1.003442in}{0.783805in}}%
\pgfpathcurveto{\pgfqpoint{1.003442in}{0.772909in}}{\pgfqpoint{1.007771in}{0.762459in}}{\pgfqpoint{1.015475in}{0.754754in}}%
\pgfpathcurveto{\pgfqpoint{1.023179in}{0.747050in}}{\pgfqpoint{1.033630in}{0.742721in}}{\pgfqpoint{1.044526in}{0.742721in}}%
\pgfpathlineto{\pgfqpoint{1.044526in}{0.742721in}}%
\pgfpathclose%
\pgfusepath{stroke}%
\end{pgfscope}%
\begin{pgfscope}%
\pgfpathrectangle{\pgfqpoint{0.688192in}{0.670138in}}{\pgfqpoint{7.111808in}{5.129862in}}%
\pgfusepath{clip}%
\pgfsetbuttcap%
\pgfsetroundjoin%
\pgfsetlinewidth{1.003750pt}%
\definecolor{currentstroke}{rgb}{0.000000,0.000000,0.000000}%
\pgfsetstrokecolor{currentstroke}%
\pgfsetdash{}{0pt}%
\pgfpathmoveto{\pgfqpoint{6.048059in}{5.126928in}}%
\pgfpathcurveto{\pgfqpoint{6.058955in}{5.126928in}}{\pgfqpoint{6.069405in}{5.131257in}}{\pgfqpoint{6.077110in}{5.138961in}}%
\pgfpathcurveto{\pgfqpoint{6.084814in}{5.146665in}}{\pgfqpoint{6.089143in}{5.157116in}}{\pgfqpoint{6.089143in}{5.168012in}}%
\pgfpathcurveto{\pgfqpoint{6.089143in}{5.178907in}}{\pgfqpoint{6.084814in}{5.189358in}}{\pgfqpoint{6.077110in}{5.197062in}}%
\pgfpathcurveto{\pgfqpoint{6.069405in}{5.204767in}}{\pgfqpoint{6.058955in}{5.209096in}}{\pgfqpoint{6.048059in}{5.209096in}}%
\pgfpathcurveto{\pgfqpoint{6.037164in}{5.209096in}}{\pgfqpoint{6.026713in}{5.204767in}}{\pgfqpoint{6.019008in}{5.197062in}}%
\pgfpathcurveto{\pgfqpoint{6.011304in}{5.189358in}}{\pgfqpoint{6.006975in}{5.178907in}}{\pgfqpoint{6.006975in}{5.168012in}}%
\pgfpathcurveto{\pgfqpoint{6.006975in}{5.157116in}}{\pgfqpoint{6.011304in}{5.146665in}}{\pgfqpoint{6.019008in}{5.138961in}}%
\pgfpathcurveto{\pgfqpoint{6.026713in}{5.131257in}}{\pgfqpoint{6.037164in}{5.126928in}}{\pgfqpoint{6.048059in}{5.126928in}}%
\pgfpathlineto{\pgfqpoint{6.048059in}{5.126928in}}%
\pgfpathclose%
\pgfusepath{stroke}%
\end{pgfscope}%
\begin{pgfscope}%
\pgfpathrectangle{\pgfqpoint{0.688192in}{0.670138in}}{\pgfqpoint{7.111808in}{5.129862in}}%
\pgfusepath{clip}%
\pgfsetbuttcap%
\pgfsetroundjoin%
\pgfsetlinewidth{1.003750pt}%
\definecolor{currentstroke}{rgb}{0.000000,0.000000,0.000000}%
\pgfsetstrokecolor{currentstroke}%
\pgfsetdash{}{0pt}%
\pgfpathmoveto{\pgfqpoint{1.050667in}{0.739240in}}%
\pgfpathcurveto{\pgfqpoint{1.061563in}{0.739240in}}{\pgfqpoint{1.072014in}{0.743568in}}{\pgfqpoint{1.079718in}{0.751273in}}%
\pgfpathcurveto{\pgfqpoint{1.087422in}{0.758977in}}{\pgfqpoint{1.091751in}{0.769428in}}{\pgfqpoint{1.091751in}{0.780323in}}%
\pgfpathcurveto{\pgfqpoint{1.091751in}{0.791219in}}{\pgfqpoint{1.087422in}{0.801670in}}{\pgfqpoint{1.079718in}{0.809374in}}%
\pgfpathcurveto{\pgfqpoint{1.072014in}{0.817078in}}{\pgfqpoint{1.061563in}{0.821407in}}{\pgfqpoint{1.050667in}{0.821407in}}%
\pgfpathcurveto{\pgfqpoint{1.039772in}{0.821407in}}{\pgfqpoint{1.029321in}{0.817078in}}{\pgfqpoint{1.021617in}{0.809374in}}%
\pgfpathcurveto{\pgfqpoint{1.013912in}{0.801670in}}{\pgfqpoint{1.009583in}{0.791219in}}{\pgfqpoint{1.009583in}{0.780323in}}%
\pgfpathcurveto{\pgfqpoint{1.009583in}{0.769428in}}{\pgfqpoint{1.013912in}{0.758977in}}{\pgfqpoint{1.021617in}{0.751273in}}%
\pgfpathcurveto{\pgfqpoint{1.029321in}{0.743568in}}{\pgfqpoint{1.039772in}{0.739240in}}{\pgfqpoint{1.050667in}{0.739240in}}%
\pgfpathlineto{\pgfqpoint{1.050667in}{0.739240in}}%
\pgfpathclose%
\pgfusepath{stroke}%
\end{pgfscope}%
\begin{pgfscope}%
\pgfpathrectangle{\pgfqpoint{0.688192in}{0.670138in}}{\pgfqpoint{7.111808in}{5.129862in}}%
\pgfusepath{clip}%
\pgfsetbuttcap%
\pgfsetroundjoin%
\pgfsetlinewidth{1.003750pt}%
\definecolor{currentstroke}{rgb}{0.000000,0.000000,0.000000}%
\pgfsetstrokecolor{currentstroke}%
\pgfsetdash{}{0pt}%
\pgfpathmoveto{\pgfqpoint{1.259457in}{0.716052in}}%
\pgfpathcurveto{\pgfqpoint{1.270352in}{0.716052in}}{\pgfqpoint{1.280803in}{0.720381in}}{\pgfqpoint{1.288507in}{0.728085in}}%
\pgfpathcurveto{\pgfqpoint{1.296212in}{0.735789in}}{\pgfqpoint{1.300541in}{0.746240in}}{\pgfqpoint{1.300541in}{0.757136in}}%
\pgfpathcurveto{\pgfqpoint{1.300541in}{0.768031in}}{\pgfqpoint{1.296212in}{0.778482in}}{\pgfqpoint{1.288507in}{0.786187in}}%
\pgfpathcurveto{\pgfqpoint{1.280803in}{0.793891in}}{\pgfqpoint{1.270352in}{0.798220in}}{\pgfqpoint{1.259457in}{0.798220in}}%
\pgfpathcurveto{\pgfqpoint{1.248561in}{0.798220in}}{\pgfqpoint{1.238110in}{0.793891in}}{\pgfqpoint{1.230406in}{0.786187in}}%
\pgfpathcurveto{\pgfqpoint{1.222702in}{0.778482in}}{\pgfqpoint{1.218373in}{0.768031in}}{\pgfqpoint{1.218373in}{0.757136in}}%
\pgfpathcurveto{\pgfqpoint{1.218373in}{0.746240in}}{\pgfqpoint{1.222702in}{0.735789in}}{\pgfqpoint{1.230406in}{0.728085in}}%
\pgfpathcurveto{\pgfqpoint{1.238110in}{0.720381in}}{\pgfqpoint{1.248561in}{0.716052in}}{\pgfqpoint{1.259457in}{0.716052in}}%
\pgfpathlineto{\pgfqpoint{1.259457in}{0.716052in}}%
\pgfpathclose%
\pgfusepath{stroke}%
\end{pgfscope}%
\begin{pgfscope}%
\pgfpathrectangle{\pgfqpoint{0.688192in}{0.670138in}}{\pgfqpoint{7.111808in}{5.129862in}}%
\pgfusepath{clip}%
\pgfsetbuttcap%
\pgfsetroundjoin%
\pgfsetlinewidth{1.003750pt}%
\definecolor{currentstroke}{rgb}{0.000000,0.000000,0.000000}%
\pgfsetstrokecolor{currentstroke}%
\pgfsetdash{}{0pt}%
\pgfpathmoveto{\pgfqpoint{5.928879in}{0.732232in}}%
\pgfpathcurveto{\pgfqpoint{5.939775in}{0.732232in}}{\pgfqpoint{5.950225in}{0.736560in}}{\pgfqpoint{5.957930in}{0.744265in}}%
\pgfpathcurveto{\pgfqpoint{5.965634in}{0.751969in}}{\pgfqpoint{5.969963in}{0.762420in}}{\pgfqpoint{5.969963in}{0.773315in}}%
\pgfpathcurveto{\pgfqpoint{5.969963in}{0.784211in}}{\pgfqpoint{5.965634in}{0.794662in}}{\pgfqpoint{5.957930in}{0.802366in}}%
\pgfpathcurveto{\pgfqpoint{5.950225in}{0.810071in}}{\pgfqpoint{5.939775in}{0.814399in}}{\pgfqpoint{5.928879in}{0.814399in}}%
\pgfpathcurveto{\pgfqpoint{5.917983in}{0.814399in}}{\pgfqpoint{5.907533in}{0.810071in}}{\pgfqpoint{5.899828in}{0.802366in}}%
\pgfpathcurveto{\pgfqpoint{5.892124in}{0.794662in}}{\pgfqpoint{5.887795in}{0.784211in}}{\pgfqpoint{5.887795in}{0.773315in}}%
\pgfpathcurveto{\pgfqpoint{5.887795in}{0.762420in}}{\pgfqpoint{5.892124in}{0.751969in}}{\pgfqpoint{5.899828in}{0.744265in}}%
\pgfpathcurveto{\pgfqpoint{5.907533in}{0.736560in}}{\pgfqpoint{5.917983in}{0.732232in}}{\pgfqpoint{5.928879in}{0.732232in}}%
\pgfpathlineto{\pgfqpoint{5.928879in}{0.732232in}}%
\pgfpathclose%
\pgfusepath{stroke}%
\end{pgfscope}%
\begin{pgfscope}%
\pgfpathrectangle{\pgfqpoint{0.688192in}{0.670138in}}{\pgfqpoint{7.111808in}{5.129862in}}%
\pgfusepath{clip}%
\pgfsetbuttcap%
\pgfsetroundjoin%
\pgfsetlinewidth{1.003750pt}%
\definecolor{currentstroke}{rgb}{0.000000,0.000000,0.000000}%
\pgfsetstrokecolor{currentstroke}%
\pgfsetdash{}{0pt}%
\pgfpathmoveto{\pgfqpoint{1.414589in}{0.710834in}}%
\pgfpathcurveto{\pgfqpoint{1.425484in}{0.710834in}}{\pgfqpoint{1.435935in}{0.715163in}}{\pgfqpoint{1.443639in}{0.722867in}}%
\pgfpathcurveto{\pgfqpoint{1.451344in}{0.730572in}}{\pgfqpoint{1.455672in}{0.741022in}}{\pgfqpoint{1.455672in}{0.751918in}}%
\pgfpathcurveto{\pgfqpoint{1.455672in}{0.762814in}}{\pgfqpoint{1.451344in}{0.773264in}}{\pgfqpoint{1.443639in}{0.780969in}}%
\pgfpathcurveto{\pgfqpoint{1.435935in}{0.788673in}}{\pgfqpoint{1.425484in}{0.793002in}}{\pgfqpoint{1.414589in}{0.793002in}}%
\pgfpathcurveto{\pgfqpoint{1.403693in}{0.793002in}}{\pgfqpoint{1.393242in}{0.788673in}}{\pgfqpoint{1.385538in}{0.780969in}}%
\pgfpathcurveto{\pgfqpoint{1.377834in}{0.773264in}}{\pgfqpoint{1.373505in}{0.762814in}}{\pgfqpoint{1.373505in}{0.751918in}}%
\pgfpathcurveto{\pgfqpoint{1.373505in}{0.741022in}}{\pgfqpoint{1.377834in}{0.730572in}}{\pgfqpoint{1.385538in}{0.722867in}}%
\pgfpathcurveto{\pgfqpoint{1.393242in}{0.715163in}}{\pgfqpoint{1.403693in}{0.710834in}}{\pgfqpoint{1.414589in}{0.710834in}}%
\pgfpathlineto{\pgfqpoint{1.414589in}{0.710834in}}%
\pgfpathclose%
\pgfusepath{stroke}%
\end{pgfscope}%
\begin{pgfscope}%
\pgfpathrectangle{\pgfqpoint{0.688192in}{0.670138in}}{\pgfqpoint{7.111808in}{5.129862in}}%
\pgfusepath{clip}%
\pgfsetbuttcap%
\pgfsetroundjoin%
\pgfsetlinewidth{1.003750pt}%
\definecolor{currentstroke}{rgb}{0.000000,0.000000,0.000000}%
\pgfsetstrokecolor{currentstroke}%
\pgfsetdash{}{0pt}%
\pgfpathmoveto{\pgfqpoint{2.331314in}{0.681791in}}%
\pgfpathcurveto{\pgfqpoint{2.342210in}{0.681791in}}{\pgfqpoint{2.352661in}{0.686120in}}{\pgfqpoint{2.360365in}{0.693824in}}%
\pgfpathcurveto{\pgfqpoint{2.368069in}{0.701529in}}{\pgfqpoint{2.372398in}{0.711979in}}{\pgfqpoint{2.372398in}{0.722875in}}%
\pgfpathcurveto{\pgfqpoint{2.372398in}{0.733771in}}{\pgfqpoint{2.368069in}{0.744221in}}{\pgfqpoint{2.360365in}{0.751926in}}%
\pgfpathcurveto{\pgfqpoint{2.352661in}{0.759630in}}{\pgfqpoint{2.342210in}{0.763959in}}{\pgfqpoint{2.331314in}{0.763959in}}%
\pgfpathcurveto{\pgfqpoint{2.320419in}{0.763959in}}{\pgfqpoint{2.309968in}{0.759630in}}{\pgfqpoint{2.302264in}{0.751926in}}%
\pgfpathcurveto{\pgfqpoint{2.294559in}{0.744221in}}{\pgfqpoint{2.290231in}{0.733771in}}{\pgfqpoint{2.290231in}{0.722875in}}%
\pgfpathcurveto{\pgfqpoint{2.290231in}{0.711979in}}{\pgfqpoint{2.294559in}{0.701529in}}{\pgfqpoint{2.302264in}{0.693824in}}%
\pgfpathcurveto{\pgfqpoint{2.309968in}{0.686120in}}{\pgfqpoint{2.320419in}{0.681791in}}{\pgfqpoint{2.331314in}{0.681791in}}%
\pgfpathlineto{\pgfqpoint{2.331314in}{0.681791in}}%
\pgfpathclose%
\pgfusepath{stroke}%
\end{pgfscope}%
\begin{pgfscope}%
\pgfpathrectangle{\pgfqpoint{0.688192in}{0.670138in}}{\pgfqpoint{7.111808in}{5.129862in}}%
\pgfusepath{clip}%
\pgfsetbuttcap%
\pgfsetroundjoin%
\pgfsetlinewidth{1.003750pt}%
\definecolor{currentstroke}{rgb}{0.000000,0.000000,0.000000}%
\pgfsetstrokecolor{currentstroke}%
\pgfsetdash{}{0pt}%
\pgfpathmoveto{\pgfqpoint{0.782559in}{1.071983in}}%
\pgfpathcurveto{\pgfqpoint{0.793455in}{1.071983in}}{\pgfqpoint{0.803905in}{1.076312in}}{\pgfqpoint{0.811610in}{1.084017in}}%
\pgfpathcurveto{\pgfqpoint{0.819314in}{1.091721in}}{\pgfqpoint{0.823643in}{1.102172in}}{\pgfqpoint{0.823643in}{1.113067in}}%
\pgfpathcurveto{\pgfqpoint{0.823643in}{1.123963in}}{\pgfqpoint{0.819314in}{1.134414in}}{\pgfqpoint{0.811610in}{1.142118in}}%
\pgfpathcurveto{\pgfqpoint{0.803905in}{1.149822in}}{\pgfqpoint{0.793455in}{1.154151in}}{\pgfqpoint{0.782559in}{1.154151in}}%
\pgfpathcurveto{\pgfqpoint{0.771663in}{1.154151in}}{\pgfqpoint{0.761213in}{1.149822in}}{\pgfqpoint{0.753508in}{1.142118in}}%
\pgfpathcurveto{\pgfqpoint{0.745804in}{1.134414in}}{\pgfqpoint{0.741475in}{1.123963in}}{\pgfqpoint{0.741475in}{1.113067in}}%
\pgfpathcurveto{\pgfqpoint{0.741475in}{1.102172in}}{\pgfqpoint{0.745804in}{1.091721in}}{\pgfqpoint{0.753508in}{1.084017in}}%
\pgfpathcurveto{\pgfqpoint{0.761213in}{1.076312in}}{\pgfqpoint{0.771663in}{1.071983in}}{\pgfqpoint{0.782559in}{1.071983in}}%
\pgfpathlineto{\pgfqpoint{0.782559in}{1.071983in}}%
\pgfpathclose%
\pgfusepath{stroke}%
\end{pgfscope}%
\begin{pgfscope}%
\pgfpathrectangle{\pgfqpoint{0.688192in}{0.670138in}}{\pgfqpoint{7.111808in}{5.129862in}}%
\pgfusepath{clip}%
\pgfsetbuttcap%
\pgfsetroundjoin%
\pgfsetlinewidth{1.003750pt}%
\definecolor{currentstroke}{rgb}{0.000000,0.000000,0.000000}%
\pgfsetstrokecolor{currentstroke}%
\pgfsetdash{}{0pt}%
\pgfpathmoveto{\pgfqpoint{5.507614in}{0.631795in}}%
\pgfpathcurveto{\pgfqpoint{5.518510in}{0.631795in}}{\pgfqpoint{5.528960in}{0.636124in}}{\pgfqpoint{5.536665in}{0.643829in}}%
\pgfpathcurveto{\pgfqpoint{5.544369in}{0.651533in}}{\pgfqpoint{5.548698in}{0.661984in}}{\pgfqpoint{5.548698in}{0.672879in}}%
\pgfpathcurveto{\pgfqpoint{5.548698in}{0.683775in}}{\pgfqpoint{5.544369in}{0.694226in}}{\pgfqpoint{5.536665in}{0.701930in}}%
\pgfpathcurveto{\pgfqpoint{5.528960in}{0.709634in}}{\pgfqpoint{5.518510in}{0.713963in}}{\pgfqpoint{5.507614in}{0.713963in}}%
\pgfpathcurveto{\pgfqpoint{5.496718in}{0.713963in}}{\pgfqpoint{5.486268in}{0.709634in}}{\pgfqpoint{5.478563in}{0.701930in}}%
\pgfpathcurveto{\pgfqpoint{5.470859in}{0.694226in}}{\pgfqpoint{5.466530in}{0.683775in}}{\pgfqpoint{5.466530in}{0.672879in}}%
\pgfpathcurveto{\pgfqpoint{5.466530in}{0.661984in}}{\pgfqpoint{5.470859in}{0.651533in}}{\pgfqpoint{5.478563in}{0.643829in}}%
\pgfpathcurveto{\pgfqpoint{5.486268in}{0.636124in}}{\pgfqpoint{5.496718in}{0.631795in}}{\pgfqpoint{5.507614in}{0.631795in}}%
\pgfusepath{stroke}%
\end{pgfscope}%
\begin{pgfscope}%
\pgfpathrectangle{\pgfqpoint{0.688192in}{0.670138in}}{\pgfqpoint{7.111808in}{5.129862in}}%
\pgfusepath{clip}%
\pgfsetbuttcap%
\pgfsetroundjoin%
\pgfsetlinewidth{1.003750pt}%
\definecolor{currentstroke}{rgb}{0.000000,0.000000,0.000000}%
\pgfsetstrokecolor{currentstroke}%
\pgfsetdash{}{0pt}%
\pgfpathmoveto{\pgfqpoint{1.759569in}{0.698796in}}%
\pgfpathcurveto{\pgfqpoint{1.770464in}{0.698796in}}{\pgfqpoint{1.780915in}{0.703125in}}{\pgfqpoint{1.788619in}{0.710830in}}%
\pgfpathcurveto{\pgfqpoint{1.796324in}{0.718534in}}{\pgfqpoint{1.800652in}{0.728985in}}{\pgfqpoint{1.800652in}{0.739880in}}%
\pgfpathcurveto{\pgfqpoint{1.800652in}{0.750776in}}{\pgfqpoint{1.796324in}{0.761227in}}{\pgfqpoint{1.788619in}{0.768931in}}%
\pgfpathcurveto{\pgfqpoint{1.780915in}{0.776635in}}{\pgfqpoint{1.770464in}{0.780964in}}{\pgfqpoint{1.759569in}{0.780964in}}%
\pgfpathcurveto{\pgfqpoint{1.748673in}{0.780964in}}{\pgfqpoint{1.738222in}{0.776635in}}{\pgfqpoint{1.730518in}{0.768931in}}%
\pgfpathcurveto{\pgfqpoint{1.722814in}{0.761227in}}{\pgfqpoint{1.718485in}{0.750776in}}{\pgfqpoint{1.718485in}{0.739880in}}%
\pgfpathcurveto{\pgfqpoint{1.718485in}{0.728985in}}{\pgfqpoint{1.722814in}{0.718534in}}{\pgfqpoint{1.730518in}{0.710830in}}%
\pgfpathcurveto{\pgfqpoint{1.738222in}{0.703125in}}{\pgfqpoint{1.748673in}{0.698796in}}{\pgfqpoint{1.759569in}{0.698796in}}%
\pgfpathlineto{\pgfqpoint{1.759569in}{0.698796in}}%
\pgfpathclose%
\pgfusepath{stroke}%
\end{pgfscope}%
\begin{pgfscope}%
\pgfpathrectangle{\pgfqpoint{0.688192in}{0.670138in}}{\pgfqpoint{7.111808in}{5.129862in}}%
\pgfusepath{clip}%
\pgfsetbuttcap%
\pgfsetroundjoin%
\pgfsetlinewidth{1.003750pt}%
\definecolor{currentstroke}{rgb}{0.000000,0.000000,0.000000}%
\pgfsetstrokecolor{currentstroke}%
\pgfsetdash{}{0pt}%
\pgfpathmoveto{\pgfqpoint{1.045306in}{0.741536in}}%
\pgfpathcurveto{\pgfqpoint{1.056202in}{0.741536in}}{\pgfqpoint{1.066653in}{0.745865in}}{\pgfqpoint{1.074357in}{0.753569in}}%
\pgfpathcurveto{\pgfqpoint{1.082061in}{0.761274in}}{\pgfqpoint{1.086390in}{0.771725in}}{\pgfqpoint{1.086390in}{0.782620in}}%
\pgfpathcurveto{\pgfqpoint{1.086390in}{0.793516in}}{\pgfqpoint{1.082061in}{0.803967in}}{\pgfqpoint{1.074357in}{0.811671in}}%
\pgfpathcurveto{\pgfqpoint{1.066653in}{0.819375in}}{\pgfqpoint{1.056202in}{0.823704in}}{\pgfqpoint{1.045306in}{0.823704in}}%
\pgfpathcurveto{\pgfqpoint{1.034411in}{0.823704in}}{\pgfqpoint{1.023960in}{0.819375in}}{\pgfqpoint{1.016256in}{0.811671in}}%
\pgfpathcurveto{\pgfqpoint{1.008551in}{0.803967in}}{\pgfqpoint{1.004222in}{0.793516in}}{\pgfqpoint{1.004222in}{0.782620in}}%
\pgfpathcurveto{\pgfqpoint{1.004222in}{0.771725in}}{\pgfqpoint{1.008551in}{0.761274in}}{\pgfqpoint{1.016256in}{0.753569in}}%
\pgfpathcurveto{\pgfqpoint{1.023960in}{0.745865in}}{\pgfqpoint{1.034411in}{0.741536in}}{\pgfqpoint{1.045306in}{0.741536in}}%
\pgfpathlineto{\pgfqpoint{1.045306in}{0.741536in}}%
\pgfpathclose%
\pgfusepath{stroke}%
\end{pgfscope}%
\begin{pgfscope}%
\pgfpathrectangle{\pgfqpoint{0.688192in}{0.670138in}}{\pgfqpoint{7.111808in}{5.129862in}}%
\pgfusepath{clip}%
\pgfsetbuttcap%
\pgfsetroundjoin%
\pgfsetlinewidth{1.003750pt}%
\definecolor{currentstroke}{rgb}{0.000000,0.000000,0.000000}%
\pgfsetstrokecolor{currentstroke}%
\pgfsetdash{}{0pt}%
\pgfpathmoveto{\pgfqpoint{5.556869in}{1.422817in}}%
\pgfpathcurveto{\pgfqpoint{5.567765in}{1.422817in}}{\pgfqpoint{5.578216in}{1.427145in}}{\pgfqpoint{5.585920in}{1.434850in}}%
\pgfpathcurveto{\pgfqpoint{5.593624in}{1.442554in}}{\pgfqpoint{5.597953in}{1.453005in}}{\pgfqpoint{5.597953in}{1.463900in}}%
\pgfpathcurveto{\pgfqpoint{5.597953in}{1.474796in}}{\pgfqpoint{5.593624in}{1.485247in}}{\pgfqpoint{5.585920in}{1.492951in}}%
\pgfpathcurveto{\pgfqpoint{5.578216in}{1.500655in}}{\pgfqpoint{5.567765in}{1.504984in}}{\pgfqpoint{5.556869in}{1.504984in}}%
\pgfpathcurveto{\pgfqpoint{5.545974in}{1.504984in}}{\pgfqpoint{5.535523in}{1.500655in}}{\pgfqpoint{5.527819in}{1.492951in}}%
\pgfpathcurveto{\pgfqpoint{5.520114in}{1.485247in}}{\pgfqpoint{5.515785in}{1.474796in}}{\pgfqpoint{5.515785in}{1.463900in}}%
\pgfpathcurveto{\pgfqpoint{5.515785in}{1.453005in}}{\pgfqpoint{5.520114in}{1.442554in}}{\pgfqpoint{5.527819in}{1.434850in}}%
\pgfpathcurveto{\pgfqpoint{5.535523in}{1.427145in}}{\pgfqpoint{5.545974in}{1.422817in}}{\pgfqpoint{5.556869in}{1.422817in}}%
\pgfpathlineto{\pgfqpoint{5.556869in}{1.422817in}}%
\pgfpathclose%
\pgfusepath{stroke}%
\end{pgfscope}%
\begin{pgfscope}%
\pgfpathrectangle{\pgfqpoint{0.688192in}{0.670138in}}{\pgfqpoint{7.111808in}{5.129862in}}%
\pgfusepath{clip}%
\pgfsetbuttcap%
\pgfsetroundjoin%
\pgfsetlinewidth{1.003750pt}%
\definecolor{currentstroke}{rgb}{0.000000,0.000000,0.000000}%
\pgfsetstrokecolor{currentstroke}%
\pgfsetdash{}{0pt}%
\pgfpathmoveto{\pgfqpoint{1.759569in}{0.698796in}}%
\pgfpathcurveto{\pgfqpoint{1.770464in}{0.698796in}}{\pgfqpoint{1.780915in}{0.703125in}}{\pgfqpoint{1.788619in}{0.710830in}}%
\pgfpathcurveto{\pgfqpoint{1.796324in}{0.718534in}}{\pgfqpoint{1.800652in}{0.728985in}}{\pgfqpoint{1.800652in}{0.739880in}}%
\pgfpathcurveto{\pgfqpoint{1.800652in}{0.750776in}}{\pgfqpoint{1.796324in}{0.761227in}}{\pgfqpoint{1.788619in}{0.768931in}}%
\pgfpathcurveto{\pgfqpoint{1.780915in}{0.776635in}}{\pgfqpoint{1.770464in}{0.780964in}}{\pgfqpoint{1.759569in}{0.780964in}}%
\pgfpathcurveto{\pgfqpoint{1.748673in}{0.780964in}}{\pgfqpoint{1.738222in}{0.776635in}}{\pgfqpoint{1.730518in}{0.768931in}}%
\pgfpathcurveto{\pgfqpoint{1.722814in}{0.761227in}}{\pgfqpoint{1.718485in}{0.750776in}}{\pgfqpoint{1.718485in}{0.739880in}}%
\pgfpathcurveto{\pgfqpoint{1.718485in}{0.728985in}}{\pgfqpoint{1.722814in}{0.718534in}}{\pgfqpoint{1.730518in}{0.710830in}}%
\pgfpathcurveto{\pgfqpoint{1.738222in}{0.703125in}}{\pgfqpoint{1.748673in}{0.698796in}}{\pgfqpoint{1.759569in}{0.698796in}}%
\pgfpathlineto{\pgfqpoint{1.759569in}{0.698796in}}%
\pgfpathclose%
\pgfusepath{stroke}%
\end{pgfscope}%
\begin{pgfscope}%
\pgfpathrectangle{\pgfqpoint{0.688192in}{0.670138in}}{\pgfqpoint{7.111808in}{5.129862in}}%
\pgfusepath{clip}%
\pgfsetbuttcap%
\pgfsetroundjoin%
\pgfsetlinewidth{1.003750pt}%
\definecolor{currentstroke}{rgb}{0.000000,0.000000,0.000000}%
\pgfsetstrokecolor{currentstroke}%
\pgfsetdash{}{0pt}%
\pgfpathmoveto{\pgfqpoint{6.235447in}{1.342252in}}%
\pgfpathcurveto{\pgfqpoint{6.246343in}{1.342252in}}{\pgfqpoint{6.256793in}{1.346580in}}{\pgfqpoint{6.264498in}{1.354285in}}%
\pgfpathcurveto{\pgfqpoint{6.272202in}{1.361989in}}{\pgfqpoint{6.276531in}{1.372440in}}{\pgfqpoint{6.276531in}{1.383336in}}%
\pgfpathcurveto{\pgfqpoint{6.276531in}{1.394231in}}{\pgfqpoint{6.272202in}{1.404682in}}{\pgfqpoint{6.264498in}{1.412386in}}%
\pgfpathcurveto{\pgfqpoint{6.256793in}{1.420091in}}{\pgfqpoint{6.246343in}{1.424419in}}{\pgfqpoint{6.235447in}{1.424419in}}%
\pgfpathcurveto{\pgfqpoint{6.224551in}{1.424419in}}{\pgfqpoint{6.214101in}{1.420091in}}{\pgfqpoint{6.206396in}{1.412386in}}%
\pgfpathcurveto{\pgfqpoint{6.198692in}{1.404682in}}{\pgfqpoint{6.194363in}{1.394231in}}{\pgfqpoint{6.194363in}{1.383336in}}%
\pgfpathcurveto{\pgfqpoint{6.194363in}{1.372440in}}{\pgfqpoint{6.198692in}{1.361989in}}{\pgfqpoint{6.206396in}{1.354285in}}%
\pgfpathcurveto{\pgfqpoint{6.214101in}{1.346580in}}{\pgfqpoint{6.224551in}{1.342252in}}{\pgfqpoint{6.235447in}{1.342252in}}%
\pgfpathlineto{\pgfqpoint{6.235447in}{1.342252in}}%
\pgfpathclose%
\pgfusepath{stroke}%
\end{pgfscope}%
\begin{pgfscope}%
\pgfpathrectangle{\pgfqpoint{0.688192in}{0.670138in}}{\pgfqpoint{7.111808in}{5.129862in}}%
\pgfusepath{clip}%
\pgfsetbuttcap%
\pgfsetroundjoin%
\pgfsetlinewidth{1.003750pt}%
\definecolor{currentstroke}{rgb}{0.000000,0.000000,0.000000}%
\pgfsetstrokecolor{currentstroke}%
\pgfsetdash{}{0pt}%
\pgfpathmoveto{\pgfqpoint{0.886861in}{0.869726in}}%
\pgfpathcurveto{\pgfqpoint{0.897756in}{0.869726in}}{\pgfqpoint{0.908207in}{0.874055in}}{\pgfqpoint{0.915911in}{0.881759in}}%
\pgfpathcurveto{\pgfqpoint{0.923616in}{0.889463in}}{\pgfqpoint{0.927945in}{0.899914in}}{\pgfqpoint{0.927945in}{0.910810in}}%
\pgfpathcurveto{\pgfqpoint{0.927945in}{0.921705in}}{\pgfqpoint{0.923616in}{0.932156in}}{\pgfqpoint{0.915911in}{0.939861in}}%
\pgfpathcurveto{\pgfqpoint{0.908207in}{0.947565in}}{\pgfqpoint{0.897756in}{0.951894in}}{\pgfqpoint{0.886861in}{0.951894in}}%
\pgfpathcurveto{\pgfqpoint{0.875965in}{0.951894in}}{\pgfqpoint{0.865514in}{0.947565in}}{\pgfqpoint{0.857810in}{0.939861in}}%
\pgfpathcurveto{\pgfqpoint{0.850106in}{0.932156in}}{\pgfqpoint{0.845777in}{0.921705in}}{\pgfqpoint{0.845777in}{0.910810in}}%
\pgfpathcurveto{\pgfqpoint{0.845777in}{0.899914in}}{\pgfqpoint{0.850106in}{0.889463in}}{\pgfqpoint{0.857810in}{0.881759in}}%
\pgfpathcurveto{\pgfqpoint{0.865514in}{0.874055in}}{\pgfqpoint{0.875965in}{0.869726in}}{\pgfqpoint{0.886861in}{0.869726in}}%
\pgfpathlineto{\pgfqpoint{0.886861in}{0.869726in}}%
\pgfpathclose%
\pgfusepath{stroke}%
\end{pgfscope}%
\begin{pgfscope}%
\pgfpathrectangle{\pgfqpoint{0.688192in}{0.670138in}}{\pgfqpoint{7.111808in}{5.129862in}}%
\pgfusepath{clip}%
\pgfsetbuttcap%
\pgfsetroundjoin%
\pgfsetlinewidth{1.003750pt}%
\definecolor{currentstroke}{rgb}{0.000000,0.000000,0.000000}%
\pgfsetstrokecolor{currentstroke}%
\pgfsetdash{}{0pt}%
\pgfpathmoveto{\pgfqpoint{1.120660in}{0.719620in}}%
\pgfpathcurveto{\pgfqpoint{1.131555in}{0.719620in}}{\pgfqpoint{1.142006in}{0.723949in}}{\pgfqpoint{1.149710in}{0.731653in}}%
\pgfpathcurveto{\pgfqpoint{1.157415in}{0.739358in}}{\pgfqpoint{1.161744in}{0.749809in}}{\pgfqpoint{1.161744in}{0.760704in}}%
\pgfpathcurveto{\pgfqpoint{1.161744in}{0.771600in}}{\pgfqpoint{1.157415in}{0.782051in}}{\pgfqpoint{1.149710in}{0.789755in}}%
\pgfpathcurveto{\pgfqpoint{1.142006in}{0.797459in}}{\pgfqpoint{1.131555in}{0.801788in}}{\pgfqpoint{1.120660in}{0.801788in}}%
\pgfpathcurveto{\pgfqpoint{1.109764in}{0.801788in}}{\pgfqpoint{1.099313in}{0.797459in}}{\pgfqpoint{1.091609in}{0.789755in}}%
\pgfpathcurveto{\pgfqpoint{1.083905in}{0.782051in}}{\pgfqpoint{1.079576in}{0.771600in}}{\pgfqpoint{1.079576in}{0.760704in}}%
\pgfpathcurveto{\pgfqpoint{1.079576in}{0.749809in}}{\pgfqpoint{1.083905in}{0.739358in}}{\pgfqpoint{1.091609in}{0.731653in}}%
\pgfpathcurveto{\pgfqpoint{1.099313in}{0.723949in}}{\pgfqpoint{1.109764in}{0.719620in}}{\pgfqpoint{1.120660in}{0.719620in}}%
\pgfpathlineto{\pgfqpoint{1.120660in}{0.719620in}}%
\pgfpathclose%
\pgfusepath{stroke}%
\end{pgfscope}%
\begin{pgfscope}%
\pgfpathrectangle{\pgfqpoint{0.688192in}{0.670138in}}{\pgfqpoint{7.111808in}{5.129862in}}%
\pgfusepath{clip}%
\pgfsetbuttcap%
\pgfsetroundjoin%
\pgfsetlinewidth{1.003750pt}%
\definecolor{currentstroke}{rgb}{0.000000,0.000000,0.000000}%
\pgfsetstrokecolor{currentstroke}%
\pgfsetdash{}{0pt}%
\pgfpathmoveto{\pgfqpoint{6.103052in}{1.273456in}}%
\pgfpathcurveto{\pgfqpoint{6.113948in}{1.273456in}}{\pgfqpoint{6.124398in}{1.277785in}}{\pgfqpoint{6.132103in}{1.285489in}}%
\pgfpathcurveto{\pgfqpoint{6.139807in}{1.293194in}}{\pgfqpoint{6.144136in}{1.303644in}}{\pgfqpoint{6.144136in}{1.314540in}}%
\pgfpathcurveto{\pgfqpoint{6.144136in}{1.325435in}}{\pgfqpoint{6.139807in}{1.335886in}}{\pgfqpoint{6.132103in}{1.343591in}}%
\pgfpathcurveto{\pgfqpoint{6.124398in}{1.351295in}}{\pgfqpoint{6.113948in}{1.355624in}}{\pgfqpoint{6.103052in}{1.355624in}}%
\pgfpathcurveto{\pgfqpoint{6.092157in}{1.355624in}}{\pgfqpoint{6.081706in}{1.351295in}}{\pgfqpoint{6.074001in}{1.343591in}}%
\pgfpathcurveto{\pgfqpoint{6.066297in}{1.335886in}}{\pgfqpoint{6.061968in}{1.325435in}}{\pgfqpoint{6.061968in}{1.314540in}}%
\pgfpathcurveto{\pgfqpoint{6.061968in}{1.303644in}}{\pgfqpoint{6.066297in}{1.293194in}}{\pgfqpoint{6.074001in}{1.285489in}}%
\pgfpathcurveto{\pgfqpoint{6.081706in}{1.277785in}}{\pgfqpoint{6.092157in}{1.273456in}}{\pgfqpoint{6.103052in}{1.273456in}}%
\pgfpathlineto{\pgfqpoint{6.103052in}{1.273456in}}%
\pgfpathclose%
\pgfusepath{stroke}%
\end{pgfscope}%
\begin{pgfscope}%
\pgfpathrectangle{\pgfqpoint{0.688192in}{0.670138in}}{\pgfqpoint{7.111808in}{5.129862in}}%
\pgfusepath{clip}%
\pgfsetbuttcap%
\pgfsetroundjoin%
\pgfsetlinewidth{1.003750pt}%
\definecolor{currentstroke}{rgb}{0.000000,0.000000,0.000000}%
\pgfsetstrokecolor{currentstroke}%
\pgfsetdash{}{0pt}%
\pgfpathmoveto{\pgfqpoint{1.759569in}{0.698796in}}%
\pgfpathcurveto{\pgfqpoint{1.770464in}{0.698796in}}{\pgfqpoint{1.780915in}{0.703125in}}{\pgfqpoint{1.788619in}{0.710830in}}%
\pgfpathcurveto{\pgfqpoint{1.796324in}{0.718534in}}{\pgfqpoint{1.800652in}{0.728985in}}{\pgfqpoint{1.800652in}{0.739880in}}%
\pgfpathcurveto{\pgfqpoint{1.800652in}{0.750776in}}{\pgfqpoint{1.796324in}{0.761227in}}{\pgfqpoint{1.788619in}{0.768931in}}%
\pgfpathcurveto{\pgfqpoint{1.780915in}{0.776635in}}{\pgfqpoint{1.770464in}{0.780964in}}{\pgfqpoint{1.759569in}{0.780964in}}%
\pgfpathcurveto{\pgfqpoint{1.748673in}{0.780964in}}{\pgfqpoint{1.738222in}{0.776635in}}{\pgfqpoint{1.730518in}{0.768931in}}%
\pgfpathcurveto{\pgfqpoint{1.722814in}{0.761227in}}{\pgfqpoint{1.718485in}{0.750776in}}{\pgfqpoint{1.718485in}{0.739880in}}%
\pgfpathcurveto{\pgfqpoint{1.718485in}{0.728985in}}{\pgfqpoint{1.722814in}{0.718534in}}{\pgfqpoint{1.730518in}{0.710830in}}%
\pgfpathcurveto{\pgfqpoint{1.738222in}{0.703125in}}{\pgfqpoint{1.748673in}{0.698796in}}{\pgfqpoint{1.759569in}{0.698796in}}%
\pgfpathlineto{\pgfqpoint{1.759569in}{0.698796in}}%
\pgfpathclose%
\pgfusepath{stroke}%
\end{pgfscope}%
\begin{pgfscope}%
\pgfpathrectangle{\pgfqpoint{0.688192in}{0.670138in}}{\pgfqpoint{7.111808in}{5.129862in}}%
\pgfusepath{clip}%
\pgfsetbuttcap%
\pgfsetroundjoin%
\pgfsetlinewidth{1.003750pt}%
\definecolor{currentstroke}{rgb}{0.000000,0.000000,0.000000}%
\pgfsetstrokecolor{currentstroke}%
\pgfsetdash{}{0pt}%
\pgfpathmoveto{\pgfqpoint{0.901363in}{0.856548in}}%
\pgfpathcurveto{\pgfqpoint{0.912259in}{0.856548in}}{\pgfqpoint{0.922710in}{0.860877in}}{\pgfqpoint{0.930414in}{0.868582in}}%
\pgfpathcurveto{\pgfqpoint{0.938118in}{0.876286in}}{\pgfqpoint{0.942447in}{0.886737in}}{\pgfqpoint{0.942447in}{0.897632in}}%
\pgfpathcurveto{\pgfqpoint{0.942447in}{0.908528in}}{\pgfqpoint{0.938118in}{0.918979in}}{\pgfqpoint{0.930414in}{0.926683in}}%
\pgfpathcurveto{\pgfqpoint{0.922710in}{0.934387in}}{\pgfqpoint{0.912259in}{0.938716in}}{\pgfqpoint{0.901363in}{0.938716in}}%
\pgfpathcurveto{\pgfqpoint{0.890468in}{0.938716in}}{\pgfqpoint{0.880017in}{0.934387in}}{\pgfqpoint{0.872313in}{0.926683in}}%
\pgfpathcurveto{\pgfqpoint{0.864608in}{0.918979in}}{\pgfqpoint{0.860280in}{0.908528in}}{\pgfqpoint{0.860280in}{0.897632in}}%
\pgfpathcurveto{\pgfqpoint{0.860280in}{0.886737in}}{\pgfqpoint{0.864608in}{0.876286in}}{\pgfqpoint{0.872313in}{0.868582in}}%
\pgfpathcurveto{\pgfqpoint{0.880017in}{0.860877in}}{\pgfqpoint{0.890468in}{0.856548in}}{\pgfqpoint{0.901363in}{0.856548in}}%
\pgfpathlineto{\pgfqpoint{0.901363in}{0.856548in}}%
\pgfpathclose%
\pgfusepath{stroke}%
\end{pgfscope}%
\begin{pgfscope}%
\pgfpathrectangle{\pgfqpoint{0.688192in}{0.670138in}}{\pgfqpoint{7.111808in}{5.129862in}}%
\pgfusepath{clip}%
\pgfsetbuttcap%
\pgfsetroundjoin%
\pgfsetlinewidth{1.003750pt}%
\definecolor{currentstroke}{rgb}{0.000000,0.000000,0.000000}%
\pgfsetstrokecolor{currentstroke}%
\pgfsetdash{}{0pt}%
\pgfpathmoveto{\pgfqpoint{0.789763in}{1.038554in}}%
\pgfpathcurveto{\pgfqpoint{0.800659in}{1.038554in}}{\pgfqpoint{0.811109in}{1.042883in}}{\pgfqpoint{0.818814in}{1.050587in}}%
\pgfpathcurveto{\pgfqpoint{0.826518in}{1.058291in}}{\pgfqpoint{0.830847in}{1.068742in}}{\pgfqpoint{0.830847in}{1.079638in}}%
\pgfpathcurveto{\pgfqpoint{0.830847in}{1.090533in}}{\pgfqpoint{0.826518in}{1.100984in}}{\pgfqpoint{0.818814in}{1.108688in}}%
\pgfpathcurveto{\pgfqpoint{0.811109in}{1.116393in}}{\pgfqpoint{0.800659in}{1.120722in}}{\pgfqpoint{0.789763in}{1.120722in}}%
\pgfpathcurveto{\pgfqpoint{0.778868in}{1.120722in}}{\pgfqpoint{0.768417in}{1.116393in}}{\pgfqpoint{0.760712in}{1.108688in}}%
\pgfpathcurveto{\pgfqpoint{0.753008in}{1.100984in}}{\pgfqpoint{0.748679in}{1.090533in}}{\pgfqpoint{0.748679in}{1.079638in}}%
\pgfpathcurveto{\pgfqpoint{0.748679in}{1.068742in}}{\pgfqpoint{0.753008in}{1.058291in}}{\pgfqpoint{0.760712in}{1.050587in}}%
\pgfpathcurveto{\pgfqpoint{0.768417in}{1.042883in}}{\pgfqpoint{0.778868in}{1.038554in}}{\pgfqpoint{0.789763in}{1.038554in}}%
\pgfpathlineto{\pgfqpoint{0.789763in}{1.038554in}}%
\pgfpathclose%
\pgfusepath{stroke}%
\end{pgfscope}%
\begin{pgfscope}%
\pgfpathrectangle{\pgfqpoint{0.688192in}{0.670138in}}{\pgfqpoint{7.111808in}{5.129862in}}%
\pgfusepath{clip}%
\pgfsetbuttcap%
\pgfsetroundjoin%
\pgfsetlinewidth{1.003750pt}%
\definecolor{currentstroke}{rgb}{0.000000,0.000000,0.000000}%
\pgfsetstrokecolor{currentstroke}%
\pgfsetdash{}{0pt}%
\pgfpathmoveto{\pgfqpoint{1.050667in}{0.739240in}}%
\pgfpathcurveto{\pgfqpoint{1.061563in}{0.739240in}}{\pgfqpoint{1.072014in}{0.743568in}}{\pgfqpoint{1.079718in}{0.751273in}}%
\pgfpathcurveto{\pgfqpoint{1.087422in}{0.758977in}}{\pgfqpoint{1.091751in}{0.769428in}}{\pgfqpoint{1.091751in}{0.780323in}}%
\pgfpathcurveto{\pgfqpoint{1.091751in}{0.791219in}}{\pgfqpoint{1.087422in}{0.801670in}}{\pgfqpoint{1.079718in}{0.809374in}}%
\pgfpathcurveto{\pgfqpoint{1.072014in}{0.817078in}}{\pgfqpoint{1.061563in}{0.821407in}}{\pgfqpoint{1.050667in}{0.821407in}}%
\pgfpathcurveto{\pgfqpoint{1.039772in}{0.821407in}}{\pgfqpoint{1.029321in}{0.817078in}}{\pgfqpoint{1.021617in}{0.809374in}}%
\pgfpathcurveto{\pgfqpoint{1.013912in}{0.801670in}}{\pgfqpoint{1.009583in}{0.791219in}}{\pgfqpoint{1.009583in}{0.780323in}}%
\pgfpathcurveto{\pgfqpoint{1.009583in}{0.769428in}}{\pgfqpoint{1.013912in}{0.758977in}}{\pgfqpoint{1.021617in}{0.751273in}}%
\pgfpathcurveto{\pgfqpoint{1.029321in}{0.743568in}}{\pgfqpoint{1.039772in}{0.739240in}}{\pgfqpoint{1.050667in}{0.739240in}}%
\pgfpathlineto{\pgfqpoint{1.050667in}{0.739240in}}%
\pgfpathclose%
\pgfusepath{stroke}%
\end{pgfscope}%
\begin{pgfscope}%
\pgfpathrectangle{\pgfqpoint{0.688192in}{0.670138in}}{\pgfqpoint{7.111808in}{5.129862in}}%
\pgfusepath{clip}%
\pgfsetbuttcap%
\pgfsetroundjoin%
\pgfsetlinewidth{1.003750pt}%
\definecolor{currentstroke}{rgb}{0.000000,0.000000,0.000000}%
\pgfsetstrokecolor{currentstroke}%
\pgfsetdash{}{0pt}%
\pgfpathmoveto{\pgfqpoint{0.732629in}{1.453719in}}%
\pgfpathcurveto{\pgfqpoint{0.743524in}{1.453719in}}{\pgfqpoint{0.753975in}{1.458048in}}{\pgfqpoint{0.761679in}{1.465752in}}%
\pgfpathcurveto{\pgfqpoint{0.769384in}{1.473457in}}{\pgfqpoint{0.773713in}{1.483907in}}{\pgfqpoint{0.773713in}{1.494803in}}%
\pgfpathcurveto{\pgfqpoint{0.773713in}{1.505698in}}{\pgfqpoint{0.769384in}{1.516149in}}{\pgfqpoint{0.761679in}{1.523854in}}%
\pgfpathcurveto{\pgfqpoint{0.753975in}{1.531558in}}{\pgfqpoint{0.743524in}{1.535887in}}{\pgfqpoint{0.732629in}{1.535887in}}%
\pgfpathcurveto{\pgfqpoint{0.721733in}{1.535887in}}{\pgfqpoint{0.711282in}{1.531558in}}{\pgfqpoint{0.703578in}{1.523854in}}%
\pgfpathcurveto{\pgfqpoint{0.695874in}{1.516149in}}{\pgfqpoint{0.691545in}{1.505698in}}{\pgfqpoint{0.691545in}{1.494803in}}%
\pgfpathcurveto{\pgfqpoint{0.691545in}{1.483907in}}{\pgfqpoint{0.695874in}{1.473457in}}{\pgfqpoint{0.703578in}{1.465752in}}%
\pgfpathcurveto{\pgfqpoint{0.711282in}{1.458048in}}{\pgfqpoint{0.721733in}{1.453719in}}{\pgfqpoint{0.732629in}{1.453719in}}%
\pgfpathlineto{\pgfqpoint{0.732629in}{1.453719in}}%
\pgfpathclose%
\pgfusepath{stroke}%
\end{pgfscope}%
\begin{pgfscope}%
\pgfpathrectangle{\pgfqpoint{0.688192in}{0.670138in}}{\pgfqpoint{7.111808in}{5.129862in}}%
\pgfusepath{clip}%
\pgfsetbuttcap%
\pgfsetroundjoin%
\pgfsetlinewidth{1.003750pt}%
\definecolor{currentstroke}{rgb}{0.000000,0.000000,0.000000}%
\pgfsetstrokecolor{currentstroke}%
\pgfsetdash{}{0pt}%
\pgfpathmoveto{\pgfqpoint{3.523301in}{0.651040in}}%
\pgfpathcurveto{\pgfqpoint{3.534196in}{0.651040in}}{\pgfqpoint{3.544647in}{0.655369in}}{\pgfqpoint{3.552351in}{0.663073in}}%
\pgfpathcurveto{\pgfqpoint{3.560056in}{0.670778in}}{\pgfqpoint{3.564385in}{0.681229in}}{\pgfqpoint{3.564385in}{0.692124in}}%
\pgfpathcurveto{\pgfqpoint{3.564385in}{0.703020in}}{\pgfqpoint{3.560056in}{0.713470in}}{\pgfqpoint{3.552351in}{0.721175in}}%
\pgfpathcurveto{\pgfqpoint{3.544647in}{0.728879in}}{\pgfqpoint{3.534196in}{0.733208in}}{\pgfqpoint{3.523301in}{0.733208in}}%
\pgfpathcurveto{\pgfqpoint{3.512405in}{0.733208in}}{\pgfqpoint{3.501954in}{0.728879in}}{\pgfqpoint{3.494250in}{0.721175in}}%
\pgfpathcurveto{\pgfqpoint{3.486546in}{0.713470in}}{\pgfqpoint{3.482217in}{0.703020in}}{\pgfqpoint{3.482217in}{0.692124in}}%
\pgfpathcurveto{\pgfqpoint{3.482217in}{0.681229in}}{\pgfqpoint{3.486546in}{0.670778in}}{\pgfqpoint{3.494250in}{0.663073in}}%
\pgfpathcurveto{\pgfqpoint{3.501954in}{0.655369in}}{\pgfqpoint{3.512405in}{0.651040in}}{\pgfqpoint{3.523301in}{0.651040in}}%
\pgfusepath{stroke}%
\end{pgfscope}%
\begin{pgfscope}%
\pgfpathrectangle{\pgfqpoint{0.688192in}{0.670138in}}{\pgfqpoint{7.111808in}{5.129862in}}%
\pgfusepath{clip}%
\pgfsetbuttcap%
\pgfsetroundjoin%
\pgfsetlinewidth{1.003750pt}%
\definecolor{currentstroke}{rgb}{0.000000,0.000000,0.000000}%
\pgfsetstrokecolor{currentstroke}%
\pgfsetdash{}{0pt}%
\pgfpathmoveto{\pgfqpoint{0.881277in}{0.900428in}}%
\pgfpathcurveto{\pgfqpoint{0.892173in}{0.900428in}}{\pgfqpoint{0.902624in}{0.904756in}}{\pgfqpoint{0.910328in}{0.912461in}}%
\pgfpathcurveto{\pgfqpoint{0.918032in}{0.920165in}}{\pgfqpoint{0.922361in}{0.930616in}}{\pgfqpoint{0.922361in}{0.941511in}}%
\pgfpathcurveto{\pgfqpoint{0.922361in}{0.952407in}}{\pgfqpoint{0.918032in}{0.962858in}}{\pgfqpoint{0.910328in}{0.970562in}}%
\pgfpathcurveto{\pgfqpoint{0.902624in}{0.978266in}}{\pgfqpoint{0.892173in}{0.982595in}}{\pgfqpoint{0.881277in}{0.982595in}}%
\pgfpathcurveto{\pgfqpoint{0.870382in}{0.982595in}}{\pgfqpoint{0.859931in}{0.978266in}}{\pgfqpoint{0.852226in}{0.970562in}}%
\pgfpathcurveto{\pgfqpoint{0.844522in}{0.962858in}}{\pgfqpoint{0.840193in}{0.952407in}}{\pgfqpoint{0.840193in}{0.941511in}}%
\pgfpathcurveto{\pgfqpoint{0.840193in}{0.930616in}}{\pgfqpoint{0.844522in}{0.920165in}}{\pgfqpoint{0.852226in}{0.912461in}}%
\pgfpathcurveto{\pgfqpoint{0.859931in}{0.904756in}}{\pgfqpoint{0.870382in}{0.900428in}}{\pgfqpoint{0.881277in}{0.900428in}}%
\pgfpathlineto{\pgfqpoint{0.881277in}{0.900428in}}%
\pgfpathclose%
\pgfusepath{stroke}%
\end{pgfscope}%
\begin{pgfscope}%
\pgfpathrectangle{\pgfqpoint{0.688192in}{0.670138in}}{\pgfqpoint{7.111808in}{5.129862in}}%
\pgfusepath{clip}%
\pgfsetbuttcap%
\pgfsetroundjoin%
\pgfsetlinewidth{1.003750pt}%
\definecolor{currentstroke}{rgb}{0.000000,0.000000,0.000000}%
\pgfsetstrokecolor{currentstroke}%
\pgfsetdash{}{0pt}%
\pgfpathmoveto{\pgfqpoint{0.790083in}{1.032876in}}%
\pgfpathcurveto{\pgfqpoint{0.800978in}{1.032876in}}{\pgfqpoint{0.811429in}{1.037204in}}{\pgfqpoint{0.819133in}{1.044909in}}%
\pgfpathcurveto{\pgfqpoint{0.826838in}{1.052613in}}{\pgfqpoint{0.831167in}{1.063064in}}{\pgfqpoint{0.831167in}{1.073959in}}%
\pgfpathcurveto{\pgfqpoint{0.831167in}{1.084855in}}{\pgfqpoint{0.826838in}{1.095306in}}{\pgfqpoint{0.819133in}{1.103010in}}%
\pgfpathcurveto{\pgfqpoint{0.811429in}{1.110715in}}{\pgfqpoint{0.800978in}{1.115043in}}{\pgfqpoint{0.790083in}{1.115043in}}%
\pgfpathcurveto{\pgfqpoint{0.779187in}{1.115043in}}{\pgfqpoint{0.768736in}{1.110715in}}{\pgfqpoint{0.761032in}{1.103010in}}%
\pgfpathcurveto{\pgfqpoint{0.753328in}{1.095306in}}{\pgfqpoint{0.748999in}{1.084855in}}{\pgfqpoint{0.748999in}{1.073959in}}%
\pgfpathcurveto{\pgfqpoint{0.748999in}{1.063064in}}{\pgfqpoint{0.753328in}{1.052613in}}{\pgfqpoint{0.761032in}{1.044909in}}%
\pgfpathcurveto{\pgfqpoint{0.768736in}{1.037204in}}{\pgfqpoint{0.779187in}{1.032876in}}{\pgfqpoint{0.790083in}{1.032876in}}%
\pgfpathlineto{\pgfqpoint{0.790083in}{1.032876in}}%
\pgfpathclose%
\pgfusepath{stroke}%
\end{pgfscope}%
\begin{pgfscope}%
\pgfpathrectangle{\pgfqpoint{0.688192in}{0.670138in}}{\pgfqpoint{7.111808in}{5.129862in}}%
\pgfusepath{clip}%
\pgfsetbuttcap%
\pgfsetroundjoin%
\pgfsetlinewidth{1.003750pt}%
\definecolor{currentstroke}{rgb}{0.000000,0.000000,0.000000}%
\pgfsetstrokecolor{currentstroke}%
\pgfsetdash{}{0pt}%
\pgfpathmoveto{\pgfqpoint{1.341138in}{0.712313in}}%
\pgfpathcurveto{\pgfqpoint{1.352034in}{0.712313in}}{\pgfqpoint{1.362484in}{0.716642in}}{\pgfqpoint{1.370189in}{0.724347in}}%
\pgfpathcurveto{\pgfqpoint{1.377893in}{0.732051in}}{\pgfqpoint{1.382222in}{0.742502in}}{\pgfqpoint{1.382222in}{0.753397in}}%
\pgfpathcurveto{\pgfqpoint{1.382222in}{0.764293in}}{\pgfqpoint{1.377893in}{0.774744in}}{\pgfqpoint{1.370189in}{0.782448in}}%
\pgfpathcurveto{\pgfqpoint{1.362484in}{0.790152in}}{\pgfqpoint{1.352034in}{0.794481in}}{\pgfqpoint{1.341138in}{0.794481in}}%
\pgfpathcurveto{\pgfqpoint{1.330243in}{0.794481in}}{\pgfqpoint{1.319792in}{0.790152in}}{\pgfqpoint{1.312087in}{0.782448in}}%
\pgfpathcurveto{\pgfqpoint{1.304383in}{0.774744in}}{\pgfqpoint{1.300054in}{0.764293in}}{\pgfqpoint{1.300054in}{0.753397in}}%
\pgfpathcurveto{\pgfqpoint{1.300054in}{0.742502in}}{\pgfqpoint{1.304383in}{0.732051in}}{\pgfqpoint{1.312087in}{0.724347in}}%
\pgfpathcurveto{\pgfqpoint{1.319792in}{0.716642in}}{\pgfqpoint{1.330243in}{0.712313in}}{\pgfqpoint{1.341138in}{0.712313in}}%
\pgfpathlineto{\pgfqpoint{1.341138in}{0.712313in}}%
\pgfpathclose%
\pgfusepath{stroke}%
\end{pgfscope}%
\begin{pgfscope}%
\pgfpathrectangle{\pgfqpoint{0.688192in}{0.670138in}}{\pgfqpoint{7.111808in}{5.129862in}}%
\pgfusepath{clip}%
\pgfsetbuttcap%
\pgfsetroundjoin%
\pgfsetlinewidth{1.003750pt}%
\definecolor{currentstroke}{rgb}{0.000000,0.000000,0.000000}%
\pgfsetstrokecolor{currentstroke}%
\pgfsetdash{}{0pt}%
\pgfpathmoveto{\pgfqpoint{0.957757in}{0.807940in}}%
\pgfpathcurveto{\pgfqpoint{0.968653in}{0.807940in}}{\pgfqpoint{0.979103in}{0.812269in}}{\pgfqpoint{0.986808in}{0.819974in}}%
\pgfpathcurveto{\pgfqpoint{0.994512in}{0.827678in}}{\pgfqpoint{0.998841in}{0.838129in}}{\pgfqpoint{0.998841in}{0.849024in}}%
\pgfpathcurveto{\pgfqpoint{0.998841in}{0.859920in}}{\pgfqpoint{0.994512in}{0.870371in}}{\pgfqpoint{0.986808in}{0.878075in}}%
\pgfpathcurveto{\pgfqpoint{0.979103in}{0.885779in}}{\pgfqpoint{0.968653in}{0.890108in}}{\pgfqpoint{0.957757in}{0.890108in}}%
\pgfpathcurveto{\pgfqpoint{0.946862in}{0.890108in}}{\pgfqpoint{0.936411in}{0.885779in}}{\pgfqpoint{0.928706in}{0.878075in}}%
\pgfpathcurveto{\pgfqpoint{0.921002in}{0.870371in}}{\pgfqpoint{0.916673in}{0.859920in}}{\pgfqpoint{0.916673in}{0.849024in}}%
\pgfpathcurveto{\pgfqpoint{0.916673in}{0.838129in}}{\pgfqpoint{0.921002in}{0.827678in}}{\pgfqpoint{0.928706in}{0.819974in}}%
\pgfpathcurveto{\pgfqpoint{0.936411in}{0.812269in}}{\pgfqpoint{0.946862in}{0.807940in}}{\pgfqpoint{0.957757in}{0.807940in}}%
\pgfpathlineto{\pgfqpoint{0.957757in}{0.807940in}}%
\pgfpathclose%
\pgfusepath{stroke}%
\end{pgfscope}%
\begin{pgfscope}%
\pgfpathrectangle{\pgfqpoint{0.688192in}{0.670138in}}{\pgfqpoint{7.111808in}{5.129862in}}%
\pgfusepath{clip}%
\pgfsetbuttcap%
\pgfsetroundjoin%
\pgfsetlinewidth{1.003750pt}%
\definecolor{currentstroke}{rgb}{0.000000,0.000000,0.000000}%
\pgfsetstrokecolor{currentstroke}%
\pgfsetdash{}{0pt}%
\pgfpathmoveto{\pgfqpoint{4.954913in}{0.636044in}}%
\pgfpathcurveto{\pgfqpoint{4.965809in}{0.636044in}}{\pgfqpoint{4.976259in}{0.640373in}}{\pgfqpoint{4.983964in}{0.648077in}}%
\pgfpathcurveto{\pgfqpoint{4.991668in}{0.655781in}}{\pgfqpoint{4.995997in}{0.666232in}}{\pgfqpoint{4.995997in}{0.677128in}}%
\pgfpathcurveto{\pgfqpoint{4.995997in}{0.688023in}}{\pgfqpoint{4.991668in}{0.698474in}}{\pgfqpoint{4.983964in}{0.706179in}}%
\pgfpathcurveto{\pgfqpoint{4.976259in}{0.713883in}}{\pgfqpoint{4.965809in}{0.718212in}}{\pgfqpoint{4.954913in}{0.718212in}}%
\pgfpathcurveto{\pgfqpoint{4.944017in}{0.718212in}}{\pgfqpoint{4.933567in}{0.713883in}}{\pgfqpoint{4.925862in}{0.706179in}}%
\pgfpathcurveto{\pgfqpoint{4.918158in}{0.698474in}}{\pgfqpoint{4.913829in}{0.688023in}}{\pgfqpoint{4.913829in}{0.677128in}}%
\pgfpathcurveto{\pgfqpoint{4.913829in}{0.666232in}}{\pgfqpoint{4.918158in}{0.655781in}}{\pgfqpoint{4.925862in}{0.648077in}}%
\pgfpathcurveto{\pgfqpoint{4.933567in}{0.640373in}}{\pgfqpoint{4.944017in}{0.636044in}}{\pgfqpoint{4.954913in}{0.636044in}}%
\pgfusepath{stroke}%
\end{pgfscope}%
\begin{pgfscope}%
\pgfpathrectangle{\pgfqpoint{0.688192in}{0.670138in}}{\pgfqpoint{7.111808in}{5.129862in}}%
\pgfusepath{clip}%
\pgfsetbuttcap%
\pgfsetroundjoin%
\pgfsetlinewidth{1.003750pt}%
\definecolor{currentstroke}{rgb}{0.000000,0.000000,0.000000}%
\pgfsetstrokecolor{currentstroke}%
\pgfsetdash{}{0pt}%
\pgfpathmoveto{\pgfqpoint{1.069116in}{0.729808in}}%
\pgfpathcurveto{\pgfqpoint{1.080011in}{0.729808in}}{\pgfqpoint{1.090462in}{0.734137in}}{\pgfqpoint{1.098166in}{0.741841in}}%
\pgfpathcurveto{\pgfqpoint{1.105871in}{0.749545in}}{\pgfqpoint{1.110200in}{0.759996in}}{\pgfqpoint{1.110200in}{0.770892in}}%
\pgfpathcurveto{\pgfqpoint{1.110200in}{0.781787in}}{\pgfqpoint{1.105871in}{0.792238in}}{\pgfqpoint{1.098166in}{0.799942in}}%
\pgfpathcurveto{\pgfqpoint{1.090462in}{0.807647in}}{\pgfqpoint{1.080011in}{0.811976in}}{\pgfqpoint{1.069116in}{0.811976in}}%
\pgfpathcurveto{\pgfqpoint{1.058220in}{0.811976in}}{\pgfqpoint{1.047769in}{0.807647in}}{\pgfqpoint{1.040065in}{0.799942in}}%
\pgfpathcurveto{\pgfqpoint{1.032361in}{0.792238in}}{\pgfqpoint{1.028032in}{0.781787in}}{\pgfqpoint{1.028032in}{0.770892in}}%
\pgfpathcurveto{\pgfqpoint{1.028032in}{0.759996in}}{\pgfqpoint{1.032361in}{0.749545in}}{\pgfqpoint{1.040065in}{0.741841in}}%
\pgfpathcurveto{\pgfqpoint{1.047769in}{0.734137in}}{\pgfqpoint{1.058220in}{0.729808in}}{\pgfqpoint{1.069116in}{0.729808in}}%
\pgfpathlineto{\pgfqpoint{1.069116in}{0.729808in}}%
\pgfpathclose%
\pgfusepath{stroke}%
\end{pgfscope}%
\begin{pgfscope}%
\pgfpathrectangle{\pgfqpoint{0.688192in}{0.670138in}}{\pgfqpoint{7.111808in}{5.129862in}}%
\pgfusepath{clip}%
\pgfsetbuttcap%
\pgfsetroundjoin%
\pgfsetlinewidth{1.003750pt}%
\definecolor{currentstroke}{rgb}{0.000000,0.000000,0.000000}%
\pgfsetstrokecolor{currentstroke}%
\pgfsetdash{}{0pt}%
\pgfpathmoveto{\pgfqpoint{2.443271in}{0.678334in}}%
\pgfpathcurveto{\pgfqpoint{2.454167in}{0.678334in}}{\pgfqpoint{2.464618in}{0.682663in}}{\pgfqpoint{2.472322in}{0.690367in}}%
\pgfpathcurveto{\pgfqpoint{2.480026in}{0.698072in}}{\pgfqpoint{2.484355in}{0.708523in}}{\pgfqpoint{2.484355in}{0.719418in}}%
\pgfpathcurveto{\pgfqpoint{2.484355in}{0.730314in}}{\pgfqpoint{2.480026in}{0.740764in}}{\pgfqpoint{2.472322in}{0.748469in}}%
\pgfpathcurveto{\pgfqpoint{2.464618in}{0.756173in}}{\pgfqpoint{2.454167in}{0.760502in}}{\pgfqpoint{2.443271in}{0.760502in}}%
\pgfpathcurveto{\pgfqpoint{2.432376in}{0.760502in}}{\pgfqpoint{2.421925in}{0.756173in}}{\pgfqpoint{2.414221in}{0.748469in}}%
\pgfpathcurveto{\pgfqpoint{2.406516in}{0.740764in}}{\pgfqpoint{2.402188in}{0.730314in}}{\pgfqpoint{2.402188in}{0.719418in}}%
\pgfpathcurveto{\pgfqpoint{2.402188in}{0.708523in}}{\pgfqpoint{2.406516in}{0.698072in}}{\pgfqpoint{2.414221in}{0.690367in}}%
\pgfpathcurveto{\pgfqpoint{2.421925in}{0.682663in}}{\pgfqpoint{2.432376in}{0.678334in}}{\pgfqpoint{2.443271in}{0.678334in}}%
\pgfpathlineto{\pgfqpoint{2.443271in}{0.678334in}}%
\pgfpathclose%
\pgfusepath{stroke}%
\end{pgfscope}%
\begin{pgfscope}%
\pgfpathrectangle{\pgfqpoint{0.688192in}{0.670138in}}{\pgfqpoint{7.111808in}{5.129862in}}%
\pgfusepath{clip}%
\pgfsetbuttcap%
\pgfsetroundjoin%
\pgfsetlinewidth{1.003750pt}%
\definecolor{currentstroke}{rgb}{0.000000,0.000000,0.000000}%
\pgfsetstrokecolor{currentstroke}%
\pgfsetdash{}{0pt}%
\pgfpathmoveto{\pgfqpoint{0.880316in}{0.913368in}}%
\pgfpathcurveto{\pgfqpoint{0.891212in}{0.913368in}}{\pgfqpoint{0.901663in}{0.917697in}}{\pgfqpoint{0.909367in}{0.925401in}}%
\pgfpathcurveto{\pgfqpoint{0.917071in}{0.933106in}}{\pgfqpoint{0.921400in}{0.943557in}}{\pgfqpoint{0.921400in}{0.954452in}}%
\pgfpathcurveto{\pgfqpoint{0.921400in}{0.965348in}}{\pgfqpoint{0.917071in}{0.975799in}}{\pgfqpoint{0.909367in}{0.983503in}}%
\pgfpathcurveto{\pgfqpoint{0.901663in}{0.991207in}}{\pgfqpoint{0.891212in}{0.995536in}}{\pgfqpoint{0.880316in}{0.995536in}}%
\pgfpathcurveto{\pgfqpoint{0.869421in}{0.995536in}}{\pgfqpoint{0.858970in}{0.991207in}}{\pgfqpoint{0.851266in}{0.983503in}}%
\pgfpathcurveto{\pgfqpoint{0.843561in}{0.975799in}}{\pgfqpoint{0.839232in}{0.965348in}}{\pgfqpoint{0.839232in}{0.954452in}}%
\pgfpathcurveto{\pgfqpoint{0.839232in}{0.943557in}}{\pgfqpoint{0.843561in}{0.933106in}}{\pgfqpoint{0.851266in}{0.925401in}}%
\pgfpathcurveto{\pgfqpoint{0.858970in}{0.917697in}}{\pgfqpoint{0.869421in}{0.913368in}}{\pgfqpoint{0.880316in}{0.913368in}}%
\pgfpathlineto{\pgfqpoint{0.880316in}{0.913368in}}%
\pgfpathclose%
\pgfusepath{stroke}%
\end{pgfscope}%
\begin{pgfscope}%
\pgfpathrectangle{\pgfqpoint{0.688192in}{0.670138in}}{\pgfqpoint{7.111808in}{5.129862in}}%
\pgfusepath{clip}%
\pgfsetbuttcap%
\pgfsetroundjoin%
\pgfsetlinewidth{1.003750pt}%
\definecolor{currentstroke}{rgb}{0.000000,0.000000,0.000000}%
\pgfsetstrokecolor{currentstroke}%
\pgfsetdash{}{0pt}%
\pgfpathmoveto{\pgfqpoint{0.951297in}{0.809397in}}%
\pgfpathcurveto{\pgfqpoint{0.962193in}{0.809397in}}{\pgfqpoint{0.972643in}{0.813726in}}{\pgfqpoint{0.980348in}{0.821431in}}%
\pgfpathcurveto{\pgfqpoint{0.988052in}{0.829135in}}{\pgfqpoint{0.992381in}{0.839586in}}{\pgfqpoint{0.992381in}{0.850481in}}%
\pgfpathcurveto{\pgfqpoint{0.992381in}{0.861377in}}{\pgfqpoint{0.988052in}{0.871828in}}{\pgfqpoint{0.980348in}{0.879532in}}%
\pgfpathcurveto{\pgfqpoint{0.972643in}{0.887236in}}{\pgfqpoint{0.962193in}{0.891565in}}{\pgfqpoint{0.951297in}{0.891565in}}%
\pgfpathcurveto{\pgfqpoint{0.940402in}{0.891565in}}{\pgfqpoint{0.929951in}{0.887236in}}{\pgfqpoint{0.922246in}{0.879532in}}%
\pgfpathcurveto{\pgfqpoint{0.914542in}{0.871828in}}{\pgfqpoint{0.910213in}{0.861377in}}{\pgfqpoint{0.910213in}{0.850481in}}%
\pgfpathcurveto{\pgfqpoint{0.910213in}{0.839586in}}{\pgfqpoint{0.914542in}{0.829135in}}{\pgfqpoint{0.922246in}{0.821431in}}%
\pgfpathcurveto{\pgfqpoint{0.929951in}{0.813726in}}{\pgfqpoint{0.940402in}{0.809397in}}{\pgfqpoint{0.951297in}{0.809397in}}%
\pgfpathlineto{\pgfqpoint{0.951297in}{0.809397in}}%
\pgfpathclose%
\pgfusepath{stroke}%
\end{pgfscope}%
\begin{pgfscope}%
\pgfpathrectangle{\pgfqpoint{0.688192in}{0.670138in}}{\pgfqpoint{7.111808in}{5.129862in}}%
\pgfusepath{clip}%
\pgfsetbuttcap%
\pgfsetroundjoin%
\pgfsetlinewidth{1.003750pt}%
\definecolor{currentstroke}{rgb}{0.000000,0.000000,0.000000}%
\pgfsetstrokecolor{currentstroke}%
\pgfsetdash{}{0pt}%
\pgfpathmoveto{\pgfqpoint{6.226527in}{1.339270in}}%
\pgfpathcurveto{\pgfqpoint{6.237423in}{1.339270in}}{\pgfqpoint{6.247873in}{1.343599in}}{\pgfqpoint{6.255578in}{1.351303in}}%
\pgfpathcurveto{\pgfqpoint{6.263282in}{1.359007in}}{\pgfqpoint{6.267611in}{1.369458in}}{\pgfqpoint{6.267611in}{1.380354in}}%
\pgfpathcurveto{\pgfqpoint{6.267611in}{1.391249in}}{\pgfqpoint{6.263282in}{1.401700in}}{\pgfqpoint{6.255578in}{1.409404in}}%
\pgfpathcurveto{\pgfqpoint{6.247873in}{1.417109in}}{\pgfqpoint{6.237423in}{1.421438in}}{\pgfqpoint{6.226527in}{1.421438in}}%
\pgfpathcurveto{\pgfqpoint{6.215631in}{1.421438in}}{\pgfqpoint{6.205181in}{1.417109in}}{\pgfqpoint{6.197476in}{1.409404in}}%
\pgfpathcurveto{\pgfqpoint{6.189772in}{1.401700in}}{\pgfqpoint{6.185443in}{1.391249in}}{\pgfqpoint{6.185443in}{1.380354in}}%
\pgfpathcurveto{\pgfqpoint{6.185443in}{1.369458in}}{\pgfqpoint{6.189772in}{1.359007in}}{\pgfqpoint{6.197476in}{1.351303in}}%
\pgfpathcurveto{\pgfqpoint{6.205181in}{1.343599in}}{\pgfqpoint{6.215631in}{1.339270in}}{\pgfqpoint{6.226527in}{1.339270in}}%
\pgfpathlineto{\pgfqpoint{6.226527in}{1.339270in}}%
\pgfpathclose%
\pgfusepath{stroke}%
\end{pgfscope}%
\begin{pgfscope}%
\pgfpathrectangle{\pgfqpoint{0.688192in}{0.670138in}}{\pgfqpoint{7.111808in}{5.129862in}}%
\pgfusepath{clip}%
\pgfsetbuttcap%
\pgfsetroundjoin%
\pgfsetlinewidth{1.003750pt}%
\definecolor{currentstroke}{rgb}{0.000000,0.000000,0.000000}%
\pgfsetstrokecolor{currentstroke}%
\pgfsetdash{}{0pt}%
\pgfpathmoveto{\pgfqpoint{1.044526in}{0.742721in}}%
\pgfpathcurveto{\pgfqpoint{1.055421in}{0.742721in}}{\pgfqpoint{1.065872in}{0.747050in}}{\pgfqpoint{1.073577in}{0.754754in}}%
\pgfpathcurveto{\pgfqpoint{1.081281in}{0.762459in}}{\pgfqpoint{1.085610in}{0.772909in}}{\pgfqpoint{1.085610in}{0.783805in}}%
\pgfpathcurveto{\pgfqpoint{1.085610in}{0.794701in}}{\pgfqpoint{1.081281in}{0.805151in}}{\pgfqpoint{1.073577in}{0.812856in}}%
\pgfpathcurveto{\pgfqpoint{1.065872in}{0.820560in}}{\pgfqpoint{1.055421in}{0.824889in}}{\pgfqpoint{1.044526in}{0.824889in}}%
\pgfpathcurveto{\pgfqpoint{1.033630in}{0.824889in}}{\pgfqpoint{1.023179in}{0.820560in}}{\pgfqpoint{1.015475in}{0.812856in}}%
\pgfpathcurveto{\pgfqpoint{1.007771in}{0.805151in}}{\pgfqpoint{1.003442in}{0.794701in}}{\pgfqpoint{1.003442in}{0.783805in}}%
\pgfpathcurveto{\pgfqpoint{1.003442in}{0.772909in}}{\pgfqpoint{1.007771in}{0.762459in}}{\pgfqpoint{1.015475in}{0.754754in}}%
\pgfpathcurveto{\pgfqpoint{1.023179in}{0.747050in}}{\pgfqpoint{1.033630in}{0.742721in}}{\pgfqpoint{1.044526in}{0.742721in}}%
\pgfpathlineto{\pgfqpoint{1.044526in}{0.742721in}}%
\pgfpathclose%
\pgfusepath{stroke}%
\end{pgfscope}%
\begin{pgfscope}%
\pgfpathrectangle{\pgfqpoint{0.688192in}{0.670138in}}{\pgfqpoint{7.111808in}{5.129862in}}%
\pgfusepath{clip}%
\pgfsetbuttcap%
\pgfsetroundjoin%
\pgfsetlinewidth{1.003750pt}%
\definecolor{currentstroke}{rgb}{0.000000,0.000000,0.000000}%
\pgfsetstrokecolor{currentstroke}%
\pgfsetdash{}{0pt}%
\pgfpathmoveto{\pgfqpoint{2.629436in}{1.010748in}}%
\pgfpathcurveto{\pgfqpoint{2.640332in}{1.010748in}}{\pgfqpoint{2.650782in}{1.015076in}}{\pgfqpoint{2.658487in}{1.022781in}}%
\pgfpathcurveto{\pgfqpoint{2.666191in}{1.030485in}}{\pgfqpoint{2.670520in}{1.040936in}}{\pgfqpoint{2.670520in}{1.051831in}}%
\pgfpathcurveto{\pgfqpoint{2.670520in}{1.062727in}}{\pgfqpoint{2.666191in}{1.073178in}}{\pgfqpoint{2.658487in}{1.080882in}}%
\pgfpathcurveto{\pgfqpoint{2.650782in}{1.088586in}}{\pgfqpoint{2.640332in}{1.092915in}}{\pgfqpoint{2.629436in}{1.092915in}}%
\pgfpathcurveto{\pgfqpoint{2.618540in}{1.092915in}}{\pgfqpoint{2.608090in}{1.088586in}}{\pgfqpoint{2.600385in}{1.080882in}}%
\pgfpathcurveto{\pgfqpoint{2.592681in}{1.073178in}}{\pgfqpoint{2.588352in}{1.062727in}}{\pgfqpoint{2.588352in}{1.051831in}}%
\pgfpathcurveto{\pgfqpoint{2.588352in}{1.040936in}}{\pgfqpoint{2.592681in}{1.030485in}}{\pgfqpoint{2.600385in}{1.022781in}}%
\pgfpathcurveto{\pgfqpoint{2.608090in}{1.015076in}}{\pgfqpoint{2.618540in}{1.010748in}}{\pgfqpoint{2.629436in}{1.010748in}}%
\pgfpathlineto{\pgfqpoint{2.629436in}{1.010748in}}%
\pgfpathclose%
\pgfusepath{stroke}%
\end{pgfscope}%
\begin{pgfscope}%
\pgfpathrectangle{\pgfqpoint{0.688192in}{0.670138in}}{\pgfqpoint{7.111808in}{5.129862in}}%
\pgfusepath{clip}%
\pgfsetbuttcap%
\pgfsetroundjoin%
\pgfsetlinewidth{1.003750pt}%
\definecolor{currentstroke}{rgb}{0.000000,0.000000,0.000000}%
\pgfsetstrokecolor{currentstroke}%
\pgfsetdash{}{0pt}%
\pgfpathmoveto{\pgfqpoint{1.414589in}{0.710834in}}%
\pgfpathcurveto{\pgfqpoint{1.425484in}{0.710834in}}{\pgfqpoint{1.435935in}{0.715163in}}{\pgfqpoint{1.443639in}{0.722867in}}%
\pgfpathcurveto{\pgfqpoint{1.451344in}{0.730572in}}{\pgfqpoint{1.455672in}{0.741022in}}{\pgfqpoint{1.455672in}{0.751918in}}%
\pgfpathcurveto{\pgfqpoint{1.455672in}{0.762814in}}{\pgfqpoint{1.451344in}{0.773264in}}{\pgfqpoint{1.443639in}{0.780969in}}%
\pgfpathcurveto{\pgfqpoint{1.435935in}{0.788673in}}{\pgfqpoint{1.425484in}{0.793002in}}{\pgfqpoint{1.414589in}{0.793002in}}%
\pgfpathcurveto{\pgfqpoint{1.403693in}{0.793002in}}{\pgfqpoint{1.393242in}{0.788673in}}{\pgfqpoint{1.385538in}{0.780969in}}%
\pgfpathcurveto{\pgfqpoint{1.377834in}{0.773264in}}{\pgfqpoint{1.373505in}{0.762814in}}{\pgfqpoint{1.373505in}{0.751918in}}%
\pgfpathcurveto{\pgfqpoint{1.373505in}{0.741022in}}{\pgfqpoint{1.377834in}{0.730572in}}{\pgfqpoint{1.385538in}{0.722867in}}%
\pgfpathcurveto{\pgfqpoint{1.393242in}{0.715163in}}{\pgfqpoint{1.403693in}{0.710834in}}{\pgfqpoint{1.414589in}{0.710834in}}%
\pgfpathlineto{\pgfqpoint{1.414589in}{0.710834in}}%
\pgfpathclose%
\pgfusepath{stroke}%
\end{pgfscope}%
\begin{pgfscope}%
\pgfpathrectangle{\pgfqpoint{0.688192in}{0.670138in}}{\pgfqpoint{7.111808in}{5.129862in}}%
\pgfusepath{clip}%
\pgfsetbuttcap%
\pgfsetroundjoin%
\pgfsetlinewidth{1.003750pt}%
\definecolor{currentstroke}{rgb}{0.000000,0.000000,0.000000}%
\pgfsetstrokecolor{currentstroke}%
\pgfsetdash{}{0pt}%
\pgfpathmoveto{\pgfqpoint{1.610058in}{0.705530in}}%
\pgfpathcurveto{\pgfqpoint{1.620954in}{0.705530in}}{\pgfqpoint{1.631404in}{0.709859in}}{\pgfqpoint{1.639109in}{0.717563in}}%
\pgfpathcurveto{\pgfqpoint{1.646813in}{0.725267in}}{\pgfqpoint{1.651142in}{0.735718in}}{\pgfqpoint{1.651142in}{0.746614in}}%
\pgfpathcurveto{\pgfqpoint{1.651142in}{0.757509in}}{\pgfqpoint{1.646813in}{0.767960in}}{\pgfqpoint{1.639109in}{0.775664in}}%
\pgfpathcurveto{\pgfqpoint{1.631404in}{0.783369in}}{\pgfqpoint{1.620954in}{0.787697in}}{\pgfqpoint{1.610058in}{0.787697in}}%
\pgfpathcurveto{\pgfqpoint{1.599163in}{0.787697in}}{\pgfqpoint{1.588712in}{0.783369in}}{\pgfqpoint{1.581007in}{0.775664in}}%
\pgfpathcurveto{\pgfqpoint{1.573303in}{0.767960in}}{\pgfqpoint{1.568974in}{0.757509in}}{\pgfqpoint{1.568974in}{0.746614in}}%
\pgfpathcurveto{\pgfqpoint{1.568974in}{0.735718in}}{\pgfqpoint{1.573303in}{0.725267in}}{\pgfqpoint{1.581007in}{0.717563in}}%
\pgfpathcurveto{\pgfqpoint{1.588712in}{0.709859in}}{\pgfqpoint{1.599163in}{0.705530in}}{\pgfqpoint{1.610058in}{0.705530in}}%
\pgfpathlineto{\pgfqpoint{1.610058in}{0.705530in}}%
\pgfpathclose%
\pgfusepath{stroke}%
\end{pgfscope}%
\begin{pgfscope}%
\pgfpathrectangle{\pgfqpoint{0.688192in}{0.670138in}}{\pgfqpoint{7.111808in}{5.129862in}}%
\pgfusepath{clip}%
\pgfsetbuttcap%
\pgfsetroundjoin%
\pgfsetlinewidth{1.003750pt}%
\definecolor{currentstroke}{rgb}{0.000000,0.000000,0.000000}%
\pgfsetstrokecolor{currentstroke}%
\pgfsetdash{}{0pt}%
\pgfpathmoveto{\pgfqpoint{1.840914in}{0.696069in}}%
\pgfpathcurveto{\pgfqpoint{1.851809in}{0.696069in}}{\pgfqpoint{1.862260in}{0.700398in}}{\pgfqpoint{1.869964in}{0.708103in}}%
\pgfpathcurveto{\pgfqpoint{1.877669in}{0.715807in}}{\pgfqpoint{1.881997in}{0.726258in}}{\pgfqpoint{1.881997in}{0.737153in}}%
\pgfpathcurveto{\pgfqpoint{1.881997in}{0.748049in}}{\pgfqpoint{1.877669in}{0.758500in}}{\pgfqpoint{1.869964in}{0.766204in}}%
\pgfpathcurveto{\pgfqpoint{1.862260in}{0.773908in}}{\pgfqpoint{1.851809in}{0.778237in}}{\pgfqpoint{1.840914in}{0.778237in}}%
\pgfpathcurveto{\pgfqpoint{1.830018in}{0.778237in}}{\pgfqpoint{1.819567in}{0.773908in}}{\pgfqpoint{1.811863in}{0.766204in}}%
\pgfpathcurveto{\pgfqpoint{1.804159in}{0.758500in}}{\pgfqpoint{1.799830in}{0.748049in}}{\pgfqpoint{1.799830in}{0.737153in}}%
\pgfpathcurveto{\pgfqpoint{1.799830in}{0.726258in}}{\pgfqpoint{1.804159in}{0.715807in}}{\pgfqpoint{1.811863in}{0.708103in}}%
\pgfpathcurveto{\pgfqpoint{1.819567in}{0.700398in}}{\pgfqpoint{1.830018in}{0.696069in}}{\pgfqpoint{1.840914in}{0.696069in}}%
\pgfpathlineto{\pgfqpoint{1.840914in}{0.696069in}}%
\pgfpathclose%
\pgfusepath{stroke}%
\end{pgfscope}%
\begin{pgfscope}%
\pgfpathrectangle{\pgfqpoint{0.688192in}{0.670138in}}{\pgfqpoint{7.111808in}{5.129862in}}%
\pgfusepath{clip}%
\pgfsetbuttcap%
\pgfsetroundjoin%
\pgfsetlinewidth{1.003750pt}%
\definecolor{currentstroke}{rgb}{0.000000,0.000000,0.000000}%
\pgfsetstrokecolor{currentstroke}%
\pgfsetdash{}{0pt}%
\pgfpathmoveto{\pgfqpoint{1.069116in}{0.729808in}}%
\pgfpathcurveto{\pgfqpoint{1.080011in}{0.729808in}}{\pgfqpoint{1.090462in}{0.734137in}}{\pgfqpoint{1.098166in}{0.741841in}}%
\pgfpathcurveto{\pgfqpoint{1.105871in}{0.749545in}}{\pgfqpoint{1.110200in}{0.759996in}}{\pgfqpoint{1.110200in}{0.770892in}}%
\pgfpathcurveto{\pgfqpoint{1.110200in}{0.781787in}}{\pgfqpoint{1.105871in}{0.792238in}}{\pgfqpoint{1.098166in}{0.799942in}}%
\pgfpathcurveto{\pgfqpoint{1.090462in}{0.807647in}}{\pgfqpoint{1.080011in}{0.811976in}}{\pgfqpoint{1.069116in}{0.811976in}}%
\pgfpathcurveto{\pgfqpoint{1.058220in}{0.811976in}}{\pgfqpoint{1.047769in}{0.807647in}}{\pgfqpoint{1.040065in}{0.799942in}}%
\pgfpathcurveto{\pgfqpoint{1.032361in}{0.792238in}}{\pgfqpoint{1.028032in}{0.781787in}}{\pgfqpoint{1.028032in}{0.770892in}}%
\pgfpathcurveto{\pgfqpoint{1.028032in}{0.759996in}}{\pgfqpoint{1.032361in}{0.749545in}}{\pgfqpoint{1.040065in}{0.741841in}}%
\pgfpathcurveto{\pgfqpoint{1.047769in}{0.734137in}}{\pgfqpoint{1.058220in}{0.729808in}}{\pgfqpoint{1.069116in}{0.729808in}}%
\pgfpathlineto{\pgfqpoint{1.069116in}{0.729808in}}%
\pgfpathclose%
\pgfusepath{stroke}%
\end{pgfscope}%
\begin{pgfscope}%
\pgfpathrectangle{\pgfqpoint{0.688192in}{0.670138in}}{\pgfqpoint{7.111808in}{5.129862in}}%
\pgfusepath{clip}%
\pgfsetbuttcap%
\pgfsetroundjoin%
\pgfsetlinewidth{1.003750pt}%
\definecolor{currentstroke}{rgb}{0.000000,0.000000,0.000000}%
\pgfsetstrokecolor{currentstroke}%
\pgfsetdash{}{0pt}%
\pgfpathmoveto{\pgfqpoint{0.886861in}{0.869726in}}%
\pgfpathcurveto{\pgfqpoint{0.897756in}{0.869726in}}{\pgfqpoint{0.908207in}{0.874055in}}{\pgfqpoint{0.915911in}{0.881759in}}%
\pgfpathcurveto{\pgfqpoint{0.923616in}{0.889463in}}{\pgfqpoint{0.927945in}{0.899914in}}{\pgfqpoint{0.927945in}{0.910810in}}%
\pgfpathcurveto{\pgfqpoint{0.927945in}{0.921705in}}{\pgfqpoint{0.923616in}{0.932156in}}{\pgfqpoint{0.915911in}{0.939861in}}%
\pgfpathcurveto{\pgfqpoint{0.908207in}{0.947565in}}{\pgfqpoint{0.897756in}{0.951894in}}{\pgfqpoint{0.886861in}{0.951894in}}%
\pgfpathcurveto{\pgfqpoint{0.875965in}{0.951894in}}{\pgfqpoint{0.865514in}{0.947565in}}{\pgfqpoint{0.857810in}{0.939861in}}%
\pgfpathcurveto{\pgfqpoint{0.850106in}{0.932156in}}{\pgfqpoint{0.845777in}{0.921705in}}{\pgfqpoint{0.845777in}{0.910810in}}%
\pgfpathcurveto{\pgfqpoint{0.845777in}{0.899914in}}{\pgfqpoint{0.850106in}{0.889463in}}{\pgfqpoint{0.857810in}{0.881759in}}%
\pgfpathcurveto{\pgfqpoint{0.865514in}{0.874055in}}{\pgfqpoint{0.875965in}{0.869726in}}{\pgfqpoint{0.886861in}{0.869726in}}%
\pgfpathlineto{\pgfqpoint{0.886861in}{0.869726in}}%
\pgfpathclose%
\pgfusepath{stroke}%
\end{pgfscope}%
\begin{pgfscope}%
\pgfpathrectangle{\pgfqpoint{0.688192in}{0.670138in}}{\pgfqpoint{7.111808in}{5.129862in}}%
\pgfusepath{clip}%
\pgfsetbuttcap%
\pgfsetroundjoin%
\pgfsetlinewidth{1.003750pt}%
\definecolor{currentstroke}{rgb}{0.000000,0.000000,0.000000}%
\pgfsetstrokecolor{currentstroke}%
\pgfsetdash{}{0pt}%
\pgfpathmoveto{\pgfqpoint{2.443271in}{0.678334in}}%
\pgfpathcurveto{\pgfqpoint{2.454167in}{0.678334in}}{\pgfqpoint{2.464618in}{0.682663in}}{\pgfqpoint{2.472322in}{0.690367in}}%
\pgfpathcurveto{\pgfqpoint{2.480026in}{0.698072in}}{\pgfqpoint{2.484355in}{0.708523in}}{\pgfqpoint{2.484355in}{0.719418in}}%
\pgfpathcurveto{\pgfqpoint{2.484355in}{0.730314in}}{\pgfqpoint{2.480026in}{0.740764in}}{\pgfqpoint{2.472322in}{0.748469in}}%
\pgfpathcurveto{\pgfqpoint{2.464618in}{0.756173in}}{\pgfqpoint{2.454167in}{0.760502in}}{\pgfqpoint{2.443271in}{0.760502in}}%
\pgfpathcurveto{\pgfqpoint{2.432376in}{0.760502in}}{\pgfqpoint{2.421925in}{0.756173in}}{\pgfqpoint{2.414221in}{0.748469in}}%
\pgfpathcurveto{\pgfqpoint{2.406516in}{0.740764in}}{\pgfqpoint{2.402188in}{0.730314in}}{\pgfqpoint{2.402188in}{0.719418in}}%
\pgfpathcurveto{\pgfqpoint{2.402188in}{0.708523in}}{\pgfqpoint{2.406516in}{0.698072in}}{\pgfqpoint{2.414221in}{0.690367in}}%
\pgfpathcurveto{\pgfqpoint{2.421925in}{0.682663in}}{\pgfqpoint{2.432376in}{0.678334in}}{\pgfqpoint{2.443271in}{0.678334in}}%
\pgfpathlineto{\pgfqpoint{2.443271in}{0.678334in}}%
\pgfpathclose%
\pgfusepath{stroke}%
\end{pgfscope}%
\begin{pgfscope}%
\pgfpathrectangle{\pgfqpoint{0.688192in}{0.670138in}}{\pgfqpoint{7.111808in}{5.129862in}}%
\pgfusepath{clip}%
\pgfsetbuttcap%
\pgfsetroundjoin%
\pgfsetlinewidth{1.003750pt}%
\definecolor{currentstroke}{rgb}{0.000000,0.000000,0.000000}%
\pgfsetstrokecolor{currentstroke}%
\pgfsetdash{}{0pt}%
\pgfpathmoveto{\pgfqpoint{2.443271in}{0.678334in}}%
\pgfpathcurveto{\pgfqpoint{2.454167in}{0.678334in}}{\pgfqpoint{2.464618in}{0.682663in}}{\pgfqpoint{2.472322in}{0.690367in}}%
\pgfpathcurveto{\pgfqpoint{2.480026in}{0.698072in}}{\pgfqpoint{2.484355in}{0.708523in}}{\pgfqpoint{2.484355in}{0.719418in}}%
\pgfpathcurveto{\pgfqpoint{2.484355in}{0.730314in}}{\pgfqpoint{2.480026in}{0.740764in}}{\pgfqpoint{2.472322in}{0.748469in}}%
\pgfpathcurveto{\pgfqpoint{2.464618in}{0.756173in}}{\pgfqpoint{2.454167in}{0.760502in}}{\pgfqpoint{2.443271in}{0.760502in}}%
\pgfpathcurveto{\pgfqpoint{2.432376in}{0.760502in}}{\pgfqpoint{2.421925in}{0.756173in}}{\pgfqpoint{2.414221in}{0.748469in}}%
\pgfpathcurveto{\pgfqpoint{2.406516in}{0.740764in}}{\pgfqpoint{2.402188in}{0.730314in}}{\pgfqpoint{2.402188in}{0.719418in}}%
\pgfpathcurveto{\pgfqpoint{2.402188in}{0.708523in}}{\pgfqpoint{2.406516in}{0.698072in}}{\pgfqpoint{2.414221in}{0.690367in}}%
\pgfpathcurveto{\pgfqpoint{2.421925in}{0.682663in}}{\pgfqpoint{2.432376in}{0.678334in}}{\pgfqpoint{2.443271in}{0.678334in}}%
\pgfpathlineto{\pgfqpoint{2.443271in}{0.678334in}}%
\pgfpathclose%
\pgfusepath{stroke}%
\end{pgfscope}%
\begin{pgfscope}%
\pgfpathrectangle{\pgfqpoint{0.688192in}{0.670138in}}{\pgfqpoint{7.111808in}{5.129862in}}%
\pgfusepath{clip}%
\pgfsetbuttcap%
\pgfsetroundjoin%
\pgfsetlinewidth{1.003750pt}%
\definecolor{currentstroke}{rgb}{0.000000,0.000000,0.000000}%
\pgfsetstrokecolor{currentstroke}%
\pgfsetdash{}{0pt}%
\pgfpathmoveto{\pgfqpoint{1.448644in}{0.707802in}}%
\pgfpathcurveto{\pgfqpoint{1.459539in}{0.707802in}}{\pgfqpoint{1.469990in}{0.712131in}}{\pgfqpoint{1.477694in}{0.719835in}}%
\pgfpathcurveto{\pgfqpoint{1.485399in}{0.727539in}}{\pgfqpoint{1.489728in}{0.737990in}}{\pgfqpoint{1.489728in}{0.748886in}}%
\pgfpathcurveto{\pgfqpoint{1.489728in}{0.759781in}}{\pgfqpoint{1.485399in}{0.770232in}}{\pgfqpoint{1.477694in}{0.777936in}}%
\pgfpathcurveto{\pgfqpoint{1.469990in}{0.785641in}}{\pgfqpoint{1.459539in}{0.789970in}}{\pgfqpoint{1.448644in}{0.789970in}}%
\pgfpathcurveto{\pgfqpoint{1.437748in}{0.789970in}}{\pgfqpoint{1.427297in}{0.785641in}}{\pgfqpoint{1.419593in}{0.777936in}}%
\pgfpathcurveto{\pgfqpoint{1.411889in}{0.770232in}}{\pgfqpoint{1.407560in}{0.759781in}}{\pgfqpoint{1.407560in}{0.748886in}}%
\pgfpathcurveto{\pgfqpoint{1.407560in}{0.737990in}}{\pgfqpoint{1.411889in}{0.727539in}}{\pgfqpoint{1.419593in}{0.719835in}}%
\pgfpathcurveto{\pgfqpoint{1.427297in}{0.712131in}}{\pgfqpoint{1.437748in}{0.707802in}}{\pgfqpoint{1.448644in}{0.707802in}}%
\pgfpathlineto{\pgfqpoint{1.448644in}{0.707802in}}%
\pgfpathclose%
\pgfusepath{stroke}%
\end{pgfscope}%
\begin{pgfscope}%
\pgfpathrectangle{\pgfqpoint{0.688192in}{0.670138in}}{\pgfqpoint{7.111808in}{5.129862in}}%
\pgfusepath{clip}%
\pgfsetbuttcap%
\pgfsetroundjoin%
\pgfsetlinewidth{1.003750pt}%
\definecolor{currentstroke}{rgb}{0.000000,0.000000,0.000000}%
\pgfsetstrokecolor{currentstroke}%
\pgfsetdash{}{0pt}%
\pgfpathmoveto{\pgfqpoint{1.308402in}{0.713449in}}%
\pgfpathcurveto{\pgfqpoint{1.319298in}{0.713449in}}{\pgfqpoint{1.329748in}{0.717778in}}{\pgfqpoint{1.337453in}{0.725482in}}%
\pgfpathcurveto{\pgfqpoint{1.345157in}{0.733187in}}{\pgfqpoint{1.349486in}{0.743637in}}{\pgfqpoint{1.349486in}{0.754533in}}%
\pgfpathcurveto{\pgfqpoint{1.349486in}{0.765429in}}{\pgfqpoint{1.345157in}{0.775879in}}{\pgfqpoint{1.337453in}{0.783584in}}%
\pgfpathcurveto{\pgfqpoint{1.329748in}{0.791288in}}{\pgfqpoint{1.319298in}{0.795617in}}{\pgfqpoint{1.308402in}{0.795617in}}%
\pgfpathcurveto{\pgfqpoint{1.297506in}{0.795617in}}{\pgfqpoint{1.287056in}{0.791288in}}{\pgfqpoint{1.279351in}{0.783584in}}%
\pgfpathcurveto{\pgfqpoint{1.271647in}{0.775879in}}{\pgfqpoint{1.267318in}{0.765429in}}{\pgfqpoint{1.267318in}{0.754533in}}%
\pgfpathcurveto{\pgfqpoint{1.267318in}{0.743637in}}{\pgfqpoint{1.271647in}{0.733187in}}{\pgfqpoint{1.279351in}{0.725482in}}%
\pgfpathcurveto{\pgfqpoint{1.287056in}{0.717778in}}{\pgfqpoint{1.297506in}{0.713449in}}{\pgfqpoint{1.308402in}{0.713449in}}%
\pgfpathlineto{\pgfqpoint{1.308402in}{0.713449in}}%
\pgfpathclose%
\pgfusepath{stroke}%
\end{pgfscope}%
\begin{pgfscope}%
\pgfpathrectangle{\pgfqpoint{0.688192in}{0.670138in}}{\pgfqpoint{7.111808in}{5.129862in}}%
\pgfusepath{clip}%
\pgfsetbuttcap%
\pgfsetroundjoin%
\pgfsetlinewidth{1.003750pt}%
\definecolor{currentstroke}{rgb}{0.000000,0.000000,0.000000}%
\pgfsetstrokecolor{currentstroke}%
\pgfsetdash{}{0pt}%
\pgfpathmoveto{\pgfqpoint{1.685511in}{0.701118in}}%
\pgfpathcurveto{\pgfqpoint{1.696407in}{0.701118in}}{\pgfqpoint{1.706857in}{0.705447in}}{\pgfqpoint{1.714562in}{0.713151in}}%
\pgfpathcurveto{\pgfqpoint{1.722266in}{0.720855in}}{\pgfqpoint{1.726595in}{0.731306in}}{\pgfqpoint{1.726595in}{0.742202in}}%
\pgfpathcurveto{\pgfqpoint{1.726595in}{0.753097in}}{\pgfqpoint{1.722266in}{0.763548in}}{\pgfqpoint{1.714562in}{0.771252in}}%
\pgfpathcurveto{\pgfqpoint{1.706857in}{0.778957in}}{\pgfqpoint{1.696407in}{0.783286in}}{\pgfqpoint{1.685511in}{0.783286in}}%
\pgfpathcurveto{\pgfqpoint{1.674615in}{0.783286in}}{\pgfqpoint{1.664165in}{0.778957in}}{\pgfqpoint{1.656460in}{0.771252in}}%
\pgfpathcurveto{\pgfqpoint{1.648756in}{0.763548in}}{\pgfqpoint{1.644427in}{0.753097in}}{\pgfqpoint{1.644427in}{0.742202in}}%
\pgfpathcurveto{\pgfqpoint{1.644427in}{0.731306in}}{\pgfqpoint{1.648756in}{0.720855in}}{\pgfqpoint{1.656460in}{0.713151in}}%
\pgfpathcurveto{\pgfqpoint{1.664165in}{0.705447in}}{\pgfqpoint{1.674615in}{0.701118in}}{\pgfqpoint{1.685511in}{0.701118in}}%
\pgfpathlineto{\pgfqpoint{1.685511in}{0.701118in}}%
\pgfpathclose%
\pgfusepath{stroke}%
\end{pgfscope}%
\begin{pgfscope}%
\pgfpathrectangle{\pgfqpoint{0.688192in}{0.670138in}}{\pgfqpoint{7.111808in}{5.129862in}}%
\pgfusepath{clip}%
\pgfsetbuttcap%
\pgfsetroundjoin%
\pgfsetlinewidth{1.003750pt}%
\definecolor{currentstroke}{rgb}{0.000000,0.000000,0.000000}%
\pgfsetstrokecolor{currentstroke}%
\pgfsetdash{}{0pt}%
\pgfpathmoveto{\pgfqpoint{1.351667in}{0.711476in}}%
\pgfpathcurveto{\pgfqpoint{1.362562in}{0.711476in}}{\pgfqpoint{1.373013in}{0.715805in}}{\pgfqpoint{1.380718in}{0.723510in}}%
\pgfpathcurveto{\pgfqpoint{1.388422in}{0.731214in}}{\pgfqpoint{1.392751in}{0.741665in}}{\pgfqpoint{1.392751in}{0.752560in}}%
\pgfpathcurveto{\pgfqpoint{1.392751in}{0.763456in}}{\pgfqpoint{1.388422in}{0.773907in}}{\pgfqpoint{1.380718in}{0.781611in}}%
\pgfpathcurveto{\pgfqpoint{1.373013in}{0.789315in}}{\pgfqpoint{1.362562in}{0.793644in}}{\pgfqpoint{1.351667in}{0.793644in}}%
\pgfpathcurveto{\pgfqpoint{1.340771in}{0.793644in}}{\pgfqpoint{1.330321in}{0.789315in}}{\pgfqpoint{1.322616in}{0.781611in}}%
\pgfpathcurveto{\pgfqpoint{1.314912in}{0.773907in}}{\pgfqpoint{1.310583in}{0.763456in}}{\pgfqpoint{1.310583in}{0.752560in}}%
\pgfpathcurveto{\pgfqpoint{1.310583in}{0.741665in}}{\pgfqpoint{1.314912in}{0.731214in}}{\pgfqpoint{1.322616in}{0.723510in}}%
\pgfpathcurveto{\pgfqpoint{1.330321in}{0.715805in}}{\pgfqpoint{1.340771in}{0.711476in}}{\pgfqpoint{1.351667in}{0.711476in}}%
\pgfpathlineto{\pgfqpoint{1.351667in}{0.711476in}}%
\pgfpathclose%
\pgfusepath{stroke}%
\end{pgfscope}%
\begin{pgfscope}%
\pgfpathrectangle{\pgfqpoint{0.688192in}{0.670138in}}{\pgfqpoint{7.111808in}{5.129862in}}%
\pgfusepath{clip}%
\pgfsetbuttcap%
\pgfsetroundjoin%
\pgfsetlinewidth{1.003750pt}%
\definecolor{currentstroke}{rgb}{0.000000,0.000000,0.000000}%
\pgfsetstrokecolor{currentstroke}%
\pgfsetdash{}{0pt}%
\pgfpathmoveto{\pgfqpoint{0.928243in}{0.841597in}}%
\pgfpathcurveto{\pgfqpoint{0.939139in}{0.841597in}}{\pgfqpoint{0.949589in}{0.845926in}}{\pgfqpoint{0.957294in}{0.853631in}}%
\pgfpathcurveto{\pgfqpoint{0.964998in}{0.861335in}}{\pgfqpoint{0.969327in}{0.871786in}}{\pgfqpoint{0.969327in}{0.882681in}}%
\pgfpathcurveto{\pgfqpoint{0.969327in}{0.893577in}}{\pgfqpoint{0.964998in}{0.904028in}}{\pgfqpoint{0.957294in}{0.911732in}}%
\pgfpathcurveto{\pgfqpoint{0.949589in}{0.919436in}}{\pgfqpoint{0.939139in}{0.923765in}}{\pgfqpoint{0.928243in}{0.923765in}}%
\pgfpathcurveto{\pgfqpoint{0.917347in}{0.923765in}}{\pgfqpoint{0.906897in}{0.919436in}}{\pgfqpoint{0.899192in}{0.911732in}}%
\pgfpathcurveto{\pgfqpoint{0.891488in}{0.904028in}}{\pgfqpoint{0.887159in}{0.893577in}}{\pgfqpoint{0.887159in}{0.882681in}}%
\pgfpathcurveto{\pgfqpoint{0.887159in}{0.871786in}}{\pgfqpoint{0.891488in}{0.861335in}}{\pgfqpoint{0.899192in}{0.853631in}}%
\pgfpathcurveto{\pgfqpoint{0.906897in}{0.845926in}}{\pgfqpoint{0.917347in}{0.841597in}}{\pgfqpoint{0.928243in}{0.841597in}}%
\pgfpathlineto{\pgfqpoint{0.928243in}{0.841597in}}%
\pgfpathclose%
\pgfusepath{stroke}%
\end{pgfscope}%
\begin{pgfscope}%
\pgfpathrectangle{\pgfqpoint{0.688192in}{0.670138in}}{\pgfqpoint{7.111808in}{5.129862in}}%
\pgfusepath{clip}%
\pgfsetbuttcap%
\pgfsetroundjoin%
\pgfsetlinewidth{1.003750pt}%
\definecolor{currentstroke}{rgb}{0.000000,0.000000,0.000000}%
\pgfsetstrokecolor{currentstroke}%
\pgfsetdash{}{0pt}%
\pgfpathmoveto{\pgfqpoint{5.988842in}{0.629054in}}%
\pgfpathcurveto{\pgfqpoint{5.999738in}{0.629054in}}{\pgfqpoint{6.010189in}{0.633383in}}{\pgfqpoint{6.017893in}{0.641087in}}%
\pgfpathcurveto{\pgfqpoint{6.025598in}{0.648792in}}{\pgfqpoint{6.029926in}{0.659242in}}{\pgfqpoint{6.029926in}{0.670138in}}%
\pgfpathcurveto{\pgfqpoint{6.029926in}{0.681034in}}{\pgfqpoint{6.025598in}{0.691484in}}{\pgfqpoint{6.017893in}{0.699189in}}%
\pgfpathcurveto{\pgfqpoint{6.010189in}{0.706893in}}{\pgfqpoint{5.999738in}{0.711222in}}{\pgfqpoint{5.988842in}{0.711222in}}%
\pgfpathcurveto{\pgfqpoint{5.977947in}{0.711222in}}{\pgfqpoint{5.967496in}{0.706893in}}{\pgfqpoint{5.959792in}{0.699189in}}%
\pgfpathcurveto{\pgfqpoint{5.952087in}{0.691484in}}{\pgfqpoint{5.947759in}{0.681034in}}{\pgfqpoint{5.947759in}{0.670138in}}%
\pgfpathcurveto{\pgfqpoint{5.947759in}{0.659242in}}{\pgfqpoint{5.952087in}{0.648792in}}{\pgfqpoint{5.959792in}{0.641087in}}%
\pgfpathcurveto{\pgfqpoint{5.967496in}{0.633383in}}{\pgfqpoint{5.977947in}{0.629054in}}{\pgfqpoint{5.988842in}{0.629054in}}%
\pgfusepath{stroke}%
\end{pgfscope}%
\begin{pgfscope}%
\pgfpathrectangle{\pgfqpoint{0.688192in}{0.670138in}}{\pgfqpoint{7.111808in}{5.129862in}}%
\pgfusepath{clip}%
\pgfsetbuttcap%
\pgfsetroundjoin%
\pgfsetlinewidth{1.003750pt}%
\definecolor{currentstroke}{rgb}{0.000000,0.000000,0.000000}%
\pgfsetstrokecolor{currentstroke}%
\pgfsetdash{}{0pt}%
\pgfpathmoveto{\pgfqpoint{5.621788in}{3.919132in}}%
\pgfpathcurveto{\pgfqpoint{5.632683in}{3.919132in}}{\pgfqpoint{5.643134in}{3.923461in}}{\pgfqpoint{5.650838in}{3.931165in}}%
\pgfpathcurveto{\pgfqpoint{5.658543in}{3.938870in}}{\pgfqpoint{5.662872in}{3.949321in}}{\pgfqpoint{5.662872in}{3.960216in}}%
\pgfpathcurveto{\pgfqpoint{5.662872in}{3.971112in}}{\pgfqpoint{5.658543in}{3.981563in}}{\pgfqpoint{5.650838in}{3.989267in}}%
\pgfpathcurveto{\pgfqpoint{5.643134in}{3.996971in}}{\pgfqpoint{5.632683in}{4.001300in}}{\pgfqpoint{5.621788in}{4.001300in}}%
\pgfpathcurveto{\pgfqpoint{5.610892in}{4.001300in}}{\pgfqpoint{5.600441in}{3.996971in}}{\pgfqpoint{5.592737in}{3.989267in}}%
\pgfpathcurveto{\pgfqpoint{5.585033in}{3.981563in}}{\pgfqpoint{5.580704in}{3.971112in}}{\pgfqpoint{5.580704in}{3.960216in}}%
\pgfpathcurveto{\pgfqpoint{5.580704in}{3.949321in}}{\pgfqpoint{5.585033in}{3.938870in}}{\pgfqpoint{5.592737in}{3.931165in}}%
\pgfpathcurveto{\pgfqpoint{5.600441in}{3.923461in}}{\pgfqpoint{5.610892in}{3.919132in}}{\pgfqpoint{5.621788in}{3.919132in}}%
\pgfpathlineto{\pgfqpoint{5.621788in}{3.919132in}}%
\pgfpathclose%
\pgfusepath{stroke}%
\end{pgfscope}%
\begin{pgfscope}%
\pgfpathrectangle{\pgfqpoint{0.688192in}{0.670138in}}{\pgfqpoint{7.111808in}{5.129862in}}%
\pgfusepath{clip}%
\pgfsetbuttcap%
\pgfsetroundjoin%
\pgfsetlinewidth{1.003750pt}%
\definecolor{currentstroke}{rgb}{0.000000,0.000000,0.000000}%
\pgfsetstrokecolor{currentstroke}%
\pgfsetdash{}{0pt}%
\pgfpathmoveto{\pgfqpoint{4.176025in}{0.641739in}}%
\pgfpathcurveto{\pgfqpoint{4.186921in}{0.641739in}}{\pgfqpoint{4.197371in}{0.646068in}}{\pgfqpoint{4.205076in}{0.653772in}}%
\pgfpathcurveto{\pgfqpoint{4.212780in}{0.661476in}}{\pgfqpoint{4.217109in}{0.671927in}}{\pgfqpoint{4.217109in}{0.682823in}}%
\pgfpathcurveto{\pgfqpoint{4.217109in}{0.693718in}}{\pgfqpoint{4.212780in}{0.704169in}}{\pgfqpoint{4.205076in}{0.711873in}}%
\pgfpathcurveto{\pgfqpoint{4.197371in}{0.719578in}}{\pgfqpoint{4.186921in}{0.723906in}}{\pgfqpoint{4.176025in}{0.723906in}}%
\pgfpathcurveto{\pgfqpoint{4.165129in}{0.723906in}}{\pgfqpoint{4.154679in}{0.719578in}}{\pgfqpoint{4.146974in}{0.711873in}}%
\pgfpathcurveto{\pgfqpoint{4.139270in}{0.704169in}}{\pgfqpoint{4.134941in}{0.693718in}}{\pgfqpoint{4.134941in}{0.682823in}}%
\pgfpathcurveto{\pgfqpoint{4.134941in}{0.671927in}}{\pgfqpoint{4.139270in}{0.661476in}}{\pgfqpoint{4.146974in}{0.653772in}}%
\pgfpathcurveto{\pgfqpoint{4.154679in}{0.646068in}}{\pgfqpoint{4.165129in}{0.641739in}}{\pgfqpoint{4.176025in}{0.641739in}}%
\pgfusepath{stroke}%
\end{pgfscope}%
\begin{pgfscope}%
\pgfpathrectangle{\pgfqpoint{0.688192in}{0.670138in}}{\pgfqpoint{7.111808in}{5.129862in}}%
\pgfusepath{clip}%
\pgfsetbuttcap%
\pgfsetroundjoin%
\pgfsetlinewidth{1.003750pt}%
\definecolor{currentstroke}{rgb}{0.000000,0.000000,0.000000}%
\pgfsetstrokecolor{currentstroke}%
\pgfsetdash{}{0pt}%
\pgfpathmoveto{\pgfqpoint{1.044526in}{0.742721in}}%
\pgfpathcurveto{\pgfqpoint{1.055421in}{0.742721in}}{\pgfqpoint{1.065872in}{0.747050in}}{\pgfqpoint{1.073577in}{0.754754in}}%
\pgfpathcurveto{\pgfqpoint{1.081281in}{0.762459in}}{\pgfqpoint{1.085610in}{0.772909in}}{\pgfqpoint{1.085610in}{0.783805in}}%
\pgfpathcurveto{\pgfqpoint{1.085610in}{0.794701in}}{\pgfqpoint{1.081281in}{0.805151in}}{\pgfqpoint{1.073577in}{0.812856in}}%
\pgfpathcurveto{\pgfqpoint{1.065872in}{0.820560in}}{\pgfqpoint{1.055421in}{0.824889in}}{\pgfqpoint{1.044526in}{0.824889in}}%
\pgfpathcurveto{\pgfqpoint{1.033630in}{0.824889in}}{\pgfqpoint{1.023179in}{0.820560in}}{\pgfqpoint{1.015475in}{0.812856in}}%
\pgfpathcurveto{\pgfqpoint{1.007771in}{0.805151in}}{\pgfqpoint{1.003442in}{0.794701in}}{\pgfqpoint{1.003442in}{0.783805in}}%
\pgfpathcurveto{\pgfqpoint{1.003442in}{0.772909in}}{\pgfqpoint{1.007771in}{0.762459in}}{\pgfqpoint{1.015475in}{0.754754in}}%
\pgfpathcurveto{\pgfqpoint{1.023179in}{0.747050in}}{\pgfqpoint{1.033630in}{0.742721in}}{\pgfqpoint{1.044526in}{0.742721in}}%
\pgfpathlineto{\pgfqpoint{1.044526in}{0.742721in}}%
\pgfpathclose%
\pgfusepath{stroke}%
\end{pgfscope}%
\begin{pgfscope}%
\pgfpathrectangle{\pgfqpoint{0.688192in}{0.670138in}}{\pgfqpoint{7.111808in}{5.129862in}}%
\pgfusepath{clip}%
\pgfsetbuttcap%
\pgfsetroundjoin%
\pgfsetlinewidth{1.003750pt}%
\definecolor{currentstroke}{rgb}{0.000000,0.000000,0.000000}%
\pgfsetstrokecolor{currentstroke}%
\pgfsetdash{}{0pt}%
\pgfpathmoveto{\pgfqpoint{4.954913in}{0.636044in}}%
\pgfpathcurveto{\pgfqpoint{4.965809in}{0.636044in}}{\pgfqpoint{4.976259in}{0.640373in}}{\pgfqpoint{4.983964in}{0.648077in}}%
\pgfpathcurveto{\pgfqpoint{4.991668in}{0.655781in}}{\pgfqpoint{4.995997in}{0.666232in}}{\pgfqpoint{4.995997in}{0.677128in}}%
\pgfpathcurveto{\pgfqpoint{4.995997in}{0.688023in}}{\pgfqpoint{4.991668in}{0.698474in}}{\pgfqpoint{4.983964in}{0.706179in}}%
\pgfpathcurveto{\pgfqpoint{4.976259in}{0.713883in}}{\pgfqpoint{4.965809in}{0.718212in}}{\pgfqpoint{4.954913in}{0.718212in}}%
\pgfpathcurveto{\pgfqpoint{4.944017in}{0.718212in}}{\pgfqpoint{4.933567in}{0.713883in}}{\pgfqpoint{4.925862in}{0.706179in}}%
\pgfpathcurveto{\pgfqpoint{4.918158in}{0.698474in}}{\pgfqpoint{4.913829in}{0.688023in}}{\pgfqpoint{4.913829in}{0.677128in}}%
\pgfpathcurveto{\pgfqpoint{4.913829in}{0.666232in}}{\pgfqpoint{4.918158in}{0.655781in}}{\pgfqpoint{4.925862in}{0.648077in}}%
\pgfpathcurveto{\pgfqpoint{4.933567in}{0.640373in}}{\pgfqpoint{4.944017in}{0.636044in}}{\pgfqpoint{4.954913in}{0.636044in}}%
\pgfusepath{stroke}%
\end{pgfscope}%
\begin{pgfscope}%
\pgfpathrectangle{\pgfqpoint{0.688192in}{0.670138in}}{\pgfqpoint{7.111808in}{5.129862in}}%
\pgfusepath{clip}%
\pgfsetbuttcap%
\pgfsetroundjoin%
\pgfsetlinewidth{1.003750pt}%
\definecolor{currentstroke}{rgb}{0.000000,0.000000,0.000000}%
\pgfsetstrokecolor{currentstroke}%
\pgfsetdash{}{0pt}%
\pgfpathmoveto{\pgfqpoint{4.279209in}{1.434102in}}%
\pgfpathcurveto{\pgfqpoint{4.290104in}{1.434102in}}{\pgfqpoint{4.300555in}{1.438430in}}{\pgfqpoint{4.308259in}{1.446135in}}%
\pgfpathcurveto{\pgfqpoint{4.315964in}{1.453839in}}{\pgfqpoint{4.320293in}{1.464290in}}{\pgfqpoint{4.320293in}{1.475185in}}%
\pgfpathcurveto{\pgfqpoint{4.320293in}{1.486081in}}{\pgfqpoint{4.315964in}{1.496532in}}{\pgfqpoint{4.308259in}{1.504236in}}%
\pgfpathcurveto{\pgfqpoint{4.300555in}{1.511940in}}{\pgfqpoint{4.290104in}{1.516269in}}{\pgfqpoint{4.279209in}{1.516269in}}%
\pgfpathcurveto{\pgfqpoint{4.268313in}{1.516269in}}{\pgfqpoint{4.257862in}{1.511940in}}{\pgfqpoint{4.250158in}{1.504236in}}%
\pgfpathcurveto{\pgfqpoint{4.242454in}{1.496532in}}{\pgfqpoint{4.238125in}{1.486081in}}{\pgfqpoint{4.238125in}{1.475185in}}%
\pgfpathcurveto{\pgfqpoint{4.238125in}{1.464290in}}{\pgfqpoint{4.242454in}{1.453839in}}{\pgfqpoint{4.250158in}{1.446135in}}%
\pgfpathcurveto{\pgfqpoint{4.257862in}{1.438430in}}{\pgfqpoint{4.268313in}{1.434102in}}{\pgfqpoint{4.279209in}{1.434102in}}%
\pgfpathlineto{\pgfqpoint{4.279209in}{1.434102in}}%
\pgfpathclose%
\pgfusepath{stroke}%
\end{pgfscope}%
\begin{pgfscope}%
\pgfpathrectangle{\pgfqpoint{0.688192in}{0.670138in}}{\pgfqpoint{7.111808in}{5.129862in}}%
\pgfusepath{clip}%
\pgfsetbuttcap%
\pgfsetroundjoin%
\pgfsetlinewidth{1.003750pt}%
\definecolor{currentstroke}{rgb}{0.000000,0.000000,0.000000}%
\pgfsetstrokecolor{currentstroke}%
\pgfsetdash{}{0pt}%
\pgfpathmoveto{\pgfqpoint{2.346054in}{0.680598in}}%
\pgfpathcurveto{\pgfqpoint{2.356949in}{0.680598in}}{\pgfqpoint{2.367400in}{0.684927in}}{\pgfqpoint{2.375104in}{0.692631in}}%
\pgfpathcurveto{\pgfqpoint{2.382809in}{0.700336in}}{\pgfqpoint{2.387137in}{0.710787in}}{\pgfqpoint{2.387137in}{0.721682in}}%
\pgfpathcurveto{\pgfqpoint{2.387137in}{0.732578in}}{\pgfqpoint{2.382809in}{0.743028in}}{\pgfqpoint{2.375104in}{0.750733in}}%
\pgfpathcurveto{\pgfqpoint{2.367400in}{0.758437in}}{\pgfqpoint{2.356949in}{0.762766in}}{\pgfqpoint{2.346054in}{0.762766in}}%
\pgfpathcurveto{\pgfqpoint{2.335158in}{0.762766in}}{\pgfqpoint{2.324707in}{0.758437in}}{\pgfqpoint{2.317003in}{0.750733in}}%
\pgfpathcurveto{\pgfqpoint{2.309299in}{0.743028in}}{\pgfqpoint{2.304970in}{0.732578in}}{\pgfqpoint{2.304970in}{0.721682in}}%
\pgfpathcurveto{\pgfqpoint{2.304970in}{0.710787in}}{\pgfqpoint{2.309299in}{0.700336in}}{\pgfqpoint{2.317003in}{0.692631in}}%
\pgfpathcurveto{\pgfqpoint{2.324707in}{0.684927in}}{\pgfqpoint{2.335158in}{0.680598in}}{\pgfqpoint{2.346054in}{0.680598in}}%
\pgfpathlineto{\pgfqpoint{2.346054in}{0.680598in}}%
\pgfpathclose%
\pgfusepath{stroke}%
\end{pgfscope}%
\begin{pgfscope}%
\pgfpathrectangle{\pgfqpoint{0.688192in}{0.670138in}}{\pgfqpoint{7.111808in}{5.129862in}}%
\pgfusepath{clip}%
\pgfsetbuttcap%
\pgfsetroundjoin%
\pgfsetlinewidth{1.003750pt}%
\definecolor{currentstroke}{rgb}{0.000000,0.000000,0.000000}%
\pgfsetstrokecolor{currentstroke}%
\pgfsetdash{}{0pt}%
\pgfpathmoveto{\pgfqpoint{0.886861in}{0.869726in}}%
\pgfpathcurveto{\pgfqpoint{0.897756in}{0.869726in}}{\pgfqpoint{0.908207in}{0.874055in}}{\pgfqpoint{0.915911in}{0.881759in}}%
\pgfpathcurveto{\pgfqpoint{0.923616in}{0.889463in}}{\pgfqpoint{0.927945in}{0.899914in}}{\pgfqpoint{0.927945in}{0.910810in}}%
\pgfpathcurveto{\pgfqpoint{0.927945in}{0.921705in}}{\pgfqpoint{0.923616in}{0.932156in}}{\pgfqpoint{0.915911in}{0.939861in}}%
\pgfpathcurveto{\pgfqpoint{0.908207in}{0.947565in}}{\pgfqpoint{0.897756in}{0.951894in}}{\pgfqpoint{0.886861in}{0.951894in}}%
\pgfpathcurveto{\pgfqpoint{0.875965in}{0.951894in}}{\pgfqpoint{0.865514in}{0.947565in}}{\pgfqpoint{0.857810in}{0.939861in}}%
\pgfpathcurveto{\pgfqpoint{0.850106in}{0.932156in}}{\pgfqpoint{0.845777in}{0.921705in}}{\pgfqpoint{0.845777in}{0.910810in}}%
\pgfpathcurveto{\pgfqpoint{0.845777in}{0.899914in}}{\pgfqpoint{0.850106in}{0.889463in}}{\pgfqpoint{0.857810in}{0.881759in}}%
\pgfpathcurveto{\pgfqpoint{0.865514in}{0.874055in}}{\pgfqpoint{0.875965in}{0.869726in}}{\pgfqpoint{0.886861in}{0.869726in}}%
\pgfpathlineto{\pgfqpoint{0.886861in}{0.869726in}}%
\pgfpathclose%
\pgfusepath{stroke}%
\end{pgfscope}%
\begin{pgfscope}%
\pgfpathrectangle{\pgfqpoint{0.688192in}{0.670138in}}{\pgfqpoint{7.111808in}{5.129862in}}%
\pgfusepath{clip}%
\pgfsetbuttcap%
\pgfsetroundjoin%
\pgfsetlinewidth{1.003750pt}%
\definecolor{currentstroke}{rgb}{0.000000,0.000000,0.000000}%
\pgfsetstrokecolor{currentstroke}%
\pgfsetdash{}{0pt}%
\pgfpathmoveto{\pgfqpoint{2.371204in}{0.938172in}}%
\pgfpathcurveto{\pgfqpoint{2.382100in}{0.938172in}}{\pgfqpoint{2.392551in}{0.942501in}}{\pgfqpoint{2.400255in}{0.950205in}}%
\pgfpathcurveto{\pgfqpoint{2.407959in}{0.957910in}}{\pgfqpoint{2.412288in}{0.968361in}}{\pgfqpoint{2.412288in}{0.979256in}}%
\pgfpathcurveto{\pgfqpoint{2.412288in}{0.990152in}}{\pgfqpoint{2.407959in}{1.000602in}}{\pgfqpoint{2.400255in}{1.008307in}}%
\pgfpathcurveto{\pgfqpoint{2.392551in}{1.016011in}}{\pgfqpoint{2.382100in}{1.020340in}}{\pgfqpoint{2.371204in}{1.020340in}}%
\pgfpathcurveto{\pgfqpoint{2.360309in}{1.020340in}}{\pgfqpoint{2.349858in}{1.016011in}}{\pgfqpoint{2.342153in}{1.008307in}}%
\pgfpathcurveto{\pgfqpoint{2.334449in}{1.000602in}}{\pgfqpoint{2.330120in}{0.990152in}}{\pgfqpoint{2.330120in}{0.979256in}}%
\pgfpathcurveto{\pgfqpoint{2.330120in}{0.968361in}}{\pgfqpoint{2.334449in}{0.957910in}}{\pgfqpoint{2.342153in}{0.950205in}}%
\pgfpathcurveto{\pgfqpoint{2.349858in}{0.942501in}}{\pgfqpoint{2.360309in}{0.938172in}}{\pgfqpoint{2.371204in}{0.938172in}}%
\pgfpathlineto{\pgfqpoint{2.371204in}{0.938172in}}%
\pgfpathclose%
\pgfusepath{stroke}%
\end{pgfscope}%
\begin{pgfscope}%
\pgfpathrectangle{\pgfqpoint{0.688192in}{0.670138in}}{\pgfqpoint{7.111808in}{5.129862in}}%
\pgfusepath{clip}%
\pgfsetbuttcap%
\pgfsetroundjoin%
\pgfsetlinewidth{1.003750pt}%
\definecolor{currentstroke}{rgb}{0.000000,0.000000,0.000000}%
\pgfsetstrokecolor{currentstroke}%
\pgfsetdash{}{0pt}%
\pgfpathmoveto{\pgfqpoint{0.881277in}{0.900428in}}%
\pgfpathcurveto{\pgfqpoint{0.892173in}{0.900428in}}{\pgfqpoint{0.902624in}{0.904756in}}{\pgfqpoint{0.910328in}{0.912461in}}%
\pgfpathcurveto{\pgfqpoint{0.918032in}{0.920165in}}{\pgfqpoint{0.922361in}{0.930616in}}{\pgfqpoint{0.922361in}{0.941511in}}%
\pgfpathcurveto{\pgfqpoint{0.922361in}{0.952407in}}{\pgfqpoint{0.918032in}{0.962858in}}{\pgfqpoint{0.910328in}{0.970562in}}%
\pgfpathcurveto{\pgfqpoint{0.902624in}{0.978266in}}{\pgfqpoint{0.892173in}{0.982595in}}{\pgfqpoint{0.881277in}{0.982595in}}%
\pgfpathcurveto{\pgfqpoint{0.870382in}{0.982595in}}{\pgfqpoint{0.859931in}{0.978266in}}{\pgfqpoint{0.852226in}{0.970562in}}%
\pgfpathcurveto{\pgfqpoint{0.844522in}{0.962858in}}{\pgfqpoint{0.840193in}{0.952407in}}{\pgfqpoint{0.840193in}{0.941511in}}%
\pgfpathcurveto{\pgfqpoint{0.840193in}{0.930616in}}{\pgfqpoint{0.844522in}{0.920165in}}{\pgfqpoint{0.852226in}{0.912461in}}%
\pgfpathcurveto{\pgfqpoint{0.859931in}{0.904756in}}{\pgfqpoint{0.870382in}{0.900428in}}{\pgfqpoint{0.881277in}{0.900428in}}%
\pgfpathlineto{\pgfqpoint{0.881277in}{0.900428in}}%
\pgfpathclose%
\pgfusepath{stroke}%
\end{pgfscope}%
\begin{pgfscope}%
\pgfpathrectangle{\pgfqpoint{0.688192in}{0.670138in}}{\pgfqpoint{7.111808in}{5.129862in}}%
\pgfusepath{clip}%
\pgfsetbuttcap%
\pgfsetroundjoin%
\pgfsetlinewidth{1.003750pt}%
\definecolor{currentstroke}{rgb}{0.000000,0.000000,0.000000}%
\pgfsetstrokecolor{currentstroke}%
\pgfsetdash{}{0pt}%
\pgfpathmoveto{\pgfqpoint{0.931763in}{0.833664in}}%
\pgfpathcurveto{\pgfqpoint{0.942659in}{0.833664in}}{\pgfqpoint{0.953110in}{0.837992in}}{\pgfqpoint{0.960814in}{0.845697in}}%
\pgfpathcurveto{\pgfqpoint{0.968519in}{0.853401in}}{\pgfqpoint{0.972847in}{0.863852in}}{\pgfqpoint{0.972847in}{0.874747in}}%
\pgfpathcurveto{\pgfqpoint{0.972847in}{0.885643in}}{\pgfqpoint{0.968519in}{0.896094in}}{\pgfqpoint{0.960814in}{0.903798in}}%
\pgfpathcurveto{\pgfqpoint{0.953110in}{0.911502in}}{\pgfqpoint{0.942659in}{0.915831in}}{\pgfqpoint{0.931763in}{0.915831in}}%
\pgfpathcurveto{\pgfqpoint{0.920868in}{0.915831in}}{\pgfqpoint{0.910417in}{0.911502in}}{\pgfqpoint{0.902713in}{0.903798in}}%
\pgfpathcurveto{\pgfqpoint{0.895008in}{0.896094in}}{\pgfqpoint{0.890680in}{0.885643in}}{\pgfqpoint{0.890680in}{0.874747in}}%
\pgfpathcurveto{\pgfqpoint{0.890680in}{0.863852in}}{\pgfqpoint{0.895008in}{0.853401in}}{\pgfqpoint{0.902713in}{0.845697in}}%
\pgfpathcurveto{\pgfqpoint{0.910417in}{0.837992in}}{\pgfqpoint{0.920868in}{0.833664in}}{\pgfqpoint{0.931763in}{0.833664in}}%
\pgfpathlineto{\pgfqpoint{0.931763in}{0.833664in}}%
\pgfpathclose%
\pgfusepath{stroke}%
\end{pgfscope}%
\begin{pgfscope}%
\pgfpathrectangle{\pgfqpoint{0.688192in}{0.670138in}}{\pgfqpoint{7.111808in}{5.129862in}}%
\pgfusepath{clip}%
\pgfsetbuttcap%
\pgfsetroundjoin%
\pgfsetlinewidth{1.003750pt}%
\definecolor{currentstroke}{rgb}{0.000000,0.000000,0.000000}%
\pgfsetstrokecolor{currentstroke}%
\pgfsetdash{}{0pt}%
\pgfpathmoveto{\pgfqpoint{0.794194in}{1.022605in}}%
\pgfpathcurveto{\pgfqpoint{0.805090in}{1.022605in}}{\pgfqpoint{0.815541in}{1.026933in}}{\pgfqpoint{0.823245in}{1.034638in}}%
\pgfpathcurveto{\pgfqpoint{0.830949in}{1.042342in}}{\pgfqpoint{0.835278in}{1.052793in}}{\pgfqpoint{0.835278in}{1.063688in}}%
\pgfpathcurveto{\pgfqpoint{0.835278in}{1.074584in}}{\pgfqpoint{0.830949in}{1.085035in}}{\pgfqpoint{0.823245in}{1.092739in}}%
\pgfpathcurveto{\pgfqpoint{0.815541in}{1.100443in}}{\pgfqpoint{0.805090in}{1.104772in}}{\pgfqpoint{0.794194in}{1.104772in}}%
\pgfpathcurveto{\pgfqpoint{0.783299in}{1.104772in}}{\pgfqpoint{0.772848in}{1.100443in}}{\pgfqpoint{0.765144in}{1.092739in}}%
\pgfpathcurveto{\pgfqpoint{0.757439in}{1.085035in}}{\pgfqpoint{0.753110in}{1.074584in}}{\pgfqpoint{0.753110in}{1.063688in}}%
\pgfpathcurveto{\pgfqpoint{0.753110in}{1.052793in}}{\pgfqpoint{0.757439in}{1.042342in}}{\pgfqpoint{0.765144in}{1.034638in}}%
\pgfpathcurveto{\pgfqpoint{0.772848in}{1.026933in}}{\pgfqpoint{0.783299in}{1.022605in}}{\pgfqpoint{0.794194in}{1.022605in}}%
\pgfpathlineto{\pgfqpoint{0.794194in}{1.022605in}}%
\pgfpathclose%
\pgfusepath{stroke}%
\end{pgfscope}%
\begin{pgfscope}%
\pgfpathrectangle{\pgfqpoint{0.688192in}{0.670138in}}{\pgfqpoint{7.111808in}{5.129862in}}%
\pgfusepath{clip}%
\pgfsetbuttcap%
\pgfsetroundjoin%
\pgfsetlinewidth{1.003750pt}%
\definecolor{currentstroke}{rgb}{0.000000,0.000000,0.000000}%
\pgfsetstrokecolor{currentstroke}%
\pgfsetdash{}{0pt}%
\pgfpathmoveto{\pgfqpoint{1.101880in}{0.720677in}}%
\pgfpathcurveto{\pgfqpoint{1.112775in}{0.720677in}}{\pgfqpoint{1.123226in}{0.725006in}}{\pgfqpoint{1.130930in}{0.732710in}}%
\pgfpathcurveto{\pgfqpoint{1.138635in}{0.740415in}}{\pgfqpoint{1.142964in}{0.750865in}}{\pgfqpoint{1.142964in}{0.761761in}}%
\pgfpathcurveto{\pgfqpoint{1.142964in}{0.772656in}}{\pgfqpoint{1.138635in}{0.783107in}}{\pgfqpoint{1.130930in}{0.790812in}}%
\pgfpathcurveto{\pgfqpoint{1.123226in}{0.798516in}}{\pgfqpoint{1.112775in}{0.802845in}}{\pgfqpoint{1.101880in}{0.802845in}}%
\pgfpathcurveto{\pgfqpoint{1.090984in}{0.802845in}}{\pgfqpoint{1.080533in}{0.798516in}}{\pgfqpoint{1.072829in}{0.790812in}}%
\pgfpathcurveto{\pgfqpoint{1.065125in}{0.783107in}}{\pgfqpoint{1.060796in}{0.772656in}}{\pgfqpoint{1.060796in}{0.761761in}}%
\pgfpathcurveto{\pgfqpoint{1.060796in}{0.750865in}}{\pgfqpoint{1.065125in}{0.740415in}}{\pgfqpoint{1.072829in}{0.732710in}}%
\pgfpathcurveto{\pgfqpoint{1.080533in}{0.725006in}}{\pgfqpoint{1.090984in}{0.720677in}}{\pgfqpoint{1.101880in}{0.720677in}}%
\pgfpathlineto{\pgfqpoint{1.101880in}{0.720677in}}%
\pgfpathclose%
\pgfusepath{stroke}%
\end{pgfscope}%
\begin{pgfscope}%
\pgfpathrectangle{\pgfqpoint{0.688192in}{0.670138in}}{\pgfqpoint{7.111808in}{5.129862in}}%
\pgfusepath{clip}%
\pgfsetbuttcap%
\pgfsetroundjoin%
\pgfsetlinewidth{1.003750pt}%
\definecolor{currentstroke}{rgb}{0.000000,0.000000,0.000000}%
\pgfsetstrokecolor{currentstroke}%
\pgfsetdash{}{0pt}%
\pgfpathmoveto{\pgfqpoint{2.272359in}{3.391657in}}%
\pgfpathcurveto{\pgfqpoint{2.283254in}{3.391657in}}{\pgfqpoint{2.293705in}{3.395986in}}{\pgfqpoint{2.301409in}{3.403690in}}%
\pgfpathcurveto{\pgfqpoint{2.309114in}{3.411395in}}{\pgfqpoint{2.313443in}{3.421845in}}{\pgfqpoint{2.313443in}{3.432741in}}%
\pgfpathcurveto{\pgfqpoint{2.313443in}{3.443637in}}{\pgfqpoint{2.309114in}{3.454087in}}{\pgfqpoint{2.301409in}{3.461792in}}%
\pgfpathcurveto{\pgfqpoint{2.293705in}{3.469496in}}{\pgfqpoint{2.283254in}{3.473825in}}{\pgfqpoint{2.272359in}{3.473825in}}%
\pgfpathcurveto{\pgfqpoint{2.261463in}{3.473825in}}{\pgfqpoint{2.251012in}{3.469496in}}{\pgfqpoint{2.243308in}{3.461792in}}%
\pgfpathcurveto{\pgfqpoint{2.235604in}{3.454087in}}{\pgfqpoint{2.231275in}{3.443637in}}{\pgfqpoint{2.231275in}{3.432741in}}%
\pgfpathcurveto{\pgfqpoint{2.231275in}{3.421845in}}{\pgfqpoint{2.235604in}{3.411395in}}{\pgfqpoint{2.243308in}{3.403690in}}%
\pgfpathcurveto{\pgfqpoint{2.251012in}{3.395986in}}{\pgfqpoint{2.261463in}{3.391657in}}{\pgfqpoint{2.272359in}{3.391657in}}%
\pgfpathlineto{\pgfqpoint{2.272359in}{3.391657in}}%
\pgfpathclose%
\pgfusepath{stroke}%
\end{pgfscope}%
\begin{pgfscope}%
\pgfpathrectangle{\pgfqpoint{0.688192in}{0.670138in}}{\pgfqpoint{7.111808in}{5.129862in}}%
\pgfusepath{clip}%
\pgfsetbuttcap%
\pgfsetroundjoin%
\pgfsetlinewidth{1.003750pt}%
\definecolor{currentstroke}{rgb}{0.000000,0.000000,0.000000}%
\pgfsetstrokecolor{currentstroke}%
\pgfsetdash{}{0pt}%
\pgfpathmoveto{\pgfqpoint{0.957757in}{0.807940in}}%
\pgfpathcurveto{\pgfqpoint{0.968653in}{0.807940in}}{\pgfqpoint{0.979103in}{0.812269in}}{\pgfqpoint{0.986808in}{0.819974in}}%
\pgfpathcurveto{\pgfqpoint{0.994512in}{0.827678in}}{\pgfqpoint{0.998841in}{0.838129in}}{\pgfqpoint{0.998841in}{0.849024in}}%
\pgfpathcurveto{\pgfqpoint{0.998841in}{0.859920in}}{\pgfqpoint{0.994512in}{0.870371in}}{\pgfqpoint{0.986808in}{0.878075in}}%
\pgfpathcurveto{\pgfqpoint{0.979103in}{0.885779in}}{\pgfqpoint{0.968653in}{0.890108in}}{\pgfqpoint{0.957757in}{0.890108in}}%
\pgfpathcurveto{\pgfqpoint{0.946862in}{0.890108in}}{\pgfqpoint{0.936411in}{0.885779in}}{\pgfqpoint{0.928706in}{0.878075in}}%
\pgfpathcurveto{\pgfqpoint{0.921002in}{0.870371in}}{\pgfqpoint{0.916673in}{0.859920in}}{\pgfqpoint{0.916673in}{0.849024in}}%
\pgfpathcurveto{\pgfqpoint{0.916673in}{0.838129in}}{\pgfqpoint{0.921002in}{0.827678in}}{\pgfqpoint{0.928706in}{0.819974in}}%
\pgfpathcurveto{\pgfqpoint{0.936411in}{0.812269in}}{\pgfqpoint{0.946862in}{0.807940in}}{\pgfqpoint{0.957757in}{0.807940in}}%
\pgfpathlineto{\pgfqpoint{0.957757in}{0.807940in}}%
\pgfpathclose%
\pgfusepath{stroke}%
\end{pgfscope}%
\begin{pgfscope}%
\pgfpathrectangle{\pgfqpoint{0.688192in}{0.670138in}}{\pgfqpoint{7.111808in}{5.129862in}}%
\pgfusepath{clip}%
\pgfsetbuttcap%
\pgfsetroundjoin%
\pgfsetlinewidth{1.003750pt}%
\definecolor{currentstroke}{rgb}{0.000000,0.000000,0.000000}%
\pgfsetstrokecolor{currentstroke}%
\pgfsetdash{}{0pt}%
\pgfpathmoveto{\pgfqpoint{1.120660in}{0.719620in}}%
\pgfpathcurveto{\pgfqpoint{1.131555in}{0.719620in}}{\pgfqpoint{1.142006in}{0.723949in}}{\pgfqpoint{1.149710in}{0.731653in}}%
\pgfpathcurveto{\pgfqpoint{1.157415in}{0.739358in}}{\pgfqpoint{1.161744in}{0.749809in}}{\pgfqpoint{1.161744in}{0.760704in}}%
\pgfpathcurveto{\pgfqpoint{1.161744in}{0.771600in}}{\pgfqpoint{1.157415in}{0.782051in}}{\pgfqpoint{1.149710in}{0.789755in}}%
\pgfpathcurveto{\pgfqpoint{1.142006in}{0.797459in}}{\pgfqpoint{1.131555in}{0.801788in}}{\pgfqpoint{1.120660in}{0.801788in}}%
\pgfpathcurveto{\pgfqpoint{1.109764in}{0.801788in}}{\pgfqpoint{1.099313in}{0.797459in}}{\pgfqpoint{1.091609in}{0.789755in}}%
\pgfpathcurveto{\pgfqpoint{1.083905in}{0.782051in}}{\pgfqpoint{1.079576in}{0.771600in}}{\pgfqpoint{1.079576in}{0.760704in}}%
\pgfpathcurveto{\pgfqpoint{1.079576in}{0.749809in}}{\pgfqpoint{1.083905in}{0.739358in}}{\pgfqpoint{1.091609in}{0.731653in}}%
\pgfpathcurveto{\pgfqpoint{1.099313in}{0.723949in}}{\pgfqpoint{1.109764in}{0.719620in}}{\pgfqpoint{1.120660in}{0.719620in}}%
\pgfpathlineto{\pgfqpoint{1.120660in}{0.719620in}}%
\pgfpathclose%
\pgfusepath{stroke}%
\end{pgfscope}%
\begin{pgfscope}%
\pgfpathrectangle{\pgfqpoint{0.688192in}{0.670138in}}{\pgfqpoint{7.111808in}{5.129862in}}%
\pgfusepath{clip}%
\pgfsetbuttcap%
\pgfsetroundjoin%
\pgfsetlinewidth{1.003750pt}%
\definecolor{currentstroke}{rgb}{0.000000,0.000000,0.000000}%
\pgfsetstrokecolor{currentstroke}%
\pgfsetdash{}{0pt}%
\pgfpathmoveto{\pgfqpoint{1.679564in}{0.702362in}}%
\pgfpathcurveto{\pgfqpoint{1.690460in}{0.702362in}}{\pgfqpoint{1.700911in}{0.706691in}}{\pgfqpoint{1.708615in}{0.714395in}}%
\pgfpathcurveto{\pgfqpoint{1.716319in}{0.722099in}}{\pgfqpoint{1.720648in}{0.732550in}}{\pgfqpoint{1.720648in}{0.743446in}}%
\pgfpathcurveto{\pgfqpoint{1.720648in}{0.754341in}}{\pgfqpoint{1.716319in}{0.764792in}}{\pgfqpoint{1.708615in}{0.772496in}}%
\pgfpathcurveto{\pgfqpoint{1.700911in}{0.780201in}}{\pgfqpoint{1.690460in}{0.784529in}}{\pgfqpoint{1.679564in}{0.784529in}}%
\pgfpathcurveto{\pgfqpoint{1.668669in}{0.784529in}}{\pgfqpoint{1.658218in}{0.780201in}}{\pgfqpoint{1.650514in}{0.772496in}}%
\pgfpathcurveto{\pgfqpoint{1.642809in}{0.764792in}}{\pgfqpoint{1.638480in}{0.754341in}}{\pgfqpoint{1.638480in}{0.743446in}}%
\pgfpathcurveto{\pgfqpoint{1.638480in}{0.732550in}}{\pgfqpoint{1.642809in}{0.722099in}}{\pgfqpoint{1.650514in}{0.714395in}}%
\pgfpathcurveto{\pgfqpoint{1.658218in}{0.706691in}}{\pgfqpoint{1.668669in}{0.702362in}}{\pgfqpoint{1.679564in}{0.702362in}}%
\pgfpathlineto{\pgfqpoint{1.679564in}{0.702362in}}%
\pgfpathclose%
\pgfusepath{stroke}%
\end{pgfscope}%
\begin{pgfscope}%
\pgfpathrectangle{\pgfqpoint{0.688192in}{0.670138in}}{\pgfqpoint{7.111808in}{5.129862in}}%
\pgfusepath{clip}%
\pgfsetbuttcap%
\pgfsetroundjoin%
\pgfsetlinewidth{1.003750pt}%
\definecolor{currentstroke}{rgb}{0.000000,0.000000,0.000000}%
\pgfsetstrokecolor{currentstroke}%
\pgfsetdash{}{0pt}%
\pgfpathmoveto{\pgfqpoint{5.261268in}{0.632886in}}%
\pgfpathcurveto{\pgfqpoint{5.272163in}{0.632886in}}{\pgfqpoint{5.282614in}{0.637215in}}{\pgfqpoint{5.290319in}{0.644920in}}%
\pgfpathcurveto{\pgfqpoint{5.298023in}{0.652624in}}{\pgfqpoint{5.302352in}{0.663075in}}{\pgfqpoint{5.302352in}{0.673970in}}%
\pgfpathcurveto{\pgfqpoint{5.302352in}{0.684866in}}{\pgfqpoint{5.298023in}{0.695317in}}{\pgfqpoint{5.290319in}{0.703021in}}%
\pgfpathcurveto{\pgfqpoint{5.282614in}{0.710725in}}{\pgfqpoint{5.272163in}{0.715054in}}{\pgfqpoint{5.261268in}{0.715054in}}%
\pgfpathcurveto{\pgfqpoint{5.250372in}{0.715054in}}{\pgfqpoint{5.239921in}{0.710725in}}{\pgfqpoint{5.232217in}{0.703021in}}%
\pgfpathcurveto{\pgfqpoint{5.224513in}{0.695317in}}{\pgfqpoint{5.220184in}{0.684866in}}{\pgfqpoint{5.220184in}{0.673970in}}%
\pgfpathcurveto{\pgfqpoint{5.220184in}{0.663075in}}{\pgfqpoint{5.224513in}{0.652624in}}{\pgfqpoint{5.232217in}{0.644920in}}%
\pgfpathcurveto{\pgfqpoint{5.239921in}{0.637215in}}{\pgfqpoint{5.250372in}{0.632886in}}{\pgfqpoint{5.261268in}{0.632886in}}%
\pgfusepath{stroke}%
\end{pgfscope}%
\begin{pgfscope}%
\pgfpathrectangle{\pgfqpoint{0.688192in}{0.670138in}}{\pgfqpoint{7.111808in}{5.129862in}}%
\pgfusepath{clip}%
\pgfsetbuttcap%
\pgfsetroundjoin%
\pgfsetlinewidth{1.003750pt}%
\definecolor{currentstroke}{rgb}{0.000000,0.000000,0.000000}%
\pgfsetstrokecolor{currentstroke}%
\pgfsetdash{}{0pt}%
\pgfpathmoveto{\pgfqpoint{0.881277in}{0.900428in}}%
\pgfpathcurveto{\pgfqpoint{0.892173in}{0.900428in}}{\pgfqpoint{0.902624in}{0.904756in}}{\pgfqpoint{0.910328in}{0.912461in}}%
\pgfpathcurveto{\pgfqpoint{0.918032in}{0.920165in}}{\pgfqpoint{0.922361in}{0.930616in}}{\pgfqpoint{0.922361in}{0.941511in}}%
\pgfpathcurveto{\pgfqpoint{0.922361in}{0.952407in}}{\pgfqpoint{0.918032in}{0.962858in}}{\pgfqpoint{0.910328in}{0.970562in}}%
\pgfpathcurveto{\pgfqpoint{0.902624in}{0.978266in}}{\pgfqpoint{0.892173in}{0.982595in}}{\pgfqpoint{0.881277in}{0.982595in}}%
\pgfpathcurveto{\pgfqpoint{0.870382in}{0.982595in}}{\pgfqpoint{0.859931in}{0.978266in}}{\pgfqpoint{0.852226in}{0.970562in}}%
\pgfpathcurveto{\pgfqpoint{0.844522in}{0.962858in}}{\pgfqpoint{0.840193in}{0.952407in}}{\pgfqpoint{0.840193in}{0.941511in}}%
\pgfpathcurveto{\pgfqpoint{0.840193in}{0.930616in}}{\pgfqpoint{0.844522in}{0.920165in}}{\pgfqpoint{0.852226in}{0.912461in}}%
\pgfpathcurveto{\pgfqpoint{0.859931in}{0.904756in}}{\pgfqpoint{0.870382in}{0.900428in}}{\pgfqpoint{0.881277in}{0.900428in}}%
\pgfpathlineto{\pgfqpoint{0.881277in}{0.900428in}}%
\pgfpathclose%
\pgfusepath{stroke}%
\end{pgfscope}%
\begin{pgfscope}%
\pgfpathrectangle{\pgfqpoint{0.688192in}{0.670138in}}{\pgfqpoint{7.111808in}{5.129862in}}%
\pgfusepath{clip}%
\pgfsetbuttcap%
\pgfsetroundjoin%
\pgfsetlinewidth{1.003750pt}%
\definecolor{currentstroke}{rgb}{0.000000,0.000000,0.000000}%
\pgfsetstrokecolor{currentstroke}%
\pgfsetdash{}{0pt}%
\pgfpathmoveto{\pgfqpoint{0.899610in}{0.859043in}}%
\pgfpathcurveto{\pgfqpoint{0.910506in}{0.859043in}}{\pgfqpoint{0.920957in}{0.863371in}}{\pgfqpoint{0.928661in}{0.871076in}}%
\pgfpathcurveto{\pgfqpoint{0.936365in}{0.878780in}}{\pgfqpoint{0.940694in}{0.889231in}}{\pgfqpoint{0.940694in}{0.900126in}}%
\pgfpathcurveto{\pgfqpoint{0.940694in}{0.911022in}}{\pgfqpoint{0.936365in}{0.921473in}}{\pgfqpoint{0.928661in}{0.929177in}}%
\pgfpathcurveto{\pgfqpoint{0.920957in}{0.936881in}}{\pgfqpoint{0.910506in}{0.941210in}}{\pgfqpoint{0.899610in}{0.941210in}}%
\pgfpathcurveto{\pgfqpoint{0.888715in}{0.941210in}}{\pgfqpoint{0.878264in}{0.936881in}}{\pgfqpoint{0.870560in}{0.929177in}}%
\pgfpathcurveto{\pgfqpoint{0.862855in}{0.921473in}}{\pgfqpoint{0.858526in}{0.911022in}}{\pgfqpoint{0.858526in}{0.900126in}}%
\pgfpathcurveto{\pgfqpoint{0.858526in}{0.889231in}}{\pgfqpoint{0.862855in}{0.878780in}}{\pgfqpoint{0.870560in}{0.871076in}}%
\pgfpathcurveto{\pgfqpoint{0.878264in}{0.863371in}}{\pgfqpoint{0.888715in}{0.859043in}}{\pgfqpoint{0.899610in}{0.859043in}}%
\pgfpathlineto{\pgfqpoint{0.899610in}{0.859043in}}%
\pgfpathclose%
\pgfusepath{stroke}%
\end{pgfscope}%
\begin{pgfscope}%
\pgfpathrectangle{\pgfqpoint{0.688192in}{0.670138in}}{\pgfqpoint{7.111808in}{5.129862in}}%
\pgfusepath{clip}%
\pgfsetbuttcap%
\pgfsetroundjoin%
\pgfsetlinewidth{1.003750pt}%
\definecolor{currentstroke}{rgb}{0.000000,0.000000,0.000000}%
\pgfsetstrokecolor{currentstroke}%
\pgfsetdash{}{0pt}%
\pgfpathmoveto{\pgfqpoint{4.832581in}{0.867024in}}%
\pgfpathcurveto{\pgfqpoint{4.843476in}{0.867024in}}{\pgfqpoint{4.853927in}{0.871353in}}{\pgfqpoint{4.861631in}{0.879057in}}%
\pgfpathcurveto{\pgfqpoint{4.869336in}{0.886762in}}{\pgfqpoint{4.873665in}{0.897212in}}{\pgfqpoint{4.873665in}{0.908108in}}%
\pgfpathcurveto{\pgfqpoint{4.873665in}{0.919003in}}{\pgfqpoint{4.869336in}{0.929454in}}{\pgfqpoint{4.861631in}{0.937159in}}%
\pgfpathcurveto{\pgfqpoint{4.853927in}{0.944863in}}{\pgfqpoint{4.843476in}{0.949192in}}{\pgfqpoint{4.832581in}{0.949192in}}%
\pgfpathcurveto{\pgfqpoint{4.821685in}{0.949192in}}{\pgfqpoint{4.811234in}{0.944863in}}{\pgfqpoint{4.803530in}{0.937159in}}%
\pgfpathcurveto{\pgfqpoint{4.795826in}{0.929454in}}{\pgfqpoint{4.791497in}{0.919003in}}{\pgfqpoint{4.791497in}{0.908108in}}%
\pgfpathcurveto{\pgfqpoint{4.791497in}{0.897212in}}{\pgfqpoint{4.795826in}{0.886762in}}{\pgfqpoint{4.803530in}{0.879057in}}%
\pgfpathcurveto{\pgfqpoint{4.811234in}{0.871353in}}{\pgfqpoint{4.821685in}{0.867024in}}{\pgfqpoint{4.832581in}{0.867024in}}%
\pgfpathlineto{\pgfqpoint{4.832581in}{0.867024in}}%
\pgfpathclose%
\pgfusepath{stroke}%
\end{pgfscope}%
\begin{pgfscope}%
\pgfpathrectangle{\pgfqpoint{0.688192in}{0.670138in}}{\pgfqpoint{7.111808in}{5.129862in}}%
\pgfusepath{clip}%
\pgfsetbuttcap%
\pgfsetroundjoin%
\pgfsetlinewidth{1.003750pt}%
\definecolor{currentstroke}{rgb}{0.000000,0.000000,0.000000}%
\pgfsetstrokecolor{currentstroke}%
\pgfsetdash{}{0pt}%
\pgfpathmoveto{\pgfqpoint{1.341138in}{0.712313in}}%
\pgfpathcurveto{\pgfqpoint{1.352034in}{0.712313in}}{\pgfqpoint{1.362484in}{0.716642in}}{\pgfqpoint{1.370189in}{0.724347in}}%
\pgfpathcurveto{\pgfqpoint{1.377893in}{0.732051in}}{\pgfqpoint{1.382222in}{0.742502in}}{\pgfqpoint{1.382222in}{0.753397in}}%
\pgfpathcurveto{\pgfqpoint{1.382222in}{0.764293in}}{\pgfqpoint{1.377893in}{0.774744in}}{\pgfqpoint{1.370189in}{0.782448in}}%
\pgfpathcurveto{\pgfqpoint{1.362484in}{0.790152in}}{\pgfqpoint{1.352034in}{0.794481in}}{\pgfqpoint{1.341138in}{0.794481in}}%
\pgfpathcurveto{\pgfqpoint{1.330243in}{0.794481in}}{\pgfqpoint{1.319792in}{0.790152in}}{\pgfqpoint{1.312087in}{0.782448in}}%
\pgfpathcurveto{\pgfqpoint{1.304383in}{0.774744in}}{\pgfqpoint{1.300054in}{0.764293in}}{\pgfqpoint{1.300054in}{0.753397in}}%
\pgfpathcurveto{\pgfqpoint{1.300054in}{0.742502in}}{\pgfqpoint{1.304383in}{0.732051in}}{\pgfqpoint{1.312087in}{0.724347in}}%
\pgfpathcurveto{\pgfqpoint{1.319792in}{0.716642in}}{\pgfqpoint{1.330243in}{0.712313in}}{\pgfqpoint{1.341138in}{0.712313in}}%
\pgfpathlineto{\pgfqpoint{1.341138in}{0.712313in}}%
\pgfpathclose%
\pgfusepath{stroke}%
\end{pgfscope}%
\begin{pgfscope}%
\pgfpathrectangle{\pgfqpoint{0.688192in}{0.670138in}}{\pgfqpoint{7.111808in}{5.129862in}}%
\pgfusepath{clip}%
\pgfsetbuttcap%
\pgfsetroundjoin%
\pgfsetlinewidth{1.003750pt}%
\definecolor{currentstroke}{rgb}{0.000000,0.000000,0.000000}%
\pgfsetstrokecolor{currentstroke}%
\pgfsetdash{}{0pt}%
\pgfpathmoveto{\pgfqpoint{4.634758in}{0.639654in}}%
\pgfpathcurveto{\pgfqpoint{4.645653in}{0.639654in}}{\pgfqpoint{4.656104in}{0.643983in}}{\pgfqpoint{4.663808in}{0.651687in}}%
\pgfpathcurveto{\pgfqpoint{4.671513in}{0.659391in}}{\pgfqpoint{4.675841in}{0.669842in}}{\pgfqpoint{4.675841in}{0.680738in}}%
\pgfpathcurveto{\pgfqpoint{4.675841in}{0.691633in}}{\pgfqpoint{4.671513in}{0.702084in}}{\pgfqpoint{4.663808in}{0.709789in}}%
\pgfpathcurveto{\pgfqpoint{4.656104in}{0.717493in}}{\pgfqpoint{4.645653in}{0.721822in}}{\pgfqpoint{4.634758in}{0.721822in}}%
\pgfpathcurveto{\pgfqpoint{4.623862in}{0.721822in}}{\pgfqpoint{4.613411in}{0.717493in}}{\pgfqpoint{4.605707in}{0.709789in}}%
\pgfpathcurveto{\pgfqpoint{4.598003in}{0.702084in}}{\pgfqpoint{4.593674in}{0.691633in}}{\pgfqpoint{4.593674in}{0.680738in}}%
\pgfpathcurveto{\pgfqpoint{4.593674in}{0.669842in}}{\pgfqpoint{4.598003in}{0.659391in}}{\pgfqpoint{4.605707in}{0.651687in}}%
\pgfpathcurveto{\pgfqpoint{4.613411in}{0.643983in}}{\pgfqpoint{4.623862in}{0.639654in}}{\pgfqpoint{4.634758in}{0.639654in}}%
\pgfusepath{stroke}%
\end{pgfscope}%
\begin{pgfscope}%
\pgfpathrectangle{\pgfqpoint{0.688192in}{0.670138in}}{\pgfqpoint{7.111808in}{5.129862in}}%
\pgfusepath{clip}%
\pgfsetbuttcap%
\pgfsetroundjoin%
\pgfsetlinewidth{1.003750pt}%
\definecolor{currentstroke}{rgb}{0.000000,0.000000,0.000000}%
\pgfsetstrokecolor{currentstroke}%
\pgfsetdash{}{0pt}%
\pgfpathmoveto{\pgfqpoint{3.248690in}{4.902676in}}%
\pgfpathcurveto{\pgfqpoint{3.259585in}{4.902676in}}{\pgfqpoint{3.270036in}{4.907005in}}{\pgfqpoint{3.277741in}{4.914709in}}%
\pgfpathcurveto{\pgfqpoint{3.285445in}{4.922413in}}{\pgfqpoint{3.289774in}{4.932864in}}{\pgfqpoint{3.289774in}{4.943760in}}%
\pgfpathcurveto{\pgfqpoint{3.289774in}{4.954655in}}{\pgfqpoint{3.285445in}{4.965106in}}{\pgfqpoint{3.277741in}{4.972810in}}%
\pgfpathcurveto{\pgfqpoint{3.270036in}{4.980515in}}{\pgfqpoint{3.259585in}{4.984844in}}{\pgfqpoint{3.248690in}{4.984844in}}%
\pgfpathcurveto{\pgfqpoint{3.237794in}{4.984844in}}{\pgfqpoint{3.227343in}{4.980515in}}{\pgfqpoint{3.219639in}{4.972810in}}%
\pgfpathcurveto{\pgfqpoint{3.211935in}{4.965106in}}{\pgfqpoint{3.207606in}{4.954655in}}{\pgfqpoint{3.207606in}{4.943760in}}%
\pgfpathcurveto{\pgfqpoint{3.207606in}{4.932864in}}{\pgfqpoint{3.211935in}{4.922413in}}{\pgfqpoint{3.219639in}{4.914709in}}%
\pgfpathcurveto{\pgfqpoint{3.227343in}{4.907005in}}{\pgfqpoint{3.237794in}{4.902676in}}{\pgfqpoint{3.248690in}{4.902676in}}%
\pgfpathlineto{\pgfqpoint{3.248690in}{4.902676in}}%
\pgfpathclose%
\pgfusepath{stroke}%
\end{pgfscope}%
\begin{pgfscope}%
\pgfpathrectangle{\pgfqpoint{0.688192in}{0.670138in}}{\pgfqpoint{7.111808in}{5.129862in}}%
\pgfusepath{clip}%
\pgfsetbuttcap%
\pgfsetroundjoin%
\pgfsetlinewidth{1.003750pt}%
\definecolor{currentstroke}{rgb}{0.000000,0.000000,0.000000}%
\pgfsetstrokecolor{currentstroke}%
\pgfsetdash{}{0pt}%
\pgfpathmoveto{\pgfqpoint{0.790083in}{1.032876in}}%
\pgfpathcurveto{\pgfqpoint{0.800978in}{1.032876in}}{\pgfqpoint{0.811429in}{1.037204in}}{\pgfqpoint{0.819133in}{1.044909in}}%
\pgfpathcurveto{\pgfqpoint{0.826838in}{1.052613in}}{\pgfqpoint{0.831167in}{1.063064in}}{\pgfqpoint{0.831167in}{1.073959in}}%
\pgfpathcurveto{\pgfqpoint{0.831167in}{1.084855in}}{\pgfqpoint{0.826838in}{1.095306in}}{\pgfqpoint{0.819133in}{1.103010in}}%
\pgfpathcurveto{\pgfqpoint{0.811429in}{1.110715in}}{\pgfqpoint{0.800978in}{1.115043in}}{\pgfqpoint{0.790083in}{1.115043in}}%
\pgfpathcurveto{\pgfqpoint{0.779187in}{1.115043in}}{\pgfqpoint{0.768736in}{1.110715in}}{\pgfqpoint{0.761032in}{1.103010in}}%
\pgfpathcurveto{\pgfqpoint{0.753328in}{1.095306in}}{\pgfqpoint{0.748999in}{1.084855in}}{\pgfqpoint{0.748999in}{1.073959in}}%
\pgfpathcurveto{\pgfqpoint{0.748999in}{1.063064in}}{\pgfqpoint{0.753328in}{1.052613in}}{\pgfqpoint{0.761032in}{1.044909in}}%
\pgfpathcurveto{\pgfqpoint{0.768736in}{1.037204in}}{\pgfqpoint{0.779187in}{1.032876in}}{\pgfqpoint{0.790083in}{1.032876in}}%
\pgfpathlineto{\pgfqpoint{0.790083in}{1.032876in}}%
\pgfpathclose%
\pgfusepath{stroke}%
\end{pgfscope}%
\begin{pgfscope}%
\pgfpathrectangle{\pgfqpoint{0.688192in}{0.670138in}}{\pgfqpoint{7.111808in}{5.129862in}}%
\pgfusepath{clip}%
\pgfsetbuttcap%
\pgfsetroundjoin%
\pgfsetlinewidth{1.003750pt}%
\definecolor{currentstroke}{rgb}{0.000000,0.000000,0.000000}%
\pgfsetstrokecolor{currentstroke}%
\pgfsetdash{}{0pt}%
\pgfpathmoveto{\pgfqpoint{1.441321in}{0.708382in}}%
\pgfpathcurveto{\pgfqpoint{1.452217in}{0.708382in}}{\pgfqpoint{1.462667in}{0.712711in}}{\pgfqpoint{1.470372in}{0.720416in}}%
\pgfpathcurveto{\pgfqpoint{1.478076in}{0.728120in}}{\pgfqpoint{1.482405in}{0.738571in}}{\pgfqpoint{1.482405in}{0.749466in}}%
\pgfpathcurveto{\pgfqpoint{1.482405in}{0.760362in}}{\pgfqpoint{1.478076in}{0.770813in}}{\pgfqpoint{1.470372in}{0.778517in}}%
\pgfpathcurveto{\pgfqpoint{1.462667in}{0.786221in}}{\pgfqpoint{1.452217in}{0.790550in}}{\pgfqpoint{1.441321in}{0.790550in}}%
\pgfpathcurveto{\pgfqpoint{1.430425in}{0.790550in}}{\pgfqpoint{1.419975in}{0.786221in}}{\pgfqpoint{1.412270in}{0.778517in}}%
\pgfpathcurveto{\pgfqpoint{1.404566in}{0.770813in}}{\pgfqpoint{1.400237in}{0.760362in}}{\pgfqpoint{1.400237in}{0.749466in}}%
\pgfpathcurveto{\pgfqpoint{1.400237in}{0.738571in}}{\pgfqpoint{1.404566in}{0.728120in}}{\pgfqpoint{1.412270in}{0.720416in}}%
\pgfpathcurveto{\pgfqpoint{1.419975in}{0.712711in}}{\pgfqpoint{1.430425in}{0.708382in}}{\pgfqpoint{1.441321in}{0.708382in}}%
\pgfpathlineto{\pgfqpoint{1.441321in}{0.708382in}}%
\pgfpathclose%
\pgfusepath{stroke}%
\end{pgfscope}%
\begin{pgfscope}%
\pgfpathrectangle{\pgfqpoint{0.688192in}{0.670138in}}{\pgfqpoint{7.111808in}{5.129862in}}%
\pgfusepath{clip}%
\pgfsetbuttcap%
\pgfsetroundjoin%
\pgfsetlinewidth{1.003750pt}%
\definecolor{currentstroke}{rgb}{0.000000,0.000000,0.000000}%
\pgfsetstrokecolor{currentstroke}%
\pgfsetdash{}{0pt}%
\pgfpathmoveto{\pgfqpoint{3.523301in}{0.651040in}}%
\pgfpathcurveto{\pgfqpoint{3.534196in}{0.651040in}}{\pgfqpoint{3.544647in}{0.655369in}}{\pgfqpoint{3.552351in}{0.663073in}}%
\pgfpathcurveto{\pgfqpoint{3.560056in}{0.670778in}}{\pgfqpoint{3.564385in}{0.681229in}}{\pgfqpoint{3.564385in}{0.692124in}}%
\pgfpathcurveto{\pgfqpoint{3.564385in}{0.703020in}}{\pgfqpoint{3.560056in}{0.713470in}}{\pgfqpoint{3.552351in}{0.721175in}}%
\pgfpathcurveto{\pgfqpoint{3.544647in}{0.728879in}}{\pgfqpoint{3.534196in}{0.733208in}}{\pgfqpoint{3.523301in}{0.733208in}}%
\pgfpathcurveto{\pgfqpoint{3.512405in}{0.733208in}}{\pgfqpoint{3.501954in}{0.728879in}}{\pgfqpoint{3.494250in}{0.721175in}}%
\pgfpathcurveto{\pgfqpoint{3.486546in}{0.713470in}}{\pgfqpoint{3.482217in}{0.703020in}}{\pgfqpoint{3.482217in}{0.692124in}}%
\pgfpathcurveto{\pgfqpoint{3.482217in}{0.681229in}}{\pgfqpoint{3.486546in}{0.670778in}}{\pgfqpoint{3.494250in}{0.663073in}}%
\pgfpathcurveto{\pgfqpoint{3.501954in}{0.655369in}}{\pgfqpoint{3.512405in}{0.651040in}}{\pgfqpoint{3.523301in}{0.651040in}}%
\pgfusepath{stroke}%
\end{pgfscope}%
\begin{pgfscope}%
\pgfpathrectangle{\pgfqpoint{0.688192in}{0.670138in}}{\pgfqpoint{7.111808in}{5.129862in}}%
\pgfusepath{clip}%
\pgfsetbuttcap%
\pgfsetroundjoin%
\pgfsetlinewidth{1.003750pt}%
\definecolor{currentstroke}{rgb}{0.000000,0.000000,0.000000}%
\pgfsetstrokecolor{currentstroke}%
\pgfsetdash{}{0pt}%
\pgfpathmoveto{\pgfqpoint{2.934581in}{0.664746in}}%
\pgfpathcurveto{\pgfqpoint{2.945476in}{0.664746in}}{\pgfqpoint{2.955927in}{0.669075in}}{\pgfqpoint{2.963631in}{0.676779in}}%
\pgfpathcurveto{\pgfqpoint{2.971336in}{0.684484in}}{\pgfqpoint{2.975665in}{0.694934in}}{\pgfqpoint{2.975665in}{0.705830in}}%
\pgfpathcurveto{\pgfqpoint{2.975665in}{0.716726in}}{\pgfqpoint{2.971336in}{0.727176in}}{\pgfqpoint{2.963631in}{0.734881in}}%
\pgfpathcurveto{\pgfqpoint{2.955927in}{0.742585in}}{\pgfqpoint{2.945476in}{0.746914in}}{\pgfqpoint{2.934581in}{0.746914in}}%
\pgfpathcurveto{\pgfqpoint{2.923685in}{0.746914in}}{\pgfqpoint{2.913234in}{0.742585in}}{\pgfqpoint{2.905530in}{0.734881in}}%
\pgfpathcurveto{\pgfqpoint{2.897826in}{0.727176in}}{\pgfqpoint{2.893497in}{0.716726in}}{\pgfqpoint{2.893497in}{0.705830in}}%
\pgfpathcurveto{\pgfqpoint{2.893497in}{0.694934in}}{\pgfqpoint{2.897826in}{0.684484in}}{\pgfqpoint{2.905530in}{0.676779in}}%
\pgfpathcurveto{\pgfqpoint{2.913234in}{0.669075in}}{\pgfqpoint{2.923685in}{0.664746in}}{\pgfqpoint{2.934581in}{0.664746in}}%
\pgfusepath{stroke}%
\end{pgfscope}%
\begin{pgfscope}%
\pgfpathrectangle{\pgfqpoint{0.688192in}{0.670138in}}{\pgfqpoint{7.111808in}{5.129862in}}%
\pgfusepath{clip}%
\pgfsetbuttcap%
\pgfsetroundjoin%
\pgfsetlinewidth{1.003750pt}%
\definecolor{currentstroke}{rgb}{0.000000,0.000000,0.000000}%
\pgfsetstrokecolor{currentstroke}%
\pgfsetdash{}{0pt}%
\pgfpathmoveto{\pgfqpoint{2.934581in}{0.664746in}}%
\pgfpathcurveto{\pgfqpoint{2.945476in}{0.664746in}}{\pgfqpoint{2.955927in}{0.669075in}}{\pgfqpoint{2.963631in}{0.676779in}}%
\pgfpathcurveto{\pgfqpoint{2.971336in}{0.684484in}}{\pgfqpoint{2.975665in}{0.694934in}}{\pgfqpoint{2.975665in}{0.705830in}}%
\pgfpathcurveto{\pgfqpoint{2.975665in}{0.716726in}}{\pgfqpoint{2.971336in}{0.727176in}}{\pgfqpoint{2.963631in}{0.734881in}}%
\pgfpathcurveto{\pgfqpoint{2.955927in}{0.742585in}}{\pgfqpoint{2.945476in}{0.746914in}}{\pgfqpoint{2.934581in}{0.746914in}}%
\pgfpathcurveto{\pgfqpoint{2.923685in}{0.746914in}}{\pgfqpoint{2.913234in}{0.742585in}}{\pgfqpoint{2.905530in}{0.734881in}}%
\pgfpathcurveto{\pgfqpoint{2.897826in}{0.727176in}}{\pgfqpoint{2.893497in}{0.716726in}}{\pgfqpoint{2.893497in}{0.705830in}}%
\pgfpathcurveto{\pgfqpoint{2.893497in}{0.694934in}}{\pgfqpoint{2.897826in}{0.684484in}}{\pgfqpoint{2.905530in}{0.676779in}}%
\pgfpathcurveto{\pgfqpoint{2.913234in}{0.669075in}}{\pgfqpoint{2.923685in}{0.664746in}}{\pgfqpoint{2.934581in}{0.664746in}}%
\pgfusepath{stroke}%
\end{pgfscope}%
\begin{pgfscope}%
\pgfpathrectangle{\pgfqpoint{0.688192in}{0.670138in}}{\pgfqpoint{7.111808in}{5.129862in}}%
\pgfusepath{clip}%
\pgfsetbuttcap%
\pgfsetroundjoin%
\pgfsetlinewidth{1.003750pt}%
\definecolor{currentstroke}{rgb}{0.000000,0.000000,0.000000}%
\pgfsetstrokecolor{currentstroke}%
\pgfsetdash{}{0pt}%
\pgfpathmoveto{\pgfqpoint{2.890311in}{0.666323in}}%
\pgfpathcurveto{\pgfqpoint{2.901207in}{0.666323in}}{\pgfqpoint{2.911658in}{0.670651in}}{\pgfqpoint{2.919362in}{0.678356in}}%
\pgfpathcurveto{\pgfqpoint{2.927066in}{0.686060in}}{\pgfqpoint{2.931395in}{0.696511in}}{\pgfqpoint{2.931395in}{0.707406in}}%
\pgfpathcurveto{\pgfqpoint{2.931395in}{0.718302in}}{\pgfqpoint{2.927066in}{0.728753in}}{\pgfqpoint{2.919362in}{0.736457in}}%
\pgfpathcurveto{\pgfqpoint{2.911658in}{0.744162in}}{\pgfqpoint{2.901207in}{0.748490in}}{\pgfqpoint{2.890311in}{0.748490in}}%
\pgfpathcurveto{\pgfqpoint{2.879416in}{0.748490in}}{\pgfqpoint{2.868965in}{0.744162in}}{\pgfqpoint{2.861260in}{0.736457in}}%
\pgfpathcurveto{\pgfqpoint{2.853556in}{0.728753in}}{\pgfqpoint{2.849227in}{0.718302in}}{\pgfqpoint{2.849227in}{0.707406in}}%
\pgfpathcurveto{\pgfqpoint{2.849227in}{0.696511in}}{\pgfqpoint{2.853556in}{0.686060in}}{\pgfqpoint{2.861260in}{0.678356in}}%
\pgfpathcurveto{\pgfqpoint{2.868965in}{0.670651in}}{\pgfqpoint{2.879416in}{0.666323in}}{\pgfqpoint{2.890311in}{0.666323in}}%
\pgfusepath{stroke}%
\end{pgfscope}%
\begin{pgfscope}%
\pgfpathrectangle{\pgfqpoint{0.688192in}{0.670138in}}{\pgfqpoint{7.111808in}{5.129862in}}%
\pgfusepath{clip}%
\pgfsetbuttcap%
\pgfsetroundjoin%
\pgfsetlinewidth{1.003750pt}%
\definecolor{currentstroke}{rgb}{0.000000,0.000000,0.000000}%
\pgfsetstrokecolor{currentstroke}%
\pgfsetdash{}{0pt}%
\pgfpathmoveto{\pgfqpoint{1.050667in}{0.739240in}}%
\pgfpathcurveto{\pgfqpoint{1.061563in}{0.739240in}}{\pgfqpoint{1.072014in}{0.743568in}}{\pgfqpoint{1.079718in}{0.751273in}}%
\pgfpathcurveto{\pgfqpoint{1.087422in}{0.758977in}}{\pgfqpoint{1.091751in}{0.769428in}}{\pgfqpoint{1.091751in}{0.780323in}}%
\pgfpathcurveto{\pgfqpoint{1.091751in}{0.791219in}}{\pgfqpoint{1.087422in}{0.801670in}}{\pgfqpoint{1.079718in}{0.809374in}}%
\pgfpathcurveto{\pgfqpoint{1.072014in}{0.817078in}}{\pgfqpoint{1.061563in}{0.821407in}}{\pgfqpoint{1.050667in}{0.821407in}}%
\pgfpathcurveto{\pgfqpoint{1.039772in}{0.821407in}}{\pgfqpoint{1.029321in}{0.817078in}}{\pgfqpoint{1.021617in}{0.809374in}}%
\pgfpathcurveto{\pgfqpoint{1.013912in}{0.801670in}}{\pgfqpoint{1.009583in}{0.791219in}}{\pgfqpoint{1.009583in}{0.780323in}}%
\pgfpathcurveto{\pgfqpoint{1.009583in}{0.769428in}}{\pgfqpoint{1.013912in}{0.758977in}}{\pgfqpoint{1.021617in}{0.751273in}}%
\pgfpathcurveto{\pgfqpoint{1.029321in}{0.743568in}}{\pgfqpoint{1.039772in}{0.739240in}}{\pgfqpoint{1.050667in}{0.739240in}}%
\pgfpathlineto{\pgfqpoint{1.050667in}{0.739240in}}%
\pgfpathclose%
\pgfusepath{stroke}%
\end{pgfscope}%
\begin{pgfscope}%
\pgfpathrectangle{\pgfqpoint{0.688192in}{0.670138in}}{\pgfqpoint{7.111808in}{5.129862in}}%
\pgfusepath{clip}%
\pgfsetbuttcap%
\pgfsetroundjoin%
\pgfsetlinewidth{1.003750pt}%
\definecolor{currentstroke}{rgb}{0.000000,0.000000,0.000000}%
\pgfsetstrokecolor{currentstroke}%
\pgfsetdash{}{0pt}%
\pgfpathmoveto{\pgfqpoint{0.880316in}{0.913368in}}%
\pgfpathcurveto{\pgfqpoint{0.891212in}{0.913368in}}{\pgfqpoint{0.901663in}{0.917697in}}{\pgfqpoint{0.909367in}{0.925401in}}%
\pgfpathcurveto{\pgfqpoint{0.917071in}{0.933106in}}{\pgfqpoint{0.921400in}{0.943557in}}{\pgfqpoint{0.921400in}{0.954452in}}%
\pgfpathcurveto{\pgfqpoint{0.921400in}{0.965348in}}{\pgfqpoint{0.917071in}{0.975799in}}{\pgfqpoint{0.909367in}{0.983503in}}%
\pgfpathcurveto{\pgfqpoint{0.901663in}{0.991207in}}{\pgfqpoint{0.891212in}{0.995536in}}{\pgfqpoint{0.880316in}{0.995536in}}%
\pgfpathcurveto{\pgfqpoint{0.869421in}{0.995536in}}{\pgfqpoint{0.858970in}{0.991207in}}{\pgfqpoint{0.851266in}{0.983503in}}%
\pgfpathcurveto{\pgfqpoint{0.843561in}{0.975799in}}{\pgfqpoint{0.839232in}{0.965348in}}{\pgfqpoint{0.839232in}{0.954452in}}%
\pgfpathcurveto{\pgfqpoint{0.839232in}{0.943557in}}{\pgfqpoint{0.843561in}{0.933106in}}{\pgfqpoint{0.851266in}{0.925401in}}%
\pgfpathcurveto{\pgfqpoint{0.858970in}{0.917697in}}{\pgfqpoint{0.869421in}{0.913368in}}{\pgfqpoint{0.880316in}{0.913368in}}%
\pgfpathlineto{\pgfqpoint{0.880316in}{0.913368in}}%
\pgfpathclose%
\pgfusepath{stroke}%
\end{pgfscope}%
\begin{pgfscope}%
\pgfpathrectangle{\pgfqpoint{0.688192in}{0.670138in}}{\pgfqpoint{7.111808in}{5.129862in}}%
\pgfusepath{clip}%
\pgfsetbuttcap%
\pgfsetroundjoin%
\pgfsetlinewidth{1.003750pt}%
\definecolor{currentstroke}{rgb}{0.000000,0.000000,0.000000}%
\pgfsetstrokecolor{currentstroke}%
\pgfsetdash{}{0pt}%
\pgfpathmoveto{\pgfqpoint{0.899610in}{0.859043in}}%
\pgfpathcurveto{\pgfqpoint{0.910506in}{0.859043in}}{\pgfqpoint{0.920957in}{0.863371in}}{\pgfqpoint{0.928661in}{0.871076in}}%
\pgfpathcurveto{\pgfqpoint{0.936365in}{0.878780in}}{\pgfqpoint{0.940694in}{0.889231in}}{\pgfqpoint{0.940694in}{0.900126in}}%
\pgfpathcurveto{\pgfqpoint{0.940694in}{0.911022in}}{\pgfqpoint{0.936365in}{0.921473in}}{\pgfqpoint{0.928661in}{0.929177in}}%
\pgfpathcurveto{\pgfqpoint{0.920957in}{0.936881in}}{\pgfqpoint{0.910506in}{0.941210in}}{\pgfqpoint{0.899610in}{0.941210in}}%
\pgfpathcurveto{\pgfqpoint{0.888715in}{0.941210in}}{\pgfqpoint{0.878264in}{0.936881in}}{\pgfqpoint{0.870560in}{0.929177in}}%
\pgfpathcurveto{\pgfqpoint{0.862855in}{0.921473in}}{\pgfqpoint{0.858526in}{0.911022in}}{\pgfqpoint{0.858526in}{0.900126in}}%
\pgfpathcurveto{\pgfqpoint{0.858526in}{0.889231in}}{\pgfqpoint{0.862855in}{0.878780in}}{\pgfqpoint{0.870560in}{0.871076in}}%
\pgfpathcurveto{\pgfqpoint{0.878264in}{0.863371in}}{\pgfqpoint{0.888715in}{0.859043in}}{\pgfqpoint{0.899610in}{0.859043in}}%
\pgfpathlineto{\pgfqpoint{0.899610in}{0.859043in}}%
\pgfpathclose%
\pgfusepath{stroke}%
\end{pgfscope}%
\begin{pgfscope}%
\pgfpathrectangle{\pgfqpoint{0.688192in}{0.670138in}}{\pgfqpoint{7.111808in}{5.129862in}}%
\pgfusepath{clip}%
\pgfsetbuttcap%
\pgfsetroundjoin%
\pgfsetlinewidth{1.003750pt}%
\definecolor{currentstroke}{rgb}{0.000000,0.000000,0.000000}%
\pgfsetstrokecolor{currentstroke}%
\pgfsetdash{}{0pt}%
\pgfpathmoveto{\pgfqpoint{2.869035in}{0.669385in}}%
\pgfpathcurveto{\pgfqpoint{2.879931in}{0.669385in}}{\pgfqpoint{2.890382in}{0.673714in}}{\pgfqpoint{2.898086in}{0.681419in}}%
\pgfpathcurveto{\pgfqpoint{2.905790in}{0.689123in}}{\pgfqpoint{2.910119in}{0.699574in}}{\pgfqpoint{2.910119in}{0.710469in}}%
\pgfpathcurveto{\pgfqpoint{2.910119in}{0.721365in}}{\pgfqpoint{2.905790in}{0.731816in}}{\pgfqpoint{2.898086in}{0.739520in}}%
\pgfpathcurveto{\pgfqpoint{2.890382in}{0.747224in}}{\pgfqpoint{2.879931in}{0.751553in}}{\pgfqpoint{2.869035in}{0.751553in}}%
\pgfpathcurveto{\pgfqpoint{2.858140in}{0.751553in}}{\pgfqpoint{2.847689in}{0.747224in}}{\pgfqpoint{2.839985in}{0.739520in}}%
\pgfpathcurveto{\pgfqpoint{2.832280in}{0.731816in}}{\pgfqpoint{2.827952in}{0.721365in}}{\pgfqpoint{2.827952in}{0.710469in}}%
\pgfpathcurveto{\pgfqpoint{2.827952in}{0.699574in}}{\pgfqpoint{2.832280in}{0.689123in}}{\pgfqpoint{2.839985in}{0.681419in}}%
\pgfpathcurveto{\pgfqpoint{2.847689in}{0.673714in}}{\pgfqpoint{2.858140in}{0.669385in}}{\pgfqpoint{2.869035in}{0.669385in}}%
\pgfpathlineto{\pgfqpoint{2.869035in}{0.669385in}}%
\pgfpathclose%
\pgfusepath{stroke}%
\end{pgfscope}%
\begin{pgfscope}%
\pgfpathrectangle{\pgfqpoint{0.688192in}{0.670138in}}{\pgfqpoint{7.111808in}{5.129862in}}%
\pgfusepath{clip}%
\pgfsetbuttcap%
\pgfsetroundjoin%
\pgfsetlinewidth{1.003750pt}%
\definecolor{currentstroke}{rgb}{0.000000,0.000000,0.000000}%
\pgfsetstrokecolor{currentstroke}%
\pgfsetdash{}{0pt}%
\pgfpathmoveto{\pgfqpoint{4.946549in}{3.101874in}}%
\pgfpathcurveto{\pgfqpoint{4.957445in}{3.101874in}}{\pgfqpoint{4.967896in}{3.106203in}}{\pgfqpoint{4.975600in}{3.113907in}}%
\pgfpathcurveto{\pgfqpoint{4.983304in}{3.121612in}}{\pgfqpoint{4.987633in}{3.132062in}}{\pgfqpoint{4.987633in}{3.142958in}}%
\pgfpathcurveto{\pgfqpoint{4.987633in}{3.153854in}}{\pgfqpoint{4.983304in}{3.164304in}}{\pgfqpoint{4.975600in}{3.172009in}}%
\pgfpathcurveto{\pgfqpoint{4.967896in}{3.179713in}}{\pgfqpoint{4.957445in}{3.184042in}}{\pgfqpoint{4.946549in}{3.184042in}}%
\pgfpathcurveto{\pgfqpoint{4.935654in}{3.184042in}}{\pgfqpoint{4.925203in}{3.179713in}}{\pgfqpoint{4.917499in}{3.172009in}}%
\pgfpathcurveto{\pgfqpoint{4.909794in}{3.164304in}}{\pgfqpoint{4.905465in}{3.153854in}}{\pgfqpoint{4.905465in}{3.142958in}}%
\pgfpathcurveto{\pgfqpoint{4.905465in}{3.132062in}}{\pgfqpoint{4.909794in}{3.121612in}}{\pgfqpoint{4.917499in}{3.113907in}}%
\pgfpathcurveto{\pgfqpoint{4.925203in}{3.106203in}}{\pgfqpoint{4.935654in}{3.101874in}}{\pgfqpoint{4.946549in}{3.101874in}}%
\pgfpathlineto{\pgfqpoint{4.946549in}{3.101874in}}%
\pgfpathclose%
\pgfusepath{stroke}%
\end{pgfscope}%
\begin{pgfscope}%
\pgfpathrectangle{\pgfqpoint{0.688192in}{0.670138in}}{\pgfqpoint{7.111808in}{5.129862in}}%
\pgfusepath{clip}%
\pgfsetbuttcap%
\pgfsetroundjoin%
\pgfsetlinewidth{1.003750pt}%
\definecolor{currentstroke}{rgb}{0.000000,0.000000,0.000000}%
\pgfsetstrokecolor{currentstroke}%
\pgfsetdash{}{0pt}%
\pgfpathmoveto{\pgfqpoint{0.901363in}{0.856548in}}%
\pgfpathcurveto{\pgfqpoint{0.912259in}{0.856548in}}{\pgfqpoint{0.922710in}{0.860877in}}{\pgfqpoint{0.930414in}{0.868582in}}%
\pgfpathcurveto{\pgfqpoint{0.938118in}{0.876286in}}{\pgfqpoint{0.942447in}{0.886737in}}{\pgfqpoint{0.942447in}{0.897632in}}%
\pgfpathcurveto{\pgfqpoint{0.942447in}{0.908528in}}{\pgfqpoint{0.938118in}{0.918979in}}{\pgfqpoint{0.930414in}{0.926683in}}%
\pgfpathcurveto{\pgfqpoint{0.922710in}{0.934387in}}{\pgfqpoint{0.912259in}{0.938716in}}{\pgfqpoint{0.901363in}{0.938716in}}%
\pgfpathcurveto{\pgfqpoint{0.890468in}{0.938716in}}{\pgfqpoint{0.880017in}{0.934387in}}{\pgfqpoint{0.872313in}{0.926683in}}%
\pgfpathcurveto{\pgfqpoint{0.864608in}{0.918979in}}{\pgfqpoint{0.860280in}{0.908528in}}{\pgfqpoint{0.860280in}{0.897632in}}%
\pgfpathcurveto{\pgfqpoint{0.860280in}{0.886737in}}{\pgfqpoint{0.864608in}{0.876286in}}{\pgfqpoint{0.872313in}{0.868582in}}%
\pgfpathcurveto{\pgfqpoint{0.880017in}{0.860877in}}{\pgfqpoint{0.890468in}{0.856548in}}{\pgfqpoint{0.901363in}{0.856548in}}%
\pgfpathlineto{\pgfqpoint{0.901363in}{0.856548in}}%
\pgfpathclose%
\pgfusepath{stroke}%
\end{pgfscope}%
\begin{pgfscope}%
\pgfpathrectangle{\pgfqpoint{0.688192in}{0.670138in}}{\pgfqpoint{7.111808in}{5.129862in}}%
\pgfusepath{clip}%
\pgfsetbuttcap%
\pgfsetroundjoin%
\pgfsetlinewidth{1.003750pt}%
\definecolor{currentstroke}{rgb}{0.000000,0.000000,0.000000}%
\pgfsetstrokecolor{currentstroke}%
\pgfsetdash{}{0pt}%
\pgfpathmoveto{\pgfqpoint{2.821283in}{0.839158in}}%
\pgfpathcurveto{\pgfqpoint{2.832179in}{0.839158in}}{\pgfqpoint{2.842630in}{0.843486in}}{\pgfqpoint{2.850334in}{0.851191in}}%
\pgfpathcurveto{\pgfqpoint{2.858038in}{0.858895in}}{\pgfqpoint{2.862367in}{0.869346in}}{\pgfqpoint{2.862367in}{0.880241in}}%
\pgfpathcurveto{\pgfqpoint{2.862367in}{0.891137in}}{\pgfqpoint{2.858038in}{0.901588in}}{\pgfqpoint{2.850334in}{0.909292in}}%
\pgfpathcurveto{\pgfqpoint{2.842630in}{0.916996in}}{\pgfqpoint{2.832179in}{0.921325in}}{\pgfqpoint{2.821283in}{0.921325in}}%
\pgfpathcurveto{\pgfqpoint{2.810388in}{0.921325in}}{\pgfqpoint{2.799937in}{0.916996in}}{\pgfqpoint{2.792232in}{0.909292in}}%
\pgfpathcurveto{\pgfqpoint{2.784528in}{0.901588in}}{\pgfqpoint{2.780199in}{0.891137in}}{\pgfqpoint{2.780199in}{0.880241in}}%
\pgfpathcurveto{\pgfqpoint{2.780199in}{0.869346in}}{\pgfqpoint{2.784528in}{0.858895in}}{\pgfqpoint{2.792232in}{0.851191in}}%
\pgfpathcurveto{\pgfqpoint{2.799937in}{0.843486in}}{\pgfqpoint{2.810388in}{0.839158in}}{\pgfqpoint{2.821283in}{0.839158in}}%
\pgfpathlineto{\pgfqpoint{2.821283in}{0.839158in}}%
\pgfpathclose%
\pgfusepath{stroke}%
\end{pgfscope}%
\begin{pgfscope}%
\pgfpathrectangle{\pgfqpoint{0.688192in}{0.670138in}}{\pgfqpoint{7.111808in}{5.129862in}}%
\pgfusepath{clip}%
\pgfsetbuttcap%
\pgfsetroundjoin%
\pgfsetlinewidth{1.003750pt}%
\definecolor{currentstroke}{rgb}{0.000000,0.000000,0.000000}%
\pgfsetstrokecolor{currentstroke}%
\pgfsetdash{}{0pt}%
\pgfpathmoveto{\pgfqpoint{0.782559in}{1.071983in}}%
\pgfpathcurveto{\pgfqpoint{0.793455in}{1.071983in}}{\pgfqpoint{0.803905in}{1.076312in}}{\pgfqpoint{0.811610in}{1.084017in}}%
\pgfpathcurveto{\pgfqpoint{0.819314in}{1.091721in}}{\pgfqpoint{0.823643in}{1.102172in}}{\pgfqpoint{0.823643in}{1.113067in}}%
\pgfpathcurveto{\pgfqpoint{0.823643in}{1.123963in}}{\pgfqpoint{0.819314in}{1.134414in}}{\pgfqpoint{0.811610in}{1.142118in}}%
\pgfpathcurveto{\pgfqpoint{0.803905in}{1.149822in}}{\pgfqpoint{0.793455in}{1.154151in}}{\pgfqpoint{0.782559in}{1.154151in}}%
\pgfpathcurveto{\pgfqpoint{0.771663in}{1.154151in}}{\pgfqpoint{0.761213in}{1.149822in}}{\pgfqpoint{0.753508in}{1.142118in}}%
\pgfpathcurveto{\pgfqpoint{0.745804in}{1.134414in}}{\pgfqpoint{0.741475in}{1.123963in}}{\pgfqpoint{0.741475in}{1.113067in}}%
\pgfpathcurveto{\pgfqpoint{0.741475in}{1.102172in}}{\pgfqpoint{0.745804in}{1.091721in}}{\pgfqpoint{0.753508in}{1.084017in}}%
\pgfpathcurveto{\pgfqpoint{0.761213in}{1.076312in}}{\pgfqpoint{0.771663in}{1.071983in}}{\pgfqpoint{0.782559in}{1.071983in}}%
\pgfpathlineto{\pgfqpoint{0.782559in}{1.071983in}}%
\pgfpathclose%
\pgfusepath{stroke}%
\end{pgfscope}%
\begin{pgfscope}%
\pgfpathrectangle{\pgfqpoint{0.688192in}{0.670138in}}{\pgfqpoint{7.111808in}{5.129862in}}%
\pgfusepath{clip}%
\pgfsetbuttcap%
\pgfsetroundjoin%
\pgfsetlinewidth{1.003750pt}%
\definecolor{currentstroke}{rgb}{0.000000,0.000000,0.000000}%
\pgfsetstrokecolor{currentstroke}%
\pgfsetdash{}{0pt}%
\pgfpathmoveto{\pgfqpoint{2.360638in}{2.745826in}}%
\pgfpathcurveto{\pgfqpoint{2.371534in}{2.745826in}}{\pgfqpoint{2.381985in}{2.750155in}}{\pgfqpoint{2.389689in}{2.757859in}}%
\pgfpathcurveto{\pgfqpoint{2.397393in}{2.765564in}}{\pgfqpoint{2.401722in}{2.776014in}}{\pgfqpoint{2.401722in}{2.786910in}}%
\pgfpathcurveto{\pgfqpoint{2.401722in}{2.797806in}}{\pgfqpoint{2.397393in}{2.808256in}}{\pgfqpoint{2.389689in}{2.815961in}}%
\pgfpathcurveto{\pgfqpoint{2.381985in}{2.823665in}}{\pgfqpoint{2.371534in}{2.827994in}}{\pgfqpoint{2.360638in}{2.827994in}}%
\pgfpathcurveto{\pgfqpoint{2.349743in}{2.827994in}}{\pgfqpoint{2.339292in}{2.823665in}}{\pgfqpoint{2.331588in}{2.815961in}}%
\pgfpathcurveto{\pgfqpoint{2.323883in}{2.808256in}}{\pgfqpoint{2.319554in}{2.797806in}}{\pgfqpoint{2.319554in}{2.786910in}}%
\pgfpathcurveto{\pgfqpoint{2.319554in}{2.776014in}}{\pgfqpoint{2.323883in}{2.765564in}}{\pgfqpoint{2.331588in}{2.757859in}}%
\pgfpathcurveto{\pgfqpoint{2.339292in}{2.750155in}}{\pgfqpoint{2.349743in}{2.745826in}}{\pgfqpoint{2.360638in}{2.745826in}}%
\pgfpathlineto{\pgfqpoint{2.360638in}{2.745826in}}%
\pgfpathclose%
\pgfusepath{stroke}%
\end{pgfscope}%
\begin{pgfscope}%
\pgfpathrectangle{\pgfqpoint{0.688192in}{0.670138in}}{\pgfqpoint{7.111808in}{5.129862in}}%
\pgfusepath{clip}%
\pgfsetbuttcap%
\pgfsetroundjoin%
\pgfsetlinewidth{1.003750pt}%
\definecolor{currentstroke}{rgb}{0.000000,0.000000,0.000000}%
\pgfsetstrokecolor{currentstroke}%
\pgfsetdash{}{0pt}%
\pgfpathmoveto{\pgfqpoint{3.139880in}{0.960028in}}%
\pgfpathcurveto{\pgfqpoint{3.150775in}{0.960028in}}{\pgfqpoint{3.161226in}{0.964357in}}{\pgfqpoint{3.168930in}{0.972061in}}%
\pgfpathcurveto{\pgfqpoint{3.176635in}{0.979766in}}{\pgfqpoint{3.180964in}{0.990217in}}{\pgfqpoint{3.180964in}{1.001112in}}%
\pgfpathcurveto{\pgfqpoint{3.180964in}{1.012008in}}{\pgfqpoint{3.176635in}{1.022458in}}{\pgfqpoint{3.168930in}{1.030163in}}%
\pgfpathcurveto{\pgfqpoint{3.161226in}{1.037867in}}{\pgfqpoint{3.150775in}{1.042196in}}{\pgfqpoint{3.139880in}{1.042196in}}%
\pgfpathcurveto{\pgfqpoint{3.128984in}{1.042196in}}{\pgfqpoint{3.118533in}{1.037867in}}{\pgfqpoint{3.110829in}{1.030163in}}%
\pgfpathcurveto{\pgfqpoint{3.103125in}{1.022458in}}{\pgfqpoint{3.098796in}{1.012008in}}{\pgfqpoint{3.098796in}{1.001112in}}%
\pgfpathcurveto{\pgfqpoint{3.098796in}{0.990217in}}{\pgfqpoint{3.103125in}{0.979766in}}{\pgfqpoint{3.110829in}{0.972061in}}%
\pgfpathcurveto{\pgfqpoint{3.118533in}{0.964357in}}{\pgfqpoint{3.128984in}{0.960028in}}{\pgfqpoint{3.139880in}{0.960028in}}%
\pgfpathlineto{\pgfqpoint{3.139880in}{0.960028in}}%
\pgfpathclose%
\pgfusepath{stroke}%
\end{pgfscope}%
\begin{pgfscope}%
\pgfpathrectangle{\pgfqpoint{0.688192in}{0.670138in}}{\pgfqpoint{7.111808in}{5.129862in}}%
\pgfusepath{clip}%
\pgfsetbuttcap%
\pgfsetroundjoin%
\pgfsetlinewidth{1.003750pt}%
\definecolor{currentstroke}{rgb}{0.000000,0.000000,0.000000}%
\pgfsetstrokecolor{currentstroke}%
\pgfsetdash{}{0pt}%
\pgfpathmoveto{\pgfqpoint{1.840914in}{0.696069in}}%
\pgfpathcurveto{\pgfqpoint{1.851809in}{0.696069in}}{\pgfqpoint{1.862260in}{0.700398in}}{\pgfqpoint{1.869964in}{0.708103in}}%
\pgfpathcurveto{\pgfqpoint{1.877669in}{0.715807in}}{\pgfqpoint{1.881997in}{0.726258in}}{\pgfqpoint{1.881997in}{0.737153in}}%
\pgfpathcurveto{\pgfqpoint{1.881997in}{0.748049in}}{\pgfqpoint{1.877669in}{0.758500in}}{\pgfqpoint{1.869964in}{0.766204in}}%
\pgfpathcurveto{\pgfqpoint{1.862260in}{0.773908in}}{\pgfqpoint{1.851809in}{0.778237in}}{\pgfqpoint{1.840914in}{0.778237in}}%
\pgfpathcurveto{\pgfqpoint{1.830018in}{0.778237in}}{\pgfqpoint{1.819567in}{0.773908in}}{\pgfqpoint{1.811863in}{0.766204in}}%
\pgfpathcurveto{\pgfqpoint{1.804159in}{0.758500in}}{\pgfqpoint{1.799830in}{0.748049in}}{\pgfqpoint{1.799830in}{0.737153in}}%
\pgfpathcurveto{\pgfqpoint{1.799830in}{0.726258in}}{\pgfqpoint{1.804159in}{0.715807in}}{\pgfqpoint{1.811863in}{0.708103in}}%
\pgfpathcurveto{\pgfqpoint{1.819567in}{0.700398in}}{\pgfqpoint{1.830018in}{0.696069in}}{\pgfqpoint{1.840914in}{0.696069in}}%
\pgfpathlineto{\pgfqpoint{1.840914in}{0.696069in}}%
\pgfpathclose%
\pgfusepath{stroke}%
\end{pgfscope}%
\begin{pgfscope}%
\pgfpathrectangle{\pgfqpoint{0.688192in}{0.670138in}}{\pgfqpoint{7.111808in}{5.129862in}}%
\pgfusepath{clip}%
\pgfsetbuttcap%
\pgfsetroundjoin%
\pgfsetlinewidth{1.003750pt}%
\definecolor{currentstroke}{rgb}{0.000000,0.000000,0.000000}%
\pgfsetstrokecolor{currentstroke}%
\pgfsetdash{}{0pt}%
\pgfpathmoveto{\pgfqpoint{0.882669in}{0.877855in}}%
\pgfpathcurveto{\pgfqpoint{0.893565in}{0.877855in}}{\pgfqpoint{0.904016in}{0.882184in}}{\pgfqpoint{0.911720in}{0.889888in}}%
\pgfpathcurveto{\pgfqpoint{0.919424in}{0.897592in}}{\pgfqpoint{0.923753in}{0.908043in}}{\pgfqpoint{0.923753in}{0.918939in}}%
\pgfpathcurveto{\pgfqpoint{0.923753in}{0.929834in}}{\pgfqpoint{0.919424in}{0.940285in}}{\pgfqpoint{0.911720in}{0.947989in}}%
\pgfpathcurveto{\pgfqpoint{0.904016in}{0.955694in}}{\pgfqpoint{0.893565in}{0.960023in}}{\pgfqpoint{0.882669in}{0.960023in}}%
\pgfpathcurveto{\pgfqpoint{0.871774in}{0.960023in}}{\pgfqpoint{0.861323in}{0.955694in}}{\pgfqpoint{0.853619in}{0.947989in}}%
\pgfpathcurveto{\pgfqpoint{0.845914in}{0.940285in}}{\pgfqpoint{0.841585in}{0.929834in}}{\pgfqpoint{0.841585in}{0.918939in}}%
\pgfpathcurveto{\pgfqpoint{0.841585in}{0.908043in}}{\pgfqpoint{0.845914in}{0.897592in}}{\pgfqpoint{0.853619in}{0.889888in}}%
\pgfpathcurveto{\pgfqpoint{0.861323in}{0.882184in}}{\pgfqpoint{0.871774in}{0.877855in}}{\pgfqpoint{0.882669in}{0.877855in}}%
\pgfpathlineto{\pgfqpoint{0.882669in}{0.877855in}}%
\pgfpathclose%
\pgfusepath{stroke}%
\end{pgfscope}%
\begin{pgfscope}%
\pgfpathrectangle{\pgfqpoint{0.688192in}{0.670138in}}{\pgfqpoint{7.111808in}{5.129862in}}%
\pgfusepath{clip}%
\pgfsetbuttcap%
\pgfsetroundjoin%
\pgfsetlinewidth{1.003750pt}%
\definecolor{currentstroke}{rgb}{0.000000,0.000000,0.000000}%
\pgfsetstrokecolor{currentstroke}%
\pgfsetdash{}{0pt}%
\pgfpathmoveto{\pgfqpoint{0.885952in}{0.873580in}}%
\pgfpathcurveto{\pgfqpoint{0.896848in}{0.873580in}}{\pgfqpoint{0.907298in}{0.877909in}}{\pgfqpoint{0.915003in}{0.885613in}}%
\pgfpathcurveto{\pgfqpoint{0.922707in}{0.893317in}}{\pgfqpoint{0.927036in}{0.903768in}}{\pgfqpoint{0.927036in}{0.914664in}}%
\pgfpathcurveto{\pgfqpoint{0.927036in}{0.925559in}}{\pgfqpoint{0.922707in}{0.936010in}}{\pgfqpoint{0.915003in}{0.943714in}}%
\pgfpathcurveto{\pgfqpoint{0.907298in}{0.951419in}}{\pgfqpoint{0.896848in}{0.955748in}}{\pgfqpoint{0.885952in}{0.955748in}}%
\pgfpathcurveto{\pgfqpoint{0.875056in}{0.955748in}}{\pgfqpoint{0.864606in}{0.951419in}}{\pgfqpoint{0.856901in}{0.943714in}}%
\pgfpathcurveto{\pgfqpoint{0.849197in}{0.936010in}}{\pgfqpoint{0.844868in}{0.925559in}}{\pgfqpoint{0.844868in}{0.914664in}}%
\pgfpathcurveto{\pgfqpoint{0.844868in}{0.903768in}}{\pgfqpoint{0.849197in}{0.893317in}}{\pgfqpoint{0.856901in}{0.885613in}}%
\pgfpathcurveto{\pgfqpoint{0.864606in}{0.877909in}}{\pgfqpoint{0.875056in}{0.873580in}}{\pgfqpoint{0.885952in}{0.873580in}}%
\pgfpathlineto{\pgfqpoint{0.885952in}{0.873580in}}%
\pgfpathclose%
\pgfusepath{stroke}%
\end{pgfscope}%
\begin{pgfscope}%
\pgfpathrectangle{\pgfqpoint{0.688192in}{0.670138in}}{\pgfqpoint{7.111808in}{5.129862in}}%
\pgfusepath{clip}%
\pgfsetbuttcap%
\pgfsetroundjoin%
\pgfsetlinewidth{1.003750pt}%
\definecolor{currentstroke}{rgb}{0.000000,0.000000,0.000000}%
\pgfsetstrokecolor{currentstroke}%
\pgfsetdash{}{0pt}%
\pgfpathmoveto{\pgfqpoint{3.583057in}{0.870778in}}%
\pgfpathcurveto{\pgfqpoint{3.593952in}{0.870778in}}{\pgfqpoint{3.604403in}{0.875107in}}{\pgfqpoint{3.612108in}{0.882811in}}%
\pgfpathcurveto{\pgfqpoint{3.619812in}{0.890516in}}{\pgfqpoint{3.624141in}{0.900966in}}{\pgfqpoint{3.624141in}{0.911862in}}%
\pgfpathcurveto{\pgfqpoint{3.624141in}{0.922758in}}{\pgfqpoint{3.619812in}{0.933208in}}{\pgfqpoint{3.612108in}{0.940913in}}%
\pgfpathcurveto{\pgfqpoint{3.604403in}{0.948617in}}{\pgfqpoint{3.593952in}{0.952946in}}{\pgfqpoint{3.583057in}{0.952946in}}%
\pgfpathcurveto{\pgfqpoint{3.572161in}{0.952946in}}{\pgfqpoint{3.561710in}{0.948617in}}{\pgfqpoint{3.554006in}{0.940913in}}%
\pgfpathcurveto{\pgfqpoint{3.546302in}{0.933208in}}{\pgfqpoint{3.541973in}{0.922758in}}{\pgfqpoint{3.541973in}{0.911862in}}%
\pgfpathcurveto{\pgfqpoint{3.541973in}{0.900966in}}{\pgfqpoint{3.546302in}{0.890516in}}{\pgfqpoint{3.554006in}{0.882811in}}%
\pgfpathcurveto{\pgfqpoint{3.561710in}{0.875107in}}{\pgfqpoint{3.572161in}{0.870778in}}{\pgfqpoint{3.583057in}{0.870778in}}%
\pgfpathlineto{\pgfqpoint{3.583057in}{0.870778in}}%
\pgfpathclose%
\pgfusepath{stroke}%
\end{pgfscope}%
\begin{pgfscope}%
\pgfpathrectangle{\pgfqpoint{0.688192in}{0.670138in}}{\pgfqpoint{7.111808in}{5.129862in}}%
\pgfusepath{clip}%
\pgfsetbuttcap%
\pgfsetroundjoin%
\pgfsetlinewidth{1.003750pt}%
\definecolor{currentstroke}{rgb}{0.000000,0.000000,0.000000}%
\pgfsetstrokecolor{currentstroke}%
\pgfsetdash{}{0pt}%
\pgfpathmoveto{\pgfqpoint{0.931763in}{0.833664in}}%
\pgfpathcurveto{\pgfqpoint{0.942659in}{0.833664in}}{\pgfqpoint{0.953110in}{0.837992in}}{\pgfqpoint{0.960814in}{0.845697in}}%
\pgfpathcurveto{\pgfqpoint{0.968519in}{0.853401in}}{\pgfqpoint{0.972847in}{0.863852in}}{\pgfqpoint{0.972847in}{0.874747in}}%
\pgfpathcurveto{\pgfqpoint{0.972847in}{0.885643in}}{\pgfqpoint{0.968519in}{0.896094in}}{\pgfqpoint{0.960814in}{0.903798in}}%
\pgfpathcurveto{\pgfqpoint{0.953110in}{0.911502in}}{\pgfqpoint{0.942659in}{0.915831in}}{\pgfqpoint{0.931763in}{0.915831in}}%
\pgfpathcurveto{\pgfqpoint{0.920868in}{0.915831in}}{\pgfqpoint{0.910417in}{0.911502in}}{\pgfqpoint{0.902713in}{0.903798in}}%
\pgfpathcurveto{\pgfqpoint{0.895008in}{0.896094in}}{\pgfqpoint{0.890680in}{0.885643in}}{\pgfqpoint{0.890680in}{0.874747in}}%
\pgfpathcurveto{\pgfqpoint{0.890680in}{0.863852in}}{\pgfqpoint{0.895008in}{0.853401in}}{\pgfqpoint{0.902713in}{0.845697in}}%
\pgfpathcurveto{\pgfqpoint{0.910417in}{0.837992in}}{\pgfqpoint{0.920868in}{0.833664in}}{\pgfqpoint{0.931763in}{0.833664in}}%
\pgfpathlineto{\pgfqpoint{0.931763in}{0.833664in}}%
\pgfpathclose%
\pgfusepath{stroke}%
\end{pgfscope}%
\begin{pgfscope}%
\pgfpathrectangle{\pgfqpoint{0.688192in}{0.670138in}}{\pgfqpoint{7.111808in}{5.129862in}}%
\pgfusepath{clip}%
\pgfsetbuttcap%
\pgfsetroundjoin%
\pgfsetlinewidth{1.003750pt}%
\definecolor{currentstroke}{rgb}{0.000000,0.000000,0.000000}%
\pgfsetstrokecolor{currentstroke}%
\pgfsetdash{}{0pt}%
\pgfpathmoveto{\pgfqpoint{6.014792in}{1.437812in}}%
\pgfpathcurveto{\pgfqpoint{6.025688in}{1.437812in}}{\pgfqpoint{6.036139in}{1.442141in}}{\pgfqpoint{6.043843in}{1.449845in}}%
\pgfpathcurveto{\pgfqpoint{6.051547in}{1.457549in}}{\pgfqpoint{6.055876in}{1.468000in}}{\pgfqpoint{6.055876in}{1.478896in}}%
\pgfpathcurveto{\pgfqpoint{6.055876in}{1.489791in}}{\pgfqpoint{6.051547in}{1.500242in}}{\pgfqpoint{6.043843in}{1.507947in}}%
\pgfpathcurveto{\pgfqpoint{6.036139in}{1.515651in}}{\pgfqpoint{6.025688in}{1.519980in}}{\pgfqpoint{6.014792in}{1.519980in}}%
\pgfpathcurveto{\pgfqpoint{6.003897in}{1.519980in}}{\pgfqpoint{5.993446in}{1.515651in}}{\pgfqpoint{5.985741in}{1.507947in}}%
\pgfpathcurveto{\pgfqpoint{5.978037in}{1.500242in}}{\pgfqpoint{5.973708in}{1.489791in}}{\pgfqpoint{5.973708in}{1.478896in}}%
\pgfpathcurveto{\pgfqpoint{5.973708in}{1.468000in}}{\pgfqpoint{5.978037in}{1.457549in}}{\pgfqpoint{5.985741in}{1.449845in}}%
\pgfpathcurveto{\pgfqpoint{5.993446in}{1.442141in}}{\pgfqpoint{6.003897in}{1.437812in}}{\pgfqpoint{6.014792in}{1.437812in}}%
\pgfpathlineto{\pgfqpoint{6.014792in}{1.437812in}}%
\pgfpathclose%
\pgfusepath{stroke}%
\end{pgfscope}%
\begin{pgfscope}%
\pgfpathrectangle{\pgfqpoint{0.688192in}{0.670138in}}{\pgfqpoint{7.111808in}{5.129862in}}%
\pgfusepath{clip}%
\pgfsetbuttcap%
\pgfsetroundjoin%
\pgfsetlinewidth{1.003750pt}%
\definecolor{currentstroke}{rgb}{0.000000,0.000000,0.000000}%
\pgfsetstrokecolor{currentstroke}%
\pgfsetdash{}{0pt}%
\pgfpathmoveto{\pgfqpoint{4.549016in}{0.640923in}}%
\pgfpathcurveto{\pgfqpoint{4.559912in}{0.640923in}}{\pgfqpoint{4.570362in}{0.645252in}}{\pgfqpoint{4.578067in}{0.652956in}}%
\pgfpathcurveto{\pgfqpoint{4.585771in}{0.660660in}}{\pgfqpoint{4.590100in}{0.671111in}}{\pgfqpoint{4.590100in}{0.682007in}}%
\pgfpathcurveto{\pgfqpoint{4.590100in}{0.692902in}}{\pgfqpoint{4.585771in}{0.703353in}}{\pgfqpoint{4.578067in}{0.711057in}}%
\pgfpathcurveto{\pgfqpoint{4.570362in}{0.718762in}}{\pgfqpoint{4.559912in}{0.723090in}}{\pgfqpoint{4.549016in}{0.723090in}}%
\pgfpathcurveto{\pgfqpoint{4.538120in}{0.723090in}}{\pgfqpoint{4.527670in}{0.718762in}}{\pgfqpoint{4.519965in}{0.711057in}}%
\pgfpathcurveto{\pgfqpoint{4.512261in}{0.703353in}}{\pgfqpoint{4.507932in}{0.692902in}}{\pgfqpoint{4.507932in}{0.682007in}}%
\pgfpathcurveto{\pgfqpoint{4.507932in}{0.671111in}}{\pgfqpoint{4.512261in}{0.660660in}}{\pgfqpoint{4.519965in}{0.652956in}}%
\pgfpathcurveto{\pgfqpoint{4.527670in}{0.645252in}}{\pgfqpoint{4.538120in}{0.640923in}}{\pgfqpoint{4.549016in}{0.640923in}}%
\pgfusepath{stroke}%
\end{pgfscope}%
\begin{pgfscope}%
\pgfpathrectangle{\pgfqpoint{0.688192in}{0.670138in}}{\pgfqpoint{7.111808in}{5.129862in}}%
\pgfusepath{clip}%
\pgfsetbuttcap%
\pgfsetroundjoin%
\pgfsetlinewidth{1.003750pt}%
\definecolor{currentstroke}{rgb}{0.000000,0.000000,0.000000}%
\pgfsetstrokecolor{currentstroke}%
\pgfsetdash{}{0pt}%
\pgfpathmoveto{\pgfqpoint{5.112107in}{0.634839in}}%
\pgfpathcurveto{\pgfqpoint{5.123002in}{0.634839in}}{\pgfqpoint{5.133453in}{0.639168in}}{\pgfqpoint{5.141158in}{0.646872in}}%
\pgfpathcurveto{\pgfqpoint{5.148862in}{0.654577in}}{\pgfqpoint{5.153191in}{0.665027in}}{\pgfqpoint{5.153191in}{0.675923in}}%
\pgfpathcurveto{\pgfqpoint{5.153191in}{0.686819in}}{\pgfqpoint{5.148862in}{0.697269in}}{\pgfqpoint{5.141158in}{0.704974in}}%
\pgfpathcurveto{\pgfqpoint{5.133453in}{0.712678in}}{\pgfqpoint{5.123002in}{0.717007in}}{\pgfqpoint{5.112107in}{0.717007in}}%
\pgfpathcurveto{\pgfqpoint{5.101211in}{0.717007in}}{\pgfqpoint{5.090761in}{0.712678in}}{\pgfqpoint{5.083056in}{0.704974in}}%
\pgfpathcurveto{\pgfqpoint{5.075352in}{0.697269in}}{\pgfqpoint{5.071023in}{0.686819in}}{\pgfqpoint{5.071023in}{0.675923in}}%
\pgfpathcurveto{\pgfqpoint{5.071023in}{0.665027in}}{\pgfqpoint{5.075352in}{0.654577in}}{\pgfqpoint{5.083056in}{0.646872in}}%
\pgfpathcurveto{\pgfqpoint{5.090761in}{0.639168in}}{\pgfqpoint{5.101211in}{0.634839in}}{\pgfqpoint{5.112107in}{0.634839in}}%
\pgfusepath{stroke}%
\end{pgfscope}%
\begin{pgfscope}%
\pgfpathrectangle{\pgfqpoint{0.688192in}{0.670138in}}{\pgfqpoint{7.111808in}{5.129862in}}%
\pgfusepath{clip}%
\pgfsetbuttcap%
\pgfsetroundjoin%
\pgfsetlinewidth{1.003750pt}%
\definecolor{currentstroke}{rgb}{0.000000,0.000000,0.000000}%
\pgfsetstrokecolor{currentstroke}%
\pgfsetdash{}{0pt}%
\pgfpathmoveto{\pgfqpoint{0.774443in}{1.098542in}}%
\pgfpathcurveto{\pgfqpoint{0.785338in}{1.098542in}}{\pgfqpoint{0.795789in}{1.102871in}}{\pgfqpoint{0.803493in}{1.110576in}}%
\pgfpathcurveto{\pgfqpoint{0.811198in}{1.118280in}}{\pgfqpoint{0.815526in}{1.128731in}}{\pgfqpoint{0.815526in}{1.139626in}}%
\pgfpathcurveto{\pgfqpoint{0.815526in}{1.150522in}}{\pgfqpoint{0.811198in}{1.160973in}}{\pgfqpoint{0.803493in}{1.168677in}}%
\pgfpathcurveto{\pgfqpoint{0.795789in}{1.176381in}}{\pgfqpoint{0.785338in}{1.180710in}}{\pgfqpoint{0.774443in}{1.180710in}}%
\pgfpathcurveto{\pgfqpoint{0.763547in}{1.180710in}}{\pgfqpoint{0.753096in}{1.176381in}}{\pgfqpoint{0.745392in}{1.168677in}}%
\pgfpathcurveto{\pgfqpoint{0.737688in}{1.160973in}}{\pgfqpoint{0.733359in}{1.150522in}}{\pgfqpoint{0.733359in}{1.139626in}}%
\pgfpathcurveto{\pgfqpoint{0.733359in}{1.128731in}}{\pgfqpoint{0.737688in}{1.118280in}}{\pgfqpoint{0.745392in}{1.110576in}}%
\pgfpathcurveto{\pgfqpoint{0.753096in}{1.102871in}}{\pgfqpoint{0.763547in}{1.098542in}}{\pgfqpoint{0.774443in}{1.098542in}}%
\pgfpathlineto{\pgfqpoint{0.774443in}{1.098542in}}%
\pgfpathclose%
\pgfusepath{stroke}%
\end{pgfscope}%
\begin{pgfscope}%
\pgfpathrectangle{\pgfqpoint{0.688192in}{0.670138in}}{\pgfqpoint{7.111808in}{5.129862in}}%
\pgfusepath{clip}%
\pgfsetbuttcap%
\pgfsetroundjoin%
\pgfsetlinewidth{1.003750pt}%
\definecolor{currentstroke}{rgb}{0.000000,0.000000,0.000000}%
\pgfsetstrokecolor{currentstroke}%
\pgfsetdash{}{0pt}%
\pgfpathmoveto{\pgfqpoint{1.106601in}{0.720443in}}%
\pgfpathcurveto{\pgfqpoint{1.117497in}{0.720443in}}{\pgfqpoint{1.127948in}{0.724772in}}{\pgfqpoint{1.135652in}{0.732476in}}%
\pgfpathcurveto{\pgfqpoint{1.143357in}{0.740180in}}{\pgfqpoint{1.147685in}{0.750631in}}{\pgfqpoint{1.147685in}{0.761527in}}%
\pgfpathcurveto{\pgfqpoint{1.147685in}{0.772422in}}{\pgfqpoint{1.143357in}{0.782873in}}{\pgfqpoint{1.135652in}{0.790577in}}%
\pgfpathcurveto{\pgfqpoint{1.127948in}{0.798282in}}{\pgfqpoint{1.117497in}{0.802611in}}{\pgfqpoint{1.106601in}{0.802611in}}%
\pgfpathcurveto{\pgfqpoint{1.095706in}{0.802611in}}{\pgfqpoint{1.085255in}{0.798282in}}{\pgfqpoint{1.077551in}{0.790577in}}%
\pgfpathcurveto{\pgfqpoint{1.069846in}{0.782873in}}{\pgfqpoint{1.065518in}{0.772422in}}{\pgfqpoint{1.065518in}{0.761527in}}%
\pgfpathcurveto{\pgfqpoint{1.065518in}{0.750631in}}{\pgfqpoint{1.069846in}{0.740180in}}{\pgfqpoint{1.077551in}{0.732476in}}%
\pgfpathcurveto{\pgfqpoint{1.085255in}{0.724772in}}{\pgfqpoint{1.095706in}{0.720443in}}{\pgfqpoint{1.106601in}{0.720443in}}%
\pgfpathlineto{\pgfqpoint{1.106601in}{0.720443in}}%
\pgfpathclose%
\pgfusepath{stroke}%
\end{pgfscope}%
\begin{pgfscope}%
\pgfpathrectangle{\pgfqpoint{0.688192in}{0.670138in}}{\pgfqpoint{7.111808in}{5.129862in}}%
\pgfusepath{clip}%
\pgfsetbuttcap%
\pgfsetroundjoin%
\pgfsetlinewidth{1.003750pt}%
\definecolor{currentstroke}{rgb}{0.000000,0.000000,0.000000}%
\pgfsetstrokecolor{currentstroke}%
\pgfsetdash{}{0pt}%
\pgfpathmoveto{\pgfqpoint{0.880316in}{0.913368in}}%
\pgfpathcurveto{\pgfqpoint{0.891212in}{0.913368in}}{\pgfqpoint{0.901663in}{0.917697in}}{\pgfqpoint{0.909367in}{0.925401in}}%
\pgfpathcurveto{\pgfqpoint{0.917071in}{0.933106in}}{\pgfqpoint{0.921400in}{0.943557in}}{\pgfqpoint{0.921400in}{0.954452in}}%
\pgfpathcurveto{\pgfqpoint{0.921400in}{0.965348in}}{\pgfqpoint{0.917071in}{0.975799in}}{\pgfqpoint{0.909367in}{0.983503in}}%
\pgfpathcurveto{\pgfqpoint{0.901663in}{0.991207in}}{\pgfqpoint{0.891212in}{0.995536in}}{\pgfqpoint{0.880316in}{0.995536in}}%
\pgfpathcurveto{\pgfqpoint{0.869421in}{0.995536in}}{\pgfqpoint{0.858970in}{0.991207in}}{\pgfqpoint{0.851266in}{0.983503in}}%
\pgfpathcurveto{\pgfqpoint{0.843561in}{0.975799in}}{\pgfqpoint{0.839232in}{0.965348in}}{\pgfqpoint{0.839232in}{0.954452in}}%
\pgfpathcurveto{\pgfqpoint{0.839232in}{0.943557in}}{\pgfqpoint{0.843561in}{0.933106in}}{\pgfqpoint{0.851266in}{0.925401in}}%
\pgfpathcurveto{\pgfqpoint{0.858970in}{0.917697in}}{\pgfqpoint{0.869421in}{0.913368in}}{\pgfqpoint{0.880316in}{0.913368in}}%
\pgfpathlineto{\pgfqpoint{0.880316in}{0.913368in}}%
\pgfpathclose%
\pgfusepath{stroke}%
\end{pgfscope}%
\begin{pgfscope}%
\pgfpathrectangle{\pgfqpoint{0.688192in}{0.670138in}}{\pgfqpoint{7.111808in}{5.129862in}}%
\pgfusepath{clip}%
\pgfsetbuttcap%
\pgfsetroundjoin%
\pgfsetlinewidth{1.003750pt}%
\definecolor{currentstroke}{rgb}{0.000000,0.000000,0.000000}%
\pgfsetstrokecolor{currentstroke}%
\pgfsetdash{}{0pt}%
\pgfpathmoveto{\pgfqpoint{5.988842in}{0.629054in}}%
\pgfpathcurveto{\pgfqpoint{5.999738in}{0.629054in}}{\pgfqpoint{6.010189in}{0.633383in}}{\pgfqpoint{6.017893in}{0.641087in}}%
\pgfpathcurveto{\pgfqpoint{6.025598in}{0.648792in}}{\pgfqpoint{6.029926in}{0.659242in}}{\pgfqpoint{6.029926in}{0.670138in}}%
\pgfpathcurveto{\pgfqpoint{6.029926in}{0.681034in}}{\pgfqpoint{6.025598in}{0.691484in}}{\pgfqpoint{6.017893in}{0.699189in}}%
\pgfpathcurveto{\pgfqpoint{6.010189in}{0.706893in}}{\pgfqpoint{5.999738in}{0.711222in}}{\pgfqpoint{5.988842in}{0.711222in}}%
\pgfpathcurveto{\pgfqpoint{5.977947in}{0.711222in}}{\pgfqpoint{5.967496in}{0.706893in}}{\pgfqpoint{5.959792in}{0.699189in}}%
\pgfpathcurveto{\pgfqpoint{5.952087in}{0.691484in}}{\pgfqpoint{5.947759in}{0.681034in}}{\pgfqpoint{5.947759in}{0.670138in}}%
\pgfpathcurveto{\pgfqpoint{5.947759in}{0.659242in}}{\pgfqpoint{5.952087in}{0.648792in}}{\pgfqpoint{5.959792in}{0.641087in}}%
\pgfpathcurveto{\pgfqpoint{5.967496in}{0.633383in}}{\pgfqpoint{5.977947in}{0.629054in}}{\pgfqpoint{5.988842in}{0.629054in}}%
\pgfusepath{stroke}%
\end{pgfscope}%
\begin{pgfscope}%
\pgfpathrectangle{\pgfqpoint{0.688192in}{0.670138in}}{\pgfqpoint{7.111808in}{5.129862in}}%
\pgfusepath{clip}%
\pgfsetbuttcap%
\pgfsetroundjoin%
\pgfsetlinewidth{1.003750pt}%
\definecolor{currentstroke}{rgb}{0.000000,0.000000,0.000000}%
\pgfsetstrokecolor{currentstroke}%
\pgfsetdash{}{0pt}%
\pgfpathmoveto{\pgfqpoint{0.688192in}{5.214003in}}%
\pgfpathcurveto{\pgfqpoint{0.699087in}{5.214003in}}{\pgfqpoint{0.709538in}{5.218332in}}{\pgfqpoint{0.717242in}{5.226037in}}%
\pgfpathcurveto{\pgfqpoint{0.724947in}{5.233741in}}{\pgfqpoint{0.729275in}{5.244192in}}{\pgfqpoint{0.729275in}{5.255087in}}%
\pgfpathcurveto{\pgfqpoint{0.729275in}{5.265983in}}{\pgfqpoint{0.724947in}{5.276434in}}{\pgfqpoint{0.717242in}{5.284138in}}%
\pgfpathcurveto{\pgfqpoint{0.709538in}{5.291842in}}{\pgfqpoint{0.699087in}{5.296171in}}{\pgfqpoint{0.688192in}{5.296171in}}%
\pgfpathcurveto{\pgfqpoint{0.677296in}{5.296171in}}{\pgfqpoint{0.666845in}{5.291842in}}{\pgfqpoint{0.659141in}{5.284138in}}%
\pgfpathcurveto{\pgfqpoint{0.651436in}{5.276434in}}{\pgfqpoint{0.647108in}{5.265983in}}{\pgfqpoint{0.647108in}{5.255087in}}%
\pgfpathcurveto{\pgfqpoint{0.647108in}{5.244192in}}{\pgfqpoint{0.651436in}{5.233741in}}{\pgfqpoint{0.659141in}{5.226037in}}%
\pgfpathcurveto{\pgfqpoint{0.666845in}{5.218332in}}{\pgfqpoint{0.677296in}{5.214003in}}{\pgfqpoint{0.688192in}{5.214003in}}%
\pgfpathlineto{\pgfqpoint{0.688192in}{5.214003in}}%
\pgfpathclose%
\pgfusepath{stroke}%
\end{pgfscope}%
\begin{pgfscope}%
\pgfpathrectangle{\pgfqpoint{0.688192in}{0.670138in}}{\pgfqpoint{7.111808in}{5.129862in}}%
\pgfusepath{clip}%
\pgfsetbuttcap%
\pgfsetroundjoin%
\pgfsetlinewidth{1.003750pt}%
\definecolor{currentstroke}{rgb}{0.000000,0.000000,0.000000}%
\pgfsetstrokecolor{currentstroke}%
\pgfsetdash{}{0pt}%
\pgfpathmoveto{\pgfqpoint{0.882669in}{0.877855in}}%
\pgfpathcurveto{\pgfqpoint{0.893565in}{0.877855in}}{\pgfqpoint{0.904016in}{0.882184in}}{\pgfqpoint{0.911720in}{0.889888in}}%
\pgfpathcurveto{\pgfqpoint{0.919424in}{0.897592in}}{\pgfqpoint{0.923753in}{0.908043in}}{\pgfqpoint{0.923753in}{0.918939in}}%
\pgfpathcurveto{\pgfqpoint{0.923753in}{0.929834in}}{\pgfqpoint{0.919424in}{0.940285in}}{\pgfqpoint{0.911720in}{0.947989in}}%
\pgfpathcurveto{\pgfqpoint{0.904016in}{0.955694in}}{\pgfqpoint{0.893565in}{0.960023in}}{\pgfqpoint{0.882669in}{0.960023in}}%
\pgfpathcurveto{\pgfqpoint{0.871774in}{0.960023in}}{\pgfqpoint{0.861323in}{0.955694in}}{\pgfqpoint{0.853619in}{0.947989in}}%
\pgfpathcurveto{\pgfqpoint{0.845914in}{0.940285in}}{\pgfqpoint{0.841585in}{0.929834in}}{\pgfqpoint{0.841585in}{0.918939in}}%
\pgfpathcurveto{\pgfqpoint{0.841585in}{0.908043in}}{\pgfqpoint{0.845914in}{0.897592in}}{\pgfqpoint{0.853619in}{0.889888in}}%
\pgfpathcurveto{\pgfqpoint{0.861323in}{0.882184in}}{\pgfqpoint{0.871774in}{0.877855in}}{\pgfqpoint{0.882669in}{0.877855in}}%
\pgfpathlineto{\pgfqpoint{0.882669in}{0.877855in}}%
\pgfpathclose%
\pgfusepath{stroke}%
\end{pgfscope}%
\begin{pgfscope}%
\pgfpathrectangle{\pgfqpoint{0.688192in}{0.670138in}}{\pgfqpoint{7.111808in}{5.129862in}}%
\pgfusepath{clip}%
\pgfsetbuttcap%
\pgfsetroundjoin%
\pgfsetlinewidth{1.003750pt}%
\definecolor{currentstroke}{rgb}{0.000000,0.000000,0.000000}%
\pgfsetstrokecolor{currentstroke}%
\pgfsetdash{}{0pt}%
\pgfpathmoveto{\pgfqpoint{1.141797in}{0.719357in}}%
\pgfpathcurveto{\pgfqpoint{1.152693in}{0.719357in}}{\pgfqpoint{1.163143in}{0.723686in}}{\pgfqpoint{1.170848in}{0.731391in}}%
\pgfpathcurveto{\pgfqpoint{1.178552in}{0.739095in}}{\pgfqpoint{1.182881in}{0.749546in}}{\pgfqpoint{1.182881in}{0.760441in}}%
\pgfpathcurveto{\pgfqpoint{1.182881in}{0.771337in}}{\pgfqpoint{1.178552in}{0.781788in}}{\pgfqpoint{1.170848in}{0.789492in}}%
\pgfpathcurveto{\pgfqpoint{1.163143in}{0.797196in}}{\pgfqpoint{1.152693in}{0.801525in}}{\pgfqpoint{1.141797in}{0.801525in}}%
\pgfpathcurveto{\pgfqpoint{1.130902in}{0.801525in}}{\pgfqpoint{1.120451in}{0.797196in}}{\pgfqpoint{1.112746in}{0.789492in}}%
\pgfpathcurveto{\pgfqpoint{1.105042in}{0.781788in}}{\pgfqpoint{1.100713in}{0.771337in}}{\pgfqpoint{1.100713in}{0.760441in}}%
\pgfpathcurveto{\pgfqpoint{1.100713in}{0.749546in}}{\pgfqpoint{1.105042in}{0.739095in}}{\pgfqpoint{1.112746in}{0.731391in}}%
\pgfpathcurveto{\pgfqpoint{1.120451in}{0.723686in}}{\pgfqpoint{1.130902in}{0.719357in}}{\pgfqpoint{1.141797in}{0.719357in}}%
\pgfpathlineto{\pgfqpoint{1.141797in}{0.719357in}}%
\pgfpathclose%
\pgfusepath{stroke}%
\end{pgfscope}%
\begin{pgfscope}%
\pgfpathrectangle{\pgfqpoint{0.688192in}{0.670138in}}{\pgfqpoint{7.111808in}{5.129862in}}%
\pgfusepath{clip}%
\pgfsetbuttcap%
\pgfsetroundjoin%
\pgfsetlinewidth{1.003750pt}%
\definecolor{currentstroke}{rgb}{0.000000,0.000000,0.000000}%
\pgfsetstrokecolor{currentstroke}%
\pgfsetdash{}{0pt}%
\pgfpathmoveto{\pgfqpoint{0.886861in}{0.869726in}}%
\pgfpathcurveto{\pgfqpoint{0.897756in}{0.869726in}}{\pgfqpoint{0.908207in}{0.874055in}}{\pgfqpoint{0.915911in}{0.881759in}}%
\pgfpathcurveto{\pgfqpoint{0.923616in}{0.889463in}}{\pgfqpoint{0.927945in}{0.899914in}}{\pgfqpoint{0.927945in}{0.910810in}}%
\pgfpathcurveto{\pgfqpoint{0.927945in}{0.921705in}}{\pgfqpoint{0.923616in}{0.932156in}}{\pgfqpoint{0.915911in}{0.939861in}}%
\pgfpathcurveto{\pgfqpoint{0.908207in}{0.947565in}}{\pgfqpoint{0.897756in}{0.951894in}}{\pgfqpoint{0.886861in}{0.951894in}}%
\pgfpathcurveto{\pgfqpoint{0.875965in}{0.951894in}}{\pgfqpoint{0.865514in}{0.947565in}}{\pgfqpoint{0.857810in}{0.939861in}}%
\pgfpathcurveto{\pgfqpoint{0.850106in}{0.932156in}}{\pgfqpoint{0.845777in}{0.921705in}}{\pgfqpoint{0.845777in}{0.910810in}}%
\pgfpathcurveto{\pgfqpoint{0.845777in}{0.899914in}}{\pgfqpoint{0.850106in}{0.889463in}}{\pgfqpoint{0.857810in}{0.881759in}}%
\pgfpathcurveto{\pgfqpoint{0.865514in}{0.874055in}}{\pgfqpoint{0.875965in}{0.869726in}}{\pgfqpoint{0.886861in}{0.869726in}}%
\pgfpathlineto{\pgfqpoint{0.886861in}{0.869726in}}%
\pgfpathclose%
\pgfusepath{stroke}%
\end{pgfscope}%
\begin{pgfscope}%
\pgfpathrectangle{\pgfqpoint{0.688192in}{0.670138in}}{\pgfqpoint{7.111808in}{5.129862in}}%
\pgfusepath{clip}%
\pgfsetbuttcap%
\pgfsetroundjoin%
\pgfsetlinewidth{1.003750pt}%
\definecolor{currentstroke}{rgb}{0.000000,0.000000,0.000000}%
\pgfsetstrokecolor{currentstroke}%
\pgfsetdash{}{0pt}%
\pgfpathmoveto{\pgfqpoint{1.090088in}{0.725570in}}%
\pgfpathcurveto{\pgfqpoint{1.100984in}{0.725570in}}{\pgfqpoint{1.111434in}{0.729898in}}{\pgfqpoint{1.119139in}{0.737603in}}%
\pgfpathcurveto{\pgfqpoint{1.126843in}{0.745307in}}{\pgfqpoint{1.131172in}{0.755758in}}{\pgfqpoint{1.131172in}{0.766654in}}%
\pgfpathcurveto{\pgfqpoint{1.131172in}{0.777549in}}{\pgfqpoint{1.126843in}{0.788000in}}{\pgfqpoint{1.119139in}{0.795704in}}%
\pgfpathcurveto{\pgfqpoint{1.111434in}{0.803409in}}{\pgfqpoint{1.100984in}{0.807737in}}{\pgfqpoint{1.090088in}{0.807737in}}%
\pgfpathcurveto{\pgfqpoint{1.079192in}{0.807737in}}{\pgfqpoint{1.068742in}{0.803409in}}{\pgfqpoint{1.061037in}{0.795704in}}%
\pgfpathcurveto{\pgfqpoint{1.053333in}{0.788000in}}{\pgfqpoint{1.049004in}{0.777549in}}{\pgfqpoint{1.049004in}{0.766654in}}%
\pgfpathcurveto{\pgfqpoint{1.049004in}{0.755758in}}{\pgfqpoint{1.053333in}{0.745307in}}{\pgfqpoint{1.061037in}{0.737603in}}%
\pgfpathcurveto{\pgfqpoint{1.068742in}{0.729898in}}{\pgfqpoint{1.079192in}{0.725570in}}{\pgfqpoint{1.090088in}{0.725570in}}%
\pgfpathlineto{\pgfqpoint{1.090088in}{0.725570in}}%
\pgfpathclose%
\pgfusepath{stroke}%
\end{pgfscope}%
\begin{pgfscope}%
\pgfpathrectangle{\pgfqpoint{0.688192in}{0.670138in}}{\pgfqpoint{7.111808in}{5.129862in}}%
\pgfusepath{clip}%
\pgfsetbuttcap%
\pgfsetroundjoin%
\pgfsetlinewidth{1.003750pt}%
\definecolor{currentstroke}{rgb}{0.000000,0.000000,0.000000}%
\pgfsetstrokecolor{currentstroke}%
\pgfsetdash{}{0pt}%
\pgfpathmoveto{\pgfqpoint{4.923457in}{0.834585in}}%
\pgfpathcurveto{\pgfqpoint{4.934352in}{0.834585in}}{\pgfqpoint{4.944803in}{0.838914in}}{\pgfqpoint{4.952507in}{0.846618in}}%
\pgfpathcurveto{\pgfqpoint{4.960212in}{0.854323in}}{\pgfqpoint{4.964540in}{0.864773in}}{\pgfqpoint{4.964540in}{0.875669in}}%
\pgfpathcurveto{\pgfqpoint{4.964540in}{0.886565in}}{\pgfqpoint{4.960212in}{0.897015in}}{\pgfqpoint{4.952507in}{0.904720in}}%
\pgfpathcurveto{\pgfqpoint{4.944803in}{0.912424in}}{\pgfqpoint{4.934352in}{0.916753in}}{\pgfqpoint{4.923457in}{0.916753in}}%
\pgfpathcurveto{\pgfqpoint{4.912561in}{0.916753in}}{\pgfqpoint{4.902110in}{0.912424in}}{\pgfqpoint{4.894406in}{0.904720in}}%
\pgfpathcurveto{\pgfqpoint{4.886702in}{0.897015in}}{\pgfqpoint{4.882373in}{0.886565in}}{\pgfqpoint{4.882373in}{0.875669in}}%
\pgfpathcurveto{\pgfqpoint{4.882373in}{0.864773in}}{\pgfqpoint{4.886702in}{0.854323in}}{\pgfqpoint{4.894406in}{0.846618in}}%
\pgfpathcurveto{\pgfqpoint{4.902110in}{0.838914in}}{\pgfqpoint{4.912561in}{0.834585in}}{\pgfqpoint{4.923457in}{0.834585in}}%
\pgfpathlineto{\pgfqpoint{4.923457in}{0.834585in}}%
\pgfpathclose%
\pgfusepath{stroke}%
\end{pgfscope}%
\begin{pgfscope}%
\pgfpathrectangle{\pgfqpoint{0.688192in}{0.670138in}}{\pgfqpoint{7.111808in}{5.129862in}}%
\pgfusepath{clip}%
\pgfsetbuttcap%
\pgfsetroundjoin%
\pgfsetlinewidth{1.003750pt}%
\definecolor{currentstroke}{rgb}{0.000000,0.000000,0.000000}%
\pgfsetstrokecolor{currentstroke}%
\pgfsetdash{}{0pt}%
\pgfpathmoveto{\pgfqpoint{1.351667in}{0.711476in}}%
\pgfpathcurveto{\pgfqpoint{1.362562in}{0.711476in}}{\pgfqpoint{1.373013in}{0.715805in}}{\pgfqpoint{1.380718in}{0.723510in}}%
\pgfpathcurveto{\pgfqpoint{1.388422in}{0.731214in}}{\pgfqpoint{1.392751in}{0.741665in}}{\pgfqpoint{1.392751in}{0.752560in}}%
\pgfpathcurveto{\pgfqpoint{1.392751in}{0.763456in}}{\pgfqpoint{1.388422in}{0.773907in}}{\pgfqpoint{1.380718in}{0.781611in}}%
\pgfpathcurveto{\pgfqpoint{1.373013in}{0.789315in}}{\pgfqpoint{1.362562in}{0.793644in}}{\pgfqpoint{1.351667in}{0.793644in}}%
\pgfpathcurveto{\pgfqpoint{1.340771in}{0.793644in}}{\pgfqpoint{1.330321in}{0.789315in}}{\pgfqpoint{1.322616in}{0.781611in}}%
\pgfpathcurveto{\pgfqpoint{1.314912in}{0.773907in}}{\pgfqpoint{1.310583in}{0.763456in}}{\pgfqpoint{1.310583in}{0.752560in}}%
\pgfpathcurveto{\pgfqpoint{1.310583in}{0.741665in}}{\pgfqpoint{1.314912in}{0.731214in}}{\pgfqpoint{1.322616in}{0.723510in}}%
\pgfpathcurveto{\pgfqpoint{1.330321in}{0.715805in}}{\pgfqpoint{1.340771in}{0.711476in}}{\pgfqpoint{1.351667in}{0.711476in}}%
\pgfpathlineto{\pgfqpoint{1.351667in}{0.711476in}}%
\pgfpathclose%
\pgfusepath{stroke}%
\end{pgfscope}%
\begin{pgfscope}%
\pgfpathrectangle{\pgfqpoint{0.688192in}{0.670138in}}{\pgfqpoint{7.111808in}{5.129862in}}%
\pgfusepath{clip}%
\pgfsetbuttcap%
\pgfsetroundjoin%
\pgfsetlinewidth{1.003750pt}%
\definecolor{currentstroke}{rgb}{0.000000,0.000000,0.000000}%
\pgfsetstrokecolor{currentstroke}%
\pgfsetdash{}{0pt}%
\pgfpathmoveto{\pgfqpoint{0.899610in}{0.859043in}}%
\pgfpathcurveto{\pgfqpoint{0.910506in}{0.859043in}}{\pgfqpoint{0.920957in}{0.863371in}}{\pgfqpoint{0.928661in}{0.871076in}}%
\pgfpathcurveto{\pgfqpoint{0.936365in}{0.878780in}}{\pgfqpoint{0.940694in}{0.889231in}}{\pgfqpoint{0.940694in}{0.900126in}}%
\pgfpathcurveto{\pgfqpoint{0.940694in}{0.911022in}}{\pgfqpoint{0.936365in}{0.921473in}}{\pgfqpoint{0.928661in}{0.929177in}}%
\pgfpathcurveto{\pgfqpoint{0.920957in}{0.936881in}}{\pgfqpoint{0.910506in}{0.941210in}}{\pgfqpoint{0.899610in}{0.941210in}}%
\pgfpathcurveto{\pgfqpoint{0.888715in}{0.941210in}}{\pgfqpoint{0.878264in}{0.936881in}}{\pgfqpoint{0.870560in}{0.929177in}}%
\pgfpathcurveto{\pgfqpoint{0.862855in}{0.921473in}}{\pgfqpoint{0.858526in}{0.911022in}}{\pgfqpoint{0.858526in}{0.900126in}}%
\pgfpathcurveto{\pgfqpoint{0.858526in}{0.889231in}}{\pgfqpoint{0.862855in}{0.878780in}}{\pgfqpoint{0.870560in}{0.871076in}}%
\pgfpathcurveto{\pgfqpoint{0.878264in}{0.863371in}}{\pgfqpoint{0.888715in}{0.859043in}}{\pgfqpoint{0.899610in}{0.859043in}}%
\pgfpathlineto{\pgfqpoint{0.899610in}{0.859043in}}%
\pgfpathclose%
\pgfusepath{stroke}%
\end{pgfscope}%
\begin{pgfscope}%
\pgfpathrectangle{\pgfqpoint{0.688192in}{0.670138in}}{\pgfqpoint{7.111808in}{5.129862in}}%
\pgfusepath{clip}%
\pgfsetbuttcap%
\pgfsetroundjoin%
\pgfsetlinewidth{1.003750pt}%
\definecolor{currentstroke}{rgb}{0.000000,0.000000,0.000000}%
\pgfsetstrokecolor{currentstroke}%
\pgfsetdash{}{0pt}%
\pgfpathmoveto{\pgfqpoint{1.421141in}{0.710539in}}%
\pgfpathcurveto{\pgfqpoint{1.432036in}{0.710539in}}{\pgfqpoint{1.442487in}{0.714867in}}{\pgfqpoint{1.450191in}{0.722572in}}%
\pgfpathcurveto{\pgfqpoint{1.457896in}{0.730276in}}{\pgfqpoint{1.462225in}{0.740727in}}{\pgfqpoint{1.462225in}{0.751622in}}%
\pgfpathcurveto{\pgfqpoint{1.462225in}{0.762518in}}{\pgfqpoint{1.457896in}{0.772969in}}{\pgfqpoint{1.450191in}{0.780673in}}%
\pgfpathcurveto{\pgfqpoint{1.442487in}{0.788377in}}{\pgfqpoint{1.432036in}{0.792706in}}{\pgfqpoint{1.421141in}{0.792706in}}%
\pgfpathcurveto{\pgfqpoint{1.410245in}{0.792706in}}{\pgfqpoint{1.399794in}{0.788377in}}{\pgfqpoint{1.392090in}{0.780673in}}%
\pgfpathcurveto{\pgfqpoint{1.384386in}{0.772969in}}{\pgfqpoint{1.380057in}{0.762518in}}{\pgfqpoint{1.380057in}{0.751622in}}%
\pgfpathcurveto{\pgfqpoint{1.380057in}{0.740727in}}{\pgfqpoint{1.384386in}{0.730276in}}{\pgfqpoint{1.392090in}{0.722572in}}%
\pgfpathcurveto{\pgfqpoint{1.399794in}{0.714867in}}{\pgfqpoint{1.410245in}{0.710539in}}{\pgfqpoint{1.421141in}{0.710539in}}%
\pgfpathlineto{\pgfqpoint{1.421141in}{0.710539in}}%
\pgfpathclose%
\pgfusepath{stroke}%
\end{pgfscope}%
\begin{pgfscope}%
\pgfpathrectangle{\pgfqpoint{0.688192in}{0.670138in}}{\pgfqpoint{7.111808in}{5.129862in}}%
\pgfusepath{clip}%
\pgfsetbuttcap%
\pgfsetroundjoin%
\pgfsetlinewidth{1.003750pt}%
\definecolor{currentstroke}{rgb}{0.000000,0.000000,0.000000}%
\pgfsetstrokecolor{currentstroke}%
\pgfsetdash{}{0pt}%
\pgfpathmoveto{\pgfqpoint{4.603680in}{0.639749in}}%
\pgfpathcurveto{\pgfqpoint{4.614576in}{0.639749in}}{\pgfqpoint{4.625026in}{0.644077in}}{\pgfqpoint{4.632731in}{0.651782in}}%
\pgfpathcurveto{\pgfqpoint{4.640435in}{0.659486in}}{\pgfqpoint{4.644764in}{0.669937in}}{\pgfqpoint{4.644764in}{0.680832in}}%
\pgfpathcurveto{\pgfqpoint{4.644764in}{0.691728in}}{\pgfqpoint{4.640435in}{0.702179in}}{\pgfqpoint{4.632731in}{0.709883in}}%
\pgfpathcurveto{\pgfqpoint{4.625026in}{0.717587in}}{\pgfqpoint{4.614576in}{0.721916in}}{\pgfqpoint{4.603680in}{0.721916in}}%
\pgfpathcurveto{\pgfqpoint{4.592784in}{0.721916in}}{\pgfqpoint{4.582334in}{0.717587in}}{\pgfqpoint{4.574629in}{0.709883in}}%
\pgfpathcurveto{\pgfqpoint{4.566925in}{0.702179in}}{\pgfqpoint{4.562596in}{0.691728in}}{\pgfqpoint{4.562596in}{0.680832in}}%
\pgfpathcurveto{\pgfqpoint{4.562596in}{0.669937in}}{\pgfqpoint{4.566925in}{0.659486in}}{\pgfqpoint{4.574629in}{0.651782in}}%
\pgfpathcurveto{\pgfqpoint{4.582334in}{0.644077in}}{\pgfqpoint{4.592784in}{0.639749in}}{\pgfqpoint{4.603680in}{0.639749in}}%
\pgfusepath{stroke}%
\end{pgfscope}%
\begin{pgfscope}%
\pgfpathrectangle{\pgfqpoint{0.688192in}{0.670138in}}{\pgfqpoint{7.111808in}{5.129862in}}%
\pgfusepath{clip}%
\pgfsetbuttcap%
\pgfsetroundjoin%
\pgfsetlinewidth{1.003750pt}%
\definecolor{currentstroke}{rgb}{0.000000,0.000000,0.000000}%
\pgfsetstrokecolor{currentstroke}%
\pgfsetdash{}{0pt}%
\pgfpathmoveto{\pgfqpoint{2.821283in}{0.839158in}}%
\pgfpathcurveto{\pgfqpoint{2.832179in}{0.839158in}}{\pgfqpoint{2.842630in}{0.843486in}}{\pgfqpoint{2.850334in}{0.851191in}}%
\pgfpathcurveto{\pgfqpoint{2.858038in}{0.858895in}}{\pgfqpoint{2.862367in}{0.869346in}}{\pgfqpoint{2.862367in}{0.880241in}}%
\pgfpathcurveto{\pgfqpoint{2.862367in}{0.891137in}}{\pgfqpoint{2.858038in}{0.901588in}}{\pgfqpoint{2.850334in}{0.909292in}}%
\pgfpathcurveto{\pgfqpoint{2.842630in}{0.916996in}}{\pgfqpoint{2.832179in}{0.921325in}}{\pgfqpoint{2.821283in}{0.921325in}}%
\pgfpathcurveto{\pgfqpoint{2.810388in}{0.921325in}}{\pgfqpoint{2.799937in}{0.916996in}}{\pgfqpoint{2.792232in}{0.909292in}}%
\pgfpathcurveto{\pgfqpoint{2.784528in}{0.901588in}}{\pgfqpoint{2.780199in}{0.891137in}}{\pgfqpoint{2.780199in}{0.880241in}}%
\pgfpathcurveto{\pgfqpoint{2.780199in}{0.869346in}}{\pgfqpoint{2.784528in}{0.858895in}}{\pgfqpoint{2.792232in}{0.851191in}}%
\pgfpathcurveto{\pgfqpoint{2.799937in}{0.843486in}}{\pgfqpoint{2.810388in}{0.839158in}}{\pgfqpoint{2.821283in}{0.839158in}}%
\pgfpathlineto{\pgfqpoint{2.821283in}{0.839158in}}%
\pgfpathclose%
\pgfusepath{stroke}%
\end{pgfscope}%
\begin{pgfscope}%
\pgfpathrectangle{\pgfqpoint{0.688192in}{0.670138in}}{\pgfqpoint{7.111808in}{5.129862in}}%
\pgfusepath{clip}%
\pgfsetbuttcap%
\pgfsetroundjoin%
\pgfsetlinewidth{1.003750pt}%
\definecolor{currentstroke}{rgb}{0.000000,0.000000,0.000000}%
\pgfsetstrokecolor{currentstroke}%
\pgfsetdash{}{0pt}%
\pgfpathmoveto{\pgfqpoint{3.482566in}{1.154944in}}%
\pgfpathcurveto{\pgfqpoint{3.493461in}{1.154944in}}{\pgfqpoint{3.503912in}{1.159273in}}{\pgfqpoint{3.511616in}{1.166977in}}%
\pgfpathcurveto{\pgfqpoint{3.519321in}{1.174682in}}{\pgfqpoint{3.523650in}{1.185132in}}{\pgfqpoint{3.523650in}{1.196028in}}%
\pgfpathcurveto{\pgfqpoint{3.523650in}{1.206924in}}{\pgfqpoint{3.519321in}{1.217374in}}{\pgfqpoint{3.511616in}{1.225079in}}%
\pgfpathcurveto{\pgfqpoint{3.503912in}{1.232783in}}{\pgfqpoint{3.493461in}{1.237112in}}{\pgfqpoint{3.482566in}{1.237112in}}%
\pgfpathcurveto{\pgfqpoint{3.471670in}{1.237112in}}{\pgfqpoint{3.461219in}{1.232783in}}{\pgfqpoint{3.453515in}{1.225079in}}%
\pgfpathcurveto{\pgfqpoint{3.445811in}{1.217374in}}{\pgfqpoint{3.441482in}{1.206924in}}{\pgfqpoint{3.441482in}{1.196028in}}%
\pgfpathcurveto{\pgfqpoint{3.441482in}{1.185132in}}{\pgfqpoint{3.445811in}{1.174682in}}{\pgfqpoint{3.453515in}{1.166977in}}%
\pgfpathcurveto{\pgfqpoint{3.461219in}{1.159273in}}{\pgfqpoint{3.471670in}{1.154944in}}{\pgfqpoint{3.482566in}{1.154944in}}%
\pgfpathlineto{\pgfqpoint{3.482566in}{1.154944in}}%
\pgfpathclose%
\pgfusepath{stroke}%
\end{pgfscope}%
\begin{pgfscope}%
\pgfpathrectangle{\pgfqpoint{0.688192in}{0.670138in}}{\pgfqpoint{7.111808in}{5.129862in}}%
\pgfusepath{clip}%
\pgfsetbuttcap%
\pgfsetroundjoin%
\pgfsetlinewidth{1.003750pt}%
\definecolor{currentstroke}{rgb}{0.000000,0.000000,0.000000}%
\pgfsetstrokecolor{currentstroke}%
\pgfsetdash{}{0pt}%
\pgfpathmoveto{\pgfqpoint{4.176025in}{0.641739in}}%
\pgfpathcurveto{\pgfqpoint{4.186921in}{0.641739in}}{\pgfqpoint{4.197371in}{0.646068in}}{\pgfqpoint{4.205076in}{0.653772in}}%
\pgfpathcurveto{\pgfqpoint{4.212780in}{0.661476in}}{\pgfqpoint{4.217109in}{0.671927in}}{\pgfqpoint{4.217109in}{0.682823in}}%
\pgfpathcurveto{\pgfqpoint{4.217109in}{0.693718in}}{\pgfqpoint{4.212780in}{0.704169in}}{\pgfqpoint{4.205076in}{0.711873in}}%
\pgfpathcurveto{\pgfqpoint{4.197371in}{0.719578in}}{\pgfqpoint{4.186921in}{0.723906in}}{\pgfqpoint{4.176025in}{0.723906in}}%
\pgfpathcurveto{\pgfqpoint{4.165129in}{0.723906in}}{\pgfqpoint{4.154679in}{0.719578in}}{\pgfqpoint{4.146974in}{0.711873in}}%
\pgfpathcurveto{\pgfqpoint{4.139270in}{0.704169in}}{\pgfqpoint{4.134941in}{0.693718in}}{\pgfqpoint{4.134941in}{0.682823in}}%
\pgfpathcurveto{\pgfqpoint{4.134941in}{0.671927in}}{\pgfqpoint{4.139270in}{0.661476in}}{\pgfqpoint{4.146974in}{0.653772in}}%
\pgfpathcurveto{\pgfqpoint{4.154679in}{0.646068in}}{\pgfqpoint{4.165129in}{0.641739in}}{\pgfqpoint{4.176025in}{0.641739in}}%
\pgfusepath{stroke}%
\end{pgfscope}%
\begin{pgfscope}%
\pgfpathrectangle{\pgfqpoint{0.688192in}{0.670138in}}{\pgfqpoint{7.111808in}{5.129862in}}%
\pgfusepath{clip}%
\pgfsetbuttcap%
\pgfsetroundjoin%
\pgfsetlinewidth{1.003750pt}%
\definecolor{currentstroke}{rgb}{0.000000,0.000000,0.000000}%
\pgfsetstrokecolor{currentstroke}%
\pgfsetdash{}{0pt}%
\pgfpathmoveto{\pgfqpoint{0.789763in}{1.038554in}}%
\pgfpathcurveto{\pgfqpoint{0.800659in}{1.038554in}}{\pgfqpoint{0.811109in}{1.042883in}}{\pgfqpoint{0.818814in}{1.050587in}}%
\pgfpathcurveto{\pgfqpoint{0.826518in}{1.058291in}}{\pgfqpoint{0.830847in}{1.068742in}}{\pgfqpoint{0.830847in}{1.079638in}}%
\pgfpathcurveto{\pgfqpoint{0.830847in}{1.090533in}}{\pgfqpoint{0.826518in}{1.100984in}}{\pgfqpoint{0.818814in}{1.108688in}}%
\pgfpathcurveto{\pgfqpoint{0.811109in}{1.116393in}}{\pgfqpoint{0.800659in}{1.120722in}}{\pgfqpoint{0.789763in}{1.120722in}}%
\pgfpathcurveto{\pgfqpoint{0.778868in}{1.120722in}}{\pgfqpoint{0.768417in}{1.116393in}}{\pgfqpoint{0.760712in}{1.108688in}}%
\pgfpathcurveto{\pgfqpoint{0.753008in}{1.100984in}}{\pgfqpoint{0.748679in}{1.090533in}}{\pgfqpoint{0.748679in}{1.079638in}}%
\pgfpathcurveto{\pgfqpoint{0.748679in}{1.068742in}}{\pgfqpoint{0.753008in}{1.058291in}}{\pgfqpoint{0.760712in}{1.050587in}}%
\pgfpathcurveto{\pgfqpoint{0.768417in}{1.042883in}}{\pgfqpoint{0.778868in}{1.038554in}}{\pgfqpoint{0.789763in}{1.038554in}}%
\pgfpathlineto{\pgfqpoint{0.789763in}{1.038554in}}%
\pgfpathclose%
\pgfusepath{stroke}%
\end{pgfscope}%
\begin{pgfscope}%
\pgfpathrectangle{\pgfqpoint{0.688192in}{0.670138in}}{\pgfqpoint{7.111808in}{5.129862in}}%
\pgfusepath{clip}%
\pgfsetbuttcap%
\pgfsetroundjoin%
\pgfsetlinewidth{1.003750pt}%
\definecolor{currentstroke}{rgb}{0.000000,0.000000,0.000000}%
\pgfsetstrokecolor{currentstroke}%
\pgfsetdash{}{0pt}%
\pgfpathmoveto{\pgfqpoint{4.182175in}{0.641642in}}%
\pgfpathcurveto{\pgfqpoint{4.193071in}{0.641642in}}{\pgfqpoint{4.203522in}{0.645971in}}{\pgfqpoint{4.211226in}{0.653675in}}%
\pgfpathcurveto{\pgfqpoint{4.218930in}{0.661380in}}{\pgfqpoint{4.223259in}{0.671830in}}{\pgfqpoint{4.223259in}{0.682726in}}%
\pgfpathcurveto{\pgfqpoint{4.223259in}{0.693621in}}{\pgfqpoint{4.218930in}{0.704072in}}{\pgfqpoint{4.211226in}{0.711777in}}%
\pgfpathcurveto{\pgfqpoint{4.203522in}{0.719481in}}{\pgfqpoint{4.193071in}{0.723810in}}{\pgfqpoint{4.182175in}{0.723810in}}%
\pgfpathcurveto{\pgfqpoint{4.171280in}{0.723810in}}{\pgfqpoint{4.160829in}{0.719481in}}{\pgfqpoint{4.153125in}{0.711777in}}%
\pgfpathcurveto{\pgfqpoint{4.145420in}{0.704072in}}{\pgfqpoint{4.141091in}{0.693621in}}{\pgfqpoint{4.141091in}{0.682726in}}%
\pgfpathcurveto{\pgfqpoint{4.141091in}{0.671830in}}{\pgfqpoint{4.145420in}{0.661380in}}{\pgfqpoint{4.153125in}{0.653675in}}%
\pgfpathcurveto{\pgfqpoint{4.160829in}{0.645971in}}{\pgfqpoint{4.171280in}{0.641642in}}{\pgfqpoint{4.182175in}{0.641642in}}%
\pgfusepath{stroke}%
\end{pgfscope}%
\begin{pgfscope}%
\pgfpathrectangle{\pgfqpoint{0.688192in}{0.670138in}}{\pgfqpoint{7.111808in}{5.129862in}}%
\pgfusepath{clip}%
\pgfsetbuttcap%
\pgfsetroundjoin%
\pgfsetlinewidth{1.003750pt}%
\definecolor{currentstroke}{rgb}{0.000000,0.000000,0.000000}%
\pgfsetstrokecolor{currentstroke}%
\pgfsetdash{}{0pt}%
\pgfpathmoveto{\pgfqpoint{2.497167in}{0.675229in}}%
\pgfpathcurveto{\pgfqpoint{2.508063in}{0.675229in}}{\pgfqpoint{2.518514in}{0.679557in}}{\pgfqpoint{2.526218in}{0.687262in}}%
\pgfpathcurveto{\pgfqpoint{2.533922in}{0.694966in}}{\pgfqpoint{2.538251in}{0.705417in}}{\pgfqpoint{2.538251in}{0.716312in}}%
\pgfpathcurveto{\pgfqpoint{2.538251in}{0.727208in}}{\pgfqpoint{2.533922in}{0.737659in}}{\pgfqpoint{2.526218in}{0.745363in}}%
\pgfpathcurveto{\pgfqpoint{2.518514in}{0.753067in}}{\pgfqpoint{2.508063in}{0.757396in}}{\pgfqpoint{2.497167in}{0.757396in}}%
\pgfpathcurveto{\pgfqpoint{2.486272in}{0.757396in}}{\pgfqpoint{2.475821in}{0.753067in}}{\pgfqpoint{2.468117in}{0.745363in}}%
\pgfpathcurveto{\pgfqpoint{2.460412in}{0.737659in}}{\pgfqpoint{2.456084in}{0.727208in}}{\pgfqpoint{2.456084in}{0.716312in}}%
\pgfpathcurveto{\pgfqpoint{2.456084in}{0.705417in}}{\pgfqpoint{2.460412in}{0.694966in}}{\pgfqpoint{2.468117in}{0.687262in}}%
\pgfpathcurveto{\pgfqpoint{2.475821in}{0.679557in}}{\pgfqpoint{2.486272in}{0.675229in}}{\pgfqpoint{2.497167in}{0.675229in}}%
\pgfpathlineto{\pgfqpoint{2.497167in}{0.675229in}}%
\pgfpathclose%
\pgfusepath{stroke}%
\end{pgfscope}%
\begin{pgfscope}%
\pgfpathrectangle{\pgfqpoint{0.688192in}{0.670138in}}{\pgfqpoint{7.111808in}{5.129862in}}%
\pgfusepath{clip}%
\pgfsetbuttcap%
\pgfsetroundjoin%
\pgfsetlinewidth{1.003750pt}%
\definecolor{currentstroke}{rgb}{0.000000,0.000000,0.000000}%
\pgfsetstrokecolor{currentstroke}%
\pgfsetdash{}{0pt}%
\pgfpathmoveto{\pgfqpoint{1.920423in}{0.692661in}}%
\pgfpathcurveto{\pgfqpoint{1.931318in}{0.692661in}}{\pgfqpoint{1.941769in}{0.696990in}}{\pgfqpoint{1.949473in}{0.704695in}}%
\pgfpathcurveto{\pgfqpoint{1.957178in}{0.712399in}}{\pgfqpoint{1.961506in}{0.722850in}}{\pgfqpoint{1.961506in}{0.733745in}}%
\pgfpathcurveto{\pgfqpoint{1.961506in}{0.744641in}}{\pgfqpoint{1.957178in}{0.755092in}}{\pgfqpoint{1.949473in}{0.762796in}}%
\pgfpathcurveto{\pgfqpoint{1.941769in}{0.770500in}}{\pgfqpoint{1.931318in}{0.774829in}}{\pgfqpoint{1.920423in}{0.774829in}}%
\pgfpathcurveto{\pgfqpoint{1.909527in}{0.774829in}}{\pgfqpoint{1.899076in}{0.770500in}}{\pgfqpoint{1.891372in}{0.762796in}}%
\pgfpathcurveto{\pgfqpoint{1.883668in}{0.755092in}}{\pgfqpoint{1.879339in}{0.744641in}}{\pgfqpoint{1.879339in}{0.733745in}}%
\pgfpathcurveto{\pgfqpoint{1.879339in}{0.722850in}}{\pgfqpoint{1.883668in}{0.712399in}}{\pgfqpoint{1.891372in}{0.704695in}}%
\pgfpathcurveto{\pgfqpoint{1.899076in}{0.696990in}}{\pgfqpoint{1.909527in}{0.692661in}}{\pgfqpoint{1.920423in}{0.692661in}}%
\pgfpathlineto{\pgfqpoint{1.920423in}{0.692661in}}%
\pgfpathclose%
\pgfusepath{stroke}%
\end{pgfscope}%
\begin{pgfscope}%
\pgfpathrectangle{\pgfqpoint{0.688192in}{0.670138in}}{\pgfqpoint{7.111808in}{5.129862in}}%
\pgfusepath{clip}%
\pgfsetbuttcap%
\pgfsetroundjoin%
\pgfsetlinewidth{1.003750pt}%
\definecolor{currentstroke}{rgb}{0.000000,0.000000,0.000000}%
\pgfsetstrokecolor{currentstroke}%
\pgfsetdash{}{0pt}%
\pgfpathmoveto{\pgfqpoint{1.666551in}{0.703510in}}%
\pgfpathcurveto{\pgfqpoint{1.677446in}{0.703510in}}{\pgfqpoint{1.687897in}{0.707839in}}{\pgfqpoint{1.695601in}{0.715543in}}%
\pgfpathcurveto{\pgfqpoint{1.703306in}{0.723248in}}{\pgfqpoint{1.707635in}{0.733698in}}{\pgfqpoint{1.707635in}{0.744594in}}%
\pgfpathcurveto{\pgfqpoint{1.707635in}{0.755489in}}{\pgfqpoint{1.703306in}{0.765940in}}{\pgfqpoint{1.695601in}{0.773645in}}%
\pgfpathcurveto{\pgfqpoint{1.687897in}{0.781349in}}{\pgfqpoint{1.677446in}{0.785678in}}{\pgfqpoint{1.666551in}{0.785678in}}%
\pgfpathcurveto{\pgfqpoint{1.655655in}{0.785678in}}{\pgfqpoint{1.645204in}{0.781349in}}{\pgfqpoint{1.637500in}{0.773645in}}%
\pgfpathcurveto{\pgfqpoint{1.629796in}{0.765940in}}{\pgfqpoint{1.625467in}{0.755489in}}{\pgfqpoint{1.625467in}{0.744594in}}%
\pgfpathcurveto{\pgfqpoint{1.625467in}{0.733698in}}{\pgfqpoint{1.629796in}{0.723248in}}{\pgfqpoint{1.637500in}{0.715543in}}%
\pgfpathcurveto{\pgfqpoint{1.645204in}{0.707839in}}{\pgfqpoint{1.655655in}{0.703510in}}{\pgfqpoint{1.666551in}{0.703510in}}%
\pgfpathlineto{\pgfqpoint{1.666551in}{0.703510in}}%
\pgfpathclose%
\pgfusepath{stroke}%
\end{pgfscope}%
\begin{pgfscope}%
\pgfpathrectangle{\pgfqpoint{0.688192in}{0.670138in}}{\pgfqpoint{7.111808in}{5.129862in}}%
\pgfusepath{clip}%
\pgfsetbuttcap%
\pgfsetroundjoin%
\pgfsetlinewidth{1.003750pt}%
\definecolor{currentstroke}{rgb}{0.000000,0.000000,0.000000}%
\pgfsetstrokecolor{currentstroke}%
\pgfsetdash{}{0pt}%
\pgfpathmoveto{\pgfqpoint{1.141797in}{0.719357in}}%
\pgfpathcurveto{\pgfqpoint{1.152693in}{0.719357in}}{\pgfqpoint{1.163143in}{0.723686in}}{\pgfqpoint{1.170848in}{0.731391in}}%
\pgfpathcurveto{\pgfqpoint{1.178552in}{0.739095in}}{\pgfqpoint{1.182881in}{0.749546in}}{\pgfqpoint{1.182881in}{0.760441in}}%
\pgfpathcurveto{\pgfqpoint{1.182881in}{0.771337in}}{\pgfqpoint{1.178552in}{0.781788in}}{\pgfqpoint{1.170848in}{0.789492in}}%
\pgfpathcurveto{\pgfqpoint{1.163143in}{0.797196in}}{\pgfqpoint{1.152693in}{0.801525in}}{\pgfqpoint{1.141797in}{0.801525in}}%
\pgfpathcurveto{\pgfqpoint{1.130902in}{0.801525in}}{\pgfqpoint{1.120451in}{0.797196in}}{\pgfqpoint{1.112746in}{0.789492in}}%
\pgfpathcurveto{\pgfqpoint{1.105042in}{0.781788in}}{\pgfqpoint{1.100713in}{0.771337in}}{\pgfqpoint{1.100713in}{0.760441in}}%
\pgfpathcurveto{\pgfqpoint{1.100713in}{0.749546in}}{\pgfqpoint{1.105042in}{0.739095in}}{\pgfqpoint{1.112746in}{0.731391in}}%
\pgfpathcurveto{\pgfqpoint{1.120451in}{0.723686in}}{\pgfqpoint{1.130902in}{0.719357in}}{\pgfqpoint{1.141797in}{0.719357in}}%
\pgfpathlineto{\pgfqpoint{1.141797in}{0.719357in}}%
\pgfpathclose%
\pgfusepath{stroke}%
\end{pgfscope}%
\begin{pgfscope}%
\pgfpathrectangle{\pgfqpoint{0.688192in}{0.670138in}}{\pgfqpoint{7.111808in}{5.129862in}}%
\pgfusepath{clip}%
\pgfsetbuttcap%
\pgfsetroundjoin%
\pgfsetlinewidth{1.003750pt}%
\definecolor{currentstroke}{rgb}{0.000000,0.000000,0.000000}%
\pgfsetstrokecolor{currentstroke}%
\pgfsetdash{}{0pt}%
\pgfpathmoveto{\pgfqpoint{0.715926in}{2.236095in}}%
\pgfpathcurveto{\pgfqpoint{0.726821in}{2.236095in}}{\pgfqpoint{0.737272in}{2.240424in}}{\pgfqpoint{0.744976in}{2.248129in}}%
\pgfpathcurveto{\pgfqpoint{0.752681in}{2.255833in}}{\pgfqpoint{0.757010in}{2.266284in}}{\pgfqpoint{0.757010in}{2.277179in}}%
\pgfpathcurveto{\pgfqpoint{0.757010in}{2.288075in}}{\pgfqpoint{0.752681in}{2.298526in}}{\pgfqpoint{0.744976in}{2.306230in}}%
\pgfpathcurveto{\pgfqpoint{0.737272in}{2.313934in}}{\pgfqpoint{0.726821in}{2.318263in}}{\pgfqpoint{0.715926in}{2.318263in}}%
\pgfpathcurveto{\pgfqpoint{0.705030in}{2.318263in}}{\pgfqpoint{0.694579in}{2.313934in}}{\pgfqpoint{0.686875in}{2.306230in}}%
\pgfpathcurveto{\pgfqpoint{0.679171in}{2.298526in}}{\pgfqpoint{0.674842in}{2.288075in}}{\pgfqpoint{0.674842in}{2.277179in}}%
\pgfpathcurveto{\pgfqpoint{0.674842in}{2.266284in}}{\pgfqpoint{0.679171in}{2.255833in}}{\pgfqpoint{0.686875in}{2.248129in}}%
\pgfpathcurveto{\pgfqpoint{0.694579in}{2.240424in}}{\pgfqpoint{0.705030in}{2.236095in}}{\pgfqpoint{0.715926in}{2.236095in}}%
\pgfpathlineto{\pgfqpoint{0.715926in}{2.236095in}}%
\pgfpathclose%
\pgfusepath{stroke}%
\end{pgfscope}%
\begin{pgfscope}%
\pgfpathrectangle{\pgfqpoint{0.688192in}{0.670138in}}{\pgfqpoint{7.111808in}{5.129862in}}%
\pgfusepath{clip}%
\pgfsetbuttcap%
\pgfsetroundjoin%
\pgfsetlinewidth{1.003750pt}%
\definecolor{currentstroke}{rgb}{0.000000,0.000000,0.000000}%
\pgfsetstrokecolor{currentstroke}%
\pgfsetdash{}{0pt}%
\pgfpathmoveto{\pgfqpoint{1.091394in}{0.722252in}}%
\pgfpathcurveto{\pgfqpoint{1.102290in}{0.722252in}}{\pgfqpoint{1.112741in}{0.726581in}}{\pgfqpoint{1.120445in}{0.734285in}}%
\pgfpathcurveto{\pgfqpoint{1.128149in}{0.741989in}}{\pgfqpoint{1.132478in}{0.752440in}}{\pgfqpoint{1.132478in}{0.763336in}}%
\pgfpathcurveto{\pgfqpoint{1.132478in}{0.774231in}}{\pgfqpoint{1.128149in}{0.784682in}}{\pgfqpoint{1.120445in}{0.792386in}}%
\pgfpathcurveto{\pgfqpoint{1.112741in}{0.800091in}}{\pgfqpoint{1.102290in}{0.804420in}}{\pgfqpoint{1.091394in}{0.804420in}}%
\pgfpathcurveto{\pgfqpoint{1.080499in}{0.804420in}}{\pgfqpoint{1.070048in}{0.800091in}}{\pgfqpoint{1.062344in}{0.792386in}}%
\pgfpathcurveto{\pgfqpoint{1.054639in}{0.784682in}}{\pgfqpoint{1.050310in}{0.774231in}}{\pgfqpoint{1.050310in}{0.763336in}}%
\pgfpathcurveto{\pgfqpoint{1.050310in}{0.752440in}}{\pgfqpoint{1.054639in}{0.741989in}}{\pgfqpoint{1.062344in}{0.734285in}}%
\pgfpathcurveto{\pgfqpoint{1.070048in}{0.726581in}}{\pgfqpoint{1.080499in}{0.722252in}}{\pgfqpoint{1.091394in}{0.722252in}}%
\pgfpathlineto{\pgfqpoint{1.091394in}{0.722252in}}%
\pgfpathclose%
\pgfusepath{stroke}%
\end{pgfscope}%
\begin{pgfscope}%
\pgfpathrectangle{\pgfqpoint{0.688192in}{0.670138in}}{\pgfqpoint{7.111808in}{5.129862in}}%
\pgfusepath{clip}%
\pgfsetbuttcap%
\pgfsetroundjoin%
\pgfsetlinewidth{1.003750pt}%
\definecolor{currentstroke}{rgb}{0.000000,0.000000,0.000000}%
\pgfsetstrokecolor{currentstroke}%
\pgfsetdash{}{0pt}%
\pgfpathmoveto{\pgfqpoint{2.333738in}{3.417663in}}%
\pgfpathcurveto{\pgfqpoint{2.344634in}{3.417663in}}{\pgfqpoint{2.355085in}{3.421992in}}{\pgfqpoint{2.362789in}{3.429696in}}%
\pgfpathcurveto{\pgfqpoint{2.370493in}{3.437401in}}{\pgfqpoint{2.374822in}{3.447852in}}{\pgfqpoint{2.374822in}{3.458747in}}%
\pgfpathcurveto{\pgfqpoint{2.374822in}{3.469643in}}{\pgfqpoint{2.370493in}{3.480093in}}{\pgfqpoint{2.362789in}{3.487798in}}%
\pgfpathcurveto{\pgfqpoint{2.355085in}{3.495502in}}{\pgfqpoint{2.344634in}{3.499831in}}{\pgfqpoint{2.333738in}{3.499831in}}%
\pgfpathcurveto{\pgfqpoint{2.322843in}{3.499831in}}{\pgfqpoint{2.312392in}{3.495502in}}{\pgfqpoint{2.304688in}{3.487798in}}%
\pgfpathcurveto{\pgfqpoint{2.296983in}{3.480093in}}{\pgfqpoint{2.292655in}{3.469643in}}{\pgfqpoint{2.292655in}{3.458747in}}%
\pgfpathcurveto{\pgfqpoint{2.292655in}{3.447852in}}{\pgfqpoint{2.296983in}{3.437401in}}{\pgfqpoint{2.304688in}{3.429696in}}%
\pgfpathcurveto{\pgfqpoint{2.312392in}{3.421992in}}{\pgfqpoint{2.322843in}{3.417663in}}{\pgfqpoint{2.333738in}{3.417663in}}%
\pgfpathlineto{\pgfqpoint{2.333738in}{3.417663in}}%
\pgfpathclose%
\pgfusepath{stroke}%
\end{pgfscope}%
\begin{pgfscope}%
\pgfpathrectangle{\pgfqpoint{0.688192in}{0.670138in}}{\pgfqpoint{7.111808in}{5.129862in}}%
\pgfusepath{clip}%
\pgfsetbuttcap%
\pgfsetroundjoin%
\pgfsetlinewidth{1.003750pt}%
\definecolor{currentstroke}{rgb}{0.000000,0.000000,0.000000}%
\pgfsetstrokecolor{currentstroke}%
\pgfsetdash{}{0pt}%
\pgfpathmoveto{\pgfqpoint{0.928243in}{0.841597in}}%
\pgfpathcurveto{\pgfqpoint{0.939139in}{0.841597in}}{\pgfqpoint{0.949589in}{0.845926in}}{\pgfqpoint{0.957294in}{0.853631in}}%
\pgfpathcurveto{\pgfqpoint{0.964998in}{0.861335in}}{\pgfqpoint{0.969327in}{0.871786in}}{\pgfqpoint{0.969327in}{0.882681in}}%
\pgfpathcurveto{\pgfqpoint{0.969327in}{0.893577in}}{\pgfqpoint{0.964998in}{0.904028in}}{\pgfqpoint{0.957294in}{0.911732in}}%
\pgfpathcurveto{\pgfqpoint{0.949589in}{0.919436in}}{\pgfqpoint{0.939139in}{0.923765in}}{\pgfqpoint{0.928243in}{0.923765in}}%
\pgfpathcurveto{\pgfqpoint{0.917347in}{0.923765in}}{\pgfqpoint{0.906897in}{0.919436in}}{\pgfqpoint{0.899192in}{0.911732in}}%
\pgfpathcurveto{\pgfqpoint{0.891488in}{0.904028in}}{\pgfqpoint{0.887159in}{0.893577in}}{\pgfqpoint{0.887159in}{0.882681in}}%
\pgfpathcurveto{\pgfqpoint{0.887159in}{0.871786in}}{\pgfqpoint{0.891488in}{0.861335in}}{\pgfqpoint{0.899192in}{0.853631in}}%
\pgfpathcurveto{\pgfqpoint{0.906897in}{0.845926in}}{\pgfqpoint{0.917347in}{0.841597in}}{\pgfqpoint{0.928243in}{0.841597in}}%
\pgfpathlineto{\pgfqpoint{0.928243in}{0.841597in}}%
\pgfpathclose%
\pgfusepath{stroke}%
\end{pgfscope}%
\begin{pgfscope}%
\pgfpathrectangle{\pgfqpoint{0.688192in}{0.670138in}}{\pgfqpoint{7.111808in}{5.129862in}}%
\pgfusepath{clip}%
\pgfsetbuttcap%
\pgfsetroundjoin%
\pgfsetlinewidth{1.003750pt}%
\definecolor{currentstroke}{rgb}{0.000000,0.000000,0.000000}%
\pgfsetstrokecolor{currentstroke}%
\pgfsetdash{}{0pt}%
\pgfpathmoveto{\pgfqpoint{1.389197in}{0.711266in}}%
\pgfpathcurveto{\pgfqpoint{1.400093in}{0.711266in}}{\pgfqpoint{1.410544in}{0.715595in}}{\pgfqpoint{1.418248in}{0.723299in}}%
\pgfpathcurveto{\pgfqpoint{1.425952in}{0.731004in}}{\pgfqpoint{1.430281in}{0.741454in}}{\pgfqpoint{1.430281in}{0.752350in}}%
\pgfpathcurveto{\pgfqpoint{1.430281in}{0.763246in}}{\pgfqpoint{1.425952in}{0.773696in}}{\pgfqpoint{1.418248in}{0.781401in}}%
\pgfpathcurveto{\pgfqpoint{1.410544in}{0.789105in}}{\pgfqpoint{1.400093in}{0.793434in}}{\pgfqpoint{1.389197in}{0.793434in}}%
\pgfpathcurveto{\pgfqpoint{1.378302in}{0.793434in}}{\pgfqpoint{1.367851in}{0.789105in}}{\pgfqpoint{1.360147in}{0.781401in}}%
\pgfpathcurveto{\pgfqpoint{1.352442in}{0.773696in}}{\pgfqpoint{1.348113in}{0.763246in}}{\pgfqpoint{1.348113in}{0.752350in}}%
\pgfpathcurveto{\pgfqpoint{1.348113in}{0.741454in}}{\pgfqpoint{1.352442in}{0.731004in}}{\pgfqpoint{1.360147in}{0.723299in}}%
\pgfpathcurveto{\pgfqpoint{1.367851in}{0.715595in}}{\pgfqpoint{1.378302in}{0.711266in}}{\pgfqpoint{1.389197in}{0.711266in}}%
\pgfpathlineto{\pgfqpoint{1.389197in}{0.711266in}}%
\pgfpathclose%
\pgfusepath{stroke}%
\end{pgfscope}%
\begin{pgfscope}%
\pgfpathrectangle{\pgfqpoint{0.688192in}{0.670138in}}{\pgfqpoint{7.111808in}{5.129862in}}%
\pgfusepath{clip}%
\pgfsetbuttcap%
\pgfsetroundjoin%
\pgfsetlinewidth{1.003750pt}%
\definecolor{currentstroke}{rgb}{0.000000,0.000000,0.000000}%
\pgfsetstrokecolor{currentstroke}%
\pgfsetdash{}{0pt}%
\pgfpathmoveto{\pgfqpoint{4.586127in}{0.640844in}}%
\pgfpathcurveto{\pgfqpoint{4.597023in}{0.640844in}}{\pgfqpoint{4.607473in}{0.645173in}}{\pgfqpoint{4.615178in}{0.652877in}}%
\pgfpathcurveto{\pgfqpoint{4.622882in}{0.660582in}}{\pgfqpoint{4.627211in}{0.671032in}}{\pgfqpoint{4.627211in}{0.681928in}}%
\pgfpathcurveto{\pgfqpoint{4.627211in}{0.692824in}}{\pgfqpoint{4.622882in}{0.703274in}}{\pgfqpoint{4.615178in}{0.710979in}}%
\pgfpathcurveto{\pgfqpoint{4.607473in}{0.718683in}}{\pgfqpoint{4.597023in}{0.723012in}}{\pgfqpoint{4.586127in}{0.723012in}}%
\pgfpathcurveto{\pgfqpoint{4.575231in}{0.723012in}}{\pgfqpoint{4.564781in}{0.718683in}}{\pgfqpoint{4.557076in}{0.710979in}}%
\pgfpathcurveto{\pgfqpoint{4.549372in}{0.703274in}}{\pgfqpoint{4.545043in}{0.692824in}}{\pgfqpoint{4.545043in}{0.681928in}}%
\pgfpathcurveto{\pgfqpoint{4.545043in}{0.671032in}}{\pgfqpoint{4.549372in}{0.660582in}}{\pgfqpoint{4.557076in}{0.652877in}}%
\pgfpathcurveto{\pgfqpoint{4.564781in}{0.645173in}}{\pgfqpoint{4.575231in}{0.640844in}}{\pgfqpoint{4.586127in}{0.640844in}}%
\pgfusepath{stroke}%
\end{pgfscope}%
\begin{pgfscope}%
\pgfpathrectangle{\pgfqpoint{0.688192in}{0.670138in}}{\pgfqpoint{7.111808in}{5.129862in}}%
\pgfusepath{clip}%
\pgfsetbuttcap%
\pgfsetroundjoin%
\pgfsetlinewidth{1.003750pt}%
\definecolor{currentstroke}{rgb}{0.000000,0.000000,0.000000}%
\pgfsetstrokecolor{currentstroke}%
\pgfsetdash{}{0pt}%
\pgfpathmoveto{\pgfqpoint{0.901363in}{0.856548in}}%
\pgfpathcurveto{\pgfqpoint{0.912259in}{0.856548in}}{\pgfqpoint{0.922710in}{0.860877in}}{\pgfqpoint{0.930414in}{0.868582in}}%
\pgfpathcurveto{\pgfqpoint{0.938118in}{0.876286in}}{\pgfqpoint{0.942447in}{0.886737in}}{\pgfqpoint{0.942447in}{0.897632in}}%
\pgfpathcurveto{\pgfqpoint{0.942447in}{0.908528in}}{\pgfqpoint{0.938118in}{0.918979in}}{\pgfqpoint{0.930414in}{0.926683in}}%
\pgfpathcurveto{\pgfqpoint{0.922710in}{0.934387in}}{\pgfqpoint{0.912259in}{0.938716in}}{\pgfqpoint{0.901363in}{0.938716in}}%
\pgfpathcurveto{\pgfqpoint{0.890468in}{0.938716in}}{\pgfqpoint{0.880017in}{0.934387in}}{\pgfqpoint{0.872313in}{0.926683in}}%
\pgfpathcurveto{\pgfqpoint{0.864608in}{0.918979in}}{\pgfqpoint{0.860280in}{0.908528in}}{\pgfqpoint{0.860280in}{0.897632in}}%
\pgfpathcurveto{\pgfqpoint{0.860280in}{0.886737in}}{\pgfqpoint{0.864608in}{0.876286in}}{\pgfqpoint{0.872313in}{0.868582in}}%
\pgfpathcurveto{\pgfqpoint{0.880017in}{0.860877in}}{\pgfqpoint{0.890468in}{0.856548in}}{\pgfqpoint{0.901363in}{0.856548in}}%
\pgfpathlineto{\pgfqpoint{0.901363in}{0.856548in}}%
\pgfpathclose%
\pgfusepath{stroke}%
\end{pgfscope}%
\begin{pgfscope}%
\pgfpathrectangle{\pgfqpoint{0.688192in}{0.670138in}}{\pgfqpoint{7.111808in}{5.129862in}}%
\pgfusepath{clip}%
\pgfsetbuttcap%
\pgfsetroundjoin%
\pgfsetlinewidth{1.003750pt}%
\definecolor{currentstroke}{rgb}{0.000000,0.000000,0.000000}%
\pgfsetstrokecolor{currentstroke}%
\pgfsetdash{}{0pt}%
\pgfpathmoveto{\pgfqpoint{1.389197in}{0.711266in}}%
\pgfpathcurveto{\pgfqpoint{1.400093in}{0.711266in}}{\pgfqpoint{1.410544in}{0.715595in}}{\pgfqpoint{1.418248in}{0.723299in}}%
\pgfpathcurveto{\pgfqpoint{1.425952in}{0.731004in}}{\pgfqpoint{1.430281in}{0.741454in}}{\pgfqpoint{1.430281in}{0.752350in}}%
\pgfpathcurveto{\pgfqpoint{1.430281in}{0.763246in}}{\pgfqpoint{1.425952in}{0.773696in}}{\pgfqpoint{1.418248in}{0.781401in}}%
\pgfpathcurveto{\pgfqpoint{1.410544in}{0.789105in}}{\pgfqpoint{1.400093in}{0.793434in}}{\pgfqpoint{1.389197in}{0.793434in}}%
\pgfpathcurveto{\pgfqpoint{1.378302in}{0.793434in}}{\pgfqpoint{1.367851in}{0.789105in}}{\pgfqpoint{1.360147in}{0.781401in}}%
\pgfpathcurveto{\pgfqpoint{1.352442in}{0.773696in}}{\pgfqpoint{1.348113in}{0.763246in}}{\pgfqpoint{1.348113in}{0.752350in}}%
\pgfpathcurveto{\pgfqpoint{1.348113in}{0.741454in}}{\pgfqpoint{1.352442in}{0.731004in}}{\pgfqpoint{1.360147in}{0.723299in}}%
\pgfpathcurveto{\pgfqpoint{1.367851in}{0.715595in}}{\pgfqpoint{1.378302in}{0.711266in}}{\pgfqpoint{1.389197in}{0.711266in}}%
\pgfpathlineto{\pgfqpoint{1.389197in}{0.711266in}}%
\pgfpathclose%
\pgfusepath{stroke}%
\end{pgfscope}%
\begin{pgfscope}%
\pgfpathrectangle{\pgfqpoint{0.688192in}{0.670138in}}{\pgfqpoint{7.111808in}{5.129862in}}%
\pgfusepath{clip}%
\pgfsetbuttcap%
\pgfsetroundjoin%
\pgfsetlinewidth{1.003750pt}%
\definecolor{currentstroke}{rgb}{0.000000,0.000000,0.000000}%
\pgfsetstrokecolor{currentstroke}%
\pgfsetdash{}{0pt}%
\pgfpathmoveto{\pgfqpoint{2.628270in}{0.672002in}}%
\pgfpathcurveto{\pgfqpoint{2.639166in}{0.672002in}}{\pgfqpoint{2.649617in}{0.676331in}}{\pgfqpoint{2.657321in}{0.684035in}}%
\pgfpathcurveto{\pgfqpoint{2.665025in}{0.691739in}}{\pgfqpoint{2.669354in}{0.702190in}}{\pgfqpoint{2.669354in}{0.713086in}}%
\pgfpathcurveto{\pgfqpoint{2.669354in}{0.723981in}}{\pgfqpoint{2.665025in}{0.734432in}}{\pgfqpoint{2.657321in}{0.742137in}}%
\pgfpathcurveto{\pgfqpoint{2.649617in}{0.749841in}}{\pgfqpoint{2.639166in}{0.754170in}}{\pgfqpoint{2.628270in}{0.754170in}}%
\pgfpathcurveto{\pgfqpoint{2.617375in}{0.754170in}}{\pgfqpoint{2.606924in}{0.749841in}}{\pgfqpoint{2.599220in}{0.742137in}}%
\pgfpathcurveto{\pgfqpoint{2.591515in}{0.734432in}}{\pgfqpoint{2.587186in}{0.723981in}}{\pgfqpoint{2.587186in}{0.713086in}}%
\pgfpathcurveto{\pgfqpoint{2.587186in}{0.702190in}}{\pgfqpoint{2.591515in}{0.691739in}}{\pgfqpoint{2.599220in}{0.684035in}}%
\pgfpathcurveto{\pgfqpoint{2.606924in}{0.676331in}}{\pgfqpoint{2.617375in}{0.672002in}}{\pgfqpoint{2.628270in}{0.672002in}}%
\pgfpathlineto{\pgfqpoint{2.628270in}{0.672002in}}%
\pgfpathclose%
\pgfusepath{stroke}%
\end{pgfscope}%
\begin{pgfscope}%
\pgfpathrectangle{\pgfqpoint{0.688192in}{0.670138in}}{\pgfqpoint{7.111808in}{5.129862in}}%
\pgfusepath{clip}%
\pgfsetbuttcap%
\pgfsetroundjoin%
\pgfsetlinewidth{1.003750pt}%
\definecolor{currentstroke}{rgb}{0.000000,0.000000,0.000000}%
\pgfsetstrokecolor{currentstroke}%
\pgfsetdash{}{0pt}%
\pgfpathmoveto{\pgfqpoint{2.758352in}{1.055333in}}%
\pgfpathcurveto{\pgfqpoint{2.769248in}{1.055333in}}{\pgfqpoint{2.779699in}{1.059662in}}{\pgfqpoint{2.787403in}{1.067366in}}%
\pgfpathcurveto{\pgfqpoint{2.795107in}{1.075070in}}{\pgfqpoint{2.799436in}{1.085521in}}{\pgfqpoint{2.799436in}{1.096417in}}%
\pgfpathcurveto{\pgfqpoint{2.799436in}{1.107312in}}{\pgfqpoint{2.795107in}{1.117763in}}{\pgfqpoint{2.787403in}{1.125467in}}%
\pgfpathcurveto{\pgfqpoint{2.779699in}{1.133172in}}{\pgfqpoint{2.769248in}{1.137501in}}{\pgfqpoint{2.758352in}{1.137501in}}%
\pgfpathcurveto{\pgfqpoint{2.747457in}{1.137501in}}{\pgfqpoint{2.737006in}{1.133172in}}{\pgfqpoint{2.729301in}{1.125467in}}%
\pgfpathcurveto{\pgfqpoint{2.721597in}{1.117763in}}{\pgfqpoint{2.717268in}{1.107312in}}{\pgfqpoint{2.717268in}{1.096417in}}%
\pgfpathcurveto{\pgfqpoint{2.717268in}{1.085521in}}{\pgfqpoint{2.721597in}{1.075070in}}{\pgfqpoint{2.729301in}{1.067366in}}%
\pgfpathcurveto{\pgfqpoint{2.737006in}{1.059662in}}{\pgfqpoint{2.747457in}{1.055333in}}{\pgfqpoint{2.758352in}{1.055333in}}%
\pgfpathlineto{\pgfqpoint{2.758352in}{1.055333in}}%
\pgfpathclose%
\pgfusepath{stroke}%
\end{pgfscope}%
\begin{pgfscope}%
\pgfpathrectangle{\pgfqpoint{0.688192in}{0.670138in}}{\pgfqpoint{7.111808in}{5.129862in}}%
\pgfusepath{clip}%
\pgfsetbuttcap%
\pgfsetroundjoin%
\pgfsetlinewidth{1.003750pt}%
\definecolor{currentstroke}{rgb}{0.000000,0.000000,0.000000}%
\pgfsetstrokecolor{currentstroke}%
\pgfsetdash{}{0pt}%
\pgfpathmoveto{\pgfqpoint{0.937645in}{0.823724in}}%
\pgfpathcurveto{\pgfqpoint{0.948540in}{0.823724in}}{\pgfqpoint{0.958991in}{0.828053in}}{\pgfqpoint{0.966695in}{0.835757in}}%
\pgfpathcurveto{\pgfqpoint{0.974400in}{0.843462in}}{\pgfqpoint{0.978729in}{0.853912in}}{\pgfqpoint{0.978729in}{0.864808in}}%
\pgfpathcurveto{\pgfqpoint{0.978729in}{0.875703in}}{\pgfqpoint{0.974400in}{0.886154in}}{\pgfqpoint{0.966695in}{0.893859in}}%
\pgfpathcurveto{\pgfqpoint{0.958991in}{0.901563in}}{\pgfqpoint{0.948540in}{0.905892in}}{\pgfqpoint{0.937645in}{0.905892in}}%
\pgfpathcurveto{\pgfqpoint{0.926749in}{0.905892in}}{\pgfqpoint{0.916298in}{0.901563in}}{\pgfqpoint{0.908594in}{0.893859in}}%
\pgfpathcurveto{\pgfqpoint{0.900890in}{0.886154in}}{\pgfqpoint{0.896561in}{0.875703in}}{\pgfqpoint{0.896561in}{0.864808in}}%
\pgfpathcurveto{\pgfqpoint{0.896561in}{0.853912in}}{\pgfqpoint{0.900890in}{0.843462in}}{\pgfqpoint{0.908594in}{0.835757in}}%
\pgfpathcurveto{\pgfqpoint{0.916298in}{0.828053in}}{\pgfqpoint{0.926749in}{0.823724in}}{\pgfqpoint{0.937645in}{0.823724in}}%
\pgfpathlineto{\pgfqpoint{0.937645in}{0.823724in}}%
\pgfpathclose%
\pgfusepath{stroke}%
\end{pgfscope}%
\begin{pgfscope}%
\pgfpathrectangle{\pgfqpoint{0.688192in}{0.670138in}}{\pgfqpoint{7.111808in}{5.129862in}}%
\pgfusepath{clip}%
\pgfsetbuttcap%
\pgfsetroundjoin%
\pgfsetlinewidth{1.003750pt}%
\definecolor{currentstroke}{rgb}{0.000000,0.000000,0.000000}%
\pgfsetstrokecolor{currentstroke}%
\pgfsetdash{}{0pt}%
\pgfpathmoveto{\pgfqpoint{1.707667in}{0.699316in}}%
\pgfpathcurveto{\pgfqpoint{1.718562in}{0.699316in}}{\pgfqpoint{1.729013in}{0.703645in}}{\pgfqpoint{1.736717in}{0.711350in}}%
\pgfpathcurveto{\pgfqpoint{1.744422in}{0.719054in}}{\pgfqpoint{1.748751in}{0.729505in}}{\pgfqpoint{1.748751in}{0.740400in}}%
\pgfpathcurveto{\pgfqpoint{1.748751in}{0.751296in}}{\pgfqpoint{1.744422in}{0.761747in}}{\pgfqpoint{1.736717in}{0.769451in}}%
\pgfpathcurveto{\pgfqpoint{1.729013in}{0.777155in}}{\pgfqpoint{1.718562in}{0.781484in}}{\pgfqpoint{1.707667in}{0.781484in}}%
\pgfpathcurveto{\pgfqpoint{1.696771in}{0.781484in}}{\pgfqpoint{1.686320in}{0.777155in}}{\pgfqpoint{1.678616in}{0.769451in}}%
\pgfpathcurveto{\pgfqpoint{1.670912in}{0.761747in}}{\pgfqpoint{1.666583in}{0.751296in}}{\pgfqpoint{1.666583in}{0.740400in}}%
\pgfpathcurveto{\pgfqpoint{1.666583in}{0.729505in}}{\pgfqpoint{1.670912in}{0.719054in}}{\pgfqpoint{1.678616in}{0.711350in}}%
\pgfpathcurveto{\pgfqpoint{1.686320in}{0.703645in}}{\pgfqpoint{1.696771in}{0.699316in}}{\pgfqpoint{1.707667in}{0.699316in}}%
\pgfpathlineto{\pgfqpoint{1.707667in}{0.699316in}}%
\pgfpathclose%
\pgfusepath{stroke}%
\end{pgfscope}%
\begin{pgfscope}%
\pgfpathrectangle{\pgfqpoint{0.688192in}{0.670138in}}{\pgfqpoint{7.111808in}{5.129862in}}%
\pgfusepath{clip}%
\pgfsetbuttcap%
\pgfsetroundjoin%
\pgfsetlinewidth{1.003750pt}%
\definecolor{currentstroke}{rgb}{0.000000,0.000000,0.000000}%
\pgfsetstrokecolor{currentstroke}%
\pgfsetdash{}{0pt}%
\pgfpathmoveto{\pgfqpoint{1.389197in}{0.711266in}}%
\pgfpathcurveto{\pgfqpoint{1.400093in}{0.711266in}}{\pgfqpoint{1.410544in}{0.715595in}}{\pgfqpoint{1.418248in}{0.723299in}}%
\pgfpathcurveto{\pgfqpoint{1.425952in}{0.731004in}}{\pgfqpoint{1.430281in}{0.741454in}}{\pgfqpoint{1.430281in}{0.752350in}}%
\pgfpathcurveto{\pgfqpoint{1.430281in}{0.763246in}}{\pgfqpoint{1.425952in}{0.773696in}}{\pgfqpoint{1.418248in}{0.781401in}}%
\pgfpathcurveto{\pgfqpoint{1.410544in}{0.789105in}}{\pgfqpoint{1.400093in}{0.793434in}}{\pgfqpoint{1.389197in}{0.793434in}}%
\pgfpathcurveto{\pgfqpoint{1.378302in}{0.793434in}}{\pgfqpoint{1.367851in}{0.789105in}}{\pgfqpoint{1.360147in}{0.781401in}}%
\pgfpathcurveto{\pgfqpoint{1.352442in}{0.773696in}}{\pgfqpoint{1.348113in}{0.763246in}}{\pgfqpoint{1.348113in}{0.752350in}}%
\pgfpathcurveto{\pgfqpoint{1.348113in}{0.741454in}}{\pgfqpoint{1.352442in}{0.731004in}}{\pgfqpoint{1.360147in}{0.723299in}}%
\pgfpathcurveto{\pgfqpoint{1.367851in}{0.715595in}}{\pgfqpoint{1.378302in}{0.711266in}}{\pgfqpoint{1.389197in}{0.711266in}}%
\pgfpathlineto{\pgfqpoint{1.389197in}{0.711266in}}%
\pgfpathclose%
\pgfusepath{stroke}%
\end{pgfscope}%
\begin{pgfscope}%
\pgfpathrectangle{\pgfqpoint{0.688192in}{0.670138in}}{\pgfqpoint{7.111808in}{5.129862in}}%
\pgfusepath{clip}%
\pgfsetbuttcap%
\pgfsetroundjoin%
\pgfsetlinewidth{1.003750pt}%
\definecolor{currentstroke}{rgb}{0.000000,0.000000,0.000000}%
\pgfsetstrokecolor{currentstroke}%
\pgfsetdash{}{0pt}%
\pgfpathmoveto{\pgfqpoint{1.106601in}{0.720443in}}%
\pgfpathcurveto{\pgfqpoint{1.117497in}{0.720443in}}{\pgfqpoint{1.127948in}{0.724772in}}{\pgfqpoint{1.135652in}{0.732476in}}%
\pgfpathcurveto{\pgfqpoint{1.143357in}{0.740180in}}{\pgfqpoint{1.147685in}{0.750631in}}{\pgfqpoint{1.147685in}{0.761527in}}%
\pgfpathcurveto{\pgfqpoint{1.147685in}{0.772422in}}{\pgfqpoint{1.143357in}{0.782873in}}{\pgfqpoint{1.135652in}{0.790577in}}%
\pgfpathcurveto{\pgfqpoint{1.127948in}{0.798282in}}{\pgfqpoint{1.117497in}{0.802611in}}{\pgfqpoint{1.106601in}{0.802611in}}%
\pgfpathcurveto{\pgfqpoint{1.095706in}{0.802611in}}{\pgfqpoint{1.085255in}{0.798282in}}{\pgfqpoint{1.077551in}{0.790577in}}%
\pgfpathcurveto{\pgfqpoint{1.069846in}{0.782873in}}{\pgfqpoint{1.065518in}{0.772422in}}{\pgfqpoint{1.065518in}{0.761527in}}%
\pgfpathcurveto{\pgfqpoint{1.065518in}{0.750631in}}{\pgfqpoint{1.069846in}{0.740180in}}{\pgfqpoint{1.077551in}{0.732476in}}%
\pgfpathcurveto{\pgfqpoint{1.085255in}{0.724772in}}{\pgfqpoint{1.095706in}{0.720443in}}{\pgfqpoint{1.106601in}{0.720443in}}%
\pgfpathlineto{\pgfqpoint{1.106601in}{0.720443in}}%
\pgfpathclose%
\pgfusepath{stroke}%
\end{pgfscope}%
\begin{pgfscope}%
\pgfpathrectangle{\pgfqpoint{0.688192in}{0.670138in}}{\pgfqpoint{7.111808in}{5.129862in}}%
\pgfusepath{clip}%
\pgfsetbuttcap%
\pgfsetroundjoin%
\pgfsetlinewidth{1.003750pt}%
\definecolor{currentstroke}{rgb}{0.000000,0.000000,0.000000}%
\pgfsetstrokecolor{currentstroke}%
\pgfsetdash{}{0pt}%
\pgfpathmoveto{\pgfqpoint{1.419139in}{0.710712in}}%
\pgfpathcurveto{\pgfqpoint{1.430035in}{0.710712in}}{\pgfqpoint{1.440485in}{0.715041in}}{\pgfqpoint{1.448190in}{0.722745in}}%
\pgfpathcurveto{\pgfqpoint{1.455894in}{0.730450in}}{\pgfqpoint{1.460223in}{0.740900in}}{\pgfqpoint{1.460223in}{0.751796in}}%
\pgfpathcurveto{\pgfqpoint{1.460223in}{0.762691in}}{\pgfqpoint{1.455894in}{0.773142in}}{\pgfqpoint{1.448190in}{0.780847in}}%
\pgfpathcurveto{\pgfqpoint{1.440485in}{0.788551in}}{\pgfqpoint{1.430035in}{0.792880in}}{\pgfqpoint{1.419139in}{0.792880in}}%
\pgfpathcurveto{\pgfqpoint{1.408244in}{0.792880in}}{\pgfqpoint{1.397793in}{0.788551in}}{\pgfqpoint{1.390088in}{0.780847in}}%
\pgfpathcurveto{\pgfqpoint{1.382384in}{0.773142in}}{\pgfqpoint{1.378055in}{0.762691in}}{\pgfqpoint{1.378055in}{0.751796in}}%
\pgfpathcurveto{\pgfqpoint{1.378055in}{0.740900in}}{\pgfqpoint{1.382384in}{0.730450in}}{\pgfqpoint{1.390088in}{0.722745in}}%
\pgfpathcurveto{\pgfqpoint{1.397793in}{0.715041in}}{\pgfqpoint{1.408244in}{0.710712in}}{\pgfqpoint{1.419139in}{0.710712in}}%
\pgfpathlineto{\pgfqpoint{1.419139in}{0.710712in}}%
\pgfpathclose%
\pgfusepath{stroke}%
\end{pgfscope}%
\begin{pgfscope}%
\pgfpathrectangle{\pgfqpoint{0.688192in}{0.670138in}}{\pgfqpoint{7.111808in}{5.129862in}}%
\pgfusepath{clip}%
\pgfsetbuttcap%
\pgfsetroundjoin%
\pgfsetlinewidth{1.003750pt}%
\definecolor{currentstroke}{rgb}{0.000000,0.000000,0.000000}%
\pgfsetstrokecolor{currentstroke}%
\pgfsetdash{}{0pt}%
\pgfpathmoveto{\pgfqpoint{0.856742in}{0.923169in}}%
\pgfpathcurveto{\pgfqpoint{0.867638in}{0.923169in}}{\pgfqpoint{0.878089in}{0.927498in}}{\pgfqpoint{0.885793in}{0.935202in}}%
\pgfpathcurveto{\pgfqpoint{0.893497in}{0.942907in}}{\pgfqpoint{0.897826in}{0.953357in}}{\pgfqpoint{0.897826in}{0.964253in}}%
\pgfpathcurveto{\pgfqpoint{0.897826in}{0.975149in}}{\pgfqpoint{0.893497in}{0.985599in}}{\pgfqpoint{0.885793in}{0.993304in}}%
\pgfpathcurveto{\pgfqpoint{0.878089in}{1.001008in}}{\pgfqpoint{0.867638in}{1.005337in}}{\pgfqpoint{0.856742in}{1.005337in}}%
\pgfpathcurveto{\pgfqpoint{0.845847in}{1.005337in}}{\pgfqpoint{0.835396in}{1.001008in}}{\pgfqpoint{0.827691in}{0.993304in}}%
\pgfpathcurveto{\pgfqpoint{0.819987in}{0.985599in}}{\pgfqpoint{0.815658in}{0.975149in}}{\pgfqpoint{0.815658in}{0.964253in}}%
\pgfpathcurveto{\pgfqpoint{0.815658in}{0.953357in}}{\pgfqpoint{0.819987in}{0.942907in}}{\pgfqpoint{0.827691in}{0.935202in}}%
\pgfpathcurveto{\pgfqpoint{0.835396in}{0.927498in}}{\pgfqpoint{0.845847in}{0.923169in}}{\pgfqpoint{0.856742in}{0.923169in}}%
\pgfpathlineto{\pgfqpoint{0.856742in}{0.923169in}}%
\pgfpathclose%
\pgfusepath{stroke}%
\end{pgfscope}%
\begin{pgfscope}%
\pgfpathrectangle{\pgfqpoint{0.688192in}{0.670138in}}{\pgfqpoint{7.111808in}{5.129862in}}%
\pgfusepath{clip}%
\pgfsetbuttcap%
\pgfsetroundjoin%
\pgfsetlinewidth{1.003750pt}%
\definecolor{currentstroke}{rgb}{0.000000,0.000000,0.000000}%
\pgfsetstrokecolor{currentstroke}%
\pgfsetdash{}{0pt}%
\pgfpathmoveto{\pgfqpoint{1.666551in}{0.703510in}}%
\pgfpathcurveto{\pgfqpoint{1.677446in}{0.703510in}}{\pgfqpoint{1.687897in}{0.707839in}}{\pgfqpoint{1.695601in}{0.715543in}}%
\pgfpathcurveto{\pgfqpoint{1.703306in}{0.723248in}}{\pgfqpoint{1.707635in}{0.733698in}}{\pgfqpoint{1.707635in}{0.744594in}}%
\pgfpathcurveto{\pgfqpoint{1.707635in}{0.755489in}}{\pgfqpoint{1.703306in}{0.765940in}}{\pgfqpoint{1.695601in}{0.773645in}}%
\pgfpathcurveto{\pgfqpoint{1.687897in}{0.781349in}}{\pgfqpoint{1.677446in}{0.785678in}}{\pgfqpoint{1.666551in}{0.785678in}}%
\pgfpathcurveto{\pgfqpoint{1.655655in}{0.785678in}}{\pgfqpoint{1.645204in}{0.781349in}}{\pgfqpoint{1.637500in}{0.773645in}}%
\pgfpathcurveto{\pgfqpoint{1.629796in}{0.765940in}}{\pgfqpoint{1.625467in}{0.755489in}}{\pgfqpoint{1.625467in}{0.744594in}}%
\pgfpathcurveto{\pgfqpoint{1.625467in}{0.733698in}}{\pgfqpoint{1.629796in}{0.723248in}}{\pgfqpoint{1.637500in}{0.715543in}}%
\pgfpathcurveto{\pgfqpoint{1.645204in}{0.707839in}}{\pgfqpoint{1.655655in}{0.703510in}}{\pgfqpoint{1.666551in}{0.703510in}}%
\pgfpathlineto{\pgfqpoint{1.666551in}{0.703510in}}%
\pgfpathclose%
\pgfusepath{stroke}%
\end{pgfscope}%
\begin{pgfscope}%
\pgfpathrectangle{\pgfqpoint{0.688192in}{0.670138in}}{\pgfqpoint{7.111808in}{5.129862in}}%
\pgfusepath{clip}%
\pgfsetbuttcap%
\pgfsetroundjoin%
\pgfsetlinewidth{1.003750pt}%
\definecolor{currentstroke}{rgb}{0.000000,0.000000,0.000000}%
\pgfsetstrokecolor{currentstroke}%
\pgfsetdash{}{0pt}%
\pgfpathmoveto{\pgfqpoint{3.789959in}{1.083285in}}%
\pgfpathcurveto{\pgfqpoint{3.800855in}{1.083285in}}{\pgfqpoint{3.811306in}{1.087614in}}{\pgfqpoint{3.819010in}{1.095318in}}%
\pgfpathcurveto{\pgfqpoint{3.826714in}{1.103022in}}{\pgfqpoint{3.831043in}{1.113473in}}{\pgfqpoint{3.831043in}{1.124369in}}%
\pgfpathcurveto{\pgfqpoint{3.831043in}{1.135264in}}{\pgfqpoint{3.826714in}{1.145715in}}{\pgfqpoint{3.819010in}{1.153419in}}%
\pgfpathcurveto{\pgfqpoint{3.811306in}{1.161124in}}{\pgfqpoint{3.800855in}{1.165453in}}{\pgfqpoint{3.789959in}{1.165453in}}%
\pgfpathcurveto{\pgfqpoint{3.779064in}{1.165453in}}{\pgfqpoint{3.768613in}{1.161124in}}{\pgfqpoint{3.760909in}{1.153419in}}%
\pgfpathcurveto{\pgfqpoint{3.753204in}{1.145715in}}{\pgfqpoint{3.748876in}{1.135264in}}{\pgfqpoint{3.748876in}{1.124369in}}%
\pgfpathcurveto{\pgfqpoint{3.748876in}{1.113473in}}{\pgfqpoint{3.753204in}{1.103022in}}{\pgfqpoint{3.760909in}{1.095318in}}%
\pgfpathcurveto{\pgfqpoint{3.768613in}{1.087614in}}{\pgfqpoint{3.779064in}{1.083285in}}{\pgfqpoint{3.789959in}{1.083285in}}%
\pgfpathlineto{\pgfqpoint{3.789959in}{1.083285in}}%
\pgfpathclose%
\pgfusepath{stroke}%
\end{pgfscope}%
\begin{pgfscope}%
\pgfpathrectangle{\pgfqpoint{0.688192in}{0.670138in}}{\pgfqpoint{7.111808in}{5.129862in}}%
\pgfusepath{clip}%
\pgfsetbuttcap%
\pgfsetroundjoin%
\pgfsetlinewidth{1.003750pt}%
\definecolor{currentstroke}{rgb}{0.000000,0.000000,0.000000}%
\pgfsetstrokecolor{currentstroke}%
\pgfsetdash{}{0pt}%
\pgfpathmoveto{\pgfqpoint{0.895312in}{0.866174in}}%
\pgfpathcurveto{\pgfqpoint{0.906208in}{0.866174in}}{\pgfqpoint{0.916658in}{0.870503in}}{\pgfqpoint{0.924363in}{0.878207in}}%
\pgfpathcurveto{\pgfqpoint{0.932067in}{0.885911in}}{\pgfqpoint{0.936396in}{0.896362in}}{\pgfqpoint{0.936396in}{0.907258in}}%
\pgfpathcurveto{\pgfqpoint{0.936396in}{0.918153in}}{\pgfqpoint{0.932067in}{0.928604in}}{\pgfqpoint{0.924363in}{0.936308in}}%
\pgfpathcurveto{\pgfqpoint{0.916658in}{0.944013in}}{\pgfqpoint{0.906208in}{0.948341in}}{\pgfqpoint{0.895312in}{0.948341in}}%
\pgfpathcurveto{\pgfqpoint{0.884416in}{0.948341in}}{\pgfqpoint{0.873966in}{0.944013in}}{\pgfqpoint{0.866261in}{0.936308in}}%
\pgfpathcurveto{\pgfqpoint{0.858557in}{0.928604in}}{\pgfqpoint{0.854228in}{0.918153in}}{\pgfqpoint{0.854228in}{0.907258in}}%
\pgfpathcurveto{\pgfqpoint{0.854228in}{0.896362in}}{\pgfqpoint{0.858557in}{0.885911in}}{\pgfqpoint{0.866261in}{0.878207in}}%
\pgfpathcurveto{\pgfqpoint{0.873966in}{0.870503in}}{\pgfqpoint{0.884416in}{0.866174in}}{\pgfqpoint{0.895312in}{0.866174in}}%
\pgfpathlineto{\pgfqpoint{0.895312in}{0.866174in}}%
\pgfpathclose%
\pgfusepath{stroke}%
\end{pgfscope}%
\begin{pgfscope}%
\pgfpathrectangle{\pgfqpoint{0.688192in}{0.670138in}}{\pgfqpoint{7.111808in}{5.129862in}}%
\pgfusepath{clip}%
\pgfsetbuttcap%
\pgfsetroundjoin%
\pgfsetlinewidth{1.003750pt}%
\definecolor{currentstroke}{rgb}{0.000000,0.000000,0.000000}%
\pgfsetstrokecolor{currentstroke}%
\pgfsetdash{}{0pt}%
\pgfpathmoveto{\pgfqpoint{2.217170in}{0.684781in}}%
\pgfpathcurveto{\pgfqpoint{2.228066in}{0.684781in}}{\pgfqpoint{2.238517in}{0.689109in}}{\pgfqpoint{2.246221in}{0.696814in}}%
\pgfpathcurveto{\pgfqpoint{2.253925in}{0.704518in}}{\pgfqpoint{2.258254in}{0.714969in}}{\pgfqpoint{2.258254in}{0.725864in}}%
\pgfpathcurveto{\pgfqpoint{2.258254in}{0.736760in}}{\pgfqpoint{2.253925in}{0.747211in}}{\pgfqpoint{2.246221in}{0.754915in}}%
\pgfpathcurveto{\pgfqpoint{2.238517in}{0.762620in}}{\pgfqpoint{2.228066in}{0.766948in}}{\pgfqpoint{2.217170in}{0.766948in}}%
\pgfpathcurveto{\pgfqpoint{2.206275in}{0.766948in}}{\pgfqpoint{2.195824in}{0.762620in}}{\pgfqpoint{2.188120in}{0.754915in}}%
\pgfpathcurveto{\pgfqpoint{2.180415in}{0.747211in}}{\pgfqpoint{2.176087in}{0.736760in}}{\pgfqpoint{2.176087in}{0.725864in}}%
\pgfpathcurveto{\pgfqpoint{2.176087in}{0.714969in}}{\pgfqpoint{2.180415in}{0.704518in}}{\pgfqpoint{2.188120in}{0.696814in}}%
\pgfpathcurveto{\pgfqpoint{2.195824in}{0.689109in}}{\pgfqpoint{2.206275in}{0.684781in}}{\pgfqpoint{2.217170in}{0.684781in}}%
\pgfpathlineto{\pgfqpoint{2.217170in}{0.684781in}}%
\pgfpathclose%
\pgfusepath{stroke}%
\end{pgfscope}%
\begin{pgfscope}%
\pgfpathrectangle{\pgfqpoint{0.688192in}{0.670138in}}{\pgfqpoint{7.111808in}{5.129862in}}%
\pgfusepath{clip}%
\pgfsetbuttcap%
\pgfsetroundjoin%
\pgfsetlinewidth{1.003750pt}%
\definecolor{currentstroke}{rgb}{0.000000,0.000000,0.000000}%
\pgfsetstrokecolor{currentstroke}%
\pgfsetdash{}{0pt}%
\pgfpathmoveto{\pgfqpoint{1.070830in}{0.725886in}}%
\pgfpathcurveto{\pgfqpoint{1.081726in}{0.725886in}}{\pgfqpoint{1.092176in}{0.730215in}}{\pgfqpoint{1.099881in}{0.737920in}}%
\pgfpathcurveto{\pgfqpoint{1.107585in}{0.745624in}}{\pgfqpoint{1.111914in}{0.756075in}}{\pgfqpoint{1.111914in}{0.766970in}}%
\pgfpathcurveto{\pgfqpoint{1.111914in}{0.777866in}}{\pgfqpoint{1.107585in}{0.788317in}}{\pgfqpoint{1.099881in}{0.796021in}}%
\pgfpathcurveto{\pgfqpoint{1.092176in}{0.803725in}}{\pgfqpoint{1.081726in}{0.808054in}}{\pgfqpoint{1.070830in}{0.808054in}}%
\pgfpathcurveto{\pgfqpoint{1.059934in}{0.808054in}}{\pgfqpoint{1.049484in}{0.803725in}}{\pgfqpoint{1.041779in}{0.796021in}}%
\pgfpathcurveto{\pgfqpoint{1.034075in}{0.788317in}}{\pgfqpoint{1.029746in}{0.777866in}}{\pgfqpoint{1.029746in}{0.766970in}}%
\pgfpathcurveto{\pgfqpoint{1.029746in}{0.756075in}}{\pgfqpoint{1.034075in}{0.745624in}}{\pgfqpoint{1.041779in}{0.737920in}}%
\pgfpathcurveto{\pgfqpoint{1.049484in}{0.730215in}}{\pgfqpoint{1.059934in}{0.725886in}}{\pgfqpoint{1.070830in}{0.725886in}}%
\pgfpathlineto{\pgfqpoint{1.070830in}{0.725886in}}%
\pgfpathclose%
\pgfusepath{stroke}%
\end{pgfscope}%
\begin{pgfscope}%
\pgfpathrectangle{\pgfqpoint{0.688192in}{0.670138in}}{\pgfqpoint{7.111808in}{5.129862in}}%
\pgfusepath{clip}%
\pgfsetbuttcap%
\pgfsetroundjoin%
\pgfsetlinewidth{1.003750pt}%
\definecolor{currentstroke}{rgb}{0.000000,0.000000,0.000000}%
\pgfsetstrokecolor{currentstroke}%
\pgfsetdash{}{0pt}%
\pgfpathmoveto{\pgfqpoint{0.688192in}{5.214003in}}%
\pgfpathcurveto{\pgfqpoint{0.699087in}{5.214003in}}{\pgfqpoint{0.709538in}{5.218332in}}{\pgfqpoint{0.717242in}{5.226037in}}%
\pgfpathcurveto{\pgfqpoint{0.724947in}{5.233741in}}{\pgfqpoint{0.729275in}{5.244192in}}{\pgfqpoint{0.729275in}{5.255087in}}%
\pgfpathcurveto{\pgfqpoint{0.729275in}{5.265983in}}{\pgfqpoint{0.724947in}{5.276434in}}{\pgfqpoint{0.717242in}{5.284138in}}%
\pgfpathcurveto{\pgfqpoint{0.709538in}{5.291842in}}{\pgfqpoint{0.699087in}{5.296171in}}{\pgfqpoint{0.688192in}{5.296171in}}%
\pgfpathcurveto{\pgfqpoint{0.677296in}{5.296171in}}{\pgfqpoint{0.666845in}{5.291842in}}{\pgfqpoint{0.659141in}{5.284138in}}%
\pgfpathcurveto{\pgfqpoint{0.651436in}{5.276434in}}{\pgfqpoint{0.647108in}{5.265983in}}{\pgfqpoint{0.647108in}{5.255087in}}%
\pgfpathcurveto{\pgfqpoint{0.647108in}{5.244192in}}{\pgfqpoint{0.651436in}{5.233741in}}{\pgfqpoint{0.659141in}{5.226037in}}%
\pgfpathcurveto{\pgfqpoint{0.666845in}{5.218332in}}{\pgfqpoint{0.677296in}{5.214003in}}{\pgfqpoint{0.688192in}{5.214003in}}%
\pgfpathlineto{\pgfqpoint{0.688192in}{5.214003in}}%
\pgfpathclose%
\pgfusepath{stroke}%
\end{pgfscope}%
\begin{pgfscope}%
\pgfpathrectangle{\pgfqpoint{0.688192in}{0.670138in}}{\pgfqpoint{7.111808in}{5.129862in}}%
\pgfusepath{clip}%
\pgfsetbuttcap%
\pgfsetroundjoin%
\pgfsetlinewidth{1.003750pt}%
\definecolor{currentstroke}{rgb}{0.000000,0.000000,0.000000}%
\pgfsetstrokecolor{currentstroke}%
\pgfsetdash{}{0pt}%
\pgfpathmoveto{\pgfqpoint{0.957757in}{0.807940in}}%
\pgfpathcurveto{\pgfqpoint{0.968653in}{0.807940in}}{\pgfqpoint{0.979103in}{0.812269in}}{\pgfqpoint{0.986808in}{0.819974in}}%
\pgfpathcurveto{\pgfqpoint{0.994512in}{0.827678in}}{\pgfqpoint{0.998841in}{0.838129in}}{\pgfqpoint{0.998841in}{0.849024in}}%
\pgfpathcurveto{\pgfqpoint{0.998841in}{0.859920in}}{\pgfqpoint{0.994512in}{0.870371in}}{\pgfqpoint{0.986808in}{0.878075in}}%
\pgfpathcurveto{\pgfqpoint{0.979103in}{0.885779in}}{\pgfqpoint{0.968653in}{0.890108in}}{\pgfqpoint{0.957757in}{0.890108in}}%
\pgfpathcurveto{\pgfqpoint{0.946862in}{0.890108in}}{\pgfqpoint{0.936411in}{0.885779in}}{\pgfqpoint{0.928706in}{0.878075in}}%
\pgfpathcurveto{\pgfqpoint{0.921002in}{0.870371in}}{\pgfqpoint{0.916673in}{0.859920in}}{\pgfqpoint{0.916673in}{0.849024in}}%
\pgfpathcurveto{\pgfqpoint{0.916673in}{0.838129in}}{\pgfqpoint{0.921002in}{0.827678in}}{\pgfqpoint{0.928706in}{0.819974in}}%
\pgfpathcurveto{\pgfqpoint{0.936411in}{0.812269in}}{\pgfqpoint{0.946862in}{0.807940in}}{\pgfqpoint{0.957757in}{0.807940in}}%
\pgfpathlineto{\pgfqpoint{0.957757in}{0.807940in}}%
\pgfpathclose%
\pgfusepath{stroke}%
\end{pgfscope}%
\begin{pgfscope}%
\pgfpathrectangle{\pgfqpoint{0.688192in}{0.670138in}}{\pgfqpoint{7.111808in}{5.129862in}}%
\pgfusepath{clip}%
\pgfsetbuttcap%
\pgfsetroundjoin%
\pgfsetlinewidth{1.003750pt}%
\definecolor{currentstroke}{rgb}{0.000000,0.000000,0.000000}%
\pgfsetstrokecolor{currentstroke}%
\pgfsetdash{}{0pt}%
\pgfpathmoveto{\pgfqpoint{6.235447in}{1.342252in}}%
\pgfpathcurveto{\pgfqpoint{6.246343in}{1.342252in}}{\pgfqpoint{6.256793in}{1.346580in}}{\pgfqpoint{6.264498in}{1.354285in}}%
\pgfpathcurveto{\pgfqpoint{6.272202in}{1.361989in}}{\pgfqpoint{6.276531in}{1.372440in}}{\pgfqpoint{6.276531in}{1.383336in}}%
\pgfpathcurveto{\pgfqpoint{6.276531in}{1.394231in}}{\pgfqpoint{6.272202in}{1.404682in}}{\pgfqpoint{6.264498in}{1.412386in}}%
\pgfpathcurveto{\pgfqpoint{6.256793in}{1.420091in}}{\pgfqpoint{6.246343in}{1.424419in}}{\pgfqpoint{6.235447in}{1.424419in}}%
\pgfpathcurveto{\pgfqpoint{6.224551in}{1.424419in}}{\pgfqpoint{6.214101in}{1.420091in}}{\pgfqpoint{6.206396in}{1.412386in}}%
\pgfpathcurveto{\pgfqpoint{6.198692in}{1.404682in}}{\pgfqpoint{6.194363in}{1.394231in}}{\pgfqpoint{6.194363in}{1.383336in}}%
\pgfpathcurveto{\pgfqpoint{6.194363in}{1.372440in}}{\pgfqpoint{6.198692in}{1.361989in}}{\pgfqpoint{6.206396in}{1.354285in}}%
\pgfpathcurveto{\pgfqpoint{6.214101in}{1.346580in}}{\pgfqpoint{6.224551in}{1.342252in}}{\pgfqpoint{6.235447in}{1.342252in}}%
\pgfpathlineto{\pgfqpoint{6.235447in}{1.342252in}}%
\pgfpathclose%
\pgfusepath{stroke}%
\end{pgfscope}%
\begin{pgfscope}%
\pgfpathrectangle{\pgfqpoint{0.688192in}{0.670138in}}{\pgfqpoint{7.111808in}{5.129862in}}%
\pgfusepath{clip}%
\pgfsetbuttcap%
\pgfsetroundjoin%
\pgfsetlinewidth{1.003750pt}%
\definecolor{currentstroke}{rgb}{0.000000,0.000000,0.000000}%
\pgfsetstrokecolor{currentstroke}%
\pgfsetdash{}{0pt}%
\pgfpathmoveto{\pgfqpoint{0.895312in}{0.866174in}}%
\pgfpathcurveto{\pgfqpoint{0.906208in}{0.866174in}}{\pgfqpoint{0.916658in}{0.870503in}}{\pgfqpoint{0.924363in}{0.878207in}}%
\pgfpathcurveto{\pgfqpoint{0.932067in}{0.885911in}}{\pgfqpoint{0.936396in}{0.896362in}}{\pgfqpoint{0.936396in}{0.907258in}}%
\pgfpathcurveto{\pgfqpoint{0.936396in}{0.918153in}}{\pgfqpoint{0.932067in}{0.928604in}}{\pgfqpoint{0.924363in}{0.936308in}}%
\pgfpathcurveto{\pgfqpoint{0.916658in}{0.944013in}}{\pgfqpoint{0.906208in}{0.948341in}}{\pgfqpoint{0.895312in}{0.948341in}}%
\pgfpathcurveto{\pgfqpoint{0.884416in}{0.948341in}}{\pgfqpoint{0.873966in}{0.944013in}}{\pgfqpoint{0.866261in}{0.936308in}}%
\pgfpathcurveto{\pgfqpoint{0.858557in}{0.928604in}}{\pgfqpoint{0.854228in}{0.918153in}}{\pgfqpoint{0.854228in}{0.907258in}}%
\pgfpathcurveto{\pgfqpoint{0.854228in}{0.896362in}}{\pgfqpoint{0.858557in}{0.885911in}}{\pgfqpoint{0.866261in}{0.878207in}}%
\pgfpathcurveto{\pgfqpoint{0.873966in}{0.870503in}}{\pgfqpoint{0.884416in}{0.866174in}}{\pgfqpoint{0.895312in}{0.866174in}}%
\pgfpathlineto{\pgfqpoint{0.895312in}{0.866174in}}%
\pgfpathclose%
\pgfusepath{stroke}%
\end{pgfscope}%
\begin{pgfscope}%
\pgfpathrectangle{\pgfqpoint{0.688192in}{0.670138in}}{\pgfqpoint{7.111808in}{5.129862in}}%
\pgfusepath{clip}%
\pgfsetbuttcap%
\pgfsetroundjoin%
\pgfsetlinewidth{1.003750pt}%
\definecolor{currentstroke}{rgb}{0.000000,0.000000,0.000000}%
\pgfsetstrokecolor{currentstroke}%
\pgfsetdash{}{0pt}%
\pgfpathmoveto{\pgfqpoint{4.634758in}{0.639654in}}%
\pgfpathcurveto{\pgfqpoint{4.645653in}{0.639654in}}{\pgfqpoint{4.656104in}{0.643983in}}{\pgfqpoint{4.663808in}{0.651687in}}%
\pgfpathcurveto{\pgfqpoint{4.671513in}{0.659391in}}{\pgfqpoint{4.675841in}{0.669842in}}{\pgfqpoint{4.675841in}{0.680738in}}%
\pgfpathcurveto{\pgfqpoint{4.675841in}{0.691633in}}{\pgfqpoint{4.671513in}{0.702084in}}{\pgfqpoint{4.663808in}{0.709789in}}%
\pgfpathcurveto{\pgfqpoint{4.656104in}{0.717493in}}{\pgfqpoint{4.645653in}{0.721822in}}{\pgfqpoint{4.634758in}{0.721822in}}%
\pgfpathcurveto{\pgfqpoint{4.623862in}{0.721822in}}{\pgfqpoint{4.613411in}{0.717493in}}{\pgfqpoint{4.605707in}{0.709789in}}%
\pgfpathcurveto{\pgfqpoint{4.598003in}{0.702084in}}{\pgfqpoint{4.593674in}{0.691633in}}{\pgfqpoint{4.593674in}{0.680738in}}%
\pgfpathcurveto{\pgfqpoint{4.593674in}{0.669842in}}{\pgfqpoint{4.598003in}{0.659391in}}{\pgfqpoint{4.605707in}{0.651687in}}%
\pgfpathcurveto{\pgfqpoint{4.613411in}{0.643983in}}{\pgfqpoint{4.623862in}{0.639654in}}{\pgfqpoint{4.634758in}{0.639654in}}%
\pgfusepath{stroke}%
\end{pgfscope}%
\begin{pgfscope}%
\pgfpathrectangle{\pgfqpoint{0.688192in}{0.670138in}}{\pgfqpoint{7.111808in}{5.129862in}}%
\pgfusepath{clip}%
\pgfsetbuttcap%
\pgfsetroundjoin%
\pgfsetlinewidth{1.003750pt}%
\definecolor{currentstroke}{rgb}{0.000000,0.000000,0.000000}%
\pgfsetstrokecolor{currentstroke}%
\pgfsetdash{}{0pt}%
\pgfpathmoveto{\pgfqpoint{1.141797in}{0.719357in}}%
\pgfpathcurveto{\pgfqpoint{1.152693in}{0.719357in}}{\pgfqpoint{1.163143in}{0.723686in}}{\pgfqpoint{1.170848in}{0.731391in}}%
\pgfpathcurveto{\pgfqpoint{1.178552in}{0.739095in}}{\pgfqpoint{1.182881in}{0.749546in}}{\pgfqpoint{1.182881in}{0.760441in}}%
\pgfpathcurveto{\pgfqpoint{1.182881in}{0.771337in}}{\pgfqpoint{1.178552in}{0.781788in}}{\pgfqpoint{1.170848in}{0.789492in}}%
\pgfpathcurveto{\pgfqpoint{1.163143in}{0.797196in}}{\pgfqpoint{1.152693in}{0.801525in}}{\pgfqpoint{1.141797in}{0.801525in}}%
\pgfpathcurveto{\pgfqpoint{1.130902in}{0.801525in}}{\pgfqpoint{1.120451in}{0.797196in}}{\pgfqpoint{1.112746in}{0.789492in}}%
\pgfpathcurveto{\pgfqpoint{1.105042in}{0.781788in}}{\pgfqpoint{1.100713in}{0.771337in}}{\pgfqpoint{1.100713in}{0.760441in}}%
\pgfpathcurveto{\pgfqpoint{1.100713in}{0.749546in}}{\pgfqpoint{1.105042in}{0.739095in}}{\pgfqpoint{1.112746in}{0.731391in}}%
\pgfpathcurveto{\pgfqpoint{1.120451in}{0.723686in}}{\pgfqpoint{1.130902in}{0.719357in}}{\pgfqpoint{1.141797in}{0.719357in}}%
\pgfpathlineto{\pgfqpoint{1.141797in}{0.719357in}}%
\pgfpathclose%
\pgfusepath{stroke}%
\end{pgfscope}%
\begin{pgfscope}%
\pgfpathrectangle{\pgfqpoint{0.688192in}{0.670138in}}{\pgfqpoint{7.111808in}{5.129862in}}%
\pgfusepath{clip}%
\pgfsetbuttcap%
\pgfsetroundjoin%
\pgfsetlinewidth{1.003750pt}%
\definecolor{currentstroke}{rgb}{0.000000,0.000000,0.000000}%
\pgfsetstrokecolor{currentstroke}%
\pgfsetdash{}{0pt}%
\pgfpathmoveto{\pgfqpoint{1.389197in}{0.711266in}}%
\pgfpathcurveto{\pgfqpoint{1.400093in}{0.711266in}}{\pgfqpoint{1.410544in}{0.715595in}}{\pgfqpoint{1.418248in}{0.723299in}}%
\pgfpathcurveto{\pgfqpoint{1.425952in}{0.731004in}}{\pgfqpoint{1.430281in}{0.741454in}}{\pgfqpoint{1.430281in}{0.752350in}}%
\pgfpathcurveto{\pgfqpoint{1.430281in}{0.763246in}}{\pgfqpoint{1.425952in}{0.773696in}}{\pgfqpoint{1.418248in}{0.781401in}}%
\pgfpathcurveto{\pgfqpoint{1.410544in}{0.789105in}}{\pgfqpoint{1.400093in}{0.793434in}}{\pgfqpoint{1.389197in}{0.793434in}}%
\pgfpathcurveto{\pgfqpoint{1.378302in}{0.793434in}}{\pgfqpoint{1.367851in}{0.789105in}}{\pgfqpoint{1.360147in}{0.781401in}}%
\pgfpathcurveto{\pgfqpoint{1.352442in}{0.773696in}}{\pgfqpoint{1.348113in}{0.763246in}}{\pgfqpoint{1.348113in}{0.752350in}}%
\pgfpathcurveto{\pgfqpoint{1.348113in}{0.741454in}}{\pgfqpoint{1.352442in}{0.731004in}}{\pgfqpoint{1.360147in}{0.723299in}}%
\pgfpathcurveto{\pgfqpoint{1.367851in}{0.715595in}}{\pgfqpoint{1.378302in}{0.711266in}}{\pgfqpoint{1.389197in}{0.711266in}}%
\pgfpathlineto{\pgfqpoint{1.389197in}{0.711266in}}%
\pgfpathclose%
\pgfusepath{stroke}%
\end{pgfscope}%
\begin{pgfscope}%
\pgfpathrectangle{\pgfqpoint{0.688192in}{0.670138in}}{\pgfqpoint{7.111808in}{5.129862in}}%
\pgfusepath{clip}%
\pgfsetbuttcap%
\pgfsetroundjoin%
\pgfsetlinewidth{1.003750pt}%
\definecolor{currentstroke}{rgb}{0.000000,0.000000,0.000000}%
\pgfsetstrokecolor{currentstroke}%
\pgfsetdash{}{0pt}%
\pgfpathmoveto{\pgfqpoint{2.333738in}{3.417663in}}%
\pgfpathcurveto{\pgfqpoint{2.344634in}{3.417663in}}{\pgfqpoint{2.355085in}{3.421992in}}{\pgfqpoint{2.362789in}{3.429696in}}%
\pgfpathcurveto{\pgfqpoint{2.370493in}{3.437401in}}{\pgfqpoint{2.374822in}{3.447852in}}{\pgfqpoint{2.374822in}{3.458747in}}%
\pgfpathcurveto{\pgfqpoint{2.374822in}{3.469643in}}{\pgfqpoint{2.370493in}{3.480093in}}{\pgfqpoint{2.362789in}{3.487798in}}%
\pgfpathcurveto{\pgfqpoint{2.355085in}{3.495502in}}{\pgfqpoint{2.344634in}{3.499831in}}{\pgfqpoint{2.333738in}{3.499831in}}%
\pgfpathcurveto{\pgfqpoint{2.322843in}{3.499831in}}{\pgfqpoint{2.312392in}{3.495502in}}{\pgfqpoint{2.304688in}{3.487798in}}%
\pgfpathcurveto{\pgfqpoint{2.296983in}{3.480093in}}{\pgfqpoint{2.292655in}{3.469643in}}{\pgfqpoint{2.292655in}{3.458747in}}%
\pgfpathcurveto{\pgfqpoint{2.292655in}{3.447852in}}{\pgfqpoint{2.296983in}{3.437401in}}{\pgfqpoint{2.304688in}{3.429696in}}%
\pgfpathcurveto{\pgfqpoint{2.312392in}{3.421992in}}{\pgfqpoint{2.322843in}{3.417663in}}{\pgfqpoint{2.333738in}{3.417663in}}%
\pgfpathlineto{\pgfqpoint{2.333738in}{3.417663in}}%
\pgfpathclose%
\pgfusepath{stroke}%
\end{pgfscope}%
\begin{pgfscope}%
\pgfpathrectangle{\pgfqpoint{0.688192in}{0.670138in}}{\pgfqpoint{7.111808in}{5.129862in}}%
\pgfusepath{clip}%
\pgfsetbuttcap%
\pgfsetroundjoin%
\pgfsetlinewidth{1.003750pt}%
\definecolor{currentstroke}{rgb}{0.000000,0.000000,0.000000}%
\pgfsetstrokecolor{currentstroke}%
\pgfsetdash{}{0pt}%
\pgfpathmoveto{\pgfqpoint{5.112107in}{0.634839in}}%
\pgfpathcurveto{\pgfqpoint{5.123002in}{0.634839in}}{\pgfqpoint{5.133453in}{0.639168in}}{\pgfqpoint{5.141158in}{0.646872in}}%
\pgfpathcurveto{\pgfqpoint{5.148862in}{0.654577in}}{\pgfqpoint{5.153191in}{0.665027in}}{\pgfqpoint{5.153191in}{0.675923in}}%
\pgfpathcurveto{\pgfqpoint{5.153191in}{0.686819in}}{\pgfqpoint{5.148862in}{0.697269in}}{\pgfqpoint{5.141158in}{0.704974in}}%
\pgfpathcurveto{\pgfqpoint{5.133453in}{0.712678in}}{\pgfqpoint{5.123002in}{0.717007in}}{\pgfqpoint{5.112107in}{0.717007in}}%
\pgfpathcurveto{\pgfqpoint{5.101211in}{0.717007in}}{\pgfqpoint{5.090761in}{0.712678in}}{\pgfqpoint{5.083056in}{0.704974in}}%
\pgfpathcurveto{\pgfqpoint{5.075352in}{0.697269in}}{\pgfqpoint{5.071023in}{0.686819in}}{\pgfqpoint{5.071023in}{0.675923in}}%
\pgfpathcurveto{\pgfqpoint{5.071023in}{0.665027in}}{\pgfqpoint{5.075352in}{0.654577in}}{\pgfqpoint{5.083056in}{0.646872in}}%
\pgfpathcurveto{\pgfqpoint{5.090761in}{0.639168in}}{\pgfqpoint{5.101211in}{0.634839in}}{\pgfqpoint{5.112107in}{0.634839in}}%
\pgfusepath{stroke}%
\end{pgfscope}%
\begin{pgfscope}%
\pgfpathrectangle{\pgfqpoint{0.688192in}{0.670138in}}{\pgfqpoint{7.111808in}{5.129862in}}%
\pgfusepath{clip}%
\pgfsetbuttcap%
\pgfsetroundjoin%
\pgfsetlinewidth{1.003750pt}%
\definecolor{currentstroke}{rgb}{0.000000,0.000000,0.000000}%
\pgfsetstrokecolor{currentstroke}%
\pgfsetdash{}{0pt}%
\pgfpathmoveto{\pgfqpoint{5.530080in}{5.911460in}}%
\pgfpathcurveto{\pgfqpoint{5.540975in}{5.911460in}}{\pgfqpoint{5.551426in}{5.915789in}}{\pgfqpoint{5.559130in}{5.923493in}}%
\pgfpathcurveto{\pgfqpoint{5.566835in}{5.931197in}}{\pgfqpoint{5.571163in}{5.941648in}}{\pgfqpoint{5.571163in}{5.952544in}}%
\pgfpathcurveto{\pgfqpoint{5.571163in}{5.963439in}}{\pgfqpoint{5.566835in}{5.973890in}}{\pgfqpoint{5.559130in}{5.981595in}}%
\pgfpathcurveto{\pgfqpoint{5.551426in}{5.989299in}}{\pgfqpoint{5.540975in}{5.993628in}}{\pgfqpoint{5.530080in}{5.993628in}}%
\pgfpathcurveto{\pgfqpoint{5.519184in}{5.993628in}}{\pgfqpoint{5.508733in}{5.989299in}}{\pgfqpoint{5.501029in}{5.981595in}}%
\pgfpathcurveto{\pgfqpoint{5.493324in}{5.973890in}}{\pgfqpoint{5.488996in}{5.963439in}}{\pgfqpoint{5.488996in}{5.952544in}}%
\pgfpathcurveto{\pgfqpoint{5.488996in}{5.941648in}}{\pgfqpoint{5.493324in}{5.931197in}}{\pgfqpoint{5.501029in}{5.923493in}}%
\pgfpathcurveto{\pgfqpoint{5.508733in}{5.915789in}}{\pgfqpoint{5.519184in}{5.911460in}}{\pgfqpoint{5.530080in}{5.911460in}}%
\pgfusepath{stroke}%
\end{pgfscope}%
\begin{pgfscope}%
\pgfpathrectangle{\pgfqpoint{0.688192in}{0.670138in}}{\pgfqpoint{7.111808in}{5.129862in}}%
\pgfusepath{clip}%
\pgfsetbuttcap%
\pgfsetroundjoin%
\pgfsetlinewidth{1.003750pt}%
\definecolor{currentstroke}{rgb}{0.000000,0.000000,0.000000}%
\pgfsetstrokecolor{currentstroke}%
\pgfsetdash{}{0pt}%
\pgfpathmoveto{\pgfqpoint{0.886403in}{0.869810in}}%
\pgfpathcurveto{\pgfqpoint{0.897298in}{0.869810in}}{\pgfqpoint{0.907749in}{0.874139in}}{\pgfqpoint{0.915453in}{0.881844in}}%
\pgfpathcurveto{\pgfqpoint{0.923158in}{0.889548in}}{\pgfqpoint{0.927487in}{0.899999in}}{\pgfqpoint{0.927487in}{0.910894in}}%
\pgfpathcurveto{\pgfqpoint{0.927487in}{0.921790in}}{\pgfqpoint{0.923158in}{0.932241in}}{\pgfqpoint{0.915453in}{0.939945in}}%
\pgfpathcurveto{\pgfqpoint{0.907749in}{0.947649in}}{\pgfqpoint{0.897298in}{0.951978in}}{\pgfqpoint{0.886403in}{0.951978in}}%
\pgfpathcurveto{\pgfqpoint{0.875507in}{0.951978in}}{\pgfqpoint{0.865056in}{0.947649in}}{\pgfqpoint{0.857352in}{0.939945in}}%
\pgfpathcurveto{\pgfqpoint{0.849648in}{0.932241in}}{\pgfqpoint{0.845319in}{0.921790in}}{\pgfqpoint{0.845319in}{0.910894in}}%
\pgfpathcurveto{\pgfqpoint{0.845319in}{0.899999in}}{\pgfqpoint{0.849648in}{0.889548in}}{\pgfqpoint{0.857352in}{0.881844in}}%
\pgfpathcurveto{\pgfqpoint{0.865056in}{0.874139in}}{\pgfqpoint{0.875507in}{0.869810in}}{\pgfqpoint{0.886403in}{0.869810in}}%
\pgfpathlineto{\pgfqpoint{0.886403in}{0.869810in}}%
\pgfpathclose%
\pgfusepath{stroke}%
\end{pgfscope}%
\begin{pgfscope}%
\pgfpathrectangle{\pgfqpoint{0.688192in}{0.670138in}}{\pgfqpoint{7.111808in}{5.129862in}}%
\pgfusepath{clip}%
\pgfsetbuttcap%
\pgfsetroundjoin%
\pgfsetlinewidth{1.003750pt}%
\definecolor{currentstroke}{rgb}{0.000000,0.000000,0.000000}%
\pgfsetstrokecolor{currentstroke}%
\pgfsetdash{}{0pt}%
\pgfpathmoveto{\pgfqpoint{0.931867in}{0.825133in}}%
\pgfpathcurveto{\pgfqpoint{0.942763in}{0.825133in}}{\pgfqpoint{0.953213in}{0.829462in}}{\pgfqpoint{0.960918in}{0.837167in}}%
\pgfpathcurveto{\pgfqpoint{0.968622in}{0.844871in}}{\pgfqpoint{0.972951in}{0.855322in}}{\pgfqpoint{0.972951in}{0.866217in}}%
\pgfpathcurveto{\pgfqpoint{0.972951in}{0.877113in}}{\pgfqpoint{0.968622in}{0.887564in}}{\pgfqpoint{0.960918in}{0.895268in}}%
\pgfpathcurveto{\pgfqpoint{0.953213in}{0.902972in}}{\pgfqpoint{0.942763in}{0.907301in}}{\pgfqpoint{0.931867in}{0.907301in}}%
\pgfpathcurveto{\pgfqpoint{0.920971in}{0.907301in}}{\pgfqpoint{0.910521in}{0.902972in}}{\pgfqpoint{0.902816in}{0.895268in}}%
\pgfpathcurveto{\pgfqpoint{0.895112in}{0.887564in}}{\pgfqpoint{0.890783in}{0.877113in}}{\pgfqpoint{0.890783in}{0.866217in}}%
\pgfpathcurveto{\pgfqpoint{0.890783in}{0.855322in}}{\pgfqpoint{0.895112in}{0.844871in}}{\pgfqpoint{0.902816in}{0.837167in}}%
\pgfpathcurveto{\pgfqpoint{0.910521in}{0.829462in}}{\pgfqpoint{0.920971in}{0.825133in}}{\pgfqpoint{0.931867in}{0.825133in}}%
\pgfpathlineto{\pgfqpoint{0.931867in}{0.825133in}}%
\pgfpathclose%
\pgfusepath{stroke}%
\end{pgfscope}%
\begin{pgfscope}%
\pgfpathrectangle{\pgfqpoint{0.688192in}{0.670138in}}{\pgfqpoint{7.111808in}{5.129862in}}%
\pgfusepath{clip}%
\pgfsetbuttcap%
\pgfsetroundjoin%
\pgfsetlinewidth{1.003750pt}%
\definecolor{currentstroke}{rgb}{0.000000,0.000000,0.000000}%
\pgfsetstrokecolor{currentstroke}%
\pgfsetdash{}{0pt}%
\pgfpathmoveto{\pgfqpoint{1.141797in}{0.719357in}}%
\pgfpathcurveto{\pgfqpoint{1.152693in}{0.719357in}}{\pgfqpoint{1.163143in}{0.723686in}}{\pgfqpoint{1.170848in}{0.731391in}}%
\pgfpathcurveto{\pgfqpoint{1.178552in}{0.739095in}}{\pgfqpoint{1.182881in}{0.749546in}}{\pgfqpoint{1.182881in}{0.760441in}}%
\pgfpathcurveto{\pgfqpoint{1.182881in}{0.771337in}}{\pgfqpoint{1.178552in}{0.781788in}}{\pgfqpoint{1.170848in}{0.789492in}}%
\pgfpathcurveto{\pgfqpoint{1.163143in}{0.797196in}}{\pgfqpoint{1.152693in}{0.801525in}}{\pgfqpoint{1.141797in}{0.801525in}}%
\pgfpathcurveto{\pgfqpoint{1.130902in}{0.801525in}}{\pgfqpoint{1.120451in}{0.797196in}}{\pgfqpoint{1.112746in}{0.789492in}}%
\pgfpathcurveto{\pgfqpoint{1.105042in}{0.781788in}}{\pgfqpoint{1.100713in}{0.771337in}}{\pgfqpoint{1.100713in}{0.760441in}}%
\pgfpathcurveto{\pgfqpoint{1.100713in}{0.749546in}}{\pgfqpoint{1.105042in}{0.739095in}}{\pgfqpoint{1.112746in}{0.731391in}}%
\pgfpathcurveto{\pgfqpoint{1.120451in}{0.723686in}}{\pgfqpoint{1.130902in}{0.719357in}}{\pgfqpoint{1.141797in}{0.719357in}}%
\pgfpathlineto{\pgfqpoint{1.141797in}{0.719357in}}%
\pgfpathclose%
\pgfusepath{stroke}%
\end{pgfscope}%
\begin{pgfscope}%
\pgfpathrectangle{\pgfqpoint{0.688192in}{0.670138in}}{\pgfqpoint{7.111808in}{5.129862in}}%
\pgfusepath{clip}%
\pgfsetbuttcap%
\pgfsetroundjoin%
\pgfsetlinewidth{1.003750pt}%
\definecolor{currentstroke}{rgb}{0.000000,0.000000,0.000000}%
\pgfsetstrokecolor{currentstroke}%
\pgfsetdash{}{0pt}%
\pgfpathmoveto{\pgfqpoint{0.782559in}{1.071983in}}%
\pgfpathcurveto{\pgfqpoint{0.793455in}{1.071983in}}{\pgfqpoint{0.803905in}{1.076312in}}{\pgfqpoint{0.811610in}{1.084017in}}%
\pgfpathcurveto{\pgfqpoint{0.819314in}{1.091721in}}{\pgfqpoint{0.823643in}{1.102172in}}{\pgfqpoint{0.823643in}{1.113067in}}%
\pgfpathcurveto{\pgfqpoint{0.823643in}{1.123963in}}{\pgfqpoint{0.819314in}{1.134414in}}{\pgfqpoint{0.811610in}{1.142118in}}%
\pgfpathcurveto{\pgfqpoint{0.803905in}{1.149822in}}{\pgfqpoint{0.793455in}{1.154151in}}{\pgfqpoint{0.782559in}{1.154151in}}%
\pgfpathcurveto{\pgfqpoint{0.771663in}{1.154151in}}{\pgfqpoint{0.761213in}{1.149822in}}{\pgfqpoint{0.753508in}{1.142118in}}%
\pgfpathcurveto{\pgfqpoint{0.745804in}{1.134414in}}{\pgfqpoint{0.741475in}{1.123963in}}{\pgfqpoint{0.741475in}{1.113067in}}%
\pgfpathcurveto{\pgfqpoint{0.741475in}{1.102172in}}{\pgfqpoint{0.745804in}{1.091721in}}{\pgfqpoint{0.753508in}{1.084017in}}%
\pgfpathcurveto{\pgfqpoint{0.761213in}{1.076312in}}{\pgfqpoint{0.771663in}{1.071983in}}{\pgfqpoint{0.782559in}{1.071983in}}%
\pgfpathlineto{\pgfqpoint{0.782559in}{1.071983in}}%
\pgfpathclose%
\pgfusepath{stroke}%
\end{pgfscope}%
\begin{pgfscope}%
\pgfpathrectangle{\pgfqpoint{0.688192in}{0.670138in}}{\pgfqpoint{7.111808in}{5.129862in}}%
\pgfusepath{clip}%
\pgfsetbuttcap%
\pgfsetroundjoin%
\pgfsetlinewidth{1.003750pt}%
\definecolor{currentstroke}{rgb}{0.000000,0.000000,0.000000}%
\pgfsetstrokecolor{currentstroke}%
\pgfsetdash{}{0pt}%
\pgfpathmoveto{\pgfqpoint{5.696293in}{0.630769in}}%
\pgfpathcurveto{\pgfqpoint{5.707189in}{0.630769in}}{\pgfqpoint{5.717639in}{0.635098in}}{\pgfqpoint{5.725344in}{0.642802in}}%
\pgfpathcurveto{\pgfqpoint{5.733048in}{0.650507in}}{\pgfqpoint{5.737377in}{0.660958in}}{\pgfqpoint{5.737377in}{0.671853in}}%
\pgfpathcurveto{\pgfqpoint{5.737377in}{0.682749in}}{\pgfqpoint{5.733048in}{0.693200in}}{\pgfqpoint{5.725344in}{0.700904in}}%
\pgfpathcurveto{\pgfqpoint{5.717639in}{0.708608in}}{\pgfqpoint{5.707189in}{0.712937in}}{\pgfqpoint{5.696293in}{0.712937in}}%
\pgfpathcurveto{\pgfqpoint{5.685397in}{0.712937in}}{\pgfqpoint{5.674947in}{0.708608in}}{\pgfqpoint{5.667242in}{0.700904in}}%
\pgfpathcurveto{\pgfqpoint{5.659538in}{0.693200in}}{\pgfqpoint{5.655209in}{0.682749in}}{\pgfqpoint{5.655209in}{0.671853in}}%
\pgfpathcurveto{\pgfqpoint{5.655209in}{0.660958in}}{\pgfqpoint{5.659538in}{0.650507in}}{\pgfqpoint{5.667242in}{0.642802in}}%
\pgfpathcurveto{\pgfqpoint{5.674947in}{0.635098in}}{\pgfqpoint{5.685397in}{0.630769in}}{\pgfqpoint{5.696293in}{0.630769in}}%
\pgfusepath{stroke}%
\end{pgfscope}%
\begin{pgfscope}%
\pgfpathrectangle{\pgfqpoint{0.688192in}{0.670138in}}{\pgfqpoint{7.111808in}{5.129862in}}%
\pgfusepath{clip}%
\pgfsetbuttcap%
\pgfsetroundjoin%
\pgfsetlinewidth{1.003750pt}%
\definecolor{currentstroke}{rgb}{0.000000,0.000000,0.000000}%
\pgfsetstrokecolor{currentstroke}%
\pgfsetdash{}{0pt}%
\pgfpathmoveto{\pgfqpoint{1.351667in}{0.711476in}}%
\pgfpathcurveto{\pgfqpoint{1.362562in}{0.711476in}}{\pgfqpoint{1.373013in}{0.715805in}}{\pgfqpoint{1.380718in}{0.723510in}}%
\pgfpathcurveto{\pgfqpoint{1.388422in}{0.731214in}}{\pgfqpoint{1.392751in}{0.741665in}}{\pgfqpoint{1.392751in}{0.752560in}}%
\pgfpathcurveto{\pgfqpoint{1.392751in}{0.763456in}}{\pgfqpoint{1.388422in}{0.773907in}}{\pgfqpoint{1.380718in}{0.781611in}}%
\pgfpathcurveto{\pgfqpoint{1.373013in}{0.789315in}}{\pgfqpoint{1.362562in}{0.793644in}}{\pgfqpoint{1.351667in}{0.793644in}}%
\pgfpathcurveto{\pgfqpoint{1.340771in}{0.793644in}}{\pgfqpoint{1.330321in}{0.789315in}}{\pgfqpoint{1.322616in}{0.781611in}}%
\pgfpathcurveto{\pgfqpoint{1.314912in}{0.773907in}}{\pgfqpoint{1.310583in}{0.763456in}}{\pgfqpoint{1.310583in}{0.752560in}}%
\pgfpathcurveto{\pgfqpoint{1.310583in}{0.741665in}}{\pgfqpoint{1.314912in}{0.731214in}}{\pgfqpoint{1.322616in}{0.723510in}}%
\pgfpathcurveto{\pgfqpoint{1.330321in}{0.715805in}}{\pgfqpoint{1.340771in}{0.711476in}}{\pgfqpoint{1.351667in}{0.711476in}}%
\pgfpathlineto{\pgfqpoint{1.351667in}{0.711476in}}%
\pgfpathclose%
\pgfusepath{stroke}%
\end{pgfscope}%
\begin{pgfscope}%
\pgfpathrectangle{\pgfqpoint{0.688192in}{0.670138in}}{\pgfqpoint{7.111808in}{5.129862in}}%
\pgfusepath{clip}%
\pgfsetbuttcap%
\pgfsetroundjoin%
\pgfsetlinewidth{1.003750pt}%
\definecolor{currentstroke}{rgb}{0.000000,0.000000,0.000000}%
\pgfsetstrokecolor{currentstroke}%
\pgfsetdash{}{0pt}%
\pgfpathmoveto{\pgfqpoint{0.755774in}{1.165660in}}%
\pgfpathcurveto{\pgfqpoint{0.766670in}{1.165660in}}{\pgfqpoint{0.777120in}{1.169989in}}{\pgfqpoint{0.784825in}{1.177693in}}%
\pgfpathcurveto{\pgfqpoint{0.792529in}{1.185398in}}{\pgfqpoint{0.796858in}{1.195848in}}{\pgfqpoint{0.796858in}{1.206744in}}%
\pgfpathcurveto{\pgfqpoint{0.796858in}{1.217640in}}{\pgfqpoint{0.792529in}{1.228090in}}{\pgfqpoint{0.784825in}{1.235795in}}%
\pgfpathcurveto{\pgfqpoint{0.777120in}{1.243499in}}{\pgfqpoint{0.766670in}{1.247828in}}{\pgfqpoint{0.755774in}{1.247828in}}%
\pgfpathcurveto{\pgfqpoint{0.744878in}{1.247828in}}{\pgfqpoint{0.734428in}{1.243499in}}{\pgfqpoint{0.726723in}{1.235795in}}%
\pgfpathcurveto{\pgfqpoint{0.719019in}{1.228090in}}{\pgfqpoint{0.714690in}{1.217640in}}{\pgfqpoint{0.714690in}{1.206744in}}%
\pgfpathcurveto{\pgfqpoint{0.714690in}{1.195848in}}{\pgfqpoint{0.719019in}{1.185398in}}{\pgfqpoint{0.726723in}{1.177693in}}%
\pgfpathcurveto{\pgfqpoint{0.734428in}{1.169989in}}{\pgfqpoint{0.744878in}{1.165660in}}{\pgfqpoint{0.755774in}{1.165660in}}%
\pgfpathlineto{\pgfqpoint{0.755774in}{1.165660in}}%
\pgfpathclose%
\pgfusepath{stroke}%
\end{pgfscope}%
\begin{pgfscope}%
\pgfpathrectangle{\pgfqpoint{0.688192in}{0.670138in}}{\pgfqpoint{7.111808in}{5.129862in}}%
\pgfusepath{clip}%
\pgfsetbuttcap%
\pgfsetroundjoin%
\pgfsetlinewidth{1.003750pt}%
\definecolor{currentstroke}{rgb}{0.000000,0.000000,0.000000}%
\pgfsetstrokecolor{currentstroke}%
\pgfsetdash{}{0pt}%
\pgfpathmoveto{\pgfqpoint{2.887549in}{5.756388in}}%
\pgfpathcurveto{\pgfqpoint{2.898445in}{5.756388in}}{\pgfqpoint{2.908896in}{5.760716in}}{\pgfqpoint{2.916600in}{5.768421in}}%
\pgfpathcurveto{\pgfqpoint{2.924305in}{5.776125in}}{\pgfqpoint{2.928633in}{5.786576in}}{\pgfqpoint{2.928633in}{5.797472in}}%
\pgfpathcurveto{\pgfqpoint{2.928633in}{5.808367in}}{\pgfqpoint{2.924305in}{5.818818in}}{\pgfqpoint{2.916600in}{5.826522in}}%
\pgfpathcurveto{\pgfqpoint{2.908896in}{5.834227in}}{\pgfqpoint{2.898445in}{5.838555in}}{\pgfqpoint{2.887549in}{5.838555in}}%
\pgfpathcurveto{\pgfqpoint{2.876654in}{5.838555in}}{\pgfqpoint{2.866203in}{5.834227in}}{\pgfqpoint{2.858499in}{5.826522in}}%
\pgfpathcurveto{\pgfqpoint{2.850794in}{5.818818in}}{\pgfqpoint{2.846466in}{5.808367in}}{\pgfqpoint{2.846466in}{5.797472in}}%
\pgfpathcurveto{\pgfqpoint{2.846466in}{5.786576in}}{\pgfqpoint{2.850794in}{5.776125in}}{\pgfqpoint{2.858499in}{5.768421in}}%
\pgfpathcurveto{\pgfqpoint{2.866203in}{5.760716in}}{\pgfqpoint{2.876654in}{5.756388in}}{\pgfqpoint{2.887549in}{5.756388in}}%
\pgfpathlineto{\pgfqpoint{2.887549in}{5.756388in}}%
\pgfpathclose%
\pgfusepath{stroke}%
\end{pgfscope}%
\begin{pgfscope}%
\pgfpathrectangle{\pgfqpoint{0.688192in}{0.670138in}}{\pgfqpoint{7.111808in}{5.129862in}}%
\pgfusepath{clip}%
\pgfsetbuttcap%
\pgfsetroundjoin%
\pgfsetlinewidth{1.003750pt}%
\definecolor{currentstroke}{rgb}{0.000000,0.000000,0.000000}%
\pgfsetstrokecolor{currentstroke}%
\pgfsetdash{}{0pt}%
\pgfpathmoveto{\pgfqpoint{5.556869in}{1.422817in}}%
\pgfpathcurveto{\pgfqpoint{5.567765in}{1.422817in}}{\pgfqpoint{5.578216in}{1.427145in}}{\pgfqpoint{5.585920in}{1.434850in}}%
\pgfpathcurveto{\pgfqpoint{5.593624in}{1.442554in}}{\pgfqpoint{5.597953in}{1.453005in}}{\pgfqpoint{5.597953in}{1.463900in}}%
\pgfpathcurveto{\pgfqpoint{5.597953in}{1.474796in}}{\pgfqpoint{5.593624in}{1.485247in}}{\pgfqpoint{5.585920in}{1.492951in}}%
\pgfpathcurveto{\pgfqpoint{5.578216in}{1.500655in}}{\pgfqpoint{5.567765in}{1.504984in}}{\pgfqpoint{5.556869in}{1.504984in}}%
\pgfpathcurveto{\pgfqpoint{5.545974in}{1.504984in}}{\pgfqpoint{5.535523in}{1.500655in}}{\pgfqpoint{5.527819in}{1.492951in}}%
\pgfpathcurveto{\pgfqpoint{5.520114in}{1.485247in}}{\pgfqpoint{5.515785in}{1.474796in}}{\pgfqpoint{5.515785in}{1.463900in}}%
\pgfpathcurveto{\pgfqpoint{5.515785in}{1.453005in}}{\pgfqpoint{5.520114in}{1.442554in}}{\pgfqpoint{5.527819in}{1.434850in}}%
\pgfpathcurveto{\pgfqpoint{5.535523in}{1.427145in}}{\pgfqpoint{5.545974in}{1.422817in}}{\pgfqpoint{5.556869in}{1.422817in}}%
\pgfpathlineto{\pgfqpoint{5.556869in}{1.422817in}}%
\pgfpathclose%
\pgfusepath{stroke}%
\end{pgfscope}%
\begin{pgfscope}%
\pgfpathrectangle{\pgfqpoint{0.688192in}{0.670138in}}{\pgfqpoint{7.111808in}{5.129862in}}%
\pgfusepath{clip}%
\pgfsetbuttcap%
\pgfsetroundjoin%
\pgfsetlinewidth{1.003750pt}%
\definecolor{currentstroke}{rgb}{0.000000,0.000000,0.000000}%
\pgfsetstrokecolor{currentstroke}%
\pgfsetdash{}{0pt}%
\pgfpathmoveto{\pgfqpoint{2.406189in}{0.678386in}}%
\pgfpathcurveto{\pgfqpoint{2.417084in}{0.678386in}}{\pgfqpoint{2.427535in}{0.682715in}}{\pgfqpoint{2.435240in}{0.690419in}}%
\pgfpathcurveto{\pgfqpoint{2.442944in}{0.698123in}}{\pgfqpoint{2.447273in}{0.708574in}}{\pgfqpoint{2.447273in}{0.719470in}}%
\pgfpathcurveto{\pgfqpoint{2.447273in}{0.730365in}}{\pgfqpoint{2.442944in}{0.740816in}}{\pgfqpoint{2.435240in}{0.748521in}}%
\pgfpathcurveto{\pgfqpoint{2.427535in}{0.756225in}}{\pgfqpoint{2.417084in}{0.760554in}}{\pgfqpoint{2.406189in}{0.760554in}}%
\pgfpathcurveto{\pgfqpoint{2.395293in}{0.760554in}}{\pgfqpoint{2.384843in}{0.756225in}}{\pgfqpoint{2.377138in}{0.748521in}}%
\pgfpathcurveto{\pgfqpoint{2.369434in}{0.740816in}}{\pgfqpoint{2.365105in}{0.730365in}}{\pgfqpoint{2.365105in}{0.719470in}}%
\pgfpathcurveto{\pgfqpoint{2.365105in}{0.708574in}}{\pgfqpoint{2.369434in}{0.698123in}}{\pgfqpoint{2.377138in}{0.690419in}}%
\pgfpathcurveto{\pgfqpoint{2.384843in}{0.682715in}}{\pgfqpoint{2.395293in}{0.678386in}}{\pgfqpoint{2.406189in}{0.678386in}}%
\pgfpathlineto{\pgfqpoint{2.406189in}{0.678386in}}%
\pgfpathclose%
\pgfusepath{stroke}%
\end{pgfscope}%
\begin{pgfscope}%
\pgfpathrectangle{\pgfqpoint{0.688192in}{0.670138in}}{\pgfqpoint{7.111808in}{5.129862in}}%
\pgfusepath{clip}%
\pgfsetbuttcap%
\pgfsetroundjoin%
\pgfsetlinewidth{1.003750pt}%
\definecolor{currentstroke}{rgb}{0.000000,0.000000,0.000000}%
\pgfsetstrokecolor{currentstroke}%
\pgfsetdash{}{0pt}%
\pgfpathmoveto{\pgfqpoint{1.441321in}{0.708382in}}%
\pgfpathcurveto{\pgfqpoint{1.452217in}{0.708382in}}{\pgfqpoint{1.462667in}{0.712711in}}{\pgfqpoint{1.470372in}{0.720416in}}%
\pgfpathcurveto{\pgfqpoint{1.478076in}{0.728120in}}{\pgfqpoint{1.482405in}{0.738571in}}{\pgfqpoint{1.482405in}{0.749466in}}%
\pgfpathcurveto{\pgfqpoint{1.482405in}{0.760362in}}{\pgfqpoint{1.478076in}{0.770813in}}{\pgfqpoint{1.470372in}{0.778517in}}%
\pgfpathcurveto{\pgfqpoint{1.462667in}{0.786221in}}{\pgfqpoint{1.452217in}{0.790550in}}{\pgfqpoint{1.441321in}{0.790550in}}%
\pgfpathcurveto{\pgfqpoint{1.430425in}{0.790550in}}{\pgfqpoint{1.419975in}{0.786221in}}{\pgfqpoint{1.412270in}{0.778517in}}%
\pgfpathcurveto{\pgfqpoint{1.404566in}{0.770813in}}{\pgfqpoint{1.400237in}{0.760362in}}{\pgfqpoint{1.400237in}{0.749466in}}%
\pgfpathcurveto{\pgfqpoint{1.400237in}{0.738571in}}{\pgfqpoint{1.404566in}{0.728120in}}{\pgfqpoint{1.412270in}{0.720416in}}%
\pgfpathcurveto{\pgfqpoint{1.419975in}{0.712711in}}{\pgfqpoint{1.430425in}{0.708382in}}{\pgfqpoint{1.441321in}{0.708382in}}%
\pgfpathlineto{\pgfqpoint{1.441321in}{0.708382in}}%
\pgfpathclose%
\pgfusepath{stroke}%
\end{pgfscope}%
\begin{pgfscope}%
\pgfpathrectangle{\pgfqpoint{0.688192in}{0.670138in}}{\pgfqpoint{7.111808in}{5.129862in}}%
\pgfusepath{clip}%
\pgfsetbuttcap%
\pgfsetroundjoin%
\pgfsetlinewidth{1.003750pt}%
\definecolor{currentstroke}{rgb}{0.000000,0.000000,0.000000}%
\pgfsetstrokecolor{currentstroke}%
\pgfsetdash{}{0pt}%
\pgfpathmoveto{\pgfqpoint{0.722061in}{1.890620in}}%
\pgfpathcurveto{\pgfqpoint{0.732957in}{1.890620in}}{\pgfqpoint{0.743408in}{1.894949in}}{\pgfqpoint{0.751112in}{1.902653in}}%
\pgfpathcurveto{\pgfqpoint{0.758816in}{1.910357in}}{\pgfqpoint{0.763145in}{1.920808in}}{\pgfqpoint{0.763145in}{1.931704in}}%
\pgfpathcurveto{\pgfqpoint{0.763145in}{1.942599in}}{\pgfqpoint{0.758816in}{1.953050in}}{\pgfqpoint{0.751112in}{1.960754in}}%
\pgfpathcurveto{\pgfqpoint{0.743408in}{1.968459in}}{\pgfqpoint{0.732957in}{1.972788in}}{\pgfqpoint{0.722061in}{1.972788in}}%
\pgfpathcurveto{\pgfqpoint{0.711166in}{1.972788in}}{\pgfqpoint{0.700715in}{1.968459in}}{\pgfqpoint{0.693011in}{1.960754in}}%
\pgfpathcurveto{\pgfqpoint{0.685306in}{1.953050in}}{\pgfqpoint{0.680977in}{1.942599in}}{\pgfqpoint{0.680977in}{1.931704in}}%
\pgfpathcurveto{\pgfqpoint{0.680977in}{1.920808in}}{\pgfqpoint{0.685306in}{1.910357in}}{\pgfqpoint{0.693011in}{1.902653in}}%
\pgfpathcurveto{\pgfqpoint{0.700715in}{1.894949in}}{\pgfqpoint{0.711166in}{1.890620in}}{\pgfqpoint{0.722061in}{1.890620in}}%
\pgfpathlineto{\pgfqpoint{0.722061in}{1.890620in}}%
\pgfpathclose%
\pgfusepath{stroke}%
\end{pgfscope}%
\begin{pgfscope}%
\pgfpathrectangle{\pgfqpoint{0.688192in}{0.670138in}}{\pgfqpoint{7.111808in}{5.129862in}}%
\pgfusepath{clip}%
\pgfsetbuttcap%
\pgfsetroundjoin%
\pgfsetlinewidth{1.003750pt}%
\definecolor{currentstroke}{rgb}{0.000000,0.000000,0.000000}%
\pgfsetstrokecolor{currentstroke}%
\pgfsetdash{}{0pt}%
\pgfpathmoveto{\pgfqpoint{3.916062in}{0.893951in}}%
\pgfpathcurveto{\pgfqpoint{3.926957in}{0.893951in}}{\pgfqpoint{3.937408in}{0.898280in}}{\pgfqpoint{3.945112in}{0.905984in}}%
\pgfpathcurveto{\pgfqpoint{3.952817in}{0.913688in}}{\pgfqpoint{3.957146in}{0.924139in}}{\pgfqpoint{3.957146in}{0.935035in}}%
\pgfpathcurveto{\pgfqpoint{3.957146in}{0.945930in}}{\pgfqpoint{3.952817in}{0.956381in}}{\pgfqpoint{3.945112in}{0.964085in}}%
\pgfpathcurveto{\pgfqpoint{3.937408in}{0.971790in}}{\pgfqpoint{3.926957in}{0.976118in}}{\pgfqpoint{3.916062in}{0.976118in}}%
\pgfpathcurveto{\pgfqpoint{3.905166in}{0.976118in}}{\pgfqpoint{3.894715in}{0.971790in}}{\pgfqpoint{3.887011in}{0.964085in}}%
\pgfpathcurveto{\pgfqpoint{3.879307in}{0.956381in}}{\pgfqpoint{3.874978in}{0.945930in}}{\pgfqpoint{3.874978in}{0.935035in}}%
\pgfpathcurveto{\pgfqpoint{3.874978in}{0.924139in}}{\pgfqpoint{3.879307in}{0.913688in}}{\pgfqpoint{3.887011in}{0.905984in}}%
\pgfpathcurveto{\pgfqpoint{3.894715in}{0.898280in}}{\pgfqpoint{3.905166in}{0.893951in}}{\pgfqpoint{3.916062in}{0.893951in}}%
\pgfpathlineto{\pgfqpoint{3.916062in}{0.893951in}}%
\pgfpathclose%
\pgfusepath{stroke}%
\end{pgfscope}%
\begin{pgfscope}%
\pgfpathrectangle{\pgfqpoint{0.688192in}{0.670138in}}{\pgfqpoint{7.111808in}{5.129862in}}%
\pgfusepath{clip}%
\pgfsetbuttcap%
\pgfsetroundjoin%
\pgfsetlinewidth{1.003750pt}%
\definecolor{currentstroke}{rgb}{0.000000,0.000000,0.000000}%
\pgfsetstrokecolor{currentstroke}%
\pgfsetdash{}{0pt}%
\pgfpathmoveto{\pgfqpoint{4.391763in}{0.641577in}}%
\pgfpathcurveto{\pgfqpoint{4.402658in}{0.641577in}}{\pgfqpoint{4.413109in}{0.645906in}}{\pgfqpoint{4.420813in}{0.653610in}}%
\pgfpathcurveto{\pgfqpoint{4.428518in}{0.661315in}}{\pgfqpoint{4.432846in}{0.671766in}}{\pgfqpoint{4.432846in}{0.682661in}}%
\pgfpathcurveto{\pgfqpoint{4.432846in}{0.693557in}}{\pgfqpoint{4.428518in}{0.704007in}}{\pgfqpoint{4.420813in}{0.711712in}}%
\pgfpathcurveto{\pgfqpoint{4.413109in}{0.719416in}}{\pgfqpoint{4.402658in}{0.723745in}}{\pgfqpoint{4.391763in}{0.723745in}}%
\pgfpathcurveto{\pgfqpoint{4.380867in}{0.723745in}}{\pgfqpoint{4.370416in}{0.719416in}}{\pgfqpoint{4.362712in}{0.711712in}}%
\pgfpathcurveto{\pgfqpoint{4.355007in}{0.704007in}}{\pgfqpoint{4.350679in}{0.693557in}}{\pgfqpoint{4.350679in}{0.682661in}}%
\pgfpathcurveto{\pgfqpoint{4.350679in}{0.671766in}}{\pgfqpoint{4.355007in}{0.661315in}}{\pgfqpoint{4.362712in}{0.653610in}}%
\pgfpathcurveto{\pgfqpoint{4.370416in}{0.645906in}}{\pgfqpoint{4.380867in}{0.641577in}}{\pgfqpoint{4.391763in}{0.641577in}}%
\pgfusepath{stroke}%
\end{pgfscope}%
\begin{pgfscope}%
\pgfpathrectangle{\pgfqpoint{0.688192in}{0.670138in}}{\pgfqpoint{7.111808in}{5.129862in}}%
\pgfusepath{clip}%
\pgfsetbuttcap%
\pgfsetroundjoin%
\pgfsetlinewidth{1.003750pt}%
\definecolor{currentstroke}{rgb}{0.000000,0.000000,0.000000}%
\pgfsetstrokecolor{currentstroke}%
\pgfsetdash{}{0pt}%
\pgfpathmoveto{\pgfqpoint{2.497167in}{0.675229in}}%
\pgfpathcurveto{\pgfqpoint{2.508063in}{0.675229in}}{\pgfqpoint{2.518514in}{0.679557in}}{\pgfqpoint{2.526218in}{0.687262in}}%
\pgfpathcurveto{\pgfqpoint{2.533922in}{0.694966in}}{\pgfqpoint{2.538251in}{0.705417in}}{\pgfqpoint{2.538251in}{0.716312in}}%
\pgfpathcurveto{\pgfqpoint{2.538251in}{0.727208in}}{\pgfqpoint{2.533922in}{0.737659in}}{\pgfqpoint{2.526218in}{0.745363in}}%
\pgfpathcurveto{\pgfqpoint{2.518514in}{0.753067in}}{\pgfqpoint{2.508063in}{0.757396in}}{\pgfqpoint{2.497167in}{0.757396in}}%
\pgfpathcurveto{\pgfqpoint{2.486272in}{0.757396in}}{\pgfqpoint{2.475821in}{0.753067in}}{\pgfqpoint{2.468117in}{0.745363in}}%
\pgfpathcurveto{\pgfqpoint{2.460412in}{0.737659in}}{\pgfqpoint{2.456084in}{0.727208in}}{\pgfqpoint{2.456084in}{0.716312in}}%
\pgfpathcurveto{\pgfqpoint{2.456084in}{0.705417in}}{\pgfqpoint{2.460412in}{0.694966in}}{\pgfqpoint{2.468117in}{0.687262in}}%
\pgfpathcurveto{\pgfqpoint{2.475821in}{0.679557in}}{\pgfqpoint{2.486272in}{0.675229in}}{\pgfqpoint{2.497167in}{0.675229in}}%
\pgfpathlineto{\pgfqpoint{2.497167in}{0.675229in}}%
\pgfpathclose%
\pgfusepath{stroke}%
\end{pgfscope}%
\begin{pgfscope}%
\pgfpathrectangle{\pgfqpoint{0.688192in}{0.670138in}}{\pgfqpoint{7.111808in}{5.129862in}}%
\pgfusepath{clip}%
\pgfsetbuttcap%
\pgfsetroundjoin%
\pgfsetlinewidth{1.003750pt}%
\definecolor{currentstroke}{rgb}{0.000000,0.000000,0.000000}%
\pgfsetstrokecolor{currentstroke}%
\pgfsetdash{}{0pt}%
\pgfpathmoveto{\pgfqpoint{1.351667in}{0.711476in}}%
\pgfpathcurveto{\pgfqpoint{1.362562in}{0.711476in}}{\pgfqpoint{1.373013in}{0.715805in}}{\pgfqpoint{1.380718in}{0.723510in}}%
\pgfpathcurveto{\pgfqpoint{1.388422in}{0.731214in}}{\pgfqpoint{1.392751in}{0.741665in}}{\pgfqpoint{1.392751in}{0.752560in}}%
\pgfpathcurveto{\pgfqpoint{1.392751in}{0.763456in}}{\pgfqpoint{1.388422in}{0.773907in}}{\pgfqpoint{1.380718in}{0.781611in}}%
\pgfpathcurveto{\pgfqpoint{1.373013in}{0.789315in}}{\pgfqpoint{1.362562in}{0.793644in}}{\pgfqpoint{1.351667in}{0.793644in}}%
\pgfpathcurveto{\pgfqpoint{1.340771in}{0.793644in}}{\pgfqpoint{1.330321in}{0.789315in}}{\pgfqpoint{1.322616in}{0.781611in}}%
\pgfpathcurveto{\pgfqpoint{1.314912in}{0.773907in}}{\pgfqpoint{1.310583in}{0.763456in}}{\pgfqpoint{1.310583in}{0.752560in}}%
\pgfpathcurveto{\pgfqpoint{1.310583in}{0.741665in}}{\pgfqpoint{1.314912in}{0.731214in}}{\pgfqpoint{1.322616in}{0.723510in}}%
\pgfpathcurveto{\pgfqpoint{1.330321in}{0.715805in}}{\pgfqpoint{1.340771in}{0.711476in}}{\pgfqpoint{1.351667in}{0.711476in}}%
\pgfpathlineto{\pgfqpoint{1.351667in}{0.711476in}}%
\pgfpathclose%
\pgfusepath{stroke}%
\end{pgfscope}%
\begin{pgfscope}%
\pgfpathrectangle{\pgfqpoint{0.688192in}{0.670138in}}{\pgfqpoint{7.111808in}{5.129862in}}%
\pgfusepath{clip}%
\pgfsetbuttcap%
\pgfsetroundjoin%
\pgfsetlinewidth{1.003750pt}%
\definecolor{currentstroke}{rgb}{0.000000,0.000000,0.000000}%
\pgfsetstrokecolor{currentstroke}%
\pgfsetdash{}{0pt}%
\pgfpathmoveto{\pgfqpoint{0.881277in}{0.900428in}}%
\pgfpathcurveto{\pgfqpoint{0.892173in}{0.900428in}}{\pgfqpoint{0.902624in}{0.904756in}}{\pgfqpoint{0.910328in}{0.912461in}}%
\pgfpathcurveto{\pgfqpoint{0.918032in}{0.920165in}}{\pgfqpoint{0.922361in}{0.930616in}}{\pgfqpoint{0.922361in}{0.941511in}}%
\pgfpathcurveto{\pgfqpoint{0.922361in}{0.952407in}}{\pgfqpoint{0.918032in}{0.962858in}}{\pgfqpoint{0.910328in}{0.970562in}}%
\pgfpathcurveto{\pgfqpoint{0.902624in}{0.978266in}}{\pgfqpoint{0.892173in}{0.982595in}}{\pgfqpoint{0.881277in}{0.982595in}}%
\pgfpathcurveto{\pgfqpoint{0.870382in}{0.982595in}}{\pgfqpoint{0.859931in}{0.978266in}}{\pgfqpoint{0.852226in}{0.970562in}}%
\pgfpathcurveto{\pgfqpoint{0.844522in}{0.962858in}}{\pgfqpoint{0.840193in}{0.952407in}}{\pgfqpoint{0.840193in}{0.941511in}}%
\pgfpathcurveto{\pgfqpoint{0.840193in}{0.930616in}}{\pgfqpoint{0.844522in}{0.920165in}}{\pgfqpoint{0.852226in}{0.912461in}}%
\pgfpathcurveto{\pgfqpoint{0.859931in}{0.904756in}}{\pgfqpoint{0.870382in}{0.900428in}}{\pgfqpoint{0.881277in}{0.900428in}}%
\pgfpathlineto{\pgfqpoint{0.881277in}{0.900428in}}%
\pgfpathclose%
\pgfusepath{stroke}%
\end{pgfscope}%
\begin{pgfscope}%
\pgfpathrectangle{\pgfqpoint{0.688192in}{0.670138in}}{\pgfqpoint{7.111808in}{5.129862in}}%
\pgfusepath{clip}%
\pgfsetbuttcap%
\pgfsetroundjoin%
\pgfsetlinewidth{1.003750pt}%
\definecolor{currentstroke}{rgb}{0.000000,0.000000,0.000000}%
\pgfsetstrokecolor{currentstroke}%
\pgfsetdash{}{0pt}%
\pgfpathmoveto{\pgfqpoint{4.462854in}{0.641380in}}%
\pgfpathcurveto{\pgfqpoint{4.473749in}{0.641380in}}{\pgfqpoint{4.484200in}{0.645709in}}{\pgfqpoint{4.491904in}{0.653414in}}%
\pgfpathcurveto{\pgfqpoint{4.499609in}{0.661118in}}{\pgfqpoint{4.503938in}{0.671569in}}{\pgfqpoint{4.503938in}{0.682464in}}%
\pgfpathcurveto{\pgfqpoint{4.503938in}{0.693360in}}{\pgfqpoint{4.499609in}{0.703811in}}{\pgfqpoint{4.491904in}{0.711515in}}%
\pgfpathcurveto{\pgfqpoint{4.484200in}{0.719219in}}{\pgfqpoint{4.473749in}{0.723548in}}{\pgfqpoint{4.462854in}{0.723548in}}%
\pgfpathcurveto{\pgfqpoint{4.451958in}{0.723548in}}{\pgfqpoint{4.441507in}{0.719219in}}{\pgfqpoint{4.433803in}{0.711515in}}%
\pgfpathcurveto{\pgfqpoint{4.426099in}{0.703811in}}{\pgfqpoint{4.421770in}{0.693360in}}{\pgfqpoint{4.421770in}{0.682464in}}%
\pgfpathcurveto{\pgfqpoint{4.421770in}{0.671569in}}{\pgfqpoint{4.426099in}{0.661118in}}{\pgfqpoint{4.433803in}{0.653414in}}%
\pgfpathcurveto{\pgfqpoint{4.441507in}{0.645709in}}{\pgfqpoint{4.451958in}{0.641380in}}{\pgfqpoint{4.462854in}{0.641380in}}%
\pgfusepath{stroke}%
\end{pgfscope}%
\begin{pgfscope}%
\pgfpathrectangle{\pgfqpoint{0.688192in}{0.670138in}}{\pgfqpoint{7.111808in}{5.129862in}}%
\pgfusepath{clip}%
\pgfsetbuttcap%
\pgfsetroundjoin%
\pgfsetlinewidth{1.003750pt}%
\definecolor{currentstroke}{rgb}{0.000000,0.000000,0.000000}%
\pgfsetstrokecolor{currentstroke}%
\pgfsetdash{}{0pt}%
\pgfpathmoveto{\pgfqpoint{1.050628in}{0.739811in}}%
\pgfpathcurveto{\pgfqpoint{1.061524in}{0.739811in}}{\pgfqpoint{1.071974in}{0.744140in}}{\pgfqpoint{1.079679in}{0.751844in}}%
\pgfpathcurveto{\pgfqpoint{1.087383in}{0.759549in}}{\pgfqpoint{1.091712in}{0.769999in}}{\pgfqpoint{1.091712in}{0.780895in}}%
\pgfpathcurveto{\pgfqpoint{1.091712in}{0.791790in}}{\pgfqpoint{1.087383in}{0.802241in}}{\pgfqpoint{1.079679in}{0.809946in}}%
\pgfpathcurveto{\pgfqpoint{1.071974in}{0.817650in}}{\pgfqpoint{1.061524in}{0.821979in}}{\pgfqpoint{1.050628in}{0.821979in}}%
\pgfpathcurveto{\pgfqpoint{1.039732in}{0.821979in}}{\pgfqpoint{1.029282in}{0.817650in}}{\pgfqpoint{1.021577in}{0.809946in}}%
\pgfpathcurveto{\pgfqpoint{1.013873in}{0.802241in}}{\pgfqpoint{1.009544in}{0.791790in}}{\pgfqpoint{1.009544in}{0.780895in}}%
\pgfpathcurveto{\pgfqpoint{1.009544in}{0.769999in}}{\pgfqpoint{1.013873in}{0.759549in}}{\pgfqpoint{1.021577in}{0.751844in}}%
\pgfpathcurveto{\pgfqpoint{1.029282in}{0.744140in}}{\pgfqpoint{1.039732in}{0.739811in}}{\pgfqpoint{1.050628in}{0.739811in}}%
\pgfpathlineto{\pgfqpoint{1.050628in}{0.739811in}}%
\pgfpathclose%
\pgfusepath{stroke}%
\end{pgfscope}%
\begin{pgfscope}%
\pgfpathrectangle{\pgfqpoint{0.688192in}{0.670138in}}{\pgfqpoint{7.111808in}{5.129862in}}%
\pgfusepath{clip}%
\pgfsetbuttcap%
\pgfsetroundjoin%
\pgfsetlinewidth{1.003750pt}%
\definecolor{currentstroke}{rgb}{0.000000,0.000000,0.000000}%
\pgfsetstrokecolor{currentstroke}%
\pgfsetdash{}{0pt}%
\pgfpathmoveto{\pgfqpoint{0.885952in}{0.873580in}}%
\pgfpathcurveto{\pgfqpoint{0.896848in}{0.873580in}}{\pgfqpoint{0.907298in}{0.877909in}}{\pgfqpoint{0.915003in}{0.885613in}}%
\pgfpathcurveto{\pgfqpoint{0.922707in}{0.893317in}}{\pgfqpoint{0.927036in}{0.903768in}}{\pgfqpoint{0.927036in}{0.914664in}}%
\pgfpathcurveto{\pgfqpoint{0.927036in}{0.925559in}}{\pgfqpoint{0.922707in}{0.936010in}}{\pgfqpoint{0.915003in}{0.943714in}}%
\pgfpathcurveto{\pgfqpoint{0.907298in}{0.951419in}}{\pgfqpoint{0.896848in}{0.955748in}}{\pgfqpoint{0.885952in}{0.955748in}}%
\pgfpathcurveto{\pgfqpoint{0.875056in}{0.955748in}}{\pgfqpoint{0.864606in}{0.951419in}}{\pgfqpoint{0.856901in}{0.943714in}}%
\pgfpathcurveto{\pgfqpoint{0.849197in}{0.936010in}}{\pgfqpoint{0.844868in}{0.925559in}}{\pgfqpoint{0.844868in}{0.914664in}}%
\pgfpathcurveto{\pgfqpoint{0.844868in}{0.903768in}}{\pgfqpoint{0.849197in}{0.893317in}}{\pgfqpoint{0.856901in}{0.885613in}}%
\pgfpathcurveto{\pgfqpoint{0.864606in}{0.877909in}}{\pgfqpoint{0.875056in}{0.873580in}}{\pgfqpoint{0.885952in}{0.873580in}}%
\pgfpathlineto{\pgfqpoint{0.885952in}{0.873580in}}%
\pgfpathclose%
\pgfusepath{stroke}%
\end{pgfscope}%
\begin{pgfscope}%
\pgfpathrectangle{\pgfqpoint{0.688192in}{0.670138in}}{\pgfqpoint{7.111808in}{5.129862in}}%
\pgfusepath{clip}%
\pgfsetbuttcap%
\pgfsetroundjoin%
\pgfsetlinewidth{1.003750pt}%
\definecolor{currentstroke}{rgb}{0.000000,0.000000,0.000000}%
\pgfsetstrokecolor{currentstroke}%
\pgfsetdash{}{0pt}%
\pgfpathmoveto{\pgfqpoint{0.782559in}{1.071983in}}%
\pgfpathcurveto{\pgfqpoint{0.793455in}{1.071983in}}{\pgfqpoint{0.803905in}{1.076312in}}{\pgfqpoint{0.811610in}{1.084017in}}%
\pgfpathcurveto{\pgfqpoint{0.819314in}{1.091721in}}{\pgfqpoint{0.823643in}{1.102172in}}{\pgfqpoint{0.823643in}{1.113067in}}%
\pgfpathcurveto{\pgfqpoint{0.823643in}{1.123963in}}{\pgfqpoint{0.819314in}{1.134414in}}{\pgfqpoint{0.811610in}{1.142118in}}%
\pgfpathcurveto{\pgfqpoint{0.803905in}{1.149822in}}{\pgfqpoint{0.793455in}{1.154151in}}{\pgfqpoint{0.782559in}{1.154151in}}%
\pgfpathcurveto{\pgfqpoint{0.771663in}{1.154151in}}{\pgfqpoint{0.761213in}{1.149822in}}{\pgfqpoint{0.753508in}{1.142118in}}%
\pgfpathcurveto{\pgfqpoint{0.745804in}{1.134414in}}{\pgfqpoint{0.741475in}{1.123963in}}{\pgfqpoint{0.741475in}{1.113067in}}%
\pgfpathcurveto{\pgfqpoint{0.741475in}{1.102172in}}{\pgfqpoint{0.745804in}{1.091721in}}{\pgfqpoint{0.753508in}{1.084017in}}%
\pgfpathcurveto{\pgfqpoint{0.761213in}{1.076312in}}{\pgfqpoint{0.771663in}{1.071983in}}{\pgfqpoint{0.782559in}{1.071983in}}%
\pgfpathlineto{\pgfqpoint{0.782559in}{1.071983in}}%
\pgfpathclose%
\pgfusepath{stroke}%
\end{pgfscope}%
\begin{pgfscope}%
\pgfpathrectangle{\pgfqpoint{0.688192in}{0.670138in}}{\pgfqpoint{7.111808in}{5.129862in}}%
\pgfusepath{clip}%
\pgfsetbuttcap%
\pgfsetroundjoin%
\pgfsetlinewidth{1.003750pt}%
\definecolor{currentstroke}{rgb}{0.000000,0.000000,0.000000}%
\pgfsetstrokecolor{currentstroke}%
\pgfsetdash{}{0pt}%
\pgfpathmoveto{\pgfqpoint{1.027669in}{0.756388in}}%
\pgfpathcurveto{\pgfqpoint{1.038564in}{0.756388in}}{\pgfqpoint{1.049015in}{0.760717in}}{\pgfqpoint{1.056719in}{0.768422in}}%
\pgfpathcurveto{\pgfqpoint{1.064424in}{0.776126in}}{\pgfqpoint{1.068753in}{0.786577in}}{\pgfqpoint{1.068753in}{0.797472in}}%
\pgfpathcurveto{\pgfqpoint{1.068753in}{0.808368in}}{\pgfqpoint{1.064424in}{0.818819in}}{\pgfqpoint{1.056719in}{0.826523in}}%
\pgfpathcurveto{\pgfqpoint{1.049015in}{0.834227in}}{\pgfqpoint{1.038564in}{0.838556in}}{\pgfqpoint{1.027669in}{0.838556in}}%
\pgfpathcurveto{\pgfqpoint{1.016773in}{0.838556in}}{\pgfqpoint{1.006322in}{0.834227in}}{\pgfqpoint{0.998618in}{0.826523in}}%
\pgfpathcurveto{\pgfqpoint{0.990914in}{0.818819in}}{\pgfqpoint{0.986585in}{0.808368in}}{\pgfqpoint{0.986585in}{0.797472in}}%
\pgfpathcurveto{\pgfqpoint{0.986585in}{0.786577in}}{\pgfqpoint{0.990914in}{0.776126in}}{\pgfqpoint{0.998618in}{0.768422in}}%
\pgfpathcurveto{\pgfqpoint{1.006322in}{0.760717in}}{\pgfqpoint{1.016773in}{0.756388in}}{\pgfqpoint{1.027669in}{0.756388in}}%
\pgfpathlineto{\pgfqpoint{1.027669in}{0.756388in}}%
\pgfpathclose%
\pgfusepath{stroke}%
\end{pgfscope}%
\begin{pgfscope}%
\pgfpathrectangle{\pgfqpoint{0.688192in}{0.670138in}}{\pgfqpoint{7.111808in}{5.129862in}}%
\pgfusepath{clip}%
\pgfsetbuttcap%
\pgfsetroundjoin%
\pgfsetlinewidth{1.003750pt}%
\definecolor{currentstroke}{rgb}{0.000000,0.000000,0.000000}%
\pgfsetstrokecolor{currentstroke}%
\pgfsetdash{}{0pt}%
\pgfpathmoveto{\pgfqpoint{5.696293in}{0.630769in}}%
\pgfpathcurveto{\pgfqpoint{5.707189in}{0.630769in}}{\pgfqpoint{5.717639in}{0.635098in}}{\pgfqpoint{5.725344in}{0.642802in}}%
\pgfpathcurveto{\pgfqpoint{5.733048in}{0.650507in}}{\pgfqpoint{5.737377in}{0.660958in}}{\pgfqpoint{5.737377in}{0.671853in}}%
\pgfpathcurveto{\pgfqpoint{5.737377in}{0.682749in}}{\pgfqpoint{5.733048in}{0.693200in}}{\pgfqpoint{5.725344in}{0.700904in}}%
\pgfpathcurveto{\pgfqpoint{5.717639in}{0.708608in}}{\pgfqpoint{5.707189in}{0.712937in}}{\pgfqpoint{5.696293in}{0.712937in}}%
\pgfpathcurveto{\pgfqpoint{5.685397in}{0.712937in}}{\pgfqpoint{5.674947in}{0.708608in}}{\pgfqpoint{5.667242in}{0.700904in}}%
\pgfpathcurveto{\pgfqpoint{5.659538in}{0.693200in}}{\pgfqpoint{5.655209in}{0.682749in}}{\pgfqpoint{5.655209in}{0.671853in}}%
\pgfpathcurveto{\pgfqpoint{5.655209in}{0.660958in}}{\pgfqpoint{5.659538in}{0.650507in}}{\pgfqpoint{5.667242in}{0.642802in}}%
\pgfpathcurveto{\pgfqpoint{5.674947in}{0.635098in}}{\pgfqpoint{5.685397in}{0.630769in}}{\pgfqpoint{5.696293in}{0.630769in}}%
\pgfusepath{stroke}%
\end{pgfscope}%
\begin{pgfscope}%
\pgfpathrectangle{\pgfqpoint{0.688192in}{0.670138in}}{\pgfqpoint{7.111808in}{5.129862in}}%
\pgfusepath{clip}%
\pgfsetbuttcap%
\pgfsetroundjoin%
\pgfsetlinewidth{1.003750pt}%
\definecolor{currentstroke}{rgb}{0.000000,0.000000,0.000000}%
\pgfsetstrokecolor{currentstroke}%
\pgfsetdash{}{0pt}%
\pgfpathmoveto{\pgfqpoint{1.713082in}{0.698892in}}%
\pgfpathcurveto{\pgfqpoint{1.723978in}{0.698892in}}{\pgfqpoint{1.734428in}{0.703221in}}{\pgfqpoint{1.742133in}{0.710926in}}%
\pgfpathcurveto{\pgfqpoint{1.749837in}{0.718630in}}{\pgfqpoint{1.754166in}{0.729081in}}{\pgfqpoint{1.754166in}{0.739976in}}%
\pgfpathcurveto{\pgfqpoint{1.754166in}{0.750872in}}{\pgfqpoint{1.749837in}{0.761323in}}{\pgfqpoint{1.742133in}{0.769027in}}%
\pgfpathcurveto{\pgfqpoint{1.734428in}{0.776731in}}{\pgfqpoint{1.723978in}{0.781060in}}{\pgfqpoint{1.713082in}{0.781060in}}%
\pgfpathcurveto{\pgfqpoint{1.702186in}{0.781060in}}{\pgfqpoint{1.691736in}{0.776731in}}{\pgfqpoint{1.684031in}{0.769027in}}%
\pgfpathcurveto{\pgfqpoint{1.676327in}{0.761323in}}{\pgfqpoint{1.671998in}{0.750872in}}{\pgfqpoint{1.671998in}{0.739976in}}%
\pgfpathcurveto{\pgfqpoint{1.671998in}{0.729081in}}{\pgfqpoint{1.676327in}{0.718630in}}{\pgfqpoint{1.684031in}{0.710926in}}%
\pgfpathcurveto{\pgfqpoint{1.691736in}{0.703221in}}{\pgfqpoint{1.702186in}{0.698892in}}{\pgfqpoint{1.713082in}{0.698892in}}%
\pgfpathlineto{\pgfqpoint{1.713082in}{0.698892in}}%
\pgfpathclose%
\pgfusepath{stroke}%
\end{pgfscope}%
\begin{pgfscope}%
\pgfpathrectangle{\pgfqpoint{0.688192in}{0.670138in}}{\pgfqpoint{7.111808in}{5.129862in}}%
\pgfusepath{clip}%
\pgfsetbuttcap%
\pgfsetroundjoin%
\pgfsetlinewidth{1.003750pt}%
\definecolor{currentstroke}{rgb}{0.000000,0.000000,0.000000}%
\pgfsetstrokecolor{currentstroke}%
\pgfsetdash{}{0pt}%
\pgfpathmoveto{\pgfqpoint{5.802873in}{1.509242in}}%
\pgfpathcurveto{\pgfqpoint{5.813768in}{1.509242in}}{\pgfqpoint{5.824219in}{1.513571in}}{\pgfqpoint{5.831923in}{1.521275in}}%
\pgfpathcurveto{\pgfqpoint{5.839628in}{1.528979in}}{\pgfqpoint{5.843956in}{1.539430in}}{\pgfqpoint{5.843956in}{1.550326in}}%
\pgfpathcurveto{\pgfqpoint{5.843956in}{1.561221in}}{\pgfqpoint{5.839628in}{1.571672in}}{\pgfqpoint{5.831923in}{1.579376in}}%
\pgfpathcurveto{\pgfqpoint{5.824219in}{1.587081in}}{\pgfqpoint{5.813768in}{1.591410in}}{\pgfqpoint{5.802873in}{1.591410in}}%
\pgfpathcurveto{\pgfqpoint{5.791977in}{1.591410in}}{\pgfqpoint{5.781526in}{1.587081in}}{\pgfqpoint{5.773822in}{1.579376in}}%
\pgfpathcurveto{\pgfqpoint{5.766118in}{1.571672in}}{\pgfqpoint{5.761789in}{1.561221in}}{\pgfqpoint{5.761789in}{1.550326in}}%
\pgfpathcurveto{\pgfqpoint{5.761789in}{1.539430in}}{\pgfqpoint{5.766118in}{1.528979in}}{\pgfqpoint{5.773822in}{1.521275in}}%
\pgfpathcurveto{\pgfqpoint{5.781526in}{1.513571in}}{\pgfqpoint{5.791977in}{1.509242in}}{\pgfqpoint{5.802873in}{1.509242in}}%
\pgfpathlineto{\pgfqpoint{5.802873in}{1.509242in}}%
\pgfpathclose%
\pgfusepath{stroke}%
\end{pgfscope}%
\begin{pgfscope}%
\pgfpathrectangle{\pgfqpoint{0.688192in}{0.670138in}}{\pgfqpoint{7.111808in}{5.129862in}}%
\pgfusepath{clip}%
\pgfsetbuttcap%
\pgfsetroundjoin%
\pgfsetlinewidth{1.003750pt}%
\definecolor{currentstroke}{rgb}{0.000000,0.000000,0.000000}%
\pgfsetstrokecolor{currentstroke}%
\pgfsetdash{}{0pt}%
\pgfpathmoveto{\pgfqpoint{0.715926in}{2.236095in}}%
\pgfpathcurveto{\pgfqpoint{0.726821in}{2.236095in}}{\pgfqpoint{0.737272in}{2.240424in}}{\pgfqpoint{0.744976in}{2.248129in}}%
\pgfpathcurveto{\pgfqpoint{0.752681in}{2.255833in}}{\pgfqpoint{0.757010in}{2.266284in}}{\pgfqpoint{0.757010in}{2.277179in}}%
\pgfpathcurveto{\pgfqpoint{0.757010in}{2.288075in}}{\pgfqpoint{0.752681in}{2.298526in}}{\pgfqpoint{0.744976in}{2.306230in}}%
\pgfpathcurveto{\pgfqpoint{0.737272in}{2.313934in}}{\pgfqpoint{0.726821in}{2.318263in}}{\pgfqpoint{0.715926in}{2.318263in}}%
\pgfpathcurveto{\pgfqpoint{0.705030in}{2.318263in}}{\pgfqpoint{0.694579in}{2.313934in}}{\pgfqpoint{0.686875in}{2.306230in}}%
\pgfpathcurveto{\pgfqpoint{0.679171in}{2.298526in}}{\pgfqpoint{0.674842in}{2.288075in}}{\pgfqpoint{0.674842in}{2.277179in}}%
\pgfpathcurveto{\pgfqpoint{0.674842in}{2.266284in}}{\pgfqpoint{0.679171in}{2.255833in}}{\pgfqpoint{0.686875in}{2.248129in}}%
\pgfpathcurveto{\pgfqpoint{0.694579in}{2.240424in}}{\pgfqpoint{0.705030in}{2.236095in}}{\pgfqpoint{0.715926in}{2.236095in}}%
\pgfpathlineto{\pgfqpoint{0.715926in}{2.236095in}}%
\pgfpathclose%
\pgfusepath{stroke}%
\end{pgfscope}%
\begin{pgfscope}%
\pgfpathrectangle{\pgfqpoint{0.688192in}{0.670138in}}{\pgfqpoint{7.111808in}{5.129862in}}%
\pgfusepath{clip}%
\pgfsetbuttcap%
\pgfsetroundjoin%
\pgfsetlinewidth{1.003750pt}%
\definecolor{currentstroke}{rgb}{0.000000,0.000000,0.000000}%
\pgfsetstrokecolor{currentstroke}%
\pgfsetdash{}{0pt}%
\pgfpathmoveto{\pgfqpoint{0.886403in}{0.869810in}}%
\pgfpathcurveto{\pgfqpoint{0.897298in}{0.869810in}}{\pgfqpoint{0.907749in}{0.874139in}}{\pgfqpoint{0.915453in}{0.881844in}}%
\pgfpathcurveto{\pgfqpoint{0.923158in}{0.889548in}}{\pgfqpoint{0.927487in}{0.899999in}}{\pgfqpoint{0.927487in}{0.910894in}}%
\pgfpathcurveto{\pgfqpoint{0.927487in}{0.921790in}}{\pgfqpoint{0.923158in}{0.932241in}}{\pgfqpoint{0.915453in}{0.939945in}}%
\pgfpathcurveto{\pgfqpoint{0.907749in}{0.947649in}}{\pgfqpoint{0.897298in}{0.951978in}}{\pgfqpoint{0.886403in}{0.951978in}}%
\pgfpathcurveto{\pgfqpoint{0.875507in}{0.951978in}}{\pgfqpoint{0.865056in}{0.947649in}}{\pgfqpoint{0.857352in}{0.939945in}}%
\pgfpathcurveto{\pgfqpoint{0.849648in}{0.932241in}}{\pgfqpoint{0.845319in}{0.921790in}}{\pgfqpoint{0.845319in}{0.910894in}}%
\pgfpathcurveto{\pgfqpoint{0.845319in}{0.899999in}}{\pgfqpoint{0.849648in}{0.889548in}}{\pgfqpoint{0.857352in}{0.881844in}}%
\pgfpathcurveto{\pgfqpoint{0.865056in}{0.874139in}}{\pgfqpoint{0.875507in}{0.869810in}}{\pgfqpoint{0.886403in}{0.869810in}}%
\pgfpathlineto{\pgfqpoint{0.886403in}{0.869810in}}%
\pgfpathclose%
\pgfusepath{stroke}%
\end{pgfscope}%
\begin{pgfscope}%
\pgfpathrectangle{\pgfqpoint{0.688192in}{0.670138in}}{\pgfqpoint{7.111808in}{5.129862in}}%
\pgfusepath{clip}%
\pgfsetbuttcap%
\pgfsetroundjoin%
\pgfsetlinewidth{1.003750pt}%
\definecolor{currentstroke}{rgb}{0.000000,0.000000,0.000000}%
\pgfsetstrokecolor{currentstroke}%
\pgfsetdash{}{0pt}%
\pgfpathmoveto{\pgfqpoint{1.120660in}{0.719620in}}%
\pgfpathcurveto{\pgfqpoint{1.131555in}{0.719620in}}{\pgfqpoint{1.142006in}{0.723949in}}{\pgfqpoint{1.149710in}{0.731653in}}%
\pgfpathcurveto{\pgfqpoint{1.157415in}{0.739358in}}{\pgfqpoint{1.161744in}{0.749809in}}{\pgfqpoint{1.161744in}{0.760704in}}%
\pgfpathcurveto{\pgfqpoint{1.161744in}{0.771600in}}{\pgfqpoint{1.157415in}{0.782051in}}{\pgfqpoint{1.149710in}{0.789755in}}%
\pgfpathcurveto{\pgfqpoint{1.142006in}{0.797459in}}{\pgfqpoint{1.131555in}{0.801788in}}{\pgfqpoint{1.120660in}{0.801788in}}%
\pgfpathcurveto{\pgfqpoint{1.109764in}{0.801788in}}{\pgfqpoint{1.099313in}{0.797459in}}{\pgfqpoint{1.091609in}{0.789755in}}%
\pgfpathcurveto{\pgfqpoint{1.083905in}{0.782051in}}{\pgfqpoint{1.079576in}{0.771600in}}{\pgfqpoint{1.079576in}{0.760704in}}%
\pgfpathcurveto{\pgfqpoint{1.079576in}{0.749809in}}{\pgfqpoint{1.083905in}{0.739358in}}{\pgfqpoint{1.091609in}{0.731653in}}%
\pgfpathcurveto{\pgfqpoint{1.099313in}{0.723949in}}{\pgfqpoint{1.109764in}{0.719620in}}{\pgfqpoint{1.120660in}{0.719620in}}%
\pgfpathlineto{\pgfqpoint{1.120660in}{0.719620in}}%
\pgfpathclose%
\pgfusepath{stroke}%
\end{pgfscope}%
\begin{pgfscope}%
\pgfpathrectangle{\pgfqpoint{0.688192in}{0.670138in}}{\pgfqpoint{7.111808in}{5.129862in}}%
\pgfusepath{clip}%
\pgfsetbuttcap%
\pgfsetroundjoin%
\pgfsetlinewidth{1.003750pt}%
\definecolor{currentstroke}{rgb}{0.000000,0.000000,0.000000}%
\pgfsetstrokecolor{currentstroke}%
\pgfsetdash{}{0pt}%
\pgfpathmoveto{\pgfqpoint{1.852165in}{0.694984in}}%
\pgfpathcurveto{\pgfqpoint{1.863060in}{0.694984in}}{\pgfqpoint{1.873511in}{0.699313in}}{\pgfqpoint{1.881215in}{0.707017in}}%
\pgfpathcurveto{\pgfqpoint{1.888920in}{0.714722in}}{\pgfqpoint{1.893248in}{0.725173in}}{\pgfqpoint{1.893248in}{0.736068in}}%
\pgfpathcurveto{\pgfqpoint{1.893248in}{0.746964in}}{\pgfqpoint{1.888920in}{0.757415in}}{\pgfqpoint{1.881215in}{0.765119in}}%
\pgfpathcurveto{\pgfqpoint{1.873511in}{0.772823in}}{\pgfqpoint{1.863060in}{0.777152in}}{\pgfqpoint{1.852165in}{0.777152in}}%
\pgfpathcurveto{\pgfqpoint{1.841269in}{0.777152in}}{\pgfqpoint{1.830818in}{0.772823in}}{\pgfqpoint{1.823114in}{0.765119in}}%
\pgfpathcurveto{\pgfqpoint{1.815410in}{0.757415in}}{\pgfqpoint{1.811081in}{0.746964in}}{\pgfqpoint{1.811081in}{0.736068in}}%
\pgfpathcurveto{\pgfqpoint{1.811081in}{0.725173in}}{\pgfqpoint{1.815410in}{0.714722in}}{\pgfqpoint{1.823114in}{0.707017in}}%
\pgfpathcurveto{\pgfqpoint{1.830818in}{0.699313in}}{\pgfqpoint{1.841269in}{0.694984in}}{\pgfqpoint{1.852165in}{0.694984in}}%
\pgfpathlineto{\pgfqpoint{1.852165in}{0.694984in}}%
\pgfpathclose%
\pgfusepath{stroke}%
\end{pgfscope}%
\begin{pgfscope}%
\pgfpathrectangle{\pgfqpoint{0.688192in}{0.670138in}}{\pgfqpoint{7.111808in}{5.129862in}}%
\pgfusepath{clip}%
\pgfsetbuttcap%
\pgfsetroundjoin%
\pgfsetlinewidth{1.003750pt}%
\definecolor{currentstroke}{rgb}{0.000000,0.000000,0.000000}%
\pgfsetstrokecolor{currentstroke}%
\pgfsetdash{}{0pt}%
\pgfpathmoveto{\pgfqpoint{5.927705in}{1.360343in}}%
\pgfpathcurveto{\pgfqpoint{5.938601in}{1.360343in}}{\pgfqpoint{5.949052in}{1.364672in}}{\pgfqpoint{5.956756in}{1.372376in}}%
\pgfpathcurveto{\pgfqpoint{5.964460in}{1.380080in}}{\pgfqpoint{5.968789in}{1.390531in}}{\pgfqpoint{5.968789in}{1.401427in}}%
\pgfpathcurveto{\pgfqpoint{5.968789in}{1.412322in}}{\pgfqpoint{5.964460in}{1.422773in}}{\pgfqpoint{5.956756in}{1.430477in}}%
\pgfpathcurveto{\pgfqpoint{5.949052in}{1.438182in}}{\pgfqpoint{5.938601in}{1.442511in}}{\pgfqpoint{5.927705in}{1.442511in}}%
\pgfpathcurveto{\pgfqpoint{5.916810in}{1.442511in}}{\pgfqpoint{5.906359in}{1.438182in}}{\pgfqpoint{5.898654in}{1.430477in}}%
\pgfpathcurveto{\pgfqpoint{5.890950in}{1.422773in}}{\pgfqpoint{5.886621in}{1.412322in}}{\pgfqpoint{5.886621in}{1.401427in}}%
\pgfpathcurveto{\pgfqpoint{5.886621in}{1.390531in}}{\pgfqpoint{5.890950in}{1.380080in}}{\pgfqpoint{5.898654in}{1.372376in}}%
\pgfpathcurveto{\pgfqpoint{5.906359in}{1.364672in}}{\pgfqpoint{5.916810in}{1.360343in}}{\pgfqpoint{5.927705in}{1.360343in}}%
\pgfpathlineto{\pgfqpoint{5.927705in}{1.360343in}}%
\pgfpathclose%
\pgfusepath{stroke}%
\end{pgfscope}%
\begin{pgfscope}%
\pgfpathrectangle{\pgfqpoint{0.688192in}{0.670138in}}{\pgfqpoint{7.111808in}{5.129862in}}%
\pgfusepath{clip}%
\pgfsetbuttcap%
\pgfsetroundjoin%
\pgfsetlinewidth{1.003750pt}%
\definecolor{currentstroke}{rgb}{0.000000,0.000000,0.000000}%
\pgfsetstrokecolor{currentstroke}%
\pgfsetdash{}{0pt}%
\pgfpathmoveto{\pgfqpoint{1.050667in}{0.739240in}}%
\pgfpathcurveto{\pgfqpoint{1.061563in}{0.739240in}}{\pgfqpoint{1.072014in}{0.743568in}}{\pgfqpoint{1.079718in}{0.751273in}}%
\pgfpathcurveto{\pgfqpoint{1.087422in}{0.758977in}}{\pgfqpoint{1.091751in}{0.769428in}}{\pgfqpoint{1.091751in}{0.780323in}}%
\pgfpathcurveto{\pgfqpoint{1.091751in}{0.791219in}}{\pgfqpoint{1.087422in}{0.801670in}}{\pgfqpoint{1.079718in}{0.809374in}}%
\pgfpathcurveto{\pgfqpoint{1.072014in}{0.817078in}}{\pgfqpoint{1.061563in}{0.821407in}}{\pgfqpoint{1.050667in}{0.821407in}}%
\pgfpathcurveto{\pgfqpoint{1.039772in}{0.821407in}}{\pgfqpoint{1.029321in}{0.817078in}}{\pgfqpoint{1.021617in}{0.809374in}}%
\pgfpathcurveto{\pgfqpoint{1.013912in}{0.801670in}}{\pgfqpoint{1.009583in}{0.791219in}}{\pgfqpoint{1.009583in}{0.780323in}}%
\pgfpathcurveto{\pgfqpoint{1.009583in}{0.769428in}}{\pgfqpoint{1.013912in}{0.758977in}}{\pgfqpoint{1.021617in}{0.751273in}}%
\pgfpathcurveto{\pgfqpoint{1.029321in}{0.743568in}}{\pgfqpoint{1.039772in}{0.739240in}}{\pgfqpoint{1.050667in}{0.739240in}}%
\pgfpathlineto{\pgfqpoint{1.050667in}{0.739240in}}%
\pgfpathclose%
\pgfusepath{stroke}%
\end{pgfscope}%
\begin{pgfscope}%
\pgfpathrectangle{\pgfqpoint{0.688192in}{0.670138in}}{\pgfqpoint{7.111808in}{5.129862in}}%
\pgfusepath{clip}%
\pgfsetbuttcap%
\pgfsetroundjoin%
\pgfsetlinewidth{1.003750pt}%
\definecolor{currentstroke}{rgb}{0.000000,0.000000,0.000000}%
\pgfsetstrokecolor{currentstroke}%
\pgfsetdash{}{0pt}%
\pgfpathmoveto{\pgfqpoint{5.988842in}{0.629054in}}%
\pgfpathcurveto{\pgfqpoint{5.999738in}{0.629054in}}{\pgfqpoint{6.010189in}{0.633383in}}{\pgfqpoint{6.017893in}{0.641087in}}%
\pgfpathcurveto{\pgfqpoint{6.025598in}{0.648792in}}{\pgfqpoint{6.029926in}{0.659242in}}{\pgfqpoint{6.029926in}{0.670138in}}%
\pgfpathcurveto{\pgfqpoint{6.029926in}{0.681034in}}{\pgfqpoint{6.025598in}{0.691484in}}{\pgfqpoint{6.017893in}{0.699189in}}%
\pgfpathcurveto{\pgfqpoint{6.010189in}{0.706893in}}{\pgfqpoint{5.999738in}{0.711222in}}{\pgfqpoint{5.988842in}{0.711222in}}%
\pgfpathcurveto{\pgfqpoint{5.977947in}{0.711222in}}{\pgfqpoint{5.967496in}{0.706893in}}{\pgfqpoint{5.959792in}{0.699189in}}%
\pgfpathcurveto{\pgfqpoint{5.952087in}{0.691484in}}{\pgfqpoint{5.947759in}{0.681034in}}{\pgfqpoint{5.947759in}{0.670138in}}%
\pgfpathcurveto{\pgfqpoint{5.947759in}{0.659242in}}{\pgfqpoint{5.952087in}{0.648792in}}{\pgfqpoint{5.959792in}{0.641087in}}%
\pgfpathcurveto{\pgfqpoint{5.967496in}{0.633383in}}{\pgfqpoint{5.977947in}{0.629054in}}{\pgfqpoint{5.988842in}{0.629054in}}%
\pgfusepath{stroke}%
\end{pgfscope}%
\begin{pgfscope}%
\pgfpathrectangle{\pgfqpoint{0.688192in}{0.670138in}}{\pgfqpoint{7.111808in}{5.129862in}}%
\pgfusepath{clip}%
\pgfsetbuttcap%
\pgfsetroundjoin%
\pgfsetlinewidth{1.003750pt}%
\definecolor{currentstroke}{rgb}{0.000000,0.000000,0.000000}%
\pgfsetstrokecolor{currentstroke}%
\pgfsetdash{}{0pt}%
\pgfpathmoveto{\pgfqpoint{4.920836in}{0.872575in}}%
\pgfpathcurveto{\pgfqpoint{4.931732in}{0.872575in}}{\pgfqpoint{4.942183in}{0.876904in}}{\pgfqpoint{4.949887in}{0.884609in}}%
\pgfpathcurveto{\pgfqpoint{4.957591in}{0.892313in}}{\pgfqpoint{4.961920in}{0.902764in}}{\pgfqpoint{4.961920in}{0.913659in}}%
\pgfpathcurveto{\pgfqpoint{4.961920in}{0.924555in}}{\pgfqpoint{4.957591in}{0.935006in}}{\pgfqpoint{4.949887in}{0.942710in}}%
\pgfpathcurveto{\pgfqpoint{4.942183in}{0.950414in}}{\pgfqpoint{4.931732in}{0.954743in}}{\pgfqpoint{4.920836in}{0.954743in}}%
\pgfpathcurveto{\pgfqpoint{4.909941in}{0.954743in}}{\pgfqpoint{4.899490in}{0.950414in}}{\pgfqpoint{4.891786in}{0.942710in}}%
\pgfpathcurveto{\pgfqpoint{4.884081in}{0.935006in}}{\pgfqpoint{4.879753in}{0.924555in}}{\pgfqpoint{4.879753in}{0.913659in}}%
\pgfpathcurveto{\pgfqpoint{4.879753in}{0.902764in}}{\pgfqpoint{4.884081in}{0.892313in}}{\pgfqpoint{4.891786in}{0.884609in}}%
\pgfpathcurveto{\pgfqpoint{4.899490in}{0.876904in}}{\pgfqpoint{4.909941in}{0.872575in}}{\pgfqpoint{4.920836in}{0.872575in}}%
\pgfpathlineto{\pgfqpoint{4.920836in}{0.872575in}}%
\pgfpathclose%
\pgfusepath{stroke}%
\end{pgfscope}%
\begin{pgfscope}%
\pgfpathrectangle{\pgfqpoint{0.688192in}{0.670138in}}{\pgfqpoint{7.111808in}{5.129862in}}%
\pgfusepath{clip}%
\pgfsetbuttcap%
\pgfsetroundjoin%
\pgfsetlinewidth{1.003750pt}%
\definecolor{currentstroke}{rgb}{0.000000,0.000000,0.000000}%
\pgfsetstrokecolor{currentstroke}%
\pgfsetdash{}{0pt}%
\pgfpathmoveto{\pgfqpoint{1.685511in}{0.701118in}}%
\pgfpathcurveto{\pgfqpoint{1.696407in}{0.701118in}}{\pgfqpoint{1.706857in}{0.705447in}}{\pgfqpoint{1.714562in}{0.713151in}}%
\pgfpathcurveto{\pgfqpoint{1.722266in}{0.720855in}}{\pgfqpoint{1.726595in}{0.731306in}}{\pgfqpoint{1.726595in}{0.742202in}}%
\pgfpathcurveto{\pgfqpoint{1.726595in}{0.753097in}}{\pgfqpoint{1.722266in}{0.763548in}}{\pgfqpoint{1.714562in}{0.771252in}}%
\pgfpathcurveto{\pgfqpoint{1.706857in}{0.778957in}}{\pgfqpoint{1.696407in}{0.783286in}}{\pgfqpoint{1.685511in}{0.783286in}}%
\pgfpathcurveto{\pgfqpoint{1.674615in}{0.783286in}}{\pgfqpoint{1.664165in}{0.778957in}}{\pgfqpoint{1.656460in}{0.771252in}}%
\pgfpathcurveto{\pgfqpoint{1.648756in}{0.763548in}}{\pgfqpoint{1.644427in}{0.753097in}}{\pgfqpoint{1.644427in}{0.742202in}}%
\pgfpathcurveto{\pgfqpoint{1.644427in}{0.731306in}}{\pgfqpoint{1.648756in}{0.720855in}}{\pgfqpoint{1.656460in}{0.713151in}}%
\pgfpathcurveto{\pgfqpoint{1.664165in}{0.705447in}}{\pgfqpoint{1.674615in}{0.701118in}}{\pgfqpoint{1.685511in}{0.701118in}}%
\pgfpathlineto{\pgfqpoint{1.685511in}{0.701118in}}%
\pgfpathclose%
\pgfusepath{stroke}%
\end{pgfscope}%
\begin{pgfscope}%
\pgfpathrectangle{\pgfqpoint{0.688192in}{0.670138in}}{\pgfqpoint{7.111808in}{5.129862in}}%
\pgfusepath{clip}%
\pgfsetbuttcap%
\pgfsetroundjoin%
\pgfsetlinewidth{1.003750pt}%
\definecolor{currentstroke}{rgb}{0.000000,0.000000,0.000000}%
\pgfsetstrokecolor{currentstroke}%
\pgfsetdash{}{0pt}%
\pgfpathmoveto{\pgfqpoint{1.019140in}{0.759981in}}%
\pgfpathcurveto{\pgfqpoint{1.030035in}{0.759981in}}{\pgfqpoint{1.040486in}{0.764309in}}{\pgfqpoint{1.048191in}{0.772014in}}%
\pgfpathcurveto{\pgfqpoint{1.055895in}{0.779718in}}{\pgfqpoint{1.060224in}{0.790169in}}{\pgfqpoint{1.060224in}{0.801064in}}%
\pgfpathcurveto{\pgfqpoint{1.060224in}{0.811960in}}{\pgfqpoint{1.055895in}{0.822411in}}{\pgfqpoint{1.048191in}{0.830115in}}%
\pgfpathcurveto{\pgfqpoint{1.040486in}{0.837820in}}{\pgfqpoint{1.030035in}{0.842148in}}{\pgfqpoint{1.019140in}{0.842148in}}%
\pgfpathcurveto{\pgfqpoint{1.008244in}{0.842148in}}{\pgfqpoint{0.997794in}{0.837820in}}{\pgfqpoint{0.990089in}{0.830115in}}%
\pgfpathcurveto{\pgfqpoint{0.982385in}{0.822411in}}{\pgfqpoint{0.978056in}{0.811960in}}{\pgfqpoint{0.978056in}{0.801064in}}%
\pgfpathcurveto{\pgfqpoint{0.978056in}{0.790169in}}{\pgfqpoint{0.982385in}{0.779718in}}{\pgfqpoint{0.990089in}{0.772014in}}%
\pgfpathcurveto{\pgfqpoint{0.997794in}{0.764309in}}{\pgfqpoint{1.008244in}{0.759981in}}{\pgfqpoint{1.019140in}{0.759981in}}%
\pgfpathlineto{\pgfqpoint{1.019140in}{0.759981in}}%
\pgfpathclose%
\pgfusepath{stroke}%
\end{pgfscope}%
\begin{pgfscope}%
\pgfpathrectangle{\pgfqpoint{0.688192in}{0.670138in}}{\pgfqpoint{7.111808in}{5.129862in}}%
\pgfusepath{clip}%
\pgfsetbuttcap%
\pgfsetroundjoin%
\pgfsetlinewidth{1.003750pt}%
\definecolor{currentstroke}{rgb}{0.000000,0.000000,0.000000}%
\pgfsetstrokecolor{currentstroke}%
\pgfsetdash{}{0pt}%
\pgfpathmoveto{\pgfqpoint{0.688192in}{5.214003in}}%
\pgfpathcurveto{\pgfqpoint{0.699087in}{5.214003in}}{\pgfqpoint{0.709538in}{5.218332in}}{\pgfqpoint{0.717242in}{5.226037in}}%
\pgfpathcurveto{\pgfqpoint{0.724947in}{5.233741in}}{\pgfqpoint{0.729275in}{5.244192in}}{\pgfqpoint{0.729275in}{5.255087in}}%
\pgfpathcurveto{\pgfqpoint{0.729275in}{5.265983in}}{\pgfqpoint{0.724947in}{5.276434in}}{\pgfqpoint{0.717242in}{5.284138in}}%
\pgfpathcurveto{\pgfqpoint{0.709538in}{5.291842in}}{\pgfqpoint{0.699087in}{5.296171in}}{\pgfqpoint{0.688192in}{5.296171in}}%
\pgfpathcurveto{\pgfqpoint{0.677296in}{5.296171in}}{\pgfqpoint{0.666845in}{5.291842in}}{\pgfqpoint{0.659141in}{5.284138in}}%
\pgfpathcurveto{\pgfqpoint{0.651436in}{5.276434in}}{\pgfqpoint{0.647108in}{5.265983in}}{\pgfqpoint{0.647108in}{5.255087in}}%
\pgfpathcurveto{\pgfqpoint{0.647108in}{5.244192in}}{\pgfqpoint{0.651436in}{5.233741in}}{\pgfqpoint{0.659141in}{5.226037in}}%
\pgfpathcurveto{\pgfqpoint{0.666845in}{5.218332in}}{\pgfqpoint{0.677296in}{5.214003in}}{\pgfqpoint{0.688192in}{5.214003in}}%
\pgfpathlineto{\pgfqpoint{0.688192in}{5.214003in}}%
\pgfpathclose%
\pgfusepath{stroke}%
\end{pgfscope}%
\begin{pgfscope}%
\pgfpathrectangle{\pgfqpoint{0.688192in}{0.670138in}}{\pgfqpoint{7.111808in}{5.129862in}}%
\pgfusepath{clip}%
\pgfsetbuttcap%
\pgfsetroundjoin%
\pgfsetlinewidth{1.003750pt}%
\definecolor{currentstroke}{rgb}{0.000000,0.000000,0.000000}%
\pgfsetstrokecolor{currentstroke}%
\pgfsetdash{}{0pt}%
\pgfpathmoveto{\pgfqpoint{0.886861in}{0.869726in}}%
\pgfpathcurveto{\pgfqpoint{0.897756in}{0.869726in}}{\pgfqpoint{0.908207in}{0.874055in}}{\pgfqpoint{0.915911in}{0.881759in}}%
\pgfpathcurveto{\pgfqpoint{0.923616in}{0.889463in}}{\pgfqpoint{0.927945in}{0.899914in}}{\pgfqpoint{0.927945in}{0.910810in}}%
\pgfpathcurveto{\pgfqpoint{0.927945in}{0.921705in}}{\pgfqpoint{0.923616in}{0.932156in}}{\pgfqpoint{0.915911in}{0.939861in}}%
\pgfpathcurveto{\pgfqpoint{0.908207in}{0.947565in}}{\pgfqpoint{0.897756in}{0.951894in}}{\pgfqpoint{0.886861in}{0.951894in}}%
\pgfpathcurveto{\pgfqpoint{0.875965in}{0.951894in}}{\pgfqpoint{0.865514in}{0.947565in}}{\pgfqpoint{0.857810in}{0.939861in}}%
\pgfpathcurveto{\pgfqpoint{0.850106in}{0.932156in}}{\pgfqpoint{0.845777in}{0.921705in}}{\pgfqpoint{0.845777in}{0.910810in}}%
\pgfpathcurveto{\pgfqpoint{0.845777in}{0.899914in}}{\pgfqpoint{0.850106in}{0.889463in}}{\pgfqpoint{0.857810in}{0.881759in}}%
\pgfpathcurveto{\pgfqpoint{0.865514in}{0.874055in}}{\pgfqpoint{0.875965in}{0.869726in}}{\pgfqpoint{0.886861in}{0.869726in}}%
\pgfpathlineto{\pgfqpoint{0.886861in}{0.869726in}}%
\pgfpathclose%
\pgfusepath{stroke}%
\end{pgfscope}%
\begin{pgfscope}%
\pgfpathrectangle{\pgfqpoint{0.688192in}{0.670138in}}{\pgfqpoint{7.111808in}{5.129862in}}%
\pgfusepath{clip}%
\pgfsetbuttcap%
\pgfsetroundjoin%
\pgfsetlinewidth{1.003750pt}%
\definecolor{currentstroke}{rgb}{0.000000,0.000000,0.000000}%
\pgfsetstrokecolor{currentstroke}%
\pgfsetdash{}{0pt}%
\pgfpathmoveto{\pgfqpoint{0.755774in}{1.165660in}}%
\pgfpathcurveto{\pgfqpoint{0.766670in}{1.165660in}}{\pgfqpoint{0.777120in}{1.169989in}}{\pgfqpoint{0.784825in}{1.177693in}}%
\pgfpathcurveto{\pgfqpoint{0.792529in}{1.185398in}}{\pgfqpoint{0.796858in}{1.195848in}}{\pgfqpoint{0.796858in}{1.206744in}}%
\pgfpathcurveto{\pgfqpoint{0.796858in}{1.217640in}}{\pgfqpoint{0.792529in}{1.228090in}}{\pgfqpoint{0.784825in}{1.235795in}}%
\pgfpathcurveto{\pgfqpoint{0.777120in}{1.243499in}}{\pgfqpoint{0.766670in}{1.247828in}}{\pgfqpoint{0.755774in}{1.247828in}}%
\pgfpathcurveto{\pgfqpoint{0.744878in}{1.247828in}}{\pgfqpoint{0.734428in}{1.243499in}}{\pgfqpoint{0.726723in}{1.235795in}}%
\pgfpathcurveto{\pgfqpoint{0.719019in}{1.228090in}}{\pgfqpoint{0.714690in}{1.217640in}}{\pgfqpoint{0.714690in}{1.206744in}}%
\pgfpathcurveto{\pgfqpoint{0.714690in}{1.195848in}}{\pgfqpoint{0.719019in}{1.185398in}}{\pgfqpoint{0.726723in}{1.177693in}}%
\pgfpathcurveto{\pgfqpoint{0.734428in}{1.169989in}}{\pgfqpoint{0.744878in}{1.165660in}}{\pgfqpoint{0.755774in}{1.165660in}}%
\pgfpathlineto{\pgfqpoint{0.755774in}{1.165660in}}%
\pgfpathclose%
\pgfusepath{stroke}%
\end{pgfscope}%
\begin{pgfscope}%
\pgfpathrectangle{\pgfqpoint{0.688192in}{0.670138in}}{\pgfqpoint{7.111808in}{5.129862in}}%
\pgfusepath{clip}%
\pgfsetbuttcap%
\pgfsetroundjoin%
\pgfsetlinewidth{1.003750pt}%
\definecolor{currentstroke}{rgb}{0.000000,0.000000,0.000000}%
\pgfsetstrokecolor{currentstroke}%
\pgfsetdash{}{0pt}%
\pgfpathmoveto{\pgfqpoint{4.970414in}{0.635715in}}%
\pgfpathcurveto{\pgfqpoint{4.981310in}{0.635715in}}{\pgfqpoint{4.991761in}{0.640044in}}{\pgfqpoint{4.999465in}{0.647748in}}%
\pgfpathcurveto{\pgfqpoint{5.007169in}{0.655453in}}{\pgfqpoint{5.011498in}{0.665904in}}{\pgfqpoint{5.011498in}{0.676799in}}%
\pgfpathcurveto{\pgfqpoint{5.011498in}{0.687695in}}{\pgfqpoint{5.007169in}{0.698146in}}{\pgfqpoint{4.999465in}{0.705850in}}%
\pgfpathcurveto{\pgfqpoint{4.991761in}{0.713554in}}{\pgfqpoint{4.981310in}{0.717883in}}{\pgfqpoint{4.970414in}{0.717883in}}%
\pgfpathcurveto{\pgfqpoint{4.959519in}{0.717883in}}{\pgfqpoint{4.949068in}{0.713554in}}{\pgfqpoint{4.941363in}{0.705850in}}%
\pgfpathcurveto{\pgfqpoint{4.933659in}{0.698146in}}{\pgfqpoint{4.929330in}{0.687695in}}{\pgfqpoint{4.929330in}{0.676799in}}%
\pgfpathcurveto{\pgfqpoint{4.929330in}{0.665904in}}{\pgfqpoint{4.933659in}{0.655453in}}{\pgfqpoint{4.941363in}{0.647748in}}%
\pgfpathcurveto{\pgfqpoint{4.949068in}{0.640044in}}{\pgfqpoint{4.959519in}{0.635715in}}{\pgfqpoint{4.970414in}{0.635715in}}%
\pgfusepath{stroke}%
\end{pgfscope}%
\begin{pgfscope}%
\pgfpathrectangle{\pgfqpoint{0.688192in}{0.670138in}}{\pgfqpoint{7.111808in}{5.129862in}}%
\pgfusepath{clip}%
\pgfsetbuttcap%
\pgfsetroundjoin%
\pgfsetlinewidth{1.003750pt}%
\definecolor{currentstroke}{rgb}{0.000000,0.000000,0.000000}%
\pgfsetstrokecolor{currentstroke}%
\pgfsetdash{}{0pt}%
\pgfpathmoveto{\pgfqpoint{3.387640in}{0.994225in}}%
\pgfpathcurveto{\pgfqpoint{3.398535in}{0.994225in}}{\pgfqpoint{3.408986in}{0.998554in}}{\pgfqpoint{3.416690in}{1.006258in}}%
\pgfpathcurveto{\pgfqpoint{3.424395in}{1.013963in}}{\pgfqpoint{3.428724in}{1.024413in}}{\pgfqpoint{3.428724in}{1.035309in}}%
\pgfpathcurveto{\pgfqpoint{3.428724in}{1.046205in}}{\pgfqpoint{3.424395in}{1.056655in}}{\pgfqpoint{3.416690in}{1.064360in}}%
\pgfpathcurveto{\pgfqpoint{3.408986in}{1.072064in}}{\pgfqpoint{3.398535in}{1.076393in}}{\pgfqpoint{3.387640in}{1.076393in}}%
\pgfpathcurveto{\pgfqpoint{3.376744in}{1.076393in}}{\pgfqpoint{3.366293in}{1.072064in}}{\pgfqpoint{3.358589in}{1.064360in}}%
\pgfpathcurveto{\pgfqpoint{3.350885in}{1.056655in}}{\pgfqpoint{3.346556in}{1.046205in}}{\pgfqpoint{3.346556in}{1.035309in}}%
\pgfpathcurveto{\pgfqpoint{3.346556in}{1.024413in}}{\pgfqpoint{3.350885in}{1.013963in}}{\pgfqpoint{3.358589in}{1.006258in}}%
\pgfpathcurveto{\pgfqpoint{3.366293in}{0.998554in}}{\pgfqpoint{3.376744in}{0.994225in}}{\pgfqpoint{3.387640in}{0.994225in}}%
\pgfpathlineto{\pgfqpoint{3.387640in}{0.994225in}}%
\pgfpathclose%
\pgfusepath{stroke}%
\end{pgfscope}%
\begin{pgfscope}%
\pgfpathrectangle{\pgfqpoint{0.688192in}{0.670138in}}{\pgfqpoint{7.111808in}{5.129862in}}%
\pgfusepath{clip}%
\pgfsetbuttcap%
\pgfsetroundjoin%
\pgfsetlinewidth{1.003750pt}%
\definecolor{currentstroke}{rgb}{0.000000,0.000000,0.000000}%
\pgfsetstrokecolor{currentstroke}%
\pgfsetdash{}{0pt}%
\pgfpathmoveto{\pgfqpoint{0.881277in}{0.900428in}}%
\pgfpathcurveto{\pgfqpoint{0.892173in}{0.900428in}}{\pgfqpoint{0.902624in}{0.904756in}}{\pgfqpoint{0.910328in}{0.912461in}}%
\pgfpathcurveto{\pgfqpoint{0.918032in}{0.920165in}}{\pgfqpoint{0.922361in}{0.930616in}}{\pgfqpoint{0.922361in}{0.941511in}}%
\pgfpathcurveto{\pgfqpoint{0.922361in}{0.952407in}}{\pgfqpoint{0.918032in}{0.962858in}}{\pgfqpoint{0.910328in}{0.970562in}}%
\pgfpathcurveto{\pgfqpoint{0.902624in}{0.978266in}}{\pgfqpoint{0.892173in}{0.982595in}}{\pgfqpoint{0.881277in}{0.982595in}}%
\pgfpathcurveto{\pgfqpoint{0.870382in}{0.982595in}}{\pgfqpoint{0.859931in}{0.978266in}}{\pgfqpoint{0.852226in}{0.970562in}}%
\pgfpathcurveto{\pgfqpoint{0.844522in}{0.962858in}}{\pgfqpoint{0.840193in}{0.952407in}}{\pgfqpoint{0.840193in}{0.941511in}}%
\pgfpathcurveto{\pgfqpoint{0.840193in}{0.930616in}}{\pgfqpoint{0.844522in}{0.920165in}}{\pgfqpoint{0.852226in}{0.912461in}}%
\pgfpathcurveto{\pgfqpoint{0.859931in}{0.904756in}}{\pgfqpoint{0.870382in}{0.900428in}}{\pgfqpoint{0.881277in}{0.900428in}}%
\pgfpathlineto{\pgfqpoint{0.881277in}{0.900428in}}%
\pgfpathclose%
\pgfusepath{stroke}%
\end{pgfscope}%
\begin{pgfscope}%
\pgfpathrectangle{\pgfqpoint{0.688192in}{0.670138in}}{\pgfqpoint{7.111808in}{5.129862in}}%
\pgfusepath{clip}%
\pgfsetbuttcap%
\pgfsetroundjoin%
\pgfsetlinewidth{1.003750pt}%
\definecolor{currentstroke}{rgb}{0.000000,0.000000,0.000000}%
\pgfsetstrokecolor{currentstroke}%
\pgfsetdash{}{0pt}%
\pgfpathmoveto{\pgfqpoint{0.794194in}{1.022605in}}%
\pgfpathcurveto{\pgfqpoint{0.805090in}{1.022605in}}{\pgfqpoint{0.815541in}{1.026933in}}{\pgfqpoint{0.823245in}{1.034638in}}%
\pgfpathcurveto{\pgfqpoint{0.830949in}{1.042342in}}{\pgfqpoint{0.835278in}{1.052793in}}{\pgfqpoint{0.835278in}{1.063688in}}%
\pgfpathcurveto{\pgfqpoint{0.835278in}{1.074584in}}{\pgfqpoint{0.830949in}{1.085035in}}{\pgfqpoint{0.823245in}{1.092739in}}%
\pgfpathcurveto{\pgfqpoint{0.815541in}{1.100443in}}{\pgfqpoint{0.805090in}{1.104772in}}{\pgfqpoint{0.794194in}{1.104772in}}%
\pgfpathcurveto{\pgfqpoint{0.783299in}{1.104772in}}{\pgfqpoint{0.772848in}{1.100443in}}{\pgfqpoint{0.765144in}{1.092739in}}%
\pgfpathcurveto{\pgfqpoint{0.757439in}{1.085035in}}{\pgfqpoint{0.753110in}{1.074584in}}{\pgfqpoint{0.753110in}{1.063688in}}%
\pgfpathcurveto{\pgfqpoint{0.753110in}{1.052793in}}{\pgfqpoint{0.757439in}{1.042342in}}{\pgfqpoint{0.765144in}{1.034638in}}%
\pgfpathcurveto{\pgfqpoint{0.772848in}{1.026933in}}{\pgfqpoint{0.783299in}{1.022605in}}{\pgfqpoint{0.794194in}{1.022605in}}%
\pgfpathlineto{\pgfqpoint{0.794194in}{1.022605in}}%
\pgfpathclose%
\pgfusepath{stroke}%
\end{pgfscope}%
\begin{pgfscope}%
\pgfpathrectangle{\pgfqpoint{0.688192in}{0.670138in}}{\pgfqpoint{7.111808in}{5.129862in}}%
\pgfusepath{clip}%
\pgfsetbuttcap%
\pgfsetroundjoin%
\pgfsetlinewidth{1.003750pt}%
\definecolor{currentstroke}{rgb}{0.000000,0.000000,0.000000}%
\pgfsetstrokecolor{currentstroke}%
\pgfsetdash{}{0pt}%
\pgfpathmoveto{\pgfqpoint{1.320633in}{0.712781in}}%
\pgfpathcurveto{\pgfqpoint{1.331529in}{0.712781in}}{\pgfqpoint{1.341980in}{0.717110in}}{\pgfqpoint{1.349684in}{0.724814in}}%
\pgfpathcurveto{\pgfqpoint{1.357388in}{0.732518in}}{\pgfqpoint{1.361717in}{0.742969in}}{\pgfqpoint{1.361717in}{0.753865in}}%
\pgfpathcurveto{\pgfqpoint{1.361717in}{0.764760in}}{\pgfqpoint{1.357388in}{0.775211in}}{\pgfqpoint{1.349684in}{0.782915in}}%
\pgfpathcurveto{\pgfqpoint{1.341980in}{0.790620in}}{\pgfqpoint{1.331529in}{0.794949in}}{\pgfqpoint{1.320633in}{0.794949in}}%
\pgfpathcurveto{\pgfqpoint{1.309738in}{0.794949in}}{\pgfqpoint{1.299287in}{0.790620in}}{\pgfqpoint{1.291583in}{0.782915in}}%
\pgfpathcurveto{\pgfqpoint{1.283878in}{0.775211in}}{\pgfqpoint{1.279549in}{0.764760in}}{\pgfqpoint{1.279549in}{0.753865in}}%
\pgfpathcurveto{\pgfqpoint{1.279549in}{0.742969in}}{\pgfqpoint{1.283878in}{0.732518in}}{\pgfqpoint{1.291583in}{0.724814in}}%
\pgfpathcurveto{\pgfqpoint{1.299287in}{0.717110in}}{\pgfqpoint{1.309738in}{0.712781in}}{\pgfqpoint{1.320633in}{0.712781in}}%
\pgfpathlineto{\pgfqpoint{1.320633in}{0.712781in}}%
\pgfpathclose%
\pgfusepath{stroke}%
\end{pgfscope}%
\begin{pgfscope}%
\pgfpathrectangle{\pgfqpoint{0.688192in}{0.670138in}}{\pgfqpoint{7.111808in}{5.129862in}}%
\pgfusepath{clip}%
\pgfsetbuttcap%
\pgfsetroundjoin%
\pgfsetlinewidth{1.003750pt}%
\definecolor{currentstroke}{rgb}{0.000000,0.000000,0.000000}%
\pgfsetstrokecolor{currentstroke}%
\pgfsetdash{}{0pt}%
\pgfpathmoveto{\pgfqpoint{2.983194in}{0.663522in}}%
\pgfpathcurveto{\pgfqpoint{2.994090in}{0.663522in}}{\pgfqpoint{3.004540in}{0.667850in}}{\pgfqpoint{3.012245in}{0.675555in}}%
\pgfpathcurveto{\pgfqpoint{3.019949in}{0.683259in}}{\pgfqpoint{3.024278in}{0.693710in}}{\pgfqpoint{3.024278in}{0.704605in}}%
\pgfpathcurveto{\pgfqpoint{3.024278in}{0.715501in}}{\pgfqpoint{3.019949in}{0.725952in}}{\pgfqpoint{3.012245in}{0.733656in}}%
\pgfpathcurveto{\pgfqpoint{3.004540in}{0.741360in}}{\pgfqpoint{2.994090in}{0.745689in}}{\pgfqpoint{2.983194in}{0.745689in}}%
\pgfpathcurveto{\pgfqpoint{2.972299in}{0.745689in}}{\pgfqpoint{2.961848in}{0.741360in}}{\pgfqpoint{2.954143in}{0.733656in}}%
\pgfpathcurveto{\pgfqpoint{2.946439in}{0.725952in}}{\pgfqpoint{2.942110in}{0.715501in}}{\pgfqpoint{2.942110in}{0.704605in}}%
\pgfpathcurveto{\pgfqpoint{2.942110in}{0.693710in}}{\pgfqpoint{2.946439in}{0.683259in}}{\pgfqpoint{2.954143in}{0.675555in}}%
\pgfpathcurveto{\pgfqpoint{2.961848in}{0.667850in}}{\pgfqpoint{2.972299in}{0.663522in}}{\pgfqpoint{2.983194in}{0.663522in}}%
\pgfusepath{stroke}%
\end{pgfscope}%
\begin{pgfscope}%
\pgfpathrectangle{\pgfqpoint{0.688192in}{0.670138in}}{\pgfqpoint{7.111808in}{5.129862in}}%
\pgfusepath{clip}%
\pgfsetbuttcap%
\pgfsetroundjoin%
\pgfsetlinewidth{1.003750pt}%
\definecolor{currentstroke}{rgb}{0.000000,0.000000,0.000000}%
\pgfsetstrokecolor{currentstroke}%
\pgfsetdash{}{0pt}%
\pgfpathmoveto{\pgfqpoint{4.549016in}{0.640923in}}%
\pgfpathcurveto{\pgfqpoint{4.559912in}{0.640923in}}{\pgfqpoint{4.570362in}{0.645252in}}{\pgfqpoint{4.578067in}{0.652956in}}%
\pgfpathcurveto{\pgfqpoint{4.585771in}{0.660660in}}{\pgfqpoint{4.590100in}{0.671111in}}{\pgfqpoint{4.590100in}{0.682007in}}%
\pgfpathcurveto{\pgfqpoint{4.590100in}{0.692902in}}{\pgfqpoint{4.585771in}{0.703353in}}{\pgfqpoint{4.578067in}{0.711057in}}%
\pgfpathcurveto{\pgfqpoint{4.570362in}{0.718762in}}{\pgfqpoint{4.559912in}{0.723090in}}{\pgfqpoint{4.549016in}{0.723090in}}%
\pgfpathcurveto{\pgfqpoint{4.538120in}{0.723090in}}{\pgfqpoint{4.527670in}{0.718762in}}{\pgfqpoint{4.519965in}{0.711057in}}%
\pgfpathcurveto{\pgfqpoint{4.512261in}{0.703353in}}{\pgfqpoint{4.507932in}{0.692902in}}{\pgfqpoint{4.507932in}{0.682007in}}%
\pgfpathcurveto{\pgfqpoint{4.507932in}{0.671111in}}{\pgfqpoint{4.512261in}{0.660660in}}{\pgfqpoint{4.519965in}{0.652956in}}%
\pgfpathcurveto{\pgfqpoint{4.527670in}{0.645252in}}{\pgfqpoint{4.538120in}{0.640923in}}{\pgfqpoint{4.549016in}{0.640923in}}%
\pgfusepath{stroke}%
\end{pgfscope}%
\begin{pgfscope}%
\pgfpathrectangle{\pgfqpoint{0.688192in}{0.670138in}}{\pgfqpoint{7.111808in}{5.129862in}}%
\pgfusepath{clip}%
\pgfsetbuttcap%
\pgfsetroundjoin%
\pgfsetlinewidth{1.003750pt}%
\definecolor{currentstroke}{rgb}{0.000000,0.000000,0.000000}%
\pgfsetstrokecolor{currentstroke}%
\pgfsetdash{}{0pt}%
\pgfpathmoveto{\pgfqpoint{2.247280in}{1.187256in}}%
\pgfpathcurveto{\pgfqpoint{2.258176in}{1.187256in}}{\pgfqpoint{2.268627in}{1.191585in}}{\pgfqpoint{2.276331in}{1.199290in}}%
\pgfpathcurveto{\pgfqpoint{2.284035in}{1.206994in}}{\pgfqpoint{2.288364in}{1.217445in}}{\pgfqpoint{2.288364in}{1.228340in}}%
\pgfpathcurveto{\pgfqpoint{2.288364in}{1.239236in}}{\pgfqpoint{2.284035in}{1.249687in}}{\pgfqpoint{2.276331in}{1.257391in}}%
\pgfpathcurveto{\pgfqpoint{2.268627in}{1.265095in}}{\pgfqpoint{2.258176in}{1.269424in}}{\pgfqpoint{2.247280in}{1.269424in}}%
\pgfpathcurveto{\pgfqpoint{2.236385in}{1.269424in}}{\pgfqpoint{2.225934in}{1.265095in}}{\pgfqpoint{2.218230in}{1.257391in}}%
\pgfpathcurveto{\pgfqpoint{2.210525in}{1.249687in}}{\pgfqpoint{2.206196in}{1.239236in}}{\pgfqpoint{2.206196in}{1.228340in}}%
\pgfpathcurveto{\pgfqpoint{2.206196in}{1.217445in}}{\pgfqpoint{2.210525in}{1.206994in}}{\pgfqpoint{2.218230in}{1.199290in}}%
\pgfpathcurveto{\pgfqpoint{2.225934in}{1.191585in}}{\pgfqpoint{2.236385in}{1.187256in}}{\pgfqpoint{2.247280in}{1.187256in}}%
\pgfpathlineto{\pgfqpoint{2.247280in}{1.187256in}}%
\pgfpathclose%
\pgfusepath{stroke}%
\end{pgfscope}%
\begin{pgfscope}%
\pgfpathrectangle{\pgfqpoint{0.688192in}{0.670138in}}{\pgfqpoint{7.111808in}{5.129862in}}%
\pgfusepath{clip}%
\pgfsetbuttcap%
\pgfsetroundjoin%
\pgfsetlinewidth{1.003750pt}%
\definecolor{currentstroke}{rgb}{0.000000,0.000000,0.000000}%
\pgfsetstrokecolor{currentstroke}%
\pgfsetdash{}{0pt}%
\pgfpathmoveto{\pgfqpoint{1.044526in}{0.742721in}}%
\pgfpathcurveto{\pgfqpoint{1.055421in}{0.742721in}}{\pgfqpoint{1.065872in}{0.747050in}}{\pgfqpoint{1.073577in}{0.754754in}}%
\pgfpathcurveto{\pgfqpoint{1.081281in}{0.762459in}}{\pgfqpoint{1.085610in}{0.772909in}}{\pgfqpoint{1.085610in}{0.783805in}}%
\pgfpathcurveto{\pgfqpoint{1.085610in}{0.794701in}}{\pgfqpoint{1.081281in}{0.805151in}}{\pgfqpoint{1.073577in}{0.812856in}}%
\pgfpathcurveto{\pgfqpoint{1.065872in}{0.820560in}}{\pgfqpoint{1.055421in}{0.824889in}}{\pgfqpoint{1.044526in}{0.824889in}}%
\pgfpathcurveto{\pgfqpoint{1.033630in}{0.824889in}}{\pgfqpoint{1.023179in}{0.820560in}}{\pgfqpoint{1.015475in}{0.812856in}}%
\pgfpathcurveto{\pgfqpoint{1.007771in}{0.805151in}}{\pgfqpoint{1.003442in}{0.794701in}}{\pgfqpoint{1.003442in}{0.783805in}}%
\pgfpathcurveto{\pgfqpoint{1.003442in}{0.772909in}}{\pgfqpoint{1.007771in}{0.762459in}}{\pgfqpoint{1.015475in}{0.754754in}}%
\pgfpathcurveto{\pgfqpoint{1.023179in}{0.747050in}}{\pgfqpoint{1.033630in}{0.742721in}}{\pgfqpoint{1.044526in}{0.742721in}}%
\pgfpathlineto{\pgfqpoint{1.044526in}{0.742721in}}%
\pgfpathclose%
\pgfusepath{stroke}%
\end{pgfscope}%
\begin{pgfscope}%
\pgfpathrectangle{\pgfqpoint{0.688192in}{0.670138in}}{\pgfqpoint{7.111808in}{5.129862in}}%
\pgfusepath{clip}%
\pgfsetbuttcap%
\pgfsetroundjoin%
\pgfsetlinewidth{1.003750pt}%
\definecolor{currentstroke}{rgb}{0.000000,0.000000,0.000000}%
\pgfsetstrokecolor{currentstroke}%
\pgfsetdash{}{0pt}%
\pgfpathmoveto{\pgfqpoint{0.901363in}{0.856548in}}%
\pgfpathcurveto{\pgfqpoint{0.912259in}{0.856548in}}{\pgfqpoint{0.922710in}{0.860877in}}{\pgfqpoint{0.930414in}{0.868582in}}%
\pgfpathcurveto{\pgfqpoint{0.938118in}{0.876286in}}{\pgfqpoint{0.942447in}{0.886737in}}{\pgfqpoint{0.942447in}{0.897632in}}%
\pgfpathcurveto{\pgfqpoint{0.942447in}{0.908528in}}{\pgfqpoint{0.938118in}{0.918979in}}{\pgfqpoint{0.930414in}{0.926683in}}%
\pgfpathcurveto{\pgfqpoint{0.922710in}{0.934387in}}{\pgfqpoint{0.912259in}{0.938716in}}{\pgfqpoint{0.901363in}{0.938716in}}%
\pgfpathcurveto{\pgfqpoint{0.890468in}{0.938716in}}{\pgfqpoint{0.880017in}{0.934387in}}{\pgfqpoint{0.872313in}{0.926683in}}%
\pgfpathcurveto{\pgfqpoint{0.864608in}{0.918979in}}{\pgfqpoint{0.860280in}{0.908528in}}{\pgfqpoint{0.860280in}{0.897632in}}%
\pgfpathcurveto{\pgfqpoint{0.860280in}{0.886737in}}{\pgfqpoint{0.864608in}{0.876286in}}{\pgfqpoint{0.872313in}{0.868582in}}%
\pgfpathcurveto{\pgfqpoint{0.880017in}{0.860877in}}{\pgfqpoint{0.890468in}{0.856548in}}{\pgfqpoint{0.901363in}{0.856548in}}%
\pgfpathlineto{\pgfqpoint{0.901363in}{0.856548in}}%
\pgfpathclose%
\pgfusepath{stroke}%
\end{pgfscope}%
\begin{pgfscope}%
\pgfpathrectangle{\pgfqpoint{0.688192in}{0.670138in}}{\pgfqpoint{7.111808in}{5.129862in}}%
\pgfusepath{clip}%
\pgfsetbuttcap%
\pgfsetroundjoin%
\pgfsetlinewidth{1.003750pt}%
\definecolor{currentstroke}{rgb}{0.000000,0.000000,0.000000}%
\pgfsetstrokecolor{currentstroke}%
\pgfsetdash{}{0pt}%
\pgfpathmoveto{\pgfqpoint{1.050628in}{0.739811in}}%
\pgfpathcurveto{\pgfqpoint{1.061524in}{0.739811in}}{\pgfqpoint{1.071974in}{0.744140in}}{\pgfqpoint{1.079679in}{0.751844in}}%
\pgfpathcurveto{\pgfqpoint{1.087383in}{0.759549in}}{\pgfqpoint{1.091712in}{0.769999in}}{\pgfqpoint{1.091712in}{0.780895in}}%
\pgfpathcurveto{\pgfqpoint{1.091712in}{0.791790in}}{\pgfqpoint{1.087383in}{0.802241in}}{\pgfqpoint{1.079679in}{0.809946in}}%
\pgfpathcurveto{\pgfqpoint{1.071974in}{0.817650in}}{\pgfqpoint{1.061524in}{0.821979in}}{\pgfqpoint{1.050628in}{0.821979in}}%
\pgfpathcurveto{\pgfqpoint{1.039732in}{0.821979in}}{\pgfqpoint{1.029282in}{0.817650in}}{\pgfqpoint{1.021577in}{0.809946in}}%
\pgfpathcurveto{\pgfqpoint{1.013873in}{0.802241in}}{\pgfqpoint{1.009544in}{0.791790in}}{\pgfqpoint{1.009544in}{0.780895in}}%
\pgfpathcurveto{\pgfqpoint{1.009544in}{0.769999in}}{\pgfqpoint{1.013873in}{0.759549in}}{\pgfqpoint{1.021577in}{0.751844in}}%
\pgfpathcurveto{\pgfqpoint{1.029282in}{0.744140in}}{\pgfqpoint{1.039732in}{0.739811in}}{\pgfqpoint{1.050628in}{0.739811in}}%
\pgfpathlineto{\pgfqpoint{1.050628in}{0.739811in}}%
\pgfpathclose%
\pgfusepath{stroke}%
\end{pgfscope}%
\begin{pgfscope}%
\pgfpathrectangle{\pgfqpoint{0.688192in}{0.670138in}}{\pgfqpoint{7.111808in}{5.129862in}}%
\pgfusepath{clip}%
\pgfsetbuttcap%
\pgfsetroundjoin%
\pgfsetlinewidth{1.003750pt}%
\definecolor{currentstroke}{rgb}{0.000000,0.000000,0.000000}%
\pgfsetstrokecolor{currentstroke}%
\pgfsetdash{}{0pt}%
\pgfpathmoveto{\pgfqpoint{0.895312in}{0.866174in}}%
\pgfpathcurveto{\pgfqpoint{0.906208in}{0.866174in}}{\pgfqpoint{0.916658in}{0.870503in}}{\pgfqpoint{0.924363in}{0.878207in}}%
\pgfpathcurveto{\pgfqpoint{0.932067in}{0.885911in}}{\pgfqpoint{0.936396in}{0.896362in}}{\pgfqpoint{0.936396in}{0.907258in}}%
\pgfpathcurveto{\pgfqpoint{0.936396in}{0.918153in}}{\pgfqpoint{0.932067in}{0.928604in}}{\pgfqpoint{0.924363in}{0.936308in}}%
\pgfpathcurveto{\pgfqpoint{0.916658in}{0.944013in}}{\pgfqpoint{0.906208in}{0.948341in}}{\pgfqpoint{0.895312in}{0.948341in}}%
\pgfpathcurveto{\pgfqpoint{0.884416in}{0.948341in}}{\pgfqpoint{0.873966in}{0.944013in}}{\pgfqpoint{0.866261in}{0.936308in}}%
\pgfpathcurveto{\pgfqpoint{0.858557in}{0.928604in}}{\pgfqpoint{0.854228in}{0.918153in}}{\pgfqpoint{0.854228in}{0.907258in}}%
\pgfpathcurveto{\pgfqpoint{0.854228in}{0.896362in}}{\pgfqpoint{0.858557in}{0.885911in}}{\pgfqpoint{0.866261in}{0.878207in}}%
\pgfpathcurveto{\pgfqpoint{0.873966in}{0.870503in}}{\pgfqpoint{0.884416in}{0.866174in}}{\pgfqpoint{0.895312in}{0.866174in}}%
\pgfpathlineto{\pgfqpoint{0.895312in}{0.866174in}}%
\pgfpathclose%
\pgfusepath{stroke}%
\end{pgfscope}%
\begin{pgfscope}%
\pgfpathrectangle{\pgfqpoint{0.688192in}{0.670138in}}{\pgfqpoint{7.111808in}{5.129862in}}%
\pgfusepath{clip}%
\pgfsetbuttcap%
\pgfsetroundjoin%
\pgfsetlinewidth{1.003750pt}%
\definecolor{currentstroke}{rgb}{0.000000,0.000000,0.000000}%
\pgfsetstrokecolor{currentstroke}%
\pgfsetdash{}{0pt}%
\pgfpathmoveto{\pgfqpoint{1.045306in}{0.741536in}}%
\pgfpathcurveto{\pgfqpoint{1.056202in}{0.741536in}}{\pgfqpoint{1.066653in}{0.745865in}}{\pgfqpoint{1.074357in}{0.753569in}}%
\pgfpathcurveto{\pgfqpoint{1.082061in}{0.761274in}}{\pgfqpoint{1.086390in}{0.771725in}}{\pgfqpoint{1.086390in}{0.782620in}}%
\pgfpathcurveto{\pgfqpoint{1.086390in}{0.793516in}}{\pgfqpoint{1.082061in}{0.803967in}}{\pgfqpoint{1.074357in}{0.811671in}}%
\pgfpathcurveto{\pgfqpoint{1.066653in}{0.819375in}}{\pgfqpoint{1.056202in}{0.823704in}}{\pgfqpoint{1.045306in}{0.823704in}}%
\pgfpathcurveto{\pgfqpoint{1.034411in}{0.823704in}}{\pgfqpoint{1.023960in}{0.819375in}}{\pgfqpoint{1.016256in}{0.811671in}}%
\pgfpathcurveto{\pgfqpoint{1.008551in}{0.803967in}}{\pgfqpoint{1.004222in}{0.793516in}}{\pgfqpoint{1.004222in}{0.782620in}}%
\pgfpathcurveto{\pgfqpoint{1.004222in}{0.771725in}}{\pgfqpoint{1.008551in}{0.761274in}}{\pgfqpoint{1.016256in}{0.753569in}}%
\pgfpathcurveto{\pgfqpoint{1.023960in}{0.745865in}}{\pgfqpoint{1.034411in}{0.741536in}}{\pgfqpoint{1.045306in}{0.741536in}}%
\pgfpathlineto{\pgfqpoint{1.045306in}{0.741536in}}%
\pgfpathclose%
\pgfusepath{stroke}%
\end{pgfscope}%
\begin{pgfscope}%
\pgfpathrectangle{\pgfqpoint{0.688192in}{0.670138in}}{\pgfqpoint{7.111808in}{5.129862in}}%
\pgfusepath{clip}%
\pgfsetbuttcap%
\pgfsetroundjoin%
\pgfsetlinewidth{1.003750pt}%
\definecolor{currentstroke}{rgb}{0.000000,0.000000,0.000000}%
\pgfsetstrokecolor{currentstroke}%
\pgfsetdash{}{0pt}%
\pgfpathmoveto{\pgfqpoint{1.448644in}{0.707802in}}%
\pgfpathcurveto{\pgfqpoint{1.459539in}{0.707802in}}{\pgfqpoint{1.469990in}{0.712131in}}{\pgfqpoint{1.477694in}{0.719835in}}%
\pgfpathcurveto{\pgfqpoint{1.485399in}{0.727539in}}{\pgfqpoint{1.489728in}{0.737990in}}{\pgfqpoint{1.489728in}{0.748886in}}%
\pgfpathcurveto{\pgfqpoint{1.489728in}{0.759781in}}{\pgfqpoint{1.485399in}{0.770232in}}{\pgfqpoint{1.477694in}{0.777936in}}%
\pgfpathcurveto{\pgfqpoint{1.469990in}{0.785641in}}{\pgfqpoint{1.459539in}{0.789970in}}{\pgfqpoint{1.448644in}{0.789970in}}%
\pgfpathcurveto{\pgfqpoint{1.437748in}{0.789970in}}{\pgfqpoint{1.427297in}{0.785641in}}{\pgfqpoint{1.419593in}{0.777936in}}%
\pgfpathcurveto{\pgfqpoint{1.411889in}{0.770232in}}{\pgfqpoint{1.407560in}{0.759781in}}{\pgfqpoint{1.407560in}{0.748886in}}%
\pgfpathcurveto{\pgfqpoint{1.407560in}{0.737990in}}{\pgfqpoint{1.411889in}{0.727539in}}{\pgfqpoint{1.419593in}{0.719835in}}%
\pgfpathcurveto{\pgfqpoint{1.427297in}{0.712131in}}{\pgfqpoint{1.437748in}{0.707802in}}{\pgfqpoint{1.448644in}{0.707802in}}%
\pgfpathlineto{\pgfqpoint{1.448644in}{0.707802in}}%
\pgfpathclose%
\pgfusepath{stroke}%
\end{pgfscope}%
\begin{pgfscope}%
\pgfpathrectangle{\pgfqpoint{0.688192in}{0.670138in}}{\pgfqpoint{7.111808in}{5.129862in}}%
\pgfusepath{clip}%
\pgfsetbuttcap%
\pgfsetroundjoin%
\pgfsetlinewidth{1.003750pt}%
\definecolor{currentstroke}{rgb}{0.000000,0.000000,0.000000}%
\pgfsetstrokecolor{currentstroke}%
\pgfsetdash{}{0pt}%
\pgfpathmoveto{\pgfqpoint{1.120660in}{0.719620in}}%
\pgfpathcurveto{\pgfqpoint{1.131555in}{0.719620in}}{\pgfqpoint{1.142006in}{0.723949in}}{\pgfqpoint{1.149710in}{0.731653in}}%
\pgfpathcurveto{\pgfqpoint{1.157415in}{0.739358in}}{\pgfqpoint{1.161744in}{0.749809in}}{\pgfqpoint{1.161744in}{0.760704in}}%
\pgfpathcurveto{\pgfqpoint{1.161744in}{0.771600in}}{\pgfqpoint{1.157415in}{0.782051in}}{\pgfqpoint{1.149710in}{0.789755in}}%
\pgfpathcurveto{\pgfqpoint{1.142006in}{0.797459in}}{\pgfqpoint{1.131555in}{0.801788in}}{\pgfqpoint{1.120660in}{0.801788in}}%
\pgfpathcurveto{\pgfqpoint{1.109764in}{0.801788in}}{\pgfqpoint{1.099313in}{0.797459in}}{\pgfqpoint{1.091609in}{0.789755in}}%
\pgfpathcurveto{\pgfqpoint{1.083905in}{0.782051in}}{\pgfqpoint{1.079576in}{0.771600in}}{\pgfqpoint{1.079576in}{0.760704in}}%
\pgfpathcurveto{\pgfqpoint{1.079576in}{0.749809in}}{\pgfqpoint{1.083905in}{0.739358in}}{\pgfqpoint{1.091609in}{0.731653in}}%
\pgfpathcurveto{\pgfqpoint{1.099313in}{0.723949in}}{\pgfqpoint{1.109764in}{0.719620in}}{\pgfqpoint{1.120660in}{0.719620in}}%
\pgfpathlineto{\pgfqpoint{1.120660in}{0.719620in}}%
\pgfpathclose%
\pgfusepath{stroke}%
\end{pgfscope}%
\begin{pgfscope}%
\pgfpathrectangle{\pgfqpoint{0.688192in}{0.670138in}}{\pgfqpoint{7.111808in}{5.129862in}}%
\pgfusepath{clip}%
\pgfsetbuttcap%
\pgfsetroundjoin%
\pgfsetlinewidth{1.003750pt}%
\definecolor{currentstroke}{rgb}{0.000000,0.000000,0.000000}%
\pgfsetstrokecolor{currentstroke}%
\pgfsetdash{}{0pt}%
\pgfpathmoveto{\pgfqpoint{5.349103in}{4.629873in}}%
\pgfpathcurveto{\pgfqpoint{5.359999in}{4.629873in}}{\pgfqpoint{5.370450in}{4.634201in}}{\pgfqpoint{5.378154in}{4.641906in}}%
\pgfpathcurveto{\pgfqpoint{5.385858in}{4.649610in}}{\pgfqpoint{5.390187in}{4.660061in}}{\pgfqpoint{5.390187in}{4.670956in}}%
\pgfpathcurveto{\pgfqpoint{5.390187in}{4.681852in}}{\pgfqpoint{5.385858in}{4.692303in}}{\pgfqpoint{5.378154in}{4.700007in}}%
\pgfpathcurveto{\pgfqpoint{5.370450in}{4.707712in}}{\pgfqpoint{5.359999in}{4.712040in}}{\pgfqpoint{5.349103in}{4.712040in}}%
\pgfpathcurveto{\pgfqpoint{5.338208in}{4.712040in}}{\pgfqpoint{5.327757in}{4.707712in}}{\pgfqpoint{5.320053in}{4.700007in}}%
\pgfpathcurveto{\pgfqpoint{5.312348in}{4.692303in}}{\pgfqpoint{5.308019in}{4.681852in}}{\pgfqpoint{5.308019in}{4.670956in}}%
\pgfpathcurveto{\pgfqpoint{5.308019in}{4.660061in}}{\pgfqpoint{5.312348in}{4.649610in}}{\pgfqpoint{5.320053in}{4.641906in}}%
\pgfpathcurveto{\pgfqpoint{5.327757in}{4.634201in}}{\pgfqpoint{5.338208in}{4.629873in}}{\pgfqpoint{5.349103in}{4.629873in}}%
\pgfpathlineto{\pgfqpoint{5.349103in}{4.629873in}}%
\pgfpathclose%
\pgfusepath{stroke}%
\end{pgfscope}%
\begin{pgfscope}%
\pgfpathrectangle{\pgfqpoint{0.688192in}{0.670138in}}{\pgfqpoint{7.111808in}{5.129862in}}%
\pgfusepath{clip}%
\pgfsetbuttcap%
\pgfsetroundjoin%
\pgfsetlinewidth{1.003750pt}%
\definecolor{currentstroke}{rgb}{0.000000,0.000000,0.000000}%
\pgfsetstrokecolor{currentstroke}%
\pgfsetdash{}{0pt}%
\pgfpathmoveto{\pgfqpoint{1.070830in}{0.725886in}}%
\pgfpathcurveto{\pgfqpoint{1.081726in}{0.725886in}}{\pgfqpoint{1.092176in}{0.730215in}}{\pgfqpoint{1.099881in}{0.737920in}}%
\pgfpathcurveto{\pgfqpoint{1.107585in}{0.745624in}}{\pgfqpoint{1.111914in}{0.756075in}}{\pgfqpoint{1.111914in}{0.766970in}}%
\pgfpathcurveto{\pgfqpoint{1.111914in}{0.777866in}}{\pgfqpoint{1.107585in}{0.788317in}}{\pgfqpoint{1.099881in}{0.796021in}}%
\pgfpathcurveto{\pgfqpoint{1.092176in}{0.803725in}}{\pgfqpoint{1.081726in}{0.808054in}}{\pgfqpoint{1.070830in}{0.808054in}}%
\pgfpathcurveto{\pgfqpoint{1.059934in}{0.808054in}}{\pgfqpoint{1.049484in}{0.803725in}}{\pgfqpoint{1.041779in}{0.796021in}}%
\pgfpathcurveto{\pgfqpoint{1.034075in}{0.788317in}}{\pgfqpoint{1.029746in}{0.777866in}}{\pgfqpoint{1.029746in}{0.766970in}}%
\pgfpathcurveto{\pgfqpoint{1.029746in}{0.756075in}}{\pgfqpoint{1.034075in}{0.745624in}}{\pgfqpoint{1.041779in}{0.737920in}}%
\pgfpathcurveto{\pgfqpoint{1.049484in}{0.730215in}}{\pgfqpoint{1.059934in}{0.725886in}}{\pgfqpoint{1.070830in}{0.725886in}}%
\pgfpathlineto{\pgfqpoint{1.070830in}{0.725886in}}%
\pgfpathclose%
\pgfusepath{stroke}%
\end{pgfscope}%
\begin{pgfscope}%
\pgfpathrectangle{\pgfqpoint{0.688192in}{0.670138in}}{\pgfqpoint{7.111808in}{5.129862in}}%
\pgfusepath{clip}%
\pgfsetbuttcap%
\pgfsetroundjoin%
\pgfsetlinewidth{1.003750pt}%
\definecolor{currentstroke}{rgb}{0.000000,0.000000,0.000000}%
\pgfsetstrokecolor{currentstroke}%
\pgfsetdash{}{0pt}%
\pgfpathmoveto{\pgfqpoint{5.988842in}{0.629054in}}%
\pgfpathcurveto{\pgfqpoint{5.999738in}{0.629054in}}{\pgfqpoint{6.010189in}{0.633383in}}{\pgfqpoint{6.017893in}{0.641087in}}%
\pgfpathcurveto{\pgfqpoint{6.025598in}{0.648792in}}{\pgfqpoint{6.029926in}{0.659242in}}{\pgfqpoint{6.029926in}{0.670138in}}%
\pgfpathcurveto{\pgfqpoint{6.029926in}{0.681034in}}{\pgfqpoint{6.025598in}{0.691484in}}{\pgfqpoint{6.017893in}{0.699189in}}%
\pgfpathcurveto{\pgfqpoint{6.010189in}{0.706893in}}{\pgfqpoint{5.999738in}{0.711222in}}{\pgfqpoint{5.988842in}{0.711222in}}%
\pgfpathcurveto{\pgfqpoint{5.977947in}{0.711222in}}{\pgfqpoint{5.967496in}{0.706893in}}{\pgfqpoint{5.959792in}{0.699189in}}%
\pgfpathcurveto{\pgfqpoint{5.952087in}{0.691484in}}{\pgfqpoint{5.947759in}{0.681034in}}{\pgfqpoint{5.947759in}{0.670138in}}%
\pgfpathcurveto{\pgfqpoint{5.947759in}{0.659242in}}{\pgfqpoint{5.952087in}{0.648792in}}{\pgfqpoint{5.959792in}{0.641087in}}%
\pgfpathcurveto{\pgfqpoint{5.967496in}{0.633383in}}{\pgfqpoint{5.977947in}{0.629054in}}{\pgfqpoint{5.988842in}{0.629054in}}%
\pgfusepath{stroke}%
\end{pgfscope}%
\begin{pgfscope}%
\pgfpathrectangle{\pgfqpoint{0.688192in}{0.670138in}}{\pgfqpoint{7.111808in}{5.129862in}}%
\pgfusepath{clip}%
\pgfsetbuttcap%
\pgfsetroundjoin%
\pgfsetlinewidth{1.003750pt}%
\definecolor{currentstroke}{rgb}{0.000000,0.000000,0.000000}%
\pgfsetstrokecolor{currentstroke}%
\pgfsetdash{}{0pt}%
\pgfpathmoveto{\pgfqpoint{2.331314in}{0.681791in}}%
\pgfpathcurveto{\pgfqpoint{2.342210in}{0.681791in}}{\pgfqpoint{2.352661in}{0.686120in}}{\pgfqpoint{2.360365in}{0.693824in}}%
\pgfpathcurveto{\pgfqpoint{2.368069in}{0.701529in}}{\pgfqpoint{2.372398in}{0.711979in}}{\pgfqpoint{2.372398in}{0.722875in}}%
\pgfpathcurveto{\pgfqpoint{2.372398in}{0.733771in}}{\pgfqpoint{2.368069in}{0.744221in}}{\pgfqpoint{2.360365in}{0.751926in}}%
\pgfpathcurveto{\pgfqpoint{2.352661in}{0.759630in}}{\pgfqpoint{2.342210in}{0.763959in}}{\pgfqpoint{2.331314in}{0.763959in}}%
\pgfpathcurveto{\pgfqpoint{2.320419in}{0.763959in}}{\pgfqpoint{2.309968in}{0.759630in}}{\pgfqpoint{2.302264in}{0.751926in}}%
\pgfpathcurveto{\pgfqpoint{2.294559in}{0.744221in}}{\pgfqpoint{2.290231in}{0.733771in}}{\pgfqpoint{2.290231in}{0.722875in}}%
\pgfpathcurveto{\pgfqpoint{2.290231in}{0.711979in}}{\pgfqpoint{2.294559in}{0.701529in}}{\pgfqpoint{2.302264in}{0.693824in}}%
\pgfpathcurveto{\pgfqpoint{2.309968in}{0.686120in}}{\pgfqpoint{2.320419in}{0.681791in}}{\pgfqpoint{2.331314in}{0.681791in}}%
\pgfpathlineto{\pgfqpoint{2.331314in}{0.681791in}}%
\pgfpathclose%
\pgfusepath{stroke}%
\end{pgfscope}%
\begin{pgfscope}%
\pgfpathrectangle{\pgfqpoint{0.688192in}{0.670138in}}{\pgfqpoint{7.111808in}{5.129862in}}%
\pgfusepath{clip}%
\pgfsetbuttcap%
\pgfsetroundjoin%
\pgfsetlinewidth{1.003750pt}%
\definecolor{currentstroke}{rgb}{0.000000,0.000000,0.000000}%
\pgfsetstrokecolor{currentstroke}%
\pgfsetdash{}{0pt}%
\pgfpathmoveto{\pgfqpoint{0.931867in}{0.825133in}}%
\pgfpathcurveto{\pgfqpoint{0.942763in}{0.825133in}}{\pgfqpoint{0.953213in}{0.829462in}}{\pgfqpoint{0.960918in}{0.837167in}}%
\pgfpathcurveto{\pgfqpoint{0.968622in}{0.844871in}}{\pgfqpoint{0.972951in}{0.855322in}}{\pgfqpoint{0.972951in}{0.866217in}}%
\pgfpathcurveto{\pgfqpoint{0.972951in}{0.877113in}}{\pgfqpoint{0.968622in}{0.887564in}}{\pgfqpoint{0.960918in}{0.895268in}}%
\pgfpathcurveto{\pgfqpoint{0.953213in}{0.902972in}}{\pgfqpoint{0.942763in}{0.907301in}}{\pgfqpoint{0.931867in}{0.907301in}}%
\pgfpathcurveto{\pgfqpoint{0.920971in}{0.907301in}}{\pgfqpoint{0.910521in}{0.902972in}}{\pgfqpoint{0.902816in}{0.895268in}}%
\pgfpathcurveto{\pgfqpoint{0.895112in}{0.887564in}}{\pgfqpoint{0.890783in}{0.877113in}}{\pgfqpoint{0.890783in}{0.866217in}}%
\pgfpathcurveto{\pgfqpoint{0.890783in}{0.855322in}}{\pgfqpoint{0.895112in}{0.844871in}}{\pgfqpoint{0.902816in}{0.837167in}}%
\pgfpathcurveto{\pgfqpoint{0.910521in}{0.829462in}}{\pgfqpoint{0.920971in}{0.825133in}}{\pgfqpoint{0.931867in}{0.825133in}}%
\pgfpathlineto{\pgfqpoint{0.931867in}{0.825133in}}%
\pgfpathclose%
\pgfusepath{stroke}%
\end{pgfscope}%
\begin{pgfscope}%
\pgfpathrectangle{\pgfqpoint{0.688192in}{0.670138in}}{\pgfqpoint{7.111808in}{5.129862in}}%
\pgfusepath{clip}%
\pgfsetbuttcap%
\pgfsetroundjoin%
\pgfsetlinewidth{1.003750pt}%
\definecolor{currentstroke}{rgb}{0.000000,0.000000,0.000000}%
\pgfsetstrokecolor{currentstroke}%
\pgfsetdash{}{0pt}%
\pgfpathmoveto{\pgfqpoint{1.050667in}{0.739240in}}%
\pgfpathcurveto{\pgfqpoint{1.061563in}{0.739240in}}{\pgfqpoint{1.072014in}{0.743568in}}{\pgfqpoint{1.079718in}{0.751273in}}%
\pgfpathcurveto{\pgfqpoint{1.087422in}{0.758977in}}{\pgfqpoint{1.091751in}{0.769428in}}{\pgfqpoint{1.091751in}{0.780323in}}%
\pgfpathcurveto{\pgfqpoint{1.091751in}{0.791219in}}{\pgfqpoint{1.087422in}{0.801670in}}{\pgfqpoint{1.079718in}{0.809374in}}%
\pgfpathcurveto{\pgfqpoint{1.072014in}{0.817078in}}{\pgfqpoint{1.061563in}{0.821407in}}{\pgfqpoint{1.050667in}{0.821407in}}%
\pgfpathcurveto{\pgfqpoint{1.039772in}{0.821407in}}{\pgfqpoint{1.029321in}{0.817078in}}{\pgfqpoint{1.021617in}{0.809374in}}%
\pgfpathcurveto{\pgfqpoint{1.013912in}{0.801670in}}{\pgfqpoint{1.009583in}{0.791219in}}{\pgfqpoint{1.009583in}{0.780323in}}%
\pgfpathcurveto{\pgfqpoint{1.009583in}{0.769428in}}{\pgfqpoint{1.013912in}{0.758977in}}{\pgfqpoint{1.021617in}{0.751273in}}%
\pgfpathcurveto{\pgfqpoint{1.029321in}{0.743568in}}{\pgfqpoint{1.039772in}{0.739240in}}{\pgfqpoint{1.050667in}{0.739240in}}%
\pgfpathlineto{\pgfqpoint{1.050667in}{0.739240in}}%
\pgfpathclose%
\pgfusepath{stroke}%
\end{pgfscope}%
\begin{pgfscope}%
\pgfpathrectangle{\pgfqpoint{0.688192in}{0.670138in}}{\pgfqpoint{7.111808in}{5.129862in}}%
\pgfusepath{clip}%
\pgfsetbuttcap%
\pgfsetroundjoin%
\pgfsetlinewidth{1.003750pt}%
\definecolor{currentstroke}{rgb}{0.000000,0.000000,0.000000}%
\pgfsetstrokecolor{currentstroke}%
\pgfsetdash{}{0pt}%
\pgfpathmoveto{\pgfqpoint{1.417824in}{0.710820in}}%
\pgfpathcurveto{\pgfqpoint{1.428720in}{0.710820in}}{\pgfqpoint{1.439170in}{0.715149in}}{\pgfqpoint{1.446875in}{0.722854in}}%
\pgfpathcurveto{\pgfqpoint{1.454579in}{0.730558in}}{\pgfqpoint{1.458908in}{0.741009in}}{\pgfqpoint{1.458908in}{0.751904in}}%
\pgfpathcurveto{\pgfqpoint{1.458908in}{0.762800in}}{\pgfqpoint{1.454579in}{0.773251in}}{\pgfqpoint{1.446875in}{0.780955in}}%
\pgfpathcurveto{\pgfqpoint{1.439170in}{0.788659in}}{\pgfqpoint{1.428720in}{0.792988in}}{\pgfqpoint{1.417824in}{0.792988in}}%
\pgfpathcurveto{\pgfqpoint{1.406928in}{0.792988in}}{\pgfqpoint{1.396478in}{0.788659in}}{\pgfqpoint{1.388773in}{0.780955in}}%
\pgfpathcurveto{\pgfqpoint{1.381069in}{0.773251in}}{\pgfqpoint{1.376740in}{0.762800in}}{\pgfqpoint{1.376740in}{0.751904in}}%
\pgfpathcurveto{\pgfqpoint{1.376740in}{0.741009in}}{\pgfqpoint{1.381069in}{0.730558in}}{\pgfqpoint{1.388773in}{0.722854in}}%
\pgfpathcurveto{\pgfqpoint{1.396478in}{0.715149in}}{\pgfqpoint{1.406928in}{0.710820in}}{\pgfqpoint{1.417824in}{0.710820in}}%
\pgfpathlineto{\pgfqpoint{1.417824in}{0.710820in}}%
\pgfpathclose%
\pgfusepath{stroke}%
\end{pgfscope}%
\begin{pgfscope}%
\pgfpathrectangle{\pgfqpoint{0.688192in}{0.670138in}}{\pgfqpoint{7.111808in}{5.129862in}}%
\pgfusepath{clip}%
\pgfsetbuttcap%
\pgfsetroundjoin%
\pgfsetlinewidth{1.003750pt}%
\definecolor{currentstroke}{rgb}{0.000000,0.000000,0.000000}%
\pgfsetstrokecolor{currentstroke}%
\pgfsetdash{}{0pt}%
\pgfpathmoveto{\pgfqpoint{0.790083in}{1.032876in}}%
\pgfpathcurveto{\pgfqpoint{0.800978in}{1.032876in}}{\pgfqpoint{0.811429in}{1.037204in}}{\pgfqpoint{0.819133in}{1.044909in}}%
\pgfpathcurveto{\pgfqpoint{0.826838in}{1.052613in}}{\pgfqpoint{0.831167in}{1.063064in}}{\pgfqpoint{0.831167in}{1.073959in}}%
\pgfpathcurveto{\pgfqpoint{0.831167in}{1.084855in}}{\pgfqpoint{0.826838in}{1.095306in}}{\pgfqpoint{0.819133in}{1.103010in}}%
\pgfpathcurveto{\pgfqpoint{0.811429in}{1.110715in}}{\pgfqpoint{0.800978in}{1.115043in}}{\pgfqpoint{0.790083in}{1.115043in}}%
\pgfpathcurveto{\pgfqpoint{0.779187in}{1.115043in}}{\pgfqpoint{0.768736in}{1.110715in}}{\pgfqpoint{0.761032in}{1.103010in}}%
\pgfpathcurveto{\pgfqpoint{0.753328in}{1.095306in}}{\pgfqpoint{0.748999in}{1.084855in}}{\pgfqpoint{0.748999in}{1.073959in}}%
\pgfpathcurveto{\pgfqpoint{0.748999in}{1.063064in}}{\pgfqpoint{0.753328in}{1.052613in}}{\pgfqpoint{0.761032in}{1.044909in}}%
\pgfpathcurveto{\pgfqpoint{0.768736in}{1.037204in}}{\pgfqpoint{0.779187in}{1.032876in}}{\pgfqpoint{0.790083in}{1.032876in}}%
\pgfpathlineto{\pgfqpoint{0.790083in}{1.032876in}}%
\pgfpathclose%
\pgfusepath{stroke}%
\end{pgfscope}%
\begin{pgfscope}%
\pgfpathrectangle{\pgfqpoint{0.688192in}{0.670138in}}{\pgfqpoint{7.111808in}{5.129862in}}%
\pgfusepath{clip}%
\pgfsetbuttcap%
\pgfsetroundjoin%
\pgfsetlinewidth{1.003750pt}%
\definecolor{currentstroke}{rgb}{0.000000,0.000000,0.000000}%
\pgfsetstrokecolor{currentstroke}%
\pgfsetdash{}{0pt}%
\pgfpathmoveto{\pgfqpoint{0.856742in}{0.923169in}}%
\pgfpathcurveto{\pgfqpoint{0.867638in}{0.923169in}}{\pgfqpoint{0.878089in}{0.927498in}}{\pgfqpoint{0.885793in}{0.935202in}}%
\pgfpathcurveto{\pgfqpoint{0.893497in}{0.942907in}}{\pgfqpoint{0.897826in}{0.953357in}}{\pgfqpoint{0.897826in}{0.964253in}}%
\pgfpathcurveto{\pgfqpoint{0.897826in}{0.975149in}}{\pgfqpoint{0.893497in}{0.985599in}}{\pgfqpoint{0.885793in}{0.993304in}}%
\pgfpathcurveto{\pgfqpoint{0.878089in}{1.001008in}}{\pgfqpoint{0.867638in}{1.005337in}}{\pgfqpoint{0.856742in}{1.005337in}}%
\pgfpathcurveto{\pgfqpoint{0.845847in}{1.005337in}}{\pgfqpoint{0.835396in}{1.001008in}}{\pgfqpoint{0.827691in}{0.993304in}}%
\pgfpathcurveto{\pgfqpoint{0.819987in}{0.985599in}}{\pgfqpoint{0.815658in}{0.975149in}}{\pgfqpoint{0.815658in}{0.964253in}}%
\pgfpathcurveto{\pgfqpoint{0.815658in}{0.953357in}}{\pgfqpoint{0.819987in}{0.942907in}}{\pgfqpoint{0.827691in}{0.935202in}}%
\pgfpathcurveto{\pgfqpoint{0.835396in}{0.927498in}}{\pgfqpoint{0.845847in}{0.923169in}}{\pgfqpoint{0.856742in}{0.923169in}}%
\pgfpathlineto{\pgfqpoint{0.856742in}{0.923169in}}%
\pgfpathclose%
\pgfusepath{stroke}%
\end{pgfscope}%
\begin{pgfscope}%
\pgfpathrectangle{\pgfqpoint{0.688192in}{0.670138in}}{\pgfqpoint{7.111808in}{5.129862in}}%
\pgfusepath{clip}%
\pgfsetbuttcap%
\pgfsetroundjoin%
\pgfsetlinewidth{1.003750pt}%
\definecolor{currentstroke}{rgb}{0.000000,0.000000,0.000000}%
\pgfsetstrokecolor{currentstroke}%
\pgfsetdash{}{0pt}%
\pgfpathmoveto{\pgfqpoint{3.252817in}{0.657352in}}%
\pgfpathcurveto{\pgfqpoint{3.263712in}{0.657352in}}{\pgfqpoint{3.274163in}{0.661681in}}{\pgfqpoint{3.281868in}{0.669385in}}%
\pgfpathcurveto{\pgfqpoint{3.289572in}{0.677089in}}{\pgfqpoint{3.293901in}{0.687540in}}{\pgfqpoint{3.293901in}{0.698436in}}%
\pgfpathcurveto{\pgfqpoint{3.293901in}{0.709331in}}{\pgfqpoint{3.289572in}{0.719782in}}{\pgfqpoint{3.281868in}{0.727486in}}%
\pgfpathcurveto{\pgfqpoint{3.274163in}{0.735191in}}{\pgfqpoint{3.263712in}{0.739520in}}{\pgfqpoint{3.252817in}{0.739520in}}%
\pgfpathcurveto{\pgfqpoint{3.241921in}{0.739520in}}{\pgfqpoint{3.231470in}{0.735191in}}{\pgfqpoint{3.223766in}{0.727486in}}%
\pgfpathcurveto{\pgfqpoint{3.216062in}{0.719782in}}{\pgfqpoint{3.211733in}{0.709331in}}{\pgfqpoint{3.211733in}{0.698436in}}%
\pgfpathcurveto{\pgfqpoint{3.211733in}{0.687540in}}{\pgfqpoint{3.216062in}{0.677089in}}{\pgfqpoint{3.223766in}{0.669385in}}%
\pgfpathcurveto{\pgfqpoint{3.231470in}{0.661681in}}{\pgfqpoint{3.241921in}{0.657352in}}{\pgfqpoint{3.252817in}{0.657352in}}%
\pgfusepath{stroke}%
\end{pgfscope}%
\begin{pgfscope}%
\pgfpathrectangle{\pgfqpoint{0.688192in}{0.670138in}}{\pgfqpoint{7.111808in}{5.129862in}}%
\pgfusepath{clip}%
\pgfsetbuttcap%
\pgfsetroundjoin%
\pgfsetlinewidth{1.003750pt}%
\definecolor{currentstroke}{rgb}{0.000000,0.000000,0.000000}%
\pgfsetstrokecolor{currentstroke}%
\pgfsetdash{}{0pt}%
\pgfpathmoveto{\pgfqpoint{1.019140in}{0.759981in}}%
\pgfpathcurveto{\pgfqpoint{1.030035in}{0.759981in}}{\pgfqpoint{1.040486in}{0.764309in}}{\pgfqpoint{1.048191in}{0.772014in}}%
\pgfpathcurveto{\pgfqpoint{1.055895in}{0.779718in}}{\pgfqpoint{1.060224in}{0.790169in}}{\pgfqpoint{1.060224in}{0.801064in}}%
\pgfpathcurveto{\pgfqpoint{1.060224in}{0.811960in}}{\pgfqpoint{1.055895in}{0.822411in}}{\pgfqpoint{1.048191in}{0.830115in}}%
\pgfpathcurveto{\pgfqpoint{1.040486in}{0.837820in}}{\pgfqpoint{1.030035in}{0.842148in}}{\pgfqpoint{1.019140in}{0.842148in}}%
\pgfpathcurveto{\pgfqpoint{1.008244in}{0.842148in}}{\pgfqpoint{0.997794in}{0.837820in}}{\pgfqpoint{0.990089in}{0.830115in}}%
\pgfpathcurveto{\pgfqpoint{0.982385in}{0.822411in}}{\pgfqpoint{0.978056in}{0.811960in}}{\pgfqpoint{0.978056in}{0.801064in}}%
\pgfpathcurveto{\pgfqpoint{0.978056in}{0.790169in}}{\pgfqpoint{0.982385in}{0.779718in}}{\pgfqpoint{0.990089in}{0.772014in}}%
\pgfpathcurveto{\pgfqpoint{0.997794in}{0.764309in}}{\pgfqpoint{1.008244in}{0.759981in}}{\pgfqpoint{1.019140in}{0.759981in}}%
\pgfpathlineto{\pgfqpoint{1.019140in}{0.759981in}}%
\pgfpathclose%
\pgfusepath{stroke}%
\end{pgfscope}%
\begin{pgfscope}%
\pgfpathrectangle{\pgfqpoint{0.688192in}{0.670138in}}{\pgfqpoint{7.111808in}{5.129862in}}%
\pgfusepath{clip}%
\pgfsetbuttcap%
\pgfsetroundjoin%
\pgfsetlinewidth{1.003750pt}%
\definecolor{currentstroke}{rgb}{0.000000,0.000000,0.000000}%
\pgfsetstrokecolor{currentstroke}%
\pgfsetdash{}{0pt}%
\pgfpathmoveto{\pgfqpoint{1.441321in}{0.708382in}}%
\pgfpathcurveto{\pgfqpoint{1.452217in}{0.708382in}}{\pgfqpoint{1.462667in}{0.712711in}}{\pgfqpoint{1.470372in}{0.720416in}}%
\pgfpathcurveto{\pgfqpoint{1.478076in}{0.728120in}}{\pgfqpoint{1.482405in}{0.738571in}}{\pgfqpoint{1.482405in}{0.749466in}}%
\pgfpathcurveto{\pgfqpoint{1.482405in}{0.760362in}}{\pgfqpoint{1.478076in}{0.770813in}}{\pgfqpoint{1.470372in}{0.778517in}}%
\pgfpathcurveto{\pgfqpoint{1.462667in}{0.786221in}}{\pgfqpoint{1.452217in}{0.790550in}}{\pgfqpoint{1.441321in}{0.790550in}}%
\pgfpathcurveto{\pgfqpoint{1.430425in}{0.790550in}}{\pgfqpoint{1.419975in}{0.786221in}}{\pgfqpoint{1.412270in}{0.778517in}}%
\pgfpathcurveto{\pgfqpoint{1.404566in}{0.770813in}}{\pgfqpoint{1.400237in}{0.760362in}}{\pgfqpoint{1.400237in}{0.749466in}}%
\pgfpathcurveto{\pgfqpoint{1.400237in}{0.738571in}}{\pgfqpoint{1.404566in}{0.728120in}}{\pgfqpoint{1.412270in}{0.720416in}}%
\pgfpathcurveto{\pgfqpoint{1.419975in}{0.712711in}}{\pgfqpoint{1.430425in}{0.708382in}}{\pgfqpoint{1.441321in}{0.708382in}}%
\pgfpathlineto{\pgfqpoint{1.441321in}{0.708382in}}%
\pgfpathclose%
\pgfusepath{stroke}%
\end{pgfscope}%
\begin{pgfscope}%
\pgfpathrectangle{\pgfqpoint{0.688192in}{0.670138in}}{\pgfqpoint{7.111808in}{5.129862in}}%
\pgfusepath{clip}%
\pgfsetbuttcap%
\pgfsetroundjoin%
\pgfsetlinewidth{1.003750pt}%
\definecolor{currentstroke}{rgb}{0.000000,0.000000,0.000000}%
\pgfsetstrokecolor{currentstroke}%
\pgfsetdash{}{0pt}%
\pgfpathmoveto{\pgfqpoint{4.462854in}{0.641380in}}%
\pgfpathcurveto{\pgfqpoint{4.473749in}{0.641380in}}{\pgfqpoint{4.484200in}{0.645709in}}{\pgfqpoint{4.491904in}{0.653414in}}%
\pgfpathcurveto{\pgfqpoint{4.499609in}{0.661118in}}{\pgfqpoint{4.503938in}{0.671569in}}{\pgfqpoint{4.503938in}{0.682464in}}%
\pgfpathcurveto{\pgfqpoint{4.503938in}{0.693360in}}{\pgfqpoint{4.499609in}{0.703811in}}{\pgfqpoint{4.491904in}{0.711515in}}%
\pgfpathcurveto{\pgfqpoint{4.484200in}{0.719219in}}{\pgfqpoint{4.473749in}{0.723548in}}{\pgfqpoint{4.462854in}{0.723548in}}%
\pgfpathcurveto{\pgfqpoint{4.451958in}{0.723548in}}{\pgfqpoint{4.441507in}{0.719219in}}{\pgfqpoint{4.433803in}{0.711515in}}%
\pgfpathcurveto{\pgfqpoint{4.426099in}{0.703811in}}{\pgfqpoint{4.421770in}{0.693360in}}{\pgfqpoint{4.421770in}{0.682464in}}%
\pgfpathcurveto{\pgfqpoint{4.421770in}{0.671569in}}{\pgfqpoint{4.426099in}{0.661118in}}{\pgfqpoint{4.433803in}{0.653414in}}%
\pgfpathcurveto{\pgfqpoint{4.441507in}{0.645709in}}{\pgfqpoint{4.451958in}{0.641380in}}{\pgfqpoint{4.462854in}{0.641380in}}%
\pgfusepath{stroke}%
\end{pgfscope}%
\begin{pgfscope}%
\pgfpathrectangle{\pgfqpoint{0.688192in}{0.670138in}}{\pgfqpoint{7.111808in}{5.129862in}}%
\pgfusepath{clip}%
\pgfsetbuttcap%
\pgfsetroundjoin%
\pgfsetlinewidth{1.003750pt}%
\definecolor{currentstroke}{rgb}{0.000000,0.000000,0.000000}%
\pgfsetstrokecolor{currentstroke}%
\pgfsetdash{}{0pt}%
\pgfpathmoveto{\pgfqpoint{3.252817in}{0.657352in}}%
\pgfpathcurveto{\pgfqpoint{3.263712in}{0.657352in}}{\pgfqpoint{3.274163in}{0.661681in}}{\pgfqpoint{3.281868in}{0.669385in}}%
\pgfpathcurveto{\pgfqpoint{3.289572in}{0.677089in}}{\pgfqpoint{3.293901in}{0.687540in}}{\pgfqpoint{3.293901in}{0.698436in}}%
\pgfpathcurveto{\pgfqpoint{3.293901in}{0.709331in}}{\pgfqpoint{3.289572in}{0.719782in}}{\pgfqpoint{3.281868in}{0.727486in}}%
\pgfpathcurveto{\pgfqpoint{3.274163in}{0.735191in}}{\pgfqpoint{3.263712in}{0.739520in}}{\pgfqpoint{3.252817in}{0.739520in}}%
\pgfpathcurveto{\pgfqpoint{3.241921in}{0.739520in}}{\pgfqpoint{3.231470in}{0.735191in}}{\pgfqpoint{3.223766in}{0.727486in}}%
\pgfpathcurveto{\pgfqpoint{3.216062in}{0.719782in}}{\pgfqpoint{3.211733in}{0.709331in}}{\pgfqpoint{3.211733in}{0.698436in}}%
\pgfpathcurveto{\pgfqpoint{3.211733in}{0.687540in}}{\pgfqpoint{3.216062in}{0.677089in}}{\pgfqpoint{3.223766in}{0.669385in}}%
\pgfpathcurveto{\pgfqpoint{3.231470in}{0.661681in}}{\pgfqpoint{3.241921in}{0.657352in}}{\pgfqpoint{3.252817in}{0.657352in}}%
\pgfusepath{stroke}%
\end{pgfscope}%
\begin{pgfscope}%
\pgfpathrectangle{\pgfqpoint{0.688192in}{0.670138in}}{\pgfqpoint{7.111808in}{5.129862in}}%
\pgfusepath{clip}%
\pgfsetbuttcap%
\pgfsetroundjoin%
\pgfsetlinewidth{1.003750pt}%
\definecolor{currentstroke}{rgb}{0.000000,0.000000,0.000000}%
\pgfsetstrokecolor{currentstroke}%
\pgfsetdash{}{0pt}%
\pgfpathmoveto{\pgfqpoint{1.120660in}{0.719620in}}%
\pgfpathcurveto{\pgfqpoint{1.131555in}{0.719620in}}{\pgfqpoint{1.142006in}{0.723949in}}{\pgfqpoint{1.149710in}{0.731653in}}%
\pgfpathcurveto{\pgfqpoint{1.157415in}{0.739358in}}{\pgfqpoint{1.161744in}{0.749809in}}{\pgfqpoint{1.161744in}{0.760704in}}%
\pgfpathcurveto{\pgfqpoint{1.161744in}{0.771600in}}{\pgfqpoint{1.157415in}{0.782051in}}{\pgfqpoint{1.149710in}{0.789755in}}%
\pgfpathcurveto{\pgfqpoint{1.142006in}{0.797459in}}{\pgfqpoint{1.131555in}{0.801788in}}{\pgfqpoint{1.120660in}{0.801788in}}%
\pgfpathcurveto{\pgfqpoint{1.109764in}{0.801788in}}{\pgfqpoint{1.099313in}{0.797459in}}{\pgfqpoint{1.091609in}{0.789755in}}%
\pgfpathcurveto{\pgfqpoint{1.083905in}{0.782051in}}{\pgfqpoint{1.079576in}{0.771600in}}{\pgfqpoint{1.079576in}{0.760704in}}%
\pgfpathcurveto{\pgfqpoint{1.079576in}{0.749809in}}{\pgfqpoint{1.083905in}{0.739358in}}{\pgfqpoint{1.091609in}{0.731653in}}%
\pgfpathcurveto{\pgfqpoint{1.099313in}{0.723949in}}{\pgfqpoint{1.109764in}{0.719620in}}{\pgfqpoint{1.120660in}{0.719620in}}%
\pgfpathlineto{\pgfqpoint{1.120660in}{0.719620in}}%
\pgfpathclose%
\pgfusepath{stroke}%
\end{pgfscope}%
\begin{pgfscope}%
\pgfpathrectangle{\pgfqpoint{0.688192in}{0.670138in}}{\pgfqpoint{7.111808in}{5.129862in}}%
\pgfusepath{clip}%
\pgfsetbuttcap%
\pgfsetroundjoin%
\pgfsetlinewidth{1.003750pt}%
\definecolor{currentstroke}{rgb}{0.000000,0.000000,0.000000}%
\pgfsetstrokecolor{currentstroke}%
\pgfsetdash{}{0pt}%
\pgfpathmoveto{\pgfqpoint{0.782053in}{1.073732in}}%
\pgfpathcurveto{\pgfqpoint{0.792949in}{1.073732in}}{\pgfqpoint{0.803400in}{1.078061in}}{\pgfqpoint{0.811104in}{1.085765in}}%
\pgfpathcurveto{\pgfqpoint{0.818808in}{1.093469in}}{\pgfqpoint{0.823137in}{1.103920in}}{\pgfqpoint{0.823137in}{1.114816in}}%
\pgfpathcurveto{\pgfqpoint{0.823137in}{1.125711in}}{\pgfqpoint{0.818808in}{1.136162in}}{\pgfqpoint{0.811104in}{1.143866in}}%
\pgfpathcurveto{\pgfqpoint{0.803400in}{1.151571in}}{\pgfqpoint{0.792949in}{1.155899in}}{\pgfqpoint{0.782053in}{1.155899in}}%
\pgfpathcurveto{\pgfqpoint{0.771158in}{1.155899in}}{\pgfqpoint{0.760707in}{1.151571in}}{\pgfqpoint{0.753003in}{1.143866in}}%
\pgfpathcurveto{\pgfqpoint{0.745298in}{1.136162in}}{\pgfqpoint{0.740969in}{1.125711in}}{\pgfqpoint{0.740969in}{1.114816in}}%
\pgfpathcurveto{\pgfqpoint{0.740969in}{1.103920in}}{\pgfqpoint{0.745298in}{1.093469in}}{\pgfqpoint{0.753003in}{1.085765in}}%
\pgfpathcurveto{\pgfqpoint{0.760707in}{1.078061in}}{\pgfqpoint{0.771158in}{1.073732in}}{\pgfqpoint{0.782053in}{1.073732in}}%
\pgfpathlineto{\pgfqpoint{0.782053in}{1.073732in}}%
\pgfpathclose%
\pgfusepath{stroke}%
\end{pgfscope}%
\begin{pgfscope}%
\pgfpathrectangle{\pgfqpoint{0.688192in}{0.670138in}}{\pgfqpoint{7.111808in}{5.129862in}}%
\pgfusepath{clip}%
\pgfsetbuttcap%
\pgfsetroundjoin%
\pgfsetlinewidth{1.003750pt}%
\definecolor{currentstroke}{rgb}{0.000000,0.000000,0.000000}%
\pgfsetstrokecolor{currentstroke}%
\pgfsetdash{}{0pt}%
\pgfpathmoveto{\pgfqpoint{0.895312in}{0.866174in}}%
\pgfpathcurveto{\pgfqpoint{0.906208in}{0.866174in}}{\pgfqpoint{0.916658in}{0.870503in}}{\pgfqpoint{0.924363in}{0.878207in}}%
\pgfpathcurveto{\pgfqpoint{0.932067in}{0.885911in}}{\pgfqpoint{0.936396in}{0.896362in}}{\pgfqpoint{0.936396in}{0.907258in}}%
\pgfpathcurveto{\pgfqpoint{0.936396in}{0.918153in}}{\pgfqpoint{0.932067in}{0.928604in}}{\pgfqpoint{0.924363in}{0.936308in}}%
\pgfpathcurveto{\pgfqpoint{0.916658in}{0.944013in}}{\pgfqpoint{0.906208in}{0.948341in}}{\pgfqpoint{0.895312in}{0.948341in}}%
\pgfpathcurveto{\pgfqpoint{0.884416in}{0.948341in}}{\pgfqpoint{0.873966in}{0.944013in}}{\pgfqpoint{0.866261in}{0.936308in}}%
\pgfpathcurveto{\pgfqpoint{0.858557in}{0.928604in}}{\pgfqpoint{0.854228in}{0.918153in}}{\pgfqpoint{0.854228in}{0.907258in}}%
\pgfpathcurveto{\pgfqpoint{0.854228in}{0.896362in}}{\pgfqpoint{0.858557in}{0.885911in}}{\pgfqpoint{0.866261in}{0.878207in}}%
\pgfpathcurveto{\pgfqpoint{0.873966in}{0.870503in}}{\pgfqpoint{0.884416in}{0.866174in}}{\pgfqpoint{0.895312in}{0.866174in}}%
\pgfpathlineto{\pgfqpoint{0.895312in}{0.866174in}}%
\pgfpathclose%
\pgfusepath{stroke}%
\end{pgfscope}%
\begin{pgfscope}%
\pgfpathrectangle{\pgfqpoint{0.688192in}{0.670138in}}{\pgfqpoint{7.111808in}{5.129862in}}%
\pgfusepath{clip}%
\pgfsetbuttcap%
\pgfsetroundjoin%
\pgfsetlinewidth{1.003750pt}%
\definecolor{currentstroke}{rgb}{0.000000,0.000000,0.000000}%
\pgfsetstrokecolor{currentstroke}%
\pgfsetdash{}{0pt}%
\pgfpathmoveto{\pgfqpoint{4.914884in}{0.636560in}}%
\pgfpathcurveto{\pgfqpoint{4.925780in}{0.636560in}}{\pgfqpoint{4.936230in}{0.640889in}}{\pgfqpoint{4.943935in}{0.648593in}}%
\pgfpathcurveto{\pgfqpoint{4.951639in}{0.656297in}}{\pgfqpoint{4.955968in}{0.666748in}}{\pgfqpoint{4.955968in}{0.677644in}}%
\pgfpathcurveto{\pgfqpoint{4.955968in}{0.688539in}}{\pgfqpoint{4.951639in}{0.698990in}}{\pgfqpoint{4.943935in}{0.706694in}}%
\pgfpathcurveto{\pgfqpoint{4.936230in}{0.714399in}}{\pgfqpoint{4.925780in}{0.718728in}}{\pgfqpoint{4.914884in}{0.718728in}}%
\pgfpathcurveto{\pgfqpoint{4.903989in}{0.718728in}}{\pgfqpoint{4.893538in}{0.714399in}}{\pgfqpoint{4.885833in}{0.706694in}}%
\pgfpathcurveto{\pgfqpoint{4.878129in}{0.698990in}}{\pgfqpoint{4.873800in}{0.688539in}}{\pgfqpoint{4.873800in}{0.677644in}}%
\pgfpathcurveto{\pgfqpoint{4.873800in}{0.666748in}}{\pgfqpoint{4.878129in}{0.656297in}}{\pgfqpoint{4.885833in}{0.648593in}}%
\pgfpathcurveto{\pgfqpoint{4.893538in}{0.640889in}}{\pgfqpoint{4.903989in}{0.636560in}}{\pgfqpoint{4.914884in}{0.636560in}}%
\pgfusepath{stroke}%
\end{pgfscope}%
\begin{pgfscope}%
\pgfpathrectangle{\pgfqpoint{0.688192in}{0.670138in}}{\pgfqpoint{7.111808in}{5.129862in}}%
\pgfusepath{clip}%
\pgfsetbuttcap%
\pgfsetroundjoin%
\pgfsetlinewidth{1.003750pt}%
\definecolor{currentstroke}{rgb}{0.000000,0.000000,0.000000}%
\pgfsetstrokecolor{currentstroke}%
\pgfsetdash{}{0pt}%
\pgfpathmoveto{\pgfqpoint{2.195614in}{0.685221in}}%
\pgfpathcurveto{\pgfqpoint{2.206509in}{0.685221in}}{\pgfqpoint{2.216960in}{0.689550in}}{\pgfqpoint{2.224664in}{0.697254in}}%
\pgfpathcurveto{\pgfqpoint{2.232369in}{0.704959in}}{\pgfqpoint{2.236697in}{0.715409in}}{\pgfqpoint{2.236697in}{0.726305in}}%
\pgfpathcurveto{\pgfqpoint{2.236697in}{0.737200in}}{\pgfqpoint{2.232369in}{0.747651in}}{\pgfqpoint{2.224664in}{0.755356in}}%
\pgfpathcurveto{\pgfqpoint{2.216960in}{0.763060in}}{\pgfqpoint{2.206509in}{0.767389in}}{\pgfqpoint{2.195614in}{0.767389in}}%
\pgfpathcurveto{\pgfqpoint{2.184718in}{0.767389in}}{\pgfqpoint{2.174267in}{0.763060in}}{\pgfqpoint{2.166563in}{0.755356in}}%
\pgfpathcurveto{\pgfqpoint{2.158859in}{0.747651in}}{\pgfqpoint{2.154530in}{0.737200in}}{\pgfqpoint{2.154530in}{0.726305in}}%
\pgfpathcurveto{\pgfqpoint{2.154530in}{0.715409in}}{\pgfqpoint{2.158859in}{0.704959in}}{\pgfqpoint{2.166563in}{0.697254in}}%
\pgfpathcurveto{\pgfqpoint{2.174267in}{0.689550in}}{\pgfqpoint{2.184718in}{0.685221in}}{\pgfqpoint{2.195614in}{0.685221in}}%
\pgfpathlineto{\pgfqpoint{2.195614in}{0.685221in}}%
\pgfpathclose%
\pgfusepath{stroke}%
\end{pgfscope}%
\begin{pgfscope}%
\pgfpathrectangle{\pgfqpoint{0.688192in}{0.670138in}}{\pgfqpoint{7.111808in}{5.129862in}}%
\pgfusepath{clip}%
\pgfsetbuttcap%
\pgfsetroundjoin%
\pgfsetlinewidth{1.003750pt}%
\definecolor{currentstroke}{rgb}{0.000000,0.000000,0.000000}%
\pgfsetstrokecolor{currentstroke}%
\pgfsetdash{}{0pt}%
\pgfpathmoveto{\pgfqpoint{4.603680in}{0.639749in}}%
\pgfpathcurveto{\pgfqpoint{4.614576in}{0.639749in}}{\pgfqpoint{4.625026in}{0.644077in}}{\pgfqpoint{4.632731in}{0.651782in}}%
\pgfpathcurveto{\pgfqpoint{4.640435in}{0.659486in}}{\pgfqpoint{4.644764in}{0.669937in}}{\pgfqpoint{4.644764in}{0.680832in}}%
\pgfpathcurveto{\pgfqpoint{4.644764in}{0.691728in}}{\pgfqpoint{4.640435in}{0.702179in}}{\pgfqpoint{4.632731in}{0.709883in}}%
\pgfpathcurveto{\pgfqpoint{4.625026in}{0.717587in}}{\pgfqpoint{4.614576in}{0.721916in}}{\pgfqpoint{4.603680in}{0.721916in}}%
\pgfpathcurveto{\pgfqpoint{4.592784in}{0.721916in}}{\pgfqpoint{4.582334in}{0.717587in}}{\pgfqpoint{4.574629in}{0.709883in}}%
\pgfpathcurveto{\pgfqpoint{4.566925in}{0.702179in}}{\pgfqpoint{4.562596in}{0.691728in}}{\pgfqpoint{4.562596in}{0.680832in}}%
\pgfpathcurveto{\pgfqpoint{4.562596in}{0.669937in}}{\pgfqpoint{4.566925in}{0.659486in}}{\pgfqpoint{4.574629in}{0.651782in}}%
\pgfpathcurveto{\pgfqpoint{4.582334in}{0.644077in}}{\pgfqpoint{4.592784in}{0.639749in}}{\pgfqpoint{4.603680in}{0.639749in}}%
\pgfusepath{stroke}%
\end{pgfscope}%
\begin{pgfscope}%
\pgfpathrectangle{\pgfqpoint{0.688192in}{0.670138in}}{\pgfqpoint{7.111808in}{5.129862in}}%
\pgfusepath{clip}%
\pgfsetbuttcap%
\pgfsetroundjoin%
\pgfsetlinewidth{1.003750pt}%
\definecolor{currentstroke}{rgb}{0.000000,0.000000,0.000000}%
\pgfsetstrokecolor{currentstroke}%
\pgfsetdash{}{0pt}%
\pgfpathmoveto{\pgfqpoint{1.344495in}{0.711949in}}%
\pgfpathcurveto{\pgfqpoint{1.355391in}{0.711949in}}{\pgfqpoint{1.365842in}{0.716278in}}{\pgfqpoint{1.373546in}{0.723982in}}%
\pgfpathcurveto{\pgfqpoint{1.381250in}{0.731686in}}{\pgfqpoint{1.385579in}{0.742137in}}{\pgfqpoint{1.385579in}{0.753033in}}%
\pgfpathcurveto{\pgfqpoint{1.385579in}{0.763928in}}{\pgfqpoint{1.381250in}{0.774379in}}{\pgfqpoint{1.373546in}{0.782084in}}%
\pgfpathcurveto{\pgfqpoint{1.365842in}{0.789788in}}{\pgfqpoint{1.355391in}{0.794117in}}{\pgfqpoint{1.344495in}{0.794117in}}%
\pgfpathcurveto{\pgfqpoint{1.333600in}{0.794117in}}{\pgfqpoint{1.323149in}{0.789788in}}{\pgfqpoint{1.315444in}{0.782084in}}%
\pgfpathcurveto{\pgfqpoint{1.307740in}{0.774379in}}{\pgfqpoint{1.303411in}{0.763928in}}{\pgfqpoint{1.303411in}{0.753033in}}%
\pgfpathcurveto{\pgfqpoint{1.303411in}{0.742137in}}{\pgfqpoint{1.307740in}{0.731686in}}{\pgfqpoint{1.315444in}{0.723982in}}%
\pgfpathcurveto{\pgfqpoint{1.323149in}{0.716278in}}{\pgfqpoint{1.333600in}{0.711949in}}{\pgfqpoint{1.344495in}{0.711949in}}%
\pgfpathlineto{\pgfqpoint{1.344495in}{0.711949in}}%
\pgfpathclose%
\pgfusepath{stroke}%
\end{pgfscope}%
\begin{pgfscope}%
\pgfpathrectangle{\pgfqpoint{0.688192in}{0.670138in}}{\pgfqpoint{7.111808in}{5.129862in}}%
\pgfusepath{clip}%
\pgfsetbuttcap%
\pgfsetroundjoin%
\pgfsetlinewidth{1.003750pt}%
\definecolor{currentstroke}{rgb}{0.000000,0.000000,0.000000}%
\pgfsetstrokecolor{currentstroke}%
\pgfsetdash{}{0pt}%
\pgfpathmoveto{\pgfqpoint{5.537814in}{0.631347in}}%
\pgfpathcurveto{\pgfqpoint{5.548710in}{0.631347in}}{\pgfqpoint{5.559161in}{0.635676in}}{\pgfqpoint{5.566865in}{0.643380in}}%
\pgfpathcurveto{\pgfqpoint{5.574569in}{0.651084in}}{\pgfqpoint{5.578898in}{0.661535in}}{\pgfqpoint{5.578898in}{0.672431in}}%
\pgfpathcurveto{\pgfqpoint{5.578898in}{0.683326in}}{\pgfqpoint{5.574569in}{0.693777in}}{\pgfqpoint{5.566865in}{0.701481in}}%
\pgfpathcurveto{\pgfqpoint{5.559161in}{0.709186in}}{\pgfqpoint{5.548710in}{0.713515in}}{\pgfqpoint{5.537814in}{0.713515in}}%
\pgfpathcurveto{\pgfqpoint{5.526919in}{0.713515in}}{\pgfqpoint{5.516468in}{0.709186in}}{\pgfqpoint{5.508764in}{0.701481in}}%
\pgfpathcurveto{\pgfqpoint{5.501059in}{0.693777in}}{\pgfqpoint{5.496731in}{0.683326in}}{\pgfqpoint{5.496731in}{0.672431in}}%
\pgfpathcurveto{\pgfqpoint{5.496731in}{0.661535in}}{\pgfqpoint{5.501059in}{0.651084in}}{\pgfqpoint{5.508764in}{0.643380in}}%
\pgfpathcurveto{\pgfqpoint{5.516468in}{0.635676in}}{\pgfqpoint{5.526919in}{0.631347in}}{\pgfqpoint{5.537814in}{0.631347in}}%
\pgfusepath{stroke}%
\end{pgfscope}%
\begin{pgfscope}%
\pgfpathrectangle{\pgfqpoint{0.688192in}{0.670138in}}{\pgfqpoint{7.111808in}{5.129862in}}%
\pgfusepath{clip}%
\pgfsetbuttcap%
\pgfsetroundjoin%
\pgfsetlinewidth{1.003750pt}%
\definecolor{currentstroke}{rgb}{0.000000,0.000000,0.000000}%
\pgfsetstrokecolor{currentstroke}%
\pgfsetdash{}{0pt}%
\pgfpathmoveto{\pgfqpoint{1.920423in}{0.692661in}}%
\pgfpathcurveto{\pgfqpoint{1.931318in}{0.692661in}}{\pgfqpoint{1.941769in}{0.696990in}}{\pgfqpoint{1.949473in}{0.704695in}}%
\pgfpathcurveto{\pgfqpoint{1.957178in}{0.712399in}}{\pgfqpoint{1.961506in}{0.722850in}}{\pgfqpoint{1.961506in}{0.733745in}}%
\pgfpathcurveto{\pgfqpoint{1.961506in}{0.744641in}}{\pgfqpoint{1.957178in}{0.755092in}}{\pgfqpoint{1.949473in}{0.762796in}}%
\pgfpathcurveto{\pgfqpoint{1.941769in}{0.770500in}}{\pgfqpoint{1.931318in}{0.774829in}}{\pgfqpoint{1.920423in}{0.774829in}}%
\pgfpathcurveto{\pgfqpoint{1.909527in}{0.774829in}}{\pgfqpoint{1.899076in}{0.770500in}}{\pgfqpoint{1.891372in}{0.762796in}}%
\pgfpathcurveto{\pgfqpoint{1.883668in}{0.755092in}}{\pgfqpoint{1.879339in}{0.744641in}}{\pgfqpoint{1.879339in}{0.733745in}}%
\pgfpathcurveto{\pgfqpoint{1.879339in}{0.722850in}}{\pgfqpoint{1.883668in}{0.712399in}}{\pgfqpoint{1.891372in}{0.704695in}}%
\pgfpathcurveto{\pgfqpoint{1.899076in}{0.696990in}}{\pgfqpoint{1.909527in}{0.692661in}}{\pgfqpoint{1.920423in}{0.692661in}}%
\pgfpathlineto{\pgfqpoint{1.920423in}{0.692661in}}%
\pgfpathclose%
\pgfusepath{stroke}%
\end{pgfscope}%
\begin{pgfscope}%
\pgfpathrectangle{\pgfqpoint{0.688192in}{0.670138in}}{\pgfqpoint{7.111808in}{5.129862in}}%
\pgfusepath{clip}%
\pgfsetbuttcap%
\pgfsetroundjoin%
\pgfsetlinewidth{1.003750pt}%
\definecolor{currentstroke}{rgb}{0.000000,0.000000,0.000000}%
\pgfsetstrokecolor{currentstroke}%
\pgfsetdash{}{0pt}%
\pgfpathmoveto{\pgfqpoint{1.069116in}{0.729808in}}%
\pgfpathcurveto{\pgfqpoint{1.080011in}{0.729808in}}{\pgfqpoint{1.090462in}{0.734137in}}{\pgfqpoint{1.098166in}{0.741841in}}%
\pgfpathcurveto{\pgfqpoint{1.105871in}{0.749545in}}{\pgfqpoint{1.110200in}{0.759996in}}{\pgfqpoint{1.110200in}{0.770892in}}%
\pgfpathcurveto{\pgfqpoint{1.110200in}{0.781787in}}{\pgfqpoint{1.105871in}{0.792238in}}{\pgfqpoint{1.098166in}{0.799942in}}%
\pgfpathcurveto{\pgfqpoint{1.090462in}{0.807647in}}{\pgfqpoint{1.080011in}{0.811976in}}{\pgfqpoint{1.069116in}{0.811976in}}%
\pgfpathcurveto{\pgfqpoint{1.058220in}{0.811976in}}{\pgfqpoint{1.047769in}{0.807647in}}{\pgfqpoint{1.040065in}{0.799942in}}%
\pgfpathcurveto{\pgfqpoint{1.032361in}{0.792238in}}{\pgfqpoint{1.028032in}{0.781787in}}{\pgfqpoint{1.028032in}{0.770892in}}%
\pgfpathcurveto{\pgfqpoint{1.028032in}{0.759996in}}{\pgfqpoint{1.032361in}{0.749545in}}{\pgfqpoint{1.040065in}{0.741841in}}%
\pgfpathcurveto{\pgfqpoint{1.047769in}{0.734137in}}{\pgfqpoint{1.058220in}{0.729808in}}{\pgfqpoint{1.069116in}{0.729808in}}%
\pgfpathlineto{\pgfqpoint{1.069116in}{0.729808in}}%
\pgfpathclose%
\pgfusepath{stroke}%
\end{pgfscope}%
\begin{pgfscope}%
\pgfpathrectangle{\pgfqpoint{0.688192in}{0.670138in}}{\pgfqpoint{7.111808in}{5.129862in}}%
\pgfusepath{clip}%
\pgfsetbuttcap%
\pgfsetroundjoin%
\pgfsetlinewidth{1.003750pt}%
\definecolor{currentstroke}{rgb}{0.000000,0.000000,0.000000}%
\pgfsetstrokecolor{currentstroke}%
\pgfsetdash{}{0pt}%
\pgfpathmoveto{\pgfqpoint{2.346054in}{0.680598in}}%
\pgfpathcurveto{\pgfqpoint{2.356949in}{0.680598in}}{\pgfqpoint{2.367400in}{0.684927in}}{\pgfqpoint{2.375104in}{0.692631in}}%
\pgfpathcurveto{\pgfqpoint{2.382809in}{0.700336in}}{\pgfqpoint{2.387137in}{0.710787in}}{\pgfqpoint{2.387137in}{0.721682in}}%
\pgfpathcurveto{\pgfqpoint{2.387137in}{0.732578in}}{\pgfqpoint{2.382809in}{0.743028in}}{\pgfqpoint{2.375104in}{0.750733in}}%
\pgfpathcurveto{\pgfqpoint{2.367400in}{0.758437in}}{\pgfqpoint{2.356949in}{0.762766in}}{\pgfqpoint{2.346054in}{0.762766in}}%
\pgfpathcurveto{\pgfqpoint{2.335158in}{0.762766in}}{\pgfqpoint{2.324707in}{0.758437in}}{\pgfqpoint{2.317003in}{0.750733in}}%
\pgfpathcurveto{\pgfqpoint{2.309299in}{0.743028in}}{\pgfqpoint{2.304970in}{0.732578in}}{\pgfqpoint{2.304970in}{0.721682in}}%
\pgfpathcurveto{\pgfqpoint{2.304970in}{0.710787in}}{\pgfqpoint{2.309299in}{0.700336in}}{\pgfqpoint{2.317003in}{0.692631in}}%
\pgfpathcurveto{\pgfqpoint{2.324707in}{0.684927in}}{\pgfqpoint{2.335158in}{0.680598in}}{\pgfqpoint{2.346054in}{0.680598in}}%
\pgfpathlineto{\pgfqpoint{2.346054in}{0.680598in}}%
\pgfpathclose%
\pgfusepath{stroke}%
\end{pgfscope}%
\begin{pgfscope}%
\pgfpathrectangle{\pgfqpoint{0.688192in}{0.670138in}}{\pgfqpoint{7.111808in}{5.129862in}}%
\pgfusepath{clip}%
\pgfsetbuttcap%
\pgfsetroundjoin%
\pgfsetlinewidth{1.003750pt}%
\definecolor{currentstroke}{rgb}{0.000000,0.000000,0.000000}%
\pgfsetstrokecolor{currentstroke}%
\pgfsetdash{}{0pt}%
\pgfpathmoveto{\pgfqpoint{1.351667in}{0.711476in}}%
\pgfpathcurveto{\pgfqpoint{1.362562in}{0.711476in}}{\pgfqpoint{1.373013in}{0.715805in}}{\pgfqpoint{1.380718in}{0.723510in}}%
\pgfpathcurveto{\pgfqpoint{1.388422in}{0.731214in}}{\pgfqpoint{1.392751in}{0.741665in}}{\pgfqpoint{1.392751in}{0.752560in}}%
\pgfpathcurveto{\pgfqpoint{1.392751in}{0.763456in}}{\pgfqpoint{1.388422in}{0.773907in}}{\pgfqpoint{1.380718in}{0.781611in}}%
\pgfpathcurveto{\pgfqpoint{1.373013in}{0.789315in}}{\pgfqpoint{1.362562in}{0.793644in}}{\pgfqpoint{1.351667in}{0.793644in}}%
\pgfpathcurveto{\pgfqpoint{1.340771in}{0.793644in}}{\pgfqpoint{1.330321in}{0.789315in}}{\pgfqpoint{1.322616in}{0.781611in}}%
\pgfpathcurveto{\pgfqpoint{1.314912in}{0.773907in}}{\pgfqpoint{1.310583in}{0.763456in}}{\pgfqpoint{1.310583in}{0.752560in}}%
\pgfpathcurveto{\pgfqpoint{1.310583in}{0.741665in}}{\pgfqpoint{1.314912in}{0.731214in}}{\pgfqpoint{1.322616in}{0.723510in}}%
\pgfpathcurveto{\pgfqpoint{1.330321in}{0.715805in}}{\pgfqpoint{1.340771in}{0.711476in}}{\pgfqpoint{1.351667in}{0.711476in}}%
\pgfpathlineto{\pgfqpoint{1.351667in}{0.711476in}}%
\pgfpathclose%
\pgfusepath{stroke}%
\end{pgfscope}%
\begin{pgfscope}%
\pgfpathrectangle{\pgfqpoint{0.688192in}{0.670138in}}{\pgfqpoint{7.111808in}{5.129862in}}%
\pgfusepath{clip}%
\pgfsetbuttcap%
\pgfsetroundjoin%
\pgfsetlinewidth{1.003750pt}%
\definecolor{currentstroke}{rgb}{0.000000,0.000000,0.000000}%
\pgfsetstrokecolor{currentstroke}%
\pgfsetdash{}{0pt}%
\pgfpathmoveto{\pgfqpoint{1.141797in}{0.719357in}}%
\pgfpathcurveto{\pgfqpoint{1.152693in}{0.719357in}}{\pgfqpoint{1.163143in}{0.723686in}}{\pgfqpoint{1.170848in}{0.731391in}}%
\pgfpathcurveto{\pgfqpoint{1.178552in}{0.739095in}}{\pgfqpoint{1.182881in}{0.749546in}}{\pgfqpoint{1.182881in}{0.760441in}}%
\pgfpathcurveto{\pgfqpoint{1.182881in}{0.771337in}}{\pgfqpoint{1.178552in}{0.781788in}}{\pgfqpoint{1.170848in}{0.789492in}}%
\pgfpathcurveto{\pgfqpoint{1.163143in}{0.797196in}}{\pgfqpoint{1.152693in}{0.801525in}}{\pgfqpoint{1.141797in}{0.801525in}}%
\pgfpathcurveto{\pgfqpoint{1.130902in}{0.801525in}}{\pgfqpoint{1.120451in}{0.797196in}}{\pgfqpoint{1.112746in}{0.789492in}}%
\pgfpathcurveto{\pgfqpoint{1.105042in}{0.781788in}}{\pgfqpoint{1.100713in}{0.771337in}}{\pgfqpoint{1.100713in}{0.760441in}}%
\pgfpathcurveto{\pgfqpoint{1.100713in}{0.749546in}}{\pgfqpoint{1.105042in}{0.739095in}}{\pgfqpoint{1.112746in}{0.731391in}}%
\pgfpathcurveto{\pgfqpoint{1.120451in}{0.723686in}}{\pgfqpoint{1.130902in}{0.719357in}}{\pgfqpoint{1.141797in}{0.719357in}}%
\pgfpathlineto{\pgfqpoint{1.141797in}{0.719357in}}%
\pgfpathclose%
\pgfusepath{stroke}%
\end{pgfscope}%
\begin{pgfscope}%
\pgfpathrectangle{\pgfqpoint{0.688192in}{0.670138in}}{\pgfqpoint{7.111808in}{5.129862in}}%
\pgfusepath{clip}%
\pgfsetbuttcap%
\pgfsetroundjoin%
\pgfsetlinewidth{1.003750pt}%
\definecolor{currentstroke}{rgb}{0.000000,0.000000,0.000000}%
\pgfsetstrokecolor{currentstroke}%
\pgfsetdash{}{0pt}%
\pgfpathmoveto{\pgfqpoint{2.497167in}{0.675229in}}%
\pgfpathcurveto{\pgfqpoint{2.508063in}{0.675229in}}{\pgfqpoint{2.518514in}{0.679557in}}{\pgfqpoint{2.526218in}{0.687262in}}%
\pgfpathcurveto{\pgfqpoint{2.533922in}{0.694966in}}{\pgfqpoint{2.538251in}{0.705417in}}{\pgfqpoint{2.538251in}{0.716312in}}%
\pgfpathcurveto{\pgfqpoint{2.538251in}{0.727208in}}{\pgfqpoint{2.533922in}{0.737659in}}{\pgfqpoint{2.526218in}{0.745363in}}%
\pgfpathcurveto{\pgfqpoint{2.518514in}{0.753067in}}{\pgfqpoint{2.508063in}{0.757396in}}{\pgfqpoint{2.497167in}{0.757396in}}%
\pgfpathcurveto{\pgfqpoint{2.486272in}{0.757396in}}{\pgfqpoint{2.475821in}{0.753067in}}{\pgfqpoint{2.468117in}{0.745363in}}%
\pgfpathcurveto{\pgfqpoint{2.460412in}{0.737659in}}{\pgfqpoint{2.456084in}{0.727208in}}{\pgfqpoint{2.456084in}{0.716312in}}%
\pgfpathcurveto{\pgfqpoint{2.456084in}{0.705417in}}{\pgfqpoint{2.460412in}{0.694966in}}{\pgfqpoint{2.468117in}{0.687262in}}%
\pgfpathcurveto{\pgfqpoint{2.475821in}{0.679557in}}{\pgfqpoint{2.486272in}{0.675229in}}{\pgfqpoint{2.497167in}{0.675229in}}%
\pgfpathlineto{\pgfqpoint{2.497167in}{0.675229in}}%
\pgfpathclose%
\pgfusepath{stroke}%
\end{pgfscope}%
\begin{pgfscope}%
\pgfpathrectangle{\pgfqpoint{0.688192in}{0.670138in}}{\pgfqpoint{7.111808in}{5.129862in}}%
\pgfusepath{clip}%
\pgfsetbuttcap%
\pgfsetroundjoin%
\pgfsetlinewidth{1.003750pt}%
\definecolor{currentstroke}{rgb}{0.000000,0.000000,0.000000}%
\pgfsetstrokecolor{currentstroke}%
\pgfsetdash{}{0pt}%
\pgfpathmoveto{\pgfqpoint{4.042326in}{1.604690in}}%
\pgfpathcurveto{\pgfqpoint{4.053222in}{1.604690in}}{\pgfqpoint{4.063673in}{1.609019in}}{\pgfqpoint{4.071377in}{1.616723in}}%
\pgfpathcurveto{\pgfqpoint{4.079081in}{1.624427in}}{\pgfqpoint{4.083410in}{1.634878in}}{\pgfqpoint{4.083410in}{1.645774in}}%
\pgfpathcurveto{\pgfqpoint{4.083410in}{1.656669in}}{\pgfqpoint{4.079081in}{1.667120in}}{\pgfqpoint{4.071377in}{1.674824in}}%
\pgfpathcurveto{\pgfqpoint{4.063673in}{1.682529in}}{\pgfqpoint{4.053222in}{1.686857in}}{\pgfqpoint{4.042326in}{1.686857in}}%
\pgfpathcurveto{\pgfqpoint{4.031431in}{1.686857in}}{\pgfqpoint{4.020980in}{1.682529in}}{\pgfqpoint{4.013276in}{1.674824in}}%
\pgfpathcurveto{\pgfqpoint{4.005571in}{1.667120in}}{\pgfqpoint{4.001242in}{1.656669in}}{\pgfqpoint{4.001242in}{1.645774in}}%
\pgfpathcurveto{\pgfqpoint{4.001242in}{1.634878in}}{\pgfqpoint{4.005571in}{1.624427in}}{\pgfqpoint{4.013276in}{1.616723in}}%
\pgfpathcurveto{\pgfqpoint{4.020980in}{1.609019in}}{\pgfqpoint{4.031431in}{1.604690in}}{\pgfqpoint{4.042326in}{1.604690in}}%
\pgfpathlineto{\pgfqpoint{4.042326in}{1.604690in}}%
\pgfpathclose%
\pgfusepath{stroke}%
\end{pgfscope}%
\begin{pgfscope}%
\pgfpathrectangle{\pgfqpoint{0.688192in}{0.670138in}}{\pgfqpoint{7.111808in}{5.129862in}}%
\pgfusepath{clip}%
\pgfsetbuttcap%
\pgfsetroundjoin%
\pgfsetlinewidth{1.003750pt}%
\definecolor{currentstroke}{rgb}{0.000000,0.000000,0.000000}%
\pgfsetstrokecolor{currentstroke}%
\pgfsetdash{}{0pt}%
\pgfpathmoveto{\pgfqpoint{3.173632in}{0.915058in}}%
\pgfpathcurveto{\pgfqpoint{3.184527in}{0.915058in}}{\pgfqpoint{3.194978in}{0.919387in}}{\pgfqpoint{3.202683in}{0.927091in}}%
\pgfpathcurveto{\pgfqpoint{3.210387in}{0.934795in}}{\pgfqpoint{3.214716in}{0.945246in}}{\pgfqpoint{3.214716in}{0.956142in}}%
\pgfpathcurveto{\pgfqpoint{3.214716in}{0.967037in}}{\pgfqpoint{3.210387in}{0.977488in}}{\pgfqpoint{3.202683in}{0.985192in}}%
\pgfpathcurveto{\pgfqpoint{3.194978in}{0.992897in}}{\pgfqpoint{3.184527in}{0.997226in}}{\pgfqpoint{3.173632in}{0.997226in}}%
\pgfpathcurveto{\pgfqpoint{3.162736in}{0.997226in}}{\pgfqpoint{3.152286in}{0.992897in}}{\pgfqpoint{3.144581in}{0.985192in}}%
\pgfpathcurveto{\pgfqpoint{3.136877in}{0.977488in}}{\pgfqpoint{3.132548in}{0.967037in}}{\pgfqpoint{3.132548in}{0.956142in}}%
\pgfpathcurveto{\pgfqpoint{3.132548in}{0.945246in}}{\pgfqpoint{3.136877in}{0.934795in}}{\pgfqpoint{3.144581in}{0.927091in}}%
\pgfpathcurveto{\pgfqpoint{3.152286in}{0.919387in}}{\pgfqpoint{3.162736in}{0.915058in}}{\pgfqpoint{3.173632in}{0.915058in}}%
\pgfpathlineto{\pgfqpoint{3.173632in}{0.915058in}}%
\pgfpathclose%
\pgfusepath{stroke}%
\end{pgfscope}%
\begin{pgfscope}%
\pgfpathrectangle{\pgfqpoint{0.688192in}{0.670138in}}{\pgfqpoint{7.111808in}{5.129862in}}%
\pgfusepath{clip}%
\pgfsetbuttcap%
\pgfsetroundjoin%
\pgfsetlinewidth{1.003750pt}%
\definecolor{currentstroke}{rgb}{0.000000,0.000000,0.000000}%
\pgfsetstrokecolor{currentstroke}%
\pgfsetdash{}{0pt}%
\pgfpathmoveto{\pgfqpoint{1.351667in}{0.711476in}}%
\pgfpathcurveto{\pgfqpoint{1.362562in}{0.711476in}}{\pgfqpoint{1.373013in}{0.715805in}}{\pgfqpoint{1.380718in}{0.723510in}}%
\pgfpathcurveto{\pgfqpoint{1.388422in}{0.731214in}}{\pgfqpoint{1.392751in}{0.741665in}}{\pgfqpoint{1.392751in}{0.752560in}}%
\pgfpathcurveto{\pgfqpoint{1.392751in}{0.763456in}}{\pgfqpoint{1.388422in}{0.773907in}}{\pgfqpoint{1.380718in}{0.781611in}}%
\pgfpathcurveto{\pgfqpoint{1.373013in}{0.789315in}}{\pgfqpoint{1.362562in}{0.793644in}}{\pgfqpoint{1.351667in}{0.793644in}}%
\pgfpathcurveto{\pgfqpoint{1.340771in}{0.793644in}}{\pgfqpoint{1.330321in}{0.789315in}}{\pgfqpoint{1.322616in}{0.781611in}}%
\pgfpathcurveto{\pgfqpoint{1.314912in}{0.773907in}}{\pgfqpoint{1.310583in}{0.763456in}}{\pgfqpoint{1.310583in}{0.752560in}}%
\pgfpathcurveto{\pgfqpoint{1.310583in}{0.741665in}}{\pgfqpoint{1.314912in}{0.731214in}}{\pgfqpoint{1.322616in}{0.723510in}}%
\pgfpathcurveto{\pgfqpoint{1.330321in}{0.715805in}}{\pgfqpoint{1.340771in}{0.711476in}}{\pgfqpoint{1.351667in}{0.711476in}}%
\pgfpathlineto{\pgfqpoint{1.351667in}{0.711476in}}%
\pgfpathclose%
\pgfusepath{stroke}%
\end{pgfscope}%
\begin{pgfscope}%
\pgfpathrectangle{\pgfqpoint{0.688192in}{0.670138in}}{\pgfqpoint{7.111808in}{5.129862in}}%
\pgfusepath{clip}%
\pgfsetbuttcap%
\pgfsetroundjoin%
\pgfsetlinewidth{1.003750pt}%
\definecolor{currentstroke}{rgb}{0.000000,0.000000,0.000000}%
\pgfsetstrokecolor{currentstroke}%
\pgfsetdash{}{0pt}%
\pgfpathmoveto{\pgfqpoint{5.119938in}{0.809659in}}%
\pgfpathcurveto{\pgfqpoint{5.130833in}{0.809659in}}{\pgfqpoint{5.141284in}{0.813988in}}{\pgfqpoint{5.148988in}{0.821692in}}%
\pgfpathcurveto{\pgfqpoint{5.156693in}{0.829396in}}{\pgfqpoint{5.161022in}{0.839847in}}{\pgfqpoint{5.161022in}{0.850743in}}%
\pgfpathcurveto{\pgfqpoint{5.161022in}{0.861638in}}{\pgfqpoint{5.156693in}{0.872089in}}{\pgfqpoint{5.148988in}{0.879793in}}%
\pgfpathcurveto{\pgfqpoint{5.141284in}{0.887498in}}{\pgfqpoint{5.130833in}{0.891826in}}{\pgfqpoint{5.119938in}{0.891826in}}%
\pgfpathcurveto{\pgfqpoint{5.109042in}{0.891826in}}{\pgfqpoint{5.098591in}{0.887498in}}{\pgfqpoint{5.090887in}{0.879793in}}%
\pgfpathcurveto{\pgfqpoint{5.083183in}{0.872089in}}{\pgfqpoint{5.078854in}{0.861638in}}{\pgfqpoint{5.078854in}{0.850743in}}%
\pgfpathcurveto{\pgfqpoint{5.078854in}{0.839847in}}{\pgfqpoint{5.083183in}{0.829396in}}{\pgfqpoint{5.090887in}{0.821692in}}%
\pgfpathcurveto{\pgfqpoint{5.098591in}{0.813988in}}{\pgfqpoint{5.109042in}{0.809659in}}{\pgfqpoint{5.119938in}{0.809659in}}%
\pgfpathlineto{\pgfqpoint{5.119938in}{0.809659in}}%
\pgfpathclose%
\pgfusepath{stroke}%
\end{pgfscope}%
\begin{pgfscope}%
\pgfpathrectangle{\pgfqpoint{0.688192in}{0.670138in}}{\pgfqpoint{7.111808in}{5.129862in}}%
\pgfusepath{clip}%
\pgfsetbuttcap%
\pgfsetroundjoin%
\pgfsetlinewidth{1.003750pt}%
\definecolor{currentstroke}{rgb}{0.000000,0.000000,0.000000}%
\pgfsetstrokecolor{currentstroke}%
\pgfsetdash{}{0pt}%
\pgfpathmoveto{\pgfqpoint{1.071092in}{0.725847in}}%
\pgfpathcurveto{\pgfqpoint{1.081988in}{0.725847in}}{\pgfqpoint{1.092438in}{0.730176in}}{\pgfqpoint{1.100143in}{0.737880in}}%
\pgfpathcurveto{\pgfqpoint{1.107847in}{0.745585in}}{\pgfqpoint{1.112176in}{0.756035in}}{\pgfqpoint{1.112176in}{0.766931in}}%
\pgfpathcurveto{\pgfqpoint{1.112176in}{0.777826in}}{\pgfqpoint{1.107847in}{0.788277in}}{\pgfqpoint{1.100143in}{0.795982in}}%
\pgfpathcurveto{\pgfqpoint{1.092438in}{0.803686in}}{\pgfqpoint{1.081988in}{0.808015in}}{\pgfqpoint{1.071092in}{0.808015in}}%
\pgfpathcurveto{\pgfqpoint{1.060196in}{0.808015in}}{\pgfqpoint{1.049746in}{0.803686in}}{\pgfqpoint{1.042041in}{0.795982in}}%
\pgfpathcurveto{\pgfqpoint{1.034337in}{0.788277in}}{\pgfqpoint{1.030008in}{0.777826in}}{\pgfqpoint{1.030008in}{0.766931in}}%
\pgfpathcurveto{\pgfqpoint{1.030008in}{0.756035in}}{\pgfqpoint{1.034337in}{0.745585in}}{\pgfqpoint{1.042041in}{0.737880in}}%
\pgfpathcurveto{\pgfqpoint{1.049746in}{0.730176in}}{\pgfqpoint{1.060196in}{0.725847in}}{\pgfqpoint{1.071092in}{0.725847in}}%
\pgfpathlineto{\pgfqpoint{1.071092in}{0.725847in}}%
\pgfpathclose%
\pgfusepath{stroke}%
\end{pgfscope}%
\begin{pgfscope}%
\pgfpathrectangle{\pgfqpoint{0.688192in}{0.670138in}}{\pgfqpoint{7.111808in}{5.129862in}}%
\pgfusepath{clip}%
\pgfsetbuttcap%
\pgfsetroundjoin%
\pgfsetlinewidth{1.003750pt}%
\definecolor{currentstroke}{rgb}{0.000000,0.000000,0.000000}%
\pgfsetstrokecolor{currentstroke}%
\pgfsetdash{}{0pt}%
\pgfpathmoveto{\pgfqpoint{2.869035in}{0.669385in}}%
\pgfpathcurveto{\pgfqpoint{2.879931in}{0.669385in}}{\pgfqpoint{2.890382in}{0.673714in}}{\pgfqpoint{2.898086in}{0.681419in}}%
\pgfpathcurveto{\pgfqpoint{2.905790in}{0.689123in}}{\pgfqpoint{2.910119in}{0.699574in}}{\pgfqpoint{2.910119in}{0.710469in}}%
\pgfpathcurveto{\pgfqpoint{2.910119in}{0.721365in}}{\pgfqpoint{2.905790in}{0.731816in}}{\pgfqpoint{2.898086in}{0.739520in}}%
\pgfpathcurveto{\pgfqpoint{2.890382in}{0.747224in}}{\pgfqpoint{2.879931in}{0.751553in}}{\pgfqpoint{2.869035in}{0.751553in}}%
\pgfpathcurveto{\pgfqpoint{2.858140in}{0.751553in}}{\pgfqpoint{2.847689in}{0.747224in}}{\pgfqpoint{2.839985in}{0.739520in}}%
\pgfpathcurveto{\pgfqpoint{2.832280in}{0.731816in}}{\pgfqpoint{2.827952in}{0.721365in}}{\pgfqpoint{2.827952in}{0.710469in}}%
\pgfpathcurveto{\pgfqpoint{2.827952in}{0.699574in}}{\pgfqpoint{2.832280in}{0.689123in}}{\pgfqpoint{2.839985in}{0.681419in}}%
\pgfpathcurveto{\pgfqpoint{2.847689in}{0.673714in}}{\pgfqpoint{2.858140in}{0.669385in}}{\pgfqpoint{2.869035in}{0.669385in}}%
\pgfpathlineto{\pgfqpoint{2.869035in}{0.669385in}}%
\pgfpathclose%
\pgfusepath{stroke}%
\end{pgfscope}%
\begin{pgfscope}%
\pgfpathrectangle{\pgfqpoint{0.688192in}{0.670138in}}{\pgfqpoint{7.111808in}{5.129862in}}%
\pgfusepath{clip}%
\pgfsetbuttcap%
\pgfsetroundjoin%
\pgfsetlinewidth{1.003750pt}%
\definecolor{currentstroke}{rgb}{0.000000,0.000000,0.000000}%
\pgfsetstrokecolor{currentstroke}%
\pgfsetdash{}{0pt}%
\pgfpathmoveto{\pgfqpoint{0.899610in}{0.859043in}}%
\pgfpathcurveto{\pgfqpoint{0.910506in}{0.859043in}}{\pgfqpoint{0.920957in}{0.863371in}}{\pgfqpoint{0.928661in}{0.871076in}}%
\pgfpathcurveto{\pgfqpoint{0.936365in}{0.878780in}}{\pgfqpoint{0.940694in}{0.889231in}}{\pgfqpoint{0.940694in}{0.900126in}}%
\pgfpathcurveto{\pgfqpoint{0.940694in}{0.911022in}}{\pgfqpoint{0.936365in}{0.921473in}}{\pgfqpoint{0.928661in}{0.929177in}}%
\pgfpathcurveto{\pgfqpoint{0.920957in}{0.936881in}}{\pgfqpoint{0.910506in}{0.941210in}}{\pgfqpoint{0.899610in}{0.941210in}}%
\pgfpathcurveto{\pgfqpoint{0.888715in}{0.941210in}}{\pgfqpoint{0.878264in}{0.936881in}}{\pgfqpoint{0.870560in}{0.929177in}}%
\pgfpathcurveto{\pgfqpoint{0.862855in}{0.921473in}}{\pgfqpoint{0.858526in}{0.911022in}}{\pgfqpoint{0.858526in}{0.900126in}}%
\pgfpathcurveto{\pgfqpoint{0.858526in}{0.889231in}}{\pgfqpoint{0.862855in}{0.878780in}}{\pgfqpoint{0.870560in}{0.871076in}}%
\pgfpathcurveto{\pgfqpoint{0.878264in}{0.863371in}}{\pgfqpoint{0.888715in}{0.859043in}}{\pgfqpoint{0.899610in}{0.859043in}}%
\pgfpathlineto{\pgfqpoint{0.899610in}{0.859043in}}%
\pgfpathclose%
\pgfusepath{stroke}%
\end{pgfscope}%
\begin{pgfscope}%
\pgfpathrectangle{\pgfqpoint{0.688192in}{0.670138in}}{\pgfqpoint{7.111808in}{5.129862in}}%
\pgfusepath{clip}%
\pgfsetbuttcap%
\pgfsetroundjoin%
\pgfsetlinewidth{1.003750pt}%
\definecolor{currentstroke}{rgb}{0.000000,0.000000,0.000000}%
\pgfsetstrokecolor{currentstroke}%
\pgfsetdash{}{0pt}%
\pgfpathmoveto{\pgfqpoint{4.440732in}{1.890898in}}%
\pgfpathcurveto{\pgfqpoint{4.451628in}{1.890898in}}{\pgfqpoint{4.462078in}{1.895227in}}{\pgfqpoint{4.469783in}{1.902931in}}%
\pgfpathcurveto{\pgfqpoint{4.477487in}{1.910636in}}{\pgfqpoint{4.481816in}{1.921086in}}{\pgfqpoint{4.481816in}{1.931982in}}%
\pgfpathcurveto{\pgfqpoint{4.481816in}{1.942878in}}{\pgfqpoint{4.477487in}{1.953328in}}{\pgfqpoint{4.469783in}{1.961033in}}%
\pgfpathcurveto{\pgfqpoint{4.462078in}{1.968737in}}{\pgfqpoint{4.451628in}{1.973066in}}{\pgfqpoint{4.440732in}{1.973066in}}%
\pgfpathcurveto{\pgfqpoint{4.429837in}{1.973066in}}{\pgfqpoint{4.419386in}{1.968737in}}{\pgfqpoint{4.411681in}{1.961033in}}%
\pgfpathcurveto{\pgfqpoint{4.403977in}{1.953328in}}{\pgfqpoint{4.399648in}{1.942878in}}{\pgfqpoint{4.399648in}{1.931982in}}%
\pgfpathcurveto{\pgfqpoint{4.399648in}{1.921086in}}{\pgfqpoint{4.403977in}{1.910636in}}{\pgfqpoint{4.411681in}{1.902931in}}%
\pgfpathcurveto{\pgfqpoint{4.419386in}{1.895227in}}{\pgfqpoint{4.429837in}{1.890898in}}{\pgfqpoint{4.440732in}{1.890898in}}%
\pgfpathlineto{\pgfqpoint{4.440732in}{1.890898in}}%
\pgfpathclose%
\pgfusepath{stroke}%
\end{pgfscope}%
\begin{pgfscope}%
\pgfpathrectangle{\pgfqpoint{0.688192in}{0.670138in}}{\pgfqpoint{7.111808in}{5.129862in}}%
\pgfusepath{clip}%
\pgfsetbuttcap%
\pgfsetroundjoin%
\pgfsetlinewidth{1.003750pt}%
\definecolor{currentstroke}{rgb}{0.000000,0.000000,0.000000}%
\pgfsetstrokecolor{currentstroke}%
\pgfsetdash{}{0pt}%
\pgfpathmoveto{\pgfqpoint{1.050628in}{0.739811in}}%
\pgfpathcurveto{\pgfqpoint{1.061524in}{0.739811in}}{\pgfqpoint{1.071974in}{0.744140in}}{\pgfqpoint{1.079679in}{0.751844in}}%
\pgfpathcurveto{\pgfqpoint{1.087383in}{0.759549in}}{\pgfqpoint{1.091712in}{0.769999in}}{\pgfqpoint{1.091712in}{0.780895in}}%
\pgfpathcurveto{\pgfqpoint{1.091712in}{0.791790in}}{\pgfqpoint{1.087383in}{0.802241in}}{\pgfqpoint{1.079679in}{0.809946in}}%
\pgfpathcurveto{\pgfqpoint{1.071974in}{0.817650in}}{\pgfqpoint{1.061524in}{0.821979in}}{\pgfqpoint{1.050628in}{0.821979in}}%
\pgfpathcurveto{\pgfqpoint{1.039732in}{0.821979in}}{\pgfqpoint{1.029282in}{0.817650in}}{\pgfqpoint{1.021577in}{0.809946in}}%
\pgfpathcurveto{\pgfqpoint{1.013873in}{0.802241in}}{\pgfqpoint{1.009544in}{0.791790in}}{\pgfqpoint{1.009544in}{0.780895in}}%
\pgfpathcurveto{\pgfqpoint{1.009544in}{0.769999in}}{\pgfqpoint{1.013873in}{0.759549in}}{\pgfqpoint{1.021577in}{0.751844in}}%
\pgfpathcurveto{\pgfqpoint{1.029282in}{0.744140in}}{\pgfqpoint{1.039732in}{0.739811in}}{\pgfqpoint{1.050628in}{0.739811in}}%
\pgfpathlineto{\pgfqpoint{1.050628in}{0.739811in}}%
\pgfpathclose%
\pgfusepath{stroke}%
\end{pgfscope}%
\begin{pgfscope}%
\pgfpathrectangle{\pgfqpoint{0.688192in}{0.670138in}}{\pgfqpoint{7.111808in}{5.129862in}}%
\pgfusepath{clip}%
\pgfsetbuttcap%
\pgfsetroundjoin%
\pgfsetlinewidth{1.003750pt}%
\definecolor{currentstroke}{rgb}{0.000000,0.000000,0.000000}%
\pgfsetstrokecolor{currentstroke}%
\pgfsetdash{}{0pt}%
\pgfpathmoveto{\pgfqpoint{1.308402in}{0.713449in}}%
\pgfpathcurveto{\pgfqpoint{1.319298in}{0.713449in}}{\pgfqpoint{1.329748in}{0.717778in}}{\pgfqpoint{1.337453in}{0.725482in}}%
\pgfpathcurveto{\pgfqpoint{1.345157in}{0.733187in}}{\pgfqpoint{1.349486in}{0.743637in}}{\pgfqpoint{1.349486in}{0.754533in}}%
\pgfpathcurveto{\pgfqpoint{1.349486in}{0.765429in}}{\pgfqpoint{1.345157in}{0.775879in}}{\pgfqpoint{1.337453in}{0.783584in}}%
\pgfpathcurveto{\pgfqpoint{1.329748in}{0.791288in}}{\pgfqpoint{1.319298in}{0.795617in}}{\pgfqpoint{1.308402in}{0.795617in}}%
\pgfpathcurveto{\pgfqpoint{1.297506in}{0.795617in}}{\pgfqpoint{1.287056in}{0.791288in}}{\pgfqpoint{1.279351in}{0.783584in}}%
\pgfpathcurveto{\pgfqpoint{1.271647in}{0.775879in}}{\pgfqpoint{1.267318in}{0.765429in}}{\pgfqpoint{1.267318in}{0.754533in}}%
\pgfpathcurveto{\pgfqpoint{1.267318in}{0.743637in}}{\pgfqpoint{1.271647in}{0.733187in}}{\pgfqpoint{1.279351in}{0.725482in}}%
\pgfpathcurveto{\pgfqpoint{1.287056in}{0.717778in}}{\pgfqpoint{1.297506in}{0.713449in}}{\pgfqpoint{1.308402in}{0.713449in}}%
\pgfpathlineto{\pgfqpoint{1.308402in}{0.713449in}}%
\pgfpathclose%
\pgfusepath{stroke}%
\end{pgfscope}%
\begin{pgfscope}%
\pgfpathrectangle{\pgfqpoint{0.688192in}{0.670138in}}{\pgfqpoint{7.111808in}{5.129862in}}%
\pgfusepath{clip}%
\pgfsetbuttcap%
\pgfsetroundjoin%
\pgfsetlinewidth{1.003750pt}%
\definecolor{currentstroke}{rgb}{0.000000,0.000000,0.000000}%
\pgfsetstrokecolor{currentstroke}%
\pgfsetdash{}{0pt}%
\pgfpathmoveto{\pgfqpoint{3.526992in}{0.944388in}}%
\pgfpathcurveto{\pgfqpoint{3.537887in}{0.944388in}}{\pgfqpoint{3.548338in}{0.948717in}}{\pgfqpoint{3.556042in}{0.956421in}}%
\pgfpathcurveto{\pgfqpoint{3.563747in}{0.964125in}}{\pgfqpoint{3.568076in}{0.974576in}}{\pgfqpoint{3.568076in}{0.985472in}}%
\pgfpathcurveto{\pgfqpoint{3.568076in}{0.996367in}}{\pgfqpoint{3.563747in}{1.006818in}}{\pgfqpoint{3.556042in}{1.014522in}}%
\pgfpathcurveto{\pgfqpoint{3.548338in}{1.022227in}}{\pgfqpoint{3.537887in}{1.026556in}}{\pgfqpoint{3.526992in}{1.026556in}}%
\pgfpathcurveto{\pgfqpoint{3.516096in}{1.026556in}}{\pgfqpoint{3.505645in}{1.022227in}}{\pgfqpoint{3.497941in}{1.014522in}}%
\pgfpathcurveto{\pgfqpoint{3.490237in}{1.006818in}}{\pgfqpoint{3.485908in}{0.996367in}}{\pgfqpoint{3.485908in}{0.985472in}}%
\pgfpathcurveto{\pgfqpoint{3.485908in}{0.974576in}}{\pgfqpoint{3.490237in}{0.964125in}}{\pgfqpoint{3.497941in}{0.956421in}}%
\pgfpathcurveto{\pgfqpoint{3.505645in}{0.948717in}}{\pgfqpoint{3.516096in}{0.944388in}}{\pgfqpoint{3.526992in}{0.944388in}}%
\pgfpathlineto{\pgfqpoint{3.526992in}{0.944388in}}%
\pgfpathclose%
\pgfusepath{stroke}%
\end{pgfscope}%
\begin{pgfscope}%
\pgfpathrectangle{\pgfqpoint{0.688192in}{0.670138in}}{\pgfqpoint{7.111808in}{5.129862in}}%
\pgfusepath{clip}%
\pgfsetbuttcap%
\pgfsetroundjoin%
\pgfsetlinewidth{1.003750pt}%
\definecolor{currentstroke}{rgb}{0.000000,0.000000,0.000000}%
\pgfsetstrokecolor{currentstroke}%
\pgfsetdash{}{0pt}%
\pgfpathmoveto{\pgfqpoint{1.606745in}{0.705897in}}%
\pgfpathcurveto{\pgfqpoint{1.617640in}{0.705897in}}{\pgfqpoint{1.628091in}{0.710226in}}{\pgfqpoint{1.635796in}{0.717930in}}%
\pgfpathcurveto{\pgfqpoint{1.643500in}{0.725635in}}{\pgfqpoint{1.647829in}{0.736085in}}{\pgfqpoint{1.647829in}{0.746981in}}%
\pgfpathcurveto{\pgfqpoint{1.647829in}{0.757877in}}{\pgfqpoint{1.643500in}{0.768327in}}{\pgfqpoint{1.635796in}{0.776032in}}%
\pgfpathcurveto{\pgfqpoint{1.628091in}{0.783736in}}{\pgfqpoint{1.617640in}{0.788065in}}{\pgfqpoint{1.606745in}{0.788065in}}%
\pgfpathcurveto{\pgfqpoint{1.595849in}{0.788065in}}{\pgfqpoint{1.585399in}{0.783736in}}{\pgfqpoint{1.577694in}{0.776032in}}%
\pgfpathcurveto{\pgfqpoint{1.569990in}{0.768327in}}{\pgfqpoint{1.565661in}{0.757877in}}{\pgfqpoint{1.565661in}{0.746981in}}%
\pgfpathcurveto{\pgfqpoint{1.565661in}{0.736085in}}{\pgfqpoint{1.569990in}{0.725635in}}{\pgfqpoint{1.577694in}{0.717930in}}%
\pgfpathcurveto{\pgfqpoint{1.585399in}{0.710226in}}{\pgfqpoint{1.595849in}{0.705897in}}{\pgfqpoint{1.606745in}{0.705897in}}%
\pgfpathlineto{\pgfqpoint{1.606745in}{0.705897in}}%
\pgfpathclose%
\pgfusepath{stroke}%
\end{pgfscope}%
\begin{pgfscope}%
\pgfpathrectangle{\pgfqpoint{0.688192in}{0.670138in}}{\pgfqpoint{7.111808in}{5.129862in}}%
\pgfusepath{clip}%
\pgfsetbuttcap%
\pgfsetroundjoin%
\pgfsetlinewidth{1.003750pt}%
\definecolor{currentstroke}{rgb}{0.000000,0.000000,0.000000}%
\pgfsetstrokecolor{currentstroke}%
\pgfsetdash{}{0pt}%
\pgfpathmoveto{\pgfqpoint{2.382807in}{0.678670in}}%
\pgfpathcurveto{\pgfqpoint{2.393703in}{0.678670in}}{\pgfqpoint{2.404154in}{0.682999in}}{\pgfqpoint{2.411858in}{0.690703in}}%
\pgfpathcurveto{\pgfqpoint{2.419562in}{0.698407in}}{\pgfqpoint{2.423891in}{0.708858in}}{\pgfqpoint{2.423891in}{0.719754in}}%
\pgfpathcurveto{\pgfqpoint{2.423891in}{0.730649in}}{\pgfqpoint{2.419562in}{0.741100in}}{\pgfqpoint{2.411858in}{0.748804in}}%
\pgfpathcurveto{\pgfqpoint{2.404154in}{0.756509in}}{\pgfqpoint{2.393703in}{0.760838in}}{\pgfqpoint{2.382807in}{0.760838in}}%
\pgfpathcurveto{\pgfqpoint{2.371912in}{0.760838in}}{\pgfqpoint{2.361461in}{0.756509in}}{\pgfqpoint{2.353757in}{0.748804in}}%
\pgfpathcurveto{\pgfqpoint{2.346052in}{0.741100in}}{\pgfqpoint{2.341724in}{0.730649in}}{\pgfqpoint{2.341724in}{0.719754in}}%
\pgfpathcurveto{\pgfqpoint{2.341724in}{0.708858in}}{\pgfqpoint{2.346052in}{0.698407in}}{\pgfqpoint{2.353757in}{0.690703in}}%
\pgfpathcurveto{\pgfqpoint{2.361461in}{0.682999in}}{\pgfqpoint{2.371912in}{0.678670in}}{\pgfqpoint{2.382807in}{0.678670in}}%
\pgfpathlineto{\pgfqpoint{2.382807in}{0.678670in}}%
\pgfpathclose%
\pgfusepath{stroke}%
\end{pgfscope}%
\begin{pgfscope}%
\pgfpathrectangle{\pgfqpoint{0.688192in}{0.670138in}}{\pgfqpoint{7.111808in}{5.129862in}}%
\pgfusepath{clip}%
\pgfsetbuttcap%
\pgfsetroundjoin%
\pgfsetlinewidth{1.003750pt}%
\definecolor{currentstroke}{rgb}{0.000000,0.000000,0.000000}%
\pgfsetstrokecolor{currentstroke}%
\pgfsetdash{}{0pt}%
\pgfpathmoveto{\pgfqpoint{1.679156in}{0.702407in}}%
\pgfpathcurveto{\pgfqpoint{1.690051in}{0.702407in}}{\pgfqpoint{1.700502in}{0.706736in}}{\pgfqpoint{1.708206in}{0.714440in}}%
\pgfpathcurveto{\pgfqpoint{1.715911in}{0.722145in}}{\pgfqpoint{1.720240in}{0.732595in}}{\pgfqpoint{1.720240in}{0.743491in}}%
\pgfpathcurveto{\pgfqpoint{1.720240in}{0.754387in}}{\pgfqpoint{1.715911in}{0.764837in}}{\pgfqpoint{1.708206in}{0.772542in}}%
\pgfpathcurveto{\pgfqpoint{1.700502in}{0.780246in}}{\pgfqpoint{1.690051in}{0.784575in}}{\pgfqpoint{1.679156in}{0.784575in}}%
\pgfpathcurveto{\pgfqpoint{1.668260in}{0.784575in}}{\pgfqpoint{1.657809in}{0.780246in}}{\pgfqpoint{1.650105in}{0.772542in}}%
\pgfpathcurveto{\pgfqpoint{1.642401in}{0.764837in}}{\pgfqpoint{1.638072in}{0.754387in}}{\pgfqpoint{1.638072in}{0.743491in}}%
\pgfpathcurveto{\pgfqpoint{1.638072in}{0.732595in}}{\pgfqpoint{1.642401in}{0.722145in}}{\pgfqpoint{1.650105in}{0.714440in}}%
\pgfpathcurveto{\pgfqpoint{1.657809in}{0.706736in}}{\pgfqpoint{1.668260in}{0.702407in}}{\pgfqpoint{1.679156in}{0.702407in}}%
\pgfpathlineto{\pgfqpoint{1.679156in}{0.702407in}}%
\pgfpathclose%
\pgfusepath{stroke}%
\end{pgfscope}%
\begin{pgfscope}%
\pgfpathrectangle{\pgfqpoint{0.688192in}{0.670138in}}{\pgfqpoint{7.111808in}{5.129862in}}%
\pgfusepath{clip}%
\pgfsetbuttcap%
\pgfsetroundjoin%
\pgfsetlinewidth{1.003750pt}%
\definecolor{currentstroke}{rgb}{0.000000,0.000000,0.000000}%
\pgfsetstrokecolor{currentstroke}%
\pgfsetdash{}{0pt}%
\pgfpathmoveto{\pgfqpoint{4.586127in}{0.640844in}}%
\pgfpathcurveto{\pgfqpoint{4.597023in}{0.640844in}}{\pgfqpoint{4.607473in}{0.645173in}}{\pgfqpoint{4.615178in}{0.652877in}}%
\pgfpathcurveto{\pgfqpoint{4.622882in}{0.660582in}}{\pgfqpoint{4.627211in}{0.671032in}}{\pgfqpoint{4.627211in}{0.681928in}}%
\pgfpathcurveto{\pgfqpoint{4.627211in}{0.692824in}}{\pgfqpoint{4.622882in}{0.703274in}}{\pgfqpoint{4.615178in}{0.710979in}}%
\pgfpathcurveto{\pgfqpoint{4.607473in}{0.718683in}}{\pgfqpoint{4.597023in}{0.723012in}}{\pgfqpoint{4.586127in}{0.723012in}}%
\pgfpathcurveto{\pgfqpoint{4.575231in}{0.723012in}}{\pgfqpoint{4.564781in}{0.718683in}}{\pgfqpoint{4.557076in}{0.710979in}}%
\pgfpathcurveto{\pgfqpoint{4.549372in}{0.703274in}}{\pgfqpoint{4.545043in}{0.692824in}}{\pgfqpoint{4.545043in}{0.681928in}}%
\pgfpathcurveto{\pgfqpoint{4.545043in}{0.671032in}}{\pgfqpoint{4.549372in}{0.660582in}}{\pgfqpoint{4.557076in}{0.652877in}}%
\pgfpathcurveto{\pgfqpoint{4.564781in}{0.645173in}}{\pgfqpoint{4.575231in}{0.640844in}}{\pgfqpoint{4.586127in}{0.640844in}}%
\pgfusepath{stroke}%
\end{pgfscope}%
\begin{pgfscope}%
\pgfpathrectangle{\pgfqpoint{0.688192in}{0.670138in}}{\pgfqpoint{7.111808in}{5.129862in}}%
\pgfusepath{clip}%
\pgfsetbuttcap%
\pgfsetroundjoin%
\pgfsetlinewidth{1.003750pt}%
\definecolor{currentstroke}{rgb}{0.000000,0.000000,0.000000}%
\pgfsetstrokecolor{currentstroke}%
\pgfsetdash{}{0pt}%
\pgfpathmoveto{\pgfqpoint{1.441321in}{0.708382in}}%
\pgfpathcurveto{\pgfqpoint{1.452217in}{0.708382in}}{\pgfqpoint{1.462667in}{0.712711in}}{\pgfqpoint{1.470372in}{0.720416in}}%
\pgfpathcurveto{\pgfqpoint{1.478076in}{0.728120in}}{\pgfqpoint{1.482405in}{0.738571in}}{\pgfqpoint{1.482405in}{0.749466in}}%
\pgfpathcurveto{\pgfqpoint{1.482405in}{0.760362in}}{\pgfqpoint{1.478076in}{0.770813in}}{\pgfqpoint{1.470372in}{0.778517in}}%
\pgfpathcurveto{\pgfqpoint{1.462667in}{0.786221in}}{\pgfqpoint{1.452217in}{0.790550in}}{\pgfqpoint{1.441321in}{0.790550in}}%
\pgfpathcurveto{\pgfqpoint{1.430425in}{0.790550in}}{\pgfqpoint{1.419975in}{0.786221in}}{\pgfqpoint{1.412270in}{0.778517in}}%
\pgfpathcurveto{\pgfqpoint{1.404566in}{0.770813in}}{\pgfqpoint{1.400237in}{0.760362in}}{\pgfqpoint{1.400237in}{0.749466in}}%
\pgfpathcurveto{\pgfqpoint{1.400237in}{0.738571in}}{\pgfqpoint{1.404566in}{0.728120in}}{\pgfqpoint{1.412270in}{0.720416in}}%
\pgfpathcurveto{\pgfqpoint{1.419975in}{0.712711in}}{\pgfqpoint{1.430425in}{0.708382in}}{\pgfqpoint{1.441321in}{0.708382in}}%
\pgfpathlineto{\pgfqpoint{1.441321in}{0.708382in}}%
\pgfpathclose%
\pgfusepath{stroke}%
\end{pgfscope}%
\begin{pgfscope}%
\pgfpathrectangle{\pgfqpoint{0.688192in}{0.670138in}}{\pgfqpoint{7.111808in}{5.129862in}}%
\pgfusepath{clip}%
\pgfsetbuttcap%
\pgfsetroundjoin%
\pgfsetlinewidth{1.003750pt}%
\definecolor{currentstroke}{rgb}{0.000000,0.000000,0.000000}%
\pgfsetstrokecolor{currentstroke}%
\pgfsetdash{}{0pt}%
\pgfpathmoveto{\pgfqpoint{1.278130in}{0.714122in}}%
\pgfpathcurveto{\pgfqpoint{1.289026in}{0.714122in}}{\pgfqpoint{1.299477in}{0.718451in}}{\pgfqpoint{1.307181in}{0.726155in}}%
\pgfpathcurveto{\pgfqpoint{1.314885in}{0.733860in}}{\pgfqpoint{1.319214in}{0.744311in}}{\pgfqpoint{1.319214in}{0.755206in}}%
\pgfpathcurveto{\pgfqpoint{1.319214in}{0.766102in}}{\pgfqpoint{1.314885in}{0.776553in}}{\pgfqpoint{1.307181in}{0.784257in}}%
\pgfpathcurveto{\pgfqpoint{1.299477in}{0.791961in}}{\pgfqpoint{1.289026in}{0.796290in}}{\pgfqpoint{1.278130in}{0.796290in}}%
\pgfpathcurveto{\pgfqpoint{1.267235in}{0.796290in}}{\pgfqpoint{1.256784in}{0.791961in}}{\pgfqpoint{1.249080in}{0.784257in}}%
\pgfpathcurveto{\pgfqpoint{1.241375in}{0.776553in}}{\pgfqpoint{1.237047in}{0.766102in}}{\pgfqpoint{1.237047in}{0.755206in}}%
\pgfpathcurveto{\pgfqpoint{1.237047in}{0.744311in}}{\pgfqpoint{1.241375in}{0.733860in}}{\pgfqpoint{1.249080in}{0.726155in}}%
\pgfpathcurveto{\pgfqpoint{1.256784in}{0.718451in}}{\pgfqpoint{1.267235in}{0.714122in}}{\pgfqpoint{1.278130in}{0.714122in}}%
\pgfpathlineto{\pgfqpoint{1.278130in}{0.714122in}}%
\pgfpathclose%
\pgfusepath{stroke}%
\end{pgfscope}%
\begin{pgfscope}%
\pgfpathrectangle{\pgfqpoint{0.688192in}{0.670138in}}{\pgfqpoint{7.111808in}{5.129862in}}%
\pgfusepath{clip}%
\pgfsetbuttcap%
\pgfsetroundjoin%
\pgfsetlinewidth{1.003750pt}%
\definecolor{currentstroke}{rgb}{0.000000,0.000000,0.000000}%
\pgfsetstrokecolor{currentstroke}%
\pgfsetdash{}{0pt}%
\pgfpathmoveto{\pgfqpoint{6.308571in}{3.902956in}}%
\pgfpathcurveto{\pgfqpoint{6.319467in}{3.902956in}}{\pgfqpoint{6.329918in}{3.907285in}}{\pgfqpoint{6.337622in}{3.914989in}}%
\pgfpathcurveto{\pgfqpoint{6.345326in}{3.922694in}}{\pgfqpoint{6.349655in}{3.933144in}}{\pgfqpoint{6.349655in}{3.944040in}}%
\pgfpathcurveto{\pgfqpoint{6.349655in}{3.954936in}}{\pgfqpoint{6.345326in}{3.965386in}}{\pgfqpoint{6.337622in}{3.973091in}}%
\pgfpathcurveto{\pgfqpoint{6.329918in}{3.980795in}}{\pgfqpoint{6.319467in}{3.985124in}}{\pgfqpoint{6.308571in}{3.985124in}}%
\pgfpathcurveto{\pgfqpoint{6.297676in}{3.985124in}}{\pgfqpoint{6.287225in}{3.980795in}}{\pgfqpoint{6.279520in}{3.973091in}}%
\pgfpathcurveto{\pgfqpoint{6.271816in}{3.965386in}}{\pgfqpoint{6.267487in}{3.954936in}}{\pgfqpoint{6.267487in}{3.944040in}}%
\pgfpathcurveto{\pgfqpoint{6.267487in}{3.933144in}}{\pgfqpoint{6.271816in}{3.922694in}}{\pgfqpoint{6.279520in}{3.914989in}}%
\pgfpathcurveto{\pgfqpoint{6.287225in}{3.907285in}}{\pgfqpoint{6.297676in}{3.902956in}}{\pgfqpoint{6.308571in}{3.902956in}}%
\pgfpathlineto{\pgfqpoint{6.308571in}{3.902956in}}%
\pgfpathclose%
\pgfusepath{stroke}%
\end{pgfscope}%
\begin{pgfscope}%
\pgfpathrectangle{\pgfqpoint{0.688192in}{0.670138in}}{\pgfqpoint{7.111808in}{5.129862in}}%
\pgfusepath{clip}%
\pgfsetbuttcap%
\pgfsetroundjoin%
\pgfsetlinewidth{1.003750pt}%
\definecolor{currentstroke}{rgb}{0.000000,0.000000,0.000000}%
\pgfsetstrokecolor{currentstroke}%
\pgfsetdash{}{0pt}%
\pgfpathmoveto{\pgfqpoint{1.840914in}{0.696069in}}%
\pgfpathcurveto{\pgfqpoint{1.851809in}{0.696069in}}{\pgfqpoint{1.862260in}{0.700398in}}{\pgfqpoint{1.869964in}{0.708103in}}%
\pgfpathcurveto{\pgfqpoint{1.877669in}{0.715807in}}{\pgfqpoint{1.881997in}{0.726258in}}{\pgfqpoint{1.881997in}{0.737153in}}%
\pgfpathcurveto{\pgfqpoint{1.881997in}{0.748049in}}{\pgfqpoint{1.877669in}{0.758500in}}{\pgfqpoint{1.869964in}{0.766204in}}%
\pgfpathcurveto{\pgfqpoint{1.862260in}{0.773908in}}{\pgfqpoint{1.851809in}{0.778237in}}{\pgfqpoint{1.840914in}{0.778237in}}%
\pgfpathcurveto{\pgfqpoint{1.830018in}{0.778237in}}{\pgfqpoint{1.819567in}{0.773908in}}{\pgfqpoint{1.811863in}{0.766204in}}%
\pgfpathcurveto{\pgfqpoint{1.804159in}{0.758500in}}{\pgfqpoint{1.799830in}{0.748049in}}{\pgfqpoint{1.799830in}{0.737153in}}%
\pgfpathcurveto{\pgfqpoint{1.799830in}{0.726258in}}{\pgfqpoint{1.804159in}{0.715807in}}{\pgfqpoint{1.811863in}{0.708103in}}%
\pgfpathcurveto{\pgfqpoint{1.819567in}{0.700398in}}{\pgfqpoint{1.830018in}{0.696069in}}{\pgfqpoint{1.840914in}{0.696069in}}%
\pgfpathlineto{\pgfqpoint{1.840914in}{0.696069in}}%
\pgfpathclose%
\pgfusepath{stroke}%
\end{pgfscope}%
\begin{pgfscope}%
\pgfpathrectangle{\pgfqpoint{0.688192in}{0.670138in}}{\pgfqpoint{7.111808in}{5.129862in}}%
\pgfusepath{clip}%
\pgfsetbuttcap%
\pgfsetroundjoin%
\pgfsetlinewidth{1.003750pt}%
\definecolor{currentstroke}{rgb}{0.000000,0.000000,0.000000}%
\pgfsetstrokecolor{currentstroke}%
\pgfsetdash{}{0pt}%
\pgfpathmoveto{\pgfqpoint{5.261268in}{0.632886in}}%
\pgfpathcurveto{\pgfqpoint{5.272163in}{0.632886in}}{\pgfqpoint{5.282614in}{0.637215in}}{\pgfqpoint{5.290319in}{0.644920in}}%
\pgfpathcurveto{\pgfqpoint{5.298023in}{0.652624in}}{\pgfqpoint{5.302352in}{0.663075in}}{\pgfqpoint{5.302352in}{0.673970in}}%
\pgfpathcurveto{\pgfqpoint{5.302352in}{0.684866in}}{\pgfqpoint{5.298023in}{0.695317in}}{\pgfqpoint{5.290319in}{0.703021in}}%
\pgfpathcurveto{\pgfqpoint{5.282614in}{0.710725in}}{\pgfqpoint{5.272163in}{0.715054in}}{\pgfqpoint{5.261268in}{0.715054in}}%
\pgfpathcurveto{\pgfqpoint{5.250372in}{0.715054in}}{\pgfqpoint{5.239921in}{0.710725in}}{\pgfqpoint{5.232217in}{0.703021in}}%
\pgfpathcurveto{\pgfqpoint{5.224513in}{0.695317in}}{\pgfqpoint{5.220184in}{0.684866in}}{\pgfqpoint{5.220184in}{0.673970in}}%
\pgfpathcurveto{\pgfqpoint{5.220184in}{0.663075in}}{\pgfqpoint{5.224513in}{0.652624in}}{\pgfqpoint{5.232217in}{0.644920in}}%
\pgfpathcurveto{\pgfqpoint{5.239921in}{0.637215in}}{\pgfqpoint{5.250372in}{0.632886in}}{\pgfqpoint{5.261268in}{0.632886in}}%
\pgfusepath{stroke}%
\end{pgfscope}%
\begin{pgfscope}%
\pgfpathrectangle{\pgfqpoint{0.688192in}{0.670138in}}{\pgfqpoint{7.111808in}{5.129862in}}%
\pgfusepath{clip}%
\pgfsetbuttcap%
\pgfsetroundjoin%
\pgfsetlinewidth{1.003750pt}%
\definecolor{currentstroke}{rgb}{0.000000,0.000000,0.000000}%
\pgfsetstrokecolor{currentstroke}%
\pgfsetdash{}{0pt}%
\pgfpathmoveto{\pgfqpoint{0.886403in}{0.869810in}}%
\pgfpathcurveto{\pgfqpoint{0.897298in}{0.869810in}}{\pgfqpoint{0.907749in}{0.874139in}}{\pgfqpoint{0.915453in}{0.881844in}}%
\pgfpathcurveto{\pgfqpoint{0.923158in}{0.889548in}}{\pgfqpoint{0.927487in}{0.899999in}}{\pgfqpoint{0.927487in}{0.910894in}}%
\pgfpathcurveto{\pgfqpoint{0.927487in}{0.921790in}}{\pgfqpoint{0.923158in}{0.932241in}}{\pgfqpoint{0.915453in}{0.939945in}}%
\pgfpathcurveto{\pgfqpoint{0.907749in}{0.947649in}}{\pgfqpoint{0.897298in}{0.951978in}}{\pgfqpoint{0.886403in}{0.951978in}}%
\pgfpathcurveto{\pgfqpoint{0.875507in}{0.951978in}}{\pgfqpoint{0.865056in}{0.947649in}}{\pgfqpoint{0.857352in}{0.939945in}}%
\pgfpathcurveto{\pgfqpoint{0.849648in}{0.932241in}}{\pgfqpoint{0.845319in}{0.921790in}}{\pgfqpoint{0.845319in}{0.910894in}}%
\pgfpathcurveto{\pgfqpoint{0.845319in}{0.899999in}}{\pgfqpoint{0.849648in}{0.889548in}}{\pgfqpoint{0.857352in}{0.881844in}}%
\pgfpathcurveto{\pgfqpoint{0.865056in}{0.874139in}}{\pgfqpoint{0.875507in}{0.869810in}}{\pgfqpoint{0.886403in}{0.869810in}}%
\pgfpathlineto{\pgfqpoint{0.886403in}{0.869810in}}%
\pgfpathclose%
\pgfusepath{stroke}%
\end{pgfscope}%
\begin{pgfscope}%
\pgfpathrectangle{\pgfqpoint{0.688192in}{0.670138in}}{\pgfqpoint{7.111808in}{5.129862in}}%
\pgfusepath{clip}%
\pgfsetbuttcap%
\pgfsetroundjoin%
\pgfsetlinewidth{1.003750pt}%
\definecolor{currentstroke}{rgb}{0.000000,0.000000,0.000000}%
\pgfsetstrokecolor{currentstroke}%
\pgfsetdash{}{0pt}%
\pgfpathmoveto{\pgfqpoint{1.419139in}{0.710712in}}%
\pgfpathcurveto{\pgfqpoint{1.430035in}{0.710712in}}{\pgfqpoint{1.440485in}{0.715041in}}{\pgfqpoint{1.448190in}{0.722745in}}%
\pgfpathcurveto{\pgfqpoint{1.455894in}{0.730450in}}{\pgfqpoint{1.460223in}{0.740900in}}{\pgfqpoint{1.460223in}{0.751796in}}%
\pgfpathcurveto{\pgfqpoint{1.460223in}{0.762691in}}{\pgfqpoint{1.455894in}{0.773142in}}{\pgfqpoint{1.448190in}{0.780847in}}%
\pgfpathcurveto{\pgfqpoint{1.440485in}{0.788551in}}{\pgfqpoint{1.430035in}{0.792880in}}{\pgfqpoint{1.419139in}{0.792880in}}%
\pgfpathcurveto{\pgfqpoint{1.408244in}{0.792880in}}{\pgfqpoint{1.397793in}{0.788551in}}{\pgfqpoint{1.390088in}{0.780847in}}%
\pgfpathcurveto{\pgfqpoint{1.382384in}{0.773142in}}{\pgfqpoint{1.378055in}{0.762691in}}{\pgfqpoint{1.378055in}{0.751796in}}%
\pgfpathcurveto{\pgfqpoint{1.378055in}{0.740900in}}{\pgfqpoint{1.382384in}{0.730450in}}{\pgfqpoint{1.390088in}{0.722745in}}%
\pgfpathcurveto{\pgfqpoint{1.397793in}{0.715041in}}{\pgfqpoint{1.408244in}{0.710712in}}{\pgfqpoint{1.419139in}{0.710712in}}%
\pgfpathlineto{\pgfqpoint{1.419139in}{0.710712in}}%
\pgfpathclose%
\pgfusepath{stroke}%
\end{pgfscope}%
\begin{pgfscope}%
\pgfpathrectangle{\pgfqpoint{0.688192in}{0.670138in}}{\pgfqpoint{7.111808in}{5.129862in}}%
\pgfusepath{clip}%
\pgfsetbuttcap%
\pgfsetroundjoin%
\pgfsetlinewidth{1.003750pt}%
\definecolor{currentstroke}{rgb}{0.000000,0.000000,0.000000}%
\pgfsetstrokecolor{currentstroke}%
\pgfsetdash{}{0pt}%
\pgfpathmoveto{\pgfqpoint{1.417824in}{0.710820in}}%
\pgfpathcurveto{\pgfqpoint{1.428720in}{0.710820in}}{\pgfqpoint{1.439170in}{0.715149in}}{\pgfqpoint{1.446875in}{0.722854in}}%
\pgfpathcurveto{\pgfqpoint{1.454579in}{0.730558in}}{\pgfqpoint{1.458908in}{0.741009in}}{\pgfqpoint{1.458908in}{0.751904in}}%
\pgfpathcurveto{\pgfqpoint{1.458908in}{0.762800in}}{\pgfqpoint{1.454579in}{0.773251in}}{\pgfqpoint{1.446875in}{0.780955in}}%
\pgfpathcurveto{\pgfqpoint{1.439170in}{0.788659in}}{\pgfqpoint{1.428720in}{0.792988in}}{\pgfqpoint{1.417824in}{0.792988in}}%
\pgfpathcurveto{\pgfqpoint{1.406928in}{0.792988in}}{\pgfqpoint{1.396478in}{0.788659in}}{\pgfqpoint{1.388773in}{0.780955in}}%
\pgfpathcurveto{\pgfqpoint{1.381069in}{0.773251in}}{\pgfqpoint{1.376740in}{0.762800in}}{\pgfqpoint{1.376740in}{0.751904in}}%
\pgfpathcurveto{\pgfqpoint{1.376740in}{0.741009in}}{\pgfqpoint{1.381069in}{0.730558in}}{\pgfqpoint{1.388773in}{0.722854in}}%
\pgfpathcurveto{\pgfqpoint{1.396478in}{0.715149in}}{\pgfqpoint{1.406928in}{0.710820in}}{\pgfqpoint{1.417824in}{0.710820in}}%
\pgfpathlineto{\pgfqpoint{1.417824in}{0.710820in}}%
\pgfpathclose%
\pgfusepath{stroke}%
\end{pgfscope}%
\begin{pgfscope}%
\pgfpathrectangle{\pgfqpoint{0.688192in}{0.670138in}}{\pgfqpoint{7.111808in}{5.129862in}}%
\pgfusepath{clip}%
\pgfsetbuttcap%
\pgfsetroundjoin%
\pgfsetlinewidth{1.003750pt}%
\definecolor{currentstroke}{rgb}{0.000000,0.000000,0.000000}%
\pgfsetstrokecolor{currentstroke}%
\pgfsetdash{}{0pt}%
\pgfpathmoveto{\pgfqpoint{2.217170in}{0.684781in}}%
\pgfpathcurveto{\pgfqpoint{2.228066in}{0.684781in}}{\pgfqpoint{2.238517in}{0.689109in}}{\pgfqpoint{2.246221in}{0.696814in}}%
\pgfpathcurveto{\pgfqpoint{2.253925in}{0.704518in}}{\pgfqpoint{2.258254in}{0.714969in}}{\pgfqpoint{2.258254in}{0.725864in}}%
\pgfpathcurveto{\pgfqpoint{2.258254in}{0.736760in}}{\pgfqpoint{2.253925in}{0.747211in}}{\pgfqpoint{2.246221in}{0.754915in}}%
\pgfpathcurveto{\pgfqpoint{2.238517in}{0.762620in}}{\pgfqpoint{2.228066in}{0.766948in}}{\pgfqpoint{2.217170in}{0.766948in}}%
\pgfpathcurveto{\pgfqpoint{2.206275in}{0.766948in}}{\pgfqpoint{2.195824in}{0.762620in}}{\pgfqpoint{2.188120in}{0.754915in}}%
\pgfpathcurveto{\pgfqpoint{2.180415in}{0.747211in}}{\pgfqpoint{2.176087in}{0.736760in}}{\pgfqpoint{2.176087in}{0.725864in}}%
\pgfpathcurveto{\pgfqpoint{2.176087in}{0.714969in}}{\pgfqpoint{2.180415in}{0.704518in}}{\pgfqpoint{2.188120in}{0.696814in}}%
\pgfpathcurveto{\pgfqpoint{2.195824in}{0.689109in}}{\pgfqpoint{2.206275in}{0.684781in}}{\pgfqpoint{2.217170in}{0.684781in}}%
\pgfpathlineto{\pgfqpoint{2.217170in}{0.684781in}}%
\pgfpathclose%
\pgfusepath{stroke}%
\end{pgfscope}%
\begin{pgfscope}%
\pgfpathrectangle{\pgfqpoint{0.688192in}{0.670138in}}{\pgfqpoint{7.111808in}{5.129862in}}%
\pgfusepath{clip}%
\pgfsetbuttcap%
\pgfsetroundjoin%
\pgfsetlinewidth{1.003750pt}%
\definecolor{currentstroke}{rgb}{0.000000,0.000000,0.000000}%
\pgfsetstrokecolor{currentstroke}%
\pgfsetdash{}{0pt}%
\pgfpathmoveto{\pgfqpoint{4.549016in}{0.640923in}}%
\pgfpathcurveto{\pgfqpoint{4.559912in}{0.640923in}}{\pgfqpoint{4.570362in}{0.645252in}}{\pgfqpoint{4.578067in}{0.652956in}}%
\pgfpathcurveto{\pgfqpoint{4.585771in}{0.660660in}}{\pgfqpoint{4.590100in}{0.671111in}}{\pgfqpoint{4.590100in}{0.682007in}}%
\pgfpathcurveto{\pgfqpoint{4.590100in}{0.692902in}}{\pgfqpoint{4.585771in}{0.703353in}}{\pgfqpoint{4.578067in}{0.711057in}}%
\pgfpathcurveto{\pgfqpoint{4.570362in}{0.718762in}}{\pgfqpoint{4.559912in}{0.723090in}}{\pgfqpoint{4.549016in}{0.723090in}}%
\pgfpathcurveto{\pgfqpoint{4.538120in}{0.723090in}}{\pgfqpoint{4.527670in}{0.718762in}}{\pgfqpoint{4.519965in}{0.711057in}}%
\pgfpathcurveto{\pgfqpoint{4.512261in}{0.703353in}}{\pgfqpoint{4.507932in}{0.692902in}}{\pgfqpoint{4.507932in}{0.682007in}}%
\pgfpathcurveto{\pgfqpoint{4.507932in}{0.671111in}}{\pgfqpoint{4.512261in}{0.660660in}}{\pgfqpoint{4.519965in}{0.652956in}}%
\pgfpathcurveto{\pgfqpoint{4.527670in}{0.645252in}}{\pgfqpoint{4.538120in}{0.640923in}}{\pgfqpoint{4.549016in}{0.640923in}}%
\pgfusepath{stroke}%
\end{pgfscope}%
\begin{pgfscope}%
\pgfpathrectangle{\pgfqpoint{0.688192in}{0.670138in}}{\pgfqpoint{7.111808in}{5.129862in}}%
\pgfusepath{clip}%
\pgfsetbuttcap%
\pgfsetroundjoin%
\pgfsetlinewidth{1.003750pt}%
\definecolor{currentstroke}{rgb}{0.000000,0.000000,0.000000}%
\pgfsetstrokecolor{currentstroke}%
\pgfsetdash{}{0pt}%
\pgfpathmoveto{\pgfqpoint{7.327961in}{4.325583in}}%
\pgfpathcurveto{\pgfqpoint{7.338857in}{4.325583in}}{\pgfqpoint{7.349308in}{4.329911in}}{\pgfqpoint{7.357012in}{4.337616in}}%
\pgfpathcurveto{\pgfqpoint{7.364716in}{4.345320in}}{\pgfqpoint{7.369045in}{4.355771in}}{\pgfqpoint{7.369045in}{4.366667in}}%
\pgfpathcurveto{\pgfqpoint{7.369045in}{4.377562in}}{\pgfqpoint{7.364716in}{4.388013in}}{\pgfqpoint{7.357012in}{4.395717in}}%
\pgfpathcurveto{\pgfqpoint{7.349308in}{4.403422in}}{\pgfqpoint{7.338857in}{4.407750in}}{\pgfqpoint{7.327961in}{4.407750in}}%
\pgfpathcurveto{\pgfqpoint{7.317066in}{4.407750in}}{\pgfqpoint{7.306615in}{4.403422in}}{\pgfqpoint{7.298911in}{4.395717in}}%
\pgfpathcurveto{\pgfqpoint{7.291206in}{4.388013in}}{\pgfqpoint{7.286877in}{4.377562in}}{\pgfqpoint{7.286877in}{4.366667in}}%
\pgfpathcurveto{\pgfqpoint{7.286877in}{4.355771in}}{\pgfqpoint{7.291206in}{4.345320in}}{\pgfqpoint{7.298911in}{4.337616in}}%
\pgfpathcurveto{\pgfqpoint{7.306615in}{4.329911in}}{\pgfqpoint{7.317066in}{4.325583in}}{\pgfqpoint{7.327961in}{4.325583in}}%
\pgfpathlineto{\pgfqpoint{7.327961in}{4.325583in}}%
\pgfpathclose%
\pgfusepath{stroke}%
\end{pgfscope}%
\begin{pgfscope}%
\pgfpathrectangle{\pgfqpoint{0.688192in}{0.670138in}}{\pgfqpoint{7.111808in}{5.129862in}}%
\pgfusepath{clip}%
\pgfsetbuttcap%
\pgfsetroundjoin%
\pgfsetlinewidth{1.003750pt}%
\definecolor{currentstroke}{rgb}{0.000000,0.000000,0.000000}%
\pgfsetstrokecolor{currentstroke}%
\pgfsetdash{}{0pt}%
\pgfpathmoveto{\pgfqpoint{1.920423in}{0.692661in}}%
\pgfpathcurveto{\pgfqpoint{1.931318in}{0.692661in}}{\pgfqpoint{1.941769in}{0.696990in}}{\pgfqpoint{1.949473in}{0.704695in}}%
\pgfpathcurveto{\pgfqpoint{1.957178in}{0.712399in}}{\pgfqpoint{1.961506in}{0.722850in}}{\pgfqpoint{1.961506in}{0.733745in}}%
\pgfpathcurveto{\pgfqpoint{1.961506in}{0.744641in}}{\pgfqpoint{1.957178in}{0.755092in}}{\pgfqpoint{1.949473in}{0.762796in}}%
\pgfpathcurveto{\pgfqpoint{1.941769in}{0.770500in}}{\pgfqpoint{1.931318in}{0.774829in}}{\pgfqpoint{1.920423in}{0.774829in}}%
\pgfpathcurveto{\pgfqpoint{1.909527in}{0.774829in}}{\pgfqpoint{1.899076in}{0.770500in}}{\pgfqpoint{1.891372in}{0.762796in}}%
\pgfpathcurveto{\pgfqpoint{1.883668in}{0.755092in}}{\pgfqpoint{1.879339in}{0.744641in}}{\pgfqpoint{1.879339in}{0.733745in}}%
\pgfpathcurveto{\pgfqpoint{1.879339in}{0.722850in}}{\pgfqpoint{1.883668in}{0.712399in}}{\pgfqpoint{1.891372in}{0.704695in}}%
\pgfpathcurveto{\pgfqpoint{1.899076in}{0.696990in}}{\pgfqpoint{1.909527in}{0.692661in}}{\pgfqpoint{1.920423in}{0.692661in}}%
\pgfpathlineto{\pgfqpoint{1.920423in}{0.692661in}}%
\pgfpathclose%
\pgfusepath{stroke}%
\end{pgfscope}%
\begin{pgfscope}%
\pgfpathrectangle{\pgfqpoint{0.688192in}{0.670138in}}{\pgfqpoint{7.111808in}{5.129862in}}%
\pgfusepath{clip}%
\pgfsetbuttcap%
\pgfsetroundjoin%
\pgfsetlinewidth{1.003750pt}%
\definecolor{currentstroke}{rgb}{0.000000,0.000000,0.000000}%
\pgfsetstrokecolor{currentstroke}%
\pgfsetdash{}{0pt}%
\pgfpathmoveto{\pgfqpoint{6.308571in}{3.902956in}}%
\pgfpathcurveto{\pgfqpoint{6.319467in}{3.902956in}}{\pgfqpoint{6.329918in}{3.907285in}}{\pgfqpoint{6.337622in}{3.914989in}}%
\pgfpathcurveto{\pgfqpoint{6.345326in}{3.922694in}}{\pgfqpoint{6.349655in}{3.933144in}}{\pgfqpoint{6.349655in}{3.944040in}}%
\pgfpathcurveto{\pgfqpoint{6.349655in}{3.954936in}}{\pgfqpoint{6.345326in}{3.965386in}}{\pgfqpoint{6.337622in}{3.973091in}}%
\pgfpathcurveto{\pgfqpoint{6.329918in}{3.980795in}}{\pgfqpoint{6.319467in}{3.985124in}}{\pgfqpoint{6.308571in}{3.985124in}}%
\pgfpathcurveto{\pgfqpoint{6.297676in}{3.985124in}}{\pgfqpoint{6.287225in}{3.980795in}}{\pgfqpoint{6.279520in}{3.973091in}}%
\pgfpathcurveto{\pgfqpoint{6.271816in}{3.965386in}}{\pgfqpoint{6.267487in}{3.954936in}}{\pgfqpoint{6.267487in}{3.944040in}}%
\pgfpathcurveto{\pgfqpoint{6.267487in}{3.933144in}}{\pgfqpoint{6.271816in}{3.922694in}}{\pgfqpoint{6.279520in}{3.914989in}}%
\pgfpathcurveto{\pgfqpoint{6.287225in}{3.907285in}}{\pgfqpoint{6.297676in}{3.902956in}}{\pgfqpoint{6.308571in}{3.902956in}}%
\pgfpathlineto{\pgfqpoint{6.308571in}{3.902956in}}%
\pgfpathclose%
\pgfusepath{stroke}%
\end{pgfscope}%
\begin{pgfscope}%
\pgfpathrectangle{\pgfqpoint{0.688192in}{0.670138in}}{\pgfqpoint{7.111808in}{5.129862in}}%
\pgfusepath{clip}%
\pgfsetbuttcap%
\pgfsetroundjoin%
\pgfsetlinewidth{1.003750pt}%
\definecolor{currentstroke}{rgb}{0.000000,0.000000,0.000000}%
\pgfsetstrokecolor{currentstroke}%
\pgfsetdash{}{0pt}%
\pgfpathmoveto{\pgfqpoint{1.351667in}{0.711476in}}%
\pgfpathcurveto{\pgfqpoint{1.362562in}{0.711476in}}{\pgfqpoint{1.373013in}{0.715805in}}{\pgfqpoint{1.380718in}{0.723510in}}%
\pgfpathcurveto{\pgfqpoint{1.388422in}{0.731214in}}{\pgfqpoint{1.392751in}{0.741665in}}{\pgfqpoint{1.392751in}{0.752560in}}%
\pgfpathcurveto{\pgfqpoint{1.392751in}{0.763456in}}{\pgfqpoint{1.388422in}{0.773907in}}{\pgfqpoint{1.380718in}{0.781611in}}%
\pgfpathcurveto{\pgfqpoint{1.373013in}{0.789315in}}{\pgfqpoint{1.362562in}{0.793644in}}{\pgfqpoint{1.351667in}{0.793644in}}%
\pgfpathcurveto{\pgfqpoint{1.340771in}{0.793644in}}{\pgfqpoint{1.330321in}{0.789315in}}{\pgfqpoint{1.322616in}{0.781611in}}%
\pgfpathcurveto{\pgfqpoint{1.314912in}{0.773907in}}{\pgfqpoint{1.310583in}{0.763456in}}{\pgfqpoint{1.310583in}{0.752560in}}%
\pgfpathcurveto{\pgfqpoint{1.310583in}{0.741665in}}{\pgfqpoint{1.314912in}{0.731214in}}{\pgfqpoint{1.322616in}{0.723510in}}%
\pgfpathcurveto{\pgfqpoint{1.330321in}{0.715805in}}{\pgfqpoint{1.340771in}{0.711476in}}{\pgfqpoint{1.351667in}{0.711476in}}%
\pgfpathlineto{\pgfqpoint{1.351667in}{0.711476in}}%
\pgfpathclose%
\pgfusepath{stroke}%
\end{pgfscope}%
\begin{pgfscope}%
\pgfpathrectangle{\pgfqpoint{0.688192in}{0.670138in}}{\pgfqpoint{7.111808in}{5.129862in}}%
\pgfusepath{clip}%
\pgfsetbuttcap%
\pgfsetroundjoin%
\pgfsetlinewidth{1.003750pt}%
\definecolor{currentstroke}{rgb}{0.000000,0.000000,0.000000}%
\pgfsetstrokecolor{currentstroke}%
\pgfsetdash{}{0pt}%
\pgfpathmoveto{\pgfqpoint{0.886861in}{0.869726in}}%
\pgfpathcurveto{\pgfqpoint{0.897756in}{0.869726in}}{\pgfqpoint{0.908207in}{0.874055in}}{\pgfqpoint{0.915911in}{0.881759in}}%
\pgfpathcurveto{\pgfqpoint{0.923616in}{0.889463in}}{\pgfqpoint{0.927945in}{0.899914in}}{\pgfqpoint{0.927945in}{0.910810in}}%
\pgfpathcurveto{\pgfqpoint{0.927945in}{0.921705in}}{\pgfqpoint{0.923616in}{0.932156in}}{\pgfqpoint{0.915911in}{0.939861in}}%
\pgfpathcurveto{\pgfqpoint{0.908207in}{0.947565in}}{\pgfqpoint{0.897756in}{0.951894in}}{\pgfqpoint{0.886861in}{0.951894in}}%
\pgfpathcurveto{\pgfqpoint{0.875965in}{0.951894in}}{\pgfqpoint{0.865514in}{0.947565in}}{\pgfqpoint{0.857810in}{0.939861in}}%
\pgfpathcurveto{\pgfqpoint{0.850106in}{0.932156in}}{\pgfqpoint{0.845777in}{0.921705in}}{\pgfqpoint{0.845777in}{0.910810in}}%
\pgfpathcurveto{\pgfqpoint{0.845777in}{0.899914in}}{\pgfqpoint{0.850106in}{0.889463in}}{\pgfqpoint{0.857810in}{0.881759in}}%
\pgfpathcurveto{\pgfqpoint{0.865514in}{0.874055in}}{\pgfqpoint{0.875965in}{0.869726in}}{\pgfqpoint{0.886861in}{0.869726in}}%
\pgfpathlineto{\pgfqpoint{0.886861in}{0.869726in}}%
\pgfpathclose%
\pgfusepath{stroke}%
\end{pgfscope}%
\begin{pgfscope}%
\pgfpathrectangle{\pgfqpoint{0.688192in}{0.670138in}}{\pgfqpoint{7.111808in}{5.129862in}}%
\pgfusepath{clip}%
\pgfsetbuttcap%
\pgfsetroundjoin%
\pgfsetlinewidth{1.003750pt}%
\definecolor{currentstroke}{rgb}{0.000000,0.000000,0.000000}%
\pgfsetstrokecolor{currentstroke}%
\pgfsetdash{}{0pt}%
\pgfpathmoveto{\pgfqpoint{2.497167in}{0.675229in}}%
\pgfpathcurveto{\pgfqpoint{2.508063in}{0.675229in}}{\pgfqpoint{2.518514in}{0.679557in}}{\pgfqpoint{2.526218in}{0.687262in}}%
\pgfpathcurveto{\pgfqpoint{2.533922in}{0.694966in}}{\pgfqpoint{2.538251in}{0.705417in}}{\pgfqpoint{2.538251in}{0.716312in}}%
\pgfpathcurveto{\pgfqpoint{2.538251in}{0.727208in}}{\pgfqpoint{2.533922in}{0.737659in}}{\pgfqpoint{2.526218in}{0.745363in}}%
\pgfpathcurveto{\pgfqpoint{2.518514in}{0.753067in}}{\pgfqpoint{2.508063in}{0.757396in}}{\pgfqpoint{2.497167in}{0.757396in}}%
\pgfpathcurveto{\pgfqpoint{2.486272in}{0.757396in}}{\pgfqpoint{2.475821in}{0.753067in}}{\pgfqpoint{2.468117in}{0.745363in}}%
\pgfpathcurveto{\pgfqpoint{2.460412in}{0.737659in}}{\pgfqpoint{2.456084in}{0.727208in}}{\pgfqpoint{2.456084in}{0.716312in}}%
\pgfpathcurveto{\pgfqpoint{2.456084in}{0.705417in}}{\pgfqpoint{2.460412in}{0.694966in}}{\pgfqpoint{2.468117in}{0.687262in}}%
\pgfpathcurveto{\pgfqpoint{2.475821in}{0.679557in}}{\pgfqpoint{2.486272in}{0.675229in}}{\pgfqpoint{2.497167in}{0.675229in}}%
\pgfpathlineto{\pgfqpoint{2.497167in}{0.675229in}}%
\pgfpathclose%
\pgfusepath{stroke}%
\end{pgfscope}%
\begin{pgfscope}%
\pgfpathrectangle{\pgfqpoint{0.688192in}{0.670138in}}{\pgfqpoint{7.111808in}{5.129862in}}%
\pgfusepath{clip}%
\pgfsetbuttcap%
\pgfsetroundjoin%
\pgfsetlinewidth{1.003750pt}%
\definecolor{currentstroke}{rgb}{0.000000,0.000000,0.000000}%
\pgfsetstrokecolor{currentstroke}%
\pgfsetdash{}{0pt}%
\pgfpathmoveto{\pgfqpoint{0.952302in}{0.807998in}}%
\pgfpathcurveto{\pgfqpoint{0.963197in}{0.807998in}}{\pgfqpoint{0.973648in}{0.812326in}}{\pgfqpoint{0.981352in}{0.820031in}}%
\pgfpathcurveto{\pgfqpoint{0.989057in}{0.827735in}}{\pgfqpoint{0.993385in}{0.838186in}}{\pgfqpoint{0.993385in}{0.849081in}}%
\pgfpathcurveto{\pgfqpoint{0.993385in}{0.859977in}}{\pgfqpoint{0.989057in}{0.870428in}}{\pgfqpoint{0.981352in}{0.878132in}}%
\pgfpathcurveto{\pgfqpoint{0.973648in}{0.885836in}}{\pgfqpoint{0.963197in}{0.890165in}}{\pgfqpoint{0.952302in}{0.890165in}}%
\pgfpathcurveto{\pgfqpoint{0.941406in}{0.890165in}}{\pgfqpoint{0.930955in}{0.885836in}}{\pgfqpoint{0.923251in}{0.878132in}}%
\pgfpathcurveto{\pgfqpoint{0.915547in}{0.870428in}}{\pgfqpoint{0.911218in}{0.859977in}}{\pgfqpoint{0.911218in}{0.849081in}}%
\pgfpathcurveto{\pgfqpoint{0.911218in}{0.838186in}}{\pgfqpoint{0.915547in}{0.827735in}}{\pgfqpoint{0.923251in}{0.820031in}}%
\pgfpathcurveto{\pgfqpoint{0.930955in}{0.812326in}}{\pgfqpoint{0.941406in}{0.807998in}}{\pgfqpoint{0.952302in}{0.807998in}}%
\pgfpathlineto{\pgfqpoint{0.952302in}{0.807998in}}%
\pgfpathclose%
\pgfusepath{stroke}%
\end{pgfscope}%
\begin{pgfscope}%
\pgfpathrectangle{\pgfqpoint{0.688192in}{0.670138in}}{\pgfqpoint{7.111808in}{5.129862in}}%
\pgfusepath{clip}%
\pgfsetbuttcap%
\pgfsetroundjoin%
\pgfsetlinewidth{1.003750pt}%
\definecolor{currentstroke}{rgb}{0.000000,0.000000,0.000000}%
\pgfsetstrokecolor{currentstroke}%
\pgfsetdash{}{0pt}%
\pgfpathmoveto{\pgfqpoint{1.528667in}{0.706804in}}%
\pgfpathcurveto{\pgfqpoint{1.539563in}{0.706804in}}{\pgfqpoint{1.550014in}{0.711133in}}{\pgfqpoint{1.557718in}{0.718837in}}%
\pgfpathcurveto{\pgfqpoint{1.565422in}{0.726541in}}{\pgfqpoint{1.569751in}{0.736992in}}{\pgfqpoint{1.569751in}{0.747888in}}%
\pgfpathcurveto{\pgfqpoint{1.569751in}{0.758783in}}{\pgfqpoint{1.565422in}{0.769234in}}{\pgfqpoint{1.557718in}{0.776938in}}%
\pgfpathcurveto{\pgfqpoint{1.550014in}{0.784643in}}{\pgfqpoint{1.539563in}{0.788972in}}{\pgfqpoint{1.528667in}{0.788972in}}%
\pgfpathcurveto{\pgfqpoint{1.517772in}{0.788972in}}{\pgfqpoint{1.507321in}{0.784643in}}{\pgfqpoint{1.499616in}{0.776938in}}%
\pgfpathcurveto{\pgfqpoint{1.491912in}{0.769234in}}{\pgfqpoint{1.487583in}{0.758783in}}{\pgfqpoint{1.487583in}{0.747888in}}%
\pgfpathcurveto{\pgfqpoint{1.487583in}{0.736992in}}{\pgfqpoint{1.491912in}{0.726541in}}{\pgfqpoint{1.499616in}{0.718837in}}%
\pgfpathcurveto{\pgfqpoint{1.507321in}{0.711133in}}{\pgfqpoint{1.517772in}{0.706804in}}{\pgfqpoint{1.528667in}{0.706804in}}%
\pgfpathlineto{\pgfqpoint{1.528667in}{0.706804in}}%
\pgfpathclose%
\pgfusepath{stroke}%
\end{pgfscope}%
\begin{pgfscope}%
\pgfpathrectangle{\pgfqpoint{0.688192in}{0.670138in}}{\pgfqpoint{7.111808in}{5.129862in}}%
\pgfusepath{clip}%
\pgfsetbuttcap%
\pgfsetroundjoin%
\pgfsetlinewidth{1.003750pt}%
\definecolor{currentstroke}{rgb}{0.000000,0.000000,0.000000}%
\pgfsetstrokecolor{currentstroke}%
\pgfsetdash{}{0pt}%
\pgfpathmoveto{\pgfqpoint{4.305764in}{5.745541in}}%
\pgfpathcurveto{\pgfqpoint{4.316660in}{5.745541in}}{\pgfqpoint{4.327110in}{5.749870in}}{\pgfqpoint{4.334815in}{5.757575in}}%
\pgfpathcurveto{\pgfqpoint{4.342519in}{5.765279in}}{\pgfqpoint{4.346848in}{5.775730in}}{\pgfqpoint{4.346848in}{5.786625in}}%
\pgfpathcurveto{\pgfqpoint{4.346848in}{5.797521in}}{\pgfqpoint{4.342519in}{5.807972in}}{\pgfqpoint{4.334815in}{5.815676in}}%
\pgfpathcurveto{\pgfqpoint{4.327110in}{5.823380in}}{\pgfqpoint{4.316660in}{5.827709in}}{\pgfqpoint{4.305764in}{5.827709in}}%
\pgfpathcurveto{\pgfqpoint{4.294868in}{5.827709in}}{\pgfqpoint{4.284418in}{5.823380in}}{\pgfqpoint{4.276713in}{5.815676in}}%
\pgfpathcurveto{\pgfqpoint{4.269009in}{5.807972in}}{\pgfqpoint{4.264680in}{5.797521in}}{\pgfqpoint{4.264680in}{5.786625in}}%
\pgfpathcurveto{\pgfqpoint{4.264680in}{5.775730in}}{\pgfqpoint{4.269009in}{5.765279in}}{\pgfqpoint{4.276713in}{5.757575in}}%
\pgfpathcurveto{\pgfqpoint{4.284418in}{5.749870in}}{\pgfqpoint{4.294868in}{5.745541in}}{\pgfqpoint{4.305764in}{5.745541in}}%
\pgfpathlineto{\pgfqpoint{4.305764in}{5.745541in}}%
\pgfpathclose%
\pgfusepath{stroke}%
\end{pgfscope}%
\begin{pgfscope}%
\pgfpathrectangle{\pgfqpoint{0.688192in}{0.670138in}}{\pgfqpoint{7.111808in}{5.129862in}}%
\pgfusepath{clip}%
\pgfsetbuttcap%
\pgfsetroundjoin%
\pgfsetlinewidth{1.003750pt}%
\definecolor{currentstroke}{rgb}{0.000000,0.000000,0.000000}%
\pgfsetstrokecolor{currentstroke}%
\pgfsetdash{}{0pt}%
\pgfpathmoveto{\pgfqpoint{0.882669in}{0.877855in}}%
\pgfpathcurveto{\pgfqpoint{0.893565in}{0.877855in}}{\pgfqpoint{0.904016in}{0.882184in}}{\pgfqpoint{0.911720in}{0.889888in}}%
\pgfpathcurveto{\pgfqpoint{0.919424in}{0.897592in}}{\pgfqpoint{0.923753in}{0.908043in}}{\pgfqpoint{0.923753in}{0.918939in}}%
\pgfpathcurveto{\pgfqpoint{0.923753in}{0.929834in}}{\pgfqpoint{0.919424in}{0.940285in}}{\pgfqpoint{0.911720in}{0.947989in}}%
\pgfpathcurveto{\pgfqpoint{0.904016in}{0.955694in}}{\pgfqpoint{0.893565in}{0.960023in}}{\pgfqpoint{0.882669in}{0.960023in}}%
\pgfpathcurveto{\pgfqpoint{0.871774in}{0.960023in}}{\pgfqpoint{0.861323in}{0.955694in}}{\pgfqpoint{0.853619in}{0.947989in}}%
\pgfpathcurveto{\pgfqpoint{0.845914in}{0.940285in}}{\pgfqpoint{0.841585in}{0.929834in}}{\pgfqpoint{0.841585in}{0.918939in}}%
\pgfpathcurveto{\pgfqpoint{0.841585in}{0.908043in}}{\pgfqpoint{0.845914in}{0.897592in}}{\pgfqpoint{0.853619in}{0.889888in}}%
\pgfpathcurveto{\pgfqpoint{0.861323in}{0.882184in}}{\pgfqpoint{0.871774in}{0.877855in}}{\pgfqpoint{0.882669in}{0.877855in}}%
\pgfpathlineto{\pgfqpoint{0.882669in}{0.877855in}}%
\pgfpathclose%
\pgfusepath{stroke}%
\end{pgfscope}%
\begin{pgfscope}%
\pgfpathrectangle{\pgfqpoint{0.688192in}{0.670138in}}{\pgfqpoint{7.111808in}{5.129862in}}%
\pgfusepath{clip}%
\pgfsetbuttcap%
\pgfsetroundjoin%
\pgfsetlinewidth{1.003750pt}%
\definecolor{currentstroke}{rgb}{0.000000,0.000000,0.000000}%
\pgfsetstrokecolor{currentstroke}%
\pgfsetdash{}{0pt}%
\pgfpathmoveto{\pgfqpoint{0.928243in}{0.841597in}}%
\pgfpathcurveto{\pgfqpoint{0.939139in}{0.841597in}}{\pgfqpoint{0.949589in}{0.845926in}}{\pgfqpoint{0.957294in}{0.853631in}}%
\pgfpathcurveto{\pgfqpoint{0.964998in}{0.861335in}}{\pgfqpoint{0.969327in}{0.871786in}}{\pgfqpoint{0.969327in}{0.882681in}}%
\pgfpathcurveto{\pgfqpoint{0.969327in}{0.893577in}}{\pgfqpoint{0.964998in}{0.904028in}}{\pgfqpoint{0.957294in}{0.911732in}}%
\pgfpathcurveto{\pgfqpoint{0.949589in}{0.919436in}}{\pgfqpoint{0.939139in}{0.923765in}}{\pgfqpoint{0.928243in}{0.923765in}}%
\pgfpathcurveto{\pgfqpoint{0.917347in}{0.923765in}}{\pgfqpoint{0.906897in}{0.919436in}}{\pgfqpoint{0.899192in}{0.911732in}}%
\pgfpathcurveto{\pgfqpoint{0.891488in}{0.904028in}}{\pgfqpoint{0.887159in}{0.893577in}}{\pgfqpoint{0.887159in}{0.882681in}}%
\pgfpathcurveto{\pgfqpoint{0.887159in}{0.871786in}}{\pgfqpoint{0.891488in}{0.861335in}}{\pgfqpoint{0.899192in}{0.853631in}}%
\pgfpathcurveto{\pgfqpoint{0.906897in}{0.845926in}}{\pgfqpoint{0.917347in}{0.841597in}}{\pgfqpoint{0.928243in}{0.841597in}}%
\pgfpathlineto{\pgfqpoint{0.928243in}{0.841597in}}%
\pgfpathclose%
\pgfusepath{stroke}%
\end{pgfscope}%
\begin{pgfscope}%
\pgfpathrectangle{\pgfqpoint{0.688192in}{0.670138in}}{\pgfqpoint{7.111808in}{5.129862in}}%
\pgfusepath{clip}%
\pgfsetbuttcap%
\pgfsetroundjoin%
\pgfsetlinewidth{1.003750pt}%
\definecolor{currentstroke}{rgb}{0.000000,0.000000,0.000000}%
\pgfsetstrokecolor{currentstroke}%
\pgfsetdash{}{0pt}%
\pgfpathmoveto{\pgfqpoint{5.261268in}{0.632886in}}%
\pgfpathcurveto{\pgfqpoint{5.272163in}{0.632886in}}{\pgfqpoint{5.282614in}{0.637215in}}{\pgfqpoint{5.290319in}{0.644920in}}%
\pgfpathcurveto{\pgfqpoint{5.298023in}{0.652624in}}{\pgfqpoint{5.302352in}{0.663075in}}{\pgfqpoint{5.302352in}{0.673970in}}%
\pgfpathcurveto{\pgfqpoint{5.302352in}{0.684866in}}{\pgfqpoint{5.298023in}{0.695317in}}{\pgfqpoint{5.290319in}{0.703021in}}%
\pgfpathcurveto{\pgfqpoint{5.282614in}{0.710725in}}{\pgfqpoint{5.272163in}{0.715054in}}{\pgfqpoint{5.261268in}{0.715054in}}%
\pgfpathcurveto{\pgfqpoint{5.250372in}{0.715054in}}{\pgfqpoint{5.239921in}{0.710725in}}{\pgfqpoint{5.232217in}{0.703021in}}%
\pgfpathcurveto{\pgfqpoint{5.224513in}{0.695317in}}{\pgfqpoint{5.220184in}{0.684866in}}{\pgfqpoint{5.220184in}{0.673970in}}%
\pgfpathcurveto{\pgfqpoint{5.220184in}{0.663075in}}{\pgfqpoint{5.224513in}{0.652624in}}{\pgfqpoint{5.232217in}{0.644920in}}%
\pgfpathcurveto{\pgfqpoint{5.239921in}{0.637215in}}{\pgfqpoint{5.250372in}{0.632886in}}{\pgfqpoint{5.261268in}{0.632886in}}%
\pgfusepath{stroke}%
\end{pgfscope}%
\begin{pgfscope}%
\pgfpathrectangle{\pgfqpoint{0.688192in}{0.670138in}}{\pgfqpoint{7.111808in}{5.129862in}}%
\pgfusepath{clip}%
\pgfsetbuttcap%
\pgfsetroundjoin%
\pgfsetlinewidth{1.003750pt}%
\definecolor{currentstroke}{rgb}{0.000000,0.000000,0.000000}%
\pgfsetstrokecolor{currentstroke}%
\pgfsetdash{}{0pt}%
\pgfpathmoveto{\pgfqpoint{2.983194in}{0.663522in}}%
\pgfpathcurveto{\pgfqpoint{2.994090in}{0.663522in}}{\pgfqpoint{3.004540in}{0.667850in}}{\pgfqpoint{3.012245in}{0.675555in}}%
\pgfpathcurveto{\pgfqpoint{3.019949in}{0.683259in}}{\pgfqpoint{3.024278in}{0.693710in}}{\pgfqpoint{3.024278in}{0.704605in}}%
\pgfpathcurveto{\pgfqpoint{3.024278in}{0.715501in}}{\pgfqpoint{3.019949in}{0.725952in}}{\pgfqpoint{3.012245in}{0.733656in}}%
\pgfpathcurveto{\pgfqpoint{3.004540in}{0.741360in}}{\pgfqpoint{2.994090in}{0.745689in}}{\pgfqpoint{2.983194in}{0.745689in}}%
\pgfpathcurveto{\pgfqpoint{2.972299in}{0.745689in}}{\pgfqpoint{2.961848in}{0.741360in}}{\pgfqpoint{2.954143in}{0.733656in}}%
\pgfpathcurveto{\pgfqpoint{2.946439in}{0.725952in}}{\pgfqpoint{2.942110in}{0.715501in}}{\pgfqpoint{2.942110in}{0.704605in}}%
\pgfpathcurveto{\pgfqpoint{2.942110in}{0.693710in}}{\pgfqpoint{2.946439in}{0.683259in}}{\pgfqpoint{2.954143in}{0.675555in}}%
\pgfpathcurveto{\pgfqpoint{2.961848in}{0.667850in}}{\pgfqpoint{2.972299in}{0.663522in}}{\pgfqpoint{2.983194in}{0.663522in}}%
\pgfusepath{stroke}%
\end{pgfscope}%
\begin{pgfscope}%
\pgfpathrectangle{\pgfqpoint{0.688192in}{0.670138in}}{\pgfqpoint{7.111808in}{5.129862in}}%
\pgfusepath{clip}%
\pgfsetbuttcap%
\pgfsetroundjoin%
\pgfsetlinewidth{1.003750pt}%
\definecolor{currentstroke}{rgb}{0.000000,0.000000,0.000000}%
\pgfsetstrokecolor{currentstroke}%
\pgfsetdash{}{0pt}%
\pgfpathmoveto{\pgfqpoint{5.485912in}{5.828484in}}%
\pgfpathcurveto{\pgfqpoint{5.496807in}{5.828484in}}{\pgfqpoint{5.507258in}{5.832813in}}{\pgfqpoint{5.514962in}{5.840517in}}%
\pgfpathcurveto{\pgfqpoint{5.522667in}{5.848222in}}{\pgfqpoint{5.526995in}{5.858672in}}{\pgfqpoint{5.526995in}{5.869568in}}%
\pgfpathcurveto{\pgfqpoint{5.526995in}{5.880463in}}{\pgfqpoint{5.522667in}{5.890914in}}{\pgfqpoint{5.514962in}{5.898619in}}%
\pgfpathcurveto{\pgfqpoint{5.507258in}{5.906323in}}{\pgfqpoint{5.496807in}{5.910652in}}{\pgfqpoint{5.485912in}{5.910652in}}%
\pgfpathcurveto{\pgfqpoint{5.475016in}{5.910652in}}{\pgfqpoint{5.464565in}{5.906323in}}{\pgfqpoint{5.456861in}{5.898619in}}%
\pgfpathcurveto{\pgfqpoint{5.449156in}{5.890914in}}{\pgfqpoint{5.444828in}{5.880463in}}{\pgfqpoint{5.444828in}{5.869568in}}%
\pgfpathcurveto{\pgfqpoint{5.444828in}{5.858672in}}{\pgfqpoint{5.449156in}{5.848222in}}{\pgfqpoint{5.456861in}{5.840517in}}%
\pgfpathcurveto{\pgfqpoint{5.464565in}{5.832813in}}{\pgfqpoint{5.475016in}{5.828484in}}{\pgfqpoint{5.485912in}{5.828484in}}%
\pgfusepath{stroke}%
\end{pgfscope}%
\begin{pgfscope}%
\pgfpathrectangle{\pgfqpoint{0.688192in}{0.670138in}}{\pgfqpoint{7.111808in}{5.129862in}}%
\pgfusepath{clip}%
\pgfsetbuttcap%
\pgfsetroundjoin%
\pgfsetlinewidth{1.003750pt}%
\definecolor{currentstroke}{rgb}{0.000000,0.000000,0.000000}%
\pgfsetstrokecolor{currentstroke}%
\pgfsetdash{}{0pt}%
\pgfpathmoveto{\pgfqpoint{1.106601in}{0.720443in}}%
\pgfpathcurveto{\pgfqpoint{1.117497in}{0.720443in}}{\pgfqpoint{1.127948in}{0.724772in}}{\pgfqpoint{1.135652in}{0.732476in}}%
\pgfpathcurveto{\pgfqpoint{1.143357in}{0.740180in}}{\pgfqpoint{1.147685in}{0.750631in}}{\pgfqpoint{1.147685in}{0.761527in}}%
\pgfpathcurveto{\pgfqpoint{1.147685in}{0.772422in}}{\pgfqpoint{1.143357in}{0.782873in}}{\pgfqpoint{1.135652in}{0.790577in}}%
\pgfpathcurveto{\pgfqpoint{1.127948in}{0.798282in}}{\pgfqpoint{1.117497in}{0.802611in}}{\pgfqpoint{1.106601in}{0.802611in}}%
\pgfpathcurveto{\pgfqpoint{1.095706in}{0.802611in}}{\pgfqpoint{1.085255in}{0.798282in}}{\pgfqpoint{1.077551in}{0.790577in}}%
\pgfpathcurveto{\pgfqpoint{1.069846in}{0.782873in}}{\pgfqpoint{1.065518in}{0.772422in}}{\pgfqpoint{1.065518in}{0.761527in}}%
\pgfpathcurveto{\pgfqpoint{1.065518in}{0.750631in}}{\pgfqpoint{1.069846in}{0.740180in}}{\pgfqpoint{1.077551in}{0.732476in}}%
\pgfpathcurveto{\pgfqpoint{1.085255in}{0.724772in}}{\pgfqpoint{1.095706in}{0.720443in}}{\pgfqpoint{1.106601in}{0.720443in}}%
\pgfpathlineto{\pgfqpoint{1.106601in}{0.720443in}}%
\pgfpathclose%
\pgfusepath{stroke}%
\end{pgfscope}%
\begin{pgfscope}%
\pgfpathrectangle{\pgfqpoint{0.688192in}{0.670138in}}{\pgfqpoint{7.111808in}{5.129862in}}%
\pgfusepath{clip}%
\pgfsetbuttcap%
\pgfsetroundjoin%
\pgfsetlinewidth{1.003750pt}%
\definecolor{currentstroke}{rgb}{0.000000,0.000000,0.000000}%
\pgfsetstrokecolor{currentstroke}%
\pgfsetdash{}{0pt}%
\pgfpathmoveto{\pgfqpoint{1.775222in}{0.697735in}}%
\pgfpathcurveto{\pgfqpoint{1.786118in}{0.697735in}}{\pgfqpoint{1.796569in}{0.702064in}}{\pgfqpoint{1.804273in}{0.709768in}}%
\pgfpathcurveto{\pgfqpoint{1.811977in}{0.717473in}}{\pgfqpoint{1.816306in}{0.727923in}}{\pgfqpoint{1.816306in}{0.738819in}}%
\pgfpathcurveto{\pgfqpoint{1.816306in}{0.749715in}}{\pgfqpoint{1.811977in}{0.760165in}}{\pgfqpoint{1.804273in}{0.767870in}}%
\pgfpathcurveto{\pgfqpoint{1.796569in}{0.775574in}}{\pgfqpoint{1.786118in}{0.779903in}}{\pgfqpoint{1.775222in}{0.779903in}}%
\pgfpathcurveto{\pgfqpoint{1.764327in}{0.779903in}}{\pgfqpoint{1.753876in}{0.775574in}}{\pgfqpoint{1.746172in}{0.767870in}}%
\pgfpathcurveto{\pgfqpoint{1.738467in}{0.760165in}}{\pgfqpoint{1.734139in}{0.749715in}}{\pgfqpoint{1.734139in}{0.738819in}}%
\pgfpathcurveto{\pgfqpoint{1.734139in}{0.727923in}}{\pgfqpoint{1.738467in}{0.717473in}}{\pgfqpoint{1.746172in}{0.709768in}}%
\pgfpathcurveto{\pgfqpoint{1.753876in}{0.702064in}}{\pgfqpoint{1.764327in}{0.697735in}}{\pgfqpoint{1.775222in}{0.697735in}}%
\pgfpathlineto{\pgfqpoint{1.775222in}{0.697735in}}%
\pgfpathclose%
\pgfusepath{stroke}%
\end{pgfscope}%
\begin{pgfscope}%
\pgfpathrectangle{\pgfqpoint{0.688192in}{0.670138in}}{\pgfqpoint{7.111808in}{5.129862in}}%
\pgfusepath{clip}%
\pgfsetbuttcap%
\pgfsetroundjoin%
\pgfsetlinewidth{1.003750pt}%
\definecolor{currentstroke}{rgb}{0.000000,0.000000,0.000000}%
\pgfsetstrokecolor{currentstroke}%
\pgfsetdash{}{0pt}%
\pgfpathmoveto{\pgfqpoint{3.322669in}{5.217613in}}%
\pgfpathcurveto{\pgfqpoint{3.333565in}{5.217613in}}{\pgfqpoint{3.344016in}{5.221942in}}{\pgfqpoint{3.351720in}{5.229646in}}%
\pgfpathcurveto{\pgfqpoint{3.359424in}{5.237351in}}{\pgfqpoint{3.363753in}{5.247801in}}{\pgfqpoint{3.363753in}{5.258697in}}%
\pgfpathcurveto{\pgfqpoint{3.363753in}{5.269593in}}{\pgfqpoint{3.359424in}{5.280043in}}{\pgfqpoint{3.351720in}{5.287748in}}%
\pgfpathcurveto{\pgfqpoint{3.344016in}{5.295452in}}{\pgfqpoint{3.333565in}{5.299781in}}{\pgfqpoint{3.322669in}{5.299781in}}%
\pgfpathcurveto{\pgfqpoint{3.311774in}{5.299781in}}{\pgfqpoint{3.301323in}{5.295452in}}{\pgfqpoint{3.293619in}{5.287748in}}%
\pgfpathcurveto{\pgfqpoint{3.285914in}{5.280043in}}{\pgfqpoint{3.281585in}{5.269593in}}{\pgfqpoint{3.281585in}{5.258697in}}%
\pgfpathcurveto{\pgfqpoint{3.281585in}{5.247801in}}{\pgfqpoint{3.285914in}{5.237351in}}{\pgfqpoint{3.293619in}{5.229646in}}%
\pgfpathcurveto{\pgfqpoint{3.301323in}{5.221942in}}{\pgfqpoint{3.311774in}{5.217613in}}{\pgfqpoint{3.322669in}{5.217613in}}%
\pgfpathlineto{\pgfqpoint{3.322669in}{5.217613in}}%
\pgfpathclose%
\pgfusepath{stroke}%
\end{pgfscope}%
\begin{pgfscope}%
\pgfpathrectangle{\pgfqpoint{0.688192in}{0.670138in}}{\pgfqpoint{7.111808in}{5.129862in}}%
\pgfusepath{clip}%
\pgfsetbuttcap%
\pgfsetroundjoin%
\pgfsetlinewidth{1.003750pt}%
\definecolor{currentstroke}{rgb}{0.000000,0.000000,0.000000}%
\pgfsetstrokecolor{currentstroke}%
\pgfsetdash{}{0pt}%
\pgfpathmoveto{\pgfqpoint{1.685511in}{0.701118in}}%
\pgfpathcurveto{\pgfqpoint{1.696407in}{0.701118in}}{\pgfqpoint{1.706857in}{0.705447in}}{\pgfqpoint{1.714562in}{0.713151in}}%
\pgfpathcurveto{\pgfqpoint{1.722266in}{0.720855in}}{\pgfqpoint{1.726595in}{0.731306in}}{\pgfqpoint{1.726595in}{0.742202in}}%
\pgfpathcurveto{\pgfqpoint{1.726595in}{0.753097in}}{\pgfqpoint{1.722266in}{0.763548in}}{\pgfqpoint{1.714562in}{0.771252in}}%
\pgfpathcurveto{\pgfqpoint{1.706857in}{0.778957in}}{\pgfqpoint{1.696407in}{0.783286in}}{\pgfqpoint{1.685511in}{0.783286in}}%
\pgfpathcurveto{\pgfqpoint{1.674615in}{0.783286in}}{\pgfqpoint{1.664165in}{0.778957in}}{\pgfqpoint{1.656460in}{0.771252in}}%
\pgfpathcurveto{\pgfqpoint{1.648756in}{0.763548in}}{\pgfqpoint{1.644427in}{0.753097in}}{\pgfqpoint{1.644427in}{0.742202in}}%
\pgfpathcurveto{\pgfqpoint{1.644427in}{0.731306in}}{\pgfqpoint{1.648756in}{0.720855in}}{\pgfqpoint{1.656460in}{0.713151in}}%
\pgfpathcurveto{\pgfqpoint{1.664165in}{0.705447in}}{\pgfqpoint{1.674615in}{0.701118in}}{\pgfqpoint{1.685511in}{0.701118in}}%
\pgfpathlineto{\pgfqpoint{1.685511in}{0.701118in}}%
\pgfpathclose%
\pgfusepath{stroke}%
\end{pgfscope}%
\begin{pgfscope}%
\pgfpathrectangle{\pgfqpoint{0.688192in}{0.670138in}}{\pgfqpoint{7.111808in}{5.129862in}}%
\pgfusepath{clip}%
\pgfsetbuttcap%
\pgfsetroundjoin%
\pgfsetlinewidth{1.003750pt}%
\definecolor{currentstroke}{rgb}{0.000000,0.000000,0.000000}%
\pgfsetstrokecolor{currentstroke}%
\pgfsetdash{}{0pt}%
\pgfpathmoveto{\pgfqpoint{2.751273in}{1.008701in}}%
\pgfpathcurveto{\pgfqpoint{2.762169in}{1.008701in}}{\pgfqpoint{2.772620in}{1.013030in}}{\pgfqpoint{2.780324in}{1.020734in}}%
\pgfpathcurveto{\pgfqpoint{2.788028in}{1.028439in}}{\pgfqpoint{2.792357in}{1.038890in}}{\pgfqpoint{2.792357in}{1.049785in}}%
\pgfpathcurveto{\pgfqpoint{2.792357in}{1.060681in}}{\pgfqpoint{2.788028in}{1.071131in}}{\pgfqpoint{2.780324in}{1.078836in}}%
\pgfpathcurveto{\pgfqpoint{2.772620in}{1.086540in}}{\pgfqpoint{2.762169in}{1.090869in}}{\pgfqpoint{2.751273in}{1.090869in}}%
\pgfpathcurveto{\pgfqpoint{2.740378in}{1.090869in}}{\pgfqpoint{2.729927in}{1.086540in}}{\pgfqpoint{2.722223in}{1.078836in}}%
\pgfpathcurveto{\pgfqpoint{2.714518in}{1.071131in}}{\pgfqpoint{2.710189in}{1.060681in}}{\pgfqpoint{2.710189in}{1.049785in}}%
\pgfpathcurveto{\pgfqpoint{2.710189in}{1.038890in}}{\pgfqpoint{2.714518in}{1.028439in}}{\pgfqpoint{2.722223in}{1.020734in}}%
\pgfpathcurveto{\pgfqpoint{2.729927in}{1.013030in}}{\pgfqpoint{2.740378in}{1.008701in}}{\pgfqpoint{2.751273in}{1.008701in}}%
\pgfpathlineto{\pgfqpoint{2.751273in}{1.008701in}}%
\pgfpathclose%
\pgfusepath{stroke}%
\end{pgfscope}%
\begin{pgfscope}%
\pgfpathrectangle{\pgfqpoint{0.688192in}{0.670138in}}{\pgfqpoint{7.111808in}{5.129862in}}%
\pgfusepath{clip}%
\pgfsetbuttcap%
\pgfsetroundjoin%
\pgfsetlinewidth{1.003750pt}%
\definecolor{currentstroke}{rgb}{0.000000,0.000000,0.000000}%
\pgfsetstrokecolor{currentstroke}%
\pgfsetdash{}{0pt}%
\pgfpathmoveto{\pgfqpoint{1.852165in}{0.694984in}}%
\pgfpathcurveto{\pgfqpoint{1.863060in}{0.694984in}}{\pgfqpoint{1.873511in}{0.699313in}}{\pgfqpoint{1.881215in}{0.707017in}}%
\pgfpathcurveto{\pgfqpoint{1.888920in}{0.714722in}}{\pgfqpoint{1.893248in}{0.725173in}}{\pgfqpoint{1.893248in}{0.736068in}}%
\pgfpathcurveto{\pgfqpoint{1.893248in}{0.746964in}}{\pgfqpoint{1.888920in}{0.757415in}}{\pgfqpoint{1.881215in}{0.765119in}}%
\pgfpathcurveto{\pgfqpoint{1.873511in}{0.772823in}}{\pgfqpoint{1.863060in}{0.777152in}}{\pgfqpoint{1.852165in}{0.777152in}}%
\pgfpathcurveto{\pgfqpoint{1.841269in}{0.777152in}}{\pgfqpoint{1.830818in}{0.772823in}}{\pgfqpoint{1.823114in}{0.765119in}}%
\pgfpathcurveto{\pgfqpoint{1.815410in}{0.757415in}}{\pgfqpoint{1.811081in}{0.746964in}}{\pgfqpoint{1.811081in}{0.736068in}}%
\pgfpathcurveto{\pgfqpoint{1.811081in}{0.725173in}}{\pgfqpoint{1.815410in}{0.714722in}}{\pgfqpoint{1.823114in}{0.707017in}}%
\pgfpathcurveto{\pgfqpoint{1.830818in}{0.699313in}}{\pgfqpoint{1.841269in}{0.694984in}}{\pgfqpoint{1.852165in}{0.694984in}}%
\pgfpathlineto{\pgfqpoint{1.852165in}{0.694984in}}%
\pgfpathclose%
\pgfusepath{stroke}%
\end{pgfscope}%
\begin{pgfscope}%
\pgfpathrectangle{\pgfqpoint{0.688192in}{0.670138in}}{\pgfqpoint{7.111808in}{5.129862in}}%
\pgfusepath{clip}%
\pgfsetbuttcap%
\pgfsetroundjoin%
\pgfsetlinewidth{1.003750pt}%
\definecolor{currentstroke}{rgb}{0.000000,0.000000,0.000000}%
\pgfsetstrokecolor{currentstroke}%
\pgfsetdash{}{0pt}%
\pgfpathmoveto{\pgfqpoint{0.931867in}{0.825133in}}%
\pgfpathcurveto{\pgfqpoint{0.942763in}{0.825133in}}{\pgfqpoint{0.953213in}{0.829462in}}{\pgfqpoint{0.960918in}{0.837167in}}%
\pgfpathcurveto{\pgfqpoint{0.968622in}{0.844871in}}{\pgfqpoint{0.972951in}{0.855322in}}{\pgfqpoint{0.972951in}{0.866217in}}%
\pgfpathcurveto{\pgfqpoint{0.972951in}{0.877113in}}{\pgfqpoint{0.968622in}{0.887564in}}{\pgfqpoint{0.960918in}{0.895268in}}%
\pgfpathcurveto{\pgfqpoint{0.953213in}{0.902972in}}{\pgfqpoint{0.942763in}{0.907301in}}{\pgfqpoint{0.931867in}{0.907301in}}%
\pgfpathcurveto{\pgfqpoint{0.920971in}{0.907301in}}{\pgfqpoint{0.910521in}{0.902972in}}{\pgfqpoint{0.902816in}{0.895268in}}%
\pgfpathcurveto{\pgfqpoint{0.895112in}{0.887564in}}{\pgfqpoint{0.890783in}{0.877113in}}{\pgfqpoint{0.890783in}{0.866217in}}%
\pgfpathcurveto{\pgfqpoint{0.890783in}{0.855322in}}{\pgfqpoint{0.895112in}{0.844871in}}{\pgfqpoint{0.902816in}{0.837167in}}%
\pgfpathcurveto{\pgfqpoint{0.910521in}{0.829462in}}{\pgfqpoint{0.920971in}{0.825133in}}{\pgfqpoint{0.931867in}{0.825133in}}%
\pgfpathlineto{\pgfqpoint{0.931867in}{0.825133in}}%
\pgfpathclose%
\pgfusepath{stroke}%
\end{pgfscope}%
\begin{pgfscope}%
\pgfpathrectangle{\pgfqpoint{0.688192in}{0.670138in}}{\pgfqpoint{7.111808in}{5.129862in}}%
\pgfusepath{clip}%
\pgfsetbuttcap%
\pgfsetroundjoin%
\pgfsetlinewidth{1.003750pt}%
\definecolor{currentstroke}{rgb}{0.000000,0.000000,0.000000}%
\pgfsetstrokecolor{currentstroke}%
\pgfsetdash{}{0pt}%
\pgfpathmoveto{\pgfqpoint{1.448644in}{0.707802in}}%
\pgfpathcurveto{\pgfqpoint{1.459539in}{0.707802in}}{\pgfqpoint{1.469990in}{0.712131in}}{\pgfqpoint{1.477694in}{0.719835in}}%
\pgfpathcurveto{\pgfqpoint{1.485399in}{0.727539in}}{\pgfqpoint{1.489728in}{0.737990in}}{\pgfqpoint{1.489728in}{0.748886in}}%
\pgfpathcurveto{\pgfqpoint{1.489728in}{0.759781in}}{\pgfqpoint{1.485399in}{0.770232in}}{\pgfqpoint{1.477694in}{0.777936in}}%
\pgfpathcurveto{\pgfqpoint{1.469990in}{0.785641in}}{\pgfqpoint{1.459539in}{0.789970in}}{\pgfqpoint{1.448644in}{0.789970in}}%
\pgfpathcurveto{\pgfqpoint{1.437748in}{0.789970in}}{\pgfqpoint{1.427297in}{0.785641in}}{\pgfqpoint{1.419593in}{0.777936in}}%
\pgfpathcurveto{\pgfqpoint{1.411889in}{0.770232in}}{\pgfqpoint{1.407560in}{0.759781in}}{\pgfqpoint{1.407560in}{0.748886in}}%
\pgfpathcurveto{\pgfqpoint{1.407560in}{0.737990in}}{\pgfqpoint{1.411889in}{0.727539in}}{\pgfqpoint{1.419593in}{0.719835in}}%
\pgfpathcurveto{\pgfqpoint{1.427297in}{0.712131in}}{\pgfqpoint{1.437748in}{0.707802in}}{\pgfqpoint{1.448644in}{0.707802in}}%
\pgfpathlineto{\pgfqpoint{1.448644in}{0.707802in}}%
\pgfpathclose%
\pgfusepath{stroke}%
\end{pgfscope}%
\begin{pgfscope}%
\pgfpathrectangle{\pgfqpoint{0.688192in}{0.670138in}}{\pgfqpoint{7.111808in}{5.129862in}}%
\pgfusepath{clip}%
\pgfsetbuttcap%
\pgfsetroundjoin%
\pgfsetlinewidth{1.003750pt}%
\definecolor{currentstroke}{rgb}{0.000000,0.000000,0.000000}%
\pgfsetstrokecolor{currentstroke}%
\pgfsetdash{}{0pt}%
\pgfpathmoveto{\pgfqpoint{3.791217in}{0.647523in}}%
\pgfpathcurveto{\pgfqpoint{3.802112in}{0.647523in}}{\pgfqpoint{3.812563in}{0.651852in}}{\pgfqpoint{3.820267in}{0.659557in}}%
\pgfpathcurveto{\pgfqpoint{3.827972in}{0.667261in}}{\pgfqpoint{3.832300in}{0.677712in}}{\pgfqpoint{3.832300in}{0.688607in}}%
\pgfpathcurveto{\pgfqpoint{3.832300in}{0.699503in}}{\pgfqpoint{3.827972in}{0.709954in}}{\pgfqpoint{3.820267in}{0.717658in}}%
\pgfpathcurveto{\pgfqpoint{3.812563in}{0.725362in}}{\pgfqpoint{3.802112in}{0.729691in}}{\pgfqpoint{3.791217in}{0.729691in}}%
\pgfpathcurveto{\pgfqpoint{3.780321in}{0.729691in}}{\pgfqpoint{3.769870in}{0.725362in}}{\pgfqpoint{3.762166in}{0.717658in}}%
\pgfpathcurveto{\pgfqpoint{3.754462in}{0.709954in}}{\pgfqpoint{3.750133in}{0.699503in}}{\pgfqpoint{3.750133in}{0.688607in}}%
\pgfpathcurveto{\pgfqpoint{3.750133in}{0.677712in}}{\pgfqpoint{3.754462in}{0.667261in}}{\pgfqpoint{3.762166in}{0.659557in}}%
\pgfpathcurveto{\pgfqpoint{3.769870in}{0.651852in}}{\pgfqpoint{3.780321in}{0.647523in}}{\pgfqpoint{3.791217in}{0.647523in}}%
\pgfusepath{stroke}%
\end{pgfscope}%
\begin{pgfscope}%
\pgfpathrectangle{\pgfqpoint{0.688192in}{0.670138in}}{\pgfqpoint{7.111808in}{5.129862in}}%
\pgfusepath{clip}%
\pgfsetbuttcap%
\pgfsetroundjoin%
\pgfsetlinewidth{1.003750pt}%
\definecolor{currentstroke}{rgb}{0.000000,0.000000,0.000000}%
\pgfsetstrokecolor{currentstroke}%
\pgfsetdash{}{0pt}%
\pgfpathmoveto{\pgfqpoint{0.931763in}{0.833664in}}%
\pgfpathcurveto{\pgfqpoint{0.942659in}{0.833664in}}{\pgfqpoint{0.953110in}{0.837992in}}{\pgfqpoint{0.960814in}{0.845697in}}%
\pgfpathcurveto{\pgfqpoint{0.968519in}{0.853401in}}{\pgfqpoint{0.972847in}{0.863852in}}{\pgfqpoint{0.972847in}{0.874747in}}%
\pgfpathcurveto{\pgfqpoint{0.972847in}{0.885643in}}{\pgfqpoint{0.968519in}{0.896094in}}{\pgfqpoint{0.960814in}{0.903798in}}%
\pgfpathcurveto{\pgfqpoint{0.953110in}{0.911502in}}{\pgfqpoint{0.942659in}{0.915831in}}{\pgfqpoint{0.931763in}{0.915831in}}%
\pgfpathcurveto{\pgfqpoint{0.920868in}{0.915831in}}{\pgfqpoint{0.910417in}{0.911502in}}{\pgfqpoint{0.902713in}{0.903798in}}%
\pgfpathcurveto{\pgfqpoint{0.895008in}{0.896094in}}{\pgfqpoint{0.890680in}{0.885643in}}{\pgfqpoint{0.890680in}{0.874747in}}%
\pgfpathcurveto{\pgfqpoint{0.890680in}{0.863852in}}{\pgfqpoint{0.895008in}{0.853401in}}{\pgfqpoint{0.902713in}{0.845697in}}%
\pgfpathcurveto{\pgfqpoint{0.910417in}{0.837992in}}{\pgfqpoint{0.920868in}{0.833664in}}{\pgfqpoint{0.931763in}{0.833664in}}%
\pgfpathlineto{\pgfqpoint{0.931763in}{0.833664in}}%
\pgfpathclose%
\pgfusepath{stroke}%
\end{pgfscope}%
\begin{pgfscope}%
\pgfpathrectangle{\pgfqpoint{0.688192in}{0.670138in}}{\pgfqpoint{7.111808in}{5.129862in}}%
\pgfusepath{clip}%
\pgfsetbuttcap%
\pgfsetroundjoin%
\pgfsetlinewidth{1.003750pt}%
\definecolor{currentstroke}{rgb}{0.000000,0.000000,0.000000}%
\pgfsetstrokecolor{currentstroke}%
\pgfsetdash{}{0pt}%
\pgfpathmoveto{\pgfqpoint{4.737940in}{0.637018in}}%
\pgfpathcurveto{\pgfqpoint{4.748836in}{0.637018in}}{\pgfqpoint{4.759287in}{0.641347in}}{\pgfqpoint{4.766991in}{0.649051in}}%
\pgfpathcurveto{\pgfqpoint{4.774695in}{0.656756in}}{\pgfqpoint{4.779024in}{0.667207in}}{\pgfqpoint{4.779024in}{0.678102in}}%
\pgfpathcurveto{\pgfqpoint{4.779024in}{0.688998in}}{\pgfqpoint{4.774695in}{0.699448in}}{\pgfqpoint{4.766991in}{0.707153in}}%
\pgfpathcurveto{\pgfqpoint{4.759287in}{0.714857in}}{\pgfqpoint{4.748836in}{0.719186in}}{\pgfqpoint{4.737940in}{0.719186in}}%
\pgfpathcurveto{\pgfqpoint{4.727045in}{0.719186in}}{\pgfqpoint{4.716594in}{0.714857in}}{\pgfqpoint{4.708890in}{0.707153in}}%
\pgfpathcurveto{\pgfqpoint{4.701185in}{0.699448in}}{\pgfqpoint{4.696856in}{0.688998in}}{\pgfqpoint{4.696856in}{0.678102in}}%
\pgfpathcurveto{\pgfqpoint{4.696856in}{0.667207in}}{\pgfqpoint{4.701185in}{0.656756in}}{\pgfqpoint{4.708890in}{0.649051in}}%
\pgfpathcurveto{\pgfqpoint{4.716594in}{0.641347in}}{\pgfqpoint{4.727045in}{0.637018in}}{\pgfqpoint{4.737940in}{0.637018in}}%
\pgfusepath{stroke}%
\end{pgfscope}%
\begin{pgfscope}%
\pgfpathrectangle{\pgfqpoint{0.688192in}{0.670138in}}{\pgfqpoint{7.111808in}{5.129862in}}%
\pgfusepath{clip}%
\pgfsetbuttcap%
\pgfsetroundjoin%
\pgfsetlinewidth{1.003750pt}%
\definecolor{currentstroke}{rgb}{0.000000,0.000000,0.000000}%
\pgfsetstrokecolor{currentstroke}%
\pgfsetdash{}{0pt}%
\pgfpathmoveto{\pgfqpoint{5.794967in}{1.746278in}}%
\pgfpathcurveto{\pgfqpoint{5.805862in}{1.746278in}}{\pgfqpoint{5.816313in}{1.750607in}}{\pgfqpoint{5.824018in}{1.758311in}}%
\pgfpathcurveto{\pgfqpoint{5.831722in}{1.766015in}}{\pgfqpoint{5.836051in}{1.776466in}}{\pgfqpoint{5.836051in}{1.787362in}}%
\pgfpathcurveto{\pgfqpoint{5.836051in}{1.798257in}}{\pgfqpoint{5.831722in}{1.808708in}}{\pgfqpoint{5.824018in}{1.816412in}}%
\pgfpathcurveto{\pgfqpoint{5.816313in}{1.824117in}}{\pgfqpoint{5.805862in}{1.828445in}}{\pgfqpoint{5.794967in}{1.828445in}}%
\pgfpathcurveto{\pgfqpoint{5.784071in}{1.828445in}}{\pgfqpoint{5.773621in}{1.824117in}}{\pgfqpoint{5.765916in}{1.816412in}}%
\pgfpathcurveto{\pgfqpoint{5.758212in}{1.808708in}}{\pgfqpoint{5.753883in}{1.798257in}}{\pgfqpoint{5.753883in}{1.787362in}}%
\pgfpathcurveto{\pgfqpoint{5.753883in}{1.776466in}}{\pgfqpoint{5.758212in}{1.766015in}}{\pgfqpoint{5.765916in}{1.758311in}}%
\pgfpathcurveto{\pgfqpoint{5.773621in}{1.750607in}}{\pgfqpoint{5.784071in}{1.746278in}}{\pgfqpoint{5.794967in}{1.746278in}}%
\pgfpathlineto{\pgfqpoint{5.794967in}{1.746278in}}%
\pgfpathclose%
\pgfusepath{stroke}%
\end{pgfscope}%
\begin{pgfscope}%
\pgfpathrectangle{\pgfqpoint{0.688192in}{0.670138in}}{\pgfqpoint{7.111808in}{5.129862in}}%
\pgfusepath{clip}%
\pgfsetbuttcap%
\pgfsetroundjoin%
\pgfsetlinewidth{1.003750pt}%
\definecolor{currentstroke}{rgb}{0.000000,0.000000,0.000000}%
\pgfsetstrokecolor{currentstroke}%
\pgfsetdash{}{0pt}%
\pgfpathmoveto{\pgfqpoint{7.658486in}{3.930259in}}%
\pgfpathcurveto{\pgfqpoint{7.669381in}{3.930259in}}{\pgfqpoint{7.679832in}{3.934588in}}{\pgfqpoint{7.687536in}{3.942292in}}%
\pgfpathcurveto{\pgfqpoint{7.695241in}{3.949997in}}{\pgfqpoint{7.699569in}{3.960447in}}{\pgfqpoint{7.699569in}{3.971343in}}%
\pgfpathcurveto{\pgfqpoint{7.699569in}{3.982239in}}{\pgfqpoint{7.695241in}{3.992689in}}{\pgfqpoint{7.687536in}{4.000394in}}%
\pgfpathcurveto{\pgfqpoint{7.679832in}{4.008098in}}{\pgfqpoint{7.669381in}{4.012427in}}{\pgfqpoint{7.658486in}{4.012427in}}%
\pgfpathcurveto{\pgfqpoint{7.647590in}{4.012427in}}{\pgfqpoint{7.637139in}{4.008098in}}{\pgfqpoint{7.629435in}{4.000394in}}%
\pgfpathcurveto{\pgfqpoint{7.621731in}{3.992689in}}{\pgfqpoint{7.617402in}{3.982239in}}{\pgfqpoint{7.617402in}{3.971343in}}%
\pgfpathcurveto{\pgfqpoint{7.617402in}{3.960447in}}{\pgfqpoint{7.621731in}{3.949997in}}{\pgfqpoint{7.629435in}{3.942292in}}%
\pgfpathcurveto{\pgfqpoint{7.637139in}{3.934588in}}{\pgfqpoint{7.647590in}{3.930259in}}{\pgfqpoint{7.658486in}{3.930259in}}%
\pgfpathlineto{\pgfqpoint{7.658486in}{3.930259in}}%
\pgfpathclose%
\pgfusepath{stroke}%
\end{pgfscope}%
\begin{pgfscope}%
\pgfpathrectangle{\pgfqpoint{0.688192in}{0.670138in}}{\pgfqpoint{7.111808in}{5.129862in}}%
\pgfusepath{clip}%
\pgfsetbuttcap%
\pgfsetroundjoin%
\pgfsetlinewidth{1.003750pt}%
\definecolor{currentstroke}{rgb}{0.000000,0.000000,0.000000}%
\pgfsetstrokecolor{currentstroke}%
\pgfsetdash{}{0pt}%
\pgfpathmoveto{\pgfqpoint{1.389197in}{0.711266in}}%
\pgfpathcurveto{\pgfqpoint{1.400093in}{0.711266in}}{\pgfqpoint{1.410544in}{0.715595in}}{\pgfqpoint{1.418248in}{0.723299in}}%
\pgfpathcurveto{\pgfqpoint{1.425952in}{0.731004in}}{\pgfqpoint{1.430281in}{0.741454in}}{\pgfqpoint{1.430281in}{0.752350in}}%
\pgfpathcurveto{\pgfqpoint{1.430281in}{0.763246in}}{\pgfqpoint{1.425952in}{0.773696in}}{\pgfqpoint{1.418248in}{0.781401in}}%
\pgfpathcurveto{\pgfqpoint{1.410544in}{0.789105in}}{\pgfqpoint{1.400093in}{0.793434in}}{\pgfqpoint{1.389197in}{0.793434in}}%
\pgfpathcurveto{\pgfqpoint{1.378302in}{0.793434in}}{\pgfqpoint{1.367851in}{0.789105in}}{\pgfqpoint{1.360147in}{0.781401in}}%
\pgfpathcurveto{\pgfqpoint{1.352442in}{0.773696in}}{\pgfqpoint{1.348113in}{0.763246in}}{\pgfqpoint{1.348113in}{0.752350in}}%
\pgfpathcurveto{\pgfqpoint{1.348113in}{0.741454in}}{\pgfqpoint{1.352442in}{0.731004in}}{\pgfqpoint{1.360147in}{0.723299in}}%
\pgfpathcurveto{\pgfqpoint{1.367851in}{0.715595in}}{\pgfqpoint{1.378302in}{0.711266in}}{\pgfqpoint{1.389197in}{0.711266in}}%
\pgfpathlineto{\pgfqpoint{1.389197in}{0.711266in}}%
\pgfpathclose%
\pgfusepath{stroke}%
\end{pgfscope}%
\begin{pgfscope}%
\pgfpathrectangle{\pgfqpoint{0.688192in}{0.670138in}}{\pgfqpoint{7.111808in}{5.129862in}}%
\pgfusepath{clip}%
\pgfsetbuttcap%
\pgfsetroundjoin%
\pgfsetlinewidth{1.003750pt}%
\definecolor{currentstroke}{rgb}{0.000000,0.000000,0.000000}%
\pgfsetstrokecolor{currentstroke}%
\pgfsetdash{}{0pt}%
\pgfpathmoveto{\pgfqpoint{0.957757in}{0.807940in}}%
\pgfpathcurveto{\pgfqpoint{0.968653in}{0.807940in}}{\pgfqpoint{0.979103in}{0.812269in}}{\pgfqpoint{0.986808in}{0.819974in}}%
\pgfpathcurveto{\pgfqpoint{0.994512in}{0.827678in}}{\pgfqpoint{0.998841in}{0.838129in}}{\pgfqpoint{0.998841in}{0.849024in}}%
\pgfpathcurveto{\pgfqpoint{0.998841in}{0.859920in}}{\pgfqpoint{0.994512in}{0.870371in}}{\pgfqpoint{0.986808in}{0.878075in}}%
\pgfpathcurveto{\pgfqpoint{0.979103in}{0.885779in}}{\pgfqpoint{0.968653in}{0.890108in}}{\pgfqpoint{0.957757in}{0.890108in}}%
\pgfpathcurveto{\pgfqpoint{0.946862in}{0.890108in}}{\pgfqpoint{0.936411in}{0.885779in}}{\pgfqpoint{0.928706in}{0.878075in}}%
\pgfpathcurveto{\pgfqpoint{0.921002in}{0.870371in}}{\pgfqpoint{0.916673in}{0.859920in}}{\pgfqpoint{0.916673in}{0.849024in}}%
\pgfpathcurveto{\pgfqpoint{0.916673in}{0.838129in}}{\pgfqpoint{0.921002in}{0.827678in}}{\pgfqpoint{0.928706in}{0.819974in}}%
\pgfpathcurveto{\pgfqpoint{0.936411in}{0.812269in}}{\pgfqpoint{0.946862in}{0.807940in}}{\pgfqpoint{0.957757in}{0.807940in}}%
\pgfpathlineto{\pgfqpoint{0.957757in}{0.807940in}}%
\pgfpathclose%
\pgfusepath{stroke}%
\end{pgfscope}%
\begin{pgfscope}%
\pgfpathrectangle{\pgfqpoint{0.688192in}{0.670138in}}{\pgfqpoint{7.111808in}{5.129862in}}%
\pgfusepath{clip}%
\pgfsetbuttcap%
\pgfsetroundjoin%
\pgfsetlinewidth{1.003750pt}%
\definecolor{currentstroke}{rgb}{0.000000,0.000000,0.000000}%
\pgfsetstrokecolor{currentstroke}%
\pgfsetdash{}{0pt}%
\pgfpathmoveto{\pgfqpoint{0.899610in}{0.859043in}}%
\pgfpathcurveto{\pgfqpoint{0.910506in}{0.859043in}}{\pgfqpoint{0.920957in}{0.863371in}}{\pgfqpoint{0.928661in}{0.871076in}}%
\pgfpathcurveto{\pgfqpoint{0.936365in}{0.878780in}}{\pgfqpoint{0.940694in}{0.889231in}}{\pgfqpoint{0.940694in}{0.900126in}}%
\pgfpathcurveto{\pgfqpoint{0.940694in}{0.911022in}}{\pgfqpoint{0.936365in}{0.921473in}}{\pgfqpoint{0.928661in}{0.929177in}}%
\pgfpathcurveto{\pgfqpoint{0.920957in}{0.936881in}}{\pgfqpoint{0.910506in}{0.941210in}}{\pgfqpoint{0.899610in}{0.941210in}}%
\pgfpathcurveto{\pgfqpoint{0.888715in}{0.941210in}}{\pgfqpoint{0.878264in}{0.936881in}}{\pgfqpoint{0.870560in}{0.929177in}}%
\pgfpathcurveto{\pgfqpoint{0.862855in}{0.921473in}}{\pgfqpoint{0.858526in}{0.911022in}}{\pgfqpoint{0.858526in}{0.900126in}}%
\pgfpathcurveto{\pgfqpoint{0.858526in}{0.889231in}}{\pgfqpoint{0.862855in}{0.878780in}}{\pgfqpoint{0.870560in}{0.871076in}}%
\pgfpathcurveto{\pgfqpoint{0.878264in}{0.863371in}}{\pgfqpoint{0.888715in}{0.859043in}}{\pgfqpoint{0.899610in}{0.859043in}}%
\pgfpathlineto{\pgfqpoint{0.899610in}{0.859043in}}%
\pgfpathclose%
\pgfusepath{stroke}%
\end{pgfscope}%
\begin{pgfscope}%
\pgfpathrectangle{\pgfqpoint{0.688192in}{0.670138in}}{\pgfqpoint{7.111808in}{5.129862in}}%
\pgfusepath{clip}%
\pgfsetbuttcap%
\pgfsetroundjoin%
\pgfsetlinewidth{1.003750pt}%
\definecolor{currentstroke}{rgb}{0.000000,0.000000,0.000000}%
\pgfsetstrokecolor{currentstroke}%
\pgfsetdash{}{0pt}%
\pgfpathmoveto{\pgfqpoint{3.745627in}{2.077577in}}%
\pgfpathcurveto{\pgfqpoint{3.756522in}{2.077577in}}{\pgfqpoint{3.766973in}{2.081906in}}{\pgfqpoint{3.774677in}{2.089610in}}%
\pgfpathcurveto{\pgfqpoint{3.782382in}{2.097315in}}{\pgfqpoint{3.786710in}{2.107765in}}{\pgfqpoint{3.786710in}{2.118661in}}%
\pgfpathcurveto{\pgfqpoint{3.786710in}{2.129556in}}{\pgfqpoint{3.782382in}{2.140007in}}{\pgfqpoint{3.774677in}{2.147712in}}%
\pgfpathcurveto{\pgfqpoint{3.766973in}{2.155416in}}{\pgfqpoint{3.756522in}{2.159745in}}{\pgfqpoint{3.745627in}{2.159745in}}%
\pgfpathcurveto{\pgfqpoint{3.734731in}{2.159745in}}{\pgfqpoint{3.724280in}{2.155416in}}{\pgfqpoint{3.716576in}{2.147712in}}%
\pgfpathcurveto{\pgfqpoint{3.708872in}{2.140007in}}{\pgfqpoint{3.704543in}{2.129556in}}{\pgfqpoint{3.704543in}{2.118661in}}%
\pgfpathcurveto{\pgfqpoint{3.704543in}{2.107765in}}{\pgfqpoint{3.708872in}{2.097315in}}{\pgfqpoint{3.716576in}{2.089610in}}%
\pgfpathcurveto{\pgfqpoint{3.724280in}{2.081906in}}{\pgfqpoint{3.734731in}{2.077577in}}{\pgfqpoint{3.745627in}{2.077577in}}%
\pgfpathlineto{\pgfqpoint{3.745627in}{2.077577in}}%
\pgfpathclose%
\pgfusepath{stroke}%
\end{pgfscope}%
\begin{pgfscope}%
\pgfpathrectangle{\pgfqpoint{0.688192in}{0.670138in}}{\pgfqpoint{7.111808in}{5.129862in}}%
\pgfusepath{clip}%
\pgfsetbuttcap%
\pgfsetroundjoin%
\pgfsetlinewidth{1.003750pt}%
\definecolor{currentstroke}{rgb}{0.000000,0.000000,0.000000}%
\pgfsetstrokecolor{currentstroke}%
\pgfsetdash{}{0pt}%
\pgfpathmoveto{\pgfqpoint{1.278130in}{0.714122in}}%
\pgfpathcurveto{\pgfqpoint{1.289026in}{0.714122in}}{\pgfqpoint{1.299477in}{0.718451in}}{\pgfqpoint{1.307181in}{0.726155in}}%
\pgfpathcurveto{\pgfqpoint{1.314885in}{0.733860in}}{\pgfqpoint{1.319214in}{0.744311in}}{\pgfqpoint{1.319214in}{0.755206in}}%
\pgfpathcurveto{\pgfqpoint{1.319214in}{0.766102in}}{\pgfqpoint{1.314885in}{0.776553in}}{\pgfqpoint{1.307181in}{0.784257in}}%
\pgfpathcurveto{\pgfqpoint{1.299477in}{0.791961in}}{\pgfqpoint{1.289026in}{0.796290in}}{\pgfqpoint{1.278130in}{0.796290in}}%
\pgfpathcurveto{\pgfqpoint{1.267235in}{0.796290in}}{\pgfqpoint{1.256784in}{0.791961in}}{\pgfqpoint{1.249080in}{0.784257in}}%
\pgfpathcurveto{\pgfqpoint{1.241375in}{0.776553in}}{\pgfqpoint{1.237047in}{0.766102in}}{\pgfqpoint{1.237047in}{0.755206in}}%
\pgfpathcurveto{\pgfqpoint{1.237047in}{0.744311in}}{\pgfqpoint{1.241375in}{0.733860in}}{\pgfqpoint{1.249080in}{0.726155in}}%
\pgfpathcurveto{\pgfqpoint{1.256784in}{0.718451in}}{\pgfqpoint{1.267235in}{0.714122in}}{\pgfqpoint{1.278130in}{0.714122in}}%
\pgfpathlineto{\pgfqpoint{1.278130in}{0.714122in}}%
\pgfpathclose%
\pgfusepath{stroke}%
\end{pgfscope}%
\begin{pgfscope}%
\pgfpathrectangle{\pgfqpoint{0.688192in}{0.670138in}}{\pgfqpoint{7.111808in}{5.129862in}}%
\pgfusepath{clip}%
\pgfsetbuttcap%
\pgfsetroundjoin%
\pgfsetlinewidth{1.003750pt}%
\definecolor{currentstroke}{rgb}{0.000000,0.000000,0.000000}%
\pgfsetstrokecolor{currentstroke}%
\pgfsetdash{}{0pt}%
\pgfpathmoveto{\pgfqpoint{2.497167in}{0.675229in}}%
\pgfpathcurveto{\pgfqpoint{2.508063in}{0.675229in}}{\pgfqpoint{2.518514in}{0.679557in}}{\pgfqpoint{2.526218in}{0.687262in}}%
\pgfpathcurveto{\pgfqpoint{2.533922in}{0.694966in}}{\pgfqpoint{2.538251in}{0.705417in}}{\pgfqpoint{2.538251in}{0.716312in}}%
\pgfpathcurveto{\pgfqpoint{2.538251in}{0.727208in}}{\pgfqpoint{2.533922in}{0.737659in}}{\pgfqpoint{2.526218in}{0.745363in}}%
\pgfpathcurveto{\pgfqpoint{2.518514in}{0.753067in}}{\pgfqpoint{2.508063in}{0.757396in}}{\pgfqpoint{2.497167in}{0.757396in}}%
\pgfpathcurveto{\pgfqpoint{2.486272in}{0.757396in}}{\pgfqpoint{2.475821in}{0.753067in}}{\pgfqpoint{2.468117in}{0.745363in}}%
\pgfpathcurveto{\pgfqpoint{2.460412in}{0.737659in}}{\pgfqpoint{2.456084in}{0.727208in}}{\pgfqpoint{2.456084in}{0.716312in}}%
\pgfpathcurveto{\pgfqpoint{2.456084in}{0.705417in}}{\pgfqpoint{2.460412in}{0.694966in}}{\pgfqpoint{2.468117in}{0.687262in}}%
\pgfpathcurveto{\pgfqpoint{2.475821in}{0.679557in}}{\pgfqpoint{2.486272in}{0.675229in}}{\pgfqpoint{2.497167in}{0.675229in}}%
\pgfpathlineto{\pgfqpoint{2.497167in}{0.675229in}}%
\pgfpathclose%
\pgfusepath{stroke}%
\end{pgfscope}%
\begin{pgfscope}%
\pgfpathrectangle{\pgfqpoint{0.688192in}{0.670138in}}{\pgfqpoint{7.111808in}{5.129862in}}%
\pgfusepath{clip}%
\pgfsetbuttcap%
\pgfsetroundjoin%
\pgfsetlinewidth{1.003750pt}%
\definecolor{currentstroke}{rgb}{0.000000,0.000000,0.000000}%
\pgfsetstrokecolor{currentstroke}%
\pgfsetdash{}{0pt}%
\pgfpathmoveto{\pgfqpoint{1.016759in}{0.762048in}}%
\pgfpathcurveto{\pgfqpoint{1.027654in}{0.762048in}}{\pgfqpoint{1.038105in}{0.766376in}}{\pgfqpoint{1.045809in}{0.774081in}}%
\pgfpathcurveto{\pgfqpoint{1.053514in}{0.781785in}}{\pgfqpoint{1.057843in}{0.792236in}}{\pgfqpoint{1.057843in}{0.803132in}}%
\pgfpathcurveto{\pgfqpoint{1.057843in}{0.814027in}}{\pgfqpoint{1.053514in}{0.824478in}}{\pgfqpoint{1.045809in}{0.832182in}}%
\pgfpathcurveto{\pgfqpoint{1.038105in}{0.839887in}}{\pgfqpoint{1.027654in}{0.844215in}}{\pgfqpoint{1.016759in}{0.844215in}}%
\pgfpathcurveto{\pgfqpoint{1.005863in}{0.844215in}}{\pgfqpoint{0.995412in}{0.839887in}}{\pgfqpoint{0.987708in}{0.832182in}}%
\pgfpathcurveto{\pgfqpoint{0.980004in}{0.824478in}}{\pgfqpoint{0.975675in}{0.814027in}}{\pgfqpoint{0.975675in}{0.803132in}}%
\pgfpathcurveto{\pgfqpoint{0.975675in}{0.792236in}}{\pgfqpoint{0.980004in}{0.781785in}}{\pgfqpoint{0.987708in}{0.774081in}}%
\pgfpathcurveto{\pgfqpoint{0.995412in}{0.766376in}}{\pgfqpoint{1.005863in}{0.762048in}}{\pgfqpoint{1.016759in}{0.762048in}}%
\pgfpathlineto{\pgfqpoint{1.016759in}{0.762048in}}%
\pgfpathclose%
\pgfusepath{stroke}%
\end{pgfscope}%
\begin{pgfscope}%
\pgfpathrectangle{\pgfqpoint{0.688192in}{0.670138in}}{\pgfqpoint{7.111808in}{5.129862in}}%
\pgfusepath{clip}%
\pgfsetbuttcap%
\pgfsetroundjoin%
\pgfsetlinewidth{1.003750pt}%
\definecolor{currentstroke}{rgb}{0.000000,0.000000,0.000000}%
\pgfsetstrokecolor{currentstroke}%
\pgfsetdash{}{0pt}%
\pgfpathmoveto{\pgfqpoint{1.419139in}{0.710712in}}%
\pgfpathcurveto{\pgfqpoint{1.430035in}{0.710712in}}{\pgfqpoint{1.440485in}{0.715041in}}{\pgfqpoint{1.448190in}{0.722745in}}%
\pgfpathcurveto{\pgfqpoint{1.455894in}{0.730450in}}{\pgfqpoint{1.460223in}{0.740900in}}{\pgfqpoint{1.460223in}{0.751796in}}%
\pgfpathcurveto{\pgfqpoint{1.460223in}{0.762691in}}{\pgfqpoint{1.455894in}{0.773142in}}{\pgfqpoint{1.448190in}{0.780847in}}%
\pgfpathcurveto{\pgfqpoint{1.440485in}{0.788551in}}{\pgfqpoint{1.430035in}{0.792880in}}{\pgfqpoint{1.419139in}{0.792880in}}%
\pgfpathcurveto{\pgfqpoint{1.408244in}{0.792880in}}{\pgfqpoint{1.397793in}{0.788551in}}{\pgfqpoint{1.390088in}{0.780847in}}%
\pgfpathcurveto{\pgfqpoint{1.382384in}{0.773142in}}{\pgfqpoint{1.378055in}{0.762691in}}{\pgfqpoint{1.378055in}{0.751796in}}%
\pgfpathcurveto{\pgfqpoint{1.378055in}{0.740900in}}{\pgfqpoint{1.382384in}{0.730450in}}{\pgfqpoint{1.390088in}{0.722745in}}%
\pgfpathcurveto{\pgfqpoint{1.397793in}{0.715041in}}{\pgfqpoint{1.408244in}{0.710712in}}{\pgfqpoint{1.419139in}{0.710712in}}%
\pgfpathlineto{\pgfqpoint{1.419139in}{0.710712in}}%
\pgfpathclose%
\pgfusepath{stroke}%
\end{pgfscope}%
\begin{pgfscope}%
\pgfpathrectangle{\pgfqpoint{0.688192in}{0.670138in}}{\pgfqpoint{7.111808in}{5.129862in}}%
\pgfusepath{clip}%
\pgfsetbuttcap%
\pgfsetroundjoin%
\pgfsetlinewidth{1.003750pt}%
\definecolor{currentstroke}{rgb}{0.000000,0.000000,0.000000}%
\pgfsetstrokecolor{currentstroke}%
\pgfsetdash{}{0pt}%
\pgfpathmoveto{\pgfqpoint{0.790083in}{1.032876in}}%
\pgfpathcurveto{\pgfqpoint{0.800978in}{1.032876in}}{\pgfqpoint{0.811429in}{1.037204in}}{\pgfqpoint{0.819133in}{1.044909in}}%
\pgfpathcurveto{\pgfqpoint{0.826838in}{1.052613in}}{\pgfqpoint{0.831167in}{1.063064in}}{\pgfqpoint{0.831167in}{1.073959in}}%
\pgfpathcurveto{\pgfqpoint{0.831167in}{1.084855in}}{\pgfqpoint{0.826838in}{1.095306in}}{\pgfqpoint{0.819133in}{1.103010in}}%
\pgfpathcurveto{\pgfqpoint{0.811429in}{1.110715in}}{\pgfqpoint{0.800978in}{1.115043in}}{\pgfqpoint{0.790083in}{1.115043in}}%
\pgfpathcurveto{\pgfqpoint{0.779187in}{1.115043in}}{\pgfqpoint{0.768736in}{1.110715in}}{\pgfqpoint{0.761032in}{1.103010in}}%
\pgfpathcurveto{\pgfqpoint{0.753328in}{1.095306in}}{\pgfqpoint{0.748999in}{1.084855in}}{\pgfqpoint{0.748999in}{1.073959in}}%
\pgfpathcurveto{\pgfqpoint{0.748999in}{1.063064in}}{\pgfqpoint{0.753328in}{1.052613in}}{\pgfqpoint{0.761032in}{1.044909in}}%
\pgfpathcurveto{\pgfqpoint{0.768736in}{1.037204in}}{\pgfqpoint{0.779187in}{1.032876in}}{\pgfqpoint{0.790083in}{1.032876in}}%
\pgfpathlineto{\pgfqpoint{0.790083in}{1.032876in}}%
\pgfpathclose%
\pgfusepath{stroke}%
\end{pgfscope}%
\begin{pgfscope}%
\pgfpathrectangle{\pgfqpoint{0.688192in}{0.670138in}}{\pgfqpoint{7.111808in}{5.129862in}}%
\pgfusepath{clip}%
\pgfsetbuttcap%
\pgfsetroundjoin%
\pgfsetlinewidth{1.003750pt}%
\definecolor{currentstroke}{rgb}{0.000000,0.000000,0.000000}%
\pgfsetstrokecolor{currentstroke}%
\pgfsetdash{}{0pt}%
\pgfpathmoveto{\pgfqpoint{5.261268in}{0.632886in}}%
\pgfpathcurveto{\pgfqpoint{5.272163in}{0.632886in}}{\pgfqpoint{5.282614in}{0.637215in}}{\pgfqpoint{5.290319in}{0.644920in}}%
\pgfpathcurveto{\pgfqpoint{5.298023in}{0.652624in}}{\pgfqpoint{5.302352in}{0.663075in}}{\pgfqpoint{5.302352in}{0.673970in}}%
\pgfpathcurveto{\pgfqpoint{5.302352in}{0.684866in}}{\pgfqpoint{5.298023in}{0.695317in}}{\pgfqpoint{5.290319in}{0.703021in}}%
\pgfpathcurveto{\pgfqpoint{5.282614in}{0.710725in}}{\pgfqpoint{5.272163in}{0.715054in}}{\pgfqpoint{5.261268in}{0.715054in}}%
\pgfpathcurveto{\pgfqpoint{5.250372in}{0.715054in}}{\pgfqpoint{5.239921in}{0.710725in}}{\pgfqpoint{5.232217in}{0.703021in}}%
\pgfpathcurveto{\pgfqpoint{5.224513in}{0.695317in}}{\pgfqpoint{5.220184in}{0.684866in}}{\pgfqpoint{5.220184in}{0.673970in}}%
\pgfpathcurveto{\pgfqpoint{5.220184in}{0.663075in}}{\pgfqpoint{5.224513in}{0.652624in}}{\pgfqpoint{5.232217in}{0.644920in}}%
\pgfpathcurveto{\pgfqpoint{5.239921in}{0.637215in}}{\pgfqpoint{5.250372in}{0.632886in}}{\pgfqpoint{5.261268in}{0.632886in}}%
\pgfusepath{stroke}%
\end{pgfscope}%
\begin{pgfscope}%
\pgfpathrectangle{\pgfqpoint{0.688192in}{0.670138in}}{\pgfqpoint{7.111808in}{5.129862in}}%
\pgfusepath{clip}%
\pgfsetbuttcap%
\pgfsetroundjoin%
\pgfsetlinewidth{1.003750pt}%
\definecolor{currentstroke}{rgb}{0.000000,0.000000,0.000000}%
\pgfsetstrokecolor{currentstroke}%
\pgfsetdash{}{0pt}%
\pgfpathmoveto{\pgfqpoint{5.696293in}{0.630769in}}%
\pgfpathcurveto{\pgfqpoint{5.707189in}{0.630769in}}{\pgfqpoint{5.717639in}{0.635098in}}{\pgfqpoint{5.725344in}{0.642802in}}%
\pgfpathcurveto{\pgfqpoint{5.733048in}{0.650507in}}{\pgfqpoint{5.737377in}{0.660958in}}{\pgfqpoint{5.737377in}{0.671853in}}%
\pgfpathcurveto{\pgfqpoint{5.737377in}{0.682749in}}{\pgfqpoint{5.733048in}{0.693200in}}{\pgfqpoint{5.725344in}{0.700904in}}%
\pgfpathcurveto{\pgfqpoint{5.717639in}{0.708608in}}{\pgfqpoint{5.707189in}{0.712937in}}{\pgfqpoint{5.696293in}{0.712937in}}%
\pgfpathcurveto{\pgfqpoint{5.685397in}{0.712937in}}{\pgfqpoint{5.674947in}{0.708608in}}{\pgfqpoint{5.667242in}{0.700904in}}%
\pgfpathcurveto{\pgfqpoint{5.659538in}{0.693200in}}{\pgfqpoint{5.655209in}{0.682749in}}{\pgfqpoint{5.655209in}{0.671853in}}%
\pgfpathcurveto{\pgfqpoint{5.655209in}{0.660958in}}{\pgfqpoint{5.659538in}{0.650507in}}{\pgfqpoint{5.667242in}{0.642802in}}%
\pgfpathcurveto{\pgfqpoint{5.674947in}{0.635098in}}{\pgfqpoint{5.685397in}{0.630769in}}{\pgfqpoint{5.696293in}{0.630769in}}%
\pgfusepath{stroke}%
\end{pgfscope}%
\begin{pgfscope}%
\pgfpathrectangle{\pgfqpoint{0.688192in}{0.670138in}}{\pgfqpoint{7.111808in}{5.129862in}}%
\pgfusepath{clip}%
\pgfsetbuttcap%
\pgfsetroundjoin%
\pgfsetlinewidth{1.003750pt}%
\definecolor{currentstroke}{rgb}{0.000000,0.000000,0.000000}%
\pgfsetstrokecolor{currentstroke}%
\pgfsetdash{}{0pt}%
\pgfpathmoveto{\pgfqpoint{1.091394in}{0.722252in}}%
\pgfpathcurveto{\pgfqpoint{1.102290in}{0.722252in}}{\pgfqpoint{1.112741in}{0.726581in}}{\pgfqpoint{1.120445in}{0.734285in}}%
\pgfpathcurveto{\pgfqpoint{1.128149in}{0.741989in}}{\pgfqpoint{1.132478in}{0.752440in}}{\pgfqpoint{1.132478in}{0.763336in}}%
\pgfpathcurveto{\pgfqpoint{1.132478in}{0.774231in}}{\pgfqpoint{1.128149in}{0.784682in}}{\pgfqpoint{1.120445in}{0.792386in}}%
\pgfpathcurveto{\pgfqpoint{1.112741in}{0.800091in}}{\pgfqpoint{1.102290in}{0.804420in}}{\pgfqpoint{1.091394in}{0.804420in}}%
\pgfpathcurveto{\pgfqpoint{1.080499in}{0.804420in}}{\pgfqpoint{1.070048in}{0.800091in}}{\pgfqpoint{1.062344in}{0.792386in}}%
\pgfpathcurveto{\pgfqpoint{1.054639in}{0.784682in}}{\pgfqpoint{1.050310in}{0.774231in}}{\pgfqpoint{1.050310in}{0.763336in}}%
\pgfpathcurveto{\pgfqpoint{1.050310in}{0.752440in}}{\pgfqpoint{1.054639in}{0.741989in}}{\pgfqpoint{1.062344in}{0.734285in}}%
\pgfpathcurveto{\pgfqpoint{1.070048in}{0.726581in}}{\pgfqpoint{1.080499in}{0.722252in}}{\pgfqpoint{1.091394in}{0.722252in}}%
\pgfpathlineto{\pgfqpoint{1.091394in}{0.722252in}}%
\pgfpathclose%
\pgfusepath{stroke}%
\end{pgfscope}%
\begin{pgfscope}%
\pgfpathrectangle{\pgfqpoint{0.688192in}{0.670138in}}{\pgfqpoint{7.111808in}{5.129862in}}%
\pgfusepath{clip}%
\pgfsetbuttcap%
\pgfsetroundjoin%
\pgfsetlinewidth{1.003750pt}%
\definecolor{currentstroke}{rgb}{0.000000,0.000000,0.000000}%
\pgfsetstrokecolor{currentstroke}%
\pgfsetdash{}{0pt}%
\pgfpathmoveto{\pgfqpoint{1.045306in}{0.741536in}}%
\pgfpathcurveto{\pgfqpoint{1.056202in}{0.741536in}}{\pgfqpoint{1.066653in}{0.745865in}}{\pgfqpoint{1.074357in}{0.753569in}}%
\pgfpathcurveto{\pgfqpoint{1.082061in}{0.761274in}}{\pgfqpoint{1.086390in}{0.771725in}}{\pgfqpoint{1.086390in}{0.782620in}}%
\pgfpathcurveto{\pgfqpoint{1.086390in}{0.793516in}}{\pgfqpoint{1.082061in}{0.803967in}}{\pgfqpoint{1.074357in}{0.811671in}}%
\pgfpathcurveto{\pgfqpoint{1.066653in}{0.819375in}}{\pgfqpoint{1.056202in}{0.823704in}}{\pgfqpoint{1.045306in}{0.823704in}}%
\pgfpathcurveto{\pgfqpoint{1.034411in}{0.823704in}}{\pgfqpoint{1.023960in}{0.819375in}}{\pgfqpoint{1.016256in}{0.811671in}}%
\pgfpathcurveto{\pgfqpoint{1.008551in}{0.803967in}}{\pgfqpoint{1.004222in}{0.793516in}}{\pgfqpoint{1.004222in}{0.782620in}}%
\pgfpathcurveto{\pgfqpoint{1.004222in}{0.771725in}}{\pgfqpoint{1.008551in}{0.761274in}}{\pgfqpoint{1.016256in}{0.753569in}}%
\pgfpathcurveto{\pgfqpoint{1.023960in}{0.745865in}}{\pgfqpoint{1.034411in}{0.741536in}}{\pgfqpoint{1.045306in}{0.741536in}}%
\pgfpathlineto{\pgfqpoint{1.045306in}{0.741536in}}%
\pgfpathclose%
\pgfusepath{stroke}%
\end{pgfscope}%
\begin{pgfscope}%
\pgfpathrectangle{\pgfqpoint{0.688192in}{0.670138in}}{\pgfqpoint{7.111808in}{5.129862in}}%
\pgfusepath{clip}%
\pgfsetbuttcap%
\pgfsetroundjoin%
\pgfsetlinewidth{1.003750pt}%
\definecolor{currentstroke}{rgb}{0.000000,0.000000,0.000000}%
\pgfsetstrokecolor{currentstroke}%
\pgfsetdash{}{0pt}%
\pgfpathmoveto{\pgfqpoint{1.320633in}{0.712781in}}%
\pgfpathcurveto{\pgfqpoint{1.331529in}{0.712781in}}{\pgfqpoint{1.341980in}{0.717110in}}{\pgfqpoint{1.349684in}{0.724814in}}%
\pgfpathcurveto{\pgfqpoint{1.357388in}{0.732518in}}{\pgfqpoint{1.361717in}{0.742969in}}{\pgfqpoint{1.361717in}{0.753865in}}%
\pgfpathcurveto{\pgfqpoint{1.361717in}{0.764760in}}{\pgfqpoint{1.357388in}{0.775211in}}{\pgfqpoint{1.349684in}{0.782915in}}%
\pgfpathcurveto{\pgfqpoint{1.341980in}{0.790620in}}{\pgfqpoint{1.331529in}{0.794949in}}{\pgfqpoint{1.320633in}{0.794949in}}%
\pgfpathcurveto{\pgfqpoint{1.309738in}{0.794949in}}{\pgfqpoint{1.299287in}{0.790620in}}{\pgfqpoint{1.291583in}{0.782915in}}%
\pgfpathcurveto{\pgfqpoint{1.283878in}{0.775211in}}{\pgfqpoint{1.279549in}{0.764760in}}{\pgfqpoint{1.279549in}{0.753865in}}%
\pgfpathcurveto{\pgfqpoint{1.279549in}{0.742969in}}{\pgfqpoint{1.283878in}{0.732518in}}{\pgfqpoint{1.291583in}{0.724814in}}%
\pgfpathcurveto{\pgfqpoint{1.299287in}{0.717110in}}{\pgfqpoint{1.309738in}{0.712781in}}{\pgfqpoint{1.320633in}{0.712781in}}%
\pgfpathlineto{\pgfqpoint{1.320633in}{0.712781in}}%
\pgfpathclose%
\pgfusepath{stroke}%
\end{pgfscope}%
\begin{pgfscope}%
\pgfpathrectangle{\pgfqpoint{0.688192in}{0.670138in}}{\pgfqpoint{7.111808in}{5.129862in}}%
\pgfusepath{clip}%
\pgfsetbuttcap%
\pgfsetroundjoin%
\pgfsetlinewidth{1.003750pt}%
\definecolor{currentstroke}{rgb}{0.000000,0.000000,0.000000}%
\pgfsetstrokecolor{currentstroke}%
\pgfsetdash{}{0pt}%
\pgfpathmoveto{\pgfqpoint{2.869035in}{0.669385in}}%
\pgfpathcurveto{\pgfqpoint{2.879931in}{0.669385in}}{\pgfqpoint{2.890382in}{0.673714in}}{\pgfqpoint{2.898086in}{0.681419in}}%
\pgfpathcurveto{\pgfqpoint{2.905790in}{0.689123in}}{\pgfqpoint{2.910119in}{0.699574in}}{\pgfqpoint{2.910119in}{0.710469in}}%
\pgfpathcurveto{\pgfqpoint{2.910119in}{0.721365in}}{\pgfqpoint{2.905790in}{0.731816in}}{\pgfqpoint{2.898086in}{0.739520in}}%
\pgfpathcurveto{\pgfqpoint{2.890382in}{0.747224in}}{\pgfqpoint{2.879931in}{0.751553in}}{\pgfqpoint{2.869035in}{0.751553in}}%
\pgfpathcurveto{\pgfqpoint{2.858140in}{0.751553in}}{\pgfqpoint{2.847689in}{0.747224in}}{\pgfqpoint{2.839985in}{0.739520in}}%
\pgfpathcurveto{\pgfqpoint{2.832280in}{0.731816in}}{\pgfqpoint{2.827952in}{0.721365in}}{\pgfqpoint{2.827952in}{0.710469in}}%
\pgfpathcurveto{\pgfqpoint{2.827952in}{0.699574in}}{\pgfqpoint{2.832280in}{0.689123in}}{\pgfqpoint{2.839985in}{0.681419in}}%
\pgfpathcurveto{\pgfqpoint{2.847689in}{0.673714in}}{\pgfqpoint{2.858140in}{0.669385in}}{\pgfqpoint{2.869035in}{0.669385in}}%
\pgfpathlineto{\pgfqpoint{2.869035in}{0.669385in}}%
\pgfpathclose%
\pgfusepath{stroke}%
\end{pgfscope}%
\begin{pgfscope}%
\pgfpathrectangle{\pgfqpoint{0.688192in}{0.670138in}}{\pgfqpoint{7.111808in}{5.129862in}}%
\pgfusepath{clip}%
\pgfsetbuttcap%
\pgfsetroundjoin%
\pgfsetlinewidth{1.003750pt}%
\definecolor{currentstroke}{rgb}{0.000000,0.000000,0.000000}%
\pgfsetstrokecolor{currentstroke}%
\pgfsetdash{}{0pt}%
\pgfpathmoveto{\pgfqpoint{0.899610in}{0.859043in}}%
\pgfpathcurveto{\pgfqpoint{0.910506in}{0.859043in}}{\pgfqpoint{0.920957in}{0.863371in}}{\pgfqpoint{0.928661in}{0.871076in}}%
\pgfpathcurveto{\pgfqpoint{0.936365in}{0.878780in}}{\pgfqpoint{0.940694in}{0.889231in}}{\pgfqpoint{0.940694in}{0.900126in}}%
\pgfpathcurveto{\pgfqpoint{0.940694in}{0.911022in}}{\pgfqpoint{0.936365in}{0.921473in}}{\pgfqpoint{0.928661in}{0.929177in}}%
\pgfpathcurveto{\pgfqpoint{0.920957in}{0.936881in}}{\pgfqpoint{0.910506in}{0.941210in}}{\pgfqpoint{0.899610in}{0.941210in}}%
\pgfpathcurveto{\pgfqpoint{0.888715in}{0.941210in}}{\pgfqpoint{0.878264in}{0.936881in}}{\pgfqpoint{0.870560in}{0.929177in}}%
\pgfpathcurveto{\pgfqpoint{0.862855in}{0.921473in}}{\pgfqpoint{0.858526in}{0.911022in}}{\pgfqpoint{0.858526in}{0.900126in}}%
\pgfpathcurveto{\pgfqpoint{0.858526in}{0.889231in}}{\pgfqpoint{0.862855in}{0.878780in}}{\pgfqpoint{0.870560in}{0.871076in}}%
\pgfpathcurveto{\pgfqpoint{0.878264in}{0.863371in}}{\pgfqpoint{0.888715in}{0.859043in}}{\pgfqpoint{0.899610in}{0.859043in}}%
\pgfpathlineto{\pgfqpoint{0.899610in}{0.859043in}}%
\pgfpathclose%
\pgfusepath{stroke}%
\end{pgfscope}%
\begin{pgfscope}%
\pgfpathrectangle{\pgfqpoint{0.688192in}{0.670138in}}{\pgfqpoint{7.111808in}{5.129862in}}%
\pgfusepath{clip}%
\pgfsetbuttcap%
\pgfsetroundjoin%
\pgfsetlinewidth{1.003750pt}%
\definecolor{currentstroke}{rgb}{0.000000,0.000000,0.000000}%
\pgfsetstrokecolor{currentstroke}%
\pgfsetdash{}{0pt}%
\pgfpathmoveto{\pgfqpoint{1.120660in}{0.719620in}}%
\pgfpathcurveto{\pgfqpoint{1.131555in}{0.719620in}}{\pgfqpoint{1.142006in}{0.723949in}}{\pgfqpoint{1.149710in}{0.731653in}}%
\pgfpathcurveto{\pgfqpoint{1.157415in}{0.739358in}}{\pgfqpoint{1.161744in}{0.749809in}}{\pgfqpoint{1.161744in}{0.760704in}}%
\pgfpathcurveto{\pgfqpoint{1.161744in}{0.771600in}}{\pgfqpoint{1.157415in}{0.782051in}}{\pgfqpoint{1.149710in}{0.789755in}}%
\pgfpathcurveto{\pgfqpoint{1.142006in}{0.797459in}}{\pgfqpoint{1.131555in}{0.801788in}}{\pgfqpoint{1.120660in}{0.801788in}}%
\pgfpathcurveto{\pgfqpoint{1.109764in}{0.801788in}}{\pgfqpoint{1.099313in}{0.797459in}}{\pgfqpoint{1.091609in}{0.789755in}}%
\pgfpathcurveto{\pgfqpoint{1.083905in}{0.782051in}}{\pgfqpoint{1.079576in}{0.771600in}}{\pgfqpoint{1.079576in}{0.760704in}}%
\pgfpathcurveto{\pgfqpoint{1.079576in}{0.749809in}}{\pgfqpoint{1.083905in}{0.739358in}}{\pgfqpoint{1.091609in}{0.731653in}}%
\pgfpathcurveto{\pgfqpoint{1.099313in}{0.723949in}}{\pgfqpoint{1.109764in}{0.719620in}}{\pgfqpoint{1.120660in}{0.719620in}}%
\pgfpathlineto{\pgfqpoint{1.120660in}{0.719620in}}%
\pgfpathclose%
\pgfusepath{stroke}%
\end{pgfscope}%
\begin{pgfscope}%
\pgfpathrectangle{\pgfqpoint{0.688192in}{0.670138in}}{\pgfqpoint{7.111808in}{5.129862in}}%
\pgfusepath{clip}%
\pgfsetbuttcap%
\pgfsetroundjoin%
\pgfsetlinewidth{1.003750pt}%
\definecolor{currentstroke}{rgb}{0.000000,0.000000,0.000000}%
\pgfsetstrokecolor{currentstroke}%
\pgfsetdash{}{0pt}%
\pgfpathmoveto{\pgfqpoint{2.497167in}{0.675229in}}%
\pgfpathcurveto{\pgfqpoint{2.508063in}{0.675229in}}{\pgfqpoint{2.518514in}{0.679557in}}{\pgfqpoint{2.526218in}{0.687262in}}%
\pgfpathcurveto{\pgfqpoint{2.533922in}{0.694966in}}{\pgfqpoint{2.538251in}{0.705417in}}{\pgfqpoint{2.538251in}{0.716312in}}%
\pgfpathcurveto{\pgfqpoint{2.538251in}{0.727208in}}{\pgfqpoint{2.533922in}{0.737659in}}{\pgfqpoint{2.526218in}{0.745363in}}%
\pgfpathcurveto{\pgfqpoint{2.518514in}{0.753067in}}{\pgfqpoint{2.508063in}{0.757396in}}{\pgfqpoint{2.497167in}{0.757396in}}%
\pgfpathcurveto{\pgfqpoint{2.486272in}{0.757396in}}{\pgfqpoint{2.475821in}{0.753067in}}{\pgfqpoint{2.468117in}{0.745363in}}%
\pgfpathcurveto{\pgfqpoint{2.460412in}{0.737659in}}{\pgfqpoint{2.456084in}{0.727208in}}{\pgfqpoint{2.456084in}{0.716312in}}%
\pgfpathcurveto{\pgfqpoint{2.456084in}{0.705417in}}{\pgfqpoint{2.460412in}{0.694966in}}{\pgfqpoint{2.468117in}{0.687262in}}%
\pgfpathcurveto{\pgfqpoint{2.475821in}{0.679557in}}{\pgfqpoint{2.486272in}{0.675229in}}{\pgfqpoint{2.497167in}{0.675229in}}%
\pgfpathlineto{\pgfqpoint{2.497167in}{0.675229in}}%
\pgfpathclose%
\pgfusepath{stroke}%
\end{pgfscope}%
\begin{pgfscope}%
\pgfpathrectangle{\pgfqpoint{0.688192in}{0.670138in}}{\pgfqpoint{7.111808in}{5.129862in}}%
\pgfusepath{clip}%
\pgfsetbuttcap%
\pgfsetroundjoin%
\pgfsetlinewidth{1.003750pt}%
\definecolor{currentstroke}{rgb}{0.000000,0.000000,0.000000}%
\pgfsetstrokecolor{currentstroke}%
\pgfsetdash{}{0pt}%
\pgfpathmoveto{\pgfqpoint{3.134854in}{1.222230in}}%
\pgfpathcurveto{\pgfqpoint{3.145749in}{1.222230in}}{\pgfqpoint{3.156200in}{1.226558in}}{\pgfqpoint{3.163905in}{1.234263in}}%
\pgfpathcurveto{\pgfqpoint{3.171609in}{1.241967in}}{\pgfqpoint{3.175938in}{1.252418in}}{\pgfqpoint{3.175938in}{1.263314in}}%
\pgfpathcurveto{\pgfqpoint{3.175938in}{1.274209in}}{\pgfqpoint{3.171609in}{1.284660in}}{\pgfqpoint{3.163905in}{1.292364in}}%
\pgfpathcurveto{\pgfqpoint{3.156200in}{1.300069in}}{\pgfqpoint{3.145749in}{1.304397in}}{\pgfqpoint{3.134854in}{1.304397in}}%
\pgfpathcurveto{\pgfqpoint{3.123958in}{1.304397in}}{\pgfqpoint{3.113507in}{1.300069in}}{\pgfqpoint{3.105803in}{1.292364in}}%
\pgfpathcurveto{\pgfqpoint{3.098099in}{1.284660in}}{\pgfqpoint{3.093770in}{1.274209in}}{\pgfqpoint{3.093770in}{1.263314in}}%
\pgfpathcurveto{\pgfqpoint{3.093770in}{1.252418in}}{\pgfqpoint{3.098099in}{1.241967in}}{\pgfqpoint{3.105803in}{1.234263in}}%
\pgfpathcurveto{\pgfqpoint{3.113507in}{1.226558in}}{\pgfqpoint{3.123958in}{1.222230in}}{\pgfqpoint{3.134854in}{1.222230in}}%
\pgfpathlineto{\pgfqpoint{3.134854in}{1.222230in}}%
\pgfpathclose%
\pgfusepath{stroke}%
\end{pgfscope}%
\begin{pgfscope}%
\pgfpathrectangle{\pgfqpoint{0.688192in}{0.670138in}}{\pgfqpoint{7.111808in}{5.129862in}}%
\pgfusepath{clip}%
\pgfsetbuttcap%
\pgfsetroundjoin%
\pgfsetlinewidth{1.003750pt}%
\definecolor{currentstroke}{rgb}{0.000000,0.000000,0.000000}%
\pgfsetstrokecolor{currentstroke}%
\pgfsetdash{}{0pt}%
\pgfpathmoveto{\pgfqpoint{1.091394in}{0.722252in}}%
\pgfpathcurveto{\pgfqpoint{1.102290in}{0.722252in}}{\pgfqpoint{1.112741in}{0.726581in}}{\pgfqpoint{1.120445in}{0.734285in}}%
\pgfpathcurveto{\pgfqpoint{1.128149in}{0.741989in}}{\pgfqpoint{1.132478in}{0.752440in}}{\pgfqpoint{1.132478in}{0.763336in}}%
\pgfpathcurveto{\pgfqpoint{1.132478in}{0.774231in}}{\pgfqpoint{1.128149in}{0.784682in}}{\pgfqpoint{1.120445in}{0.792386in}}%
\pgfpathcurveto{\pgfqpoint{1.112741in}{0.800091in}}{\pgfqpoint{1.102290in}{0.804420in}}{\pgfqpoint{1.091394in}{0.804420in}}%
\pgfpathcurveto{\pgfqpoint{1.080499in}{0.804420in}}{\pgfqpoint{1.070048in}{0.800091in}}{\pgfqpoint{1.062344in}{0.792386in}}%
\pgfpathcurveto{\pgfqpoint{1.054639in}{0.784682in}}{\pgfqpoint{1.050310in}{0.774231in}}{\pgfqpoint{1.050310in}{0.763336in}}%
\pgfpathcurveto{\pgfqpoint{1.050310in}{0.752440in}}{\pgfqpoint{1.054639in}{0.741989in}}{\pgfqpoint{1.062344in}{0.734285in}}%
\pgfpathcurveto{\pgfqpoint{1.070048in}{0.726581in}}{\pgfqpoint{1.080499in}{0.722252in}}{\pgfqpoint{1.091394in}{0.722252in}}%
\pgfpathlineto{\pgfqpoint{1.091394in}{0.722252in}}%
\pgfpathclose%
\pgfusepath{stroke}%
\end{pgfscope}%
\begin{pgfscope}%
\pgfpathrectangle{\pgfqpoint{0.688192in}{0.670138in}}{\pgfqpoint{7.111808in}{5.129862in}}%
\pgfusepath{clip}%
\pgfsetbuttcap%
\pgfsetroundjoin%
\pgfsetlinewidth{1.003750pt}%
\definecolor{currentstroke}{rgb}{0.000000,0.000000,0.000000}%
\pgfsetstrokecolor{currentstroke}%
\pgfsetdash{}{0pt}%
\pgfpathmoveto{\pgfqpoint{0.732621in}{1.454102in}}%
\pgfpathcurveto{\pgfqpoint{0.743517in}{1.454102in}}{\pgfqpoint{0.753967in}{1.458430in}}{\pgfqpoint{0.761672in}{1.466135in}}%
\pgfpathcurveto{\pgfqpoint{0.769376in}{1.473839in}}{\pgfqpoint{0.773705in}{1.484290in}}{\pgfqpoint{0.773705in}{1.495185in}}%
\pgfpathcurveto{\pgfqpoint{0.773705in}{1.506081in}}{\pgfqpoint{0.769376in}{1.516532in}}{\pgfqpoint{0.761672in}{1.524236in}}%
\pgfpathcurveto{\pgfqpoint{0.753967in}{1.531941in}}{\pgfqpoint{0.743517in}{1.536269in}}{\pgfqpoint{0.732621in}{1.536269in}}%
\pgfpathcurveto{\pgfqpoint{0.721726in}{1.536269in}}{\pgfqpoint{0.711275in}{1.531941in}}{\pgfqpoint{0.703570in}{1.524236in}}%
\pgfpathcurveto{\pgfqpoint{0.695866in}{1.516532in}}{\pgfqpoint{0.691537in}{1.506081in}}{\pgfqpoint{0.691537in}{1.495185in}}%
\pgfpathcurveto{\pgfqpoint{0.691537in}{1.484290in}}{\pgfqpoint{0.695866in}{1.473839in}}{\pgfqpoint{0.703570in}{1.466135in}}%
\pgfpathcurveto{\pgfqpoint{0.711275in}{1.458430in}}{\pgfqpoint{0.721726in}{1.454102in}}{\pgfqpoint{0.732621in}{1.454102in}}%
\pgfpathlineto{\pgfqpoint{0.732621in}{1.454102in}}%
\pgfpathclose%
\pgfusepath{stroke}%
\end{pgfscope}%
\begin{pgfscope}%
\pgfpathrectangle{\pgfqpoint{0.688192in}{0.670138in}}{\pgfqpoint{7.111808in}{5.129862in}}%
\pgfusepath{clip}%
\pgfsetbuttcap%
\pgfsetroundjoin%
\pgfsetlinewidth{1.003750pt}%
\definecolor{currentstroke}{rgb}{0.000000,0.000000,0.000000}%
\pgfsetstrokecolor{currentstroke}%
\pgfsetdash{}{0pt}%
\pgfpathmoveto{\pgfqpoint{0.928243in}{0.841597in}}%
\pgfpathcurveto{\pgfqpoint{0.939139in}{0.841597in}}{\pgfqpoint{0.949589in}{0.845926in}}{\pgfqpoint{0.957294in}{0.853631in}}%
\pgfpathcurveto{\pgfqpoint{0.964998in}{0.861335in}}{\pgfqpoint{0.969327in}{0.871786in}}{\pgfqpoint{0.969327in}{0.882681in}}%
\pgfpathcurveto{\pgfqpoint{0.969327in}{0.893577in}}{\pgfqpoint{0.964998in}{0.904028in}}{\pgfqpoint{0.957294in}{0.911732in}}%
\pgfpathcurveto{\pgfqpoint{0.949589in}{0.919436in}}{\pgfqpoint{0.939139in}{0.923765in}}{\pgfqpoint{0.928243in}{0.923765in}}%
\pgfpathcurveto{\pgfqpoint{0.917347in}{0.923765in}}{\pgfqpoint{0.906897in}{0.919436in}}{\pgfqpoint{0.899192in}{0.911732in}}%
\pgfpathcurveto{\pgfqpoint{0.891488in}{0.904028in}}{\pgfqpoint{0.887159in}{0.893577in}}{\pgfqpoint{0.887159in}{0.882681in}}%
\pgfpathcurveto{\pgfqpoint{0.887159in}{0.871786in}}{\pgfqpoint{0.891488in}{0.861335in}}{\pgfqpoint{0.899192in}{0.853631in}}%
\pgfpathcurveto{\pgfqpoint{0.906897in}{0.845926in}}{\pgfqpoint{0.917347in}{0.841597in}}{\pgfqpoint{0.928243in}{0.841597in}}%
\pgfpathlineto{\pgfqpoint{0.928243in}{0.841597in}}%
\pgfpathclose%
\pgfusepath{stroke}%
\end{pgfscope}%
\begin{pgfscope}%
\pgfpathrectangle{\pgfqpoint{0.688192in}{0.670138in}}{\pgfqpoint{7.111808in}{5.129862in}}%
\pgfusepath{clip}%
\pgfsetbuttcap%
\pgfsetroundjoin%
\pgfsetlinewidth{1.003750pt}%
\definecolor{currentstroke}{rgb}{0.000000,0.000000,0.000000}%
\pgfsetstrokecolor{currentstroke}%
\pgfsetdash{}{0pt}%
\pgfpathmoveto{\pgfqpoint{1.045306in}{0.741536in}}%
\pgfpathcurveto{\pgfqpoint{1.056202in}{0.741536in}}{\pgfqpoint{1.066653in}{0.745865in}}{\pgfqpoint{1.074357in}{0.753569in}}%
\pgfpathcurveto{\pgfqpoint{1.082061in}{0.761274in}}{\pgfqpoint{1.086390in}{0.771725in}}{\pgfqpoint{1.086390in}{0.782620in}}%
\pgfpathcurveto{\pgfqpoint{1.086390in}{0.793516in}}{\pgfqpoint{1.082061in}{0.803967in}}{\pgfqpoint{1.074357in}{0.811671in}}%
\pgfpathcurveto{\pgfqpoint{1.066653in}{0.819375in}}{\pgfqpoint{1.056202in}{0.823704in}}{\pgfqpoint{1.045306in}{0.823704in}}%
\pgfpathcurveto{\pgfqpoint{1.034411in}{0.823704in}}{\pgfqpoint{1.023960in}{0.819375in}}{\pgfqpoint{1.016256in}{0.811671in}}%
\pgfpathcurveto{\pgfqpoint{1.008551in}{0.803967in}}{\pgfqpoint{1.004222in}{0.793516in}}{\pgfqpoint{1.004222in}{0.782620in}}%
\pgfpathcurveto{\pgfqpoint{1.004222in}{0.771725in}}{\pgfqpoint{1.008551in}{0.761274in}}{\pgfqpoint{1.016256in}{0.753569in}}%
\pgfpathcurveto{\pgfqpoint{1.023960in}{0.745865in}}{\pgfqpoint{1.034411in}{0.741536in}}{\pgfqpoint{1.045306in}{0.741536in}}%
\pgfpathlineto{\pgfqpoint{1.045306in}{0.741536in}}%
\pgfpathclose%
\pgfusepath{stroke}%
\end{pgfscope}%
\begin{pgfscope}%
\pgfpathrectangle{\pgfqpoint{0.688192in}{0.670138in}}{\pgfqpoint{7.111808in}{5.129862in}}%
\pgfusepath{clip}%
\pgfsetbuttcap%
\pgfsetroundjoin%
\pgfsetlinewidth{1.003750pt}%
\definecolor{currentstroke}{rgb}{0.000000,0.000000,0.000000}%
\pgfsetstrokecolor{currentstroke}%
\pgfsetdash{}{0pt}%
\pgfpathmoveto{\pgfqpoint{1.678976in}{0.702425in}}%
\pgfpathcurveto{\pgfqpoint{1.689871in}{0.702425in}}{\pgfqpoint{1.700322in}{0.706753in}}{\pgfqpoint{1.708027in}{0.714458in}}%
\pgfpathcurveto{\pgfqpoint{1.715731in}{0.722162in}}{\pgfqpoint{1.720060in}{0.732613in}}{\pgfqpoint{1.720060in}{0.743508in}}%
\pgfpathcurveto{\pgfqpoint{1.720060in}{0.754404in}}{\pgfqpoint{1.715731in}{0.764855in}}{\pgfqpoint{1.708027in}{0.772559in}}%
\pgfpathcurveto{\pgfqpoint{1.700322in}{0.780264in}}{\pgfqpoint{1.689871in}{0.784592in}}{\pgfqpoint{1.678976in}{0.784592in}}%
\pgfpathcurveto{\pgfqpoint{1.668080in}{0.784592in}}{\pgfqpoint{1.657630in}{0.780264in}}{\pgfqpoint{1.649925in}{0.772559in}}%
\pgfpathcurveto{\pgfqpoint{1.642221in}{0.764855in}}{\pgfqpoint{1.637892in}{0.754404in}}{\pgfqpoint{1.637892in}{0.743508in}}%
\pgfpathcurveto{\pgfqpoint{1.637892in}{0.732613in}}{\pgfqpoint{1.642221in}{0.722162in}}{\pgfqpoint{1.649925in}{0.714458in}}%
\pgfpathcurveto{\pgfqpoint{1.657630in}{0.706753in}}{\pgfqpoint{1.668080in}{0.702425in}}{\pgfqpoint{1.678976in}{0.702425in}}%
\pgfpathlineto{\pgfqpoint{1.678976in}{0.702425in}}%
\pgfpathclose%
\pgfusepath{stroke}%
\end{pgfscope}%
\begin{pgfscope}%
\pgfpathrectangle{\pgfqpoint{0.688192in}{0.670138in}}{\pgfqpoint{7.111808in}{5.129862in}}%
\pgfusepath{clip}%
\pgfsetbuttcap%
\pgfsetroundjoin%
\pgfsetlinewidth{1.003750pt}%
\definecolor{currentstroke}{rgb}{0.000000,0.000000,0.000000}%
\pgfsetstrokecolor{currentstroke}%
\pgfsetdash{}{0pt}%
\pgfpathmoveto{\pgfqpoint{4.909318in}{0.860401in}}%
\pgfpathcurveto{\pgfqpoint{4.920213in}{0.860401in}}{\pgfqpoint{4.930664in}{0.864730in}}{\pgfqpoint{4.938368in}{0.872434in}}%
\pgfpathcurveto{\pgfqpoint{4.946073in}{0.880138in}}{\pgfqpoint{4.950402in}{0.890589in}}{\pgfqpoint{4.950402in}{0.901485in}}%
\pgfpathcurveto{\pgfqpoint{4.950402in}{0.912380in}}{\pgfqpoint{4.946073in}{0.922831in}}{\pgfqpoint{4.938368in}{0.930535in}}%
\pgfpathcurveto{\pgfqpoint{4.930664in}{0.938240in}}{\pgfqpoint{4.920213in}{0.942569in}}{\pgfqpoint{4.909318in}{0.942569in}}%
\pgfpathcurveto{\pgfqpoint{4.898422in}{0.942569in}}{\pgfqpoint{4.887971in}{0.938240in}}{\pgfqpoint{4.880267in}{0.930535in}}%
\pgfpathcurveto{\pgfqpoint{4.872563in}{0.922831in}}{\pgfqpoint{4.868234in}{0.912380in}}{\pgfqpoint{4.868234in}{0.901485in}}%
\pgfpathcurveto{\pgfqpoint{4.868234in}{0.890589in}}{\pgfqpoint{4.872563in}{0.880138in}}{\pgfqpoint{4.880267in}{0.872434in}}%
\pgfpathcurveto{\pgfqpoint{4.887971in}{0.864730in}}{\pgfqpoint{4.898422in}{0.860401in}}{\pgfqpoint{4.909318in}{0.860401in}}%
\pgfpathlineto{\pgfqpoint{4.909318in}{0.860401in}}%
\pgfpathclose%
\pgfusepath{stroke}%
\end{pgfscope}%
\begin{pgfscope}%
\pgfpathrectangle{\pgfqpoint{0.688192in}{0.670138in}}{\pgfqpoint{7.111808in}{5.129862in}}%
\pgfusepath{clip}%
\pgfsetbuttcap%
\pgfsetroundjoin%
\pgfsetlinewidth{1.003750pt}%
\definecolor{currentstroke}{rgb}{0.000000,0.000000,0.000000}%
\pgfsetstrokecolor{currentstroke}%
\pgfsetdash{}{0pt}%
\pgfpathmoveto{\pgfqpoint{5.252737in}{0.633161in}}%
\pgfpathcurveto{\pgfqpoint{5.263632in}{0.633161in}}{\pgfqpoint{5.274083in}{0.637490in}}{\pgfqpoint{5.281788in}{0.645194in}}%
\pgfpathcurveto{\pgfqpoint{5.289492in}{0.652898in}}{\pgfqpoint{5.293821in}{0.663349in}}{\pgfqpoint{5.293821in}{0.674245in}}%
\pgfpathcurveto{\pgfqpoint{5.293821in}{0.685140in}}{\pgfqpoint{5.289492in}{0.695591in}}{\pgfqpoint{5.281788in}{0.703295in}}%
\pgfpathcurveto{\pgfqpoint{5.274083in}{0.711000in}}{\pgfqpoint{5.263632in}{0.715328in}}{\pgfqpoint{5.252737in}{0.715328in}}%
\pgfpathcurveto{\pgfqpoint{5.241841in}{0.715328in}}{\pgfqpoint{5.231390in}{0.711000in}}{\pgfqpoint{5.223686in}{0.703295in}}%
\pgfpathcurveto{\pgfqpoint{5.215982in}{0.695591in}}{\pgfqpoint{5.211653in}{0.685140in}}{\pgfqpoint{5.211653in}{0.674245in}}%
\pgfpathcurveto{\pgfqpoint{5.211653in}{0.663349in}}{\pgfqpoint{5.215982in}{0.652898in}}{\pgfqpoint{5.223686in}{0.645194in}}%
\pgfpathcurveto{\pgfqpoint{5.231390in}{0.637490in}}{\pgfqpoint{5.241841in}{0.633161in}}{\pgfqpoint{5.252737in}{0.633161in}}%
\pgfusepath{stroke}%
\end{pgfscope}%
\begin{pgfscope}%
\pgfpathrectangle{\pgfqpoint{0.688192in}{0.670138in}}{\pgfqpoint{7.111808in}{5.129862in}}%
\pgfusepath{clip}%
\pgfsetbuttcap%
\pgfsetroundjoin%
\pgfsetlinewidth{1.003750pt}%
\definecolor{currentstroke}{rgb}{0.000000,0.000000,0.000000}%
\pgfsetstrokecolor{currentstroke}%
\pgfsetdash{}{0pt}%
\pgfpathmoveto{\pgfqpoint{2.360638in}{2.745826in}}%
\pgfpathcurveto{\pgfqpoint{2.371534in}{2.745826in}}{\pgfqpoint{2.381985in}{2.750155in}}{\pgfqpoint{2.389689in}{2.757859in}}%
\pgfpathcurveto{\pgfqpoint{2.397393in}{2.765564in}}{\pgfqpoint{2.401722in}{2.776014in}}{\pgfqpoint{2.401722in}{2.786910in}}%
\pgfpathcurveto{\pgfqpoint{2.401722in}{2.797806in}}{\pgfqpoint{2.397393in}{2.808256in}}{\pgfqpoint{2.389689in}{2.815961in}}%
\pgfpathcurveto{\pgfqpoint{2.381985in}{2.823665in}}{\pgfqpoint{2.371534in}{2.827994in}}{\pgfqpoint{2.360638in}{2.827994in}}%
\pgfpathcurveto{\pgfqpoint{2.349743in}{2.827994in}}{\pgfqpoint{2.339292in}{2.823665in}}{\pgfqpoint{2.331588in}{2.815961in}}%
\pgfpathcurveto{\pgfqpoint{2.323883in}{2.808256in}}{\pgfqpoint{2.319554in}{2.797806in}}{\pgfqpoint{2.319554in}{2.786910in}}%
\pgfpathcurveto{\pgfqpoint{2.319554in}{2.776014in}}{\pgfqpoint{2.323883in}{2.765564in}}{\pgfqpoint{2.331588in}{2.757859in}}%
\pgfpathcurveto{\pgfqpoint{2.339292in}{2.750155in}}{\pgfqpoint{2.349743in}{2.745826in}}{\pgfqpoint{2.360638in}{2.745826in}}%
\pgfpathlineto{\pgfqpoint{2.360638in}{2.745826in}}%
\pgfpathclose%
\pgfusepath{stroke}%
\end{pgfscope}%
\begin{pgfscope}%
\pgfpathrectangle{\pgfqpoint{0.688192in}{0.670138in}}{\pgfqpoint{7.111808in}{5.129862in}}%
\pgfusepath{clip}%
\pgfsetbuttcap%
\pgfsetroundjoin%
\pgfsetlinewidth{1.003750pt}%
\definecolor{currentstroke}{rgb}{0.000000,0.000000,0.000000}%
\pgfsetstrokecolor{currentstroke}%
\pgfsetdash{}{0pt}%
\pgfpathmoveto{\pgfqpoint{1.091394in}{0.722252in}}%
\pgfpathcurveto{\pgfqpoint{1.102290in}{0.722252in}}{\pgfqpoint{1.112741in}{0.726581in}}{\pgfqpoint{1.120445in}{0.734285in}}%
\pgfpathcurveto{\pgfqpoint{1.128149in}{0.741989in}}{\pgfqpoint{1.132478in}{0.752440in}}{\pgfqpoint{1.132478in}{0.763336in}}%
\pgfpathcurveto{\pgfqpoint{1.132478in}{0.774231in}}{\pgfqpoint{1.128149in}{0.784682in}}{\pgfqpoint{1.120445in}{0.792386in}}%
\pgfpathcurveto{\pgfqpoint{1.112741in}{0.800091in}}{\pgfqpoint{1.102290in}{0.804420in}}{\pgfqpoint{1.091394in}{0.804420in}}%
\pgfpathcurveto{\pgfqpoint{1.080499in}{0.804420in}}{\pgfqpoint{1.070048in}{0.800091in}}{\pgfqpoint{1.062344in}{0.792386in}}%
\pgfpathcurveto{\pgfqpoint{1.054639in}{0.784682in}}{\pgfqpoint{1.050310in}{0.774231in}}{\pgfqpoint{1.050310in}{0.763336in}}%
\pgfpathcurveto{\pgfqpoint{1.050310in}{0.752440in}}{\pgfqpoint{1.054639in}{0.741989in}}{\pgfqpoint{1.062344in}{0.734285in}}%
\pgfpathcurveto{\pgfqpoint{1.070048in}{0.726581in}}{\pgfqpoint{1.080499in}{0.722252in}}{\pgfqpoint{1.091394in}{0.722252in}}%
\pgfpathlineto{\pgfqpoint{1.091394in}{0.722252in}}%
\pgfpathclose%
\pgfusepath{stroke}%
\end{pgfscope}%
\begin{pgfscope}%
\pgfpathrectangle{\pgfqpoint{0.688192in}{0.670138in}}{\pgfqpoint{7.111808in}{5.129862in}}%
\pgfusepath{clip}%
\pgfsetbuttcap%
\pgfsetroundjoin%
\pgfsetlinewidth{1.003750pt}%
\definecolor{currentstroke}{rgb}{0.000000,0.000000,0.000000}%
\pgfsetstrokecolor{currentstroke}%
\pgfsetdash{}{0pt}%
\pgfpathmoveto{\pgfqpoint{4.586127in}{0.640844in}}%
\pgfpathcurveto{\pgfqpoint{4.597023in}{0.640844in}}{\pgfqpoint{4.607473in}{0.645173in}}{\pgfqpoint{4.615178in}{0.652877in}}%
\pgfpathcurveto{\pgfqpoint{4.622882in}{0.660582in}}{\pgfqpoint{4.627211in}{0.671032in}}{\pgfqpoint{4.627211in}{0.681928in}}%
\pgfpathcurveto{\pgfqpoint{4.627211in}{0.692824in}}{\pgfqpoint{4.622882in}{0.703274in}}{\pgfqpoint{4.615178in}{0.710979in}}%
\pgfpathcurveto{\pgfqpoint{4.607473in}{0.718683in}}{\pgfqpoint{4.597023in}{0.723012in}}{\pgfqpoint{4.586127in}{0.723012in}}%
\pgfpathcurveto{\pgfqpoint{4.575231in}{0.723012in}}{\pgfqpoint{4.564781in}{0.718683in}}{\pgfqpoint{4.557076in}{0.710979in}}%
\pgfpathcurveto{\pgfqpoint{4.549372in}{0.703274in}}{\pgfqpoint{4.545043in}{0.692824in}}{\pgfqpoint{4.545043in}{0.681928in}}%
\pgfpathcurveto{\pgfqpoint{4.545043in}{0.671032in}}{\pgfqpoint{4.549372in}{0.660582in}}{\pgfqpoint{4.557076in}{0.652877in}}%
\pgfpathcurveto{\pgfqpoint{4.564781in}{0.645173in}}{\pgfqpoint{4.575231in}{0.640844in}}{\pgfqpoint{4.586127in}{0.640844in}}%
\pgfusepath{stroke}%
\end{pgfscope}%
\begin{pgfscope}%
\pgfpathrectangle{\pgfqpoint{0.688192in}{0.670138in}}{\pgfqpoint{7.111808in}{5.129862in}}%
\pgfusepath{clip}%
\pgfsetbuttcap%
\pgfsetroundjoin%
\pgfsetlinewidth{1.003750pt}%
\definecolor{currentstroke}{rgb}{0.000000,0.000000,0.000000}%
\pgfsetstrokecolor{currentstroke}%
\pgfsetdash{}{0pt}%
\pgfpathmoveto{\pgfqpoint{2.869035in}{0.669385in}}%
\pgfpathcurveto{\pgfqpoint{2.879931in}{0.669385in}}{\pgfqpoint{2.890382in}{0.673714in}}{\pgfqpoint{2.898086in}{0.681419in}}%
\pgfpathcurveto{\pgfqpoint{2.905790in}{0.689123in}}{\pgfqpoint{2.910119in}{0.699574in}}{\pgfqpoint{2.910119in}{0.710469in}}%
\pgfpathcurveto{\pgfqpoint{2.910119in}{0.721365in}}{\pgfqpoint{2.905790in}{0.731816in}}{\pgfqpoint{2.898086in}{0.739520in}}%
\pgfpathcurveto{\pgfqpoint{2.890382in}{0.747224in}}{\pgfqpoint{2.879931in}{0.751553in}}{\pgfqpoint{2.869035in}{0.751553in}}%
\pgfpathcurveto{\pgfqpoint{2.858140in}{0.751553in}}{\pgfqpoint{2.847689in}{0.747224in}}{\pgfqpoint{2.839985in}{0.739520in}}%
\pgfpathcurveto{\pgfqpoint{2.832280in}{0.731816in}}{\pgfqpoint{2.827952in}{0.721365in}}{\pgfqpoint{2.827952in}{0.710469in}}%
\pgfpathcurveto{\pgfqpoint{2.827952in}{0.699574in}}{\pgfqpoint{2.832280in}{0.689123in}}{\pgfqpoint{2.839985in}{0.681419in}}%
\pgfpathcurveto{\pgfqpoint{2.847689in}{0.673714in}}{\pgfqpoint{2.858140in}{0.669385in}}{\pgfqpoint{2.869035in}{0.669385in}}%
\pgfpathlineto{\pgfqpoint{2.869035in}{0.669385in}}%
\pgfpathclose%
\pgfusepath{stroke}%
\end{pgfscope}%
\begin{pgfscope}%
\pgfpathrectangle{\pgfqpoint{0.688192in}{0.670138in}}{\pgfqpoint{7.111808in}{5.129862in}}%
\pgfusepath{clip}%
\pgfsetbuttcap%
\pgfsetroundjoin%
\pgfsetlinewidth{1.003750pt}%
\definecolor{currentstroke}{rgb}{0.000000,0.000000,0.000000}%
\pgfsetstrokecolor{currentstroke}%
\pgfsetdash{}{0pt}%
\pgfpathmoveto{\pgfqpoint{1.351667in}{0.711476in}}%
\pgfpathcurveto{\pgfqpoint{1.362562in}{0.711476in}}{\pgfqpoint{1.373013in}{0.715805in}}{\pgfqpoint{1.380718in}{0.723510in}}%
\pgfpathcurveto{\pgfqpoint{1.388422in}{0.731214in}}{\pgfqpoint{1.392751in}{0.741665in}}{\pgfqpoint{1.392751in}{0.752560in}}%
\pgfpathcurveto{\pgfqpoint{1.392751in}{0.763456in}}{\pgfqpoint{1.388422in}{0.773907in}}{\pgfqpoint{1.380718in}{0.781611in}}%
\pgfpathcurveto{\pgfqpoint{1.373013in}{0.789315in}}{\pgfqpoint{1.362562in}{0.793644in}}{\pgfqpoint{1.351667in}{0.793644in}}%
\pgfpathcurveto{\pgfqpoint{1.340771in}{0.793644in}}{\pgfqpoint{1.330321in}{0.789315in}}{\pgfqpoint{1.322616in}{0.781611in}}%
\pgfpathcurveto{\pgfqpoint{1.314912in}{0.773907in}}{\pgfqpoint{1.310583in}{0.763456in}}{\pgfqpoint{1.310583in}{0.752560in}}%
\pgfpathcurveto{\pgfqpoint{1.310583in}{0.741665in}}{\pgfqpoint{1.314912in}{0.731214in}}{\pgfqpoint{1.322616in}{0.723510in}}%
\pgfpathcurveto{\pgfqpoint{1.330321in}{0.715805in}}{\pgfqpoint{1.340771in}{0.711476in}}{\pgfqpoint{1.351667in}{0.711476in}}%
\pgfpathlineto{\pgfqpoint{1.351667in}{0.711476in}}%
\pgfpathclose%
\pgfusepath{stroke}%
\end{pgfscope}%
\begin{pgfscope}%
\pgfpathrectangle{\pgfqpoint{0.688192in}{0.670138in}}{\pgfqpoint{7.111808in}{5.129862in}}%
\pgfusepath{clip}%
\pgfsetbuttcap%
\pgfsetroundjoin%
\pgfsetlinewidth{1.003750pt}%
\definecolor{currentstroke}{rgb}{0.000000,0.000000,0.000000}%
\pgfsetstrokecolor{currentstroke}%
\pgfsetdash{}{0pt}%
\pgfpathmoveto{\pgfqpoint{0.881277in}{0.900428in}}%
\pgfpathcurveto{\pgfqpoint{0.892173in}{0.900428in}}{\pgfqpoint{0.902624in}{0.904756in}}{\pgfqpoint{0.910328in}{0.912461in}}%
\pgfpathcurveto{\pgfqpoint{0.918032in}{0.920165in}}{\pgfqpoint{0.922361in}{0.930616in}}{\pgfqpoint{0.922361in}{0.941511in}}%
\pgfpathcurveto{\pgfqpoint{0.922361in}{0.952407in}}{\pgfqpoint{0.918032in}{0.962858in}}{\pgfqpoint{0.910328in}{0.970562in}}%
\pgfpathcurveto{\pgfqpoint{0.902624in}{0.978266in}}{\pgfqpoint{0.892173in}{0.982595in}}{\pgfqpoint{0.881277in}{0.982595in}}%
\pgfpathcurveto{\pgfqpoint{0.870382in}{0.982595in}}{\pgfqpoint{0.859931in}{0.978266in}}{\pgfqpoint{0.852226in}{0.970562in}}%
\pgfpathcurveto{\pgfqpoint{0.844522in}{0.962858in}}{\pgfqpoint{0.840193in}{0.952407in}}{\pgfqpoint{0.840193in}{0.941511in}}%
\pgfpathcurveto{\pgfqpoint{0.840193in}{0.930616in}}{\pgfqpoint{0.844522in}{0.920165in}}{\pgfqpoint{0.852226in}{0.912461in}}%
\pgfpathcurveto{\pgfqpoint{0.859931in}{0.904756in}}{\pgfqpoint{0.870382in}{0.900428in}}{\pgfqpoint{0.881277in}{0.900428in}}%
\pgfpathlineto{\pgfqpoint{0.881277in}{0.900428in}}%
\pgfpathclose%
\pgfusepath{stroke}%
\end{pgfscope}%
\begin{pgfscope}%
\pgfpathrectangle{\pgfqpoint{0.688192in}{0.670138in}}{\pgfqpoint{7.111808in}{5.129862in}}%
\pgfusepath{clip}%
\pgfsetbuttcap%
\pgfsetroundjoin%
\pgfsetlinewidth{1.003750pt}%
\definecolor{currentstroke}{rgb}{0.000000,0.000000,0.000000}%
\pgfsetstrokecolor{currentstroke}%
\pgfsetdash{}{0pt}%
\pgfpathmoveto{\pgfqpoint{1.628853in}{0.704118in}}%
\pgfpathcurveto{\pgfqpoint{1.639749in}{0.704118in}}{\pgfqpoint{1.650199in}{0.708447in}}{\pgfqpoint{1.657904in}{0.716151in}}%
\pgfpathcurveto{\pgfqpoint{1.665608in}{0.723856in}}{\pgfqpoint{1.669937in}{0.734306in}}{\pgfqpoint{1.669937in}{0.745202in}}%
\pgfpathcurveto{\pgfqpoint{1.669937in}{0.756098in}}{\pgfqpoint{1.665608in}{0.766548in}}{\pgfqpoint{1.657904in}{0.774253in}}%
\pgfpathcurveto{\pgfqpoint{1.650199in}{0.781957in}}{\pgfqpoint{1.639749in}{0.786286in}}{\pgfqpoint{1.628853in}{0.786286in}}%
\pgfpathcurveto{\pgfqpoint{1.617958in}{0.786286in}}{\pgfqpoint{1.607507in}{0.781957in}}{\pgfqpoint{1.599802in}{0.774253in}}%
\pgfpathcurveto{\pgfqpoint{1.592098in}{0.766548in}}{\pgfqpoint{1.587769in}{0.756098in}}{\pgfqpoint{1.587769in}{0.745202in}}%
\pgfpathcurveto{\pgfqpoint{1.587769in}{0.734306in}}{\pgfqpoint{1.592098in}{0.723856in}}{\pgfqpoint{1.599802in}{0.716151in}}%
\pgfpathcurveto{\pgfqpoint{1.607507in}{0.708447in}}{\pgfqpoint{1.617958in}{0.704118in}}{\pgfqpoint{1.628853in}{0.704118in}}%
\pgfpathlineto{\pgfqpoint{1.628853in}{0.704118in}}%
\pgfpathclose%
\pgfusepath{stroke}%
\end{pgfscope}%
\begin{pgfscope}%
\pgfpathrectangle{\pgfqpoint{0.688192in}{0.670138in}}{\pgfqpoint{7.111808in}{5.129862in}}%
\pgfusepath{clip}%
\pgfsetbuttcap%
\pgfsetroundjoin%
\pgfsetlinewidth{1.003750pt}%
\definecolor{currentstroke}{rgb}{0.000000,0.000000,0.000000}%
\pgfsetstrokecolor{currentstroke}%
\pgfsetdash{}{0pt}%
\pgfpathmoveto{\pgfqpoint{4.603680in}{0.639749in}}%
\pgfpathcurveto{\pgfqpoint{4.614576in}{0.639749in}}{\pgfqpoint{4.625026in}{0.644077in}}{\pgfqpoint{4.632731in}{0.651782in}}%
\pgfpathcurveto{\pgfqpoint{4.640435in}{0.659486in}}{\pgfqpoint{4.644764in}{0.669937in}}{\pgfqpoint{4.644764in}{0.680832in}}%
\pgfpathcurveto{\pgfqpoint{4.644764in}{0.691728in}}{\pgfqpoint{4.640435in}{0.702179in}}{\pgfqpoint{4.632731in}{0.709883in}}%
\pgfpathcurveto{\pgfqpoint{4.625026in}{0.717587in}}{\pgfqpoint{4.614576in}{0.721916in}}{\pgfqpoint{4.603680in}{0.721916in}}%
\pgfpathcurveto{\pgfqpoint{4.592784in}{0.721916in}}{\pgfqpoint{4.582334in}{0.717587in}}{\pgfqpoint{4.574629in}{0.709883in}}%
\pgfpathcurveto{\pgfqpoint{4.566925in}{0.702179in}}{\pgfqpoint{4.562596in}{0.691728in}}{\pgfqpoint{4.562596in}{0.680832in}}%
\pgfpathcurveto{\pgfqpoint{4.562596in}{0.669937in}}{\pgfqpoint{4.566925in}{0.659486in}}{\pgfqpoint{4.574629in}{0.651782in}}%
\pgfpathcurveto{\pgfqpoint{4.582334in}{0.644077in}}{\pgfqpoint{4.592784in}{0.639749in}}{\pgfqpoint{4.603680in}{0.639749in}}%
\pgfusepath{stroke}%
\end{pgfscope}%
\begin{pgfscope}%
\pgfpathrectangle{\pgfqpoint{0.688192in}{0.670138in}}{\pgfqpoint{7.111808in}{5.129862in}}%
\pgfusepath{clip}%
\pgfsetbuttcap%
\pgfsetroundjoin%
\pgfsetlinewidth{1.003750pt}%
\definecolor{currentstroke}{rgb}{0.000000,0.000000,0.000000}%
\pgfsetstrokecolor{currentstroke}%
\pgfsetdash{}{0pt}%
\pgfpathmoveto{\pgfqpoint{6.581792in}{5.491980in}}%
\pgfpathcurveto{\pgfqpoint{6.592688in}{5.491980in}}{\pgfqpoint{6.603139in}{5.496309in}}{\pgfqpoint{6.610843in}{5.504013in}}%
\pgfpathcurveto{\pgfqpoint{6.618547in}{5.511718in}}{\pgfqpoint{6.622876in}{5.522168in}}{\pgfqpoint{6.622876in}{5.533064in}}%
\pgfpathcurveto{\pgfqpoint{6.622876in}{5.543959in}}{\pgfqpoint{6.618547in}{5.554410in}}{\pgfqpoint{6.610843in}{5.562115in}}%
\pgfpathcurveto{\pgfqpoint{6.603139in}{5.569819in}}{\pgfqpoint{6.592688in}{5.574148in}}{\pgfqpoint{6.581792in}{5.574148in}}%
\pgfpathcurveto{\pgfqpoint{6.570897in}{5.574148in}}{\pgfqpoint{6.560446in}{5.569819in}}{\pgfqpoint{6.552741in}{5.562115in}}%
\pgfpathcurveto{\pgfqpoint{6.545037in}{5.554410in}}{\pgfqpoint{6.540708in}{5.543959in}}{\pgfqpoint{6.540708in}{5.533064in}}%
\pgfpathcurveto{\pgfqpoint{6.540708in}{5.522168in}}{\pgfqpoint{6.545037in}{5.511718in}}{\pgfqpoint{6.552741in}{5.504013in}}%
\pgfpathcurveto{\pgfqpoint{6.560446in}{5.496309in}}{\pgfqpoint{6.570897in}{5.491980in}}{\pgfqpoint{6.581792in}{5.491980in}}%
\pgfpathlineto{\pgfqpoint{6.581792in}{5.491980in}}%
\pgfpathclose%
\pgfusepath{stroke}%
\end{pgfscope}%
\begin{pgfscope}%
\pgfpathrectangle{\pgfqpoint{0.688192in}{0.670138in}}{\pgfqpoint{7.111808in}{5.129862in}}%
\pgfusepath{clip}%
\pgfsetbuttcap%
\pgfsetroundjoin%
\pgfsetlinewidth{1.003750pt}%
\definecolor{currentstroke}{rgb}{0.000000,0.000000,0.000000}%
\pgfsetstrokecolor{currentstroke}%
\pgfsetdash{}{0pt}%
\pgfpathmoveto{\pgfqpoint{1.448644in}{0.707802in}}%
\pgfpathcurveto{\pgfqpoint{1.459539in}{0.707802in}}{\pgfqpoint{1.469990in}{0.712131in}}{\pgfqpoint{1.477694in}{0.719835in}}%
\pgfpathcurveto{\pgfqpoint{1.485399in}{0.727539in}}{\pgfqpoint{1.489728in}{0.737990in}}{\pgfqpoint{1.489728in}{0.748886in}}%
\pgfpathcurveto{\pgfqpoint{1.489728in}{0.759781in}}{\pgfqpoint{1.485399in}{0.770232in}}{\pgfqpoint{1.477694in}{0.777936in}}%
\pgfpathcurveto{\pgfqpoint{1.469990in}{0.785641in}}{\pgfqpoint{1.459539in}{0.789970in}}{\pgfqpoint{1.448644in}{0.789970in}}%
\pgfpathcurveto{\pgfqpoint{1.437748in}{0.789970in}}{\pgfqpoint{1.427297in}{0.785641in}}{\pgfqpoint{1.419593in}{0.777936in}}%
\pgfpathcurveto{\pgfqpoint{1.411889in}{0.770232in}}{\pgfqpoint{1.407560in}{0.759781in}}{\pgfqpoint{1.407560in}{0.748886in}}%
\pgfpathcurveto{\pgfqpoint{1.407560in}{0.737990in}}{\pgfqpoint{1.411889in}{0.727539in}}{\pgfqpoint{1.419593in}{0.719835in}}%
\pgfpathcurveto{\pgfqpoint{1.427297in}{0.712131in}}{\pgfqpoint{1.437748in}{0.707802in}}{\pgfqpoint{1.448644in}{0.707802in}}%
\pgfpathlineto{\pgfqpoint{1.448644in}{0.707802in}}%
\pgfpathclose%
\pgfusepath{stroke}%
\end{pgfscope}%
\begin{pgfscope}%
\pgfpathrectangle{\pgfqpoint{0.688192in}{0.670138in}}{\pgfqpoint{7.111808in}{5.129862in}}%
\pgfusepath{clip}%
\pgfsetbuttcap%
\pgfsetroundjoin%
\pgfsetlinewidth{1.003750pt}%
\definecolor{currentstroke}{rgb}{0.000000,0.000000,0.000000}%
\pgfsetstrokecolor{currentstroke}%
\pgfsetdash{}{0pt}%
\pgfpathmoveto{\pgfqpoint{4.634758in}{0.639654in}}%
\pgfpathcurveto{\pgfqpoint{4.645653in}{0.639654in}}{\pgfqpoint{4.656104in}{0.643983in}}{\pgfqpoint{4.663808in}{0.651687in}}%
\pgfpathcurveto{\pgfqpoint{4.671513in}{0.659391in}}{\pgfqpoint{4.675841in}{0.669842in}}{\pgfqpoint{4.675841in}{0.680738in}}%
\pgfpathcurveto{\pgfqpoint{4.675841in}{0.691633in}}{\pgfqpoint{4.671513in}{0.702084in}}{\pgfqpoint{4.663808in}{0.709789in}}%
\pgfpathcurveto{\pgfqpoint{4.656104in}{0.717493in}}{\pgfqpoint{4.645653in}{0.721822in}}{\pgfqpoint{4.634758in}{0.721822in}}%
\pgfpathcurveto{\pgfqpoint{4.623862in}{0.721822in}}{\pgfqpoint{4.613411in}{0.717493in}}{\pgfqpoint{4.605707in}{0.709789in}}%
\pgfpathcurveto{\pgfqpoint{4.598003in}{0.702084in}}{\pgfqpoint{4.593674in}{0.691633in}}{\pgfqpoint{4.593674in}{0.680738in}}%
\pgfpathcurveto{\pgfqpoint{4.593674in}{0.669842in}}{\pgfqpoint{4.598003in}{0.659391in}}{\pgfqpoint{4.605707in}{0.651687in}}%
\pgfpathcurveto{\pgfqpoint{4.613411in}{0.643983in}}{\pgfqpoint{4.623862in}{0.639654in}}{\pgfqpoint{4.634758in}{0.639654in}}%
\pgfusepath{stroke}%
\end{pgfscope}%
\begin{pgfscope}%
\pgfpathrectangle{\pgfqpoint{0.688192in}{0.670138in}}{\pgfqpoint{7.111808in}{5.129862in}}%
\pgfusepath{clip}%
\pgfsetbuttcap%
\pgfsetroundjoin%
\pgfsetlinewidth{1.003750pt}%
\definecolor{currentstroke}{rgb}{0.000000,0.000000,0.000000}%
\pgfsetstrokecolor{currentstroke}%
\pgfsetdash{}{0pt}%
\pgfpathmoveto{\pgfqpoint{4.549016in}{0.640923in}}%
\pgfpathcurveto{\pgfqpoint{4.559912in}{0.640923in}}{\pgfqpoint{4.570362in}{0.645252in}}{\pgfqpoint{4.578067in}{0.652956in}}%
\pgfpathcurveto{\pgfqpoint{4.585771in}{0.660660in}}{\pgfqpoint{4.590100in}{0.671111in}}{\pgfqpoint{4.590100in}{0.682007in}}%
\pgfpathcurveto{\pgfqpoint{4.590100in}{0.692902in}}{\pgfqpoint{4.585771in}{0.703353in}}{\pgfqpoint{4.578067in}{0.711057in}}%
\pgfpathcurveto{\pgfqpoint{4.570362in}{0.718762in}}{\pgfqpoint{4.559912in}{0.723090in}}{\pgfqpoint{4.549016in}{0.723090in}}%
\pgfpathcurveto{\pgfqpoint{4.538120in}{0.723090in}}{\pgfqpoint{4.527670in}{0.718762in}}{\pgfqpoint{4.519965in}{0.711057in}}%
\pgfpathcurveto{\pgfqpoint{4.512261in}{0.703353in}}{\pgfqpoint{4.507932in}{0.692902in}}{\pgfqpoint{4.507932in}{0.682007in}}%
\pgfpathcurveto{\pgfqpoint{4.507932in}{0.671111in}}{\pgfqpoint{4.512261in}{0.660660in}}{\pgfqpoint{4.519965in}{0.652956in}}%
\pgfpathcurveto{\pgfqpoint{4.527670in}{0.645252in}}{\pgfqpoint{4.538120in}{0.640923in}}{\pgfqpoint{4.549016in}{0.640923in}}%
\pgfusepath{stroke}%
\end{pgfscope}%
\begin{pgfscope}%
\pgfpathrectangle{\pgfqpoint{0.688192in}{0.670138in}}{\pgfqpoint{7.111808in}{5.129862in}}%
\pgfusepath{clip}%
\pgfsetbuttcap%
\pgfsetroundjoin%
\pgfsetlinewidth{1.003750pt}%
\definecolor{currentstroke}{rgb}{0.000000,0.000000,0.000000}%
\pgfsetstrokecolor{currentstroke}%
\pgfsetdash{}{0pt}%
\pgfpathmoveto{\pgfqpoint{1.026002in}{0.758093in}}%
\pgfpathcurveto{\pgfqpoint{1.036898in}{0.758093in}}{\pgfqpoint{1.047348in}{0.762422in}}{\pgfqpoint{1.055053in}{0.770127in}}%
\pgfpathcurveto{\pgfqpoint{1.062757in}{0.777831in}}{\pgfqpoint{1.067086in}{0.788282in}}{\pgfqpoint{1.067086in}{0.799177in}}%
\pgfpathcurveto{\pgfqpoint{1.067086in}{0.810073in}}{\pgfqpoint{1.062757in}{0.820524in}}{\pgfqpoint{1.055053in}{0.828228in}}%
\pgfpathcurveto{\pgfqpoint{1.047348in}{0.835932in}}{\pgfqpoint{1.036898in}{0.840261in}}{\pgfqpoint{1.026002in}{0.840261in}}%
\pgfpathcurveto{\pgfqpoint{1.015106in}{0.840261in}}{\pgfqpoint{1.004656in}{0.835932in}}{\pgfqpoint{0.996951in}{0.828228in}}%
\pgfpathcurveto{\pgfqpoint{0.989247in}{0.820524in}}{\pgfqpoint{0.984918in}{0.810073in}}{\pgfqpoint{0.984918in}{0.799177in}}%
\pgfpathcurveto{\pgfqpoint{0.984918in}{0.788282in}}{\pgfqpoint{0.989247in}{0.777831in}}{\pgfqpoint{0.996951in}{0.770127in}}%
\pgfpathcurveto{\pgfqpoint{1.004656in}{0.762422in}}{\pgfqpoint{1.015106in}{0.758093in}}{\pgfqpoint{1.026002in}{0.758093in}}%
\pgfpathlineto{\pgfqpoint{1.026002in}{0.758093in}}%
\pgfpathclose%
\pgfusepath{stroke}%
\end{pgfscope}%
\begin{pgfscope}%
\pgfpathrectangle{\pgfqpoint{0.688192in}{0.670138in}}{\pgfqpoint{7.111808in}{5.129862in}}%
\pgfusepath{clip}%
\pgfsetbuttcap%
\pgfsetroundjoin%
\pgfsetlinewidth{1.003750pt}%
\definecolor{currentstroke}{rgb}{0.000000,0.000000,0.000000}%
\pgfsetstrokecolor{currentstroke}%
\pgfsetdash{}{0pt}%
\pgfpathmoveto{\pgfqpoint{2.196830in}{0.685120in}}%
\pgfpathcurveto{\pgfqpoint{2.207726in}{0.685120in}}{\pgfqpoint{2.218177in}{0.689449in}}{\pgfqpoint{2.225881in}{0.697153in}}%
\pgfpathcurveto{\pgfqpoint{2.233585in}{0.704857in}}{\pgfqpoint{2.237914in}{0.715308in}}{\pgfqpoint{2.237914in}{0.726204in}}%
\pgfpathcurveto{\pgfqpoint{2.237914in}{0.737099in}}{\pgfqpoint{2.233585in}{0.747550in}}{\pgfqpoint{2.225881in}{0.755254in}}%
\pgfpathcurveto{\pgfqpoint{2.218177in}{0.762959in}}{\pgfqpoint{2.207726in}{0.767287in}}{\pgfqpoint{2.196830in}{0.767287in}}%
\pgfpathcurveto{\pgfqpoint{2.185935in}{0.767287in}}{\pgfqpoint{2.175484in}{0.762959in}}{\pgfqpoint{2.167780in}{0.755254in}}%
\pgfpathcurveto{\pgfqpoint{2.160075in}{0.747550in}}{\pgfqpoint{2.155746in}{0.737099in}}{\pgfqpoint{2.155746in}{0.726204in}}%
\pgfpathcurveto{\pgfqpoint{2.155746in}{0.715308in}}{\pgfqpoint{2.160075in}{0.704857in}}{\pgfqpoint{2.167780in}{0.697153in}}%
\pgfpathcurveto{\pgfqpoint{2.175484in}{0.689449in}}{\pgfqpoint{2.185935in}{0.685120in}}{\pgfqpoint{2.196830in}{0.685120in}}%
\pgfpathlineto{\pgfqpoint{2.196830in}{0.685120in}}%
\pgfpathclose%
\pgfusepath{stroke}%
\end{pgfscope}%
\begin{pgfscope}%
\pgfpathrectangle{\pgfqpoint{0.688192in}{0.670138in}}{\pgfqpoint{7.111808in}{5.129862in}}%
\pgfusepath{clip}%
\pgfsetbuttcap%
\pgfsetroundjoin%
\pgfsetlinewidth{1.003750pt}%
\definecolor{currentstroke}{rgb}{0.000000,0.000000,0.000000}%
\pgfsetstrokecolor{currentstroke}%
\pgfsetdash{}{0pt}%
\pgfpathmoveto{\pgfqpoint{1.713082in}{0.698892in}}%
\pgfpathcurveto{\pgfqpoint{1.723978in}{0.698892in}}{\pgfqpoint{1.734428in}{0.703221in}}{\pgfqpoint{1.742133in}{0.710926in}}%
\pgfpathcurveto{\pgfqpoint{1.749837in}{0.718630in}}{\pgfqpoint{1.754166in}{0.729081in}}{\pgfqpoint{1.754166in}{0.739976in}}%
\pgfpathcurveto{\pgfqpoint{1.754166in}{0.750872in}}{\pgfqpoint{1.749837in}{0.761323in}}{\pgfqpoint{1.742133in}{0.769027in}}%
\pgfpathcurveto{\pgfqpoint{1.734428in}{0.776731in}}{\pgfqpoint{1.723978in}{0.781060in}}{\pgfqpoint{1.713082in}{0.781060in}}%
\pgfpathcurveto{\pgfqpoint{1.702186in}{0.781060in}}{\pgfqpoint{1.691736in}{0.776731in}}{\pgfqpoint{1.684031in}{0.769027in}}%
\pgfpathcurveto{\pgfqpoint{1.676327in}{0.761323in}}{\pgfqpoint{1.671998in}{0.750872in}}{\pgfqpoint{1.671998in}{0.739976in}}%
\pgfpathcurveto{\pgfqpoint{1.671998in}{0.729081in}}{\pgfqpoint{1.676327in}{0.718630in}}{\pgfqpoint{1.684031in}{0.710926in}}%
\pgfpathcurveto{\pgfqpoint{1.691736in}{0.703221in}}{\pgfqpoint{1.702186in}{0.698892in}}{\pgfqpoint{1.713082in}{0.698892in}}%
\pgfpathlineto{\pgfqpoint{1.713082in}{0.698892in}}%
\pgfpathclose%
\pgfusepath{stroke}%
\end{pgfscope}%
\begin{pgfscope}%
\pgfpathrectangle{\pgfqpoint{0.688192in}{0.670138in}}{\pgfqpoint{7.111808in}{5.129862in}}%
\pgfusepath{clip}%
\pgfsetbuttcap%
\pgfsetroundjoin%
\pgfsetlinewidth{1.003750pt}%
\definecolor{currentstroke}{rgb}{0.000000,0.000000,0.000000}%
\pgfsetstrokecolor{currentstroke}%
\pgfsetdash{}{0pt}%
\pgfpathmoveto{\pgfqpoint{1.775222in}{0.697735in}}%
\pgfpathcurveto{\pgfqpoint{1.786118in}{0.697735in}}{\pgfqpoint{1.796569in}{0.702064in}}{\pgfqpoint{1.804273in}{0.709768in}}%
\pgfpathcurveto{\pgfqpoint{1.811977in}{0.717473in}}{\pgfqpoint{1.816306in}{0.727923in}}{\pgfqpoint{1.816306in}{0.738819in}}%
\pgfpathcurveto{\pgfqpoint{1.816306in}{0.749715in}}{\pgfqpoint{1.811977in}{0.760165in}}{\pgfqpoint{1.804273in}{0.767870in}}%
\pgfpathcurveto{\pgfqpoint{1.796569in}{0.775574in}}{\pgfqpoint{1.786118in}{0.779903in}}{\pgfqpoint{1.775222in}{0.779903in}}%
\pgfpathcurveto{\pgfqpoint{1.764327in}{0.779903in}}{\pgfqpoint{1.753876in}{0.775574in}}{\pgfqpoint{1.746172in}{0.767870in}}%
\pgfpathcurveto{\pgfqpoint{1.738467in}{0.760165in}}{\pgfqpoint{1.734139in}{0.749715in}}{\pgfqpoint{1.734139in}{0.738819in}}%
\pgfpathcurveto{\pgfqpoint{1.734139in}{0.727923in}}{\pgfqpoint{1.738467in}{0.717473in}}{\pgfqpoint{1.746172in}{0.709768in}}%
\pgfpathcurveto{\pgfqpoint{1.753876in}{0.702064in}}{\pgfqpoint{1.764327in}{0.697735in}}{\pgfqpoint{1.775222in}{0.697735in}}%
\pgfpathlineto{\pgfqpoint{1.775222in}{0.697735in}}%
\pgfpathclose%
\pgfusepath{stroke}%
\end{pgfscope}%
\begin{pgfscope}%
\pgfpathrectangle{\pgfqpoint{0.688192in}{0.670138in}}{\pgfqpoint{7.111808in}{5.129862in}}%
\pgfusepath{clip}%
\pgfsetbuttcap%
\pgfsetroundjoin%
\pgfsetlinewidth{1.003750pt}%
\definecolor{currentstroke}{rgb}{0.000000,0.000000,0.000000}%
\pgfsetstrokecolor{currentstroke}%
\pgfsetdash{}{0pt}%
\pgfpathmoveto{\pgfqpoint{5.261268in}{0.632886in}}%
\pgfpathcurveto{\pgfqpoint{5.272163in}{0.632886in}}{\pgfqpoint{5.282614in}{0.637215in}}{\pgfqpoint{5.290319in}{0.644920in}}%
\pgfpathcurveto{\pgfqpoint{5.298023in}{0.652624in}}{\pgfqpoint{5.302352in}{0.663075in}}{\pgfqpoint{5.302352in}{0.673970in}}%
\pgfpathcurveto{\pgfqpoint{5.302352in}{0.684866in}}{\pgfqpoint{5.298023in}{0.695317in}}{\pgfqpoint{5.290319in}{0.703021in}}%
\pgfpathcurveto{\pgfqpoint{5.282614in}{0.710725in}}{\pgfqpoint{5.272163in}{0.715054in}}{\pgfqpoint{5.261268in}{0.715054in}}%
\pgfpathcurveto{\pgfqpoint{5.250372in}{0.715054in}}{\pgfqpoint{5.239921in}{0.710725in}}{\pgfqpoint{5.232217in}{0.703021in}}%
\pgfpathcurveto{\pgfqpoint{5.224513in}{0.695317in}}{\pgfqpoint{5.220184in}{0.684866in}}{\pgfqpoint{5.220184in}{0.673970in}}%
\pgfpathcurveto{\pgfqpoint{5.220184in}{0.663075in}}{\pgfqpoint{5.224513in}{0.652624in}}{\pgfqpoint{5.232217in}{0.644920in}}%
\pgfpathcurveto{\pgfqpoint{5.239921in}{0.637215in}}{\pgfqpoint{5.250372in}{0.632886in}}{\pgfqpoint{5.261268in}{0.632886in}}%
\pgfusepath{stroke}%
\end{pgfscope}%
\begin{pgfscope}%
\pgfpathrectangle{\pgfqpoint{0.688192in}{0.670138in}}{\pgfqpoint{7.111808in}{5.129862in}}%
\pgfusepath{clip}%
\pgfsetbuttcap%
\pgfsetroundjoin%
\pgfsetlinewidth{1.003750pt}%
\definecolor{currentstroke}{rgb}{0.000000,0.000000,0.000000}%
\pgfsetstrokecolor{currentstroke}%
\pgfsetdash{}{0pt}%
\pgfpathmoveto{\pgfqpoint{1.679564in}{0.702362in}}%
\pgfpathcurveto{\pgfqpoint{1.690460in}{0.702362in}}{\pgfqpoint{1.700911in}{0.706691in}}{\pgfqpoint{1.708615in}{0.714395in}}%
\pgfpathcurveto{\pgfqpoint{1.716319in}{0.722099in}}{\pgfqpoint{1.720648in}{0.732550in}}{\pgfqpoint{1.720648in}{0.743446in}}%
\pgfpathcurveto{\pgfqpoint{1.720648in}{0.754341in}}{\pgfqpoint{1.716319in}{0.764792in}}{\pgfqpoint{1.708615in}{0.772496in}}%
\pgfpathcurveto{\pgfqpoint{1.700911in}{0.780201in}}{\pgfqpoint{1.690460in}{0.784529in}}{\pgfqpoint{1.679564in}{0.784529in}}%
\pgfpathcurveto{\pgfqpoint{1.668669in}{0.784529in}}{\pgfqpoint{1.658218in}{0.780201in}}{\pgfqpoint{1.650514in}{0.772496in}}%
\pgfpathcurveto{\pgfqpoint{1.642809in}{0.764792in}}{\pgfqpoint{1.638480in}{0.754341in}}{\pgfqpoint{1.638480in}{0.743446in}}%
\pgfpathcurveto{\pgfqpoint{1.638480in}{0.732550in}}{\pgfqpoint{1.642809in}{0.722099in}}{\pgfqpoint{1.650514in}{0.714395in}}%
\pgfpathcurveto{\pgfqpoint{1.658218in}{0.706691in}}{\pgfqpoint{1.668669in}{0.702362in}}{\pgfqpoint{1.679564in}{0.702362in}}%
\pgfpathlineto{\pgfqpoint{1.679564in}{0.702362in}}%
\pgfpathclose%
\pgfusepath{stroke}%
\end{pgfscope}%
\begin{pgfscope}%
\pgfpathrectangle{\pgfqpoint{0.688192in}{0.670138in}}{\pgfqpoint{7.111808in}{5.129862in}}%
\pgfusepath{clip}%
\pgfsetbuttcap%
\pgfsetroundjoin%
\pgfsetlinewidth{1.003750pt}%
\definecolor{currentstroke}{rgb}{0.000000,0.000000,0.000000}%
\pgfsetstrokecolor{currentstroke}%
\pgfsetdash{}{0pt}%
\pgfpathmoveto{\pgfqpoint{1.400715in}{0.710979in}}%
\pgfpathcurveto{\pgfqpoint{1.411611in}{0.710979in}}{\pgfqpoint{1.422062in}{0.715308in}}{\pgfqpoint{1.429766in}{0.723012in}}%
\pgfpathcurveto{\pgfqpoint{1.437470in}{0.730717in}}{\pgfqpoint{1.441799in}{0.741167in}}{\pgfqpoint{1.441799in}{0.752063in}}%
\pgfpathcurveto{\pgfqpoint{1.441799in}{0.762959in}}{\pgfqpoint{1.437470in}{0.773409in}}{\pgfqpoint{1.429766in}{0.781114in}}%
\pgfpathcurveto{\pgfqpoint{1.422062in}{0.788818in}}{\pgfqpoint{1.411611in}{0.793147in}}{\pgfqpoint{1.400715in}{0.793147in}}%
\pgfpathcurveto{\pgfqpoint{1.389820in}{0.793147in}}{\pgfqpoint{1.379369in}{0.788818in}}{\pgfqpoint{1.371665in}{0.781114in}}%
\pgfpathcurveto{\pgfqpoint{1.363960in}{0.773409in}}{\pgfqpoint{1.359632in}{0.762959in}}{\pgfqpoint{1.359632in}{0.752063in}}%
\pgfpathcurveto{\pgfqpoint{1.359632in}{0.741167in}}{\pgfqpoint{1.363960in}{0.730717in}}{\pgfqpoint{1.371665in}{0.723012in}}%
\pgfpathcurveto{\pgfqpoint{1.379369in}{0.715308in}}{\pgfqpoint{1.389820in}{0.710979in}}{\pgfqpoint{1.400715in}{0.710979in}}%
\pgfpathlineto{\pgfqpoint{1.400715in}{0.710979in}}%
\pgfpathclose%
\pgfusepath{stroke}%
\end{pgfscope}%
\begin{pgfscope}%
\pgfpathrectangle{\pgfqpoint{0.688192in}{0.670138in}}{\pgfqpoint{7.111808in}{5.129862in}}%
\pgfusepath{clip}%
\pgfsetbuttcap%
\pgfsetroundjoin%
\pgfsetlinewidth{1.003750pt}%
\definecolor{currentstroke}{rgb}{0.000000,0.000000,0.000000}%
\pgfsetstrokecolor{currentstroke}%
\pgfsetdash{}{0pt}%
\pgfpathmoveto{\pgfqpoint{1.389197in}{0.711266in}}%
\pgfpathcurveto{\pgfqpoint{1.400093in}{0.711266in}}{\pgfqpoint{1.410544in}{0.715595in}}{\pgfqpoint{1.418248in}{0.723299in}}%
\pgfpathcurveto{\pgfqpoint{1.425952in}{0.731004in}}{\pgfqpoint{1.430281in}{0.741454in}}{\pgfqpoint{1.430281in}{0.752350in}}%
\pgfpathcurveto{\pgfqpoint{1.430281in}{0.763246in}}{\pgfqpoint{1.425952in}{0.773696in}}{\pgfqpoint{1.418248in}{0.781401in}}%
\pgfpathcurveto{\pgfqpoint{1.410544in}{0.789105in}}{\pgfqpoint{1.400093in}{0.793434in}}{\pgfqpoint{1.389197in}{0.793434in}}%
\pgfpathcurveto{\pgfqpoint{1.378302in}{0.793434in}}{\pgfqpoint{1.367851in}{0.789105in}}{\pgfqpoint{1.360147in}{0.781401in}}%
\pgfpathcurveto{\pgfqpoint{1.352442in}{0.773696in}}{\pgfqpoint{1.348113in}{0.763246in}}{\pgfqpoint{1.348113in}{0.752350in}}%
\pgfpathcurveto{\pgfqpoint{1.348113in}{0.741454in}}{\pgfqpoint{1.352442in}{0.731004in}}{\pgfqpoint{1.360147in}{0.723299in}}%
\pgfpathcurveto{\pgfqpoint{1.367851in}{0.715595in}}{\pgfqpoint{1.378302in}{0.711266in}}{\pgfqpoint{1.389197in}{0.711266in}}%
\pgfpathlineto{\pgfqpoint{1.389197in}{0.711266in}}%
\pgfpathclose%
\pgfusepath{stroke}%
\end{pgfscope}%
\begin{pgfscope}%
\pgfpathrectangle{\pgfqpoint{0.688192in}{0.670138in}}{\pgfqpoint{7.111808in}{5.129862in}}%
\pgfusepath{clip}%
\pgfsetbuttcap%
\pgfsetroundjoin%
\pgfsetlinewidth{1.003750pt}%
\definecolor{currentstroke}{rgb}{0.000000,0.000000,0.000000}%
\pgfsetstrokecolor{currentstroke}%
\pgfsetdash{}{0pt}%
\pgfpathmoveto{\pgfqpoint{2.869035in}{0.669385in}}%
\pgfpathcurveto{\pgfqpoint{2.879931in}{0.669385in}}{\pgfqpoint{2.890382in}{0.673714in}}{\pgfqpoint{2.898086in}{0.681419in}}%
\pgfpathcurveto{\pgfqpoint{2.905790in}{0.689123in}}{\pgfqpoint{2.910119in}{0.699574in}}{\pgfqpoint{2.910119in}{0.710469in}}%
\pgfpathcurveto{\pgfqpoint{2.910119in}{0.721365in}}{\pgfqpoint{2.905790in}{0.731816in}}{\pgfqpoint{2.898086in}{0.739520in}}%
\pgfpathcurveto{\pgfqpoint{2.890382in}{0.747224in}}{\pgfqpoint{2.879931in}{0.751553in}}{\pgfqpoint{2.869035in}{0.751553in}}%
\pgfpathcurveto{\pgfqpoint{2.858140in}{0.751553in}}{\pgfqpoint{2.847689in}{0.747224in}}{\pgfqpoint{2.839985in}{0.739520in}}%
\pgfpathcurveto{\pgfqpoint{2.832280in}{0.731816in}}{\pgfqpoint{2.827952in}{0.721365in}}{\pgfqpoint{2.827952in}{0.710469in}}%
\pgfpathcurveto{\pgfqpoint{2.827952in}{0.699574in}}{\pgfqpoint{2.832280in}{0.689123in}}{\pgfqpoint{2.839985in}{0.681419in}}%
\pgfpathcurveto{\pgfqpoint{2.847689in}{0.673714in}}{\pgfqpoint{2.858140in}{0.669385in}}{\pgfqpoint{2.869035in}{0.669385in}}%
\pgfpathlineto{\pgfqpoint{2.869035in}{0.669385in}}%
\pgfpathclose%
\pgfusepath{stroke}%
\end{pgfscope}%
\begin{pgfscope}%
\pgfpathrectangle{\pgfqpoint{0.688192in}{0.670138in}}{\pgfqpoint{7.111808in}{5.129862in}}%
\pgfusepath{clip}%
\pgfsetbuttcap%
\pgfsetroundjoin%
\pgfsetlinewidth{1.003750pt}%
\definecolor{currentstroke}{rgb}{0.000000,0.000000,0.000000}%
\pgfsetstrokecolor{currentstroke}%
\pgfsetdash{}{0pt}%
\pgfpathmoveto{\pgfqpoint{3.953911in}{0.644492in}}%
\pgfpathcurveto{\pgfqpoint{3.964806in}{0.644492in}}{\pgfqpoint{3.975257in}{0.648821in}}{\pgfqpoint{3.982961in}{0.656525in}}%
\pgfpathcurveto{\pgfqpoint{3.990666in}{0.664229in}}{\pgfqpoint{3.994994in}{0.674680in}}{\pgfqpoint{3.994994in}{0.685576in}}%
\pgfpathcurveto{\pgfqpoint{3.994994in}{0.696471in}}{\pgfqpoint{3.990666in}{0.706922in}}{\pgfqpoint{3.982961in}{0.714626in}}%
\pgfpathcurveto{\pgfqpoint{3.975257in}{0.722331in}}{\pgfqpoint{3.964806in}{0.726659in}}{\pgfqpoint{3.953911in}{0.726659in}}%
\pgfpathcurveto{\pgfqpoint{3.943015in}{0.726659in}}{\pgfqpoint{3.932564in}{0.722331in}}{\pgfqpoint{3.924860in}{0.714626in}}%
\pgfpathcurveto{\pgfqpoint{3.917156in}{0.706922in}}{\pgfqpoint{3.912827in}{0.696471in}}{\pgfqpoint{3.912827in}{0.685576in}}%
\pgfpathcurveto{\pgfqpoint{3.912827in}{0.674680in}}{\pgfqpoint{3.917156in}{0.664229in}}{\pgfqpoint{3.924860in}{0.656525in}}%
\pgfpathcurveto{\pgfqpoint{3.932564in}{0.648821in}}{\pgfqpoint{3.943015in}{0.644492in}}{\pgfqpoint{3.953911in}{0.644492in}}%
\pgfusepath{stroke}%
\end{pgfscope}%
\begin{pgfscope}%
\pgfpathrectangle{\pgfqpoint{0.688192in}{0.670138in}}{\pgfqpoint{7.111808in}{5.129862in}}%
\pgfusepath{clip}%
\pgfsetbuttcap%
\pgfsetroundjoin%
\pgfsetlinewidth{1.003750pt}%
\definecolor{currentstroke}{rgb}{0.000000,0.000000,0.000000}%
\pgfsetstrokecolor{currentstroke}%
\pgfsetdash{}{0pt}%
\pgfpathmoveto{\pgfqpoint{1.979583in}{4.073012in}}%
\pgfpathcurveto{\pgfqpoint{1.990479in}{4.073012in}}{\pgfqpoint{2.000930in}{4.077341in}}{\pgfqpoint{2.008634in}{4.085046in}}%
\pgfpathcurveto{\pgfqpoint{2.016338in}{4.092750in}}{\pgfqpoint{2.020667in}{4.103201in}}{\pgfqpoint{2.020667in}{4.114096in}}%
\pgfpathcurveto{\pgfqpoint{2.020667in}{4.124992in}}{\pgfqpoint{2.016338in}{4.135443in}}{\pgfqpoint{2.008634in}{4.143147in}}%
\pgfpathcurveto{\pgfqpoint{2.000930in}{4.150851in}}{\pgfqpoint{1.990479in}{4.155180in}}{\pgfqpoint{1.979583in}{4.155180in}}%
\pgfpathcurveto{\pgfqpoint{1.968688in}{4.155180in}}{\pgfqpoint{1.958237in}{4.150851in}}{\pgfqpoint{1.950533in}{4.143147in}}%
\pgfpathcurveto{\pgfqpoint{1.942828in}{4.135443in}}{\pgfqpoint{1.938499in}{4.124992in}}{\pgfqpoint{1.938499in}{4.114096in}}%
\pgfpathcurveto{\pgfqpoint{1.938499in}{4.103201in}}{\pgfqpoint{1.942828in}{4.092750in}}{\pgfqpoint{1.950533in}{4.085046in}}%
\pgfpathcurveto{\pgfqpoint{1.958237in}{4.077341in}}{\pgfqpoint{1.968688in}{4.073012in}}{\pgfqpoint{1.979583in}{4.073012in}}%
\pgfpathlineto{\pgfqpoint{1.979583in}{4.073012in}}%
\pgfpathclose%
\pgfusepath{stroke}%
\end{pgfscope}%
\begin{pgfscope}%
\pgfpathrectangle{\pgfqpoint{0.688192in}{0.670138in}}{\pgfqpoint{7.111808in}{5.129862in}}%
\pgfusepath{clip}%
\pgfsetbuttcap%
\pgfsetroundjoin%
\pgfsetlinewidth{1.003750pt}%
\definecolor{currentstroke}{rgb}{0.000000,0.000000,0.000000}%
\pgfsetstrokecolor{currentstroke}%
\pgfsetdash{}{0pt}%
\pgfpathmoveto{\pgfqpoint{7.528370in}{1.287396in}}%
\pgfpathcurveto{\pgfqpoint{7.539266in}{1.287396in}}{\pgfqpoint{7.549716in}{1.291724in}}{\pgfqpoint{7.557421in}{1.299429in}}%
\pgfpathcurveto{\pgfqpoint{7.565125in}{1.307133in}}{\pgfqpoint{7.569454in}{1.317584in}}{\pgfqpoint{7.569454in}{1.328479in}}%
\pgfpathcurveto{\pgfqpoint{7.569454in}{1.339375in}}{\pgfqpoint{7.565125in}{1.349826in}}{\pgfqpoint{7.557421in}{1.357530in}}%
\pgfpathcurveto{\pgfqpoint{7.549716in}{1.365234in}}{\pgfqpoint{7.539266in}{1.369563in}}{\pgfqpoint{7.528370in}{1.369563in}}%
\pgfpathcurveto{\pgfqpoint{7.517474in}{1.369563in}}{\pgfqpoint{7.507024in}{1.365234in}}{\pgfqpoint{7.499319in}{1.357530in}}%
\pgfpathcurveto{\pgfqpoint{7.491615in}{1.349826in}}{\pgfqpoint{7.487286in}{1.339375in}}{\pgfqpoint{7.487286in}{1.328479in}}%
\pgfpathcurveto{\pgfqpoint{7.487286in}{1.317584in}}{\pgfqpoint{7.491615in}{1.307133in}}{\pgfqpoint{7.499319in}{1.299429in}}%
\pgfpathcurveto{\pgfqpoint{7.507024in}{1.291724in}}{\pgfqpoint{7.517474in}{1.287396in}}{\pgfqpoint{7.528370in}{1.287396in}}%
\pgfpathlineto{\pgfqpoint{7.528370in}{1.287396in}}%
\pgfpathclose%
\pgfusepath{stroke}%
\end{pgfscope}%
\begin{pgfscope}%
\pgfpathrectangle{\pgfqpoint{0.688192in}{0.670138in}}{\pgfqpoint{7.111808in}{5.129862in}}%
\pgfusepath{clip}%
\pgfsetbuttcap%
\pgfsetroundjoin%
\pgfsetlinewidth{1.003750pt}%
\definecolor{currentstroke}{rgb}{0.000000,0.000000,0.000000}%
\pgfsetstrokecolor{currentstroke}%
\pgfsetdash{}{0pt}%
\pgfpathmoveto{\pgfqpoint{1.389197in}{0.711266in}}%
\pgfpathcurveto{\pgfqpoint{1.400093in}{0.711266in}}{\pgfqpoint{1.410544in}{0.715595in}}{\pgfqpoint{1.418248in}{0.723299in}}%
\pgfpathcurveto{\pgfqpoint{1.425952in}{0.731004in}}{\pgfqpoint{1.430281in}{0.741454in}}{\pgfqpoint{1.430281in}{0.752350in}}%
\pgfpathcurveto{\pgfqpoint{1.430281in}{0.763246in}}{\pgfqpoint{1.425952in}{0.773696in}}{\pgfqpoint{1.418248in}{0.781401in}}%
\pgfpathcurveto{\pgfqpoint{1.410544in}{0.789105in}}{\pgfqpoint{1.400093in}{0.793434in}}{\pgfqpoint{1.389197in}{0.793434in}}%
\pgfpathcurveto{\pgfqpoint{1.378302in}{0.793434in}}{\pgfqpoint{1.367851in}{0.789105in}}{\pgfqpoint{1.360147in}{0.781401in}}%
\pgfpathcurveto{\pgfqpoint{1.352442in}{0.773696in}}{\pgfqpoint{1.348113in}{0.763246in}}{\pgfqpoint{1.348113in}{0.752350in}}%
\pgfpathcurveto{\pgfqpoint{1.348113in}{0.741454in}}{\pgfqpoint{1.352442in}{0.731004in}}{\pgfqpoint{1.360147in}{0.723299in}}%
\pgfpathcurveto{\pgfqpoint{1.367851in}{0.715595in}}{\pgfqpoint{1.378302in}{0.711266in}}{\pgfqpoint{1.389197in}{0.711266in}}%
\pgfpathlineto{\pgfqpoint{1.389197in}{0.711266in}}%
\pgfpathclose%
\pgfusepath{stroke}%
\end{pgfscope}%
\begin{pgfscope}%
\pgfpathrectangle{\pgfqpoint{0.688192in}{0.670138in}}{\pgfqpoint{7.111808in}{5.129862in}}%
\pgfusepath{clip}%
\pgfsetbuttcap%
\pgfsetroundjoin%
\pgfsetlinewidth{1.003750pt}%
\definecolor{currentstroke}{rgb}{0.000000,0.000000,0.000000}%
\pgfsetstrokecolor{currentstroke}%
\pgfsetdash{}{0pt}%
\pgfpathmoveto{\pgfqpoint{2.176728in}{0.687516in}}%
\pgfpathcurveto{\pgfqpoint{2.187623in}{0.687516in}}{\pgfqpoint{2.198074in}{0.691845in}}{\pgfqpoint{2.205778in}{0.699549in}}%
\pgfpathcurveto{\pgfqpoint{2.213483in}{0.707254in}}{\pgfqpoint{2.217812in}{0.717704in}}{\pgfqpoint{2.217812in}{0.728600in}}%
\pgfpathcurveto{\pgfqpoint{2.217812in}{0.739496in}}{\pgfqpoint{2.213483in}{0.749946in}}{\pgfqpoint{2.205778in}{0.757651in}}%
\pgfpathcurveto{\pgfqpoint{2.198074in}{0.765355in}}{\pgfqpoint{2.187623in}{0.769684in}}{\pgfqpoint{2.176728in}{0.769684in}}%
\pgfpathcurveto{\pgfqpoint{2.165832in}{0.769684in}}{\pgfqpoint{2.155381in}{0.765355in}}{\pgfqpoint{2.147677in}{0.757651in}}%
\pgfpathcurveto{\pgfqpoint{2.139973in}{0.749946in}}{\pgfqpoint{2.135644in}{0.739496in}}{\pgfqpoint{2.135644in}{0.728600in}}%
\pgfpathcurveto{\pgfqpoint{2.135644in}{0.717704in}}{\pgfqpoint{2.139973in}{0.707254in}}{\pgfqpoint{2.147677in}{0.699549in}}%
\pgfpathcurveto{\pgfqpoint{2.155381in}{0.691845in}}{\pgfqpoint{2.165832in}{0.687516in}}{\pgfqpoint{2.176728in}{0.687516in}}%
\pgfpathlineto{\pgfqpoint{2.176728in}{0.687516in}}%
\pgfpathclose%
\pgfusepath{stroke}%
\end{pgfscope}%
\begin{pgfscope}%
\pgfpathrectangle{\pgfqpoint{0.688192in}{0.670138in}}{\pgfqpoint{7.111808in}{5.129862in}}%
\pgfusepath{clip}%
\pgfsetbuttcap%
\pgfsetroundjoin%
\pgfsetlinewidth{1.003750pt}%
\definecolor{currentstroke}{rgb}{0.000000,0.000000,0.000000}%
\pgfsetstrokecolor{currentstroke}%
\pgfsetdash{}{0pt}%
\pgfpathmoveto{\pgfqpoint{1.090088in}{0.725570in}}%
\pgfpathcurveto{\pgfqpoint{1.100984in}{0.725570in}}{\pgfqpoint{1.111434in}{0.729898in}}{\pgfqpoint{1.119139in}{0.737603in}}%
\pgfpathcurveto{\pgfqpoint{1.126843in}{0.745307in}}{\pgfqpoint{1.131172in}{0.755758in}}{\pgfqpoint{1.131172in}{0.766654in}}%
\pgfpathcurveto{\pgfqpoint{1.131172in}{0.777549in}}{\pgfqpoint{1.126843in}{0.788000in}}{\pgfqpoint{1.119139in}{0.795704in}}%
\pgfpathcurveto{\pgfqpoint{1.111434in}{0.803409in}}{\pgfqpoint{1.100984in}{0.807737in}}{\pgfqpoint{1.090088in}{0.807737in}}%
\pgfpathcurveto{\pgfqpoint{1.079192in}{0.807737in}}{\pgfqpoint{1.068742in}{0.803409in}}{\pgfqpoint{1.061037in}{0.795704in}}%
\pgfpathcurveto{\pgfqpoint{1.053333in}{0.788000in}}{\pgfqpoint{1.049004in}{0.777549in}}{\pgfqpoint{1.049004in}{0.766654in}}%
\pgfpathcurveto{\pgfqpoint{1.049004in}{0.755758in}}{\pgfqpoint{1.053333in}{0.745307in}}{\pgfqpoint{1.061037in}{0.737603in}}%
\pgfpathcurveto{\pgfqpoint{1.068742in}{0.729898in}}{\pgfqpoint{1.079192in}{0.725570in}}{\pgfqpoint{1.090088in}{0.725570in}}%
\pgfpathlineto{\pgfqpoint{1.090088in}{0.725570in}}%
\pgfpathclose%
\pgfusepath{stroke}%
\end{pgfscope}%
\begin{pgfscope}%
\pgfpathrectangle{\pgfqpoint{0.688192in}{0.670138in}}{\pgfqpoint{7.111808in}{5.129862in}}%
\pgfusepath{clip}%
\pgfsetbuttcap%
\pgfsetroundjoin%
\pgfsetlinewidth{1.003750pt}%
\definecolor{currentstroke}{rgb}{0.000000,0.000000,0.000000}%
\pgfsetstrokecolor{currentstroke}%
\pgfsetdash{}{0pt}%
\pgfpathmoveto{\pgfqpoint{3.472431in}{3.781861in}}%
\pgfpathcurveto{\pgfqpoint{3.483327in}{3.781861in}}{\pgfqpoint{3.493777in}{3.786190in}}{\pgfqpoint{3.501482in}{3.793894in}}%
\pgfpathcurveto{\pgfqpoint{3.509186in}{3.801599in}}{\pgfqpoint{3.513515in}{3.812049in}}{\pgfqpoint{3.513515in}{3.822945in}}%
\pgfpathcurveto{\pgfqpoint{3.513515in}{3.833841in}}{\pgfqpoint{3.509186in}{3.844291in}}{\pgfqpoint{3.501482in}{3.851996in}}%
\pgfpathcurveto{\pgfqpoint{3.493777in}{3.859700in}}{\pgfqpoint{3.483327in}{3.864029in}}{\pgfqpoint{3.472431in}{3.864029in}}%
\pgfpathcurveto{\pgfqpoint{3.461536in}{3.864029in}}{\pgfqpoint{3.451085in}{3.859700in}}{\pgfqpoint{3.443380in}{3.851996in}}%
\pgfpathcurveto{\pgfqpoint{3.435676in}{3.844291in}}{\pgfqpoint{3.431347in}{3.833841in}}{\pgfqpoint{3.431347in}{3.822945in}}%
\pgfpathcurveto{\pgfqpoint{3.431347in}{3.812049in}}{\pgfqpoint{3.435676in}{3.801599in}}{\pgfqpoint{3.443380in}{3.793894in}}%
\pgfpathcurveto{\pgfqpoint{3.451085in}{3.786190in}}{\pgfqpoint{3.461536in}{3.781861in}}{\pgfqpoint{3.472431in}{3.781861in}}%
\pgfpathlineto{\pgfqpoint{3.472431in}{3.781861in}}%
\pgfpathclose%
\pgfusepath{stroke}%
\end{pgfscope}%
\begin{pgfscope}%
\pgfpathrectangle{\pgfqpoint{0.688192in}{0.670138in}}{\pgfqpoint{7.111808in}{5.129862in}}%
\pgfusepath{clip}%
\pgfsetbuttcap%
\pgfsetroundjoin%
\pgfsetlinewidth{1.003750pt}%
\definecolor{currentstroke}{rgb}{0.000000,0.000000,0.000000}%
\pgfsetstrokecolor{currentstroke}%
\pgfsetdash{}{0pt}%
\pgfpathmoveto{\pgfqpoint{1.058884in}{0.734763in}}%
\pgfpathcurveto{\pgfqpoint{1.069780in}{0.734763in}}{\pgfqpoint{1.080231in}{0.739091in}}{\pgfqpoint{1.087935in}{0.746796in}}%
\pgfpathcurveto{\pgfqpoint{1.095639in}{0.754500in}}{\pgfqpoint{1.099968in}{0.764951in}}{\pgfqpoint{1.099968in}{0.775846in}}%
\pgfpathcurveto{\pgfqpoint{1.099968in}{0.786742in}}{\pgfqpoint{1.095639in}{0.797193in}}{\pgfqpoint{1.087935in}{0.804897in}}%
\pgfpathcurveto{\pgfqpoint{1.080231in}{0.812601in}}{\pgfqpoint{1.069780in}{0.816930in}}{\pgfqpoint{1.058884in}{0.816930in}}%
\pgfpathcurveto{\pgfqpoint{1.047989in}{0.816930in}}{\pgfqpoint{1.037538in}{0.812601in}}{\pgfqpoint{1.029834in}{0.804897in}}%
\pgfpathcurveto{\pgfqpoint{1.022129in}{0.797193in}}{\pgfqpoint{1.017801in}{0.786742in}}{\pgfqpoint{1.017801in}{0.775846in}}%
\pgfpathcurveto{\pgfqpoint{1.017801in}{0.764951in}}{\pgfqpoint{1.022129in}{0.754500in}}{\pgfqpoint{1.029834in}{0.746796in}}%
\pgfpathcurveto{\pgfqpoint{1.037538in}{0.739091in}}{\pgfqpoint{1.047989in}{0.734763in}}{\pgfqpoint{1.058884in}{0.734763in}}%
\pgfpathlineto{\pgfqpoint{1.058884in}{0.734763in}}%
\pgfpathclose%
\pgfusepath{stroke}%
\end{pgfscope}%
\begin{pgfscope}%
\pgfpathrectangle{\pgfqpoint{0.688192in}{0.670138in}}{\pgfqpoint{7.111808in}{5.129862in}}%
\pgfusepath{clip}%
\pgfsetbuttcap%
\pgfsetroundjoin%
\pgfsetlinewidth{1.003750pt}%
\definecolor{currentstroke}{rgb}{0.000000,0.000000,0.000000}%
\pgfsetstrokecolor{currentstroke}%
\pgfsetdash{}{0pt}%
\pgfpathmoveto{\pgfqpoint{1.610058in}{0.705530in}}%
\pgfpathcurveto{\pgfqpoint{1.620954in}{0.705530in}}{\pgfqpoint{1.631404in}{0.709859in}}{\pgfqpoint{1.639109in}{0.717563in}}%
\pgfpathcurveto{\pgfqpoint{1.646813in}{0.725267in}}{\pgfqpoint{1.651142in}{0.735718in}}{\pgfqpoint{1.651142in}{0.746614in}}%
\pgfpathcurveto{\pgfqpoint{1.651142in}{0.757509in}}{\pgfqpoint{1.646813in}{0.767960in}}{\pgfqpoint{1.639109in}{0.775664in}}%
\pgfpathcurveto{\pgfqpoint{1.631404in}{0.783369in}}{\pgfqpoint{1.620954in}{0.787697in}}{\pgfqpoint{1.610058in}{0.787697in}}%
\pgfpathcurveto{\pgfqpoint{1.599163in}{0.787697in}}{\pgfqpoint{1.588712in}{0.783369in}}{\pgfqpoint{1.581007in}{0.775664in}}%
\pgfpathcurveto{\pgfqpoint{1.573303in}{0.767960in}}{\pgfqpoint{1.568974in}{0.757509in}}{\pgfqpoint{1.568974in}{0.746614in}}%
\pgfpathcurveto{\pgfqpoint{1.568974in}{0.735718in}}{\pgfqpoint{1.573303in}{0.725267in}}{\pgfqpoint{1.581007in}{0.717563in}}%
\pgfpathcurveto{\pgfqpoint{1.588712in}{0.709859in}}{\pgfqpoint{1.599163in}{0.705530in}}{\pgfqpoint{1.610058in}{0.705530in}}%
\pgfpathlineto{\pgfqpoint{1.610058in}{0.705530in}}%
\pgfpathclose%
\pgfusepath{stroke}%
\end{pgfscope}%
\begin{pgfscope}%
\pgfpathrectangle{\pgfqpoint{0.688192in}{0.670138in}}{\pgfqpoint{7.111808in}{5.129862in}}%
\pgfusepath{clip}%
\pgfsetbuttcap%
\pgfsetroundjoin%
\pgfsetlinewidth{1.003750pt}%
\definecolor{currentstroke}{rgb}{0.000000,0.000000,0.000000}%
\pgfsetstrokecolor{currentstroke}%
\pgfsetdash{}{0pt}%
\pgfpathmoveto{\pgfqpoint{0.766804in}{1.159490in}}%
\pgfpathcurveto{\pgfqpoint{0.777699in}{1.159490in}}{\pgfqpoint{0.788150in}{1.163819in}}{\pgfqpoint{0.795854in}{1.171523in}}%
\pgfpathcurveto{\pgfqpoint{0.803559in}{1.179228in}}{\pgfqpoint{0.807888in}{1.189678in}}{\pgfqpoint{0.807888in}{1.200574in}}%
\pgfpathcurveto{\pgfqpoint{0.807888in}{1.211469in}}{\pgfqpoint{0.803559in}{1.221920in}}{\pgfqpoint{0.795854in}{1.229625in}}%
\pgfpathcurveto{\pgfqpoint{0.788150in}{1.237329in}}{\pgfqpoint{0.777699in}{1.241658in}}{\pgfqpoint{0.766804in}{1.241658in}}%
\pgfpathcurveto{\pgfqpoint{0.755908in}{1.241658in}}{\pgfqpoint{0.745457in}{1.237329in}}{\pgfqpoint{0.737753in}{1.229625in}}%
\pgfpathcurveto{\pgfqpoint{0.730049in}{1.221920in}}{\pgfqpoint{0.725720in}{1.211469in}}{\pgfqpoint{0.725720in}{1.200574in}}%
\pgfpathcurveto{\pgfqpoint{0.725720in}{1.189678in}}{\pgfqpoint{0.730049in}{1.179228in}}{\pgfqpoint{0.737753in}{1.171523in}}%
\pgfpathcurveto{\pgfqpoint{0.745457in}{1.163819in}}{\pgfqpoint{0.755908in}{1.159490in}}{\pgfqpoint{0.766804in}{1.159490in}}%
\pgfpathlineto{\pgfqpoint{0.766804in}{1.159490in}}%
\pgfpathclose%
\pgfusepath{stroke}%
\end{pgfscope}%
\begin{pgfscope}%
\pgfpathrectangle{\pgfqpoint{0.688192in}{0.670138in}}{\pgfqpoint{7.111808in}{5.129862in}}%
\pgfusepath{clip}%
\pgfsetbuttcap%
\pgfsetroundjoin%
\pgfsetlinewidth{1.003750pt}%
\definecolor{currentstroke}{rgb}{0.000000,0.000000,0.000000}%
\pgfsetstrokecolor{currentstroke}%
\pgfsetdash{}{0pt}%
\pgfpathmoveto{\pgfqpoint{2.217170in}{0.684781in}}%
\pgfpathcurveto{\pgfqpoint{2.228066in}{0.684781in}}{\pgfqpoint{2.238517in}{0.689109in}}{\pgfqpoint{2.246221in}{0.696814in}}%
\pgfpathcurveto{\pgfqpoint{2.253925in}{0.704518in}}{\pgfqpoint{2.258254in}{0.714969in}}{\pgfqpoint{2.258254in}{0.725864in}}%
\pgfpathcurveto{\pgfqpoint{2.258254in}{0.736760in}}{\pgfqpoint{2.253925in}{0.747211in}}{\pgfqpoint{2.246221in}{0.754915in}}%
\pgfpathcurveto{\pgfqpoint{2.238517in}{0.762620in}}{\pgfqpoint{2.228066in}{0.766948in}}{\pgfqpoint{2.217170in}{0.766948in}}%
\pgfpathcurveto{\pgfqpoint{2.206275in}{0.766948in}}{\pgfqpoint{2.195824in}{0.762620in}}{\pgfqpoint{2.188120in}{0.754915in}}%
\pgfpathcurveto{\pgfqpoint{2.180415in}{0.747211in}}{\pgfqpoint{2.176087in}{0.736760in}}{\pgfqpoint{2.176087in}{0.725864in}}%
\pgfpathcurveto{\pgfqpoint{2.176087in}{0.714969in}}{\pgfqpoint{2.180415in}{0.704518in}}{\pgfqpoint{2.188120in}{0.696814in}}%
\pgfpathcurveto{\pgfqpoint{2.195824in}{0.689109in}}{\pgfqpoint{2.206275in}{0.684781in}}{\pgfqpoint{2.217170in}{0.684781in}}%
\pgfpathlineto{\pgfqpoint{2.217170in}{0.684781in}}%
\pgfpathclose%
\pgfusepath{stroke}%
\end{pgfscope}%
\begin{pgfscope}%
\pgfpathrectangle{\pgfqpoint{0.688192in}{0.670138in}}{\pgfqpoint{7.111808in}{5.129862in}}%
\pgfusepath{clip}%
\pgfsetbuttcap%
\pgfsetroundjoin%
\pgfsetlinewidth{1.003750pt}%
\definecolor{currentstroke}{rgb}{0.000000,0.000000,0.000000}%
\pgfsetstrokecolor{currentstroke}%
\pgfsetdash{}{0pt}%
\pgfpathmoveto{\pgfqpoint{3.852493in}{1.710376in}}%
\pgfpathcurveto{\pgfqpoint{3.863389in}{1.710376in}}{\pgfqpoint{3.873839in}{1.714705in}}{\pgfqpoint{3.881544in}{1.722409in}}%
\pgfpathcurveto{\pgfqpoint{3.889248in}{1.730113in}}{\pgfqpoint{3.893577in}{1.740564in}}{\pgfqpoint{3.893577in}{1.751460in}}%
\pgfpathcurveto{\pgfqpoint{3.893577in}{1.762355in}}{\pgfqpoint{3.889248in}{1.772806in}}{\pgfqpoint{3.881544in}{1.780511in}}%
\pgfpathcurveto{\pgfqpoint{3.873839in}{1.788215in}}{\pgfqpoint{3.863389in}{1.792544in}}{\pgfqpoint{3.852493in}{1.792544in}}%
\pgfpathcurveto{\pgfqpoint{3.841597in}{1.792544in}}{\pgfqpoint{3.831147in}{1.788215in}}{\pgfqpoint{3.823442in}{1.780511in}}%
\pgfpathcurveto{\pgfqpoint{3.815738in}{1.772806in}}{\pgfqpoint{3.811409in}{1.762355in}}{\pgfqpoint{3.811409in}{1.751460in}}%
\pgfpathcurveto{\pgfqpoint{3.811409in}{1.740564in}}{\pgfqpoint{3.815738in}{1.730113in}}{\pgfqpoint{3.823442in}{1.722409in}}%
\pgfpathcurveto{\pgfqpoint{3.831147in}{1.714705in}}{\pgfqpoint{3.841597in}{1.710376in}}{\pgfqpoint{3.852493in}{1.710376in}}%
\pgfpathlineto{\pgfqpoint{3.852493in}{1.710376in}}%
\pgfpathclose%
\pgfusepath{stroke}%
\end{pgfscope}%
\begin{pgfscope}%
\pgfpathrectangle{\pgfqpoint{0.688192in}{0.670138in}}{\pgfqpoint{7.111808in}{5.129862in}}%
\pgfusepath{clip}%
\pgfsetbuttcap%
\pgfsetroundjoin%
\pgfsetlinewidth{1.003750pt}%
\definecolor{currentstroke}{rgb}{0.000000,0.000000,0.000000}%
\pgfsetstrokecolor{currentstroke}%
\pgfsetdash{}{0pt}%
\pgfpathmoveto{\pgfqpoint{3.249297in}{4.404813in}}%
\pgfpathcurveto{\pgfqpoint{3.260192in}{4.404813in}}{\pgfqpoint{3.270643in}{4.409142in}}{\pgfqpoint{3.278347in}{4.416846in}}%
\pgfpathcurveto{\pgfqpoint{3.286052in}{4.424550in}}{\pgfqpoint{3.290380in}{4.435001in}}{\pgfqpoint{3.290380in}{4.445897in}}%
\pgfpathcurveto{\pgfqpoint{3.290380in}{4.456792in}}{\pgfqpoint{3.286052in}{4.467243in}}{\pgfqpoint{3.278347in}{4.474947in}}%
\pgfpathcurveto{\pgfqpoint{3.270643in}{4.482652in}}{\pgfqpoint{3.260192in}{4.486981in}}{\pgfqpoint{3.249297in}{4.486981in}}%
\pgfpathcurveto{\pgfqpoint{3.238401in}{4.486981in}}{\pgfqpoint{3.227950in}{4.482652in}}{\pgfqpoint{3.220246in}{4.474947in}}%
\pgfpathcurveto{\pgfqpoint{3.212542in}{4.467243in}}{\pgfqpoint{3.208213in}{4.456792in}}{\pgfqpoint{3.208213in}{4.445897in}}%
\pgfpathcurveto{\pgfqpoint{3.208213in}{4.435001in}}{\pgfqpoint{3.212542in}{4.424550in}}{\pgfqpoint{3.220246in}{4.416846in}}%
\pgfpathcurveto{\pgfqpoint{3.227950in}{4.409142in}}{\pgfqpoint{3.238401in}{4.404813in}}{\pgfqpoint{3.249297in}{4.404813in}}%
\pgfpathlineto{\pgfqpoint{3.249297in}{4.404813in}}%
\pgfpathclose%
\pgfusepath{stroke}%
\end{pgfscope}%
\begin{pgfscope}%
\pgfpathrectangle{\pgfqpoint{0.688192in}{0.670138in}}{\pgfqpoint{7.111808in}{5.129862in}}%
\pgfusepath{clip}%
\pgfsetbuttcap%
\pgfsetroundjoin%
\pgfsetlinewidth{1.003750pt}%
\definecolor{currentstroke}{rgb}{0.000000,0.000000,0.000000}%
\pgfsetstrokecolor{currentstroke}%
\pgfsetdash{}{0pt}%
\pgfpathmoveto{\pgfqpoint{0.931763in}{0.833664in}}%
\pgfpathcurveto{\pgfqpoint{0.942659in}{0.833664in}}{\pgfqpoint{0.953110in}{0.837992in}}{\pgfqpoint{0.960814in}{0.845697in}}%
\pgfpathcurveto{\pgfqpoint{0.968519in}{0.853401in}}{\pgfqpoint{0.972847in}{0.863852in}}{\pgfqpoint{0.972847in}{0.874747in}}%
\pgfpathcurveto{\pgfqpoint{0.972847in}{0.885643in}}{\pgfqpoint{0.968519in}{0.896094in}}{\pgfqpoint{0.960814in}{0.903798in}}%
\pgfpathcurveto{\pgfqpoint{0.953110in}{0.911502in}}{\pgfqpoint{0.942659in}{0.915831in}}{\pgfqpoint{0.931763in}{0.915831in}}%
\pgfpathcurveto{\pgfqpoint{0.920868in}{0.915831in}}{\pgfqpoint{0.910417in}{0.911502in}}{\pgfqpoint{0.902713in}{0.903798in}}%
\pgfpathcurveto{\pgfqpoint{0.895008in}{0.896094in}}{\pgfqpoint{0.890680in}{0.885643in}}{\pgfqpoint{0.890680in}{0.874747in}}%
\pgfpathcurveto{\pgfqpoint{0.890680in}{0.863852in}}{\pgfqpoint{0.895008in}{0.853401in}}{\pgfqpoint{0.902713in}{0.845697in}}%
\pgfpathcurveto{\pgfqpoint{0.910417in}{0.837992in}}{\pgfqpoint{0.920868in}{0.833664in}}{\pgfqpoint{0.931763in}{0.833664in}}%
\pgfpathlineto{\pgfqpoint{0.931763in}{0.833664in}}%
\pgfpathclose%
\pgfusepath{stroke}%
\end{pgfscope}%
\begin{pgfscope}%
\pgfpathrectangle{\pgfqpoint{0.688192in}{0.670138in}}{\pgfqpoint{7.111808in}{5.129862in}}%
\pgfusepath{clip}%
\pgfsetbuttcap%
\pgfsetroundjoin%
\pgfsetlinewidth{1.003750pt}%
\definecolor{currentstroke}{rgb}{0.000000,0.000000,0.000000}%
\pgfsetstrokecolor{currentstroke}%
\pgfsetdash{}{0pt}%
\pgfpathmoveto{\pgfqpoint{0.928243in}{0.841597in}}%
\pgfpathcurveto{\pgfqpoint{0.939139in}{0.841597in}}{\pgfqpoint{0.949589in}{0.845926in}}{\pgfqpoint{0.957294in}{0.853631in}}%
\pgfpathcurveto{\pgfqpoint{0.964998in}{0.861335in}}{\pgfqpoint{0.969327in}{0.871786in}}{\pgfqpoint{0.969327in}{0.882681in}}%
\pgfpathcurveto{\pgfqpoint{0.969327in}{0.893577in}}{\pgfqpoint{0.964998in}{0.904028in}}{\pgfqpoint{0.957294in}{0.911732in}}%
\pgfpathcurveto{\pgfqpoint{0.949589in}{0.919436in}}{\pgfqpoint{0.939139in}{0.923765in}}{\pgfqpoint{0.928243in}{0.923765in}}%
\pgfpathcurveto{\pgfqpoint{0.917347in}{0.923765in}}{\pgfqpoint{0.906897in}{0.919436in}}{\pgfqpoint{0.899192in}{0.911732in}}%
\pgfpathcurveto{\pgfqpoint{0.891488in}{0.904028in}}{\pgfqpoint{0.887159in}{0.893577in}}{\pgfqpoint{0.887159in}{0.882681in}}%
\pgfpathcurveto{\pgfqpoint{0.887159in}{0.871786in}}{\pgfqpoint{0.891488in}{0.861335in}}{\pgfqpoint{0.899192in}{0.853631in}}%
\pgfpathcurveto{\pgfqpoint{0.906897in}{0.845926in}}{\pgfqpoint{0.917347in}{0.841597in}}{\pgfqpoint{0.928243in}{0.841597in}}%
\pgfpathlineto{\pgfqpoint{0.928243in}{0.841597in}}%
\pgfpathclose%
\pgfusepath{stroke}%
\end{pgfscope}%
\begin{pgfscope}%
\pgfpathrectangle{\pgfqpoint{0.688192in}{0.670138in}}{\pgfqpoint{7.111808in}{5.129862in}}%
\pgfusepath{clip}%
\pgfsetbuttcap%
\pgfsetroundjoin%
\pgfsetlinewidth{1.003750pt}%
\definecolor{currentstroke}{rgb}{0.000000,0.000000,0.000000}%
\pgfsetstrokecolor{currentstroke}%
\pgfsetdash{}{0pt}%
\pgfpathmoveto{\pgfqpoint{2.443271in}{0.678334in}}%
\pgfpathcurveto{\pgfqpoint{2.454167in}{0.678334in}}{\pgfqpoint{2.464618in}{0.682663in}}{\pgfqpoint{2.472322in}{0.690367in}}%
\pgfpathcurveto{\pgfqpoint{2.480026in}{0.698072in}}{\pgfqpoint{2.484355in}{0.708523in}}{\pgfqpoint{2.484355in}{0.719418in}}%
\pgfpathcurveto{\pgfqpoint{2.484355in}{0.730314in}}{\pgfqpoint{2.480026in}{0.740764in}}{\pgfqpoint{2.472322in}{0.748469in}}%
\pgfpathcurveto{\pgfqpoint{2.464618in}{0.756173in}}{\pgfqpoint{2.454167in}{0.760502in}}{\pgfqpoint{2.443271in}{0.760502in}}%
\pgfpathcurveto{\pgfqpoint{2.432376in}{0.760502in}}{\pgfqpoint{2.421925in}{0.756173in}}{\pgfqpoint{2.414221in}{0.748469in}}%
\pgfpathcurveto{\pgfqpoint{2.406516in}{0.740764in}}{\pgfqpoint{2.402188in}{0.730314in}}{\pgfqpoint{2.402188in}{0.719418in}}%
\pgfpathcurveto{\pgfqpoint{2.402188in}{0.708523in}}{\pgfqpoint{2.406516in}{0.698072in}}{\pgfqpoint{2.414221in}{0.690367in}}%
\pgfpathcurveto{\pgfqpoint{2.421925in}{0.682663in}}{\pgfqpoint{2.432376in}{0.678334in}}{\pgfqpoint{2.443271in}{0.678334in}}%
\pgfpathlineto{\pgfqpoint{2.443271in}{0.678334in}}%
\pgfpathclose%
\pgfusepath{stroke}%
\end{pgfscope}%
\begin{pgfscope}%
\pgfpathrectangle{\pgfqpoint{0.688192in}{0.670138in}}{\pgfqpoint{7.111808in}{5.129862in}}%
\pgfusepath{clip}%
\pgfsetbuttcap%
\pgfsetroundjoin%
\pgfsetlinewidth{1.003750pt}%
\definecolor{currentstroke}{rgb}{0.000000,0.000000,0.000000}%
\pgfsetstrokecolor{currentstroke}%
\pgfsetdash{}{0pt}%
\pgfpathmoveto{\pgfqpoint{0.931867in}{0.825133in}}%
\pgfpathcurveto{\pgfqpoint{0.942763in}{0.825133in}}{\pgfqpoint{0.953213in}{0.829462in}}{\pgfqpoint{0.960918in}{0.837167in}}%
\pgfpathcurveto{\pgfqpoint{0.968622in}{0.844871in}}{\pgfqpoint{0.972951in}{0.855322in}}{\pgfqpoint{0.972951in}{0.866217in}}%
\pgfpathcurveto{\pgfqpoint{0.972951in}{0.877113in}}{\pgfqpoint{0.968622in}{0.887564in}}{\pgfqpoint{0.960918in}{0.895268in}}%
\pgfpathcurveto{\pgfqpoint{0.953213in}{0.902972in}}{\pgfqpoint{0.942763in}{0.907301in}}{\pgfqpoint{0.931867in}{0.907301in}}%
\pgfpathcurveto{\pgfqpoint{0.920971in}{0.907301in}}{\pgfqpoint{0.910521in}{0.902972in}}{\pgfqpoint{0.902816in}{0.895268in}}%
\pgfpathcurveto{\pgfqpoint{0.895112in}{0.887564in}}{\pgfqpoint{0.890783in}{0.877113in}}{\pgfqpoint{0.890783in}{0.866217in}}%
\pgfpathcurveto{\pgfqpoint{0.890783in}{0.855322in}}{\pgfqpoint{0.895112in}{0.844871in}}{\pgfqpoint{0.902816in}{0.837167in}}%
\pgfpathcurveto{\pgfqpoint{0.910521in}{0.829462in}}{\pgfqpoint{0.920971in}{0.825133in}}{\pgfqpoint{0.931867in}{0.825133in}}%
\pgfpathlineto{\pgfqpoint{0.931867in}{0.825133in}}%
\pgfpathclose%
\pgfusepath{stroke}%
\end{pgfscope}%
\begin{pgfscope}%
\pgfpathrectangle{\pgfqpoint{0.688192in}{0.670138in}}{\pgfqpoint{7.111808in}{5.129862in}}%
\pgfusepath{clip}%
\pgfsetbuttcap%
\pgfsetroundjoin%
\pgfsetlinewidth{1.003750pt}%
\definecolor{currentstroke}{rgb}{0.000000,0.000000,0.000000}%
\pgfsetstrokecolor{currentstroke}%
\pgfsetdash{}{0pt}%
\pgfpathmoveto{\pgfqpoint{1.441321in}{0.708382in}}%
\pgfpathcurveto{\pgfqpoint{1.452217in}{0.708382in}}{\pgfqpoint{1.462667in}{0.712711in}}{\pgfqpoint{1.470372in}{0.720416in}}%
\pgfpathcurveto{\pgfqpoint{1.478076in}{0.728120in}}{\pgfqpoint{1.482405in}{0.738571in}}{\pgfqpoint{1.482405in}{0.749466in}}%
\pgfpathcurveto{\pgfqpoint{1.482405in}{0.760362in}}{\pgfqpoint{1.478076in}{0.770813in}}{\pgfqpoint{1.470372in}{0.778517in}}%
\pgfpathcurveto{\pgfqpoint{1.462667in}{0.786221in}}{\pgfqpoint{1.452217in}{0.790550in}}{\pgfqpoint{1.441321in}{0.790550in}}%
\pgfpathcurveto{\pgfqpoint{1.430425in}{0.790550in}}{\pgfqpoint{1.419975in}{0.786221in}}{\pgfqpoint{1.412270in}{0.778517in}}%
\pgfpathcurveto{\pgfqpoint{1.404566in}{0.770813in}}{\pgfqpoint{1.400237in}{0.760362in}}{\pgfqpoint{1.400237in}{0.749466in}}%
\pgfpathcurveto{\pgfqpoint{1.400237in}{0.738571in}}{\pgfqpoint{1.404566in}{0.728120in}}{\pgfqpoint{1.412270in}{0.720416in}}%
\pgfpathcurveto{\pgfqpoint{1.419975in}{0.712711in}}{\pgfqpoint{1.430425in}{0.708382in}}{\pgfqpoint{1.441321in}{0.708382in}}%
\pgfpathlineto{\pgfqpoint{1.441321in}{0.708382in}}%
\pgfpathclose%
\pgfusepath{stroke}%
\end{pgfscope}%
\begin{pgfscope}%
\pgfpathrectangle{\pgfqpoint{0.688192in}{0.670138in}}{\pgfqpoint{7.111808in}{5.129862in}}%
\pgfusepath{clip}%
\pgfsetbuttcap%
\pgfsetroundjoin%
\pgfsetlinewidth{1.003750pt}%
\definecolor{currentstroke}{rgb}{0.000000,0.000000,0.000000}%
\pgfsetstrokecolor{currentstroke}%
\pgfsetdash{}{0pt}%
\pgfpathmoveto{\pgfqpoint{6.526448in}{3.569328in}}%
\pgfpathcurveto{\pgfqpoint{6.537344in}{3.569328in}}{\pgfqpoint{6.547794in}{3.573656in}}{\pgfqpoint{6.555499in}{3.581361in}}%
\pgfpathcurveto{\pgfqpoint{6.563203in}{3.589065in}}{\pgfqpoint{6.567532in}{3.599516in}}{\pgfqpoint{6.567532in}{3.610411in}}%
\pgfpathcurveto{\pgfqpoint{6.567532in}{3.621307in}}{\pgfqpoint{6.563203in}{3.631758in}}{\pgfqpoint{6.555499in}{3.639462in}}%
\pgfpathcurveto{\pgfqpoint{6.547794in}{3.647166in}}{\pgfqpoint{6.537344in}{3.651495in}}{\pgfqpoint{6.526448in}{3.651495in}}%
\pgfpathcurveto{\pgfqpoint{6.515552in}{3.651495in}}{\pgfqpoint{6.505102in}{3.647166in}}{\pgfqpoint{6.497397in}{3.639462in}}%
\pgfpathcurveto{\pgfqpoint{6.489693in}{3.631758in}}{\pgfqpoint{6.485364in}{3.621307in}}{\pgfqpoint{6.485364in}{3.610411in}}%
\pgfpathcurveto{\pgfqpoint{6.485364in}{3.599516in}}{\pgfqpoint{6.489693in}{3.589065in}}{\pgfqpoint{6.497397in}{3.581361in}}%
\pgfpathcurveto{\pgfqpoint{6.505102in}{3.573656in}}{\pgfqpoint{6.515552in}{3.569328in}}{\pgfqpoint{6.526448in}{3.569328in}}%
\pgfpathlineto{\pgfqpoint{6.526448in}{3.569328in}}%
\pgfpathclose%
\pgfusepath{stroke}%
\end{pgfscope}%
\begin{pgfscope}%
\pgfpathrectangle{\pgfqpoint{0.688192in}{0.670138in}}{\pgfqpoint{7.111808in}{5.129862in}}%
\pgfusepath{clip}%
\pgfsetbuttcap%
\pgfsetroundjoin%
\pgfsetlinewidth{1.003750pt}%
\definecolor{currentstroke}{rgb}{0.000000,0.000000,0.000000}%
\pgfsetstrokecolor{currentstroke}%
\pgfsetdash{}{0pt}%
\pgfpathmoveto{\pgfqpoint{1.610058in}{0.705530in}}%
\pgfpathcurveto{\pgfqpoint{1.620954in}{0.705530in}}{\pgfqpoint{1.631404in}{0.709859in}}{\pgfqpoint{1.639109in}{0.717563in}}%
\pgfpathcurveto{\pgfqpoint{1.646813in}{0.725267in}}{\pgfqpoint{1.651142in}{0.735718in}}{\pgfqpoint{1.651142in}{0.746614in}}%
\pgfpathcurveto{\pgfqpoint{1.651142in}{0.757509in}}{\pgfqpoint{1.646813in}{0.767960in}}{\pgfqpoint{1.639109in}{0.775664in}}%
\pgfpathcurveto{\pgfqpoint{1.631404in}{0.783369in}}{\pgfqpoint{1.620954in}{0.787697in}}{\pgfqpoint{1.610058in}{0.787697in}}%
\pgfpathcurveto{\pgfqpoint{1.599163in}{0.787697in}}{\pgfqpoint{1.588712in}{0.783369in}}{\pgfqpoint{1.581007in}{0.775664in}}%
\pgfpathcurveto{\pgfqpoint{1.573303in}{0.767960in}}{\pgfqpoint{1.568974in}{0.757509in}}{\pgfqpoint{1.568974in}{0.746614in}}%
\pgfpathcurveto{\pgfqpoint{1.568974in}{0.735718in}}{\pgfqpoint{1.573303in}{0.725267in}}{\pgfqpoint{1.581007in}{0.717563in}}%
\pgfpathcurveto{\pgfqpoint{1.588712in}{0.709859in}}{\pgfqpoint{1.599163in}{0.705530in}}{\pgfqpoint{1.610058in}{0.705530in}}%
\pgfpathlineto{\pgfqpoint{1.610058in}{0.705530in}}%
\pgfpathclose%
\pgfusepath{stroke}%
\end{pgfscope}%
\begin{pgfscope}%
\pgfpathrectangle{\pgfqpoint{0.688192in}{0.670138in}}{\pgfqpoint{7.111808in}{5.129862in}}%
\pgfusepath{clip}%
\pgfsetbuttcap%
\pgfsetroundjoin%
\pgfsetlinewidth{1.003750pt}%
\definecolor{currentstroke}{rgb}{0.000000,0.000000,0.000000}%
\pgfsetstrokecolor{currentstroke}%
\pgfsetdash{}{0pt}%
\pgfpathmoveto{\pgfqpoint{0.901363in}{0.856548in}}%
\pgfpathcurveto{\pgfqpoint{0.912259in}{0.856548in}}{\pgfqpoint{0.922710in}{0.860877in}}{\pgfqpoint{0.930414in}{0.868582in}}%
\pgfpathcurveto{\pgfqpoint{0.938118in}{0.876286in}}{\pgfqpoint{0.942447in}{0.886737in}}{\pgfqpoint{0.942447in}{0.897632in}}%
\pgfpathcurveto{\pgfqpoint{0.942447in}{0.908528in}}{\pgfqpoint{0.938118in}{0.918979in}}{\pgfqpoint{0.930414in}{0.926683in}}%
\pgfpathcurveto{\pgfqpoint{0.922710in}{0.934387in}}{\pgfqpoint{0.912259in}{0.938716in}}{\pgfqpoint{0.901363in}{0.938716in}}%
\pgfpathcurveto{\pgfqpoint{0.890468in}{0.938716in}}{\pgfqpoint{0.880017in}{0.934387in}}{\pgfqpoint{0.872313in}{0.926683in}}%
\pgfpathcurveto{\pgfqpoint{0.864608in}{0.918979in}}{\pgfqpoint{0.860280in}{0.908528in}}{\pgfqpoint{0.860280in}{0.897632in}}%
\pgfpathcurveto{\pgfqpoint{0.860280in}{0.886737in}}{\pgfqpoint{0.864608in}{0.876286in}}{\pgfqpoint{0.872313in}{0.868582in}}%
\pgfpathcurveto{\pgfqpoint{0.880017in}{0.860877in}}{\pgfqpoint{0.890468in}{0.856548in}}{\pgfqpoint{0.901363in}{0.856548in}}%
\pgfpathlineto{\pgfqpoint{0.901363in}{0.856548in}}%
\pgfpathclose%
\pgfusepath{stroke}%
\end{pgfscope}%
\begin{pgfscope}%
\pgfpathrectangle{\pgfqpoint{0.688192in}{0.670138in}}{\pgfqpoint{7.111808in}{5.129862in}}%
\pgfusepath{clip}%
\pgfsetbuttcap%
\pgfsetroundjoin%
\pgfsetlinewidth{1.003750pt}%
\definecolor{currentstroke}{rgb}{0.000000,0.000000,0.000000}%
\pgfsetstrokecolor{currentstroke}%
\pgfsetdash{}{0pt}%
\pgfpathmoveto{\pgfqpoint{2.831736in}{1.077947in}}%
\pgfpathcurveto{\pgfqpoint{2.842631in}{1.077947in}}{\pgfqpoint{2.853082in}{1.082276in}}{\pgfqpoint{2.860786in}{1.089980in}}%
\pgfpathcurveto{\pgfqpoint{2.868491in}{1.097685in}}{\pgfqpoint{2.872820in}{1.108136in}}{\pgfqpoint{2.872820in}{1.119031in}}%
\pgfpathcurveto{\pgfqpoint{2.872820in}{1.129927in}}{\pgfqpoint{2.868491in}{1.140378in}}{\pgfqpoint{2.860786in}{1.148082in}}%
\pgfpathcurveto{\pgfqpoint{2.853082in}{1.155786in}}{\pgfqpoint{2.842631in}{1.160115in}}{\pgfqpoint{2.831736in}{1.160115in}}%
\pgfpathcurveto{\pgfqpoint{2.820840in}{1.160115in}}{\pgfqpoint{2.810389in}{1.155786in}}{\pgfqpoint{2.802685in}{1.148082in}}%
\pgfpathcurveto{\pgfqpoint{2.794981in}{1.140378in}}{\pgfqpoint{2.790652in}{1.129927in}}{\pgfqpoint{2.790652in}{1.119031in}}%
\pgfpathcurveto{\pgfqpoint{2.790652in}{1.108136in}}{\pgfqpoint{2.794981in}{1.097685in}}{\pgfqpoint{2.802685in}{1.089980in}}%
\pgfpathcurveto{\pgfqpoint{2.810389in}{1.082276in}}{\pgfqpoint{2.820840in}{1.077947in}}{\pgfqpoint{2.831736in}{1.077947in}}%
\pgfpathlineto{\pgfqpoint{2.831736in}{1.077947in}}%
\pgfpathclose%
\pgfusepath{stroke}%
\end{pgfscope}%
\begin{pgfscope}%
\pgfpathrectangle{\pgfqpoint{0.688192in}{0.670138in}}{\pgfqpoint{7.111808in}{5.129862in}}%
\pgfusepath{clip}%
\pgfsetbuttcap%
\pgfsetroundjoin%
\pgfsetlinewidth{1.003750pt}%
\definecolor{currentstroke}{rgb}{0.000000,0.000000,0.000000}%
\pgfsetstrokecolor{currentstroke}%
\pgfsetdash{}{0pt}%
\pgfpathmoveto{\pgfqpoint{3.775225in}{1.090531in}}%
\pgfpathcurveto{\pgfqpoint{3.786120in}{1.090531in}}{\pgfqpoint{3.796571in}{1.094860in}}{\pgfqpoint{3.804275in}{1.102564in}}%
\pgfpathcurveto{\pgfqpoint{3.811980in}{1.110269in}}{\pgfqpoint{3.816309in}{1.120719in}}{\pgfqpoint{3.816309in}{1.131615in}}%
\pgfpathcurveto{\pgfqpoint{3.816309in}{1.142511in}}{\pgfqpoint{3.811980in}{1.152961in}}{\pgfqpoint{3.804275in}{1.160666in}}%
\pgfpathcurveto{\pgfqpoint{3.796571in}{1.168370in}}{\pgfqpoint{3.786120in}{1.172699in}}{\pgfqpoint{3.775225in}{1.172699in}}%
\pgfpathcurveto{\pgfqpoint{3.764329in}{1.172699in}}{\pgfqpoint{3.753878in}{1.168370in}}{\pgfqpoint{3.746174in}{1.160666in}}%
\pgfpathcurveto{\pgfqpoint{3.738470in}{1.152961in}}{\pgfqpoint{3.734141in}{1.142511in}}{\pgfqpoint{3.734141in}{1.131615in}}%
\pgfpathcurveto{\pgfqpoint{3.734141in}{1.120719in}}{\pgfqpoint{3.738470in}{1.110269in}}{\pgfqpoint{3.746174in}{1.102564in}}%
\pgfpathcurveto{\pgfqpoint{3.753878in}{1.094860in}}{\pgfqpoint{3.764329in}{1.090531in}}{\pgfqpoint{3.775225in}{1.090531in}}%
\pgfpathlineto{\pgfqpoint{3.775225in}{1.090531in}}%
\pgfpathclose%
\pgfusepath{stroke}%
\end{pgfscope}%
\begin{pgfscope}%
\pgfpathrectangle{\pgfqpoint{0.688192in}{0.670138in}}{\pgfqpoint{7.111808in}{5.129862in}}%
\pgfusepath{clip}%
\pgfsetbuttcap%
\pgfsetroundjoin%
\pgfsetlinewidth{1.003750pt}%
\definecolor{currentstroke}{rgb}{0.000000,0.000000,0.000000}%
\pgfsetstrokecolor{currentstroke}%
\pgfsetdash{}{0pt}%
\pgfpathmoveto{\pgfqpoint{0.881277in}{0.900428in}}%
\pgfpathcurveto{\pgfqpoint{0.892173in}{0.900428in}}{\pgfqpoint{0.902624in}{0.904756in}}{\pgfqpoint{0.910328in}{0.912461in}}%
\pgfpathcurveto{\pgfqpoint{0.918032in}{0.920165in}}{\pgfqpoint{0.922361in}{0.930616in}}{\pgfqpoint{0.922361in}{0.941511in}}%
\pgfpathcurveto{\pgfqpoint{0.922361in}{0.952407in}}{\pgfqpoint{0.918032in}{0.962858in}}{\pgfqpoint{0.910328in}{0.970562in}}%
\pgfpathcurveto{\pgfqpoint{0.902624in}{0.978266in}}{\pgfqpoint{0.892173in}{0.982595in}}{\pgfqpoint{0.881277in}{0.982595in}}%
\pgfpathcurveto{\pgfqpoint{0.870382in}{0.982595in}}{\pgfqpoint{0.859931in}{0.978266in}}{\pgfqpoint{0.852226in}{0.970562in}}%
\pgfpathcurveto{\pgfqpoint{0.844522in}{0.962858in}}{\pgfqpoint{0.840193in}{0.952407in}}{\pgfqpoint{0.840193in}{0.941511in}}%
\pgfpathcurveto{\pgfqpoint{0.840193in}{0.930616in}}{\pgfqpoint{0.844522in}{0.920165in}}{\pgfqpoint{0.852226in}{0.912461in}}%
\pgfpathcurveto{\pgfqpoint{0.859931in}{0.904756in}}{\pgfqpoint{0.870382in}{0.900428in}}{\pgfqpoint{0.881277in}{0.900428in}}%
\pgfpathlineto{\pgfqpoint{0.881277in}{0.900428in}}%
\pgfpathclose%
\pgfusepath{stroke}%
\end{pgfscope}%
\begin{pgfscope}%
\pgfpathrectangle{\pgfqpoint{0.688192in}{0.670138in}}{\pgfqpoint{7.111808in}{5.129862in}}%
\pgfusepath{clip}%
\pgfsetbuttcap%
\pgfsetroundjoin%
\pgfsetlinewidth{1.003750pt}%
\definecolor{currentstroke}{rgb}{0.000000,0.000000,0.000000}%
\pgfsetstrokecolor{currentstroke}%
\pgfsetdash{}{0pt}%
\pgfpathmoveto{\pgfqpoint{5.807370in}{0.906675in}}%
\pgfpathcurveto{\pgfqpoint{5.818266in}{0.906675in}}{\pgfqpoint{5.828717in}{0.911004in}}{\pgfqpoint{5.836421in}{0.918708in}}%
\pgfpathcurveto{\pgfqpoint{5.844125in}{0.926412in}}{\pgfqpoint{5.848454in}{0.936863in}}{\pgfqpoint{5.848454in}{0.947759in}}%
\pgfpathcurveto{\pgfqpoint{5.848454in}{0.958654in}}{\pgfqpoint{5.844125in}{0.969105in}}{\pgfqpoint{5.836421in}{0.976809in}}%
\pgfpathcurveto{\pgfqpoint{5.828717in}{0.984514in}}{\pgfqpoint{5.818266in}{0.988843in}}{\pgfqpoint{5.807370in}{0.988843in}}%
\pgfpathcurveto{\pgfqpoint{5.796475in}{0.988843in}}{\pgfqpoint{5.786024in}{0.984514in}}{\pgfqpoint{5.778320in}{0.976809in}}%
\pgfpathcurveto{\pgfqpoint{5.770615in}{0.969105in}}{\pgfqpoint{5.766286in}{0.958654in}}{\pgfqpoint{5.766286in}{0.947759in}}%
\pgfpathcurveto{\pgfqpoint{5.766286in}{0.936863in}}{\pgfqpoint{5.770615in}{0.926412in}}{\pgfqpoint{5.778320in}{0.918708in}}%
\pgfpathcurveto{\pgfqpoint{5.786024in}{0.911004in}}{\pgfqpoint{5.796475in}{0.906675in}}{\pgfqpoint{5.807370in}{0.906675in}}%
\pgfpathlineto{\pgfqpoint{5.807370in}{0.906675in}}%
\pgfpathclose%
\pgfusepath{stroke}%
\end{pgfscope}%
\begin{pgfscope}%
\pgfpathrectangle{\pgfqpoint{0.688192in}{0.670138in}}{\pgfqpoint{7.111808in}{5.129862in}}%
\pgfusepath{clip}%
\pgfsetbuttcap%
\pgfsetroundjoin%
\pgfsetlinewidth{1.003750pt}%
\definecolor{currentstroke}{rgb}{0.000000,0.000000,0.000000}%
\pgfsetstrokecolor{currentstroke}%
\pgfsetdash{}{0pt}%
\pgfpathmoveto{\pgfqpoint{1.610058in}{0.705530in}}%
\pgfpathcurveto{\pgfqpoint{1.620954in}{0.705530in}}{\pgfqpoint{1.631404in}{0.709859in}}{\pgfqpoint{1.639109in}{0.717563in}}%
\pgfpathcurveto{\pgfqpoint{1.646813in}{0.725267in}}{\pgfqpoint{1.651142in}{0.735718in}}{\pgfqpoint{1.651142in}{0.746614in}}%
\pgfpathcurveto{\pgfqpoint{1.651142in}{0.757509in}}{\pgfqpoint{1.646813in}{0.767960in}}{\pgfqpoint{1.639109in}{0.775664in}}%
\pgfpathcurveto{\pgfqpoint{1.631404in}{0.783369in}}{\pgfqpoint{1.620954in}{0.787697in}}{\pgfqpoint{1.610058in}{0.787697in}}%
\pgfpathcurveto{\pgfqpoint{1.599163in}{0.787697in}}{\pgfqpoint{1.588712in}{0.783369in}}{\pgfqpoint{1.581007in}{0.775664in}}%
\pgfpathcurveto{\pgfqpoint{1.573303in}{0.767960in}}{\pgfqpoint{1.568974in}{0.757509in}}{\pgfqpoint{1.568974in}{0.746614in}}%
\pgfpathcurveto{\pgfqpoint{1.568974in}{0.735718in}}{\pgfqpoint{1.573303in}{0.725267in}}{\pgfqpoint{1.581007in}{0.717563in}}%
\pgfpathcurveto{\pgfqpoint{1.588712in}{0.709859in}}{\pgfqpoint{1.599163in}{0.705530in}}{\pgfqpoint{1.610058in}{0.705530in}}%
\pgfpathlineto{\pgfqpoint{1.610058in}{0.705530in}}%
\pgfpathclose%
\pgfusepath{stroke}%
\end{pgfscope}%
\begin{pgfscope}%
\pgfpathrectangle{\pgfqpoint{0.688192in}{0.670138in}}{\pgfqpoint{7.111808in}{5.129862in}}%
\pgfusepath{clip}%
\pgfsetbuttcap%
\pgfsetroundjoin%
\pgfsetlinewidth{1.003750pt}%
\definecolor{currentstroke}{rgb}{0.000000,0.000000,0.000000}%
\pgfsetstrokecolor{currentstroke}%
\pgfsetdash{}{0pt}%
\pgfpathmoveto{\pgfqpoint{0.830999in}{0.952350in}}%
\pgfpathcurveto{\pgfqpoint{0.841894in}{0.952350in}}{\pgfqpoint{0.852345in}{0.956679in}}{\pgfqpoint{0.860049in}{0.964383in}}%
\pgfpathcurveto{\pgfqpoint{0.867754in}{0.972087in}}{\pgfqpoint{0.872082in}{0.982538in}}{\pgfqpoint{0.872082in}{0.993434in}}%
\pgfpathcurveto{\pgfqpoint{0.872082in}{1.004329in}}{\pgfqpoint{0.867754in}{1.014780in}}{\pgfqpoint{0.860049in}{1.022484in}}%
\pgfpathcurveto{\pgfqpoint{0.852345in}{1.030189in}}{\pgfqpoint{0.841894in}{1.034518in}}{\pgfqpoint{0.830999in}{1.034518in}}%
\pgfpathcurveto{\pgfqpoint{0.820103in}{1.034518in}}{\pgfqpoint{0.809652in}{1.030189in}}{\pgfqpoint{0.801948in}{1.022484in}}%
\pgfpathcurveto{\pgfqpoint{0.794243in}{1.014780in}}{\pgfqpoint{0.789915in}{1.004329in}}{\pgfqpoint{0.789915in}{0.993434in}}%
\pgfpathcurveto{\pgfqpoint{0.789915in}{0.982538in}}{\pgfqpoint{0.794243in}{0.972087in}}{\pgfqpoint{0.801948in}{0.964383in}}%
\pgfpathcurveto{\pgfqpoint{0.809652in}{0.956679in}}{\pgfqpoint{0.820103in}{0.952350in}}{\pgfqpoint{0.830999in}{0.952350in}}%
\pgfpathlineto{\pgfqpoint{0.830999in}{0.952350in}}%
\pgfpathclose%
\pgfusepath{stroke}%
\end{pgfscope}%
\begin{pgfscope}%
\pgfpathrectangle{\pgfqpoint{0.688192in}{0.670138in}}{\pgfqpoint{7.111808in}{5.129862in}}%
\pgfusepath{clip}%
\pgfsetbuttcap%
\pgfsetroundjoin%
\pgfsetlinewidth{1.003750pt}%
\definecolor{currentstroke}{rgb}{0.000000,0.000000,0.000000}%
\pgfsetstrokecolor{currentstroke}%
\pgfsetdash{}{0pt}%
\pgfpathmoveto{\pgfqpoint{1.920423in}{0.692661in}}%
\pgfpathcurveto{\pgfqpoint{1.931318in}{0.692661in}}{\pgfqpoint{1.941769in}{0.696990in}}{\pgfqpoint{1.949473in}{0.704695in}}%
\pgfpathcurveto{\pgfqpoint{1.957178in}{0.712399in}}{\pgfqpoint{1.961506in}{0.722850in}}{\pgfqpoint{1.961506in}{0.733745in}}%
\pgfpathcurveto{\pgfqpoint{1.961506in}{0.744641in}}{\pgfqpoint{1.957178in}{0.755092in}}{\pgfqpoint{1.949473in}{0.762796in}}%
\pgfpathcurveto{\pgfqpoint{1.941769in}{0.770500in}}{\pgfqpoint{1.931318in}{0.774829in}}{\pgfqpoint{1.920423in}{0.774829in}}%
\pgfpathcurveto{\pgfqpoint{1.909527in}{0.774829in}}{\pgfqpoint{1.899076in}{0.770500in}}{\pgfqpoint{1.891372in}{0.762796in}}%
\pgfpathcurveto{\pgfqpoint{1.883668in}{0.755092in}}{\pgfqpoint{1.879339in}{0.744641in}}{\pgfqpoint{1.879339in}{0.733745in}}%
\pgfpathcurveto{\pgfqpoint{1.879339in}{0.722850in}}{\pgfqpoint{1.883668in}{0.712399in}}{\pgfqpoint{1.891372in}{0.704695in}}%
\pgfpathcurveto{\pgfqpoint{1.899076in}{0.696990in}}{\pgfqpoint{1.909527in}{0.692661in}}{\pgfqpoint{1.920423in}{0.692661in}}%
\pgfpathlineto{\pgfqpoint{1.920423in}{0.692661in}}%
\pgfpathclose%
\pgfusepath{stroke}%
\end{pgfscope}%
\begin{pgfscope}%
\pgfpathrectangle{\pgfqpoint{0.688192in}{0.670138in}}{\pgfqpoint{7.111808in}{5.129862in}}%
\pgfusepath{clip}%
\pgfsetbuttcap%
\pgfsetroundjoin%
\pgfsetlinewidth{1.003750pt}%
\definecolor{currentstroke}{rgb}{0.000000,0.000000,0.000000}%
\pgfsetstrokecolor{currentstroke}%
\pgfsetdash{}{0pt}%
\pgfpathmoveto{\pgfqpoint{0.719167in}{1.957052in}}%
\pgfpathcurveto{\pgfqpoint{0.730063in}{1.957052in}}{\pgfqpoint{0.740514in}{1.961381in}}{\pgfqpoint{0.748218in}{1.969085in}}%
\pgfpathcurveto{\pgfqpoint{0.755922in}{1.976789in}}{\pgfqpoint{0.760251in}{1.987240in}}{\pgfqpoint{0.760251in}{1.998136in}}%
\pgfpathcurveto{\pgfqpoint{0.760251in}{2.009031in}}{\pgfqpoint{0.755922in}{2.019482in}}{\pgfqpoint{0.748218in}{2.027186in}}%
\pgfpathcurveto{\pgfqpoint{0.740514in}{2.034891in}}{\pgfqpoint{0.730063in}{2.039220in}}{\pgfqpoint{0.719167in}{2.039220in}}%
\pgfpathcurveto{\pgfqpoint{0.708272in}{2.039220in}}{\pgfqpoint{0.697821in}{2.034891in}}{\pgfqpoint{0.690117in}{2.027186in}}%
\pgfpathcurveto{\pgfqpoint{0.682412in}{2.019482in}}{\pgfqpoint{0.678084in}{2.009031in}}{\pgfqpoint{0.678084in}{1.998136in}}%
\pgfpathcurveto{\pgfqpoint{0.678084in}{1.987240in}}{\pgfqpoint{0.682412in}{1.976789in}}{\pgfqpoint{0.690117in}{1.969085in}}%
\pgfpathcurveto{\pgfqpoint{0.697821in}{1.961381in}}{\pgfqpoint{0.708272in}{1.957052in}}{\pgfqpoint{0.719167in}{1.957052in}}%
\pgfpathlineto{\pgfqpoint{0.719167in}{1.957052in}}%
\pgfpathclose%
\pgfusepath{stroke}%
\end{pgfscope}%
\begin{pgfscope}%
\pgfpathrectangle{\pgfqpoint{0.688192in}{0.670138in}}{\pgfqpoint{7.111808in}{5.129862in}}%
\pgfusepath{clip}%
\pgfsetbuttcap%
\pgfsetroundjoin%
\pgfsetlinewidth{1.003750pt}%
\definecolor{currentstroke}{rgb}{0.000000,0.000000,0.000000}%
\pgfsetstrokecolor{currentstroke}%
\pgfsetdash{}{0pt}%
\pgfpathmoveto{\pgfqpoint{2.443271in}{0.678334in}}%
\pgfpathcurveto{\pgfqpoint{2.454167in}{0.678334in}}{\pgfqpoint{2.464618in}{0.682663in}}{\pgfqpoint{2.472322in}{0.690367in}}%
\pgfpathcurveto{\pgfqpoint{2.480026in}{0.698072in}}{\pgfqpoint{2.484355in}{0.708523in}}{\pgfqpoint{2.484355in}{0.719418in}}%
\pgfpathcurveto{\pgfqpoint{2.484355in}{0.730314in}}{\pgfqpoint{2.480026in}{0.740764in}}{\pgfqpoint{2.472322in}{0.748469in}}%
\pgfpathcurveto{\pgfqpoint{2.464618in}{0.756173in}}{\pgfqpoint{2.454167in}{0.760502in}}{\pgfqpoint{2.443271in}{0.760502in}}%
\pgfpathcurveto{\pgfqpoint{2.432376in}{0.760502in}}{\pgfqpoint{2.421925in}{0.756173in}}{\pgfqpoint{2.414221in}{0.748469in}}%
\pgfpathcurveto{\pgfqpoint{2.406516in}{0.740764in}}{\pgfqpoint{2.402188in}{0.730314in}}{\pgfqpoint{2.402188in}{0.719418in}}%
\pgfpathcurveto{\pgfqpoint{2.402188in}{0.708523in}}{\pgfqpoint{2.406516in}{0.698072in}}{\pgfqpoint{2.414221in}{0.690367in}}%
\pgfpathcurveto{\pgfqpoint{2.421925in}{0.682663in}}{\pgfqpoint{2.432376in}{0.678334in}}{\pgfqpoint{2.443271in}{0.678334in}}%
\pgfpathlineto{\pgfqpoint{2.443271in}{0.678334in}}%
\pgfpathclose%
\pgfusepath{stroke}%
\end{pgfscope}%
\begin{pgfscope}%
\pgfpathrectangle{\pgfqpoint{0.688192in}{0.670138in}}{\pgfqpoint{7.111808in}{5.129862in}}%
\pgfusepath{clip}%
\pgfsetbuttcap%
\pgfsetroundjoin%
\pgfsetlinewidth{1.003750pt}%
\definecolor{currentstroke}{rgb}{0.000000,0.000000,0.000000}%
\pgfsetstrokecolor{currentstroke}%
\pgfsetdash{}{0pt}%
\pgfpathmoveto{\pgfqpoint{1.852165in}{0.694984in}}%
\pgfpathcurveto{\pgfqpoint{1.863060in}{0.694984in}}{\pgfqpoint{1.873511in}{0.699313in}}{\pgfqpoint{1.881215in}{0.707017in}}%
\pgfpathcurveto{\pgfqpoint{1.888920in}{0.714722in}}{\pgfqpoint{1.893248in}{0.725173in}}{\pgfqpoint{1.893248in}{0.736068in}}%
\pgfpathcurveto{\pgfqpoint{1.893248in}{0.746964in}}{\pgfqpoint{1.888920in}{0.757415in}}{\pgfqpoint{1.881215in}{0.765119in}}%
\pgfpathcurveto{\pgfqpoint{1.873511in}{0.772823in}}{\pgfqpoint{1.863060in}{0.777152in}}{\pgfqpoint{1.852165in}{0.777152in}}%
\pgfpathcurveto{\pgfqpoint{1.841269in}{0.777152in}}{\pgfqpoint{1.830818in}{0.772823in}}{\pgfqpoint{1.823114in}{0.765119in}}%
\pgfpathcurveto{\pgfqpoint{1.815410in}{0.757415in}}{\pgfqpoint{1.811081in}{0.746964in}}{\pgfqpoint{1.811081in}{0.736068in}}%
\pgfpathcurveto{\pgfqpoint{1.811081in}{0.725173in}}{\pgfqpoint{1.815410in}{0.714722in}}{\pgfqpoint{1.823114in}{0.707017in}}%
\pgfpathcurveto{\pgfqpoint{1.830818in}{0.699313in}}{\pgfqpoint{1.841269in}{0.694984in}}{\pgfqpoint{1.852165in}{0.694984in}}%
\pgfpathlineto{\pgfqpoint{1.852165in}{0.694984in}}%
\pgfpathclose%
\pgfusepath{stroke}%
\end{pgfscope}%
\begin{pgfscope}%
\pgfpathrectangle{\pgfqpoint{0.688192in}{0.670138in}}{\pgfqpoint{7.111808in}{5.129862in}}%
\pgfusepath{clip}%
\pgfsetbuttcap%
\pgfsetroundjoin%
\pgfsetlinewidth{1.003750pt}%
\definecolor{currentstroke}{rgb}{0.000000,0.000000,0.000000}%
\pgfsetstrokecolor{currentstroke}%
\pgfsetdash{}{0pt}%
\pgfpathmoveto{\pgfqpoint{0.901363in}{0.856548in}}%
\pgfpathcurveto{\pgfqpoint{0.912259in}{0.856548in}}{\pgfqpoint{0.922710in}{0.860877in}}{\pgfqpoint{0.930414in}{0.868582in}}%
\pgfpathcurveto{\pgfqpoint{0.938118in}{0.876286in}}{\pgfqpoint{0.942447in}{0.886737in}}{\pgfqpoint{0.942447in}{0.897632in}}%
\pgfpathcurveto{\pgfqpoint{0.942447in}{0.908528in}}{\pgfqpoint{0.938118in}{0.918979in}}{\pgfqpoint{0.930414in}{0.926683in}}%
\pgfpathcurveto{\pgfqpoint{0.922710in}{0.934387in}}{\pgfqpoint{0.912259in}{0.938716in}}{\pgfqpoint{0.901363in}{0.938716in}}%
\pgfpathcurveto{\pgfqpoint{0.890468in}{0.938716in}}{\pgfqpoint{0.880017in}{0.934387in}}{\pgfqpoint{0.872313in}{0.926683in}}%
\pgfpathcurveto{\pgfqpoint{0.864608in}{0.918979in}}{\pgfqpoint{0.860280in}{0.908528in}}{\pgfqpoint{0.860280in}{0.897632in}}%
\pgfpathcurveto{\pgfqpoint{0.860280in}{0.886737in}}{\pgfqpoint{0.864608in}{0.876286in}}{\pgfqpoint{0.872313in}{0.868582in}}%
\pgfpathcurveto{\pgfqpoint{0.880017in}{0.860877in}}{\pgfqpoint{0.890468in}{0.856548in}}{\pgfqpoint{0.901363in}{0.856548in}}%
\pgfpathlineto{\pgfqpoint{0.901363in}{0.856548in}}%
\pgfpathclose%
\pgfusepath{stroke}%
\end{pgfscope}%
\begin{pgfscope}%
\pgfpathrectangle{\pgfqpoint{0.688192in}{0.670138in}}{\pgfqpoint{7.111808in}{5.129862in}}%
\pgfusepath{clip}%
\pgfsetbuttcap%
\pgfsetroundjoin%
\pgfsetlinewidth{1.003750pt}%
\definecolor{currentstroke}{rgb}{0.000000,0.000000,0.000000}%
\pgfsetstrokecolor{currentstroke}%
\pgfsetdash{}{0pt}%
\pgfpathmoveto{\pgfqpoint{1.920423in}{0.692661in}}%
\pgfpathcurveto{\pgfqpoint{1.931318in}{0.692661in}}{\pgfqpoint{1.941769in}{0.696990in}}{\pgfqpoint{1.949473in}{0.704695in}}%
\pgfpathcurveto{\pgfqpoint{1.957178in}{0.712399in}}{\pgfqpoint{1.961506in}{0.722850in}}{\pgfqpoint{1.961506in}{0.733745in}}%
\pgfpathcurveto{\pgfqpoint{1.961506in}{0.744641in}}{\pgfqpoint{1.957178in}{0.755092in}}{\pgfqpoint{1.949473in}{0.762796in}}%
\pgfpathcurveto{\pgfqpoint{1.941769in}{0.770500in}}{\pgfqpoint{1.931318in}{0.774829in}}{\pgfqpoint{1.920423in}{0.774829in}}%
\pgfpathcurveto{\pgfqpoint{1.909527in}{0.774829in}}{\pgfqpoint{1.899076in}{0.770500in}}{\pgfqpoint{1.891372in}{0.762796in}}%
\pgfpathcurveto{\pgfqpoint{1.883668in}{0.755092in}}{\pgfqpoint{1.879339in}{0.744641in}}{\pgfqpoint{1.879339in}{0.733745in}}%
\pgfpathcurveto{\pgfqpoint{1.879339in}{0.722850in}}{\pgfqpoint{1.883668in}{0.712399in}}{\pgfqpoint{1.891372in}{0.704695in}}%
\pgfpathcurveto{\pgfqpoint{1.899076in}{0.696990in}}{\pgfqpoint{1.909527in}{0.692661in}}{\pgfqpoint{1.920423in}{0.692661in}}%
\pgfpathlineto{\pgfqpoint{1.920423in}{0.692661in}}%
\pgfpathclose%
\pgfusepath{stroke}%
\end{pgfscope}%
\begin{pgfscope}%
\pgfpathrectangle{\pgfqpoint{0.688192in}{0.670138in}}{\pgfqpoint{7.111808in}{5.129862in}}%
\pgfusepath{clip}%
\pgfsetbuttcap%
\pgfsetroundjoin%
\pgfsetlinewidth{1.003750pt}%
\definecolor{currentstroke}{rgb}{0.000000,0.000000,0.000000}%
\pgfsetstrokecolor{currentstroke}%
\pgfsetdash{}{0pt}%
\pgfpathmoveto{\pgfqpoint{1.070830in}{0.725886in}}%
\pgfpathcurveto{\pgfqpoint{1.081726in}{0.725886in}}{\pgfqpoint{1.092176in}{0.730215in}}{\pgfqpoint{1.099881in}{0.737920in}}%
\pgfpathcurveto{\pgfqpoint{1.107585in}{0.745624in}}{\pgfqpoint{1.111914in}{0.756075in}}{\pgfqpoint{1.111914in}{0.766970in}}%
\pgfpathcurveto{\pgfqpoint{1.111914in}{0.777866in}}{\pgfqpoint{1.107585in}{0.788317in}}{\pgfqpoint{1.099881in}{0.796021in}}%
\pgfpathcurveto{\pgfqpoint{1.092176in}{0.803725in}}{\pgfqpoint{1.081726in}{0.808054in}}{\pgfqpoint{1.070830in}{0.808054in}}%
\pgfpathcurveto{\pgfqpoint{1.059934in}{0.808054in}}{\pgfqpoint{1.049484in}{0.803725in}}{\pgfqpoint{1.041779in}{0.796021in}}%
\pgfpathcurveto{\pgfqpoint{1.034075in}{0.788317in}}{\pgfqpoint{1.029746in}{0.777866in}}{\pgfqpoint{1.029746in}{0.766970in}}%
\pgfpathcurveto{\pgfqpoint{1.029746in}{0.756075in}}{\pgfqpoint{1.034075in}{0.745624in}}{\pgfqpoint{1.041779in}{0.737920in}}%
\pgfpathcurveto{\pgfqpoint{1.049484in}{0.730215in}}{\pgfqpoint{1.059934in}{0.725886in}}{\pgfqpoint{1.070830in}{0.725886in}}%
\pgfpathlineto{\pgfqpoint{1.070830in}{0.725886in}}%
\pgfpathclose%
\pgfusepath{stroke}%
\end{pgfscope}%
\begin{pgfscope}%
\pgfpathrectangle{\pgfqpoint{0.688192in}{0.670138in}}{\pgfqpoint{7.111808in}{5.129862in}}%
\pgfusepath{clip}%
\pgfsetbuttcap%
\pgfsetroundjoin%
\pgfsetlinewidth{1.003750pt}%
\definecolor{currentstroke}{rgb}{0.000000,0.000000,0.000000}%
\pgfsetstrokecolor{currentstroke}%
\pgfsetdash{}{0pt}%
\pgfpathmoveto{\pgfqpoint{5.033938in}{1.555601in}}%
\pgfpathcurveto{\pgfqpoint{5.044833in}{1.555601in}}{\pgfqpoint{5.055284in}{1.559930in}}{\pgfqpoint{5.062988in}{1.567634in}}%
\pgfpathcurveto{\pgfqpoint{5.070693in}{1.575338in}}{\pgfqpoint{5.075022in}{1.585789in}}{\pgfqpoint{5.075022in}{1.596685in}}%
\pgfpathcurveto{\pgfqpoint{5.075022in}{1.607580in}}{\pgfqpoint{5.070693in}{1.618031in}}{\pgfqpoint{5.062988in}{1.625736in}}%
\pgfpathcurveto{\pgfqpoint{5.055284in}{1.633440in}}{\pgfqpoint{5.044833in}{1.637769in}}{\pgfqpoint{5.033938in}{1.637769in}}%
\pgfpathcurveto{\pgfqpoint{5.023042in}{1.637769in}}{\pgfqpoint{5.012591in}{1.633440in}}{\pgfqpoint{5.004887in}{1.625736in}}%
\pgfpathcurveto{\pgfqpoint{4.997183in}{1.618031in}}{\pgfqpoint{4.992854in}{1.607580in}}{\pgfqpoint{4.992854in}{1.596685in}}%
\pgfpathcurveto{\pgfqpoint{4.992854in}{1.585789in}}{\pgfqpoint{4.997183in}{1.575338in}}{\pgfqpoint{5.004887in}{1.567634in}}%
\pgfpathcurveto{\pgfqpoint{5.012591in}{1.559930in}}{\pgfqpoint{5.023042in}{1.555601in}}{\pgfqpoint{5.033938in}{1.555601in}}%
\pgfpathlineto{\pgfqpoint{5.033938in}{1.555601in}}%
\pgfpathclose%
\pgfusepath{stroke}%
\end{pgfscope}%
\begin{pgfscope}%
\pgfpathrectangle{\pgfqpoint{0.688192in}{0.670138in}}{\pgfqpoint{7.111808in}{5.129862in}}%
\pgfusepath{clip}%
\pgfsetbuttcap%
\pgfsetroundjoin%
\pgfsetlinewidth{1.003750pt}%
\definecolor{currentstroke}{rgb}{0.000000,0.000000,0.000000}%
\pgfsetstrokecolor{currentstroke}%
\pgfsetdash{}{0pt}%
\pgfpathmoveto{\pgfqpoint{0.899610in}{0.859043in}}%
\pgfpathcurveto{\pgfqpoint{0.910506in}{0.859043in}}{\pgfqpoint{0.920957in}{0.863371in}}{\pgfqpoint{0.928661in}{0.871076in}}%
\pgfpathcurveto{\pgfqpoint{0.936365in}{0.878780in}}{\pgfqpoint{0.940694in}{0.889231in}}{\pgfqpoint{0.940694in}{0.900126in}}%
\pgfpathcurveto{\pgfqpoint{0.940694in}{0.911022in}}{\pgfqpoint{0.936365in}{0.921473in}}{\pgfqpoint{0.928661in}{0.929177in}}%
\pgfpathcurveto{\pgfqpoint{0.920957in}{0.936881in}}{\pgfqpoint{0.910506in}{0.941210in}}{\pgfqpoint{0.899610in}{0.941210in}}%
\pgfpathcurveto{\pgfqpoint{0.888715in}{0.941210in}}{\pgfqpoint{0.878264in}{0.936881in}}{\pgfqpoint{0.870560in}{0.929177in}}%
\pgfpathcurveto{\pgfqpoint{0.862855in}{0.921473in}}{\pgfqpoint{0.858526in}{0.911022in}}{\pgfqpoint{0.858526in}{0.900126in}}%
\pgfpathcurveto{\pgfqpoint{0.858526in}{0.889231in}}{\pgfqpoint{0.862855in}{0.878780in}}{\pgfqpoint{0.870560in}{0.871076in}}%
\pgfpathcurveto{\pgfqpoint{0.878264in}{0.863371in}}{\pgfqpoint{0.888715in}{0.859043in}}{\pgfqpoint{0.899610in}{0.859043in}}%
\pgfpathlineto{\pgfqpoint{0.899610in}{0.859043in}}%
\pgfpathclose%
\pgfusepath{stroke}%
\end{pgfscope}%
\begin{pgfscope}%
\pgfpathrectangle{\pgfqpoint{0.688192in}{0.670138in}}{\pgfqpoint{7.111808in}{5.129862in}}%
\pgfusepath{clip}%
\pgfsetbuttcap%
\pgfsetroundjoin%
\pgfsetlinewidth{1.003750pt}%
\definecolor{currentstroke}{rgb}{0.000000,0.000000,0.000000}%
\pgfsetstrokecolor{currentstroke}%
\pgfsetdash{}{0pt}%
\pgfpathmoveto{\pgfqpoint{2.497167in}{0.675229in}}%
\pgfpathcurveto{\pgfqpoint{2.508063in}{0.675229in}}{\pgfqpoint{2.518514in}{0.679557in}}{\pgfqpoint{2.526218in}{0.687262in}}%
\pgfpathcurveto{\pgfqpoint{2.533922in}{0.694966in}}{\pgfqpoint{2.538251in}{0.705417in}}{\pgfqpoint{2.538251in}{0.716312in}}%
\pgfpathcurveto{\pgfqpoint{2.538251in}{0.727208in}}{\pgfqpoint{2.533922in}{0.737659in}}{\pgfqpoint{2.526218in}{0.745363in}}%
\pgfpathcurveto{\pgfqpoint{2.518514in}{0.753067in}}{\pgfqpoint{2.508063in}{0.757396in}}{\pgfqpoint{2.497167in}{0.757396in}}%
\pgfpathcurveto{\pgfqpoint{2.486272in}{0.757396in}}{\pgfqpoint{2.475821in}{0.753067in}}{\pgfqpoint{2.468117in}{0.745363in}}%
\pgfpathcurveto{\pgfqpoint{2.460412in}{0.737659in}}{\pgfqpoint{2.456084in}{0.727208in}}{\pgfqpoint{2.456084in}{0.716312in}}%
\pgfpathcurveto{\pgfqpoint{2.456084in}{0.705417in}}{\pgfqpoint{2.460412in}{0.694966in}}{\pgfqpoint{2.468117in}{0.687262in}}%
\pgfpathcurveto{\pgfqpoint{2.475821in}{0.679557in}}{\pgfqpoint{2.486272in}{0.675229in}}{\pgfqpoint{2.497167in}{0.675229in}}%
\pgfpathlineto{\pgfqpoint{2.497167in}{0.675229in}}%
\pgfpathclose%
\pgfusepath{stroke}%
\end{pgfscope}%
\begin{pgfscope}%
\pgfpathrectangle{\pgfqpoint{0.688192in}{0.670138in}}{\pgfqpoint{7.111808in}{5.129862in}}%
\pgfusepath{clip}%
\pgfsetbuttcap%
\pgfsetroundjoin%
\pgfsetlinewidth{1.003750pt}%
\definecolor{currentstroke}{rgb}{0.000000,0.000000,0.000000}%
\pgfsetstrokecolor{currentstroke}%
\pgfsetdash{}{0pt}%
\pgfpathmoveto{\pgfqpoint{1.019151in}{0.759981in}}%
\pgfpathcurveto{\pgfqpoint{1.030047in}{0.759981in}}{\pgfqpoint{1.040498in}{0.764309in}}{\pgfqpoint{1.048202in}{0.772014in}}%
\pgfpathcurveto{\pgfqpoint{1.055906in}{0.779718in}}{\pgfqpoint{1.060235in}{0.790169in}}{\pgfqpoint{1.060235in}{0.801064in}}%
\pgfpathcurveto{\pgfqpoint{1.060235in}{0.811960in}}{\pgfqpoint{1.055906in}{0.822411in}}{\pgfqpoint{1.048202in}{0.830115in}}%
\pgfpathcurveto{\pgfqpoint{1.040498in}{0.837819in}}{\pgfqpoint{1.030047in}{0.842148in}}{\pgfqpoint{1.019151in}{0.842148in}}%
\pgfpathcurveto{\pgfqpoint{1.008256in}{0.842148in}}{\pgfqpoint{0.997805in}{0.837819in}}{\pgfqpoint{0.990101in}{0.830115in}}%
\pgfpathcurveto{\pgfqpoint{0.982396in}{0.822411in}}{\pgfqpoint{0.978067in}{0.811960in}}{\pgfqpoint{0.978067in}{0.801064in}}%
\pgfpathcurveto{\pgfqpoint{0.978067in}{0.790169in}}{\pgfqpoint{0.982396in}{0.779718in}}{\pgfqpoint{0.990101in}{0.772014in}}%
\pgfpathcurveto{\pgfqpoint{0.997805in}{0.764309in}}{\pgfqpoint{1.008256in}{0.759981in}}{\pgfqpoint{1.019151in}{0.759981in}}%
\pgfpathlineto{\pgfqpoint{1.019151in}{0.759981in}}%
\pgfpathclose%
\pgfusepath{stroke}%
\end{pgfscope}%
\begin{pgfscope}%
\pgfpathrectangle{\pgfqpoint{0.688192in}{0.670138in}}{\pgfqpoint{7.111808in}{5.129862in}}%
\pgfusepath{clip}%
\pgfsetbuttcap%
\pgfsetroundjoin%
\pgfsetlinewidth{1.003750pt}%
\definecolor{currentstroke}{rgb}{0.000000,0.000000,0.000000}%
\pgfsetstrokecolor{currentstroke}%
\pgfsetdash{}{0pt}%
\pgfpathmoveto{\pgfqpoint{0.931763in}{0.833664in}}%
\pgfpathcurveto{\pgfqpoint{0.942659in}{0.833664in}}{\pgfqpoint{0.953110in}{0.837992in}}{\pgfqpoint{0.960814in}{0.845697in}}%
\pgfpathcurveto{\pgfqpoint{0.968519in}{0.853401in}}{\pgfqpoint{0.972847in}{0.863852in}}{\pgfqpoint{0.972847in}{0.874747in}}%
\pgfpathcurveto{\pgfqpoint{0.972847in}{0.885643in}}{\pgfqpoint{0.968519in}{0.896094in}}{\pgfqpoint{0.960814in}{0.903798in}}%
\pgfpathcurveto{\pgfqpoint{0.953110in}{0.911502in}}{\pgfqpoint{0.942659in}{0.915831in}}{\pgfqpoint{0.931763in}{0.915831in}}%
\pgfpathcurveto{\pgfqpoint{0.920868in}{0.915831in}}{\pgfqpoint{0.910417in}{0.911502in}}{\pgfqpoint{0.902713in}{0.903798in}}%
\pgfpathcurveto{\pgfqpoint{0.895008in}{0.896094in}}{\pgfqpoint{0.890680in}{0.885643in}}{\pgfqpoint{0.890680in}{0.874747in}}%
\pgfpathcurveto{\pgfqpoint{0.890680in}{0.863852in}}{\pgfqpoint{0.895008in}{0.853401in}}{\pgfqpoint{0.902713in}{0.845697in}}%
\pgfpathcurveto{\pgfqpoint{0.910417in}{0.837992in}}{\pgfqpoint{0.920868in}{0.833664in}}{\pgfqpoint{0.931763in}{0.833664in}}%
\pgfpathlineto{\pgfqpoint{0.931763in}{0.833664in}}%
\pgfpathclose%
\pgfusepath{stroke}%
\end{pgfscope}%
\begin{pgfscope}%
\pgfpathrectangle{\pgfqpoint{0.688192in}{0.670138in}}{\pgfqpoint{7.111808in}{5.129862in}}%
\pgfusepath{clip}%
\pgfsetbuttcap%
\pgfsetroundjoin%
\pgfsetlinewidth{1.003750pt}%
\definecolor{currentstroke}{rgb}{0.000000,0.000000,0.000000}%
\pgfsetstrokecolor{currentstroke}%
\pgfsetdash{}{0pt}%
\pgfpathmoveto{\pgfqpoint{0.782559in}{1.071983in}}%
\pgfpathcurveto{\pgfqpoint{0.793455in}{1.071983in}}{\pgfqpoint{0.803905in}{1.076312in}}{\pgfqpoint{0.811610in}{1.084017in}}%
\pgfpathcurveto{\pgfqpoint{0.819314in}{1.091721in}}{\pgfqpoint{0.823643in}{1.102172in}}{\pgfqpoint{0.823643in}{1.113067in}}%
\pgfpathcurveto{\pgfqpoint{0.823643in}{1.123963in}}{\pgfqpoint{0.819314in}{1.134414in}}{\pgfqpoint{0.811610in}{1.142118in}}%
\pgfpathcurveto{\pgfqpoint{0.803905in}{1.149822in}}{\pgfqpoint{0.793455in}{1.154151in}}{\pgfqpoint{0.782559in}{1.154151in}}%
\pgfpathcurveto{\pgfqpoint{0.771663in}{1.154151in}}{\pgfqpoint{0.761213in}{1.149822in}}{\pgfqpoint{0.753508in}{1.142118in}}%
\pgfpathcurveto{\pgfqpoint{0.745804in}{1.134414in}}{\pgfqpoint{0.741475in}{1.123963in}}{\pgfqpoint{0.741475in}{1.113067in}}%
\pgfpathcurveto{\pgfqpoint{0.741475in}{1.102172in}}{\pgfqpoint{0.745804in}{1.091721in}}{\pgfqpoint{0.753508in}{1.084017in}}%
\pgfpathcurveto{\pgfqpoint{0.761213in}{1.076312in}}{\pgfqpoint{0.771663in}{1.071983in}}{\pgfqpoint{0.782559in}{1.071983in}}%
\pgfpathlineto{\pgfqpoint{0.782559in}{1.071983in}}%
\pgfpathclose%
\pgfusepath{stroke}%
\end{pgfscope}%
\begin{pgfscope}%
\pgfpathrectangle{\pgfqpoint{0.688192in}{0.670138in}}{\pgfqpoint{7.111808in}{5.129862in}}%
\pgfusepath{clip}%
\pgfsetbuttcap%
\pgfsetroundjoin%
\pgfsetlinewidth{1.003750pt}%
\definecolor{currentstroke}{rgb}{0.000000,0.000000,0.000000}%
\pgfsetstrokecolor{currentstroke}%
\pgfsetdash{}{0pt}%
\pgfpathmoveto{\pgfqpoint{5.175445in}{0.803065in}}%
\pgfpathcurveto{\pgfqpoint{5.186341in}{0.803065in}}{\pgfqpoint{5.196792in}{0.807394in}}{\pgfqpoint{5.204496in}{0.815099in}}%
\pgfpathcurveto{\pgfqpoint{5.212200in}{0.822803in}}{\pgfqpoint{5.216529in}{0.833254in}}{\pgfqpoint{5.216529in}{0.844149in}}%
\pgfpathcurveto{\pgfqpoint{5.216529in}{0.855045in}}{\pgfqpoint{5.212200in}{0.865496in}}{\pgfqpoint{5.204496in}{0.873200in}}%
\pgfpathcurveto{\pgfqpoint{5.196792in}{0.880904in}}{\pgfqpoint{5.186341in}{0.885233in}}{\pgfqpoint{5.175445in}{0.885233in}}%
\pgfpathcurveto{\pgfqpoint{5.164550in}{0.885233in}}{\pgfqpoint{5.154099in}{0.880904in}}{\pgfqpoint{5.146395in}{0.873200in}}%
\pgfpathcurveto{\pgfqpoint{5.138690in}{0.865496in}}{\pgfqpoint{5.134361in}{0.855045in}}{\pgfqpoint{5.134361in}{0.844149in}}%
\pgfpathcurveto{\pgfqpoint{5.134361in}{0.833254in}}{\pgfqpoint{5.138690in}{0.822803in}}{\pgfqpoint{5.146395in}{0.815099in}}%
\pgfpathcurveto{\pgfqpoint{5.154099in}{0.807394in}}{\pgfqpoint{5.164550in}{0.803065in}}{\pgfqpoint{5.175445in}{0.803065in}}%
\pgfpathlineto{\pgfqpoint{5.175445in}{0.803065in}}%
\pgfpathclose%
\pgfusepath{stroke}%
\end{pgfscope}%
\begin{pgfscope}%
\pgfpathrectangle{\pgfqpoint{0.688192in}{0.670138in}}{\pgfqpoint{7.111808in}{5.129862in}}%
\pgfusepath{clip}%
\pgfsetbuttcap%
\pgfsetroundjoin%
\pgfsetlinewidth{1.003750pt}%
\definecolor{currentstroke}{rgb}{0.000000,0.000000,0.000000}%
\pgfsetstrokecolor{currentstroke}%
\pgfsetdash{}{0pt}%
\pgfpathmoveto{\pgfqpoint{1.308402in}{0.713449in}}%
\pgfpathcurveto{\pgfqpoint{1.319298in}{0.713449in}}{\pgfqpoint{1.329748in}{0.717778in}}{\pgfqpoint{1.337453in}{0.725482in}}%
\pgfpathcurveto{\pgfqpoint{1.345157in}{0.733187in}}{\pgfqpoint{1.349486in}{0.743637in}}{\pgfqpoint{1.349486in}{0.754533in}}%
\pgfpathcurveto{\pgfqpoint{1.349486in}{0.765429in}}{\pgfqpoint{1.345157in}{0.775879in}}{\pgfqpoint{1.337453in}{0.783584in}}%
\pgfpathcurveto{\pgfqpoint{1.329748in}{0.791288in}}{\pgfqpoint{1.319298in}{0.795617in}}{\pgfqpoint{1.308402in}{0.795617in}}%
\pgfpathcurveto{\pgfqpoint{1.297506in}{0.795617in}}{\pgfqpoint{1.287056in}{0.791288in}}{\pgfqpoint{1.279351in}{0.783584in}}%
\pgfpathcurveto{\pgfqpoint{1.271647in}{0.775879in}}{\pgfqpoint{1.267318in}{0.765429in}}{\pgfqpoint{1.267318in}{0.754533in}}%
\pgfpathcurveto{\pgfqpoint{1.267318in}{0.743637in}}{\pgfqpoint{1.271647in}{0.733187in}}{\pgfqpoint{1.279351in}{0.725482in}}%
\pgfpathcurveto{\pgfqpoint{1.287056in}{0.717778in}}{\pgfqpoint{1.297506in}{0.713449in}}{\pgfqpoint{1.308402in}{0.713449in}}%
\pgfpathlineto{\pgfqpoint{1.308402in}{0.713449in}}%
\pgfpathclose%
\pgfusepath{stroke}%
\end{pgfscope}%
\begin{pgfscope}%
\pgfpathrectangle{\pgfqpoint{0.688192in}{0.670138in}}{\pgfqpoint{7.111808in}{5.129862in}}%
\pgfusepath{clip}%
\pgfsetbuttcap%
\pgfsetroundjoin%
\pgfsetlinewidth{1.003750pt}%
\definecolor{currentstroke}{rgb}{0.000000,0.000000,0.000000}%
\pgfsetstrokecolor{currentstroke}%
\pgfsetdash{}{0pt}%
\pgfpathmoveto{\pgfqpoint{1.045306in}{0.741536in}}%
\pgfpathcurveto{\pgfqpoint{1.056202in}{0.741536in}}{\pgfqpoint{1.066653in}{0.745865in}}{\pgfqpoint{1.074357in}{0.753569in}}%
\pgfpathcurveto{\pgfqpoint{1.082061in}{0.761274in}}{\pgfqpoint{1.086390in}{0.771725in}}{\pgfqpoint{1.086390in}{0.782620in}}%
\pgfpathcurveto{\pgfqpoint{1.086390in}{0.793516in}}{\pgfqpoint{1.082061in}{0.803967in}}{\pgfqpoint{1.074357in}{0.811671in}}%
\pgfpathcurveto{\pgfqpoint{1.066653in}{0.819375in}}{\pgfqpoint{1.056202in}{0.823704in}}{\pgfqpoint{1.045306in}{0.823704in}}%
\pgfpathcurveto{\pgfqpoint{1.034411in}{0.823704in}}{\pgfqpoint{1.023960in}{0.819375in}}{\pgfqpoint{1.016256in}{0.811671in}}%
\pgfpathcurveto{\pgfqpoint{1.008551in}{0.803967in}}{\pgfqpoint{1.004222in}{0.793516in}}{\pgfqpoint{1.004222in}{0.782620in}}%
\pgfpathcurveto{\pgfqpoint{1.004222in}{0.771725in}}{\pgfqpoint{1.008551in}{0.761274in}}{\pgfqpoint{1.016256in}{0.753569in}}%
\pgfpathcurveto{\pgfqpoint{1.023960in}{0.745865in}}{\pgfqpoint{1.034411in}{0.741536in}}{\pgfqpoint{1.045306in}{0.741536in}}%
\pgfpathlineto{\pgfqpoint{1.045306in}{0.741536in}}%
\pgfpathclose%
\pgfusepath{stroke}%
\end{pgfscope}%
\begin{pgfscope}%
\pgfpathrectangle{\pgfqpoint{0.688192in}{0.670138in}}{\pgfqpoint{7.111808in}{5.129862in}}%
\pgfusepath{clip}%
\pgfsetbuttcap%
\pgfsetroundjoin%
\pgfsetlinewidth{1.003750pt}%
\definecolor{currentstroke}{rgb}{0.000000,0.000000,0.000000}%
\pgfsetstrokecolor{currentstroke}%
\pgfsetdash{}{0pt}%
\pgfpathmoveto{\pgfqpoint{3.248690in}{4.902676in}}%
\pgfpathcurveto{\pgfqpoint{3.259585in}{4.902676in}}{\pgfqpoint{3.270036in}{4.907005in}}{\pgfqpoint{3.277741in}{4.914709in}}%
\pgfpathcurveto{\pgfqpoint{3.285445in}{4.922413in}}{\pgfqpoint{3.289774in}{4.932864in}}{\pgfqpoint{3.289774in}{4.943760in}}%
\pgfpathcurveto{\pgfqpoint{3.289774in}{4.954655in}}{\pgfqpoint{3.285445in}{4.965106in}}{\pgfqpoint{3.277741in}{4.972810in}}%
\pgfpathcurveto{\pgfqpoint{3.270036in}{4.980515in}}{\pgfqpoint{3.259585in}{4.984844in}}{\pgfqpoint{3.248690in}{4.984844in}}%
\pgfpathcurveto{\pgfqpoint{3.237794in}{4.984844in}}{\pgfqpoint{3.227343in}{4.980515in}}{\pgfqpoint{3.219639in}{4.972810in}}%
\pgfpathcurveto{\pgfqpoint{3.211935in}{4.965106in}}{\pgfqpoint{3.207606in}{4.954655in}}{\pgfqpoint{3.207606in}{4.943760in}}%
\pgfpathcurveto{\pgfqpoint{3.207606in}{4.932864in}}{\pgfqpoint{3.211935in}{4.922413in}}{\pgfqpoint{3.219639in}{4.914709in}}%
\pgfpathcurveto{\pgfqpoint{3.227343in}{4.907005in}}{\pgfqpoint{3.237794in}{4.902676in}}{\pgfqpoint{3.248690in}{4.902676in}}%
\pgfpathlineto{\pgfqpoint{3.248690in}{4.902676in}}%
\pgfpathclose%
\pgfusepath{stroke}%
\end{pgfscope}%
\begin{pgfscope}%
\pgfpathrectangle{\pgfqpoint{0.688192in}{0.670138in}}{\pgfqpoint{7.111808in}{5.129862in}}%
\pgfusepath{clip}%
\pgfsetbuttcap%
\pgfsetroundjoin%
\pgfsetlinewidth{1.003750pt}%
\definecolor{currentstroke}{rgb}{0.000000,0.000000,0.000000}%
\pgfsetstrokecolor{currentstroke}%
\pgfsetdash{}{0pt}%
\pgfpathmoveto{\pgfqpoint{6.281907in}{1.400901in}}%
\pgfpathcurveto{\pgfqpoint{6.292803in}{1.400901in}}{\pgfqpoint{6.303254in}{1.405230in}}{\pgfqpoint{6.310958in}{1.412934in}}%
\pgfpathcurveto{\pgfqpoint{6.318662in}{1.420639in}}{\pgfqpoint{6.322991in}{1.431090in}}{\pgfqpoint{6.322991in}{1.441985in}}%
\pgfpathcurveto{\pgfqpoint{6.322991in}{1.452881in}}{\pgfqpoint{6.318662in}{1.463331in}}{\pgfqpoint{6.310958in}{1.471036in}}%
\pgfpathcurveto{\pgfqpoint{6.303254in}{1.478740in}}{\pgfqpoint{6.292803in}{1.483069in}}{\pgfqpoint{6.281907in}{1.483069in}}%
\pgfpathcurveto{\pgfqpoint{6.271012in}{1.483069in}}{\pgfqpoint{6.260561in}{1.478740in}}{\pgfqpoint{6.252856in}{1.471036in}}%
\pgfpathcurveto{\pgfqpoint{6.245152in}{1.463331in}}{\pgfqpoint{6.240823in}{1.452881in}}{\pgfqpoint{6.240823in}{1.441985in}}%
\pgfpathcurveto{\pgfqpoint{6.240823in}{1.431090in}}{\pgfqpoint{6.245152in}{1.420639in}}{\pgfqpoint{6.252856in}{1.412934in}}%
\pgfpathcurveto{\pgfqpoint{6.260561in}{1.405230in}}{\pgfqpoint{6.271012in}{1.400901in}}{\pgfqpoint{6.281907in}{1.400901in}}%
\pgfpathlineto{\pgfqpoint{6.281907in}{1.400901in}}%
\pgfpathclose%
\pgfusepath{stroke}%
\end{pgfscope}%
\begin{pgfscope}%
\pgfpathrectangle{\pgfqpoint{0.688192in}{0.670138in}}{\pgfqpoint{7.111808in}{5.129862in}}%
\pgfusepath{clip}%
\pgfsetbuttcap%
\pgfsetroundjoin%
\pgfsetlinewidth{1.003750pt}%
\definecolor{currentstroke}{rgb}{0.000000,0.000000,0.000000}%
\pgfsetstrokecolor{currentstroke}%
\pgfsetdash{}{0pt}%
\pgfpathmoveto{\pgfqpoint{2.389781in}{3.802540in}}%
\pgfpathcurveto{\pgfqpoint{2.400676in}{3.802540in}}{\pgfqpoint{2.411127in}{3.806869in}}{\pgfqpoint{2.418831in}{3.814573in}}%
\pgfpathcurveto{\pgfqpoint{2.426536in}{3.822278in}}{\pgfqpoint{2.430864in}{3.832729in}}{\pgfqpoint{2.430864in}{3.843624in}}%
\pgfpathcurveto{\pgfqpoint{2.430864in}{3.854520in}}{\pgfqpoint{2.426536in}{3.864971in}}{\pgfqpoint{2.418831in}{3.872675in}}%
\pgfpathcurveto{\pgfqpoint{2.411127in}{3.880379in}}{\pgfqpoint{2.400676in}{3.884708in}}{\pgfqpoint{2.389781in}{3.884708in}}%
\pgfpathcurveto{\pgfqpoint{2.378885in}{3.884708in}}{\pgfqpoint{2.368434in}{3.880379in}}{\pgfqpoint{2.360730in}{3.872675in}}%
\pgfpathcurveto{\pgfqpoint{2.353026in}{3.864971in}}{\pgfqpoint{2.348697in}{3.854520in}}{\pgfqpoint{2.348697in}{3.843624in}}%
\pgfpathcurveto{\pgfqpoint{2.348697in}{3.832729in}}{\pgfqpoint{2.353026in}{3.822278in}}{\pgfqpoint{2.360730in}{3.814573in}}%
\pgfpathcurveto{\pgfqpoint{2.368434in}{3.806869in}}{\pgfqpoint{2.378885in}{3.802540in}}{\pgfqpoint{2.389781in}{3.802540in}}%
\pgfpathlineto{\pgfqpoint{2.389781in}{3.802540in}}%
\pgfpathclose%
\pgfusepath{stroke}%
\end{pgfscope}%
\begin{pgfscope}%
\pgfpathrectangle{\pgfqpoint{0.688192in}{0.670138in}}{\pgfqpoint{7.111808in}{5.129862in}}%
\pgfusepath{clip}%
\pgfsetbuttcap%
\pgfsetroundjoin%
\pgfsetlinewidth{1.003750pt}%
\definecolor{currentstroke}{rgb}{0.000000,0.000000,0.000000}%
\pgfsetstrokecolor{currentstroke}%
\pgfsetdash{}{0pt}%
\pgfpathmoveto{\pgfqpoint{1.707667in}{0.699316in}}%
\pgfpathcurveto{\pgfqpoint{1.718562in}{0.699316in}}{\pgfqpoint{1.729013in}{0.703645in}}{\pgfqpoint{1.736717in}{0.711350in}}%
\pgfpathcurveto{\pgfqpoint{1.744422in}{0.719054in}}{\pgfqpoint{1.748751in}{0.729505in}}{\pgfqpoint{1.748751in}{0.740400in}}%
\pgfpathcurveto{\pgfqpoint{1.748751in}{0.751296in}}{\pgfqpoint{1.744422in}{0.761747in}}{\pgfqpoint{1.736717in}{0.769451in}}%
\pgfpathcurveto{\pgfqpoint{1.729013in}{0.777155in}}{\pgfqpoint{1.718562in}{0.781484in}}{\pgfqpoint{1.707667in}{0.781484in}}%
\pgfpathcurveto{\pgfqpoint{1.696771in}{0.781484in}}{\pgfqpoint{1.686320in}{0.777155in}}{\pgfqpoint{1.678616in}{0.769451in}}%
\pgfpathcurveto{\pgfqpoint{1.670912in}{0.761747in}}{\pgfqpoint{1.666583in}{0.751296in}}{\pgfqpoint{1.666583in}{0.740400in}}%
\pgfpathcurveto{\pgfqpoint{1.666583in}{0.729505in}}{\pgfqpoint{1.670912in}{0.719054in}}{\pgfqpoint{1.678616in}{0.711350in}}%
\pgfpathcurveto{\pgfqpoint{1.686320in}{0.703645in}}{\pgfqpoint{1.696771in}{0.699316in}}{\pgfqpoint{1.707667in}{0.699316in}}%
\pgfpathlineto{\pgfqpoint{1.707667in}{0.699316in}}%
\pgfpathclose%
\pgfusepath{stroke}%
\end{pgfscope}%
\begin{pgfscope}%
\pgfpathrectangle{\pgfqpoint{0.688192in}{0.670138in}}{\pgfqpoint{7.111808in}{5.129862in}}%
\pgfusepath{clip}%
\pgfsetbuttcap%
\pgfsetroundjoin%
\pgfsetlinewidth{1.003750pt}%
\definecolor{currentstroke}{rgb}{0.000000,0.000000,0.000000}%
\pgfsetstrokecolor{currentstroke}%
\pgfsetdash{}{0pt}%
\pgfpathmoveto{\pgfqpoint{1.419462in}{0.710676in}}%
\pgfpathcurveto{\pgfqpoint{1.430357in}{0.710676in}}{\pgfqpoint{1.440808in}{0.715005in}}{\pgfqpoint{1.448512in}{0.722709in}}%
\pgfpathcurveto{\pgfqpoint{1.456217in}{0.730414in}}{\pgfqpoint{1.460545in}{0.740865in}}{\pgfqpoint{1.460545in}{0.751760in}}%
\pgfpathcurveto{\pgfqpoint{1.460545in}{0.762656in}}{\pgfqpoint{1.456217in}{0.773107in}}{\pgfqpoint{1.448512in}{0.780811in}}%
\pgfpathcurveto{\pgfqpoint{1.440808in}{0.788515in}}{\pgfqpoint{1.430357in}{0.792844in}}{\pgfqpoint{1.419462in}{0.792844in}}%
\pgfpathcurveto{\pgfqpoint{1.408566in}{0.792844in}}{\pgfqpoint{1.398115in}{0.788515in}}{\pgfqpoint{1.390411in}{0.780811in}}%
\pgfpathcurveto{\pgfqpoint{1.382707in}{0.773107in}}{\pgfqpoint{1.378378in}{0.762656in}}{\pgfqpoint{1.378378in}{0.751760in}}%
\pgfpathcurveto{\pgfqpoint{1.378378in}{0.740865in}}{\pgfqpoint{1.382707in}{0.730414in}}{\pgfqpoint{1.390411in}{0.722709in}}%
\pgfpathcurveto{\pgfqpoint{1.398115in}{0.715005in}}{\pgfqpoint{1.408566in}{0.710676in}}{\pgfqpoint{1.419462in}{0.710676in}}%
\pgfpathlineto{\pgfqpoint{1.419462in}{0.710676in}}%
\pgfpathclose%
\pgfusepath{stroke}%
\end{pgfscope}%
\begin{pgfscope}%
\pgfpathrectangle{\pgfqpoint{0.688192in}{0.670138in}}{\pgfqpoint{7.111808in}{5.129862in}}%
\pgfusepath{clip}%
\pgfsetbuttcap%
\pgfsetroundjoin%
\pgfsetlinewidth{1.003750pt}%
\definecolor{currentstroke}{rgb}{0.000000,0.000000,0.000000}%
\pgfsetstrokecolor{currentstroke}%
\pgfsetdash{}{0pt}%
\pgfpathmoveto{\pgfqpoint{1.916797in}{0.693801in}}%
\pgfpathcurveto{\pgfqpoint{1.927692in}{0.693801in}}{\pgfqpoint{1.938143in}{0.698130in}}{\pgfqpoint{1.945847in}{0.705835in}}%
\pgfpathcurveto{\pgfqpoint{1.953552in}{0.713539in}}{\pgfqpoint{1.957880in}{0.723990in}}{\pgfqpoint{1.957880in}{0.734885in}}%
\pgfpathcurveto{\pgfqpoint{1.957880in}{0.745781in}}{\pgfqpoint{1.953552in}{0.756232in}}{\pgfqpoint{1.945847in}{0.763936in}}%
\pgfpathcurveto{\pgfqpoint{1.938143in}{0.771640in}}{\pgfqpoint{1.927692in}{0.775969in}}{\pgfqpoint{1.916797in}{0.775969in}}%
\pgfpathcurveto{\pgfqpoint{1.905901in}{0.775969in}}{\pgfqpoint{1.895450in}{0.771640in}}{\pgfqpoint{1.887746in}{0.763936in}}%
\pgfpathcurveto{\pgfqpoint{1.880042in}{0.756232in}}{\pgfqpoint{1.875713in}{0.745781in}}{\pgfqpoint{1.875713in}{0.734885in}}%
\pgfpathcurveto{\pgfqpoint{1.875713in}{0.723990in}}{\pgfqpoint{1.880042in}{0.713539in}}{\pgfqpoint{1.887746in}{0.705835in}}%
\pgfpathcurveto{\pgfqpoint{1.895450in}{0.698130in}}{\pgfqpoint{1.905901in}{0.693801in}}{\pgfqpoint{1.916797in}{0.693801in}}%
\pgfpathlineto{\pgfqpoint{1.916797in}{0.693801in}}%
\pgfpathclose%
\pgfusepath{stroke}%
\end{pgfscope}%
\begin{pgfscope}%
\pgfpathrectangle{\pgfqpoint{0.688192in}{0.670138in}}{\pgfqpoint{7.111808in}{5.129862in}}%
\pgfusepath{clip}%
\pgfsetbuttcap%
\pgfsetroundjoin%
\pgfsetlinewidth{1.003750pt}%
\definecolor{currentstroke}{rgb}{0.000000,0.000000,0.000000}%
\pgfsetstrokecolor{currentstroke}%
\pgfsetdash{}{0pt}%
\pgfpathmoveto{\pgfqpoint{0.881277in}{0.900428in}}%
\pgfpathcurveto{\pgfqpoint{0.892173in}{0.900428in}}{\pgfqpoint{0.902624in}{0.904756in}}{\pgfqpoint{0.910328in}{0.912461in}}%
\pgfpathcurveto{\pgfqpoint{0.918032in}{0.920165in}}{\pgfqpoint{0.922361in}{0.930616in}}{\pgfqpoint{0.922361in}{0.941511in}}%
\pgfpathcurveto{\pgfqpoint{0.922361in}{0.952407in}}{\pgfqpoint{0.918032in}{0.962858in}}{\pgfqpoint{0.910328in}{0.970562in}}%
\pgfpathcurveto{\pgfqpoint{0.902624in}{0.978266in}}{\pgfqpoint{0.892173in}{0.982595in}}{\pgfqpoint{0.881277in}{0.982595in}}%
\pgfpathcurveto{\pgfqpoint{0.870382in}{0.982595in}}{\pgfqpoint{0.859931in}{0.978266in}}{\pgfqpoint{0.852226in}{0.970562in}}%
\pgfpathcurveto{\pgfqpoint{0.844522in}{0.962858in}}{\pgfqpoint{0.840193in}{0.952407in}}{\pgfqpoint{0.840193in}{0.941511in}}%
\pgfpathcurveto{\pgfqpoint{0.840193in}{0.930616in}}{\pgfqpoint{0.844522in}{0.920165in}}{\pgfqpoint{0.852226in}{0.912461in}}%
\pgfpathcurveto{\pgfqpoint{0.859931in}{0.904756in}}{\pgfqpoint{0.870382in}{0.900428in}}{\pgfqpoint{0.881277in}{0.900428in}}%
\pgfpathlineto{\pgfqpoint{0.881277in}{0.900428in}}%
\pgfpathclose%
\pgfusepath{stroke}%
\end{pgfscope}%
\begin{pgfscope}%
\pgfpathrectangle{\pgfqpoint{0.688192in}{0.670138in}}{\pgfqpoint{7.111808in}{5.129862in}}%
\pgfusepath{clip}%
\pgfsetbuttcap%
\pgfsetroundjoin%
\pgfsetlinewidth{1.003750pt}%
\definecolor{currentstroke}{rgb}{0.000000,0.000000,0.000000}%
\pgfsetstrokecolor{currentstroke}%
\pgfsetdash{}{0pt}%
\pgfpathmoveto{\pgfqpoint{1.070830in}{0.725886in}}%
\pgfpathcurveto{\pgfqpoint{1.081726in}{0.725886in}}{\pgfqpoint{1.092176in}{0.730215in}}{\pgfqpoint{1.099881in}{0.737920in}}%
\pgfpathcurveto{\pgfqpoint{1.107585in}{0.745624in}}{\pgfqpoint{1.111914in}{0.756075in}}{\pgfqpoint{1.111914in}{0.766970in}}%
\pgfpathcurveto{\pgfqpoint{1.111914in}{0.777866in}}{\pgfqpoint{1.107585in}{0.788317in}}{\pgfqpoint{1.099881in}{0.796021in}}%
\pgfpathcurveto{\pgfqpoint{1.092176in}{0.803725in}}{\pgfqpoint{1.081726in}{0.808054in}}{\pgfqpoint{1.070830in}{0.808054in}}%
\pgfpathcurveto{\pgfqpoint{1.059934in}{0.808054in}}{\pgfqpoint{1.049484in}{0.803725in}}{\pgfqpoint{1.041779in}{0.796021in}}%
\pgfpathcurveto{\pgfqpoint{1.034075in}{0.788317in}}{\pgfqpoint{1.029746in}{0.777866in}}{\pgfqpoint{1.029746in}{0.766970in}}%
\pgfpathcurveto{\pgfqpoint{1.029746in}{0.756075in}}{\pgfqpoint{1.034075in}{0.745624in}}{\pgfqpoint{1.041779in}{0.737920in}}%
\pgfpathcurveto{\pgfqpoint{1.049484in}{0.730215in}}{\pgfqpoint{1.059934in}{0.725886in}}{\pgfqpoint{1.070830in}{0.725886in}}%
\pgfpathlineto{\pgfqpoint{1.070830in}{0.725886in}}%
\pgfpathclose%
\pgfusepath{stroke}%
\end{pgfscope}%
\begin{pgfscope}%
\pgfpathrectangle{\pgfqpoint{0.688192in}{0.670138in}}{\pgfqpoint{7.111808in}{5.129862in}}%
\pgfusepath{clip}%
\pgfsetbuttcap%
\pgfsetroundjoin%
\pgfsetlinewidth{1.003750pt}%
\definecolor{currentstroke}{rgb}{0.000000,0.000000,0.000000}%
\pgfsetstrokecolor{currentstroke}%
\pgfsetdash{}{0pt}%
\pgfpathmoveto{\pgfqpoint{0.881277in}{0.900428in}}%
\pgfpathcurveto{\pgfqpoint{0.892173in}{0.900428in}}{\pgfqpoint{0.902624in}{0.904756in}}{\pgfqpoint{0.910328in}{0.912461in}}%
\pgfpathcurveto{\pgfqpoint{0.918032in}{0.920165in}}{\pgfqpoint{0.922361in}{0.930616in}}{\pgfqpoint{0.922361in}{0.941511in}}%
\pgfpathcurveto{\pgfqpoint{0.922361in}{0.952407in}}{\pgfqpoint{0.918032in}{0.962858in}}{\pgfqpoint{0.910328in}{0.970562in}}%
\pgfpathcurveto{\pgfqpoint{0.902624in}{0.978266in}}{\pgfqpoint{0.892173in}{0.982595in}}{\pgfqpoint{0.881277in}{0.982595in}}%
\pgfpathcurveto{\pgfqpoint{0.870382in}{0.982595in}}{\pgfqpoint{0.859931in}{0.978266in}}{\pgfqpoint{0.852226in}{0.970562in}}%
\pgfpathcurveto{\pgfqpoint{0.844522in}{0.962858in}}{\pgfqpoint{0.840193in}{0.952407in}}{\pgfqpoint{0.840193in}{0.941511in}}%
\pgfpathcurveto{\pgfqpoint{0.840193in}{0.930616in}}{\pgfqpoint{0.844522in}{0.920165in}}{\pgfqpoint{0.852226in}{0.912461in}}%
\pgfpathcurveto{\pgfqpoint{0.859931in}{0.904756in}}{\pgfqpoint{0.870382in}{0.900428in}}{\pgfqpoint{0.881277in}{0.900428in}}%
\pgfpathlineto{\pgfqpoint{0.881277in}{0.900428in}}%
\pgfpathclose%
\pgfusepath{stroke}%
\end{pgfscope}%
\begin{pgfscope}%
\pgfpathrectangle{\pgfqpoint{0.688192in}{0.670138in}}{\pgfqpoint{7.111808in}{5.129862in}}%
\pgfusepath{clip}%
\pgfsetbuttcap%
\pgfsetroundjoin%
\pgfsetlinewidth{1.003750pt}%
\definecolor{currentstroke}{rgb}{0.000000,0.000000,0.000000}%
\pgfsetstrokecolor{currentstroke}%
\pgfsetdash{}{0pt}%
\pgfpathmoveto{\pgfqpoint{0.928243in}{0.841597in}}%
\pgfpathcurveto{\pgfqpoint{0.939139in}{0.841597in}}{\pgfqpoint{0.949589in}{0.845926in}}{\pgfqpoint{0.957294in}{0.853631in}}%
\pgfpathcurveto{\pgfqpoint{0.964998in}{0.861335in}}{\pgfqpoint{0.969327in}{0.871786in}}{\pgfqpoint{0.969327in}{0.882681in}}%
\pgfpathcurveto{\pgfqpoint{0.969327in}{0.893577in}}{\pgfqpoint{0.964998in}{0.904028in}}{\pgfqpoint{0.957294in}{0.911732in}}%
\pgfpathcurveto{\pgfqpoint{0.949589in}{0.919436in}}{\pgfqpoint{0.939139in}{0.923765in}}{\pgfqpoint{0.928243in}{0.923765in}}%
\pgfpathcurveto{\pgfqpoint{0.917347in}{0.923765in}}{\pgfqpoint{0.906897in}{0.919436in}}{\pgfqpoint{0.899192in}{0.911732in}}%
\pgfpathcurveto{\pgfqpoint{0.891488in}{0.904028in}}{\pgfqpoint{0.887159in}{0.893577in}}{\pgfqpoint{0.887159in}{0.882681in}}%
\pgfpathcurveto{\pgfqpoint{0.887159in}{0.871786in}}{\pgfqpoint{0.891488in}{0.861335in}}{\pgfqpoint{0.899192in}{0.853631in}}%
\pgfpathcurveto{\pgfqpoint{0.906897in}{0.845926in}}{\pgfqpoint{0.917347in}{0.841597in}}{\pgfqpoint{0.928243in}{0.841597in}}%
\pgfpathlineto{\pgfqpoint{0.928243in}{0.841597in}}%
\pgfpathclose%
\pgfusepath{stroke}%
\end{pgfscope}%
\begin{pgfscope}%
\pgfpathrectangle{\pgfqpoint{0.688192in}{0.670138in}}{\pgfqpoint{7.111808in}{5.129862in}}%
\pgfusepath{clip}%
\pgfsetbuttcap%
\pgfsetroundjoin%
\pgfsetlinewidth{1.003750pt}%
\definecolor{currentstroke}{rgb}{0.000000,0.000000,0.000000}%
\pgfsetstrokecolor{currentstroke}%
\pgfsetdash{}{0pt}%
\pgfpathmoveto{\pgfqpoint{0.895312in}{0.866174in}}%
\pgfpathcurveto{\pgfqpoint{0.906208in}{0.866174in}}{\pgfqpoint{0.916658in}{0.870503in}}{\pgfqpoint{0.924363in}{0.878207in}}%
\pgfpathcurveto{\pgfqpoint{0.932067in}{0.885911in}}{\pgfqpoint{0.936396in}{0.896362in}}{\pgfqpoint{0.936396in}{0.907258in}}%
\pgfpathcurveto{\pgfqpoint{0.936396in}{0.918153in}}{\pgfqpoint{0.932067in}{0.928604in}}{\pgfqpoint{0.924363in}{0.936308in}}%
\pgfpathcurveto{\pgfqpoint{0.916658in}{0.944013in}}{\pgfqpoint{0.906208in}{0.948341in}}{\pgfqpoint{0.895312in}{0.948341in}}%
\pgfpathcurveto{\pgfqpoint{0.884416in}{0.948341in}}{\pgfqpoint{0.873966in}{0.944013in}}{\pgfqpoint{0.866261in}{0.936308in}}%
\pgfpathcurveto{\pgfqpoint{0.858557in}{0.928604in}}{\pgfqpoint{0.854228in}{0.918153in}}{\pgfqpoint{0.854228in}{0.907258in}}%
\pgfpathcurveto{\pgfqpoint{0.854228in}{0.896362in}}{\pgfqpoint{0.858557in}{0.885911in}}{\pgfqpoint{0.866261in}{0.878207in}}%
\pgfpathcurveto{\pgfqpoint{0.873966in}{0.870503in}}{\pgfqpoint{0.884416in}{0.866174in}}{\pgfqpoint{0.895312in}{0.866174in}}%
\pgfpathlineto{\pgfqpoint{0.895312in}{0.866174in}}%
\pgfpathclose%
\pgfusepath{stroke}%
\end{pgfscope}%
\begin{pgfscope}%
\pgfpathrectangle{\pgfqpoint{0.688192in}{0.670138in}}{\pgfqpoint{7.111808in}{5.129862in}}%
\pgfusepath{clip}%
\pgfsetbuttcap%
\pgfsetroundjoin%
\pgfsetlinewidth{1.003750pt}%
\definecolor{currentstroke}{rgb}{0.000000,0.000000,0.000000}%
\pgfsetstrokecolor{currentstroke}%
\pgfsetdash{}{0pt}%
\pgfpathmoveto{\pgfqpoint{0.789763in}{1.038554in}}%
\pgfpathcurveto{\pgfqpoint{0.800659in}{1.038554in}}{\pgfqpoint{0.811109in}{1.042883in}}{\pgfqpoint{0.818814in}{1.050587in}}%
\pgfpathcurveto{\pgfqpoint{0.826518in}{1.058291in}}{\pgfqpoint{0.830847in}{1.068742in}}{\pgfqpoint{0.830847in}{1.079638in}}%
\pgfpathcurveto{\pgfqpoint{0.830847in}{1.090533in}}{\pgfqpoint{0.826518in}{1.100984in}}{\pgfqpoint{0.818814in}{1.108688in}}%
\pgfpathcurveto{\pgfqpoint{0.811109in}{1.116393in}}{\pgfqpoint{0.800659in}{1.120722in}}{\pgfqpoint{0.789763in}{1.120722in}}%
\pgfpathcurveto{\pgfqpoint{0.778868in}{1.120722in}}{\pgfqpoint{0.768417in}{1.116393in}}{\pgfqpoint{0.760712in}{1.108688in}}%
\pgfpathcurveto{\pgfqpoint{0.753008in}{1.100984in}}{\pgfqpoint{0.748679in}{1.090533in}}{\pgfqpoint{0.748679in}{1.079638in}}%
\pgfpathcurveto{\pgfqpoint{0.748679in}{1.068742in}}{\pgfqpoint{0.753008in}{1.058291in}}{\pgfqpoint{0.760712in}{1.050587in}}%
\pgfpathcurveto{\pgfqpoint{0.768417in}{1.042883in}}{\pgfqpoint{0.778868in}{1.038554in}}{\pgfqpoint{0.789763in}{1.038554in}}%
\pgfpathlineto{\pgfqpoint{0.789763in}{1.038554in}}%
\pgfpathclose%
\pgfusepath{stroke}%
\end{pgfscope}%
\begin{pgfscope}%
\pgfpathrectangle{\pgfqpoint{0.688192in}{0.670138in}}{\pgfqpoint{7.111808in}{5.129862in}}%
\pgfusepath{clip}%
\pgfsetbuttcap%
\pgfsetroundjoin%
\pgfsetlinewidth{1.003750pt}%
\definecolor{currentstroke}{rgb}{0.000000,0.000000,0.000000}%
\pgfsetstrokecolor{currentstroke}%
\pgfsetdash{}{0pt}%
\pgfpathmoveto{\pgfqpoint{0.812030in}{0.993250in}}%
\pgfpathcurveto{\pgfqpoint{0.822926in}{0.993250in}}{\pgfqpoint{0.833377in}{0.997579in}}{\pgfqpoint{0.841081in}{1.005284in}}%
\pgfpathcurveto{\pgfqpoint{0.848785in}{1.012988in}}{\pgfqpoint{0.853114in}{1.023439in}}{\pgfqpoint{0.853114in}{1.034334in}}%
\pgfpathcurveto{\pgfqpoint{0.853114in}{1.045230in}}{\pgfqpoint{0.848785in}{1.055681in}}{\pgfqpoint{0.841081in}{1.063385in}}%
\pgfpathcurveto{\pgfqpoint{0.833377in}{1.071089in}}{\pgfqpoint{0.822926in}{1.075418in}}{\pgfqpoint{0.812030in}{1.075418in}}%
\pgfpathcurveto{\pgfqpoint{0.801135in}{1.075418in}}{\pgfqpoint{0.790684in}{1.071089in}}{\pgfqpoint{0.782980in}{1.063385in}}%
\pgfpathcurveto{\pgfqpoint{0.775275in}{1.055681in}}{\pgfqpoint{0.770946in}{1.045230in}}{\pgfqpoint{0.770946in}{1.034334in}}%
\pgfpathcurveto{\pgfqpoint{0.770946in}{1.023439in}}{\pgfqpoint{0.775275in}{1.012988in}}{\pgfqpoint{0.782980in}{1.005284in}}%
\pgfpathcurveto{\pgfqpoint{0.790684in}{0.997579in}}{\pgfqpoint{0.801135in}{0.993250in}}{\pgfqpoint{0.812030in}{0.993250in}}%
\pgfpathlineto{\pgfqpoint{0.812030in}{0.993250in}}%
\pgfpathclose%
\pgfusepath{stroke}%
\end{pgfscope}%
\begin{pgfscope}%
\pgfpathrectangle{\pgfqpoint{0.688192in}{0.670138in}}{\pgfqpoint{7.111808in}{5.129862in}}%
\pgfusepath{clip}%
\pgfsetbuttcap%
\pgfsetroundjoin%
\pgfsetlinewidth{1.003750pt}%
\definecolor{currentstroke}{rgb}{0.000000,0.000000,0.000000}%
\pgfsetstrokecolor{currentstroke}%
\pgfsetdash{}{0pt}%
\pgfpathmoveto{\pgfqpoint{1.091394in}{0.722252in}}%
\pgfpathcurveto{\pgfqpoint{1.102290in}{0.722252in}}{\pgfqpoint{1.112741in}{0.726581in}}{\pgfqpoint{1.120445in}{0.734285in}}%
\pgfpathcurveto{\pgfqpoint{1.128149in}{0.741989in}}{\pgfqpoint{1.132478in}{0.752440in}}{\pgfqpoint{1.132478in}{0.763336in}}%
\pgfpathcurveto{\pgfqpoint{1.132478in}{0.774231in}}{\pgfqpoint{1.128149in}{0.784682in}}{\pgfqpoint{1.120445in}{0.792386in}}%
\pgfpathcurveto{\pgfqpoint{1.112741in}{0.800091in}}{\pgfqpoint{1.102290in}{0.804420in}}{\pgfqpoint{1.091394in}{0.804420in}}%
\pgfpathcurveto{\pgfqpoint{1.080499in}{0.804420in}}{\pgfqpoint{1.070048in}{0.800091in}}{\pgfqpoint{1.062344in}{0.792386in}}%
\pgfpathcurveto{\pgfqpoint{1.054639in}{0.784682in}}{\pgfqpoint{1.050310in}{0.774231in}}{\pgfqpoint{1.050310in}{0.763336in}}%
\pgfpathcurveto{\pgfqpoint{1.050310in}{0.752440in}}{\pgfqpoint{1.054639in}{0.741989in}}{\pgfqpoint{1.062344in}{0.734285in}}%
\pgfpathcurveto{\pgfqpoint{1.070048in}{0.726581in}}{\pgfqpoint{1.080499in}{0.722252in}}{\pgfqpoint{1.091394in}{0.722252in}}%
\pgfpathlineto{\pgfqpoint{1.091394in}{0.722252in}}%
\pgfpathclose%
\pgfusepath{stroke}%
\end{pgfscope}%
\begin{pgfscope}%
\pgfpathrectangle{\pgfqpoint{0.688192in}{0.670138in}}{\pgfqpoint{7.111808in}{5.129862in}}%
\pgfusepath{clip}%
\pgfsetbuttcap%
\pgfsetroundjoin%
\pgfsetlinewidth{1.003750pt}%
\definecolor{currentstroke}{rgb}{0.000000,0.000000,0.000000}%
\pgfsetstrokecolor{currentstroke}%
\pgfsetdash{}{0pt}%
\pgfpathmoveto{\pgfqpoint{1.120660in}{0.719620in}}%
\pgfpathcurveto{\pgfqpoint{1.131555in}{0.719620in}}{\pgfqpoint{1.142006in}{0.723949in}}{\pgfqpoint{1.149710in}{0.731653in}}%
\pgfpathcurveto{\pgfqpoint{1.157415in}{0.739358in}}{\pgfqpoint{1.161744in}{0.749809in}}{\pgfqpoint{1.161744in}{0.760704in}}%
\pgfpathcurveto{\pgfqpoint{1.161744in}{0.771600in}}{\pgfqpoint{1.157415in}{0.782051in}}{\pgfqpoint{1.149710in}{0.789755in}}%
\pgfpathcurveto{\pgfqpoint{1.142006in}{0.797459in}}{\pgfqpoint{1.131555in}{0.801788in}}{\pgfqpoint{1.120660in}{0.801788in}}%
\pgfpathcurveto{\pgfqpoint{1.109764in}{0.801788in}}{\pgfqpoint{1.099313in}{0.797459in}}{\pgfqpoint{1.091609in}{0.789755in}}%
\pgfpathcurveto{\pgfqpoint{1.083905in}{0.782051in}}{\pgfqpoint{1.079576in}{0.771600in}}{\pgfqpoint{1.079576in}{0.760704in}}%
\pgfpathcurveto{\pgfqpoint{1.079576in}{0.749809in}}{\pgfqpoint{1.083905in}{0.739358in}}{\pgfqpoint{1.091609in}{0.731653in}}%
\pgfpathcurveto{\pgfqpoint{1.099313in}{0.723949in}}{\pgfqpoint{1.109764in}{0.719620in}}{\pgfqpoint{1.120660in}{0.719620in}}%
\pgfpathlineto{\pgfqpoint{1.120660in}{0.719620in}}%
\pgfpathclose%
\pgfusepath{stroke}%
\end{pgfscope}%
\begin{pgfscope}%
\pgfpathrectangle{\pgfqpoint{0.688192in}{0.670138in}}{\pgfqpoint{7.111808in}{5.129862in}}%
\pgfusepath{clip}%
\pgfsetbuttcap%
\pgfsetroundjoin%
\pgfsetlinewidth{1.003750pt}%
\definecolor{currentstroke}{rgb}{0.000000,0.000000,0.000000}%
\pgfsetstrokecolor{currentstroke}%
\pgfsetdash{}{0pt}%
\pgfpathmoveto{\pgfqpoint{0.886861in}{0.869726in}}%
\pgfpathcurveto{\pgfqpoint{0.897756in}{0.869726in}}{\pgfqpoint{0.908207in}{0.874055in}}{\pgfqpoint{0.915911in}{0.881759in}}%
\pgfpathcurveto{\pgfqpoint{0.923616in}{0.889463in}}{\pgfqpoint{0.927945in}{0.899914in}}{\pgfqpoint{0.927945in}{0.910810in}}%
\pgfpathcurveto{\pgfqpoint{0.927945in}{0.921705in}}{\pgfqpoint{0.923616in}{0.932156in}}{\pgfqpoint{0.915911in}{0.939861in}}%
\pgfpathcurveto{\pgfqpoint{0.908207in}{0.947565in}}{\pgfqpoint{0.897756in}{0.951894in}}{\pgfqpoint{0.886861in}{0.951894in}}%
\pgfpathcurveto{\pgfqpoint{0.875965in}{0.951894in}}{\pgfqpoint{0.865514in}{0.947565in}}{\pgfqpoint{0.857810in}{0.939861in}}%
\pgfpathcurveto{\pgfqpoint{0.850106in}{0.932156in}}{\pgfqpoint{0.845777in}{0.921705in}}{\pgfqpoint{0.845777in}{0.910810in}}%
\pgfpathcurveto{\pgfqpoint{0.845777in}{0.899914in}}{\pgfqpoint{0.850106in}{0.889463in}}{\pgfqpoint{0.857810in}{0.881759in}}%
\pgfpathcurveto{\pgfqpoint{0.865514in}{0.874055in}}{\pgfqpoint{0.875965in}{0.869726in}}{\pgfqpoint{0.886861in}{0.869726in}}%
\pgfpathlineto{\pgfqpoint{0.886861in}{0.869726in}}%
\pgfpathclose%
\pgfusepath{stroke}%
\end{pgfscope}%
\begin{pgfscope}%
\pgfpathrectangle{\pgfqpoint{0.688192in}{0.670138in}}{\pgfqpoint{7.111808in}{5.129862in}}%
\pgfusepath{clip}%
\pgfsetbuttcap%
\pgfsetroundjoin%
\pgfsetlinewidth{1.003750pt}%
\definecolor{currentstroke}{rgb}{0.000000,0.000000,0.000000}%
\pgfsetstrokecolor{currentstroke}%
\pgfsetdash{}{0pt}%
\pgfpathmoveto{\pgfqpoint{1.389197in}{0.711266in}}%
\pgfpathcurveto{\pgfqpoint{1.400093in}{0.711266in}}{\pgfqpoint{1.410544in}{0.715595in}}{\pgfqpoint{1.418248in}{0.723299in}}%
\pgfpathcurveto{\pgfqpoint{1.425952in}{0.731004in}}{\pgfqpoint{1.430281in}{0.741454in}}{\pgfqpoint{1.430281in}{0.752350in}}%
\pgfpathcurveto{\pgfqpoint{1.430281in}{0.763246in}}{\pgfqpoint{1.425952in}{0.773696in}}{\pgfqpoint{1.418248in}{0.781401in}}%
\pgfpathcurveto{\pgfqpoint{1.410544in}{0.789105in}}{\pgfqpoint{1.400093in}{0.793434in}}{\pgfqpoint{1.389197in}{0.793434in}}%
\pgfpathcurveto{\pgfqpoint{1.378302in}{0.793434in}}{\pgfqpoint{1.367851in}{0.789105in}}{\pgfqpoint{1.360147in}{0.781401in}}%
\pgfpathcurveto{\pgfqpoint{1.352442in}{0.773696in}}{\pgfqpoint{1.348113in}{0.763246in}}{\pgfqpoint{1.348113in}{0.752350in}}%
\pgfpathcurveto{\pgfqpoint{1.348113in}{0.741454in}}{\pgfqpoint{1.352442in}{0.731004in}}{\pgfqpoint{1.360147in}{0.723299in}}%
\pgfpathcurveto{\pgfqpoint{1.367851in}{0.715595in}}{\pgfqpoint{1.378302in}{0.711266in}}{\pgfqpoint{1.389197in}{0.711266in}}%
\pgfpathlineto{\pgfqpoint{1.389197in}{0.711266in}}%
\pgfpathclose%
\pgfusepath{stroke}%
\end{pgfscope}%
\begin{pgfscope}%
\pgfpathrectangle{\pgfqpoint{0.688192in}{0.670138in}}{\pgfqpoint{7.111808in}{5.129862in}}%
\pgfusepath{clip}%
\pgfsetbuttcap%
\pgfsetroundjoin%
\pgfsetlinewidth{1.003750pt}%
\definecolor{currentstroke}{rgb}{0.000000,0.000000,0.000000}%
\pgfsetstrokecolor{currentstroke}%
\pgfsetdash{}{0pt}%
\pgfpathmoveto{\pgfqpoint{4.603680in}{0.639749in}}%
\pgfpathcurveto{\pgfqpoint{4.614576in}{0.639749in}}{\pgfqpoint{4.625026in}{0.644077in}}{\pgfqpoint{4.632731in}{0.651782in}}%
\pgfpathcurveto{\pgfqpoint{4.640435in}{0.659486in}}{\pgfqpoint{4.644764in}{0.669937in}}{\pgfqpoint{4.644764in}{0.680832in}}%
\pgfpathcurveto{\pgfqpoint{4.644764in}{0.691728in}}{\pgfqpoint{4.640435in}{0.702179in}}{\pgfqpoint{4.632731in}{0.709883in}}%
\pgfpathcurveto{\pgfqpoint{4.625026in}{0.717587in}}{\pgfqpoint{4.614576in}{0.721916in}}{\pgfqpoint{4.603680in}{0.721916in}}%
\pgfpathcurveto{\pgfqpoint{4.592784in}{0.721916in}}{\pgfqpoint{4.582334in}{0.717587in}}{\pgfqpoint{4.574629in}{0.709883in}}%
\pgfpathcurveto{\pgfqpoint{4.566925in}{0.702179in}}{\pgfqpoint{4.562596in}{0.691728in}}{\pgfqpoint{4.562596in}{0.680832in}}%
\pgfpathcurveto{\pgfqpoint{4.562596in}{0.669937in}}{\pgfqpoint{4.566925in}{0.659486in}}{\pgfqpoint{4.574629in}{0.651782in}}%
\pgfpathcurveto{\pgfqpoint{4.582334in}{0.644077in}}{\pgfqpoint{4.592784in}{0.639749in}}{\pgfqpoint{4.603680in}{0.639749in}}%
\pgfusepath{stroke}%
\end{pgfscope}%
\begin{pgfscope}%
\pgfpathrectangle{\pgfqpoint{0.688192in}{0.670138in}}{\pgfqpoint{7.111808in}{5.129862in}}%
\pgfusepath{clip}%
\pgfsetbuttcap%
\pgfsetroundjoin%
\pgfsetlinewidth{1.003750pt}%
\definecolor{currentstroke}{rgb}{0.000000,0.000000,0.000000}%
\pgfsetstrokecolor{currentstroke}%
\pgfsetdash{}{0pt}%
\pgfpathmoveto{\pgfqpoint{2.887549in}{5.756388in}}%
\pgfpathcurveto{\pgfqpoint{2.898445in}{5.756388in}}{\pgfqpoint{2.908896in}{5.760716in}}{\pgfqpoint{2.916600in}{5.768421in}}%
\pgfpathcurveto{\pgfqpoint{2.924305in}{5.776125in}}{\pgfqpoint{2.928633in}{5.786576in}}{\pgfqpoint{2.928633in}{5.797472in}}%
\pgfpathcurveto{\pgfqpoint{2.928633in}{5.808367in}}{\pgfqpoint{2.924305in}{5.818818in}}{\pgfqpoint{2.916600in}{5.826522in}}%
\pgfpathcurveto{\pgfqpoint{2.908896in}{5.834227in}}{\pgfqpoint{2.898445in}{5.838555in}}{\pgfqpoint{2.887549in}{5.838555in}}%
\pgfpathcurveto{\pgfqpoint{2.876654in}{5.838555in}}{\pgfqpoint{2.866203in}{5.834227in}}{\pgfqpoint{2.858499in}{5.826522in}}%
\pgfpathcurveto{\pgfqpoint{2.850794in}{5.818818in}}{\pgfqpoint{2.846466in}{5.808367in}}{\pgfqpoint{2.846466in}{5.797472in}}%
\pgfpathcurveto{\pgfqpoint{2.846466in}{5.786576in}}{\pgfqpoint{2.850794in}{5.776125in}}{\pgfqpoint{2.858499in}{5.768421in}}%
\pgfpathcurveto{\pgfqpoint{2.866203in}{5.760716in}}{\pgfqpoint{2.876654in}{5.756388in}}{\pgfqpoint{2.887549in}{5.756388in}}%
\pgfpathlineto{\pgfqpoint{2.887549in}{5.756388in}}%
\pgfpathclose%
\pgfusepath{stroke}%
\end{pgfscope}%
\begin{pgfscope}%
\pgfpathrectangle{\pgfqpoint{0.688192in}{0.670138in}}{\pgfqpoint{7.111808in}{5.129862in}}%
\pgfusepath{clip}%
\pgfsetbuttcap%
\pgfsetroundjoin%
\pgfsetlinewidth{1.003750pt}%
\definecolor{currentstroke}{rgb}{0.000000,0.000000,0.000000}%
\pgfsetstrokecolor{currentstroke}%
\pgfsetdash{}{0pt}%
\pgfpathmoveto{\pgfqpoint{0.794194in}{1.022605in}}%
\pgfpathcurveto{\pgfqpoint{0.805090in}{1.022605in}}{\pgfqpoint{0.815541in}{1.026933in}}{\pgfqpoint{0.823245in}{1.034638in}}%
\pgfpathcurveto{\pgfqpoint{0.830949in}{1.042342in}}{\pgfqpoint{0.835278in}{1.052793in}}{\pgfqpoint{0.835278in}{1.063688in}}%
\pgfpathcurveto{\pgfqpoint{0.835278in}{1.074584in}}{\pgfqpoint{0.830949in}{1.085035in}}{\pgfqpoint{0.823245in}{1.092739in}}%
\pgfpathcurveto{\pgfqpoint{0.815541in}{1.100443in}}{\pgfqpoint{0.805090in}{1.104772in}}{\pgfqpoint{0.794194in}{1.104772in}}%
\pgfpathcurveto{\pgfqpoint{0.783299in}{1.104772in}}{\pgfqpoint{0.772848in}{1.100443in}}{\pgfqpoint{0.765144in}{1.092739in}}%
\pgfpathcurveto{\pgfqpoint{0.757439in}{1.085035in}}{\pgfqpoint{0.753110in}{1.074584in}}{\pgfqpoint{0.753110in}{1.063688in}}%
\pgfpathcurveto{\pgfqpoint{0.753110in}{1.052793in}}{\pgfqpoint{0.757439in}{1.042342in}}{\pgfqpoint{0.765144in}{1.034638in}}%
\pgfpathcurveto{\pgfqpoint{0.772848in}{1.026933in}}{\pgfqpoint{0.783299in}{1.022605in}}{\pgfqpoint{0.794194in}{1.022605in}}%
\pgfpathlineto{\pgfqpoint{0.794194in}{1.022605in}}%
\pgfpathclose%
\pgfusepath{stroke}%
\end{pgfscope}%
\begin{pgfscope}%
\pgfpathrectangle{\pgfqpoint{0.688192in}{0.670138in}}{\pgfqpoint{7.111808in}{5.129862in}}%
\pgfusepath{clip}%
\pgfsetbuttcap%
\pgfsetroundjoin%
\pgfsetlinewidth{1.003750pt}%
\definecolor{currentstroke}{rgb}{0.000000,0.000000,0.000000}%
\pgfsetstrokecolor{currentstroke}%
\pgfsetdash{}{0pt}%
\pgfpathmoveto{\pgfqpoint{5.640247in}{5.623015in}}%
\pgfpathcurveto{\pgfqpoint{5.651142in}{5.623015in}}{\pgfqpoint{5.661593in}{5.627344in}}{\pgfqpoint{5.669297in}{5.635048in}}%
\pgfpathcurveto{\pgfqpoint{5.677002in}{5.642752in}}{\pgfqpoint{5.681331in}{5.653203in}}{\pgfqpoint{5.681331in}{5.664099in}}%
\pgfpathcurveto{\pgfqpoint{5.681331in}{5.674994in}}{\pgfqpoint{5.677002in}{5.685445in}}{\pgfqpoint{5.669297in}{5.693149in}}%
\pgfpathcurveto{\pgfqpoint{5.661593in}{5.700854in}}{\pgfqpoint{5.651142in}{5.705182in}}{\pgfqpoint{5.640247in}{5.705182in}}%
\pgfpathcurveto{\pgfqpoint{5.629351in}{5.705182in}}{\pgfqpoint{5.618900in}{5.700854in}}{\pgfqpoint{5.611196in}{5.693149in}}%
\pgfpathcurveto{\pgfqpoint{5.603492in}{5.685445in}}{\pgfqpoint{5.599163in}{5.674994in}}{\pgfqpoint{5.599163in}{5.664099in}}%
\pgfpathcurveto{\pgfqpoint{5.599163in}{5.653203in}}{\pgfqpoint{5.603492in}{5.642752in}}{\pgfqpoint{5.611196in}{5.635048in}}%
\pgfpathcurveto{\pgfqpoint{5.618900in}{5.627344in}}{\pgfqpoint{5.629351in}{5.623015in}}{\pgfqpoint{5.640247in}{5.623015in}}%
\pgfpathlineto{\pgfqpoint{5.640247in}{5.623015in}}%
\pgfpathclose%
\pgfusepath{stroke}%
\end{pgfscope}%
\begin{pgfscope}%
\pgfpathrectangle{\pgfqpoint{0.688192in}{0.670138in}}{\pgfqpoint{7.111808in}{5.129862in}}%
\pgfusepath{clip}%
\pgfsetbuttcap%
\pgfsetroundjoin%
\pgfsetlinewidth{1.003750pt}%
\definecolor{currentstroke}{rgb}{0.000000,0.000000,0.000000}%
\pgfsetstrokecolor{currentstroke}%
\pgfsetdash{}{0pt}%
\pgfpathmoveto{\pgfqpoint{1.916797in}{0.693801in}}%
\pgfpathcurveto{\pgfqpoint{1.927692in}{0.693801in}}{\pgfqpoint{1.938143in}{0.698130in}}{\pgfqpoint{1.945847in}{0.705835in}}%
\pgfpathcurveto{\pgfqpoint{1.953552in}{0.713539in}}{\pgfqpoint{1.957880in}{0.723990in}}{\pgfqpoint{1.957880in}{0.734885in}}%
\pgfpathcurveto{\pgfqpoint{1.957880in}{0.745781in}}{\pgfqpoint{1.953552in}{0.756232in}}{\pgfqpoint{1.945847in}{0.763936in}}%
\pgfpathcurveto{\pgfqpoint{1.938143in}{0.771640in}}{\pgfqpoint{1.927692in}{0.775969in}}{\pgfqpoint{1.916797in}{0.775969in}}%
\pgfpathcurveto{\pgfqpoint{1.905901in}{0.775969in}}{\pgfqpoint{1.895450in}{0.771640in}}{\pgfqpoint{1.887746in}{0.763936in}}%
\pgfpathcurveto{\pgfqpoint{1.880042in}{0.756232in}}{\pgfqpoint{1.875713in}{0.745781in}}{\pgfqpoint{1.875713in}{0.734885in}}%
\pgfpathcurveto{\pgfqpoint{1.875713in}{0.723990in}}{\pgfqpoint{1.880042in}{0.713539in}}{\pgfqpoint{1.887746in}{0.705835in}}%
\pgfpathcurveto{\pgfqpoint{1.895450in}{0.698130in}}{\pgfqpoint{1.905901in}{0.693801in}}{\pgfqpoint{1.916797in}{0.693801in}}%
\pgfpathlineto{\pgfqpoint{1.916797in}{0.693801in}}%
\pgfpathclose%
\pgfusepath{stroke}%
\end{pgfscope}%
\begin{pgfscope}%
\pgfpathrectangle{\pgfqpoint{0.688192in}{0.670138in}}{\pgfqpoint{7.111808in}{5.129862in}}%
\pgfusepath{clip}%
\pgfsetbuttcap%
\pgfsetroundjoin%
\pgfsetlinewidth{1.003750pt}%
\definecolor{currentstroke}{rgb}{0.000000,0.000000,0.000000}%
\pgfsetstrokecolor{currentstroke}%
\pgfsetdash{}{0pt}%
\pgfpathmoveto{\pgfqpoint{2.333738in}{3.417663in}}%
\pgfpathcurveto{\pgfqpoint{2.344634in}{3.417663in}}{\pgfqpoint{2.355085in}{3.421992in}}{\pgfqpoint{2.362789in}{3.429696in}}%
\pgfpathcurveto{\pgfqpoint{2.370493in}{3.437401in}}{\pgfqpoint{2.374822in}{3.447852in}}{\pgfqpoint{2.374822in}{3.458747in}}%
\pgfpathcurveto{\pgfqpoint{2.374822in}{3.469643in}}{\pgfqpoint{2.370493in}{3.480093in}}{\pgfqpoint{2.362789in}{3.487798in}}%
\pgfpathcurveto{\pgfqpoint{2.355085in}{3.495502in}}{\pgfqpoint{2.344634in}{3.499831in}}{\pgfqpoint{2.333738in}{3.499831in}}%
\pgfpathcurveto{\pgfqpoint{2.322843in}{3.499831in}}{\pgfqpoint{2.312392in}{3.495502in}}{\pgfqpoint{2.304688in}{3.487798in}}%
\pgfpathcurveto{\pgfqpoint{2.296983in}{3.480093in}}{\pgfqpoint{2.292655in}{3.469643in}}{\pgfqpoint{2.292655in}{3.458747in}}%
\pgfpathcurveto{\pgfqpoint{2.292655in}{3.447852in}}{\pgfqpoint{2.296983in}{3.437401in}}{\pgfqpoint{2.304688in}{3.429696in}}%
\pgfpathcurveto{\pgfqpoint{2.312392in}{3.421992in}}{\pgfqpoint{2.322843in}{3.417663in}}{\pgfqpoint{2.333738in}{3.417663in}}%
\pgfpathlineto{\pgfqpoint{2.333738in}{3.417663in}}%
\pgfpathclose%
\pgfusepath{stroke}%
\end{pgfscope}%
\begin{pgfscope}%
\pgfpathrectangle{\pgfqpoint{0.688192in}{0.670138in}}{\pgfqpoint{7.111808in}{5.129862in}}%
\pgfusepath{clip}%
\pgfsetbuttcap%
\pgfsetroundjoin%
\pgfsetlinewidth{1.003750pt}%
\definecolor{currentstroke}{rgb}{0.000000,0.000000,0.000000}%
\pgfsetstrokecolor{currentstroke}%
\pgfsetdash{}{0pt}%
\pgfpathmoveto{\pgfqpoint{1.101880in}{0.720677in}}%
\pgfpathcurveto{\pgfqpoint{1.112775in}{0.720677in}}{\pgfqpoint{1.123226in}{0.725006in}}{\pgfqpoint{1.130930in}{0.732710in}}%
\pgfpathcurveto{\pgfqpoint{1.138635in}{0.740415in}}{\pgfqpoint{1.142964in}{0.750865in}}{\pgfqpoint{1.142964in}{0.761761in}}%
\pgfpathcurveto{\pgfqpoint{1.142964in}{0.772656in}}{\pgfqpoint{1.138635in}{0.783107in}}{\pgfqpoint{1.130930in}{0.790812in}}%
\pgfpathcurveto{\pgfqpoint{1.123226in}{0.798516in}}{\pgfqpoint{1.112775in}{0.802845in}}{\pgfqpoint{1.101880in}{0.802845in}}%
\pgfpathcurveto{\pgfqpoint{1.090984in}{0.802845in}}{\pgfqpoint{1.080533in}{0.798516in}}{\pgfqpoint{1.072829in}{0.790812in}}%
\pgfpathcurveto{\pgfqpoint{1.065125in}{0.783107in}}{\pgfqpoint{1.060796in}{0.772656in}}{\pgfqpoint{1.060796in}{0.761761in}}%
\pgfpathcurveto{\pgfqpoint{1.060796in}{0.750865in}}{\pgfqpoint{1.065125in}{0.740415in}}{\pgfqpoint{1.072829in}{0.732710in}}%
\pgfpathcurveto{\pgfqpoint{1.080533in}{0.725006in}}{\pgfqpoint{1.090984in}{0.720677in}}{\pgfqpoint{1.101880in}{0.720677in}}%
\pgfpathlineto{\pgfqpoint{1.101880in}{0.720677in}}%
\pgfpathclose%
\pgfusepath{stroke}%
\end{pgfscope}%
\begin{pgfscope}%
\pgfpathrectangle{\pgfqpoint{0.688192in}{0.670138in}}{\pgfqpoint{7.111808in}{5.129862in}}%
\pgfusepath{clip}%
\pgfsetbuttcap%
\pgfsetroundjoin%
\pgfsetlinewidth{1.003750pt}%
\definecolor{currentstroke}{rgb}{0.000000,0.000000,0.000000}%
\pgfsetstrokecolor{currentstroke}%
\pgfsetdash{}{0pt}%
\pgfpathmoveto{\pgfqpoint{2.543341in}{0.834271in}}%
\pgfpathcurveto{\pgfqpoint{2.554237in}{0.834271in}}{\pgfqpoint{2.564688in}{0.838600in}}{\pgfqpoint{2.572392in}{0.846304in}}%
\pgfpathcurveto{\pgfqpoint{2.580096in}{0.854009in}}{\pgfqpoint{2.584425in}{0.864459in}}{\pgfqpoint{2.584425in}{0.875355in}}%
\pgfpathcurveto{\pgfqpoint{2.584425in}{0.886251in}}{\pgfqpoint{2.580096in}{0.896701in}}{\pgfqpoint{2.572392in}{0.904406in}}%
\pgfpathcurveto{\pgfqpoint{2.564688in}{0.912110in}}{\pgfqpoint{2.554237in}{0.916439in}}{\pgfqpoint{2.543341in}{0.916439in}}%
\pgfpathcurveto{\pgfqpoint{2.532446in}{0.916439in}}{\pgfqpoint{2.521995in}{0.912110in}}{\pgfqpoint{2.514291in}{0.904406in}}%
\pgfpathcurveto{\pgfqpoint{2.506586in}{0.896701in}}{\pgfqpoint{2.502258in}{0.886251in}}{\pgfqpoint{2.502258in}{0.875355in}}%
\pgfpathcurveto{\pgfqpoint{2.502258in}{0.864459in}}{\pgfqpoint{2.506586in}{0.854009in}}{\pgfqpoint{2.514291in}{0.846304in}}%
\pgfpathcurveto{\pgfqpoint{2.521995in}{0.838600in}}{\pgfqpoint{2.532446in}{0.834271in}}{\pgfqpoint{2.543341in}{0.834271in}}%
\pgfpathlineto{\pgfqpoint{2.543341in}{0.834271in}}%
\pgfpathclose%
\pgfusepath{stroke}%
\end{pgfscope}%
\begin{pgfscope}%
\pgfpathrectangle{\pgfqpoint{0.688192in}{0.670138in}}{\pgfqpoint{7.111808in}{5.129862in}}%
\pgfusepath{clip}%
\pgfsetbuttcap%
\pgfsetroundjoin%
\pgfsetlinewidth{1.003750pt}%
\definecolor{currentstroke}{rgb}{0.000000,0.000000,0.000000}%
\pgfsetstrokecolor{currentstroke}%
\pgfsetdash{}{0pt}%
\pgfpathmoveto{\pgfqpoint{1.702822in}{0.700770in}}%
\pgfpathcurveto{\pgfqpoint{1.713718in}{0.700770in}}{\pgfqpoint{1.724169in}{0.705099in}}{\pgfqpoint{1.731873in}{0.712803in}}%
\pgfpathcurveto{\pgfqpoint{1.739577in}{0.720508in}}{\pgfqpoint{1.743906in}{0.730958in}}{\pgfqpoint{1.743906in}{0.741854in}}%
\pgfpathcurveto{\pgfqpoint{1.743906in}{0.752750in}}{\pgfqpoint{1.739577in}{0.763200in}}{\pgfqpoint{1.731873in}{0.770905in}}%
\pgfpathcurveto{\pgfqpoint{1.724169in}{0.778609in}}{\pgfqpoint{1.713718in}{0.782938in}}{\pgfqpoint{1.702822in}{0.782938in}}%
\pgfpathcurveto{\pgfqpoint{1.691927in}{0.782938in}}{\pgfqpoint{1.681476in}{0.778609in}}{\pgfqpoint{1.673772in}{0.770905in}}%
\pgfpathcurveto{\pgfqpoint{1.666067in}{0.763200in}}{\pgfqpoint{1.661738in}{0.752750in}}{\pgfqpoint{1.661738in}{0.741854in}}%
\pgfpathcurveto{\pgfqpoint{1.661738in}{0.730958in}}{\pgfqpoint{1.666067in}{0.720508in}}{\pgfqpoint{1.673772in}{0.712803in}}%
\pgfpathcurveto{\pgfqpoint{1.681476in}{0.705099in}}{\pgfqpoint{1.691927in}{0.700770in}}{\pgfqpoint{1.702822in}{0.700770in}}%
\pgfpathlineto{\pgfqpoint{1.702822in}{0.700770in}}%
\pgfpathclose%
\pgfusepath{stroke}%
\end{pgfscope}%
\begin{pgfscope}%
\pgfpathrectangle{\pgfqpoint{0.688192in}{0.670138in}}{\pgfqpoint{7.111808in}{5.129862in}}%
\pgfusepath{clip}%
\pgfsetbuttcap%
\pgfsetroundjoin%
\pgfsetlinewidth{1.003750pt}%
\definecolor{currentstroke}{rgb}{0.000000,0.000000,0.000000}%
\pgfsetstrokecolor{currentstroke}%
\pgfsetdash{}{0pt}%
\pgfpathmoveto{\pgfqpoint{2.481642in}{4.039571in}}%
\pgfpathcurveto{\pgfqpoint{2.492537in}{4.039571in}}{\pgfqpoint{2.502988in}{4.043900in}}{\pgfqpoint{2.510692in}{4.051604in}}%
\pgfpathcurveto{\pgfqpoint{2.518397in}{4.059309in}}{\pgfqpoint{2.522726in}{4.069759in}}{\pgfqpoint{2.522726in}{4.080655in}}%
\pgfpathcurveto{\pgfqpoint{2.522726in}{4.091551in}}{\pgfqpoint{2.518397in}{4.102001in}}{\pgfqpoint{2.510692in}{4.109706in}}%
\pgfpathcurveto{\pgfqpoint{2.502988in}{4.117410in}}{\pgfqpoint{2.492537in}{4.121739in}}{\pgfqpoint{2.481642in}{4.121739in}}%
\pgfpathcurveto{\pgfqpoint{2.470746in}{4.121739in}}{\pgfqpoint{2.460295in}{4.117410in}}{\pgfqpoint{2.452591in}{4.109706in}}%
\pgfpathcurveto{\pgfqpoint{2.444887in}{4.102001in}}{\pgfqpoint{2.440558in}{4.091551in}}{\pgfqpoint{2.440558in}{4.080655in}}%
\pgfpathcurveto{\pgfqpoint{2.440558in}{4.069759in}}{\pgfqpoint{2.444887in}{4.059309in}}{\pgfqpoint{2.452591in}{4.051604in}}%
\pgfpathcurveto{\pgfqpoint{2.460295in}{4.043900in}}{\pgfqpoint{2.470746in}{4.039571in}}{\pgfqpoint{2.481642in}{4.039571in}}%
\pgfpathlineto{\pgfqpoint{2.481642in}{4.039571in}}%
\pgfpathclose%
\pgfusepath{stroke}%
\end{pgfscope}%
\begin{pgfscope}%
\pgfpathrectangle{\pgfqpoint{0.688192in}{0.670138in}}{\pgfqpoint{7.111808in}{5.129862in}}%
\pgfusepath{clip}%
\pgfsetbuttcap%
\pgfsetroundjoin%
\pgfsetlinewidth{1.003750pt}%
\definecolor{currentstroke}{rgb}{0.000000,0.000000,0.000000}%
\pgfsetstrokecolor{currentstroke}%
\pgfsetdash{}{0pt}%
\pgfpathmoveto{\pgfqpoint{1.278130in}{0.714122in}}%
\pgfpathcurveto{\pgfqpoint{1.289026in}{0.714122in}}{\pgfqpoint{1.299477in}{0.718451in}}{\pgfqpoint{1.307181in}{0.726155in}}%
\pgfpathcurveto{\pgfqpoint{1.314885in}{0.733860in}}{\pgfqpoint{1.319214in}{0.744311in}}{\pgfqpoint{1.319214in}{0.755206in}}%
\pgfpathcurveto{\pgfqpoint{1.319214in}{0.766102in}}{\pgfqpoint{1.314885in}{0.776553in}}{\pgfqpoint{1.307181in}{0.784257in}}%
\pgfpathcurveto{\pgfqpoint{1.299477in}{0.791961in}}{\pgfqpoint{1.289026in}{0.796290in}}{\pgfqpoint{1.278130in}{0.796290in}}%
\pgfpathcurveto{\pgfqpoint{1.267235in}{0.796290in}}{\pgfqpoint{1.256784in}{0.791961in}}{\pgfqpoint{1.249080in}{0.784257in}}%
\pgfpathcurveto{\pgfqpoint{1.241375in}{0.776553in}}{\pgfqpoint{1.237047in}{0.766102in}}{\pgfqpoint{1.237047in}{0.755206in}}%
\pgfpathcurveto{\pgfqpoint{1.237047in}{0.744311in}}{\pgfqpoint{1.241375in}{0.733860in}}{\pgfqpoint{1.249080in}{0.726155in}}%
\pgfpathcurveto{\pgfqpoint{1.256784in}{0.718451in}}{\pgfqpoint{1.267235in}{0.714122in}}{\pgfqpoint{1.278130in}{0.714122in}}%
\pgfpathlineto{\pgfqpoint{1.278130in}{0.714122in}}%
\pgfpathclose%
\pgfusepath{stroke}%
\end{pgfscope}%
\begin{pgfscope}%
\pgfpathrectangle{\pgfqpoint{0.688192in}{0.670138in}}{\pgfqpoint{7.111808in}{5.129862in}}%
\pgfusepath{clip}%
\pgfsetbuttcap%
\pgfsetroundjoin%
\pgfsetlinewidth{1.003750pt}%
\definecolor{currentstroke}{rgb}{0.000000,0.000000,0.000000}%
\pgfsetstrokecolor{currentstroke}%
\pgfsetdash{}{0pt}%
\pgfpathmoveto{\pgfqpoint{0.901363in}{0.856548in}}%
\pgfpathcurveto{\pgfqpoint{0.912259in}{0.856548in}}{\pgfqpoint{0.922710in}{0.860877in}}{\pgfqpoint{0.930414in}{0.868582in}}%
\pgfpathcurveto{\pgfqpoint{0.938118in}{0.876286in}}{\pgfqpoint{0.942447in}{0.886737in}}{\pgfqpoint{0.942447in}{0.897632in}}%
\pgfpathcurveto{\pgfqpoint{0.942447in}{0.908528in}}{\pgfqpoint{0.938118in}{0.918979in}}{\pgfqpoint{0.930414in}{0.926683in}}%
\pgfpathcurveto{\pgfqpoint{0.922710in}{0.934387in}}{\pgfqpoint{0.912259in}{0.938716in}}{\pgfqpoint{0.901363in}{0.938716in}}%
\pgfpathcurveto{\pgfqpoint{0.890468in}{0.938716in}}{\pgfqpoint{0.880017in}{0.934387in}}{\pgfqpoint{0.872313in}{0.926683in}}%
\pgfpathcurveto{\pgfqpoint{0.864608in}{0.918979in}}{\pgfqpoint{0.860280in}{0.908528in}}{\pgfqpoint{0.860280in}{0.897632in}}%
\pgfpathcurveto{\pgfqpoint{0.860280in}{0.886737in}}{\pgfqpoint{0.864608in}{0.876286in}}{\pgfqpoint{0.872313in}{0.868582in}}%
\pgfpathcurveto{\pgfqpoint{0.880017in}{0.860877in}}{\pgfqpoint{0.890468in}{0.856548in}}{\pgfqpoint{0.901363in}{0.856548in}}%
\pgfpathlineto{\pgfqpoint{0.901363in}{0.856548in}}%
\pgfpathclose%
\pgfusepath{stroke}%
\end{pgfscope}%
\begin{pgfscope}%
\pgfpathrectangle{\pgfqpoint{0.688192in}{0.670138in}}{\pgfqpoint{7.111808in}{5.129862in}}%
\pgfusepath{clip}%
\pgfsetbuttcap%
\pgfsetroundjoin%
\pgfsetlinewidth{1.003750pt}%
\definecolor{currentstroke}{rgb}{0.000000,0.000000,0.000000}%
\pgfsetstrokecolor{currentstroke}%
\pgfsetdash{}{0pt}%
\pgfpathmoveto{\pgfqpoint{1.307540in}{0.713537in}}%
\pgfpathcurveto{\pgfqpoint{1.318436in}{0.713537in}}{\pgfqpoint{1.328887in}{0.717866in}}{\pgfqpoint{1.336591in}{0.725570in}}%
\pgfpathcurveto{\pgfqpoint{1.344295in}{0.733275in}}{\pgfqpoint{1.348624in}{0.743725in}}{\pgfqpoint{1.348624in}{0.754621in}}%
\pgfpathcurveto{\pgfqpoint{1.348624in}{0.765517in}}{\pgfqpoint{1.344295in}{0.775967in}}{\pgfqpoint{1.336591in}{0.783672in}}%
\pgfpathcurveto{\pgfqpoint{1.328887in}{0.791376in}}{\pgfqpoint{1.318436in}{0.795705in}}{\pgfqpoint{1.307540in}{0.795705in}}%
\pgfpathcurveto{\pgfqpoint{1.296645in}{0.795705in}}{\pgfqpoint{1.286194in}{0.791376in}}{\pgfqpoint{1.278489in}{0.783672in}}%
\pgfpathcurveto{\pgfqpoint{1.270785in}{0.775967in}}{\pgfqpoint{1.266456in}{0.765517in}}{\pgfqpoint{1.266456in}{0.754621in}}%
\pgfpathcurveto{\pgfqpoint{1.266456in}{0.743725in}}{\pgfqpoint{1.270785in}{0.733275in}}{\pgfqpoint{1.278489in}{0.725570in}}%
\pgfpathcurveto{\pgfqpoint{1.286194in}{0.717866in}}{\pgfqpoint{1.296645in}{0.713537in}}{\pgfqpoint{1.307540in}{0.713537in}}%
\pgfpathlineto{\pgfqpoint{1.307540in}{0.713537in}}%
\pgfpathclose%
\pgfusepath{stroke}%
\end{pgfscope}%
\begin{pgfscope}%
\pgfpathrectangle{\pgfqpoint{0.688192in}{0.670138in}}{\pgfqpoint{7.111808in}{5.129862in}}%
\pgfusepath{clip}%
\pgfsetbuttcap%
\pgfsetroundjoin%
\pgfsetlinewidth{1.003750pt}%
\definecolor{currentstroke}{rgb}{0.000000,0.000000,0.000000}%
\pgfsetstrokecolor{currentstroke}%
\pgfsetdash{}{0pt}%
\pgfpathmoveto{\pgfqpoint{1.070830in}{0.725886in}}%
\pgfpathcurveto{\pgfqpoint{1.081726in}{0.725886in}}{\pgfqpoint{1.092176in}{0.730215in}}{\pgfqpoint{1.099881in}{0.737920in}}%
\pgfpathcurveto{\pgfqpoint{1.107585in}{0.745624in}}{\pgfqpoint{1.111914in}{0.756075in}}{\pgfqpoint{1.111914in}{0.766970in}}%
\pgfpathcurveto{\pgfqpoint{1.111914in}{0.777866in}}{\pgfqpoint{1.107585in}{0.788317in}}{\pgfqpoint{1.099881in}{0.796021in}}%
\pgfpathcurveto{\pgfqpoint{1.092176in}{0.803725in}}{\pgfqpoint{1.081726in}{0.808054in}}{\pgfqpoint{1.070830in}{0.808054in}}%
\pgfpathcurveto{\pgfqpoint{1.059934in}{0.808054in}}{\pgfqpoint{1.049484in}{0.803725in}}{\pgfqpoint{1.041779in}{0.796021in}}%
\pgfpathcurveto{\pgfqpoint{1.034075in}{0.788317in}}{\pgfqpoint{1.029746in}{0.777866in}}{\pgfqpoint{1.029746in}{0.766970in}}%
\pgfpathcurveto{\pgfqpoint{1.029746in}{0.756075in}}{\pgfqpoint{1.034075in}{0.745624in}}{\pgfqpoint{1.041779in}{0.737920in}}%
\pgfpathcurveto{\pgfqpoint{1.049484in}{0.730215in}}{\pgfqpoint{1.059934in}{0.725886in}}{\pgfqpoint{1.070830in}{0.725886in}}%
\pgfpathlineto{\pgfqpoint{1.070830in}{0.725886in}}%
\pgfpathclose%
\pgfusepath{stroke}%
\end{pgfscope}%
\begin{pgfscope}%
\pgfpathrectangle{\pgfqpoint{0.688192in}{0.670138in}}{\pgfqpoint{7.111808in}{5.129862in}}%
\pgfusepath{clip}%
\pgfsetbuttcap%
\pgfsetroundjoin%
\pgfsetlinewidth{1.003750pt}%
\definecolor{currentstroke}{rgb}{0.000000,0.000000,0.000000}%
\pgfsetstrokecolor{currentstroke}%
\pgfsetdash{}{0pt}%
\pgfpathmoveto{\pgfqpoint{2.497167in}{0.675229in}}%
\pgfpathcurveto{\pgfqpoint{2.508063in}{0.675229in}}{\pgfqpoint{2.518514in}{0.679557in}}{\pgfqpoint{2.526218in}{0.687262in}}%
\pgfpathcurveto{\pgfqpoint{2.533922in}{0.694966in}}{\pgfqpoint{2.538251in}{0.705417in}}{\pgfqpoint{2.538251in}{0.716312in}}%
\pgfpathcurveto{\pgfqpoint{2.538251in}{0.727208in}}{\pgfqpoint{2.533922in}{0.737659in}}{\pgfqpoint{2.526218in}{0.745363in}}%
\pgfpathcurveto{\pgfqpoint{2.518514in}{0.753067in}}{\pgfqpoint{2.508063in}{0.757396in}}{\pgfqpoint{2.497167in}{0.757396in}}%
\pgfpathcurveto{\pgfqpoint{2.486272in}{0.757396in}}{\pgfqpoint{2.475821in}{0.753067in}}{\pgfqpoint{2.468117in}{0.745363in}}%
\pgfpathcurveto{\pgfqpoint{2.460412in}{0.737659in}}{\pgfqpoint{2.456084in}{0.727208in}}{\pgfqpoint{2.456084in}{0.716312in}}%
\pgfpathcurveto{\pgfqpoint{2.456084in}{0.705417in}}{\pgfqpoint{2.460412in}{0.694966in}}{\pgfqpoint{2.468117in}{0.687262in}}%
\pgfpathcurveto{\pgfqpoint{2.475821in}{0.679557in}}{\pgfqpoint{2.486272in}{0.675229in}}{\pgfqpoint{2.497167in}{0.675229in}}%
\pgfpathlineto{\pgfqpoint{2.497167in}{0.675229in}}%
\pgfpathclose%
\pgfusepath{stroke}%
\end{pgfscope}%
\begin{pgfscope}%
\pgfpathrectangle{\pgfqpoint{0.688192in}{0.670138in}}{\pgfqpoint{7.111808in}{5.129862in}}%
\pgfusepath{clip}%
\pgfsetbuttcap%
\pgfsetroundjoin%
\pgfsetlinewidth{1.003750pt}%
\definecolor{currentstroke}{rgb}{0.000000,0.000000,0.000000}%
\pgfsetstrokecolor{currentstroke}%
\pgfsetdash{}{0pt}%
\pgfpathmoveto{\pgfqpoint{1.389197in}{0.711266in}}%
\pgfpathcurveto{\pgfqpoint{1.400093in}{0.711266in}}{\pgfqpoint{1.410544in}{0.715595in}}{\pgfqpoint{1.418248in}{0.723299in}}%
\pgfpathcurveto{\pgfqpoint{1.425952in}{0.731004in}}{\pgfqpoint{1.430281in}{0.741454in}}{\pgfqpoint{1.430281in}{0.752350in}}%
\pgfpathcurveto{\pgfqpoint{1.430281in}{0.763246in}}{\pgfqpoint{1.425952in}{0.773696in}}{\pgfqpoint{1.418248in}{0.781401in}}%
\pgfpathcurveto{\pgfqpoint{1.410544in}{0.789105in}}{\pgfqpoint{1.400093in}{0.793434in}}{\pgfqpoint{1.389197in}{0.793434in}}%
\pgfpathcurveto{\pgfqpoint{1.378302in}{0.793434in}}{\pgfqpoint{1.367851in}{0.789105in}}{\pgfqpoint{1.360147in}{0.781401in}}%
\pgfpathcurveto{\pgfqpoint{1.352442in}{0.773696in}}{\pgfqpoint{1.348113in}{0.763246in}}{\pgfqpoint{1.348113in}{0.752350in}}%
\pgfpathcurveto{\pgfqpoint{1.348113in}{0.741454in}}{\pgfqpoint{1.352442in}{0.731004in}}{\pgfqpoint{1.360147in}{0.723299in}}%
\pgfpathcurveto{\pgfqpoint{1.367851in}{0.715595in}}{\pgfqpoint{1.378302in}{0.711266in}}{\pgfqpoint{1.389197in}{0.711266in}}%
\pgfpathlineto{\pgfqpoint{1.389197in}{0.711266in}}%
\pgfpathclose%
\pgfusepath{stroke}%
\end{pgfscope}%
\begin{pgfscope}%
\pgfpathrectangle{\pgfqpoint{0.688192in}{0.670138in}}{\pgfqpoint{7.111808in}{5.129862in}}%
\pgfusepath{clip}%
\pgfsetbuttcap%
\pgfsetroundjoin%
\pgfsetlinewidth{1.003750pt}%
\definecolor{currentstroke}{rgb}{0.000000,0.000000,0.000000}%
\pgfsetstrokecolor{currentstroke}%
\pgfsetdash{}{0pt}%
\pgfpathmoveto{\pgfqpoint{4.462854in}{0.641380in}}%
\pgfpathcurveto{\pgfqpoint{4.473749in}{0.641380in}}{\pgfqpoint{4.484200in}{0.645709in}}{\pgfqpoint{4.491904in}{0.653414in}}%
\pgfpathcurveto{\pgfqpoint{4.499609in}{0.661118in}}{\pgfqpoint{4.503938in}{0.671569in}}{\pgfqpoint{4.503938in}{0.682464in}}%
\pgfpathcurveto{\pgfqpoint{4.503938in}{0.693360in}}{\pgfqpoint{4.499609in}{0.703811in}}{\pgfqpoint{4.491904in}{0.711515in}}%
\pgfpathcurveto{\pgfqpoint{4.484200in}{0.719219in}}{\pgfqpoint{4.473749in}{0.723548in}}{\pgfqpoint{4.462854in}{0.723548in}}%
\pgfpathcurveto{\pgfqpoint{4.451958in}{0.723548in}}{\pgfqpoint{4.441507in}{0.719219in}}{\pgfqpoint{4.433803in}{0.711515in}}%
\pgfpathcurveto{\pgfqpoint{4.426099in}{0.703811in}}{\pgfqpoint{4.421770in}{0.693360in}}{\pgfqpoint{4.421770in}{0.682464in}}%
\pgfpathcurveto{\pgfqpoint{4.421770in}{0.671569in}}{\pgfqpoint{4.426099in}{0.661118in}}{\pgfqpoint{4.433803in}{0.653414in}}%
\pgfpathcurveto{\pgfqpoint{4.441507in}{0.645709in}}{\pgfqpoint{4.451958in}{0.641380in}}{\pgfqpoint{4.462854in}{0.641380in}}%
\pgfusepath{stroke}%
\end{pgfscope}%
\begin{pgfscope}%
\pgfpathrectangle{\pgfqpoint{0.688192in}{0.670138in}}{\pgfqpoint{7.111808in}{5.129862in}}%
\pgfusepath{clip}%
\pgfsetbuttcap%
\pgfsetroundjoin%
\pgfsetlinewidth{1.003750pt}%
\definecolor{currentstroke}{rgb}{0.000000,0.000000,0.000000}%
\pgfsetstrokecolor{currentstroke}%
\pgfsetdash{}{0pt}%
\pgfpathmoveto{\pgfqpoint{0.774443in}{1.098542in}}%
\pgfpathcurveto{\pgfqpoint{0.785338in}{1.098542in}}{\pgfqpoint{0.795789in}{1.102871in}}{\pgfqpoint{0.803493in}{1.110576in}}%
\pgfpathcurveto{\pgfqpoint{0.811198in}{1.118280in}}{\pgfqpoint{0.815526in}{1.128731in}}{\pgfqpoint{0.815526in}{1.139626in}}%
\pgfpathcurveto{\pgfqpoint{0.815526in}{1.150522in}}{\pgfqpoint{0.811198in}{1.160973in}}{\pgfqpoint{0.803493in}{1.168677in}}%
\pgfpathcurveto{\pgfqpoint{0.795789in}{1.176381in}}{\pgfqpoint{0.785338in}{1.180710in}}{\pgfqpoint{0.774443in}{1.180710in}}%
\pgfpathcurveto{\pgfqpoint{0.763547in}{1.180710in}}{\pgfqpoint{0.753096in}{1.176381in}}{\pgfqpoint{0.745392in}{1.168677in}}%
\pgfpathcurveto{\pgfqpoint{0.737688in}{1.160973in}}{\pgfqpoint{0.733359in}{1.150522in}}{\pgfqpoint{0.733359in}{1.139626in}}%
\pgfpathcurveto{\pgfqpoint{0.733359in}{1.128731in}}{\pgfqpoint{0.737688in}{1.118280in}}{\pgfqpoint{0.745392in}{1.110576in}}%
\pgfpathcurveto{\pgfqpoint{0.753096in}{1.102871in}}{\pgfqpoint{0.763547in}{1.098542in}}{\pgfqpoint{0.774443in}{1.098542in}}%
\pgfpathlineto{\pgfqpoint{0.774443in}{1.098542in}}%
\pgfpathclose%
\pgfusepath{stroke}%
\end{pgfscope}%
\begin{pgfscope}%
\pgfpathrectangle{\pgfqpoint{0.688192in}{0.670138in}}{\pgfqpoint{7.111808in}{5.129862in}}%
\pgfusepath{clip}%
\pgfsetbuttcap%
\pgfsetroundjoin%
\pgfsetlinewidth{1.003750pt}%
\definecolor{currentstroke}{rgb}{0.000000,0.000000,0.000000}%
\pgfsetstrokecolor{currentstroke}%
\pgfsetdash{}{0pt}%
\pgfpathmoveto{\pgfqpoint{2.464002in}{0.676386in}}%
\pgfpathcurveto{\pgfqpoint{2.474898in}{0.676386in}}{\pgfqpoint{2.485348in}{0.680715in}}{\pgfqpoint{2.493053in}{0.688419in}}%
\pgfpathcurveto{\pgfqpoint{2.500757in}{0.696124in}}{\pgfqpoint{2.505086in}{0.706574in}}{\pgfqpoint{2.505086in}{0.717470in}}%
\pgfpathcurveto{\pgfqpoint{2.505086in}{0.728366in}}{\pgfqpoint{2.500757in}{0.738816in}}{\pgfqpoint{2.493053in}{0.746521in}}%
\pgfpathcurveto{\pgfqpoint{2.485348in}{0.754225in}}{\pgfqpoint{2.474898in}{0.758554in}}{\pgfqpoint{2.464002in}{0.758554in}}%
\pgfpathcurveto{\pgfqpoint{2.453107in}{0.758554in}}{\pgfqpoint{2.442656in}{0.754225in}}{\pgfqpoint{2.434951in}{0.746521in}}%
\pgfpathcurveto{\pgfqpoint{2.427247in}{0.738816in}}{\pgfqpoint{2.422918in}{0.728366in}}{\pgfqpoint{2.422918in}{0.717470in}}%
\pgfpathcurveto{\pgfqpoint{2.422918in}{0.706574in}}{\pgfqpoint{2.427247in}{0.696124in}}{\pgfqpoint{2.434951in}{0.688419in}}%
\pgfpathcurveto{\pgfqpoint{2.442656in}{0.680715in}}{\pgfqpoint{2.453107in}{0.676386in}}{\pgfqpoint{2.464002in}{0.676386in}}%
\pgfpathlineto{\pgfqpoint{2.464002in}{0.676386in}}%
\pgfpathclose%
\pgfusepath{stroke}%
\end{pgfscope}%
\begin{pgfscope}%
\pgfpathrectangle{\pgfqpoint{0.688192in}{0.670138in}}{\pgfqpoint{7.111808in}{5.129862in}}%
\pgfusepath{clip}%
\pgfsetbuttcap%
\pgfsetroundjoin%
\pgfsetlinewidth{1.003750pt}%
\definecolor{currentstroke}{rgb}{0.000000,0.000000,0.000000}%
\pgfsetstrokecolor{currentstroke}%
\pgfsetdash{}{0pt}%
\pgfpathmoveto{\pgfqpoint{4.462854in}{0.641380in}}%
\pgfpathcurveto{\pgfqpoint{4.473749in}{0.641380in}}{\pgfqpoint{4.484200in}{0.645709in}}{\pgfqpoint{4.491904in}{0.653414in}}%
\pgfpathcurveto{\pgfqpoint{4.499609in}{0.661118in}}{\pgfqpoint{4.503938in}{0.671569in}}{\pgfqpoint{4.503938in}{0.682464in}}%
\pgfpathcurveto{\pgfqpoint{4.503938in}{0.693360in}}{\pgfqpoint{4.499609in}{0.703811in}}{\pgfqpoint{4.491904in}{0.711515in}}%
\pgfpathcurveto{\pgfqpoint{4.484200in}{0.719219in}}{\pgfqpoint{4.473749in}{0.723548in}}{\pgfqpoint{4.462854in}{0.723548in}}%
\pgfpathcurveto{\pgfqpoint{4.451958in}{0.723548in}}{\pgfqpoint{4.441507in}{0.719219in}}{\pgfqpoint{4.433803in}{0.711515in}}%
\pgfpathcurveto{\pgfqpoint{4.426099in}{0.703811in}}{\pgfqpoint{4.421770in}{0.693360in}}{\pgfqpoint{4.421770in}{0.682464in}}%
\pgfpathcurveto{\pgfqpoint{4.421770in}{0.671569in}}{\pgfqpoint{4.426099in}{0.661118in}}{\pgfqpoint{4.433803in}{0.653414in}}%
\pgfpathcurveto{\pgfqpoint{4.441507in}{0.645709in}}{\pgfqpoint{4.451958in}{0.641380in}}{\pgfqpoint{4.462854in}{0.641380in}}%
\pgfusepath{stroke}%
\end{pgfscope}%
\begin{pgfscope}%
\pgfpathrectangle{\pgfqpoint{0.688192in}{0.670138in}}{\pgfqpoint{7.111808in}{5.129862in}}%
\pgfusepath{clip}%
\pgfsetbuttcap%
\pgfsetroundjoin%
\pgfsetlinewidth{1.003750pt}%
\definecolor{currentstroke}{rgb}{0.000000,0.000000,0.000000}%
\pgfsetstrokecolor{currentstroke}%
\pgfsetdash{}{0pt}%
\pgfpathmoveto{\pgfqpoint{0.895312in}{0.866174in}}%
\pgfpathcurveto{\pgfqpoint{0.906208in}{0.866174in}}{\pgfqpoint{0.916658in}{0.870503in}}{\pgfqpoint{0.924363in}{0.878207in}}%
\pgfpathcurveto{\pgfqpoint{0.932067in}{0.885911in}}{\pgfqpoint{0.936396in}{0.896362in}}{\pgfqpoint{0.936396in}{0.907258in}}%
\pgfpathcurveto{\pgfqpoint{0.936396in}{0.918153in}}{\pgfqpoint{0.932067in}{0.928604in}}{\pgfqpoint{0.924363in}{0.936308in}}%
\pgfpathcurveto{\pgfqpoint{0.916658in}{0.944013in}}{\pgfqpoint{0.906208in}{0.948341in}}{\pgfqpoint{0.895312in}{0.948341in}}%
\pgfpathcurveto{\pgfqpoint{0.884416in}{0.948341in}}{\pgfqpoint{0.873966in}{0.944013in}}{\pgfqpoint{0.866261in}{0.936308in}}%
\pgfpathcurveto{\pgfqpoint{0.858557in}{0.928604in}}{\pgfqpoint{0.854228in}{0.918153in}}{\pgfqpoint{0.854228in}{0.907258in}}%
\pgfpathcurveto{\pgfqpoint{0.854228in}{0.896362in}}{\pgfqpoint{0.858557in}{0.885911in}}{\pgfqpoint{0.866261in}{0.878207in}}%
\pgfpathcurveto{\pgfqpoint{0.873966in}{0.870503in}}{\pgfqpoint{0.884416in}{0.866174in}}{\pgfqpoint{0.895312in}{0.866174in}}%
\pgfpathlineto{\pgfqpoint{0.895312in}{0.866174in}}%
\pgfpathclose%
\pgfusepath{stroke}%
\end{pgfscope}%
\begin{pgfscope}%
\pgfpathrectangle{\pgfqpoint{0.688192in}{0.670138in}}{\pgfqpoint{7.111808in}{5.129862in}}%
\pgfusepath{clip}%
\pgfsetbuttcap%
\pgfsetroundjoin%
\pgfsetlinewidth{1.003750pt}%
\definecolor{currentstroke}{rgb}{0.000000,0.000000,0.000000}%
\pgfsetstrokecolor{currentstroke}%
\pgfsetdash{}{0pt}%
\pgfpathmoveto{\pgfqpoint{0.812030in}{0.993250in}}%
\pgfpathcurveto{\pgfqpoint{0.822926in}{0.993250in}}{\pgfqpoint{0.833377in}{0.997579in}}{\pgfqpoint{0.841081in}{1.005284in}}%
\pgfpathcurveto{\pgfqpoint{0.848785in}{1.012988in}}{\pgfqpoint{0.853114in}{1.023439in}}{\pgfqpoint{0.853114in}{1.034334in}}%
\pgfpathcurveto{\pgfqpoint{0.853114in}{1.045230in}}{\pgfqpoint{0.848785in}{1.055681in}}{\pgfqpoint{0.841081in}{1.063385in}}%
\pgfpathcurveto{\pgfqpoint{0.833377in}{1.071089in}}{\pgfqpoint{0.822926in}{1.075418in}}{\pgfqpoint{0.812030in}{1.075418in}}%
\pgfpathcurveto{\pgfqpoint{0.801135in}{1.075418in}}{\pgfqpoint{0.790684in}{1.071089in}}{\pgfqpoint{0.782980in}{1.063385in}}%
\pgfpathcurveto{\pgfqpoint{0.775275in}{1.055681in}}{\pgfqpoint{0.770946in}{1.045230in}}{\pgfqpoint{0.770946in}{1.034334in}}%
\pgfpathcurveto{\pgfqpoint{0.770946in}{1.023439in}}{\pgfqpoint{0.775275in}{1.012988in}}{\pgfqpoint{0.782980in}{1.005284in}}%
\pgfpathcurveto{\pgfqpoint{0.790684in}{0.997579in}}{\pgfqpoint{0.801135in}{0.993250in}}{\pgfqpoint{0.812030in}{0.993250in}}%
\pgfpathlineto{\pgfqpoint{0.812030in}{0.993250in}}%
\pgfpathclose%
\pgfusepath{stroke}%
\end{pgfscope}%
\begin{pgfscope}%
\pgfpathrectangle{\pgfqpoint{0.688192in}{0.670138in}}{\pgfqpoint{7.111808in}{5.129862in}}%
\pgfusepath{clip}%
\pgfsetbuttcap%
\pgfsetroundjoin%
\pgfsetlinewidth{1.003750pt}%
\definecolor{currentstroke}{rgb}{0.000000,0.000000,0.000000}%
\pgfsetstrokecolor{currentstroke}%
\pgfsetdash{}{0pt}%
\pgfpathmoveto{\pgfqpoint{1.389197in}{0.711266in}}%
\pgfpathcurveto{\pgfqpoint{1.400093in}{0.711266in}}{\pgfqpoint{1.410544in}{0.715595in}}{\pgfqpoint{1.418248in}{0.723299in}}%
\pgfpathcurveto{\pgfqpoint{1.425952in}{0.731004in}}{\pgfqpoint{1.430281in}{0.741454in}}{\pgfqpoint{1.430281in}{0.752350in}}%
\pgfpathcurveto{\pgfqpoint{1.430281in}{0.763246in}}{\pgfqpoint{1.425952in}{0.773696in}}{\pgfqpoint{1.418248in}{0.781401in}}%
\pgfpathcurveto{\pgfqpoint{1.410544in}{0.789105in}}{\pgfqpoint{1.400093in}{0.793434in}}{\pgfqpoint{1.389197in}{0.793434in}}%
\pgfpathcurveto{\pgfqpoint{1.378302in}{0.793434in}}{\pgfqpoint{1.367851in}{0.789105in}}{\pgfqpoint{1.360147in}{0.781401in}}%
\pgfpathcurveto{\pgfqpoint{1.352442in}{0.773696in}}{\pgfqpoint{1.348113in}{0.763246in}}{\pgfqpoint{1.348113in}{0.752350in}}%
\pgfpathcurveto{\pgfqpoint{1.348113in}{0.741454in}}{\pgfqpoint{1.352442in}{0.731004in}}{\pgfqpoint{1.360147in}{0.723299in}}%
\pgfpathcurveto{\pgfqpoint{1.367851in}{0.715595in}}{\pgfqpoint{1.378302in}{0.711266in}}{\pgfqpoint{1.389197in}{0.711266in}}%
\pgfpathlineto{\pgfqpoint{1.389197in}{0.711266in}}%
\pgfpathclose%
\pgfusepath{stroke}%
\end{pgfscope}%
\begin{pgfscope}%
\pgfpathrectangle{\pgfqpoint{0.688192in}{0.670138in}}{\pgfqpoint{7.111808in}{5.129862in}}%
\pgfusepath{clip}%
\pgfsetbuttcap%
\pgfsetroundjoin%
\pgfsetlinewidth{1.003750pt}%
\definecolor{currentstroke}{rgb}{0.000000,0.000000,0.000000}%
\pgfsetstrokecolor{currentstroke}%
\pgfsetdash{}{0pt}%
\pgfpathmoveto{\pgfqpoint{4.462854in}{0.641380in}}%
\pgfpathcurveto{\pgfqpoint{4.473749in}{0.641380in}}{\pgfqpoint{4.484200in}{0.645709in}}{\pgfqpoint{4.491904in}{0.653414in}}%
\pgfpathcurveto{\pgfqpoint{4.499609in}{0.661118in}}{\pgfqpoint{4.503938in}{0.671569in}}{\pgfqpoint{4.503938in}{0.682464in}}%
\pgfpathcurveto{\pgfqpoint{4.503938in}{0.693360in}}{\pgfqpoint{4.499609in}{0.703811in}}{\pgfqpoint{4.491904in}{0.711515in}}%
\pgfpathcurveto{\pgfqpoint{4.484200in}{0.719219in}}{\pgfqpoint{4.473749in}{0.723548in}}{\pgfqpoint{4.462854in}{0.723548in}}%
\pgfpathcurveto{\pgfqpoint{4.451958in}{0.723548in}}{\pgfqpoint{4.441507in}{0.719219in}}{\pgfqpoint{4.433803in}{0.711515in}}%
\pgfpathcurveto{\pgfqpoint{4.426099in}{0.703811in}}{\pgfqpoint{4.421770in}{0.693360in}}{\pgfqpoint{4.421770in}{0.682464in}}%
\pgfpathcurveto{\pgfqpoint{4.421770in}{0.671569in}}{\pgfqpoint{4.426099in}{0.661118in}}{\pgfqpoint{4.433803in}{0.653414in}}%
\pgfpathcurveto{\pgfqpoint{4.441507in}{0.645709in}}{\pgfqpoint{4.451958in}{0.641380in}}{\pgfqpoint{4.462854in}{0.641380in}}%
\pgfusepath{stroke}%
\end{pgfscope}%
\begin{pgfscope}%
\pgfpathrectangle{\pgfqpoint{0.688192in}{0.670138in}}{\pgfqpoint{7.111808in}{5.129862in}}%
\pgfusepath{clip}%
\pgfsetbuttcap%
\pgfsetroundjoin%
\pgfsetlinewidth{1.003750pt}%
\definecolor{currentstroke}{rgb}{0.000000,0.000000,0.000000}%
\pgfsetstrokecolor{currentstroke}%
\pgfsetdash{}{0pt}%
\pgfpathmoveto{\pgfqpoint{1.419139in}{0.710712in}}%
\pgfpathcurveto{\pgfqpoint{1.430035in}{0.710712in}}{\pgfqpoint{1.440485in}{0.715041in}}{\pgfqpoint{1.448190in}{0.722745in}}%
\pgfpathcurveto{\pgfqpoint{1.455894in}{0.730450in}}{\pgfqpoint{1.460223in}{0.740900in}}{\pgfqpoint{1.460223in}{0.751796in}}%
\pgfpathcurveto{\pgfqpoint{1.460223in}{0.762691in}}{\pgfqpoint{1.455894in}{0.773142in}}{\pgfqpoint{1.448190in}{0.780847in}}%
\pgfpathcurveto{\pgfqpoint{1.440485in}{0.788551in}}{\pgfqpoint{1.430035in}{0.792880in}}{\pgfqpoint{1.419139in}{0.792880in}}%
\pgfpathcurveto{\pgfqpoint{1.408244in}{0.792880in}}{\pgfqpoint{1.397793in}{0.788551in}}{\pgfqpoint{1.390088in}{0.780847in}}%
\pgfpathcurveto{\pgfqpoint{1.382384in}{0.773142in}}{\pgfqpoint{1.378055in}{0.762691in}}{\pgfqpoint{1.378055in}{0.751796in}}%
\pgfpathcurveto{\pgfqpoint{1.378055in}{0.740900in}}{\pgfqpoint{1.382384in}{0.730450in}}{\pgfqpoint{1.390088in}{0.722745in}}%
\pgfpathcurveto{\pgfqpoint{1.397793in}{0.715041in}}{\pgfqpoint{1.408244in}{0.710712in}}{\pgfqpoint{1.419139in}{0.710712in}}%
\pgfpathlineto{\pgfqpoint{1.419139in}{0.710712in}}%
\pgfpathclose%
\pgfusepath{stroke}%
\end{pgfscope}%
\begin{pgfscope}%
\pgfpathrectangle{\pgfqpoint{0.688192in}{0.670138in}}{\pgfqpoint{7.111808in}{5.129862in}}%
\pgfusepath{clip}%
\pgfsetbuttcap%
\pgfsetroundjoin%
\pgfsetlinewidth{1.003750pt}%
\definecolor{currentstroke}{rgb}{0.000000,0.000000,0.000000}%
\pgfsetstrokecolor{currentstroke}%
\pgfsetdash{}{0pt}%
\pgfpathmoveto{\pgfqpoint{1.101880in}{0.720677in}}%
\pgfpathcurveto{\pgfqpoint{1.112775in}{0.720677in}}{\pgfqpoint{1.123226in}{0.725006in}}{\pgfqpoint{1.130930in}{0.732710in}}%
\pgfpathcurveto{\pgfqpoint{1.138635in}{0.740415in}}{\pgfqpoint{1.142964in}{0.750865in}}{\pgfqpoint{1.142964in}{0.761761in}}%
\pgfpathcurveto{\pgfqpoint{1.142964in}{0.772656in}}{\pgfqpoint{1.138635in}{0.783107in}}{\pgfqpoint{1.130930in}{0.790812in}}%
\pgfpathcurveto{\pgfqpoint{1.123226in}{0.798516in}}{\pgfqpoint{1.112775in}{0.802845in}}{\pgfqpoint{1.101880in}{0.802845in}}%
\pgfpathcurveto{\pgfqpoint{1.090984in}{0.802845in}}{\pgfqpoint{1.080533in}{0.798516in}}{\pgfqpoint{1.072829in}{0.790812in}}%
\pgfpathcurveto{\pgfqpoint{1.065125in}{0.783107in}}{\pgfqpoint{1.060796in}{0.772656in}}{\pgfqpoint{1.060796in}{0.761761in}}%
\pgfpathcurveto{\pgfqpoint{1.060796in}{0.750865in}}{\pgfqpoint{1.065125in}{0.740415in}}{\pgfqpoint{1.072829in}{0.732710in}}%
\pgfpathcurveto{\pgfqpoint{1.080533in}{0.725006in}}{\pgfqpoint{1.090984in}{0.720677in}}{\pgfqpoint{1.101880in}{0.720677in}}%
\pgfpathlineto{\pgfqpoint{1.101880in}{0.720677in}}%
\pgfpathclose%
\pgfusepath{stroke}%
\end{pgfscope}%
\begin{pgfscope}%
\pgfpathrectangle{\pgfqpoint{0.688192in}{0.670138in}}{\pgfqpoint{7.111808in}{5.129862in}}%
\pgfusepath{clip}%
\pgfsetbuttcap%
\pgfsetroundjoin%
\pgfsetlinewidth{1.003750pt}%
\definecolor{currentstroke}{rgb}{0.000000,0.000000,0.000000}%
\pgfsetstrokecolor{currentstroke}%
\pgfsetdash{}{0pt}%
\pgfpathmoveto{\pgfqpoint{0.880316in}{0.913368in}}%
\pgfpathcurveto{\pgfqpoint{0.891212in}{0.913368in}}{\pgfqpoint{0.901663in}{0.917697in}}{\pgfqpoint{0.909367in}{0.925401in}}%
\pgfpathcurveto{\pgfqpoint{0.917071in}{0.933106in}}{\pgfqpoint{0.921400in}{0.943557in}}{\pgfqpoint{0.921400in}{0.954452in}}%
\pgfpathcurveto{\pgfqpoint{0.921400in}{0.965348in}}{\pgfqpoint{0.917071in}{0.975799in}}{\pgfqpoint{0.909367in}{0.983503in}}%
\pgfpathcurveto{\pgfqpoint{0.901663in}{0.991207in}}{\pgfqpoint{0.891212in}{0.995536in}}{\pgfqpoint{0.880316in}{0.995536in}}%
\pgfpathcurveto{\pgfqpoint{0.869421in}{0.995536in}}{\pgfqpoint{0.858970in}{0.991207in}}{\pgfqpoint{0.851266in}{0.983503in}}%
\pgfpathcurveto{\pgfqpoint{0.843561in}{0.975799in}}{\pgfqpoint{0.839232in}{0.965348in}}{\pgfqpoint{0.839232in}{0.954452in}}%
\pgfpathcurveto{\pgfqpoint{0.839232in}{0.943557in}}{\pgfqpoint{0.843561in}{0.933106in}}{\pgfqpoint{0.851266in}{0.925401in}}%
\pgfpathcurveto{\pgfqpoint{0.858970in}{0.917697in}}{\pgfqpoint{0.869421in}{0.913368in}}{\pgfqpoint{0.880316in}{0.913368in}}%
\pgfpathlineto{\pgfqpoint{0.880316in}{0.913368in}}%
\pgfpathclose%
\pgfusepath{stroke}%
\end{pgfscope}%
\begin{pgfscope}%
\pgfpathrectangle{\pgfqpoint{0.688192in}{0.670138in}}{\pgfqpoint{7.111808in}{5.129862in}}%
\pgfusepath{clip}%
\pgfsetbuttcap%
\pgfsetroundjoin%
\pgfsetlinewidth{1.003750pt}%
\definecolor{currentstroke}{rgb}{0.000000,0.000000,0.000000}%
\pgfsetstrokecolor{currentstroke}%
\pgfsetdash{}{0pt}%
\pgfpathmoveto{\pgfqpoint{3.828070in}{0.910178in}}%
\pgfpathcurveto{\pgfqpoint{3.838965in}{0.910178in}}{\pgfqpoint{3.849416in}{0.914507in}}{\pgfqpoint{3.857121in}{0.922211in}}%
\pgfpathcurveto{\pgfqpoint{3.864825in}{0.929915in}}{\pgfqpoint{3.869154in}{0.940366in}}{\pgfqpoint{3.869154in}{0.951262in}}%
\pgfpathcurveto{\pgfqpoint{3.869154in}{0.962157in}}{\pgfqpoint{3.864825in}{0.972608in}}{\pgfqpoint{3.857121in}{0.980312in}}%
\pgfpathcurveto{\pgfqpoint{3.849416in}{0.988017in}}{\pgfqpoint{3.838965in}{0.992346in}}{\pgfqpoint{3.828070in}{0.992346in}}%
\pgfpathcurveto{\pgfqpoint{3.817174in}{0.992346in}}{\pgfqpoint{3.806724in}{0.988017in}}{\pgfqpoint{3.799019in}{0.980312in}}%
\pgfpathcurveto{\pgfqpoint{3.791315in}{0.972608in}}{\pgfqpoint{3.786986in}{0.962157in}}{\pgfqpoint{3.786986in}{0.951262in}}%
\pgfpathcurveto{\pgfqpoint{3.786986in}{0.940366in}}{\pgfqpoint{3.791315in}{0.929915in}}{\pgfqpoint{3.799019in}{0.922211in}}%
\pgfpathcurveto{\pgfqpoint{3.806724in}{0.914507in}}{\pgfqpoint{3.817174in}{0.910178in}}{\pgfqpoint{3.828070in}{0.910178in}}%
\pgfpathlineto{\pgfqpoint{3.828070in}{0.910178in}}%
\pgfpathclose%
\pgfusepath{stroke}%
\end{pgfscope}%
\begin{pgfscope}%
\pgfpathrectangle{\pgfqpoint{0.688192in}{0.670138in}}{\pgfqpoint{7.111808in}{5.129862in}}%
\pgfusepath{clip}%
\pgfsetbuttcap%
\pgfsetroundjoin%
\pgfsetlinewidth{1.003750pt}%
\definecolor{currentstroke}{rgb}{0.000000,0.000000,0.000000}%
\pgfsetstrokecolor{currentstroke}%
\pgfsetdash{}{0pt}%
\pgfpathmoveto{\pgfqpoint{4.923457in}{0.834585in}}%
\pgfpathcurveto{\pgfqpoint{4.934352in}{0.834585in}}{\pgfqpoint{4.944803in}{0.838914in}}{\pgfqpoint{4.952507in}{0.846618in}}%
\pgfpathcurveto{\pgfqpoint{4.960212in}{0.854323in}}{\pgfqpoint{4.964540in}{0.864773in}}{\pgfqpoint{4.964540in}{0.875669in}}%
\pgfpathcurveto{\pgfqpoint{4.964540in}{0.886565in}}{\pgfqpoint{4.960212in}{0.897015in}}{\pgfqpoint{4.952507in}{0.904720in}}%
\pgfpathcurveto{\pgfqpoint{4.944803in}{0.912424in}}{\pgfqpoint{4.934352in}{0.916753in}}{\pgfqpoint{4.923457in}{0.916753in}}%
\pgfpathcurveto{\pgfqpoint{4.912561in}{0.916753in}}{\pgfqpoint{4.902110in}{0.912424in}}{\pgfqpoint{4.894406in}{0.904720in}}%
\pgfpathcurveto{\pgfqpoint{4.886702in}{0.897015in}}{\pgfqpoint{4.882373in}{0.886565in}}{\pgfqpoint{4.882373in}{0.875669in}}%
\pgfpathcurveto{\pgfqpoint{4.882373in}{0.864773in}}{\pgfqpoint{4.886702in}{0.854323in}}{\pgfqpoint{4.894406in}{0.846618in}}%
\pgfpathcurveto{\pgfqpoint{4.902110in}{0.838914in}}{\pgfqpoint{4.912561in}{0.834585in}}{\pgfqpoint{4.923457in}{0.834585in}}%
\pgfpathlineto{\pgfqpoint{4.923457in}{0.834585in}}%
\pgfpathclose%
\pgfusepath{stroke}%
\end{pgfscope}%
\begin{pgfscope}%
\pgfpathrectangle{\pgfqpoint{0.688192in}{0.670138in}}{\pgfqpoint{7.111808in}{5.129862in}}%
\pgfusepath{clip}%
\pgfsetbuttcap%
\pgfsetroundjoin%
\pgfsetlinewidth{1.003750pt}%
\definecolor{currentstroke}{rgb}{0.000000,0.000000,0.000000}%
\pgfsetstrokecolor{currentstroke}%
\pgfsetdash{}{0pt}%
\pgfpathmoveto{\pgfqpoint{1.091394in}{0.722252in}}%
\pgfpathcurveto{\pgfqpoint{1.102290in}{0.722252in}}{\pgfqpoint{1.112741in}{0.726581in}}{\pgfqpoint{1.120445in}{0.734285in}}%
\pgfpathcurveto{\pgfqpoint{1.128149in}{0.741989in}}{\pgfqpoint{1.132478in}{0.752440in}}{\pgfqpoint{1.132478in}{0.763336in}}%
\pgfpathcurveto{\pgfqpoint{1.132478in}{0.774231in}}{\pgfqpoint{1.128149in}{0.784682in}}{\pgfqpoint{1.120445in}{0.792386in}}%
\pgfpathcurveto{\pgfqpoint{1.112741in}{0.800091in}}{\pgfqpoint{1.102290in}{0.804420in}}{\pgfqpoint{1.091394in}{0.804420in}}%
\pgfpathcurveto{\pgfqpoint{1.080499in}{0.804420in}}{\pgfqpoint{1.070048in}{0.800091in}}{\pgfqpoint{1.062344in}{0.792386in}}%
\pgfpathcurveto{\pgfqpoint{1.054639in}{0.784682in}}{\pgfqpoint{1.050310in}{0.774231in}}{\pgfqpoint{1.050310in}{0.763336in}}%
\pgfpathcurveto{\pgfqpoint{1.050310in}{0.752440in}}{\pgfqpoint{1.054639in}{0.741989in}}{\pgfqpoint{1.062344in}{0.734285in}}%
\pgfpathcurveto{\pgfqpoint{1.070048in}{0.726581in}}{\pgfqpoint{1.080499in}{0.722252in}}{\pgfqpoint{1.091394in}{0.722252in}}%
\pgfpathlineto{\pgfqpoint{1.091394in}{0.722252in}}%
\pgfpathclose%
\pgfusepath{stroke}%
\end{pgfscope}%
\begin{pgfscope}%
\pgfpathrectangle{\pgfqpoint{0.688192in}{0.670138in}}{\pgfqpoint{7.111808in}{5.129862in}}%
\pgfusepath{clip}%
\pgfsetbuttcap%
\pgfsetroundjoin%
\pgfsetlinewidth{1.003750pt}%
\definecolor{currentstroke}{rgb}{0.000000,0.000000,0.000000}%
\pgfsetstrokecolor{currentstroke}%
\pgfsetdash{}{0pt}%
\pgfpathmoveto{\pgfqpoint{0.886861in}{0.869726in}}%
\pgfpathcurveto{\pgfqpoint{0.897756in}{0.869726in}}{\pgfqpoint{0.908207in}{0.874055in}}{\pgfqpoint{0.915911in}{0.881759in}}%
\pgfpathcurveto{\pgfqpoint{0.923616in}{0.889463in}}{\pgfqpoint{0.927945in}{0.899914in}}{\pgfqpoint{0.927945in}{0.910810in}}%
\pgfpathcurveto{\pgfqpoint{0.927945in}{0.921705in}}{\pgfqpoint{0.923616in}{0.932156in}}{\pgfqpoint{0.915911in}{0.939861in}}%
\pgfpathcurveto{\pgfqpoint{0.908207in}{0.947565in}}{\pgfqpoint{0.897756in}{0.951894in}}{\pgfqpoint{0.886861in}{0.951894in}}%
\pgfpathcurveto{\pgfqpoint{0.875965in}{0.951894in}}{\pgfqpoint{0.865514in}{0.947565in}}{\pgfqpoint{0.857810in}{0.939861in}}%
\pgfpathcurveto{\pgfqpoint{0.850106in}{0.932156in}}{\pgfqpoint{0.845777in}{0.921705in}}{\pgfqpoint{0.845777in}{0.910810in}}%
\pgfpathcurveto{\pgfqpoint{0.845777in}{0.899914in}}{\pgfqpoint{0.850106in}{0.889463in}}{\pgfqpoint{0.857810in}{0.881759in}}%
\pgfpathcurveto{\pgfqpoint{0.865514in}{0.874055in}}{\pgfqpoint{0.875965in}{0.869726in}}{\pgfqpoint{0.886861in}{0.869726in}}%
\pgfpathlineto{\pgfqpoint{0.886861in}{0.869726in}}%
\pgfpathclose%
\pgfusepath{stroke}%
\end{pgfscope}%
\begin{pgfscope}%
\pgfpathrectangle{\pgfqpoint{0.688192in}{0.670138in}}{\pgfqpoint{7.111808in}{5.129862in}}%
\pgfusepath{clip}%
\pgfsetbuttcap%
\pgfsetroundjoin%
\pgfsetlinewidth{1.003750pt}%
\definecolor{currentstroke}{rgb}{0.000000,0.000000,0.000000}%
\pgfsetstrokecolor{currentstroke}%
\pgfsetdash{}{0pt}%
\pgfpathmoveto{\pgfqpoint{1.920423in}{0.692661in}}%
\pgfpathcurveto{\pgfqpoint{1.931318in}{0.692661in}}{\pgfqpoint{1.941769in}{0.696990in}}{\pgfqpoint{1.949473in}{0.704695in}}%
\pgfpathcurveto{\pgfqpoint{1.957178in}{0.712399in}}{\pgfqpoint{1.961506in}{0.722850in}}{\pgfqpoint{1.961506in}{0.733745in}}%
\pgfpathcurveto{\pgfqpoint{1.961506in}{0.744641in}}{\pgfqpoint{1.957178in}{0.755092in}}{\pgfqpoint{1.949473in}{0.762796in}}%
\pgfpathcurveto{\pgfqpoint{1.941769in}{0.770500in}}{\pgfqpoint{1.931318in}{0.774829in}}{\pgfqpoint{1.920423in}{0.774829in}}%
\pgfpathcurveto{\pgfqpoint{1.909527in}{0.774829in}}{\pgfqpoint{1.899076in}{0.770500in}}{\pgfqpoint{1.891372in}{0.762796in}}%
\pgfpathcurveto{\pgfqpoint{1.883668in}{0.755092in}}{\pgfqpoint{1.879339in}{0.744641in}}{\pgfqpoint{1.879339in}{0.733745in}}%
\pgfpathcurveto{\pgfqpoint{1.879339in}{0.722850in}}{\pgfqpoint{1.883668in}{0.712399in}}{\pgfqpoint{1.891372in}{0.704695in}}%
\pgfpathcurveto{\pgfqpoint{1.899076in}{0.696990in}}{\pgfqpoint{1.909527in}{0.692661in}}{\pgfqpoint{1.920423in}{0.692661in}}%
\pgfpathlineto{\pgfqpoint{1.920423in}{0.692661in}}%
\pgfpathclose%
\pgfusepath{stroke}%
\end{pgfscope}%
\begin{pgfscope}%
\pgfpathrectangle{\pgfqpoint{0.688192in}{0.670138in}}{\pgfqpoint{7.111808in}{5.129862in}}%
\pgfusepath{clip}%
\pgfsetbuttcap%
\pgfsetroundjoin%
\pgfsetlinewidth{1.003750pt}%
\definecolor{currentstroke}{rgb}{0.000000,0.000000,0.000000}%
\pgfsetstrokecolor{currentstroke}%
\pgfsetdash{}{0pt}%
\pgfpathmoveto{\pgfqpoint{5.640247in}{5.623015in}}%
\pgfpathcurveto{\pgfqpoint{5.651142in}{5.623015in}}{\pgfqpoint{5.661593in}{5.627344in}}{\pgfqpoint{5.669297in}{5.635048in}}%
\pgfpathcurveto{\pgfqpoint{5.677002in}{5.642752in}}{\pgfqpoint{5.681331in}{5.653203in}}{\pgfqpoint{5.681331in}{5.664099in}}%
\pgfpathcurveto{\pgfqpoint{5.681331in}{5.674994in}}{\pgfqpoint{5.677002in}{5.685445in}}{\pgfqpoint{5.669297in}{5.693149in}}%
\pgfpathcurveto{\pgfqpoint{5.661593in}{5.700854in}}{\pgfqpoint{5.651142in}{5.705182in}}{\pgfqpoint{5.640247in}{5.705182in}}%
\pgfpathcurveto{\pgfqpoint{5.629351in}{5.705182in}}{\pgfqpoint{5.618900in}{5.700854in}}{\pgfqpoint{5.611196in}{5.693149in}}%
\pgfpathcurveto{\pgfqpoint{5.603492in}{5.685445in}}{\pgfqpoint{5.599163in}{5.674994in}}{\pgfqpoint{5.599163in}{5.664099in}}%
\pgfpathcurveto{\pgfqpoint{5.599163in}{5.653203in}}{\pgfqpoint{5.603492in}{5.642752in}}{\pgfqpoint{5.611196in}{5.635048in}}%
\pgfpathcurveto{\pgfqpoint{5.618900in}{5.627344in}}{\pgfqpoint{5.629351in}{5.623015in}}{\pgfqpoint{5.640247in}{5.623015in}}%
\pgfpathlineto{\pgfqpoint{5.640247in}{5.623015in}}%
\pgfpathclose%
\pgfusepath{stroke}%
\end{pgfscope}%
\begin{pgfscope}%
\pgfpathrectangle{\pgfqpoint{0.688192in}{0.670138in}}{\pgfqpoint{7.111808in}{5.129862in}}%
\pgfusepath{clip}%
\pgfsetbuttcap%
\pgfsetroundjoin%
\pgfsetlinewidth{1.003750pt}%
\definecolor{currentstroke}{rgb}{0.000000,0.000000,0.000000}%
\pgfsetstrokecolor{currentstroke}%
\pgfsetdash{}{0pt}%
\pgfpathmoveto{\pgfqpoint{1.610058in}{0.705530in}}%
\pgfpathcurveto{\pgfqpoint{1.620954in}{0.705530in}}{\pgfqpoint{1.631404in}{0.709859in}}{\pgfqpoint{1.639109in}{0.717563in}}%
\pgfpathcurveto{\pgfqpoint{1.646813in}{0.725267in}}{\pgfqpoint{1.651142in}{0.735718in}}{\pgfqpoint{1.651142in}{0.746614in}}%
\pgfpathcurveto{\pgfqpoint{1.651142in}{0.757509in}}{\pgfqpoint{1.646813in}{0.767960in}}{\pgfqpoint{1.639109in}{0.775664in}}%
\pgfpathcurveto{\pgfqpoint{1.631404in}{0.783369in}}{\pgfqpoint{1.620954in}{0.787697in}}{\pgfqpoint{1.610058in}{0.787697in}}%
\pgfpathcurveto{\pgfqpoint{1.599163in}{0.787697in}}{\pgfqpoint{1.588712in}{0.783369in}}{\pgfqpoint{1.581007in}{0.775664in}}%
\pgfpathcurveto{\pgfqpoint{1.573303in}{0.767960in}}{\pgfqpoint{1.568974in}{0.757509in}}{\pgfqpoint{1.568974in}{0.746614in}}%
\pgfpathcurveto{\pgfqpoint{1.568974in}{0.735718in}}{\pgfqpoint{1.573303in}{0.725267in}}{\pgfqpoint{1.581007in}{0.717563in}}%
\pgfpathcurveto{\pgfqpoint{1.588712in}{0.709859in}}{\pgfqpoint{1.599163in}{0.705530in}}{\pgfqpoint{1.610058in}{0.705530in}}%
\pgfpathlineto{\pgfqpoint{1.610058in}{0.705530in}}%
\pgfpathclose%
\pgfusepath{stroke}%
\end{pgfscope}%
\begin{pgfscope}%
\pgfpathrectangle{\pgfqpoint{0.688192in}{0.670138in}}{\pgfqpoint{7.111808in}{5.129862in}}%
\pgfusepath{clip}%
\pgfsetbuttcap%
\pgfsetroundjoin%
\pgfsetlinewidth{1.003750pt}%
\definecolor{currentstroke}{rgb}{0.000000,0.000000,0.000000}%
\pgfsetstrokecolor{currentstroke}%
\pgfsetdash{}{0pt}%
\pgfpathmoveto{\pgfqpoint{1.351667in}{0.711476in}}%
\pgfpathcurveto{\pgfqpoint{1.362562in}{0.711476in}}{\pgfqpoint{1.373013in}{0.715805in}}{\pgfqpoint{1.380718in}{0.723510in}}%
\pgfpathcurveto{\pgfqpoint{1.388422in}{0.731214in}}{\pgfqpoint{1.392751in}{0.741665in}}{\pgfqpoint{1.392751in}{0.752560in}}%
\pgfpathcurveto{\pgfqpoint{1.392751in}{0.763456in}}{\pgfqpoint{1.388422in}{0.773907in}}{\pgfqpoint{1.380718in}{0.781611in}}%
\pgfpathcurveto{\pgfqpoint{1.373013in}{0.789315in}}{\pgfqpoint{1.362562in}{0.793644in}}{\pgfqpoint{1.351667in}{0.793644in}}%
\pgfpathcurveto{\pgfqpoint{1.340771in}{0.793644in}}{\pgfqpoint{1.330321in}{0.789315in}}{\pgfqpoint{1.322616in}{0.781611in}}%
\pgfpathcurveto{\pgfqpoint{1.314912in}{0.773907in}}{\pgfqpoint{1.310583in}{0.763456in}}{\pgfqpoint{1.310583in}{0.752560in}}%
\pgfpathcurveto{\pgfqpoint{1.310583in}{0.741665in}}{\pgfqpoint{1.314912in}{0.731214in}}{\pgfqpoint{1.322616in}{0.723510in}}%
\pgfpathcurveto{\pgfqpoint{1.330321in}{0.715805in}}{\pgfqpoint{1.340771in}{0.711476in}}{\pgfqpoint{1.351667in}{0.711476in}}%
\pgfpathlineto{\pgfqpoint{1.351667in}{0.711476in}}%
\pgfpathclose%
\pgfusepath{stroke}%
\end{pgfscope}%
\begin{pgfscope}%
\pgfpathrectangle{\pgfqpoint{0.688192in}{0.670138in}}{\pgfqpoint{7.111808in}{5.129862in}}%
\pgfusepath{clip}%
\pgfsetbuttcap%
\pgfsetroundjoin%
\pgfsetlinewidth{1.003750pt}%
\definecolor{currentstroke}{rgb}{0.000000,0.000000,0.000000}%
\pgfsetstrokecolor{currentstroke}%
\pgfsetdash{}{0pt}%
\pgfpathmoveto{\pgfqpoint{0.789763in}{1.038554in}}%
\pgfpathcurveto{\pgfqpoint{0.800659in}{1.038554in}}{\pgfqpoint{0.811109in}{1.042883in}}{\pgfqpoint{0.818814in}{1.050587in}}%
\pgfpathcurveto{\pgfqpoint{0.826518in}{1.058291in}}{\pgfqpoint{0.830847in}{1.068742in}}{\pgfqpoint{0.830847in}{1.079638in}}%
\pgfpathcurveto{\pgfqpoint{0.830847in}{1.090533in}}{\pgfqpoint{0.826518in}{1.100984in}}{\pgfqpoint{0.818814in}{1.108688in}}%
\pgfpathcurveto{\pgfqpoint{0.811109in}{1.116393in}}{\pgfqpoint{0.800659in}{1.120722in}}{\pgfqpoint{0.789763in}{1.120722in}}%
\pgfpathcurveto{\pgfqpoint{0.778868in}{1.120722in}}{\pgfqpoint{0.768417in}{1.116393in}}{\pgfqpoint{0.760712in}{1.108688in}}%
\pgfpathcurveto{\pgfqpoint{0.753008in}{1.100984in}}{\pgfqpoint{0.748679in}{1.090533in}}{\pgfqpoint{0.748679in}{1.079638in}}%
\pgfpathcurveto{\pgfqpoint{0.748679in}{1.068742in}}{\pgfqpoint{0.753008in}{1.058291in}}{\pgfqpoint{0.760712in}{1.050587in}}%
\pgfpathcurveto{\pgfqpoint{0.768417in}{1.042883in}}{\pgfqpoint{0.778868in}{1.038554in}}{\pgfqpoint{0.789763in}{1.038554in}}%
\pgfpathlineto{\pgfqpoint{0.789763in}{1.038554in}}%
\pgfpathclose%
\pgfusepath{stroke}%
\end{pgfscope}%
\begin{pgfscope}%
\pgfpathrectangle{\pgfqpoint{0.688192in}{0.670138in}}{\pgfqpoint{7.111808in}{5.129862in}}%
\pgfusepath{clip}%
\pgfsetbuttcap%
\pgfsetroundjoin%
\pgfsetlinewidth{1.003750pt}%
\definecolor{currentstroke}{rgb}{0.000000,0.000000,0.000000}%
\pgfsetstrokecolor{currentstroke}%
\pgfsetdash{}{0pt}%
\pgfpathmoveto{\pgfqpoint{3.172874in}{1.181780in}}%
\pgfpathcurveto{\pgfqpoint{3.183770in}{1.181780in}}{\pgfqpoint{3.194221in}{1.186109in}}{\pgfqpoint{3.201925in}{1.193813in}}%
\pgfpathcurveto{\pgfqpoint{3.209629in}{1.201517in}}{\pgfqpoint{3.213958in}{1.211968in}}{\pgfqpoint{3.213958in}{1.222864in}}%
\pgfpathcurveto{\pgfqpoint{3.213958in}{1.233759in}}{\pgfqpoint{3.209629in}{1.244210in}}{\pgfqpoint{3.201925in}{1.251914in}}%
\pgfpathcurveto{\pgfqpoint{3.194221in}{1.259619in}}{\pgfqpoint{3.183770in}{1.263948in}}{\pgfqpoint{3.172874in}{1.263948in}}%
\pgfpathcurveto{\pgfqpoint{3.161979in}{1.263948in}}{\pgfqpoint{3.151528in}{1.259619in}}{\pgfqpoint{3.143824in}{1.251914in}}%
\pgfpathcurveto{\pgfqpoint{3.136119in}{1.244210in}}{\pgfqpoint{3.131790in}{1.233759in}}{\pgfqpoint{3.131790in}{1.222864in}}%
\pgfpathcurveto{\pgfqpoint{3.131790in}{1.211968in}}{\pgfqpoint{3.136119in}{1.201517in}}{\pgfqpoint{3.143824in}{1.193813in}}%
\pgfpathcurveto{\pgfqpoint{3.151528in}{1.186109in}}{\pgfqpoint{3.161979in}{1.181780in}}{\pgfqpoint{3.172874in}{1.181780in}}%
\pgfpathlineto{\pgfqpoint{3.172874in}{1.181780in}}%
\pgfpathclose%
\pgfusepath{stroke}%
\end{pgfscope}%
\begin{pgfscope}%
\pgfpathrectangle{\pgfqpoint{0.688192in}{0.670138in}}{\pgfqpoint{7.111808in}{5.129862in}}%
\pgfusepath{clip}%
\pgfsetbuttcap%
\pgfsetroundjoin%
\pgfsetlinewidth{1.003750pt}%
\definecolor{currentstroke}{rgb}{0.000000,0.000000,0.000000}%
\pgfsetstrokecolor{currentstroke}%
\pgfsetdash{}{0pt}%
\pgfpathmoveto{\pgfqpoint{1.606745in}{0.705897in}}%
\pgfpathcurveto{\pgfqpoint{1.617640in}{0.705897in}}{\pgfqpoint{1.628091in}{0.710226in}}{\pgfqpoint{1.635796in}{0.717930in}}%
\pgfpathcurveto{\pgfqpoint{1.643500in}{0.725635in}}{\pgfqpoint{1.647829in}{0.736085in}}{\pgfqpoint{1.647829in}{0.746981in}}%
\pgfpathcurveto{\pgfqpoint{1.647829in}{0.757877in}}{\pgfqpoint{1.643500in}{0.768327in}}{\pgfqpoint{1.635796in}{0.776032in}}%
\pgfpathcurveto{\pgfqpoint{1.628091in}{0.783736in}}{\pgfqpoint{1.617640in}{0.788065in}}{\pgfqpoint{1.606745in}{0.788065in}}%
\pgfpathcurveto{\pgfqpoint{1.595849in}{0.788065in}}{\pgfqpoint{1.585399in}{0.783736in}}{\pgfqpoint{1.577694in}{0.776032in}}%
\pgfpathcurveto{\pgfqpoint{1.569990in}{0.768327in}}{\pgfqpoint{1.565661in}{0.757877in}}{\pgfqpoint{1.565661in}{0.746981in}}%
\pgfpathcurveto{\pgfqpoint{1.565661in}{0.736085in}}{\pgfqpoint{1.569990in}{0.725635in}}{\pgfqpoint{1.577694in}{0.717930in}}%
\pgfpathcurveto{\pgfqpoint{1.585399in}{0.710226in}}{\pgfqpoint{1.595849in}{0.705897in}}{\pgfqpoint{1.606745in}{0.705897in}}%
\pgfpathlineto{\pgfqpoint{1.606745in}{0.705897in}}%
\pgfpathclose%
\pgfusepath{stroke}%
\end{pgfscope}%
\begin{pgfscope}%
\pgfpathrectangle{\pgfqpoint{0.688192in}{0.670138in}}{\pgfqpoint{7.111808in}{5.129862in}}%
\pgfusepath{clip}%
\pgfsetbuttcap%
\pgfsetroundjoin%
\pgfsetlinewidth{1.003750pt}%
\definecolor{currentstroke}{rgb}{0.000000,0.000000,0.000000}%
\pgfsetstrokecolor{currentstroke}%
\pgfsetdash{}{0pt}%
\pgfpathmoveto{\pgfqpoint{0.708182in}{2.475553in}}%
\pgfpathcurveto{\pgfqpoint{0.719078in}{2.475553in}}{\pgfqpoint{0.729528in}{2.479882in}}{\pgfqpoint{0.737233in}{2.487586in}}%
\pgfpathcurveto{\pgfqpoint{0.744937in}{2.495291in}}{\pgfqpoint{0.749266in}{2.505741in}}{\pgfqpoint{0.749266in}{2.516637in}}%
\pgfpathcurveto{\pgfqpoint{0.749266in}{2.527532in}}{\pgfqpoint{0.744937in}{2.537983in}}{\pgfqpoint{0.737233in}{2.545688in}}%
\pgfpathcurveto{\pgfqpoint{0.729528in}{2.553392in}}{\pgfqpoint{0.719078in}{2.557721in}}{\pgfqpoint{0.708182in}{2.557721in}}%
\pgfpathcurveto{\pgfqpoint{0.697286in}{2.557721in}}{\pgfqpoint{0.686836in}{2.553392in}}{\pgfqpoint{0.679131in}{2.545688in}}%
\pgfpathcurveto{\pgfqpoint{0.671427in}{2.537983in}}{\pgfqpoint{0.667098in}{2.527532in}}{\pgfqpoint{0.667098in}{2.516637in}}%
\pgfpathcurveto{\pgfqpoint{0.667098in}{2.505741in}}{\pgfqpoint{0.671427in}{2.495291in}}{\pgfqpoint{0.679131in}{2.487586in}}%
\pgfpathcurveto{\pgfqpoint{0.686836in}{2.479882in}}{\pgfqpoint{0.697286in}{2.475553in}}{\pgfqpoint{0.708182in}{2.475553in}}%
\pgfpathlineto{\pgfqpoint{0.708182in}{2.475553in}}%
\pgfpathclose%
\pgfusepath{stroke}%
\end{pgfscope}%
\begin{pgfscope}%
\pgfpathrectangle{\pgfqpoint{0.688192in}{0.670138in}}{\pgfqpoint{7.111808in}{5.129862in}}%
\pgfusepath{clip}%
\pgfsetbuttcap%
\pgfsetroundjoin%
\pgfsetlinewidth{1.003750pt}%
\definecolor{currentstroke}{rgb}{0.000000,0.000000,0.000000}%
\pgfsetstrokecolor{currentstroke}%
\pgfsetdash{}{0pt}%
\pgfpathmoveto{\pgfqpoint{4.586127in}{0.640844in}}%
\pgfpathcurveto{\pgfqpoint{4.597023in}{0.640844in}}{\pgfqpoint{4.607473in}{0.645173in}}{\pgfqpoint{4.615178in}{0.652877in}}%
\pgfpathcurveto{\pgfqpoint{4.622882in}{0.660582in}}{\pgfqpoint{4.627211in}{0.671032in}}{\pgfqpoint{4.627211in}{0.681928in}}%
\pgfpathcurveto{\pgfqpoint{4.627211in}{0.692824in}}{\pgfqpoint{4.622882in}{0.703274in}}{\pgfqpoint{4.615178in}{0.710979in}}%
\pgfpathcurveto{\pgfqpoint{4.607473in}{0.718683in}}{\pgfqpoint{4.597023in}{0.723012in}}{\pgfqpoint{4.586127in}{0.723012in}}%
\pgfpathcurveto{\pgfqpoint{4.575231in}{0.723012in}}{\pgfqpoint{4.564781in}{0.718683in}}{\pgfqpoint{4.557076in}{0.710979in}}%
\pgfpathcurveto{\pgfqpoint{4.549372in}{0.703274in}}{\pgfqpoint{4.545043in}{0.692824in}}{\pgfqpoint{4.545043in}{0.681928in}}%
\pgfpathcurveto{\pgfqpoint{4.545043in}{0.671032in}}{\pgfqpoint{4.549372in}{0.660582in}}{\pgfqpoint{4.557076in}{0.652877in}}%
\pgfpathcurveto{\pgfqpoint{4.564781in}{0.645173in}}{\pgfqpoint{4.575231in}{0.640844in}}{\pgfqpoint{4.586127in}{0.640844in}}%
\pgfusepath{stroke}%
\end{pgfscope}%
\begin{pgfscope}%
\pgfpathrectangle{\pgfqpoint{0.688192in}{0.670138in}}{\pgfqpoint{7.111808in}{5.129862in}}%
\pgfusepath{clip}%
\pgfsetbuttcap%
\pgfsetroundjoin%
\pgfsetlinewidth{1.003750pt}%
\definecolor{currentstroke}{rgb}{0.000000,0.000000,0.000000}%
\pgfsetstrokecolor{currentstroke}%
\pgfsetdash{}{0pt}%
\pgfpathmoveto{\pgfqpoint{1.307540in}{0.713537in}}%
\pgfpathcurveto{\pgfqpoint{1.318436in}{0.713537in}}{\pgfqpoint{1.328887in}{0.717866in}}{\pgfqpoint{1.336591in}{0.725570in}}%
\pgfpathcurveto{\pgfqpoint{1.344295in}{0.733275in}}{\pgfqpoint{1.348624in}{0.743725in}}{\pgfqpoint{1.348624in}{0.754621in}}%
\pgfpathcurveto{\pgfqpoint{1.348624in}{0.765517in}}{\pgfqpoint{1.344295in}{0.775967in}}{\pgfqpoint{1.336591in}{0.783672in}}%
\pgfpathcurveto{\pgfqpoint{1.328887in}{0.791376in}}{\pgfqpoint{1.318436in}{0.795705in}}{\pgfqpoint{1.307540in}{0.795705in}}%
\pgfpathcurveto{\pgfqpoint{1.296645in}{0.795705in}}{\pgfqpoint{1.286194in}{0.791376in}}{\pgfqpoint{1.278489in}{0.783672in}}%
\pgfpathcurveto{\pgfqpoint{1.270785in}{0.775967in}}{\pgfqpoint{1.266456in}{0.765517in}}{\pgfqpoint{1.266456in}{0.754621in}}%
\pgfpathcurveto{\pgfqpoint{1.266456in}{0.743725in}}{\pgfqpoint{1.270785in}{0.733275in}}{\pgfqpoint{1.278489in}{0.725570in}}%
\pgfpathcurveto{\pgfqpoint{1.286194in}{0.717866in}}{\pgfqpoint{1.296645in}{0.713537in}}{\pgfqpoint{1.307540in}{0.713537in}}%
\pgfpathlineto{\pgfqpoint{1.307540in}{0.713537in}}%
\pgfpathclose%
\pgfusepath{stroke}%
\end{pgfscope}%
\begin{pgfscope}%
\pgfpathrectangle{\pgfqpoint{0.688192in}{0.670138in}}{\pgfqpoint{7.111808in}{5.129862in}}%
\pgfusepath{clip}%
\pgfsetbuttcap%
\pgfsetroundjoin%
\pgfsetlinewidth{1.003750pt}%
\definecolor{currentstroke}{rgb}{0.000000,0.000000,0.000000}%
\pgfsetstrokecolor{currentstroke}%
\pgfsetdash{}{0pt}%
\pgfpathmoveto{\pgfqpoint{1.775222in}{0.697735in}}%
\pgfpathcurveto{\pgfqpoint{1.786118in}{0.697735in}}{\pgfqpoint{1.796569in}{0.702064in}}{\pgfqpoint{1.804273in}{0.709768in}}%
\pgfpathcurveto{\pgfqpoint{1.811977in}{0.717473in}}{\pgfqpoint{1.816306in}{0.727923in}}{\pgfqpoint{1.816306in}{0.738819in}}%
\pgfpathcurveto{\pgfqpoint{1.816306in}{0.749715in}}{\pgfqpoint{1.811977in}{0.760165in}}{\pgfqpoint{1.804273in}{0.767870in}}%
\pgfpathcurveto{\pgfqpoint{1.796569in}{0.775574in}}{\pgfqpoint{1.786118in}{0.779903in}}{\pgfqpoint{1.775222in}{0.779903in}}%
\pgfpathcurveto{\pgfqpoint{1.764327in}{0.779903in}}{\pgfqpoint{1.753876in}{0.775574in}}{\pgfqpoint{1.746172in}{0.767870in}}%
\pgfpathcurveto{\pgfqpoint{1.738467in}{0.760165in}}{\pgfqpoint{1.734139in}{0.749715in}}{\pgfqpoint{1.734139in}{0.738819in}}%
\pgfpathcurveto{\pgfqpoint{1.734139in}{0.727923in}}{\pgfqpoint{1.738467in}{0.717473in}}{\pgfqpoint{1.746172in}{0.709768in}}%
\pgfpathcurveto{\pgfqpoint{1.753876in}{0.702064in}}{\pgfqpoint{1.764327in}{0.697735in}}{\pgfqpoint{1.775222in}{0.697735in}}%
\pgfpathlineto{\pgfqpoint{1.775222in}{0.697735in}}%
\pgfpathclose%
\pgfusepath{stroke}%
\end{pgfscope}%
\begin{pgfscope}%
\pgfpathrectangle{\pgfqpoint{0.688192in}{0.670138in}}{\pgfqpoint{7.111808in}{5.129862in}}%
\pgfusepath{clip}%
\pgfsetbuttcap%
\pgfsetroundjoin%
\pgfsetlinewidth{1.003750pt}%
\definecolor{currentstroke}{rgb}{0.000000,0.000000,0.000000}%
\pgfsetstrokecolor{currentstroke}%
\pgfsetdash{}{0pt}%
\pgfpathmoveto{\pgfqpoint{1.920423in}{0.692661in}}%
\pgfpathcurveto{\pgfqpoint{1.931318in}{0.692661in}}{\pgfqpoint{1.941769in}{0.696990in}}{\pgfqpoint{1.949473in}{0.704695in}}%
\pgfpathcurveto{\pgfqpoint{1.957178in}{0.712399in}}{\pgfqpoint{1.961506in}{0.722850in}}{\pgfqpoint{1.961506in}{0.733745in}}%
\pgfpathcurveto{\pgfqpoint{1.961506in}{0.744641in}}{\pgfqpoint{1.957178in}{0.755092in}}{\pgfqpoint{1.949473in}{0.762796in}}%
\pgfpathcurveto{\pgfqpoint{1.941769in}{0.770500in}}{\pgfqpoint{1.931318in}{0.774829in}}{\pgfqpoint{1.920423in}{0.774829in}}%
\pgfpathcurveto{\pgfqpoint{1.909527in}{0.774829in}}{\pgfqpoint{1.899076in}{0.770500in}}{\pgfqpoint{1.891372in}{0.762796in}}%
\pgfpathcurveto{\pgfqpoint{1.883668in}{0.755092in}}{\pgfqpoint{1.879339in}{0.744641in}}{\pgfqpoint{1.879339in}{0.733745in}}%
\pgfpathcurveto{\pgfqpoint{1.879339in}{0.722850in}}{\pgfqpoint{1.883668in}{0.712399in}}{\pgfqpoint{1.891372in}{0.704695in}}%
\pgfpathcurveto{\pgfqpoint{1.899076in}{0.696990in}}{\pgfqpoint{1.909527in}{0.692661in}}{\pgfqpoint{1.920423in}{0.692661in}}%
\pgfpathlineto{\pgfqpoint{1.920423in}{0.692661in}}%
\pgfpathclose%
\pgfusepath{stroke}%
\end{pgfscope}%
\begin{pgfscope}%
\pgfpathrectangle{\pgfqpoint{0.688192in}{0.670138in}}{\pgfqpoint{7.111808in}{5.129862in}}%
\pgfusepath{clip}%
\pgfsetbuttcap%
\pgfsetroundjoin%
\pgfsetlinewidth{1.003750pt}%
\definecolor{currentstroke}{rgb}{0.000000,0.000000,0.000000}%
\pgfsetstrokecolor{currentstroke}%
\pgfsetdash{}{0pt}%
\pgfpathmoveto{\pgfqpoint{4.724535in}{0.637252in}}%
\pgfpathcurveto{\pgfqpoint{4.735431in}{0.637252in}}{\pgfqpoint{4.745882in}{0.641581in}}{\pgfqpoint{4.753586in}{0.649285in}}%
\pgfpathcurveto{\pgfqpoint{4.761290in}{0.656989in}}{\pgfqpoint{4.765619in}{0.667440in}}{\pgfqpoint{4.765619in}{0.678336in}}%
\pgfpathcurveto{\pgfqpoint{4.765619in}{0.689231in}}{\pgfqpoint{4.761290in}{0.699682in}}{\pgfqpoint{4.753586in}{0.707387in}}%
\pgfpathcurveto{\pgfqpoint{4.745882in}{0.715091in}}{\pgfqpoint{4.735431in}{0.719420in}}{\pgfqpoint{4.724535in}{0.719420in}}%
\pgfpathcurveto{\pgfqpoint{4.713640in}{0.719420in}}{\pgfqpoint{4.703189in}{0.715091in}}{\pgfqpoint{4.695485in}{0.707387in}}%
\pgfpathcurveto{\pgfqpoint{4.687780in}{0.699682in}}{\pgfqpoint{4.683451in}{0.689231in}}{\pgfqpoint{4.683451in}{0.678336in}}%
\pgfpathcurveto{\pgfqpoint{4.683451in}{0.667440in}}{\pgfqpoint{4.687780in}{0.656989in}}{\pgfqpoint{4.695485in}{0.649285in}}%
\pgfpathcurveto{\pgfqpoint{4.703189in}{0.641581in}}{\pgfqpoint{4.713640in}{0.637252in}}{\pgfqpoint{4.724535in}{0.637252in}}%
\pgfusepath{stroke}%
\end{pgfscope}%
\begin{pgfscope}%
\pgfpathrectangle{\pgfqpoint{0.688192in}{0.670138in}}{\pgfqpoint{7.111808in}{5.129862in}}%
\pgfusepath{clip}%
\pgfsetbuttcap%
\pgfsetroundjoin%
\pgfsetlinewidth{1.003750pt}%
\definecolor{currentstroke}{rgb}{0.000000,0.000000,0.000000}%
\pgfsetstrokecolor{currentstroke}%
\pgfsetdash{}{0pt}%
\pgfpathmoveto{\pgfqpoint{1.389197in}{0.711266in}}%
\pgfpathcurveto{\pgfqpoint{1.400093in}{0.711266in}}{\pgfqpoint{1.410544in}{0.715595in}}{\pgfqpoint{1.418248in}{0.723299in}}%
\pgfpathcurveto{\pgfqpoint{1.425952in}{0.731004in}}{\pgfqpoint{1.430281in}{0.741454in}}{\pgfqpoint{1.430281in}{0.752350in}}%
\pgfpathcurveto{\pgfqpoint{1.430281in}{0.763246in}}{\pgfqpoint{1.425952in}{0.773696in}}{\pgfqpoint{1.418248in}{0.781401in}}%
\pgfpathcurveto{\pgfqpoint{1.410544in}{0.789105in}}{\pgfqpoint{1.400093in}{0.793434in}}{\pgfqpoint{1.389197in}{0.793434in}}%
\pgfpathcurveto{\pgfqpoint{1.378302in}{0.793434in}}{\pgfqpoint{1.367851in}{0.789105in}}{\pgfqpoint{1.360147in}{0.781401in}}%
\pgfpathcurveto{\pgfqpoint{1.352442in}{0.773696in}}{\pgfqpoint{1.348113in}{0.763246in}}{\pgfqpoint{1.348113in}{0.752350in}}%
\pgfpathcurveto{\pgfqpoint{1.348113in}{0.741454in}}{\pgfqpoint{1.352442in}{0.731004in}}{\pgfqpoint{1.360147in}{0.723299in}}%
\pgfpathcurveto{\pgfqpoint{1.367851in}{0.715595in}}{\pgfqpoint{1.378302in}{0.711266in}}{\pgfqpoint{1.389197in}{0.711266in}}%
\pgfpathlineto{\pgfqpoint{1.389197in}{0.711266in}}%
\pgfpathclose%
\pgfusepath{stroke}%
\end{pgfscope}%
\begin{pgfscope}%
\pgfpathrectangle{\pgfqpoint{0.688192in}{0.670138in}}{\pgfqpoint{7.111808in}{5.129862in}}%
\pgfusepath{clip}%
\pgfsetbuttcap%
\pgfsetroundjoin%
\pgfsetlinewidth{1.003750pt}%
\definecolor{currentstroke}{rgb}{0.000000,0.000000,0.000000}%
\pgfsetstrokecolor{currentstroke}%
\pgfsetdash{}{0pt}%
\pgfpathmoveto{\pgfqpoint{1.775222in}{0.697735in}}%
\pgfpathcurveto{\pgfqpoint{1.786118in}{0.697735in}}{\pgfqpoint{1.796569in}{0.702064in}}{\pgfqpoint{1.804273in}{0.709768in}}%
\pgfpathcurveto{\pgfqpoint{1.811977in}{0.717473in}}{\pgfqpoint{1.816306in}{0.727923in}}{\pgfqpoint{1.816306in}{0.738819in}}%
\pgfpathcurveto{\pgfqpoint{1.816306in}{0.749715in}}{\pgfqpoint{1.811977in}{0.760165in}}{\pgfqpoint{1.804273in}{0.767870in}}%
\pgfpathcurveto{\pgfqpoint{1.796569in}{0.775574in}}{\pgfqpoint{1.786118in}{0.779903in}}{\pgfqpoint{1.775222in}{0.779903in}}%
\pgfpathcurveto{\pgfqpoint{1.764327in}{0.779903in}}{\pgfqpoint{1.753876in}{0.775574in}}{\pgfqpoint{1.746172in}{0.767870in}}%
\pgfpathcurveto{\pgfqpoint{1.738467in}{0.760165in}}{\pgfqpoint{1.734139in}{0.749715in}}{\pgfqpoint{1.734139in}{0.738819in}}%
\pgfpathcurveto{\pgfqpoint{1.734139in}{0.727923in}}{\pgfqpoint{1.738467in}{0.717473in}}{\pgfqpoint{1.746172in}{0.709768in}}%
\pgfpathcurveto{\pgfqpoint{1.753876in}{0.702064in}}{\pgfqpoint{1.764327in}{0.697735in}}{\pgfqpoint{1.775222in}{0.697735in}}%
\pgfpathlineto{\pgfqpoint{1.775222in}{0.697735in}}%
\pgfpathclose%
\pgfusepath{stroke}%
\end{pgfscope}%
\begin{pgfscope}%
\pgfpathrectangle{\pgfqpoint{0.688192in}{0.670138in}}{\pgfqpoint{7.111808in}{5.129862in}}%
\pgfusepath{clip}%
\pgfsetbuttcap%
\pgfsetroundjoin%
\pgfsetlinewidth{1.003750pt}%
\definecolor{currentstroke}{rgb}{0.000000,0.000000,0.000000}%
\pgfsetstrokecolor{currentstroke}%
\pgfsetdash{}{0pt}%
\pgfpathmoveto{\pgfqpoint{5.530080in}{5.911460in}}%
\pgfpathcurveto{\pgfqpoint{5.540975in}{5.911460in}}{\pgfqpoint{5.551426in}{5.915789in}}{\pgfqpoint{5.559130in}{5.923493in}}%
\pgfpathcurveto{\pgfqpoint{5.566835in}{5.931197in}}{\pgfqpoint{5.571163in}{5.941648in}}{\pgfqpoint{5.571163in}{5.952544in}}%
\pgfpathcurveto{\pgfqpoint{5.571163in}{5.963439in}}{\pgfqpoint{5.566835in}{5.973890in}}{\pgfqpoint{5.559130in}{5.981595in}}%
\pgfpathcurveto{\pgfqpoint{5.551426in}{5.989299in}}{\pgfqpoint{5.540975in}{5.993628in}}{\pgfqpoint{5.530080in}{5.993628in}}%
\pgfpathcurveto{\pgfqpoint{5.519184in}{5.993628in}}{\pgfqpoint{5.508733in}{5.989299in}}{\pgfqpoint{5.501029in}{5.981595in}}%
\pgfpathcurveto{\pgfqpoint{5.493324in}{5.973890in}}{\pgfqpoint{5.488996in}{5.963439in}}{\pgfqpoint{5.488996in}{5.952544in}}%
\pgfpathcurveto{\pgfqpoint{5.488996in}{5.941648in}}{\pgfqpoint{5.493324in}{5.931197in}}{\pgfqpoint{5.501029in}{5.923493in}}%
\pgfpathcurveto{\pgfqpoint{5.508733in}{5.915789in}}{\pgfqpoint{5.519184in}{5.911460in}}{\pgfqpoint{5.530080in}{5.911460in}}%
\pgfusepath{stroke}%
\end{pgfscope}%
\begin{pgfscope}%
\pgfpathrectangle{\pgfqpoint{0.688192in}{0.670138in}}{\pgfqpoint{7.111808in}{5.129862in}}%
\pgfusepath{clip}%
\pgfsetbuttcap%
\pgfsetroundjoin%
\pgfsetlinewidth{1.003750pt}%
\definecolor{currentstroke}{rgb}{0.000000,0.000000,0.000000}%
\pgfsetstrokecolor{currentstroke}%
\pgfsetdash{}{0pt}%
\pgfpathmoveto{\pgfqpoint{0.901363in}{0.856548in}}%
\pgfpathcurveto{\pgfqpoint{0.912259in}{0.856548in}}{\pgfqpoint{0.922710in}{0.860877in}}{\pgfqpoint{0.930414in}{0.868582in}}%
\pgfpathcurveto{\pgfqpoint{0.938118in}{0.876286in}}{\pgfqpoint{0.942447in}{0.886737in}}{\pgfqpoint{0.942447in}{0.897632in}}%
\pgfpathcurveto{\pgfqpoint{0.942447in}{0.908528in}}{\pgfqpoint{0.938118in}{0.918979in}}{\pgfqpoint{0.930414in}{0.926683in}}%
\pgfpathcurveto{\pgfqpoint{0.922710in}{0.934387in}}{\pgfqpoint{0.912259in}{0.938716in}}{\pgfqpoint{0.901363in}{0.938716in}}%
\pgfpathcurveto{\pgfqpoint{0.890468in}{0.938716in}}{\pgfqpoint{0.880017in}{0.934387in}}{\pgfqpoint{0.872313in}{0.926683in}}%
\pgfpathcurveto{\pgfqpoint{0.864608in}{0.918979in}}{\pgfqpoint{0.860280in}{0.908528in}}{\pgfqpoint{0.860280in}{0.897632in}}%
\pgfpathcurveto{\pgfqpoint{0.860280in}{0.886737in}}{\pgfqpoint{0.864608in}{0.876286in}}{\pgfqpoint{0.872313in}{0.868582in}}%
\pgfpathcurveto{\pgfqpoint{0.880017in}{0.860877in}}{\pgfqpoint{0.890468in}{0.856548in}}{\pgfqpoint{0.901363in}{0.856548in}}%
\pgfpathlineto{\pgfqpoint{0.901363in}{0.856548in}}%
\pgfpathclose%
\pgfusepath{stroke}%
\end{pgfscope}%
\begin{pgfscope}%
\pgfpathrectangle{\pgfqpoint{0.688192in}{0.670138in}}{\pgfqpoint{7.111808in}{5.129862in}}%
\pgfusepath{clip}%
\pgfsetbuttcap%
\pgfsetroundjoin%
\pgfsetlinewidth{1.003750pt}%
\definecolor{currentstroke}{rgb}{0.000000,0.000000,0.000000}%
\pgfsetstrokecolor{currentstroke}%
\pgfsetdash{}{0pt}%
\pgfpathmoveto{\pgfqpoint{3.828070in}{0.910178in}}%
\pgfpathcurveto{\pgfqpoint{3.838965in}{0.910178in}}{\pgfqpoint{3.849416in}{0.914507in}}{\pgfqpoint{3.857121in}{0.922211in}}%
\pgfpathcurveto{\pgfqpoint{3.864825in}{0.929915in}}{\pgfqpoint{3.869154in}{0.940366in}}{\pgfqpoint{3.869154in}{0.951262in}}%
\pgfpathcurveto{\pgfqpoint{3.869154in}{0.962157in}}{\pgfqpoint{3.864825in}{0.972608in}}{\pgfqpoint{3.857121in}{0.980312in}}%
\pgfpathcurveto{\pgfqpoint{3.849416in}{0.988017in}}{\pgfqpoint{3.838965in}{0.992346in}}{\pgfqpoint{3.828070in}{0.992346in}}%
\pgfpathcurveto{\pgfqpoint{3.817174in}{0.992346in}}{\pgfqpoint{3.806724in}{0.988017in}}{\pgfqpoint{3.799019in}{0.980312in}}%
\pgfpathcurveto{\pgfqpoint{3.791315in}{0.972608in}}{\pgfqpoint{3.786986in}{0.962157in}}{\pgfqpoint{3.786986in}{0.951262in}}%
\pgfpathcurveto{\pgfqpoint{3.786986in}{0.940366in}}{\pgfqpoint{3.791315in}{0.929915in}}{\pgfqpoint{3.799019in}{0.922211in}}%
\pgfpathcurveto{\pgfqpoint{3.806724in}{0.914507in}}{\pgfqpoint{3.817174in}{0.910178in}}{\pgfqpoint{3.828070in}{0.910178in}}%
\pgfpathlineto{\pgfqpoint{3.828070in}{0.910178in}}%
\pgfpathclose%
\pgfusepath{stroke}%
\end{pgfscope}%
\begin{pgfscope}%
\pgfpathrectangle{\pgfqpoint{0.688192in}{0.670138in}}{\pgfqpoint{7.111808in}{5.129862in}}%
\pgfusepath{clip}%
\pgfsetbuttcap%
\pgfsetroundjoin%
\pgfsetlinewidth{1.003750pt}%
\definecolor{currentstroke}{rgb}{0.000000,0.000000,0.000000}%
\pgfsetstrokecolor{currentstroke}%
\pgfsetdash{}{0pt}%
\pgfpathmoveto{\pgfqpoint{2.632948in}{0.671720in}}%
\pgfpathcurveto{\pgfqpoint{2.643844in}{0.671720in}}{\pgfqpoint{2.654295in}{0.676049in}}{\pgfqpoint{2.661999in}{0.683753in}}%
\pgfpathcurveto{\pgfqpoint{2.669703in}{0.691457in}}{\pgfqpoint{2.674032in}{0.701908in}}{\pgfqpoint{2.674032in}{0.712804in}}%
\pgfpathcurveto{\pgfqpoint{2.674032in}{0.723699in}}{\pgfqpoint{2.669703in}{0.734150in}}{\pgfqpoint{2.661999in}{0.741854in}}%
\pgfpathcurveto{\pgfqpoint{2.654295in}{0.749559in}}{\pgfqpoint{2.643844in}{0.753888in}}{\pgfqpoint{2.632948in}{0.753888in}}%
\pgfpathcurveto{\pgfqpoint{2.622053in}{0.753888in}}{\pgfqpoint{2.611602in}{0.749559in}}{\pgfqpoint{2.603898in}{0.741854in}}%
\pgfpathcurveto{\pgfqpoint{2.596193in}{0.734150in}}{\pgfqpoint{2.591864in}{0.723699in}}{\pgfqpoint{2.591864in}{0.712804in}}%
\pgfpathcurveto{\pgfqpoint{2.591864in}{0.701908in}}{\pgfqpoint{2.596193in}{0.691457in}}{\pgfqpoint{2.603898in}{0.683753in}}%
\pgfpathcurveto{\pgfqpoint{2.611602in}{0.676049in}}{\pgfqpoint{2.622053in}{0.671720in}}{\pgfqpoint{2.632948in}{0.671720in}}%
\pgfpathlineto{\pgfqpoint{2.632948in}{0.671720in}}%
\pgfpathclose%
\pgfusepath{stroke}%
\end{pgfscope}%
\begin{pgfscope}%
\pgfpathrectangle{\pgfqpoint{0.688192in}{0.670138in}}{\pgfqpoint{7.111808in}{5.129862in}}%
\pgfusepath{clip}%
\pgfsetbuttcap%
\pgfsetroundjoin%
\pgfsetlinewidth{1.003750pt}%
\definecolor{currentstroke}{rgb}{0.000000,0.000000,0.000000}%
\pgfsetstrokecolor{currentstroke}%
\pgfsetdash{}{0pt}%
\pgfpathmoveto{\pgfqpoint{2.869035in}{0.669385in}}%
\pgfpathcurveto{\pgfqpoint{2.879931in}{0.669385in}}{\pgfqpoint{2.890382in}{0.673714in}}{\pgfqpoint{2.898086in}{0.681419in}}%
\pgfpathcurveto{\pgfqpoint{2.905790in}{0.689123in}}{\pgfqpoint{2.910119in}{0.699574in}}{\pgfqpoint{2.910119in}{0.710469in}}%
\pgfpathcurveto{\pgfqpoint{2.910119in}{0.721365in}}{\pgfqpoint{2.905790in}{0.731816in}}{\pgfqpoint{2.898086in}{0.739520in}}%
\pgfpathcurveto{\pgfqpoint{2.890382in}{0.747224in}}{\pgfqpoint{2.879931in}{0.751553in}}{\pgfqpoint{2.869035in}{0.751553in}}%
\pgfpathcurveto{\pgfqpoint{2.858140in}{0.751553in}}{\pgfqpoint{2.847689in}{0.747224in}}{\pgfqpoint{2.839985in}{0.739520in}}%
\pgfpathcurveto{\pgfqpoint{2.832280in}{0.731816in}}{\pgfqpoint{2.827952in}{0.721365in}}{\pgfqpoint{2.827952in}{0.710469in}}%
\pgfpathcurveto{\pgfqpoint{2.827952in}{0.699574in}}{\pgfqpoint{2.832280in}{0.689123in}}{\pgfqpoint{2.839985in}{0.681419in}}%
\pgfpathcurveto{\pgfqpoint{2.847689in}{0.673714in}}{\pgfqpoint{2.858140in}{0.669385in}}{\pgfqpoint{2.869035in}{0.669385in}}%
\pgfpathlineto{\pgfqpoint{2.869035in}{0.669385in}}%
\pgfpathclose%
\pgfusepath{stroke}%
\end{pgfscope}%
\begin{pgfscope}%
\pgfpathrectangle{\pgfqpoint{0.688192in}{0.670138in}}{\pgfqpoint{7.111808in}{5.129862in}}%
\pgfusepath{clip}%
\pgfsetbuttcap%
\pgfsetroundjoin%
\pgfsetlinewidth{1.003750pt}%
\definecolor{currentstroke}{rgb}{0.000000,0.000000,0.000000}%
\pgfsetstrokecolor{currentstroke}%
\pgfsetdash{}{0pt}%
\pgfpathmoveto{\pgfqpoint{4.700040in}{0.638060in}}%
\pgfpathcurveto{\pgfqpoint{4.710936in}{0.638060in}}{\pgfqpoint{4.721387in}{0.642388in}}{\pgfqpoint{4.729091in}{0.650093in}}%
\pgfpathcurveto{\pgfqpoint{4.736795in}{0.657797in}}{\pgfqpoint{4.741124in}{0.668248in}}{\pgfqpoint{4.741124in}{0.679144in}}%
\pgfpathcurveto{\pgfqpoint{4.741124in}{0.690039in}}{\pgfqpoint{4.736795in}{0.700490in}}{\pgfqpoint{4.729091in}{0.708194in}}%
\pgfpathcurveto{\pgfqpoint{4.721387in}{0.715899in}}{\pgfqpoint{4.710936in}{0.720227in}}{\pgfqpoint{4.700040in}{0.720227in}}%
\pgfpathcurveto{\pgfqpoint{4.689145in}{0.720227in}}{\pgfqpoint{4.678694in}{0.715899in}}{\pgfqpoint{4.670990in}{0.708194in}}%
\pgfpathcurveto{\pgfqpoint{4.663285in}{0.700490in}}{\pgfqpoint{4.658956in}{0.690039in}}{\pgfqpoint{4.658956in}{0.679144in}}%
\pgfpathcurveto{\pgfqpoint{4.658956in}{0.668248in}}{\pgfqpoint{4.663285in}{0.657797in}}{\pgfqpoint{4.670990in}{0.650093in}}%
\pgfpathcurveto{\pgfqpoint{4.678694in}{0.642388in}}{\pgfqpoint{4.689145in}{0.638060in}}{\pgfqpoint{4.700040in}{0.638060in}}%
\pgfusepath{stroke}%
\end{pgfscope}%
\begin{pgfscope}%
\pgfpathrectangle{\pgfqpoint{0.688192in}{0.670138in}}{\pgfqpoint{7.111808in}{5.129862in}}%
\pgfusepath{clip}%
\pgfsetbuttcap%
\pgfsetroundjoin%
\pgfsetlinewidth{1.003750pt}%
\definecolor{currentstroke}{rgb}{0.000000,0.000000,0.000000}%
\pgfsetstrokecolor{currentstroke}%
\pgfsetdash{}{0pt}%
\pgfpathmoveto{\pgfqpoint{1.414589in}{0.710834in}}%
\pgfpathcurveto{\pgfqpoint{1.425484in}{0.710834in}}{\pgfqpoint{1.435935in}{0.715163in}}{\pgfqpoint{1.443639in}{0.722867in}}%
\pgfpathcurveto{\pgfqpoint{1.451344in}{0.730572in}}{\pgfqpoint{1.455672in}{0.741022in}}{\pgfqpoint{1.455672in}{0.751918in}}%
\pgfpathcurveto{\pgfqpoint{1.455672in}{0.762814in}}{\pgfqpoint{1.451344in}{0.773264in}}{\pgfqpoint{1.443639in}{0.780969in}}%
\pgfpathcurveto{\pgfqpoint{1.435935in}{0.788673in}}{\pgfqpoint{1.425484in}{0.793002in}}{\pgfqpoint{1.414589in}{0.793002in}}%
\pgfpathcurveto{\pgfqpoint{1.403693in}{0.793002in}}{\pgfqpoint{1.393242in}{0.788673in}}{\pgfqpoint{1.385538in}{0.780969in}}%
\pgfpathcurveto{\pgfqpoint{1.377834in}{0.773264in}}{\pgfqpoint{1.373505in}{0.762814in}}{\pgfqpoint{1.373505in}{0.751918in}}%
\pgfpathcurveto{\pgfqpoint{1.373505in}{0.741022in}}{\pgfqpoint{1.377834in}{0.730572in}}{\pgfqpoint{1.385538in}{0.722867in}}%
\pgfpathcurveto{\pgfqpoint{1.393242in}{0.715163in}}{\pgfqpoint{1.403693in}{0.710834in}}{\pgfqpoint{1.414589in}{0.710834in}}%
\pgfpathlineto{\pgfqpoint{1.414589in}{0.710834in}}%
\pgfpathclose%
\pgfusepath{stroke}%
\end{pgfscope}%
\begin{pgfscope}%
\pgfpathrectangle{\pgfqpoint{0.688192in}{0.670138in}}{\pgfqpoint{7.111808in}{5.129862in}}%
\pgfusepath{clip}%
\pgfsetbuttcap%
\pgfsetroundjoin%
\pgfsetlinewidth{1.003750pt}%
\definecolor{currentstroke}{rgb}{0.000000,0.000000,0.000000}%
\pgfsetstrokecolor{currentstroke}%
\pgfsetdash{}{0pt}%
\pgfpathmoveto{\pgfqpoint{1.056340in}{0.734890in}}%
\pgfpathcurveto{\pgfqpoint{1.067235in}{0.734890in}}{\pgfqpoint{1.077686in}{0.739219in}}{\pgfqpoint{1.085391in}{0.746923in}}%
\pgfpathcurveto{\pgfqpoint{1.093095in}{0.754628in}}{\pgfqpoint{1.097424in}{0.765079in}}{\pgfqpoint{1.097424in}{0.775974in}}%
\pgfpathcurveto{\pgfqpoint{1.097424in}{0.786870in}}{\pgfqpoint{1.093095in}{0.797321in}}{\pgfqpoint{1.085391in}{0.805025in}}%
\pgfpathcurveto{\pgfqpoint{1.077686in}{0.812729in}}{\pgfqpoint{1.067235in}{0.817058in}}{\pgfqpoint{1.056340in}{0.817058in}}%
\pgfpathcurveto{\pgfqpoint{1.045444in}{0.817058in}}{\pgfqpoint{1.034993in}{0.812729in}}{\pgfqpoint{1.027289in}{0.805025in}}%
\pgfpathcurveto{\pgfqpoint{1.019585in}{0.797321in}}{\pgfqpoint{1.015256in}{0.786870in}}{\pgfqpoint{1.015256in}{0.775974in}}%
\pgfpathcurveto{\pgfqpoint{1.015256in}{0.765079in}}{\pgfqpoint{1.019585in}{0.754628in}}{\pgfqpoint{1.027289in}{0.746923in}}%
\pgfpathcurveto{\pgfqpoint{1.034993in}{0.739219in}}{\pgfqpoint{1.045444in}{0.734890in}}{\pgfqpoint{1.056340in}{0.734890in}}%
\pgfpathlineto{\pgfqpoint{1.056340in}{0.734890in}}%
\pgfpathclose%
\pgfusepath{stroke}%
\end{pgfscope}%
\begin{pgfscope}%
\pgfpathrectangle{\pgfqpoint{0.688192in}{0.670138in}}{\pgfqpoint{7.111808in}{5.129862in}}%
\pgfusepath{clip}%
\pgfsetbuttcap%
\pgfsetroundjoin%
\pgfsetlinewidth{1.003750pt}%
\definecolor{currentstroke}{rgb}{0.000000,0.000000,0.000000}%
\pgfsetstrokecolor{currentstroke}%
\pgfsetdash{}{0pt}%
\pgfpathmoveto{\pgfqpoint{1.141797in}{0.719357in}}%
\pgfpathcurveto{\pgfqpoint{1.152693in}{0.719357in}}{\pgfqpoint{1.163143in}{0.723686in}}{\pgfqpoint{1.170848in}{0.731391in}}%
\pgfpathcurveto{\pgfqpoint{1.178552in}{0.739095in}}{\pgfqpoint{1.182881in}{0.749546in}}{\pgfqpoint{1.182881in}{0.760441in}}%
\pgfpathcurveto{\pgfqpoint{1.182881in}{0.771337in}}{\pgfqpoint{1.178552in}{0.781788in}}{\pgfqpoint{1.170848in}{0.789492in}}%
\pgfpathcurveto{\pgfqpoint{1.163143in}{0.797196in}}{\pgfqpoint{1.152693in}{0.801525in}}{\pgfqpoint{1.141797in}{0.801525in}}%
\pgfpathcurveto{\pgfqpoint{1.130902in}{0.801525in}}{\pgfqpoint{1.120451in}{0.797196in}}{\pgfqpoint{1.112746in}{0.789492in}}%
\pgfpathcurveto{\pgfqpoint{1.105042in}{0.781788in}}{\pgfqpoint{1.100713in}{0.771337in}}{\pgfqpoint{1.100713in}{0.760441in}}%
\pgfpathcurveto{\pgfqpoint{1.100713in}{0.749546in}}{\pgfqpoint{1.105042in}{0.739095in}}{\pgfqpoint{1.112746in}{0.731391in}}%
\pgfpathcurveto{\pgfqpoint{1.120451in}{0.723686in}}{\pgfqpoint{1.130902in}{0.719357in}}{\pgfqpoint{1.141797in}{0.719357in}}%
\pgfpathlineto{\pgfqpoint{1.141797in}{0.719357in}}%
\pgfpathclose%
\pgfusepath{stroke}%
\end{pgfscope}%
\begin{pgfscope}%
\pgfpathrectangle{\pgfqpoint{0.688192in}{0.670138in}}{\pgfqpoint{7.111808in}{5.129862in}}%
\pgfusepath{clip}%
\pgfsetbuttcap%
\pgfsetroundjoin%
\pgfsetlinewidth{1.003750pt}%
\definecolor{currentstroke}{rgb}{0.000000,0.000000,0.000000}%
\pgfsetstrokecolor{currentstroke}%
\pgfsetdash{}{0pt}%
\pgfpathmoveto{\pgfqpoint{5.537814in}{0.631347in}}%
\pgfpathcurveto{\pgfqpoint{5.548710in}{0.631347in}}{\pgfqpoint{5.559161in}{0.635676in}}{\pgfqpoint{5.566865in}{0.643380in}}%
\pgfpathcurveto{\pgfqpoint{5.574569in}{0.651084in}}{\pgfqpoint{5.578898in}{0.661535in}}{\pgfqpoint{5.578898in}{0.672431in}}%
\pgfpathcurveto{\pgfqpoint{5.578898in}{0.683326in}}{\pgfqpoint{5.574569in}{0.693777in}}{\pgfqpoint{5.566865in}{0.701481in}}%
\pgfpathcurveto{\pgfqpoint{5.559161in}{0.709186in}}{\pgfqpoint{5.548710in}{0.713515in}}{\pgfqpoint{5.537814in}{0.713515in}}%
\pgfpathcurveto{\pgfqpoint{5.526919in}{0.713515in}}{\pgfqpoint{5.516468in}{0.709186in}}{\pgfqpoint{5.508764in}{0.701481in}}%
\pgfpathcurveto{\pgfqpoint{5.501059in}{0.693777in}}{\pgfqpoint{5.496731in}{0.683326in}}{\pgfqpoint{5.496731in}{0.672431in}}%
\pgfpathcurveto{\pgfqpoint{5.496731in}{0.661535in}}{\pgfqpoint{5.501059in}{0.651084in}}{\pgfqpoint{5.508764in}{0.643380in}}%
\pgfpathcurveto{\pgfqpoint{5.516468in}{0.635676in}}{\pgfqpoint{5.526919in}{0.631347in}}{\pgfqpoint{5.537814in}{0.631347in}}%
\pgfusepath{stroke}%
\end{pgfscope}%
\begin{pgfscope}%
\pgfpathrectangle{\pgfqpoint{0.688192in}{0.670138in}}{\pgfqpoint{7.111808in}{5.129862in}}%
\pgfusepath{clip}%
\pgfsetbuttcap%
\pgfsetroundjoin%
\pgfsetlinewidth{1.003750pt}%
\definecolor{currentstroke}{rgb}{0.000000,0.000000,0.000000}%
\pgfsetstrokecolor{currentstroke}%
\pgfsetdash{}{0pt}%
\pgfpathmoveto{\pgfqpoint{1.775222in}{0.697735in}}%
\pgfpathcurveto{\pgfqpoint{1.786118in}{0.697735in}}{\pgfqpoint{1.796569in}{0.702064in}}{\pgfqpoint{1.804273in}{0.709768in}}%
\pgfpathcurveto{\pgfqpoint{1.811977in}{0.717473in}}{\pgfqpoint{1.816306in}{0.727923in}}{\pgfqpoint{1.816306in}{0.738819in}}%
\pgfpathcurveto{\pgfqpoint{1.816306in}{0.749715in}}{\pgfqpoint{1.811977in}{0.760165in}}{\pgfqpoint{1.804273in}{0.767870in}}%
\pgfpathcurveto{\pgfqpoint{1.796569in}{0.775574in}}{\pgfqpoint{1.786118in}{0.779903in}}{\pgfqpoint{1.775222in}{0.779903in}}%
\pgfpathcurveto{\pgfqpoint{1.764327in}{0.779903in}}{\pgfqpoint{1.753876in}{0.775574in}}{\pgfqpoint{1.746172in}{0.767870in}}%
\pgfpathcurveto{\pgfqpoint{1.738467in}{0.760165in}}{\pgfqpoint{1.734139in}{0.749715in}}{\pgfqpoint{1.734139in}{0.738819in}}%
\pgfpathcurveto{\pgfqpoint{1.734139in}{0.727923in}}{\pgfqpoint{1.738467in}{0.717473in}}{\pgfqpoint{1.746172in}{0.709768in}}%
\pgfpathcurveto{\pgfqpoint{1.753876in}{0.702064in}}{\pgfqpoint{1.764327in}{0.697735in}}{\pgfqpoint{1.775222in}{0.697735in}}%
\pgfpathlineto{\pgfqpoint{1.775222in}{0.697735in}}%
\pgfpathclose%
\pgfusepath{stroke}%
\end{pgfscope}%
\begin{pgfscope}%
\pgfpathrectangle{\pgfqpoint{0.688192in}{0.670138in}}{\pgfqpoint{7.111808in}{5.129862in}}%
\pgfusepath{clip}%
\pgfsetbuttcap%
\pgfsetroundjoin%
\pgfsetlinewidth{1.003750pt}%
\definecolor{currentstroke}{rgb}{0.000000,0.000000,0.000000}%
\pgfsetstrokecolor{currentstroke}%
\pgfsetdash{}{0pt}%
\pgfpathmoveto{\pgfqpoint{0.856742in}{0.923169in}}%
\pgfpathcurveto{\pgfqpoint{0.867638in}{0.923169in}}{\pgfqpoint{0.878089in}{0.927498in}}{\pgfqpoint{0.885793in}{0.935202in}}%
\pgfpathcurveto{\pgfqpoint{0.893497in}{0.942907in}}{\pgfqpoint{0.897826in}{0.953357in}}{\pgfqpoint{0.897826in}{0.964253in}}%
\pgfpathcurveto{\pgfqpoint{0.897826in}{0.975149in}}{\pgfqpoint{0.893497in}{0.985599in}}{\pgfqpoint{0.885793in}{0.993304in}}%
\pgfpathcurveto{\pgfqpoint{0.878089in}{1.001008in}}{\pgfqpoint{0.867638in}{1.005337in}}{\pgfqpoint{0.856742in}{1.005337in}}%
\pgfpathcurveto{\pgfqpoint{0.845847in}{1.005337in}}{\pgfqpoint{0.835396in}{1.001008in}}{\pgfqpoint{0.827691in}{0.993304in}}%
\pgfpathcurveto{\pgfqpoint{0.819987in}{0.985599in}}{\pgfqpoint{0.815658in}{0.975149in}}{\pgfqpoint{0.815658in}{0.964253in}}%
\pgfpathcurveto{\pgfqpoint{0.815658in}{0.953357in}}{\pgfqpoint{0.819987in}{0.942907in}}{\pgfqpoint{0.827691in}{0.935202in}}%
\pgfpathcurveto{\pgfqpoint{0.835396in}{0.927498in}}{\pgfqpoint{0.845847in}{0.923169in}}{\pgfqpoint{0.856742in}{0.923169in}}%
\pgfpathlineto{\pgfqpoint{0.856742in}{0.923169in}}%
\pgfpathclose%
\pgfusepath{stroke}%
\end{pgfscope}%
\begin{pgfscope}%
\pgfpathrectangle{\pgfqpoint{0.688192in}{0.670138in}}{\pgfqpoint{7.111808in}{5.129862in}}%
\pgfusepath{clip}%
\pgfsetbuttcap%
\pgfsetroundjoin%
\pgfsetlinewidth{1.003750pt}%
\definecolor{currentstroke}{rgb}{0.000000,0.000000,0.000000}%
\pgfsetstrokecolor{currentstroke}%
\pgfsetdash{}{0pt}%
\pgfpathmoveto{\pgfqpoint{1.840914in}{0.696069in}}%
\pgfpathcurveto{\pgfqpoint{1.851809in}{0.696069in}}{\pgfqpoint{1.862260in}{0.700398in}}{\pgfqpoint{1.869964in}{0.708103in}}%
\pgfpathcurveto{\pgfqpoint{1.877669in}{0.715807in}}{\pgfqpoint{1.881997in}{0.726258in}}{\pgfqpoint{1.881997in}{0.737153in}}%
\pgfpathcurveto{\pgfqpoint{1.881997in}{0.748049in}}{\pgfqpoint{1.877669in}{0.758500in}}{\pgfqpoint{1.869964in}{0.766204in}}%
\pgfpathcurveto{\pgfqpoint{1.862260in}{0.773908in}}{\pgfqpoint{1.851809in}{0.778237in}}{\pgfqpoint{1.840914in}{0.778237in}}%
\pgfpathcurveto{\pgfqpoint{1.830018in}{0.778237in}}{\pgfqpoint{1.819567in}{0.773908in}}{\pgfqpoint{1.811863in}{0.766204in}}%
\pgfpathcurveto{\pgfqpoint{1.804159in}{0.758500in}}{\pgfqpoint{1.799830in}{0.748049in}}{\pgfqpoint{1.799830in}{0.737153in}}%
\pgfpathcurveto{\pgfqpoint{1.799830in}{0.726258in}}{\pgfqpoint{1.804159in}{0.715807in}}{\pgfqpoint{1.811863in}{0.708103in}}%
\pgfpathcurveto{\pgfqpoint{1.819567in}{0.700398in}}{\pgfqpoint{1.830018in}{0.696069in}}{\pgfqpoint{1.840914in}{0.696069in}}%
\pgfpathlineto{\pgfqpoint{1.840914in}{0.696069in}}%
\pgfpathclose%
\pgfusepath{stroke}%
\end{pgfscope}%
\begin{pgfscope}%
\pgfpathrectangle{\pgfqpoint{0.688192in}{0.670138in}}{\pgfqpoint{7.111808in}{5.129862in}}%
\pgfusepath{clip}%
\pgfsetbuttcap%
\pgfsetroundjoin%
\pgfsetlinewidth{1.003750pt}%
\definecolor{currentstroke}{rgb}{0.000000,0.000000,0.000000}%
\pgfsetstrokecolor{currentstroke}%
\pgfsetdash{}{0pt}%
\pgfpathmoveto{\pgfqpoint{0.691952in}{4.075841in}}%
\pgfpathcurveto{\pgfqpoint{0.702848in}{4.075841in}}{\pgfqpoint{0.713299in}{4.080170in}}{\pgfqpoint{0.721003in}{4.087874in}}%
\pgfpathcurveto{\pgfqpoint{0.728708in}{4.095579in}}{\pgfqpoint{0.733036in}{4.106029in}}{\pgfqpoint{0.733036in}{4.116925in}}%
\pgfpathcurveto{\pgfqpoint{0.733036in}{4.127821in}}{\pgfqpoint{0.728708in}{4.138271in}}{\pgfqpoint{0.721003in}{4.145976in}}%
\pgfpathcurveto{\pgfqpoint{0.713299in}{4.153680in}}{\pgfqpoint{0.702848in}{4.158009in}}{\pgfqpoint{0.691952in}{4.158009in}}%
\pgfpathcurveto{\pgfqpoint{0.681057in}{4.158009in}}{\pgfqpoint{0.670606in}{4.153680in}}{\pgfqpoint{0.662902in}{4.145976in}}%
\pgfpathcurveto{\pgfqpoint{0.655197in}{4.138271in}}{\pgfqpoint{0.650869in}{4.127821in}}{\pgfqpoint{0.650869in}{4.116925in}}%
\pgfpathcurveto{\pgfqpoint{0.650869in}{4.106029in}}{\pgfqpoint{0.655197in}{4.095579in}}{\pgfqpoint{0.662902in}{4.087874in}}%
\pgfpathcurveto{\pgfqpoint{0.670606in}{4.080170in}}{\pgfqpoint{0.681057in}{4.075841in}}{\pgfqpoint{0.691952in}{4.075841in}}%
\pgfpathlineto{\pgfqpoint{0.691952in}{4.075841in}}%
\pgfpathclose%
\pgfusepath{stroke}%
\end{pgfscope}%
\begin{pgfscope}%
\pgfpathrectangle{\pgfqpoint{0.688192in}{0.670138in}}{\pgfqpoint{7.111808in}{5.129862in}}%
\pgfusepath{clip}%
\pgfsetbuttcap%
\pgfsetroundjoin%
\pgfsetlinewidth{1.003750pt}%
\definecolor{currentstroke}{rgb}{0.000000,0.000000,0.000000}%
\pgfsetstrokecolor{currentstroke}%
\pgfsetdash{}{0pt}%
\pgfpathmoveto{\pgfqpoint{1.019140in}{0.759981in}}%
\pgfpathcurveto{\pgfqpoint{1.030035in}{0.759981in}}{\pgfqpoint{1.040486in}{0.764309in}}{\pgfqpoint{1.048191in}{0.772014in}}%
\pgfpathcurveto{\pgfqpoint{1.055895in}{0.779718in}}{\pgfqpoint{1.060224in}{0.790169in}}{\pgfqpoint{1.060224in}{0.801064in}}%
\pgfpathcurveto{\pgfqpoint{1.060224in}{0.811960in}}{\pgfqpoint{1.055895in}{0.822411in}}{\pgfqpoint{1.048191in}{0.830115in}}%
\pgfpathcurveto{\pgfqpoint{1.040486in}{0.837820in}}{\pgfqpoint{1.030035in}{0.842148in}}{\pgfqpoint{1.019140in}{0.842148in}}%
\pgfpathcurveto{\pgfqpoint{1.008244in}{0.842148in}}{\pgfqpoint{0.997794in}{0.837820in}}{\pgfqpoint{0.990089in}{0.830115in}}%
\pgfpathcurveto{\pgfqpoint{0.982385in}{0.822411in}}{\pgfqpoint{0.978056in}{0.811960in}}{\pgfqpoint{0.978056in}{0.801064in}}%
\pgfpathcurveto{\pgfqpoint{0.978056in}{0.790169in}}{\pgfqpoint{0.982385in}{0.779718in}}{\pgfqpoint{0.990089in}{0.772014in}}%
\pgfpathcurveto{\pgfqpoint{0.997794in}{0.764309in}}{\pgfqpoint{1.008244in}{0.759981in}}{\pgfqpoint{1.019140in}{0.759981in}}%
\pgfpathlineto{\pgfqpoint{1.019140in}{0.759981in}}%
\pgfpathclose%
\pgfusepath{stroke}%
\end{pgfscope}%
\begin{pgfscope}%
\pgfpathrectangle{\pgfqpoint{0.688192in}{0.670138in}}{\pgfqpoint{7.111808in}{5.129862in}}%
\pgfusepath{clip}%
\pgfsetbuttcap%
\pgfsetroundjoin%
\pgfsetlinewidth{1.003750pt}%
\definecolor{currentstroke}{rgb}{0.000000,0.000000,0.000000}%
\pgfsetstrokecolor{currentstroke}%
\pgfsetdash{}{0pt}%
\pgfpathmoveto{\pgfqpoint{0.766804in}{1.159490in}}%
\pgfpathcurveto{\pgfqpoint{0.777699in}{1.159490in}}{\pgfqpoint{0.788150in}{1.163819in}}{\pgfqpoint{0.795854in}{1.171523in}}%
\pgfpathcurveto{\pgfqpoint{0.803559in}{1.179228in}}{\pgfqpoint{0.807888in}{1.189678in}}{\pgfqpoint{0.807888in}{1.200574in}}%
\pgfpathcurveto{\pgfqpoint{0.807888in}{1.211469in}}{\pgfqpoint{0.803559in}{1.221920in}}{\pgfqpoint{0.795854in}{1.229625in}}%
\pgfpathcurveto{\pgfqpoint{0.788150in}{1.237329in}}{\pgfqpoint{0.777699in}{1.241658in}}{\pgfqpoint{0.766804in}{1.241658in}}%
\pgfpathcurveto{\pgfqpoint{0.755908in}{1.241658in}}{\pgfqpoint{0.745457in}{1.237329in}}{\pgfqpoint{0.737753in}{1.229625in}}%
\pgfpathcurveto{\pgfqpoint{0.730049in}{1.221920in}}{\pgfqpoint{0.725720in}{1.211469in}}{\pgfqpoint{0.725720in}{1.200574in}}%
\pgfpathcurveto{\pgfqpoint{0.725720in}{1.189678in}}{\pgfqpoint{0.730049in}{1.179228in}}{\pgfqpoint{0.737753in}{1.171523in}}%
\pgfpathcurveto{\pgfqpoint{0.745457in}{1.163819in}}{\pgfqpoint{0.755908in}{1.159490in}}{\pgfqpoint{0.766804in}{1.159490in}}%
\pgfpathlineto{\pgfqpoint{0.766804in}{1.159490in}}%
\pgfpathclose%
\pgfusepath{stroke}%
\end{pgfscope}%
\begin{pgfscope}%
\pgfpathrectangle{\pgfqpoint{0.688192in}{0.670138in}}{\pgfqpoint{7.111808in}{5.129862in}}%
\pgfusepath{clip}%
\pgfsetbuttcap%
\pgfsetroundjoin%
\pgfsetlinewidth{1.003750pt}%
\definecolor{currentstroke}{rgb}{0.000000,0.000000,0.000000}%
\pgfsetstrokecolor{currentstroke}%
\pgfsetdash{}{0pt}%
\pgfpathmoveto{\pgfqpoint{5.696293in}{0.630769in}}%
\pgfpathcurveto{\pgfqpoint{5.707189in}{0.630769in}}{\pgfqpoint{5.717639in}{0.635098in}}{\pgfqpoint{5.725344in}{0.642802in}}%
\pgfpathcurveto{\pgfqpoint{5.733048in}{0.650507in}}{\pgfqpoint{5.737377in}{0.660958in}}{\pgfqpoint{5.737377in}{0.671853in}}%
\pgfpathcurveto{\pgfqpoint{5.737377in}{0.682749in}}{\pgfqpoint{5.733048in}{0.693200in}}{\pgfqpoint{5.725344in}{0.700904in}}%
\pgfpathcurveto{\pgfqpoint{5.717639in}{0.708608in}}{\pgfqpoint{5.707189in}{0.712937in}}{\pgfqpoint{5.696293in}{0.712937in}}%
\pgfpathcurveto{\pgfqpoint{5.685397in}{0.712937in}}{\pgfqpoint{5.674947in}{0.708608in}}{\pgfqpoint{5.667242in}{0.700904in}}%
\pgfpathcurveto{\pgfqpoint{5.659538in}{0.693200in}}{\pgfqpoint{5.655209in}{0.682749in}}{\pgfqpoint{5.655209in}{0.671853in}}%
\pgfpathcurveto{\pgfqpoint{5.655209in}{0.660958in}}{\pgfqpoint{5.659538in}{0.650507in}}{\pgfqpoint{5.667242in}{0.642802in}}%
\pgfpathcurveto{\pgfqpoint{5.674947in}{0.635098in}}{\pgfqpoint{5.685397in}{0.630769in}}{\pgfqpoint{5.696293in}{0.630769in}}%
\pgfusepath{stroke}%
\end{pgfscope}%
\begin{pgfscope}%
\pgfpathrectangle{\pgfqpoint{0.688192in}{0.670138in}}{\pgfqpoint{7.111808in}{5.129862in}}%
\pgfusepath{clip}%
\pgfsetbuttcap%
\pgfsetroundjoin%
\pgfsetlinewidth{1.003750pt}%
\definecolor{currentstroke}{rgb}{0.000000,0.000000,0.000000}%
\pgfsetstrokecolor{currentstroke}%
\pgfsetdash{}{0pt}%
\pgfpathmoveto{\pgfqpoint{1.012547in}{0.788863in}}%
\pgfpathcurveto{\pgfqpoint{1.023443in}{0.788863in}}{\pgfqpoint{1.033894in}{0.793192in}}{\pgfqpoint{1.041598in}{0.800896in}}%
\pgfpathcurveto{\pgfqpoint{1.049302in}{0.808601in}}{\pgfqpoint{1.053631in}{0.819051in}}{\pgfqpoint{1.053631in}{0.829947in}}%
\pgfpathcurveto{\pgfqpoint{1.053631in}{0.840843in}}{\pgfqpoint{1.049302in}{0.851293in}}{\pgfqpoint{1.041598in}{0.858998in}}%
\pgfpathcurveto{\pgfqpoint{1.033894in}{0.866702in}}{\pgfqpoint{1.023443in}{0.871031in}}{\pgfqpoint{1.012547in}{0.871031in}}%
\pgfpathcurveto{\pgfqpoint{1.001652in}{0.871031in}}{\pgfqpoint{0.991201in}{0.866702in}}{\pgfqpoint{0.983497in}{0.858998in}}%
\pgfpathcurveto{\pgfqpoint{0.975792in}{0.851293in}}{\pgfqpoint{0.971463in}{0.840843in}}{\pgfqpoint{0.971463in}{0.829947in}}%
\pgfpathcurveto{\pgfqpoint{0.971463in}{0.819051in}}{\pgfqpoint{0.975792in}{0.808601in}}{\pgfqpoint{0.983497in}{0.800896in}}%
\pgfpathcurveto{\pgfqpoint{0.991201in}{0.793192in}}{\pgfqpoint{1.001652in}{0.788863in}}{\pgfqpoint{1.012547in}{0.788863in}}%
\pgfpathlineto{\pgfqpoint{1.012547in}{0.788863in}}%
\pgfpathclose%
\pgfusepath{stroke}%
\end{pgfscope}%
\begin{pgfscope}%
\pgfpathrectangle{\pgfqpoint{0.688192in}{0.670138in}}{\pgfqpoint{7.111808in}{5.129862in}}%
\pgfusepath{clip}%
\pgfsetbuttcap%
\pgfsetroundjoin%
\pgfsetlinewidth{1.003750pt}%
\definecolor{currentstroke}{rgb}{0.000000,0.000000,0.000000}%
\pgfsetstrokecolor{currentstroke}%
\pgfsetdash{}{0pt}%
\pgfpathmoveto{\pgfqpoint{4.176025in}{0.641739in}}%
\pgfpathcurveto{\pgfqpoint{4.186921in}{0.641739in}}{\pgfqpoint{4.197371in}{0.646068in}}{\pgfqpoint{4.205076in}{0.653772in}}%
\pgfpathcurveto{\pgfqpoint{4.212780in}{0.661476in}}{\pgfqpoint{4.217109in}{0.671927in}}{\pgfqpoint{4.217109in}{0.682823in}}%
\pgfpathcurveto{\pgfqpoint{4.217109in}{0.693718in}}{\pgfqpoint{4.212780in}{0.704169in}}{\pgfqpoint{4.205076in}{0.711873in}}%
\pgfpathcurveto{\pgfqpoint{4.197371in}{0.719578in}}{\pgfqpoint{4.186921in}{0.723906in}}{\pgfqpoint{4.176025in}{0.723906in}}%
\pgfpathcurveto{\pgfqpoint{4.165129in}{0.723906in}}{\pgfqpoint{4.154679in}{0.719578in}}{\pgfqpoint{4.146974in}{0.711873in}}%
\pgfpathcurveto{\pgfqpoint{4.139270in}{0.704169in}}{\pgfqpoint{4.134941in}{0.693718in}}{\pgfqpoint{4.134941in}{0.682823in}}%
\pgfpathcurveto{\pgfqpoint{4.134941in}{0.671927in}}{\pgfqpoint{4.139270in}{0.661476in}}{\pgfqpoint{4.146974in}{0.653772in}}%
\pgfpathcurveto{\pgfqpoint{4.154679in}{0.646068in}}{\pgfqpoint{4.165129in}{0.641739in}}{\pgfqpoint{4.176025in}{0.641739in}}%
\pgfusepath{stroke}%
\end{pgfscope}%
\begin{pgfscope}%
\pgfpathrectangle{\pgfqpoint{0.688192in}{0.670138in}}{\pgfqpoint{7.111808in}{5.129862in}}%
\pgfusepath{clip}%
\pgfsetbuttcap%
\pgfsetroundjoin%
\pgfsetlinewidth{1.003750pt}%
\definecolor{currentstroke}{rgb}{0.000000,0.000000,0.000000}%
\pgfsetstrokecolor{currentstroke}%
\pgfsetdash{}{0pt}%
\pgfpathmoveto{\pgfqpoint{4.909318in}{0.860401in}}%
\pgfpathcurveto{\pgfqpoint{4.920213in}{0.860401in}}{\pgfqpoint{4.930664in}{0.864730in}}{\pgfqpoint{4.938368in}{0.872434in}}%
\pgfpathcurveto{\pgfqpoint{4.946073in}{0.880138in}}{\pgfqpoint{4.950402in}{0.890589in}}{\pgfqpoint{4.950402in}{0.901485in}}%
\pgfpathcurveto{\pgfqpoint{4.950402in}{0.912380in}}{\pgfqpoint{4.946073in}{0.922831in}}{\pgfqpoint{4.938368in}{0.930535in}}%
\pgfpathcurveto{\pgfqpoint{4.930664in}{0.938240in}}{\pgfqpoint{4.920213in}{0.942569in}}{\pgfqpoint{4.909318in}{0.942569in}}%
\pgfpathcurveto{\pgfqpoint{4.898422in}{0.942569in}}{\pgfqpoint{4.887971in}{0.938240in}}{\pgfqpoint{4.880267in}{0.930535in}}%
\pgfpathcurveto{\pgfqpoint{4.872563in}{0.922831in}}{\pgfqpoint{4.868234in}{0.912380in}}{\pgfqpoint{4.868234in}{0.901485in}}%
\pgfpathcurveto{\pgfqpoint{4.868234in}{0.890589in}}{\pgfqpoint{4.872563in}{0.880138in}}{\pgfqpoint{4.880267in}{0.872434in}}%
\pgfpathcurveto{\pgfqpoint{4.887971in}{0.864730in}}{\pgfqpoint{4.898422in}{0.860401in}}{\pgfqpoint{4.909318in}{0.860401in}}%
\pgfpathlineto{\pgfqpoint{4.909318in}{0.860401in}}%
\pgfpathclose%
\pgfusepath{stroke}%
\end{pgfscope}%
\begin{pgfscope}%
\pgfpathrectangle{\pgfqpoint{0.688192in}{0.670138in}}{\pgfqpoint{7.111808in}{5.129862in}}%
\pgfusepath{clip}%
\pgfsetbuttcap%
\pgfsetroundjoin%
\pgfsetlinewidth{1.003750pt}%
\definecolor{currentstroke}{rgb}{0.000000,0.000000,0.000000}%
\pgfsetstrokecolor{currentstroke}%
\pgfsetdash{}{0pt}%
\pgfpathmoveto{\pgfqpoint{1.019140in}{0.759981in}}%
\pgfpathcurveto{\pgfqpoint{1.030035in}{0.759981in}}{\pgfqpoint{1.040486in}{0.764309in}}{\pgfqpoint{1.048191in}{0.772014in}}%
\pgfpathcurveto{\pgfqpoint{1.055895in}{0.779718in}}{\pgfqpoint{1.060224in}{0.790169in}}{\pgfqpoint{1.060224in}{0.801064in}}%
\pgfpathcurveto{\pgfqpoint{1.060224in}{0.811960in}}{\pgfqpoint{1.055895in}{0.822411in}}{\pgfqpoint{1.048191in}{0.830115in}}%
\pgfpathcurveto{\pgfqpoint{1.040486in}{0.837820in}}{\pgfqpoint{1.030035in}{0.842148in}}{\pgfqpoint{1.019140in}{0.842148in}}%
\pgfpathcurveto{\pgfqpoint{1.008244in}{0.842148in}}{\pgfqpoint{0.997794in}{0.837820in}}{\pgfqpoint{0.990089in}{0.830115in}}%
\pgfpathcurveto{\pgfqpoint{0.982385in}{0.822411in}}{\pgfqpoint{0.978056in}{0.811960in}}{\pgfqpoint{0.978056in}{0.801064in}}%
\pgfpathcurveto{\pgfqpoint{0.978056in}{0.790169in}}{\pgfqpoint{0.982385in}{0.779718in}}{\pgfqpoint{0.990089in}{0.772014in}}%
\pgfpathcurveto{\pgfqpoint{0.997794in}{0.764309in}}{\pgfqpoint{1.008244in}{0.759981in}}{\pgfqpoint{1.019140in}{0.759981in}}%
\pgfpathlineto{\pgfqpoint{1.019140in}{0.759981in}}%
\pgfpathclose%
\pgfusepath{stroke}%
\end{pgfscope}%
\begin{pgfscope}%
\pgfpathrectangle{\pgfqpoint{0.688192in}{0.670138in}}{\pgfqpoint{7.111808in}{5.129862in}}%
\pgfusepath{clip}%
\pgfsetbuttcap%
\pgfsetroundjoin%
\pgfsetlinewidth{1.003750pt}%
\definecolor{currentstroke}{rgb}{0.000000,0.000000,0.000000}%
\pgfsetstrokecolor{currentstroke}%
\pgfsetdash{}{0pt}%
\pgfpathmoveto{\pgfqpoint{1.307540in}{0.713537in}}%
\pgfpathcurveto{\pgfqpoint{1.318436in}{0.713537in}}{\pgfqpoint{1.328887in}{0.717866in}}{\pgfqpoint{1.336591in}{0.725570in}}%
\pgfpathcurveto{\pgfqpoint{1.344295in}{0.733275in}}{\pgfqpoint{1.348624in}{0.743725in}}{\pgfqpoint{1.348624in}{0.754621in}}%
\pgfpathcurveto{\pgfqpoint{1.348624in}{0.765517in}}{\pgfqpoint{1.344295in}{0.775967in}}{\pgfqpoint{1.336591in}{0.783672in}}%
\pgfpathcurveto{\pgfqpoint{1.328887in}{0.791376in}}{\pgfqpoint{1.318436in}{0.795705in}}{\pgfqpoint{1.307540in}{0.795705in}}%
\pgfpathcurveto{\pgfqpoint{1.296645in}{0.795705in}}{\pgfqpoint{1.286194in}{0.791376in}}{\pgfqpoint{1.278489in}{0.783672in}}%
\pgfpathcurveto{\pgfqpoint{1.270785in}{0.775967in}}{\pgfqpoint{1.266456in}{0.765517in}}{\pgfqpoint{1.266456in}{0.754621in}}%
\pgfpathcurveto{\pgfqpoint{1.266456in}{0.743725in}}{\pgfqpoint{1.270785in}{0.733275in}}{\pgfqpoint{1.278489in}{0.725570in}}%
\pgfpathcurveto{\pgfqpoint{1.286194in}{0.717866in}}{\pgfqpoint{1.296645in}{0.713537in}}{\pgfqpoint{1.307540in}{0.713537in}}%
\pgfpathlineto{\pgfqpoint{1.307540in}{0.713537in}}%
\pgfpathclose%
\pgfusepath{stroke}%
\end{pgfscope}%
\begin{pgfscope}%
\pgfpathrectangle{\pgfqpoint{0.688192in}{0.670138in}}{\pgfqpoint{7.111808in}{5.129862in}}%
\pgfusepath{clip}%
\pgfsetbuttcap%
\pgfsetroundjoin%
\pgfsetlinewidth{1.003750pt}%
\definecolor{currentstroke}{rgb}{0.000000,0.000000,0.000000}%
\pgfsetstrokecolor{currentstroke}%
\pgfsetdash{}{0pt}%
\pgfpathmoveto{\pgfqpoint{0.886403in}{0.869810in}}%
\pgfpathcurveto{\pgfqpoint{0.897298in}{0.869810in}}{\pgfqpoint{0.907749in}{0.874139in}}{\pgfqpoint{0.915453in}{0.881844in}}%
\pgfpathcurveto{\pgfqpoint{0.923158in}{0.889548in}}{\pgfqpoint{0.927487in}{0.899999in}}{\pgfqpoint{0.927487in}{0.910894in}}%
\pgfpathcurveto{\pgfqpoint{0.927487in}{0.921790in}}{\pgfqpoint{0.923158in}{0.932241in}}{\pgfqpoint{0.915453in}{0.939945in}}%
\pgfpathcurveto{\pgfqpoint{0.907749in}{0.947649in}}{\pgfqpoint{0.897298in}{0.951978in}}{\pgfqpoint{0.886403in}{0.951978in}}%
\pgfpathcurveto{\pgfqpoint{0.875507in}{0.951978in}}{\pgfqpoint{0.865056in}{0.947649in}}{\pgfqpoint{0.857352in}{0.939945in}}%
\pgfpathcurveto{\pgfqpoint{0.849648in}{0.932241in}}{\pgfqpoint{0.845319in}{0.921790in}}{\pgfqpoint{0.845319in}{0.910894in}}%
\pgfpathcurveto{\pgfqpoint{0.845319in}{0.899999in}}{\pgfqpoint{0.849648in}{0.889548in}}{\pgfqpoint{0.857352in}{0.881844in}}%
\pgfpathcurveto{\pgfqpoint{0.865056in}{0.874139in}}{\pgfqpoint{0.875507in}{0.869810in}}{\pgfqpoint{0.886403in}{0.869810in}}%
\pgfpathlineto{\pgfqpoint{0.886403in}{0.869810in}}%
\pgfpathclose%
\pgfusepath{stroke}%
\end{pgfscope}%
\begin{pgfscope}%
\pgfpathrectangle{\pgfqpoint{0.688192in}{0.670138in}}{\pgfqpoint{7.111808in}{5.129862in}}%
\pgfusepath{clip}%
\pgfsetbuttcap%
\pgfsetroundjoin%
\pgfsetlinewidth{1.003750pt}%
\definecolor{currentstroke}{rgb}{0.000000,0.000000,0.000000}%
\pgfsetstrokecolor{currentstroke}%
\pgfsetdash{}{0pt}%
\pgfpathmoveto{\pgfqpoint{1.351667in}{0.711476in}}%
\pgfpathcurveto{\pgfqpoint{1.362562in}{0.711476in}}{\pgfqpoint{1.373013in}{0.715805in}}{\pgfqpoint{1.380718in}{0.723510in}}%
\pgfpathcurveto{\pgfqpoint{1.388422in}{0.731214in}}{\pgfqpoint{1.392751in}{0.741665in}}{\pgfqpoint{1.392751in}{0.752560in}}%
\pgfpathcurveto{\pgfqpoint{1.392751in}{0.763456in}}{\pgfqpoint{1.388422in}{0.773907in}}{\pgfqpoint{1.380718in}{0.781611in}}%
\pgfpathcurveto{\pgfqpoint{1.373013in}{0.789315in}}{\pgfqpoint{1.362562in}{0.793644in}}{\pgfqpoint{1.351667in}{0.793644in}}%
\pgfpathcurveto{\pgfqpoint{1.340771in}{0.793644in}}{\pgfqpoint{1.330321in}{0.789315in}}{\pgfqpoint{1.322616in}{0.781611in}}%
\pgfpathcurveto{\pgfqpoint{1.314912in}{0.773907in}}{\pgfqpoint{1.310583in}{0.763456in}}{\pgfqpoint{1.310583in}{0.752560in}}%
\pgfpathcurveto{\pgfqpoint{1.310583in}{0.741665in}}{\pgfqpoint{1.314912in}{0.731214in}}{\pgfqpoint{1.322616in}{0.723510in}}%
\pgfpathcurveto{\pgfqpoint{1.330321in}{0.715805in}}{\pgfqpoint{1.340771in}{0.711476in}}{\pgfqpoint{1.351667in}{0.711476in}}%
\pgfpathlineto{\pgfqpoint{1.351667in}{0.711476in}}%
\pgfpathclose%
\pgfusepath{stroke}%
\end{pgfscope}%
\begin{pgfscope}%
\pgfpathrectangle{\pgfqpoint{0.688192in}{0.670138in}}{\pgfqpoint{7.111808in}{5.129862in}}%
\pgfusepath{clip}%
\pgfsetbuttcap%
\pgfsetroundjoin%
\pgfsetlinewidth{1.003750pt}%
\definecolor{currentstroke}{rgb}{0.000000,0.000000,0.000000}%
\pgfsetstrokecolor{currentstroke}%
\pgfsetdash{}{0pt}%
\pgfpathmoveto{\pgfqpoint{1.441321in}{0.708382in}}%
\pgfpathcurveto{\pgfqpoint{1.452217in}{0.708382in}}{\pgfqpoint{1.462667in}{0.712711in}}{\pgfqpoint{1.470372in}{0.720416in}}%
\pgfpathcurveto{\pgfqpoint{1.478076in}{0.728120in}}{\pgfqpoint{1.482405in}{0.738571in}}{\pgfqpoint{1.482405in}{0.749466in}}%
\pgfpathcurveto{\pgfqpoint{1.482405in}{0.760362in}}{\pgfqpoint{1.478076in}{0.770813in}}{\pgfqpoint{1.470372in}{0.778517in}}%
\pgfpathcurveto{\pgfqpoint{1.462667in}{0.786221in}}{\pgfqpoint{1.452217in}{0.790550in}}{\pgfqpoint{1.441321in}{0.790550in}}%
\pgfpathcurveto{\pgfqpoint{1.430425in}{0.790550in}}{\pgfqpoint{1.419975in}{0.786221in}}{\pgfqpoint{1.412270in}{0.778517in}}%
\pgfpathcurveto{\pgfqpoint{1.404566in}{0.770813in}}{\pgfqpoint{1.400237in}{0.760362in}}{\pgfqpoint{1.400237in}{0.749466in}}%
\pgfpathcurveto{\pgfqpoint{1.400237in}{0.738571in}}{\pgfqpoint{1.404566in}{0.728120in}}{\pgfqpoint{1.412270in}{0.720416in}}%
\pgfpathcurveto{\pgfqpoint{1.419975in}{0.712711in}}{\pgfqpoint{1.430425in}{0.708382in}}{\pgfqpoint{1.441321in}{0.708382in}}%
\pgfpathlineto{\pgfqpoint{1.441321in}{0.708382in}}%
\pgfpathclose%
\pgfusepath{stroke}%
\end{pgfscope}%
\begin{pgfscope}%
\pgfpathrectangle{\pgfqpoint{0.688192in}{0.670138in}}{\pgfqpoint{7.111808in}{5.129862in}}%
\pgfusepath{clip}%
\pgfsetbuttcap%
\pgfsetroundjoin%
\pgfsetlinewidth{1.003750pt}%
\definecolor{currentstroke}{rgb}{0.000000,0.000000,0.000000}%
\pgfsetstrokecolor{currentstroke}%
\pgfsetdash{}{0pt}%
\pgfpathmoveto{\pgfqpoint{1.666551in}{0.703510in}}%
\pgfpathcurveto{\pgfqpoint{1.677446in}{0.703510in}}{\pgfqpoint{1.687897in}{0.707839in}}{\pgfqpoint{1.695601in}{0.715543in}}%
\pgfpathcurveto{\pgfqpoint{1.703306in}{0.723248in}}{\pgfqpoint{1.707635in}{0.733698in}}{\pgfqpoint{1.707635in}{0.744594in}}%
\pgfpathcurveto{\pgfqpoint{1.707635in}{0.755489in}}{\pgfqpoint{1.703306in}{0.765940in}}{\pgfqpoint{1.695601in}{0.773645in}}%
\pgfpathcurveto{\pgfqpoint{1.687897in}{0.781349in}}{\pgfqpoint{1.677446in}{0.785678in}}{\pgfqpoint{1.666551in}{0.785678in}}%
\pgfpathcurveto{\pgfqpoint{1.655655in}{0.785678in}}{\pgfqpoint{1.645204in}{0.781349in}}{\pgfqpoint{1.637500in}{0.773645in}}%
\pgfpathcurveto{\pgfqpoint{1.629796in}{0.765940in}}{\pgfqpoint{1.625467in}{0.755489in}}{\pgfqpoint{1.625467in}{0.744594in}}%
\pgfpathcurveto{\pgfqpoint{1.625467in}{0.733698in}}{\pgfqpoint{1.629796in}{0.723248in}}{\pgfqpoint{1.637500in}{0.715543in}}%
\pgfpathcurveto{\pgfqpoint{1.645204in}{0.707839in}}{\pgfqpoint{1.655655in}{0.703510in}}{\pgfqpoint{1.666551in}{0.703510in}}%
\pgfpathlineto{\pgfqpoint{1.666551in}{0.703510in}}%
\pgfpathclose%
\pgfusepath{stroke}%
\end{pgfscope}%
\begin{pgfscope}%
\pgfpathrectangle{\pgfqpoint{0.688192in}{0.670138in}}{\pgfqpoint{7.111808in}{5.129862in}}%
\pgfusepath{clip}%
\pgfsetbuttcap%
\pgfsetroundjoin%
\pgfsetlinewidth{1.003750pt}%
\definecolor{currentstroke}{rgb}{0.000000,0.000000,0.000000}%
\pgfsetstrokecolor{currentstroke}%
\pgfsetdash{}{0pt}%
\pgfpathmoveto{\pgfqpoint{2.333738in}{3.417663in}}%
\pgfpathcurveto{\pgfqpoint{2.344634in}{3.417663in}}{\pgfqpoint{2.355085in}{3.421992in}}{\pgfqpoint{2.362789in}{3.429696in}}%
\pgfpathcurveto{\pgfqpoint{2.370493in}{3.437401in}}{\pgfqpoint{2.374822in}{3.447852in}}{\pgfqpoint{2.374822in}{3.458747in}}%
\pgfpathcurveto{\pgfqpoint{2.374822in}{3.469643in}}{\pgfqpoint{2.370493in}{3.480093in}}{\pgfqpoint{2.362789in}{3.487798in}}%
\pgfpathcurveto{\pgfqpoint{2.355085in}{3.495502in}}{\pgfqpoint{2.344634in}{3.499831in}}{\pgfqpoint{2.333738in}{3.499831in}}%
\pgfpathcurveto{\pgfqpoint{2.322843in}{3.499831in}}{\pgfqpoint{2.312392in}{3.495502in}}{\pgfqpoint{2.304688in}{3.487798in}}%
\pgfpathcurveto{\pgfqpoint{2.296983in}{3.480093in}}{\pgfqpoint{2.292655in}{3.469643in}}{\pgfqpoint{2.292655in}{3.458747in}}%
\pgfpathcurveto{\pgfqpoint{2.292655in}{3.447852in}}{\pgfqpoint{2.296983in}{3.437401in}}{\pgfqpoint{2.304688in}{3.429696in}}%
\pgfpathcurveto{\pgfqpoint{2.312392in}{3.421992in}}{\pgfqpoint{2.322843in}{3.417663in}}{\pgfqpoint{2.333738in}{3.417663in}}%
\pgfpathlineto{\pgfqpoint{2.333738in}{3.417663in}}%
\pgfpathclose%
\pgfusepath{stroke}%
\end{pgfscope}%
\begin{pgfscope}%
\pgfpathrectangle{\pgfqpoint{0.688192in}{0.670138in}}{\pgfqpoint{7.111808in}{5.129862in}}%
\pgfusepath{clip}%
\pgfsetbuttcap%
\pgfsetroundjoin%
\pgfsetlinewidth{1.003750pt}%
\definecolor{currentstroke}{rgb}{0.000000,0.000000,0.000000}%
\pgfsetstrokecolor{currentstroke}%
\pgfsetdash{}{0pt}%
\pgfpathmoveto{\pgfqpoint{4.549016in}{0.640923in}}%
\pgfpathcurveto{\pgfqpoint{4.559912in}{0.640923in}}{\pgfqpoint{4.570362in}{0.645252in}}{\pgfqpoint{4.578067in}{0.652956in}}%
\pgfpathcurveto{\pgfqpoint{4.585771in}{0.660660in}}{\pgfqpoint{4.590100in}{0.671111in}}{\pgfqpoint{4.590100in}{0.682007in}}%
\pgfpathcurveto{\pgfqpoint{4.590100in}{0.692902in}}{\pgfqpoint{4.585771in}{0.703353in}}{\pgfqpoint{4.578067in}{0.711057in}}%
\pgfpathcurveto{\pgfqpoint{4.570362in}{0.718762in}}{\pgfqpoint{4.559912in}{0.723090in}}{\pgfqpoint{4.549016in}{0.723090in}}%
\pgfpathcurveto{\pgfqpoint{4.538120in}{0.723090in}}{\pgfqpoint{4.527670in}{0.718762in}}{\pgfqpoint{4.519965in}{0.711057in}}%
\pgfpathcurveto{\pgfqpoint{4.512261in}{0.703353in}}{\pgfqpoint{4.507932in}{0.692902in}}{\pgfqpoint{4.507932in}{0.682007in}}%
\pgfpathcurveto{\pgfqpoint{4.507932in}{0.671111in}}{\pgfqpoint{4.512261in}{0.660660in}}{\pgfqpoint{4.519965in}{0.652956in}}%
\pgfpathcurveto{\pgfqpoint{4.527670in}{0.645252in}}{\pgfqpoint{4.538120in}{0.640923in}}{\pgfqpoint{4.549016in}{0.640923in}}%
\pgfusepath{stroke}%
\end{pgfscope}%
\begin{pgfscope}%
\pgfpathrectangle{\pgfqpoint{0.688192in}{0.670138in}}{\pgfqpoint{7.111808in}{5.129862in}}%
\pgfusepath{clip}%
\pgfsetbuttcap%
\pgfsetroundjoin%
\pgfsetlinewidth{1.003750pt}%
\definecolor{currentstroke}{rgb}{0.000000,0.000000,0.000000}%
\pgfsetstrokecolor{currentstroke}%
\pgfsetdash{}{0pt}%
\pgfpathmoveto{\pgfqpoint{1.106601in}{0.720443in}}%
\pgfpathcurveto{\pgfqpoint{1.117497in}{0.720443in}}{\pgfqpoint{1.127948in}{0.724772in}}{\pgfqpoint{1.135652in}{0.732476in}}%
\pgfpathcurveto{\pgfqpoint{1.143357in}{0.740180in}}{\pgfqpoint{1.147685in}{0.750631in}}{\pgfqpoint{1.147685in}{0.761527in}}%
\pgfpathcurveto{\pgfqpoint{1.147685in}{0.772422in}}{\pgfqpoint{1.143357in}{0.782873in}}{\pgfqpoint{1.135652in}{0.790577in}}%
\pgfpathcurveto{\pgfqpoint{1.127948in}{0.798282in}}{\pgfqpoint{1.117497in}{0.802611in}}{\pgfqpoint{1.106601in}{0.802611in}}%
\pgfpathcurveto{\pgfqpoint{1.095706in}{0.802611in}}{\pgfqpoint{1.085255in}{0.798282in}}{\pgfqpoint{1.077551in}{0.790577in}}%
\pgfpathcurveto{\pgfqpoint{1.069846in}{0.782873in}}{\pgfqpoint{1.065518in}{0.772422in}}{\pgfqpoint{1.065518in}{0.761527in}}%
\pgfpathcurveto{\pgfqpoint{1.065518in}{0.750631in}}{\pgfqpoint{1.069846in}{0.740180in}}{\pgfqpoint{1.077551in}{0.732476in}}%
\pgfpathcurveto{\pgfqpoint{1.085255in}{0.724772in}}{\pgfqpoint{1.095706in}{0.720443in}}{\pgfqpoint{1.106601in}{0.720443in}}%
\pgfpathlineto{\pgfqpoint{1.106601in}{0.720443in}}%
\pgfpathclose%
\pgfusepath{stroke}%
\end{pgfscope}%
\begin{pgfscope}%
\pgfpathrectangle{\pgfqpoint{0.688192in}{0.670138in}}{\pgfqpoint{7.111808in}{5.129862in}}%
\pgfusepath{clip}%
\pgfsetbuttcap%
\pgfsetroundjoin%
\pgfsetlinewidth{1.003750pt}%
\definecolor{currentstroke}{rgb}{0.000000,0.000000,0.000000}%
\pgfsetstrokecolor{currentstroke}%
\pgfsetdash{}{0pt}%
\pgfpathmoveto{\pgfqpoint{0.721104in}{1.921896in}}%
\pgfpathcurveto{\pgfqpoint{0.732000in}{1.921896in}}{\pgfqpoint{0.742451in}{1.926225in}}{\pgfqpoint{0.750155in}{1.933929in}}%
\pgfpathcurveto{\pgfqpoint{0.757859in}{1.941634in}}{\pgfqpoint{0.762188in}{1.952084in}}{\pgfqpoint{0.762188in}{1.962980in}}%
\pgfpathcurveto{\pgfqpoint{0.762188in}{1.973876in}}{\pgfqpoint{0.757859in}{1.984326in}}{\pgfqpoint{0.750155in}{1.992031in}}%
\pgfpathcurveto{\pgfqpoint{0.742451in}{1.999735in}}{\pgfqpoint{0.732000in}{2.004064in}}{\pgfqpoint{0.721104in}{2.004064in}}%
\pgfpathcurveto{\pgfqpoint{0.710209in}{2.004064in}}{\pgfqpoint{0.699758in}{1.999735in}}{\pgfqpoint{0.692054in}{1.992031in}}%
\pgfpathcurveto{\pgfqpoint{0.684349in}{1.984326in}}{\pgfqpoint{0.680020in}{1.973876in}}{\pgfqpoint{0.680020in}{1.962980in}}%
\pgfpathcurveto{\pgfqpoint{0.680020in}{1.952084in}}{\pgfqpoint{0.684349in}{1.941634in}}{\pgfqpoint{0.692054in}{1.933929in}}%
\pgfpathcurveto{\pgfqpoint{0.699758in}{1.926225in}}{\pgfqpoint{0.710209in}{1.921896in}}{\pgfqpoint{0.721104in}{1.921896in}}%
\pgfpathlineto{\pgfqpoint{0.721104in}{1.921896in}}%
\pgfpathclose%
\pgfusepath{stroke}%
\end{pgfscope}%
\begin{pgfscope}%
\pgfpathrectangle{\pgfqpoint{0.688192in}{0.670138in}}{\pgfqpoint{7.111808in}{5.129862in}}%
\pgfusepath{clip}%
\pgfsetbuttcap%
\pgfsetroundjoin%
\pgfsetlinewidth{1.003750pt}%
\definecolor{currentstroke}{rgb}{0.000000,0.000000,0.000000}%
\pgfsetstrokecolor{currentstroke}%
\pgfsetdash{}{0pt}%
\pgfpathmoveto{\pgfqpoint{4.946549in}{3.101874in}}%
\pgfpathcurveto{\pgfqpoint{4.957445in}{3.101874in}}{\pgfqpoint{4.967896in}{3.106203in}}{\pgfqpoint{4.975600in}{3.113907in}}%
\pgfpathcurveto{\pgfqpoint{4.983304in}{3.121612in}}{\pgfqpoint{4.987633in}{3.132062in}}{\pgfqpoint{4.987633in}{3.142958in}}%
\pgfpathcurveto{\pgfqpoint{4.987633in}{3.153854in}}{\pgfqpoint{4.983304in}{3.164304in}}{\pgfqpoint{4.975600in}{3.172009in}}%
\pgfpathcurveto{\pgfqpoint{4.967896in}{3.179713in}}{\pgfqpoint{4.957445in}{3.184042in}}{\pgfqpoint{4.946549in}{3.184042in}}%
\pgfpathcurveto{\pgfqpoint{4.935654in}{3.184042in}}{\pgfqpoint{4.925203in}{3.179713in}}{\pgfqpoint{4.917499in}{3.172009in}}%
\pgfpathcurveto{\pgfqpoint{4.909794in}{3.164304in}}{\pgfqpoint{4.905465in}{3.153854in}}{\pgfqpoint{4.905465in}{3.142958in}}%
\pgfpathcurveto{\pgfqpoint{4.905465in}{3.132062in}}{\pgfqpoint{4.909794in}{3.121612in}}{\pgfqpoint{4.917499in}{3.113907in}}%
\pgfpathcurveto{\pgfqpoint{4.925203in}{3.106203in}}{\pgfqpoint{4.935654in}{3.101874in}}{\pgfqpoint{4.946549in}{3.101874in}}%
\pgfpathlineto{\pgfqpoint{4.946549in}{3.101874in}}%
\pgfpathclose%
\pgfusepath{stroke}%
\end{pgfscope}%
\begin{pgfscope}%
\pgfpathrectangle{\pgfqpoint{0.688192in}{0.670138in}}{\pgfqpoint{7.111808in}{5.129862in}}%
\pgfusepath{clip}%
\pgfsetbuttcap%
\pgfsetroundjoin%
\pgfsetlinewidth{1.003750pt}%
\definecolor{currentstroke}{rgb}{0.000000,0.000000,0.000000}%
\pgfsetstrokecolor{currentstroke}%
\pgfsetdash{}{0pt}%
\pgfpathmoveto{\pgfqpoint{0.886861in}{0.869726in}}%
\pgfpathcurveto{\pgfqpoint{0.897756in}{0.869726in}}{\pgfqpoint{0.908207in}{0.874055in}}{\pgfqpoint{0.915911in}{0.881759in}}%
\pgfpathcurveto{\pgfqpoint{0.923616in}{0.889463in}}{\pgfqpoint{0.927945in}{0.899914in}}{\pgfqpoint{0.927945in}{0.910810in}}%
\pgfpathcurveto{\pgfqpoint{0.927945in}{0.921705in}}{\pgfqpoint{0.923616in}{0.932156in}}{\pgfqpoint{0.915911in}{0.939861in}}%
\pgfpathcurveto{\pgfqpoint{0.908207in}{0.947565in}}{\pgfqpoint{0.897756in}{0.951894in}}{\pgfqpoint{0.886861in}{0.951894in}}%
\pgfpathcurveto{\pgfqpoint{0.875965in}{0.951894in}}{\pgfqpoint{0.865514in}{0.947565in}}{\pgfqpoint{0.857810in}{0.939861in}}%
\pgfpathcurveto{\pgfqpoint{0.850106in}{0.932156in}}{\pgfqpoint{0.845777in}{0.921705in}}{\pgfqpoint{0.845777in}{0.910810in}}%
\pgfpathcurveto{\pgfqpoint{0.845777in}{0.899914in}}{\pgfqpoint{0.850106in}{0.889463in}}{\pgfqpoint{0.857810in}{0.881759in}}%
\pgfpathcurveto{\pgfqpoint{0.865514in}{0.874055in}}{\pgfqpoint{0.875965in}{0.869726in}}{\pgfqpoint{0.886861in}{0.869726in}}%
\pgfpathlineto{\pgfqpoint{0.886861in}{0.869726in}}%
\pgfpathclose%
\pgfusepath{stroke}%
\end{pgfscope}%
\begin{pgfscope}%
\pgfpathrectangle{\pgfqpoint{0.688192in}{0.670138in}}{\pgfqpoint{7.111808in}{5.129862in}}%
\pgfusepath{clip}%
\pgfsetbuttcap%
\pgfsetroundjoin%
\pgfsetlinewidth{1.003750pt}%
\definecolor{currentstroke}{rgb}{0.000000,0.000000,0.000000}%
\pgfsetstrokecolor{currentstroke}%
\pgfsetdash{}{0pt}%
\pgfpathmoveto{\pgfqpoint{1.320633in}{0.712781in}}%
\pgfpathcurveto{\pgfqpoint{1.331529in}{0.712781in}}{\pgfqpoint{1.341980in}{0.717110in}}{\pgfqpoint{1.349684in}{0.724814in}}%
\pgfpathcurveto{\pgfqpoint{1.357388in}{0.732518in}}{\pgfqpoint{1.361717in}{0.742969in}}{\pgfqpoint{1.361717in}{0.753865in}}%
\pgfpathcurveto{\pgfqpoint{1.361717in}{0.764760in}}{\pgfqpoint{1.357388in}{0.775211in}}{\pgfqpoint{1.349684in}{0.782915in}}%
\pgfpathcurveto{\pgfqpoint{1.341980in}{0.790620in}}{\pgfqpoint{1.331529in}{0.794949in}}{\pgfqpoint{1.320633in}{0.794949in}}%
\pgfpathcurveto{\pgfqpoint{1.309738in}{0.794949in}}{\pgfqpoint{1.299287in}{0.790620in}}{\pgfqpoint{1.291583in}{0.782915in}}%
\pgfpathcurveto{\pgfqpoint{1.283878in}{0.775211in}}{\pgfqpoint{1.279549in}{0.764760in}}{\pgfqpoint{1.279549in}{0.753865in}}%
\pgfpathcurveto{\pgfqpoint{1.279549in}{0.742969in}}{\pgfqpoint{1.283878in}{0.732518in}}{\pgfqpoint{1.291583in}{0.724814in}}%
\pgfpathcurveto{\pgfqpoint{1.299287in}{0.717110in}}{\pgfqpoint{1.309738in}{0.712781in}}{\pgfqpoint{1.320633in}{0.712781in}}%
\pgfpathlineto{\pgfqpoint{1.320633in}{0.712781in}}%
\pgfpathclose%
\pgfusepath{stroke}%
\end{pgfscope}%
\begin{pgfscope}%
\pgfpathrectangle{\pgfqpoint{0.688192in}{0.670138in}}{\pgfqpoint{7.111808in}{5.129862in}}%
\pgfusepath{clip}%
\pgfsetbuttcap%
\pgfsetroundjoin%
\pgfsetlinewidth{1.003750pt}%
\definecolor{currentstroke}{rgb}{0.000000,0.000000,0.000000}%
\pgfsetstrokecolor{currentstroke}%
\pgfsetdash{}{0pt}%
\pgfpathmoveto{\pgfqpoint{1.070830in}{0.725886in}}%
\pgfpathcurveto{\pgfqpoint{1.081726in}{0.725886in}}{\pgfqpoint{1.092176in}{0.730215in}}{\pgfqpoint{1.099881in}{0.737920in}}%
\pgfpathcurveto{\pgfqpoint{1.107585in}{0.745624in}}{\pgfqpoint{1.111914in}{0.756075in}}{\pgfqpoint{1.111914in}{0.766970in}}%
\pgfpathcurveto{\pgfqpoint{1.111914in}{0.777866in}}{\pgfqpoint{1.107585in}{0.788317in}}{\pgfqpoint{1.099881in}{0.796021in}}%
\pgfpathcurveto{\pgfqpoint{1.092176in}{0.803725in}}{\pgfqpoint{1.081726in}{0.808054in}}{\pgfqpoint{1.070830in}{0.808054in}}%
\pgfpathcurveto{\pgfqpoint{1.059934in}{0.808054in}}{\pgfqpoint{1.049484in}{0.803725in}}{\pgfqpoint{1.041779in}{0.796021in}}%
\pgfpathcurveto{\pgfqpoint{1.034075in}{0.788317in}}{\pgfqpoint{1.029746in}{0.777866in}}{\pgfqpoint{1.029746in}{0.766970in}}%
\pgfpathcurveto{\pgfqpoint{1.029746in}{0.756075in}}{\pgfqpoint{1.034075in}{0.745624in}}{\pgfqpoint{1.041779in}{0.737920in}}%
\pgfpathcurveto{\pgfqpoint{1.049484in}{0.730215in}}{\pgfqpoint{1.059934in}{0.725886in}}{\pgfqpoint{1.070830in}{0.725886in}}%
\pgfpathlineto{\pgfqpoint{1.070830in}{0.725886in}}%
\pgfpathclose%
\pgfusepath{stroke}%
\end{pgfscope}%
\begin{pgfscope}%
\pgfpathrectangle{\pgfqpoint{0.688192in}{0.670138in}}{\pgfqpoint{7.111808in}{5.129862in}}%
\pgfusepath{clip}%
\pgfsetbuttcap%
\pgfsetroundjoin%
\pgfsetlinewidth{1.003750pt}%
\definecolor{currentstroke}{rgb}{0.000000,0.000000,0.000000}%
\pgfsetstrokecolor{currentstroke}%
\pgfsetdash{}{0pt}%
\pgfpathmoveto{\pgfqpoint{0.789763in}{1.038554in}}%
\pgfpathcurveto{\pgfqpoint{0.800659in}{1.038554in}}{\pgfqpoint{0.811109in}{1.042883in}}{\pgfqpoint{0.818814in}{1.050587in}}%
\pgfpathcurveto{\pgfqpoint{0.826518in}{1.058291in}}{\pgfqpoint{0.830847in}{1.068742in}}{\pgfqpoint{0.830847in}{1.079638in}}%
\pgfpathcurveto{\pgfqpoint{0.830847in}{1.090533in}}{\pgfqpoint{0.826518in}{1.100984in}}{\pgfqpoint{0.818814in}{1.108688in}}%
\pgfpathcurveto{\pgfqpoint{0.811109in}{1.116393in}}{\pgfqpoint{0.800659in}{1.120722in}}{\pgfqpoint{0.789763in}{1.120722in}}%
\pgfpathcurveto{\pgfqpoint{0.778868in}{1.120722in}}{\pgfqpoint{0.768417in}{1.116393in}}{\pgfqpoint{0.760712in}{1.108688in}}%
\pgfpathcurveto{\pgfqpoint{0.753008in}{1.100984in}}{\pgfqpoint{0.748679in}{1.090533in}}{\pgfqpoint{0.748679in}{1.079638in}}%
\pgfpathcurveto{\pgfqpoint{0.748679in}{1.068742in}}{\pgfqpoint{0.753008in}{1.058291in}}{\pgfqpoint{0.760712in}{1.050587in}}%
\pgfpathcurveto{\pgfqpoint{0.768417in}{1.042883in}}{\pgfqpoint{0.778868in}{1.038554in}}{\pgfqpoint{0.789763in}{1.038554in}}%
\pgfpathlineto{\pgfqpoint{0.789763in}{1.038554in}}%
\pgfpathclose%
\pgfusepath{stroke}%
\end{pgfscope}%
\begin{pgfscope}%
\pgfpathrectangle{\pgfqpoint{0.688192in}{0.670138in}}{\pgfqpoint{7.111808in}{5.129862in}}%
\pgfusepath{clip}%
\pgfsetbuttcap%
\pgfsetroundjoin%
\pgfsetlinewidth{1.003750pt}%
\definecolor{currentstroke}{rgb}{0.000000,0.000000,0.000000}%
\pgfsetstrokecolor{currentstroke}%
\pgfsetdash{}{0pt}%
\pgfpathmoveto{\pgfqpoint{1.141797in}{0.719357in}}%
\pgfpathcurveto{\pgfqpoint{1.152693in}{0.719357in}}{\pgfqpoint{1.163143in}{0.723686in}}{\pgfqpoint{1.170848in}{0.731391in}}%
\pgfpathcurveto{\pgfqpoint{1.178552in}{0.739095in}}{\pgfqpoint{1.182881in}{0.749546in}}{\pgfqpoint{1.182881in}{0.760441in}}%
\pgfpathcurveto{\pgfqpoint{1.182881in}{0.771337in}}{\pgfqpoint{1.178552in}{0.781788in}}{\pgfqpoint{1.170848in}{0.789492in}}%
\pgfpathcurveto{\pgfqpoint{1.163143in}{0.797196in}}{\pgfqpoint{1.152693in}{0.801525in}}{\pgfqpoint{1.141797in}{0.801525in}}%
\pgfpathcurveto{\pgfqpoint{1.130902in}{0.801525in}}{\pgfqpoint{1.120451in}{0.797196in}}{\pgfqpoint{1.112746in}{0.789492in}}%
\pgfpathcurveto{\pgfqpoint{1.105042in}{0.781788in}}{\pgfqpoint{1.100713in}{0.771337in}}{\pgfqpoint{1.100713in}{0.760441in}}%
\pgfpathcurveto{\pgfqpoint{1.100713in}{0.749546in}}{\pgfqpoint{1.105042in}{0.739095in}}{\pgfqpoint{1.112746in}{0.731391in}}%
\pgfpathcurveto{\pgfqpoint{1.120451in}{0.723686in}}{\pgfqpoint{1.130902in}{0.719357in}}{\pgfqpoint{1.141797in}{0.719357in}}%
\pgfpathlineto{\pgfqpoint{1.141797in}{0.719357in}}%
\pgfpathclose%
\pgfusepath{stroke}%
\end{pgfscope}%
\begin{pgfscope}%
\pgfpathrectangle{\pgfqpoint{0.688192in}{0.670138in}}{\pgfqpoint{7.111808in}{5.129862in}}%
\pgfusepath{clip}%
\pgfsetbuttcap%
\pgfsetroundjoin%
\pgfsetlinewidth{1.003750pt}%
\definecolor{currentstroke}{rgb}{0.000000,0.000000,0.000000}%
\pgfsetstrokecolor{currentstroke}%
\pgfsetdash{}{0pt}%
\pgfpathmoveto{\pgfqpoint{4.104319in}{3.684583in}}%
\pgfpathcurveto{\pgfqpoint{4.115215in}{3.684583in}}{\pgfqpoint{4.125666in}{3.688912in}}{\pgfqpoint{4.133370in}{3.696617in}}%
\pgfpathcurveto{\pgfqpoint{4.141074in}{3.704321in}}{\pgfqpoint{4.145403in}{3.714772in}}{\pgfqpoint{4.145403in}{3.725667in}}%
\pgfpathcurveto{\pgfqpoint{4.145403in}{3.736563in}}{\pgfqpoint{4.141074in}{3.747014in}}{\pgfqpoint{4.133370in}{3.754718in}}%
\pgfpathcurveto{\pgfqpoint{4.125666in}{3.762422in}}{\pgfqpoint{4.115215in}{3.766751in}}{\pgfqpoint{4.104319in}{3.766751in}}%
\pgfpathcurveto{\pgfqpoint{4.093424in}{3.766751in}}{\pgfqpoint{4.082973in}{3.762422in}}{\pgfqpoint{4.075269in}{3.754718in}}%
\pgfpathcurveto{\pgfqpoint{4.067564in}{3.747014in}}{\pgfqpoint{4.063235in}{3.736563in}}{\pgfqpoint{4.063235in}{3.725667in}}%
\pgfpathcurveto{\pgfqpoint{4.063235in}{3.714772in}}{\pgfqpoint{4.067564in}{3.704321in}}{\pgfqpoint{4.075269in}{3.696617in}}%
\pgfpathcurveto{\pgfqpoint{4.082973in}{3.688912in}}{\pgfqpoint{4.093424in}{3.684583in}}{\pgfqpoint{4.104319in}{3.684583in}}%
\pgfpathlineto{\pgfqpoint{4.104319in}{3.684583in}}%
\pgfpathclose%
\pgfusepath{stroke}%
\end{pgfscope}%
\begin{pgfscope}%
\pgfpathrectangle{\pgfqpoint{0.688192in}{0.670138in}}{\pgfqpoint{7.111808in}{5.129862in}}%
\pgfusepath{clip}%
\pgfsetbuttcap%
\pgfsetroundjoin%
\definecolor{currentfill}{rgb}{0.172549,0.627451,0.172549}%
\pgfsetfillcolor{currentfill}%
\pgfsetlinewidth{1.003750pt}%
\definecolor{currentstroke}{rgb}{0.172549,0.627451,0.172549}%
\pgfsetstrokecolor{currentstroke}%
\pgfsetdash{}{0pt}%
\pgfsys@defobject{currentmarker}{\pgfqpoint{-0.041084in}{-0.041084in}}{\pgfqpoint{0.041084in}{0.041084in}}{%
\pgfpathmoveto{\pgfqpoint{0.000000in}{-0.041084in}}%
\pgfpathcurveto{\pgfqpoint{0.010896in}{-0.041084in}}{\pgfqpoint{0.021346in}{-0.036755in}}{\pgfqpoint{0.029051in}{-0.029051in}}%
\pgfpathcurveto{\pgfqpoint{0.036755in}{-0.021346in}}{\pgfqpoint{0.041084in}{-0.010896in}}{\pgfqpoint{0.041084in}{0.000000in}}%
\pgfpathcurveto{\pgfqpoint{0.041084in}{0.010896in}}{\pgfqpoint{0.036755in}{0.021346in}}{\pgfqpoint{0.029051in}{0.029051in}}%
\pgfpathcurveto{\pgfqpoint{0.021346in}{0.036755in}}{\pgfqpoint{0.010896in}{0.041084in}}{\pgfqpoint{0.000000in}{0.041084in}}%
\pgfpathcurveto{\pgfqpoint{-0.010896in}{0.041084in}}{\pgfqpoint{-0.021346in}{0.036755in}}{\pgfqpoint{-0.029051in}{0.029051in}}%
\pgfpathcurveto{\pgfqpoint{-0.036755in}{0.021346in}}{\pgfqpoint{-0.041084in}{0.010896in}}{\pgfqpoint{-0.041084in}{0.000000in}}%
\pgfpathcurveto{\pgfqpoint{-0.041084in}{-0.010896in}}{\pgfqpoint{-0.036755in}{-0.021346in}}{\pgfqpoint{-0.029051in}{-0.029051in}}%
\pgfpathcurveto{\pgfqpoint{-0.021346in}{-0.036755in}}{\pgfqpoint{-0.010896in}{-0.041084in}}{\pgfqpoint{0.000000in}{-0.041084in}}%
\pgfpathlineto{\pgfqpoint{0.000000in}{-0.041084in}}%
\pgfpathclose%
\pgfusepath{stroke,fill}%
}%
\begin{pgfscope}%
\pgfsys@transformshift{2.002623in}{2.069732in}%
\pgfsys@useobject{currentmarker}{}%
\end{pgfscope}%
\begin{pgfscope}%
\pgfsys@transformshift{2.577113in}{0.834635in}%
\pgfsys@useobject{currentmarker}{}%
\end{pgfscope}%
\begin{pgfscope}%
\pgfsys@transformshift{2.133693in}{0.971398in}%
\pgfsys@useobject{currentmarker}{}%
\end{pgfscope}%
\begin{pgfscope}%
\pgfsys@transformshift{2.002623in}{2.069732in}%
\pgfsys@useobject{currentmarker}{}%
\end{pgfscope}%
\begin{pgfscope}%
\pgfsys@transformshift{2.675907in}{0.794786in}%
\pgfsys@useobject{currentmarker}{}%
\end{pgfscope}%
\begin{pgfscope}%
\pgfsys@transformshift{1.604618in}{2.442321in}%
\pgfsys@useobject{currentmarker}{}%
\end{pgfscope}%
\begin{pgfscope}%
\pgfsys@transformshift{1.811007in}{1.014178in}%
\pgfsys@useobject{currentmarker}{}%
\end{pgfscope}%
\begin{pgfscope}%
\pgfsys@transformshift{2.137452in}{0.960682in}%
\pgfsys@useobject{currentmarker}{}%
\end{pgfscope}%
\begin{pgfscope}%
\pgfsys@transformshift{3.749415in}{0.772617in}%
\pgfsys@useobject{currentmarker}{}%
\end{pgfscope}%
\begin{pgfscope}%
\pgfsys@transformshift{1.740224in}{2.152752in}%
\pgfsys@useobject{currentmarker}{}%
\end{pgfscope}%
\begin{pgfscope}%
\pgfsys@transformshift{1.970906in}{1.724284in}%
\pgfsys@useobject{currentmarker}{}%
\end{pgfscope}%
\begin{pgfscope}%
\pgfsys@transformshift{1.616200in}{1.119376in}%
\pgfsys@useobject{currentmarker}{}%
\end{pgfscope}%
\begin{pgfscope}%
\pgfsys@transformshift{1.861428in}{0.911984in}%
\pgfsys@useobject{currentmarker}{}%
\end{pgfscope}%
\begin{pgfscope}%
\pgfsys@transformshift{2.683832in}{0.767507in}%
\pgfsys@useobject{currentmarker}{}%
\end{pgfscope}%
\begin{pgfscope}%
\pgfsys@transformshift{2.342420in}{0.820679in}%
\pgfsys@useobject{currentmarker}{}%
\end{pgfscope}%
\begin{pgfscope}%
\pgfsys@transformshift{2.137256in}{0.859251in}%
\pgfsys@useobject{currentmarker}{}%
\end{pgfscope}%
\begin{pgfscope}%
\pgfsys@transformshift{3.712607in}{0.763798in}%
\pgfsys@useobject{currentmarker}{}%
\end{pgfscope}%
\begin{pgfscope}%
\pgfsys@transformshift{1.935478in}{1.140459in}%
\pgfsys@useobject{currentmarker}{}%
\end{pgfscope}%
\begin{pgfscope}%
\pgfsys@transformshift{2.071820in}{0.929985in}%
\pgfsys@useobject{currentmarker}{}%
\end{pgfscope}%
\begin{pgfscope}%
\pgfsys@transformshift{2.587886in}{0.820746in}%
\pgfsys@useobject{currentmarker}{}%
\end{pgfscope}%
\begin{pgfscope}%
\pgfsys@transformshift{2.167264in}{0.917009in}%
\pgfsys@useobject{currentmarker}{}%
\end{pgfscope}%
\begin{pgfscope}%
\pgfsys@transformshift{2.192481in}{0.875862in}%
\pgfsys@useobject{currentmarker}{}%
\end{pgfscope}%
\begin{pgfscope}%
\pgfsys@transformshift{3.756237in}{0.768003in}%
\pgfsys@useobject{currentmarker}{}%
\end{pgfscope}%
\begin{pgfscope}%
\pgfsys@transformshift{1.787712in}{4.620613in}%
\pgfsys@useobject{currentmarker}{}%
\end{pgfscope}%
\begin{pgfscope}%
\pgfsys@transformshift{2.075890in}{1.177972in}%
\pgfsys@useobject{currentmarker}{}%
\end{pgfscope}%
\begin{pgfscope}%
\pgfsys@transformshift{2.205888in}{0.937526in}%
\pgfsys@useobject{currentmarker}{}%
\end{pgfscope}%
\begin{pgfscope}%
\pgfsys@transformshift{2.015412in}{0.829361in}%
\pgfsys@useobject{currentmarker}{}%
\end{pgfscope}%
\begin{pgfscope}%
\pgfsys@transformshift{1.849559in}{0.924387in}%
\pgfsys@useobject{currentmarker}{}%
\end{pgfscope}%
\begin{pgfscope}%
\pgfsys@transformshift{1.784951in}{1.056501in}%
\pgfsys@useobject{currentmarker}{}%
\end{pgfscope}%
\begin{pgfscope}%
\pgfsys@transformshift{1.667281in}{2.086122in}%
\pgfsys@useobject{currentmarker}{}%
\end{pgfscope}%
\begin{pgfscope}%
\pgfsys@transformshift{3.510642in}{0.768394in}%
\pgfsys@useobject{currentmarker}{}%
\end{pgfscope}%
\begin{pgfscope}%
\pgfsys@transformshift{2.446699in}{0.828993in}%
\pgfsys@useobject{currentmarker}{}%
\end{pgfscope}%
\begin{pgfscope}%
\pgfsys@transformshift{2.206416in}{0.850354in}%
\pgfsys@useobject{currentmarker}{}%
\end{pgfscope}%
\begin{pgfscope}%
\pgfsys@transformshift{2.273390in}{0.838871in}%
\pgfsys@useobject{currentmarker}{}%
\end{pgfscope}%
\begin{pgfscope}%
\pgfsys@transformshift{3.669906in}{0.790604in}%
\pgfsys@useobject{currentmarker}{}%
\end{pgfscope}%
\begin{pgfscope}%
\pgfsys@transformshift{2.775958in}{0.823662in}%
\pgfsys@useobject{currentmarker}{}%
\end{pgfscope}%
\begin{pgfscope}%
\pgfsys@transformshift{2.467544in}{0.829862in}%
\pgfsys@useobject{currentmarker}{}%
\end{pgfscope}%
\begin{pgfscope}%
\pgfsys@transformshift{3.766705in}{0.770312in}%
\pgfsys@useobject{currentmarker}{}%
\end{pgfscope}%
\begin{pgfscope}%
\pgfsys@transformshift{2.366651in}{0.847901in}%
\pgfsys@useobject{currentmarker}{}%
\end{pgfscope}%
\begin{pgfscope}%
\pgfsys@transformshift{1.978908in}{1.268350in}%
\pgfsys@useobject{currentmarker}{}%
\end{pgfscope}%
\begin{pgfscope}%
\pgfsys@transformshift{1.892499in}{4.580764in}%
\pgfsys@useobject{currentmarker}{}%
\end{pgfscope}%
\begin{pgfscope}%
\pgfsys@transformshift{2.995479in}{0.832502in}%
\pgfsys@useobject{currentmarker}{}%
\end{pgfscope}%
\begin{pgfscope}%
\pgfsys@transformshift{2.181522in}{0.926521in}%
\pgfsys@useobject{currentmarker}{}%
\end{pgfscope}%
\begin{pgfscope}%
\pgfsys@transformshift{2.194737in}{0.876819in}%
\pgfsys@useobject{currentmarker}{}%
\end{pgfscope}%
\begin{pgfscope}%
\pgfsys@transformshift{2.733482in}{0.845260in}%
\pgfsys@useobject{currentmarker}{}%
\end{pgfscope}%
\begin{pgfscope}%
\pgfsys@transformshift{2.249355in}{0.887340in}%
\pgfsys@useobject{currentmarker}{}%
\end{pgfscope}%
\begin{pgfscope}%
\pgfsys@transformshift{2.352434in}{0.877033in}%
\pgfsys@useobject{currentmarker}{}%
\end{pgfscope}%
\begin{pgfscope}%
\pgfsys@transformshift{2.277300in}{0.915168in}%
\pgfsys@useobject{currentmarker}{}%
\end{pgfscope}%
\begin{pgfscope}%
\pgfsys@transformshift{2.267238in}{0.918308in}%
\pgfsys@useobject{currentmarker}{}%
\end{pgfscope}%
\begin{pgfscope}%
\pgfsys@transformshift{1.664089in}{0.873927in}%
\pgfsys@useobject{currentmarker}{}%
\end{pgfscope}%
\begin{pgfscope}%
\pgfsys@transformshift{2.255328in}{0.795472in}%
\pgfsys@useobject{currentmarker}{}%
\end{pgfscope}%
\begin{pgfscope}%
\pgfsys@transformshift{1.503539in}{1.099915in}%
\pgfsys@useobject{currentmarker}{}%
\end{pgfscope}%
\begin{pgfscope}%
\pgfsys@transformshift{2.297898in}{0.787808in}%
\pgfsys@useobject{currentmarker}{}%
\end{pgfscope}%
\begin{pgfscope}%
\pgfsys@transformshift{1.837877in}{0.855892in}%
\pgfsys@useobject{currentmarker}{}%
\end{pgfscope}%
\begin{pgfscope}%
\pgfsys@transformshift{2.189198in}{0.827103in}%
\pgfsys@useobject{currentmarker}{}%
\end{pgfscope}%
\begin{pgfscope}%
\pgfsys@transformshift{1.532482in}{0.938707in}%
\pgfsys@useobject{currentmarker}{}%
\end{pgfscope}%
\begin{pgfscope}%
\pgfsys@transformshift{2.927418in}{0.758622in}%
\pgfsys@useobject{currentmarker}{}%
\end{pgfscope}%
\begin{pgfscope}%
\pgfsys@transformshift{2.099349in}{0.908918in}%
\pgfsys@useobject{currentmarker}{}%
\end{pgfscope}%
\begin{pgfscope}%
\pgfsys@transformshift{2.284703in}{0.830771in}%
\pgfsys@useobject{currentmarker}{}%
\end{pgfscope}%
\begin{pgfscope}%
\pgfsys@transformshift{2.244732in}{0.834513in}%
\pgfsys@useobject{currentmarker}{}%
\end{pgfscope}%
\begin{pgfscope}%
\pgfsys@transformshift{1.788645in}{0.939743in}%
\pgfsys@useobject{currentmarker}{}%
\end{pgfscope}%
\begin{pgfscope}%
\pgfsys@transformshift{2.922604in}{0.814960in}%
\pgfsys@useobject{currentmarker}{}%
\end{pgfscope}%
\begin{pgfscope}%
\pgfsys@transformshift{3.570530in}{0.775415in}%
\pgfsys@useobject{currentmarker}{}%
\end{pgfscope}%
\begin{pgfscope}%
\pgfsys@transformshift{2.336380in}{0.863530in}%
\pgfsys@useobject{currentmarker}{}%
\end{pgfscope}%
\begin{pgfscope}%
\pgfsys@transformshift{1.738349in}{2.346587in}%
\pgfsys@useobject{currentmarker}{}%
\end{pgfscope}%
\begin{pgfscope}%
\pgfsys@transformshift{1.695335in}{3.279285in}%
\pgfsys@useobject{currentmarker}{}%
\end{pgfscope}%
\begin{pgfscope}%
\pgfsys@transformshift{1.725514in}{2.617826in}%
\pgfsys@useobject{currentmarker}{}%
\end{pgfscope}%
\begin{pgfscope}%
\pgfsys@transformshift{1.827708in}{1.655651in}%
\pgfsys@useobject{currentmarker}{}%
\end{pgfscope}%
\begin{pgfscope}%
\pgfsys@transformshift{1.840030in}{1.269677in}%
\pgfsys@useobject{currentmarker}{}%
\end{pgfscope}%
\begin{pgfscope}%
\pgfsys@transformshift{2.037379in}{1.137810in}%
\pgfsys@useobject{currentmarker}{}%
\end{pgfscope}%
\begin{pgfscope}%
\pgfsys@transformshift{2.227778in}{0.878597in}%
\pgfsys@useobject{currentmarker}{}%
\end{pgfscope}%
\begin{pgfscope}%
\pgfsys@transformshift{2.217038in}{0.885537in}%
\pgfsys@useobject{currentmarker}{}%
\end{pgfscope}%
\begin{pgfscope}%
\pgfsys@transformshift{2.365198in}{0.866975in}%
\pgfsys@useobject{currentmarker}{}%
\end{pgfscope}%
\begin{pgfscope}%
\pgfsys@transformshift{1.783431in}{2.158831in}%
\pgfsys@useobject{currentmarker}{}%
\end{pgfscope}%
\begin{pgfscope}%
\pgfsys@transformshift{1.963318in}{2.960990in}%
\pgfsys@useobject{currentmarker}{}%
\end{pgfscope}%
\begin{pgfscope}%
\pgfsys@transformshift{1.978908in}{1.268350in}%
\pgfsys@useobject{currentmarker}{}%
\end{pgfscope}%
\begin{pgfscope}%
\pgfsys@transformshift{1.991849in}{1.987682in}%
\pgfsys@useobject{currentmarker}{}%
\end{pgfscope}%
\begin{pgfscope}%
\pgfsys@transformshift{2.060920in}{0.809191in}%
\pgfsys@useobject{currentmarker}{}%
\end{pgfscope}%
\begin{pgfscope}%
\pgfsys@transformshift{3.085911in}{0.739993in}%
\pgfsys@useobject{currentmarker}{}%
\end{pgfscope}%
\begin{pgfscope}%
\pgfsys@transformshift{1.552132in}{0.929506in}%
\pgfsys@useobject{currentmarker}{}%
\end{pgfscope}%
\begin{pgfscope}%
\pgfsys@transformshift{2.140592in}{0.760749in}%
\pgfsys@useobject{currentmarker}{}%
\end{pgfscope}%
\begin{pgfscope}%
\pgfsys@transformshift{2.270983in}{0.744746in}%
\pgfsys@useobject{currentmarker}{}%
\end{pgfscope}%
\begin{pgfscope}%
\pgfsys@transformshift{1.720111in}{0.866854in}%
\pgfsys@useobject{currentmarker}{}%
\end{pgfscope}%
\begin{pgfscope}%
\pgfsys@transformshift{1.695009in}{0.869095in}%
\pgfsys@useobject{currentmarker}{}%
\end{pgfscope}%
\begin{pgfscope}%
\pgfsys@transformshift{2.109068in}{0.825245in}%
\pgfsys@useobject{currentmarker}{}%
\end{pgfscope}%
\begin{pgfscope}%
\pgfsys@transformshift{1.617522in}{0.952845in}%
\pgfsys@useobject{currentmarker}{}%
\end{pgfscope}%
\begin{pgfscope}%
\pgfsys@transformshift{1.846036in}{0.880155in}%
\pgfsys@useobject{currentmarker}{}%
\end{pgfscope}%
\begin{pgfscope}%
\pgfsys@transformshift{1.910835in}{0.876791in}%
\pgfsys@useobject{currentmarker}{}%
\end{pgfscope}%
\begin{pgfscope}%
\pgfsys@transformshift{1.651967in}{4.844526in}%
\pgfsys@useobject{currentmarker}{}%
\end{pgfscope}%
\begin{pgfscope}%
\pgfsys@transformshift{1.655078in}{0.991609in}%
\pgfsys@useobject{currentmarker}{}%
\end{pgfscope}%
\begin{pgfscope}%
\pgfsys@transformshift{2.342409in}{0.820679in}%
\pgfsys@useobject{currentmarker}{}%
\end{pgfscope}%
\begin{pgfscope}%
\pgfsys@transformshift{1.946479in}{0.880739in}%
\pgfsys@useobject{currentmarker}{}%
\end{pgfscope}%
\begin{pgfscope}%
\pgfsys@transformshift{2.330531in}{0.827424in}%
\pgfsys@useobject{currentmarker}{}%
\end{pgfscope}%
\begin{pgfscope}%
\pgfsys@transformshift{2.158419in}{0.867076in}%
\pgfsys@useobject{currentmarker}{}%
\end{pgfscope}%
\begin{pgfscope}%
\pgfsys@transformshift{1.682128in}{1.056486in}%
\pgfsys@useobject{currentmarker}{}%
\end{pgfscope}%
\begin{pgfscope}%
\pgfsys@transformshift{2.939814in}{0.806323in}%
\pgfsys@useobject{currentmarker}{}%
\end{pgfscope}%
\begin{pgfscope}%
\pgfsys@transformshift{1.943459in}{0.968599in}%
\pgfsys@useobject{currentmarker}{}%
\end{pgfscope}%
\begin{pgfscope}%
\pgfsys@transformshift{2.255999in}{0.843211in}%
\pgfsys@useobject{currentmarker}{}%
\end{pgfscope}%
\begin{pgfscope}%
\pgfsys@transformshift{1.767626in}{1.703387in}%
\pgfsys@useobject{currentmarker}{}%
\end{pgfscope}%
\begin{pgfscope}%
\pgfsys@transformshift{1.725514in}{2.617826in}%
\pgfsys@useobject{currentmarker}{}%
\end{pgfscope}%
\begin{pgfscope}%
\pgfsys@transformshift{3.286826in}{0.817718in}%
\pgfsys@useobject{currentmarker}{}%
\end{pgfscope}%
\begin{pgfscope}%
\pgfsys@transformshift{2.388018in}{0.829140in}%
\pgfsys@useobject{currentmarker}{}%
\end{pgfscope}%
\begin{pgfscope}%
\pgfsys@transformshift{3.766696in}{0.770312in}%
\pgfsys@useobject{currentmarker}{}%
\end{pgfscope}%
\begin{pgfscope}%
\pgfsys@transformshift{3.617724in}{0.808912in}%
\pgfsys@useobject{currentmarker}{}%
\end{pgfscope}%
\begin{pgfscope}%
\pgfsys@transformshift{1.988388in}{1.047137in}%
\pgfsys@useobject{currentmarker}{}%
\end{pgfscope}%
\begin{pgfscope}%
\pgfsys@transformshift{2.175748in}{0.913320in}%
\pgfsys@useobject{currentmarker}{}%
\end{pgfscope}%
\begin{pgfscope}%
\pgfsys@transformshift{2.100041in}{0.918554in}%
\pgfsys@useobject{currentmarker}{}%
\end{pgfscope}%
\begin{pgfscope}%
\pgfsys@transformshift{1.881504in}{1.085658in}%
\pgfsys@useobject{currentmarker}{}%
\end{pgfscope}%
\begin{pgfscope}%
\pgfsys@transformshift{2.280519in}{0.849267in}%
\pgfsys@useobject{currentmarker}{}%
\end{pgfscope}%
\begin{pgfscope}%
\pgfsys@transformshift{2.728892in}{0.829347in}%
\pgfsys@useobject{currentmarker}{}%
\end{pgfscope}%
\begin{pgfscope}%
\pgfsys@transformshift{2.287913in}{0.847018in}%
\pgfsys@useobject{currentmarker}{}%
\end{pgfscope}%
\begin{pgfscope}%
\pgfsys@transformshift{2.961132in}{0.737694in}%
\pgfsys@useobject{currentmarker}{}%
\end{pgfscope}%
\begin{pgfscope}%
\pgfsys@transformshift{2.124613in}{0.760739in}%
\pgfsys@useobject{currentmarker}{}%
\end{pgfscope}%
\begin{pgfscope}%
\pgfsys@transformshift{1.517354in}{0.946872in}%
\pgfsys@useobject{currentmarker}{}%
\end{pgfscope}%
\begin{pgfscope}%
\pgfsys@transformshift{1.499158in}{1.093120in}%
\pgfsys@useobject{currentmarker}{}%
\end{pgfscope}%
\begin{pgfscope}%
\pgfsys@transformshift{1.758242in}{0.778794in}%
\pgfsys@useobject{currentmarker}{}%
\end{pgfscope}%
\begin{pgfscope}%
\pgfsys@transformshift{2.262076in}{0.754203in}%
\pgfsys@useobject{currentmarker}{}%
\end{pgfscope}%
\begin{pgfscope}%
\pgfsys@transformshift{1.692340in}{0.886791in}%
\pgfsys@useobject{currentmarker}{}%
\end{pgfscope}%
\begin{pgfscope}%
\pgfsys@transformshift{1.629593in}{0.929750in}%
\pgfsys@useobject{currentmarker}{}%
\end{pgfscope}%
\begin{pgfscope}%
\pgfsys@transformshift{2.800523in}{0.748212in}%
\pgfsys@useobject{currentmarker}{}%
\end{pgfscope}%
\begin{pgfscope}%
\pgfsys@transformshift{1.897616in}{0.829063in}%
\pgfsys@useobject{currentmarker}{}%
\end{pgfscope}%
\begin{pgfscope}%
\pgfsys@transformshift{1.797237in}{0.865306in}%
\pgfsys@useobject{currentmarker}{}%
\end{pgfscope}%
\begin{pgfscope}%
\pgfsys@transformshift{2.112024in}{0.782981in}%
\pgfsys@useobject{currentmarker}{}%
\end{pgfscope}%
\begin{pgfscope}%
\pgfsys@transformshift{2.517645in}{0.760248in}%
\pgfsys@useobject{currentmarker}{}%
\end{pgfscope}%
\begin{pgfscope}%
\pgfsys@transformshift{3.695237in}{0.756982in}%
\pgfsys@useobject{currentmarker}{}%
\end{pgfscope}%
\begin{pgfscope}%
\pgfsys@transformshift{3.811388in}{0.750642in}%
\pgfsys@useobject{currentmarker}{}%
\end{pgfscope}%
\begin{pgfscope}%
\pgfsys@transformshift{1.645167in}{1.061311in}%
\pgfsys@useobject{currentmarker}{}%
\end{pgfscope}%
\begin{pgfscope}%
\pgfsys@transformshift{1.727289in}{0.959375in}%
\pgfsys@useobject{currentmarker}{}%
\end{pgfscope}%
\begin{pgfscope}%
\pgfsys@transformshift{1.594165in}{5.173601in}%
\pgfsys@useobject{currentmarker}{}%
\end{pgfscope}%
\begin{pgfscope}%
\pgfsys@transformshift{1.981719in}{0.860080in}%
\pgfsys@useobject{currentmarker}{}%
\end{pgfscope}%
\begin{pgfscope}%
\pgfsys@transformshift{1.942974in}{0.872299in}%
\pgfsys@useobject{currentmarker}{}%
\end{pgfscope}%
\begin{pgfscope}%
\pgfsys@transformshift{2.203924in}{0.815880in}%
\pgfsys@useobject{currentmarker}{}%
\end{pgfscope}%
\begin{pgfscope}%
\pgfsys@transformshift{2.157362in}{0.822947in}%
\pgfsys@useobject{currentmarker}{}%
\end{pgfscope}%
\begin{pgfscope}%
\pgfsys@transformshift{2.560471in}{0.777847in}%
\pgfsys@useobject{currentmarker}{}%
\end{pgfscope}%
\begin{pgfscope}%
\pgfsys@transformshift{1.749304in}{0.973315in}%
\pgfsys@useobject{currentmarker}{}%
\end{pgfscope}%
\begin{pgfscope}%
\pgfsys@transformshift{1.621914in}{2.849282in}%
\pgfsys@useobject{currentmarker}{}%
\end{pgfscope}%
\begin{pgfscope}%
\pgfsys@transformshift{2.069294in}{0.853687in}%
\pgfsys@useobject{currentmarker}{}%
\end{pgfscope}%
\begin{pgfscope}%
\pgfsys@transformshift{1.910835in}{0.876791in}%
\pgfsys@useobject{currentmarker}{}%
\end{pgfscope}%
\begin{pgfscope}%
\pgfsys@transformshift{2.263043in}{0.800003in}%
\pgfsys@useobject{currentmarker}{}%
\end{pgfscope}%
\begin{pgfscope}%
\pgfsys@transformshift{2.183472in}{0.833074in}%
\pgfsys@useobject{currentmarker}{}%
\end{pgfscope}%
\begin{pgfscope}%
\pgfsys@transformshift{3.446266in}{0.772074in}%
\pgfsys@useobject{currentmarker}{}%
\end{pgfscope}%
\begin{pgfscope}%
\pgfsys@transformshift{2.867703in}{0.793955in}%
\pgfsys@useobject{currentmarker}{}%
\end{pgfscope}%
\begin{pgfscope}%
\pgfsys@transformshift{1.766551in}{1.845080in}%
\pgfsys@useobject{currentmarker}{}%
\end{pgfscope}%
\begin{pgfscope}%
\pgfsys@transformshift{1.651967in}{4.844526in}%
\pgfsys@useobject{currentmarker}{}%
\end{pgfscope}%
\begin{pgfscope}%
\pgfsys@transformshift{2.131008in}{0.864968in}%
\pgfsys@useobject{currentmarker}{}%
\end{pgfscope}%
\begin{pgfscope}%
\pgfsys@transformshift{1.874659in}{0.927675in}%
\pgfsys@useobject{currentmarker}{}%
\end{pgfscope}%
\begin{pgfscope}%
\pgfsys@transformshift{1.923467in}{0.927051in}%
\pgfsys@useobject{currentmarker}{}%
\end{pgfscope}%
\begin{pgfscope}%
\pgfsys@transformshift{1.926437in}{0.886999in}%
\pgfsys@useobject{currentmarker}{}%
\end{pgfscope}%
\begin{pgfscope}%
\pgfsys@transformshift{2.554497in}{0.804258in}%
\pgfsys@useobject{currentmarker}{}%
\end{pgfscope}%
\begin{pgfscope}%
\pgfsys@transformshift{2.218618in}{0.832498in}%
\pgfsys@useobject{currentmarker}{}%
\end{pgfscope}%
\begin{pgfscope}%
\pgfsys@transformshift{2.229538in}{0.829645in}%
\pgfsys@useobject{currentmarker}{}%
\end{pgfscope}%
\begin{pgfscope}%
\pgfsys@transformshift{1.692065in}{0.776827in}%
\pgfsys@useobject{currentmarker}{}%
\end{pgfscope}%
\begin{pgfscope}%
\pgfsys@transformshift{1.655493in}{0.848984in}%
\pgfsys@useobject{currentmarker}{}%
\end{pgfscope}%
\begin{pgfscope}%
\pgfsys@transformshift{1.686007in}{0.828545in}%
\pgfsys@useobject{currentmarker}{}%
\end{pgfscope}%
\begin{pgfscope}%
\pgfsys@transformshift{1.164330in}{2.493998in}%
\pgfsys@useobject{currentmarker}{}%
\end{pgfscope}%
\begin{pgfscope}%
\pgfsys@transformshift{3.232918in}{0.731169in}%
\pgfsys@useobject{currentmarker}{}%
\end{pgfscope}%
\begin{pgfscope}%
\pgfsys@transformshift{2.142373in}{0.748218in}%
\pgfsys@useobject{currentmarker}{}%
\end{pgfscope}%
\begin{pgfscope}%
\pgfsys@transformshift{1.530195in}{0.951425in}%
\pgfsys@useobject{currentmarker}{}%
\end{pgfscope}%
\begin{pgfscope}%
\pgfsys@transformshift{1.656415in}{0.912514in}%
\pgfsys@useobject{currentmarker}{}%
\end{pgfscope}%
\begin{pgfscope}%
\pgfsys@transformshift{1.705245in}{0.832399in}%
\pgfsys@useobject{currentmarker}{}%
\end{pgfscope}%
\begin{pgfscope}%
\pgfsys@transformshift{1.704944in}{0.836479in}%
\pgfsys@useobject{currentmarker}{}%
\end{pgfscope}%
\begin{pgfscope}%
\pgfsys@transformshift{1.418224in}{2.523706in}%
\pgfsys@useobject{currentmarker}{}%
\end{pgfscope}%
\begin{pgfscope}%
\pgfsys@transformshift{3.877349in}{0.733001in}%
\pgfsys@useobject{currentmarker}{}%
\end{pgfscope}%
\begin{pgfscope}%
\pgfsys@transformshift{3.613289in}{0.740982in}%
\pgfsys@useobject{currentmarker}{}%
\end{pgfscope}%
\begin{pgfscope}%
\pgfsys@transformshift{2.202142in}{0.762267in}%
\pgfsys@useobject{currentmarker}{}%
\end{pgfscope}%
\begin{pgfscope}%
\pgfsys@transformshift{1.786256in}{0.808961in}%
\pgfsys@useobject{currentmarker}{}%
\end{pgfscope}%
\begin{pgfscope}%
\pgfsys@transformshift{2.112024in}{0.782981in}%
\pgfsys@useobject{currentmarker}{}%
\end{pgfscope}%
\begin{pgfscope}%
\pgfsys@transformshift{1.750298in}{0.865031in}%
\pgfsys@useobject{currentmarker}{}%
\end{pgfscope}%
\begin{pgfscope}%
\pgfsys@transformshift{1.590786in}{2.843134in}%
\pgfsys@useobject{currentmarker}{}%
\end{pgfscope}%
\begin{pgfscope}%
\pgfsys@transformshift{1.523544in}{4.790932in}%
\pgfsys@useobject{currentmarker}{}%
\end{pgfscope}%
\begin{pgfscope}%
\pgfsys@transformshift{2.895425in}{0.757664in}%
\pgfsys@useobject{currentmarker}{}%
\end{pgfscope}%
\begin{pgfscope}%
\pgfsys@transformshift{2.447740in}{0.758638in}%
\pgfsys@useobject{currentmarker}{}%
\end{pgfscope}%
\begin{pgfscope}%
\pgfsys@transformshift{2.300687in}{0.760315in}%
\pgfsys@useobject{currentmarker}{}%
\end{pgfscope}%
\begin{pgfscope}%
\pgfsys@transformshift{3.755727in}{0.751840in}%
\pgfsys@useobject{currentmarker}{}%
\end{pgfscope}%
\begin{pgfscope}%
\pgfsys@transformshift{1.655833in}{0.990553in}%
\pgfsys@useobject{currentmarker}{}%
\end{pgfscope}%
\begin{pgfscope}%
\pgfsys@transformshift{1.676509in}{0.958598in}%
\pgfsys@useobject{currentmarker}{}%
\end{pgfscope}%
\begin{pgfscope}%
\pgfsys@transformshift{1.618329in}{1.880186in}%
\pgfsys@useobject{currentmarker}{}%
\end{pgfscope}%
\begin{pgfscope}%
\pgfsys@transformshift{2.035321in}{0.822388in}%
\pgfsys@useobject{currentmarker}{}%
\end{pgfscope}%
\begin{pgfscope}%
\pgfsys@transformshift{2.009060in}{0.824090in}%
\pgfsys@useobject{currentmarker}{}%
\end{pgfscope}%
\begin{pgfscope}%
\pgfsys@transformshift{1.788081in}{0.862958in}%
\pgfsys@useobject{currentmarker}{}%
\end{pgfscope}%
\begin{pgfscope}%
\pgfsys@transformshift{2.293057in}{0.785866in}%
\pgfsys@useobject{currentmarker}{}%
\end{pgfscope}%
\begin{pgfscope}%
\pgfsys@transformshift{1.692475in}{0.887997in}%
\pgfsys@useobject{currentmarker}{}%
\end{pgfscope}%
\begin{pgfscope}%
\pgfsys@transformshift{1.735067in}{0.887705in}%
\pgfsys@useobject{currentmarker}{}%
\end{pgfscope}%
\begin{pgfscope}%
\pgfsys@transformshift{2.439268in}{0.760318in}%
\pgfsys@useobject{currentmarker}{}%
\end{pgfscope}%
\begin{pgfscope}%
\pgfsys@transformshift{1.664718in}{2.040292in}%
\pgfsys@useobject{currentmarker}{}%
\end{pgfscope}%
\begin{pgfscope}%
\pgfsys@transformshift{1.978941in}{0.855978in}%
\pgfsys@useobject{currentmarker}{}%
\end{pgfscope}%
\begin{pgfscope}%
\pgfsys@transformshift{1.699309in}{0.888503in}%
\pgfsys@useobject{currentmarker}{}%
\end{pgfscope}%
\begin{pgfscope}%
\pgfsys@transformshift{2.761865in}{0.761944in}%
\pgfsys@useobject{currentmarker}{}%
\end{pgfscope}%
\begin{pgfscope}%
\pgfsys@transformshift{1.666529in}{2.776429in}%
\pgfsys@useobject{currentmarker}{}%
\end{pgfscope}%
\begin{pgfscope}%
\pgfsys@transformshift{1.712564in}{0.891127in}%
\pgfsys@useobject{currentmarker}{}%
\end{pgfscope}%
\begin{pgfscope}%
\pgfsys@transformshift{1.502211in}{0.930798in}%
\pgfsys@useobject{currentmarker}{}%
\end{pgfscope}%
\begin{pgfscope}%
\pgfsys@transformshift{1.605281in}{0.852221in}%
\pgfsys@useobject{currentmarker}{}%
\end{pgfscope}%
\begin{pgfscope}%
\pgfsys@transformshift{1.628638in}{0.851570in}%
\pgfsys@useobject{currentmarker}{}%
\end{pgfscope}%
\begin{pgfscope}%
\pgfsys@transformshift{1.453530in}{2.135612in}%
\pgfsys@useobject{currentmarker}{}%
\end{pgfscope}%
\begin{pgfscope}%
\pgfsys@transformshift{1.710741in}{0.768842in}%
\pgfsys@useobject{currentmarker}{}%
\end{pgfscope}%
\begin{pgfscope}%
\pgfsys@transformshift{2.187608in}{0.747228in}%
\pgfsys@useobject{currentmarker}{}%
\end{pgfscope}%
\begin{pgfscope}%
\pgfsys@transformshift{0.855425in}{4.235940in}%
\pgfsys@useobject{currentmarker}{}%
\end{pgfscope}%
\begin{pgfscope}%
\pgfsys@transformshift{1.458133in}{1.695918in}%
\pgfsys@useobject{currentmarker}{}%
\end{pgfscope}%
\begin{pgfscope}%
\pgfsys@transformshift{1.453244in}{2.207777in}%
\pgfsys@useobject{currentmarker}{}%
\end{pgfscope}%
\begin{pgfscope}%
\pgfsys@transformshift{4.977495in}{0.679666in}%
\pgfsys@useobject{currentmarker}{}%
\end{pgfscope}%
\begin{pgfscope}%
\pgfsys@transformshift{1.691837in}{0.826259in}%
\pgfsys@useobject{currentmarker}{}%
\end{pgfscope}%
\begin{pgfscope}%
\pgfsys@transformshift{1.728425in}{0.761284in}%
\pgfsys@useobject{currentmarker}{}%
\end{pgfscope}%
\begin{pgfscope}%
\pgfsys@transformshift{1.701504in}{0.809033in}%
\pgfsys@useobject{currentmarker}{}%
\end{pgfscope}%
\begin{pgfscope}%
\pgfsys@transformshift{1.667425in}{0.849318in}%
\pgfsys@useobject{currentmarker}{}%
\end{pgfscope}%
\begin{pgfscope}%
\pgfsys@transformshift{1.503539in}{1.099915in}%
\pgfsys@useobject{currentmarker}{}%
\end{pgfscope}%
\begin{pgfscope}%
\pgfsys@transformshift{1.521160in}{0.936762in}%
\pgfsys@useobject{currentmarker}{}%
\end{pgfscope}%
\begin{pgfscope}%
\pgfsys@transformshift{1.606897in}{0.895152in}%
\pgfsys@useobject{currentmarker}{}%
\end{pgfscope}%
\begin{pgfscope}%
\pgfsys@transformshift{1.624763in}{0.886640in}%
\pgfsys@useobject{currentmarker}{}%
\end{pgfscope}%
\begin{pgfscope}%
\pgfsys@transformshift{1.647831in}{0.873933in}%
\pgfsys@useobject{currentmarker}{}%
\end{pgfscope}%
\begin{pgfscope}%
\pgfsys@transformshift{1.646747in}{0.876057in}%
\pgfsys@useobject{currentmarker}{}%
\end{pgfscope}%
\begin{pgfscope}%
\pgfsys@transformshift{1.659191in}{0.852467in}%
\pgfsys@useobject{currentmarker}{}%
\end{pgfscope}%
\begin{pgfscope}%
\pgfsys@transformshift{1.634871in}{0.880504in}%
\pgfsys@useobject{currentmarker}{}%
\end{pgfscope}%
\begin{pgfscope}%
\pgfsys@transformshift{1.282412in}{5.004304in}%
\pgfsys@useobject{currentmarker}{}%
\end{pgfscope}%
\begin{pgfscope}%
\pgfsys@transformshift{1.459500in}{2.447155in}%
\pgfsys@useobject{currentmarker}{}%
\end{pgfscope}%
\begin{pgfscope}%
\pgfsys@transformshift{2.117985in}{0.766159in}%
\pgfsys@useobject{currentmarker}{}%
\end{pgfscope}%
\begin{pgfscope}%
\pgfsys@transformshift{1.736052in}{0.777951in}%
\pgfsys@useobject{currentmarker}{}%
\end{pgfscope}%
\begin{pgfscope}%
\pgfsys@transformshift{1.676226in}{0.850144in}%
\pgfsys@useobject{currentmarker}{}%
\end{pgfscope}%
\begin{pgfscope}%
\pgfsys@transformshift{2.449430in}{0.758634in}%
\pgfsys@useobject{currentmarker}{}%
\end{pgfscope}%
\begin{pgfscope}%
\pgfsys@transformshift{1.350198in}{5.303139in}%
\pgfsys@useobject{currentmarker}{}%
\end{pgfscope}%
\begin{pgfscope}%
\pgfsys@transformshift{3.755727in}{0.751840in}%
\pgfsys@useobject{currentmarker}{}%
\end{pgfscope}%
\begin{pgfscope}%
\pgfsys@transformshift{1.521082in}{1.181628in}%
\pgfsys@useobject{currentmarker}{}%
\end{pgfscope}%
\begin{pgfscope}%
\pgfsys@transformshift{1.505288in}{3.007886in}%
\pgfsys@useobject{currentmarker}{}%
\end{pgfscope}%
\begin{pgfscope}%
\pgfsys@transformshift{1.627313in}{0.939675in}%
\pgfsys@useobject{currentmarker}{}%
\end{pgfscope}%
\begin{pgfscope}%
\pgfsys@transformshift{1.576114in}{1.159187in}%
\pgfsys@useobject{currentmarker}{}%
\end{pgfscope}%
\begin{pgfscope}%
\pgfsys@transformshift{1.688572in}{0.898820in}%
\pgfsys@useobject{currentmarker}{}%
\end{pgfscope}%
\begin{pgfscope}%
\pgfsys@transformshift{2.373971in}{0.770351in}%
\pgfsys@useobject{currentmarker}{}%
\end{pgfscope}%
\begin{pgfscope}%
\pgfsys@transformshift{1.881814in}{0.790413in}%
\pgfsys@useobject{currentmarker}{}%
\end{pgfscope}%
\begin{pgfscope}%
\pgfsys@transformshift{1.759480in}{0.835060in}%
\pgfsys@useobject{currentmarker}{}%
\end{pgfscope}%
\begin{pgfscope}%
\pgfsys@transformshift{1.774881in}{0.858971in}%
\pgfsys@useobject{currentmarker}{}%
\end{pgfscope}%
\begin{pgfscope}%
\pgfsys@transformshift{3.152914in}{0.716174in}%
\pgfsys@useobject{currentmarker}{}%
\end{pgfscope}%
\begin{pgfscope}%
\pgfsys@transformshift{1.603681in}{0.852343in}%
\pgfsys@useobject{currentmarker}{}%
\end{pgfscope}%
\begin{pgfscope}%
\pgfsys@transformshift{1.638089in}{0.802473in}%
\pgfsys@useobject{currentmarker}{}%
\end{pgfscope}%
\begin{pgfscope}%
\pgfsys@transformshift{4.315380in}{0.688043in}%
\pgfsys@useobject{currentmarker}{}%
\end{pgfscope}%
\begin{pgfscope}%
\pgfsys@transformshift{2.961132in}{0.737694in}%
\pgfsys@useobject{currentmarker}{}%
\end{pgfscope}%
\begin{pgfscope}%
\pgfsys@transformshift{0.855425in}{4.235940in}%
\pgfsys@useobject{currentmarker}{}%
\end{pgfscope}%
\begin{pgfscope}%
\pgfsys@transformshift{2.270816in}{0.744689in}%
\pgfsys@useobject{currentmarker}{}%
\end{pgfscope}%
\begin{pgfscope}%
\pgfsys@transformshift{0.806050in}{4.540760in}%
\pgfsys@useobject{currentmarker}{}%
\end{pgfscope}%
\begin{pgfscope}%
\pgfsys@transformshift{1.837252in}{0.760809in}%
\pgfsys@useobject{currentmarker}{}%
\end{pgfscope}%
\begin{pgfscope}%
\pgfsys@transformshift{1.375497in}{1.141330in}%
\pgfsys@useobject{currentmarker}{}%
\end{pgfscope}%
\begin{pgfscope}%
\pgfsys@transformshift{2.140099in}{0.760749in}%
\pgfsys@useobject{currentmarker}{}%
\end{pgfscope}%
\begin{pgfscope}%
\pgfsys@transformshift{1.499158in}{1.093120in}%
\pgfsys@useobject{currentmarker}{}%
\end{pgfscope}%
\begin{pgfscope}%
\pgfsys@transformshift{1.710028in}{0.788729in}%
\pgfsys@useobject{currentmarker}{}%
\end{pgfscope}%
\begin{pgfscope}%
\pgfsys@transformshift{1.328683in}{3.322299in}%
\pgfsys@useobject{currentmarker}{}%
\end{pgfscope}%
\begin{pgfscope}%
\pgfsys@transformshift{1.358634in}{3.045153in}%
\pgfsys@useobject{currentmarker}{}%
\end{pgfscope}%
\begin{pgfscope}%
\pgfsys@transformshift{1.507021in}{1.013929in}%
\pgfsys@useobject{currentmarker}{}%
\end{pgfscope}%
\begin{pgfscope}%
\pgfsys@transformshift{1.812651in}{0.764280in}%
\pgfsys@useobject{currentmarker}{}%
\end{pgfscope}%
\begin{pgfscope}%
\pgfsys@transformshift{1.754459in}{0.764869in}%
\pgfsys@useobject{currentmarker}{}%
\end{pgfscope}%
\begin{pgfscope}%
\pgfsys@transformshift{1.624754in}{0.859508in}%
\pgfsys@useobject{currentmarker}{}%
\end{pgfscope}%
\begin{pgfscope}%
\pgfsys@transformshift{1.454807in}{1.684553in}%
\pgfsys@useobject{currentmarker}{}%
\end{pgfscope}%
\begin{pgfscope}%
\pgfsys@transformshift{2.140838in}{0.760749in}%
\pgfsys@useobject{currentmarker}{}%
\end{pgfscope}%
\begin{pgfscope}%
\pgfsys@transformshift{1.724505in}{0.806842in}%
\pgfsys@useobject{currentmarker}{}%
\end{pgfscope}%
\begin{pgfscope}%
\pgfsys@transformshift{1.513300in}{1.068821in}%
\pgfsys@useobject{currentmarker}{}%
\end{pgfscope}%
\begin{pgfscope}%
\pgfsys@transformshift{1.615177in}{0.937936in}%
\pgfsys@useobject{currentmarker}{}%
\end{pgfscope}%
\begin{pgfscope}%
\pgfsys@transformshift{1.636005in}{0.880232in}%
\pgfsys@useobject{currentmarker}{}%
\end{pgfscope}%
\begin{pgfscope}%
\pgfsys@transformshift{1.655493in}{0.848984in}%
\pgfsys@useobject{currentmarker}{}%
\end{pgfscope}%
\begin{pgfscope}%
\pgfsys@transformshift{3.613289in}{0.740982in}%
\pgfsys@useobject{currentmarker}{}%
\end{pgfscope}%
\begin{pgfscope}%
\pgfsys@transformshift{1.223341in}{5.145058in}%
\pgfsys@useobject{currentmarker}{}%
\end{pgfscope}%
\begin{pgfscope}%
\pgfsys@transformshift{2.305225in}{0.760315in}%
\pgfsys@useobject{currentmarker}{}%
\end{pgfscope}%
\begin{pgfscope}%
\pgfsys@transformshift{1.431523in}{3.425771in}%
\pgfsys@useobject{currentmarker}{}%
\end{pgfscope}%
\begin{pgfscope}%
\pgfsys@transformshift{1.455073in}{2.467371in}%
\pgfsys@useobject{currentmarker}{}%
\end{pgfscope}%
\begin{pgfscope}%
\pgfsys@transformshift{1.760888in}{0.804325in}%
\pgfsys@useobject{currentmarker}{}%
\end{pgfscope}%
\begin{pgfscope}%
\pgfsys@transformshift{1.350198in}{5.303139in}%
\pgfsys@useobject{currentmarker}{}%
\end{pgfscope}%
\begin{pgfscope}%
\pgfsys@transformshift{1.578178in}{0.939952in}%
\pgfsys@useobject{currentmarker}{}%
\end{pgfscope}%
\begin{pgfscope}%
\pgfsys@transformshift{2.897084in}{0.756951in}%
\pgfsys@useobject{currentmarker}{}%
\end{pgfscope}%
\begin{pgfscope}%
\pgfsys@transformshift{1.627313in}{0.939675in}%
\pgfsys@useobject{currentmarker}{}%
\end{pgfscope}%
\begin{pgfscope}%
\pgfsys@transformshift{1.853469in}{0.788449in}%
\pgfsys@useobject{currentmarker}{}%
\end{pgfscope}%
\begin{pgfscope}%
\pgfsys@transformshift{1.725899in}{0.831142in}%
\pgfsys@useobject{currentmarker}{}%
\end{pgfscope}%
\begin{pgfscope}%
\pgfsys@transformshift{2.935095in}{0.738614in}%
\pgfsys@useobject{currentmarker}{}%
\end{pgfscope}%
\begin{pgfscope}%
\pgfsys@transformshift{1.366484in}{1.065376in}%
\pgfsys@useobject{currentmarker}{}%
\end{pgfscope}%
\begin{pgfscope}%
\pgfsys@transformshift{1.716135in}{0.760511in}%
\pgfsys@useobject{currentmarker}{}%
\end{pgfscope}%
\begin{pgfscope}%
\pgfsys@transformshift{1.941094in}{0.742563in}%
\pgfsys@useobject{currentmarker}{}%
\end{pgfscope}%
\begin{pgfscope}%
\pgfsys@transformshift{1.456240in}{0.760647in}%
\pgfsys@useobject{currentmarker}{}%
\end{pgfscope}%
\begin{pgfscope}%
\pgfsys@transformshift{1.381388in}{0.778752in}%
\pgfsys@useobject{currentmarker}{}%
\end{pgfscope}%
\begin{pgfscope}%
\pgfsys@transformshift{3.988018in}{0.689972in}%
\pgfsys@useobject{currentmarker}{}%
\end{pgfscope}%
\begin{pgfscope}%
\pgfsys@transformshift{1.247065in}{4.139937in}%
\pgfsys@useobject{currentmarker}{}%
\end{pgfscope}%
\begin{pgfscope}%
\pgfsys@transformshift{3.788060in}{0.726555in}%
\pgfsys@useobject{currentmarker}{}%
\end{pgfscope}%
\begin{pgfscope}%
\pgfsys@transformshift{2.015825in}{0.760476in}%
\pgfsys@useobject{currentmarker}{}%
\end{pgfscope}%
\begin{pgfscope}%
\pgfsys@transformshift{1.710336in}{0.768426in}%
\pgfsys@useobject{currentmarker}{}%
\end{pgfscope}%
\begin{pgfscope}%
\pgfsys@transformshift{1.736069in}{0.760934in}%
\pgfsys@useobject{currentmarker}{}%
\end{pgfscope}%
\begin{pgfscope}%
\pgfsys@transformshift{1.619986in}{0.817703in}%
\pgfsys@useobject{currentmarker}{}%
\end{pgfscope}%
\begin{pgfscope}%
\pgfsys@transformshift{1.488234in}{1.059782in}%
\pgfsys@useobject{currentmarker}{}%
\end{pgfscope}%
\begin{pgfscope}%
\pgfsys@transformshift{1.561278in}{0.859087in}%
\pgfsys@useobject{currentmarker}{}%
\end{pgfscope}%
\begin{pgfscope}%
\pgfsys@transformshift{1.591048in}{0.849700in}%
\pgfsys@useobject{currentmarker}{}%
\end{pgfscope}%
\begin{pgfscope}%
\pgfsys@transformshift{1.330153in}{3.334094in}%
\pgfsys@useobject{currentmarker}{}%
\end{pgfscope}%
\begin{pgfscope}%
\pgfsys@transformshift{1.281707in}{5.102503in}%
\pgfsys@useobject{currentmarker}{}%
\end{pgfscope}%
\begin{pgfscope}%
\pgfsys@transformshift{1.381216in}{2.187384in}%
\pgfsys@useobject{currentmarker}{}%
\end{pgfscope}%
\begin{pgfscope}%
\pgfsys@transformshift{2.194335in}{0.760491in}%
\pgfsys@useobject{currentmarker}{}%
\end{pgfscope}%
\begin{pgfscope}%
\pgfsys@transformshift{1.692417in}{0.777223in}%
\pgfsys@useobject{currentmarker}{}%
\end{pgfscope}%
\begin{pgfscope}%
\pgfsys@transformshift{1.897064in}{0.762405in}%
\pgfsys@useobject{currentmarker}{}%
\end{pgfscope}%
\begin{pgfscope}%
\pgfsys@transformshift{1.654490in}{0.822655in}%
\pgfsys@useobject{currentmarker}{}%
\end{pgfscope}%
\begin{pgfscope}%
\pgfsys@transformshift{1.582585in}{0.884303in}%
\pgfsys@useobject{currentmarker}{}%
\end{pgfscope}%
\begin{pgfscope}%
\pgfsys@transformshift{1.593295in}{0.852063in}%
\pgfsys@useobject{currentmarker}{}%
\end{pgfscope}%
\begin{pgfscope}%
\pgfsys@transformshift{1.235061in}{5.185253in}%
\pgfsys@useobject{currentmarker}{}%
\end{pgfscope}%
\begin{pgfscope}%
\pgfsys@transformshift{1.414276in}{2.451419in}%
\pgfsys@useobject{currentmarker}{}%
\end{pgfscope}%
\begin{pgfscope}%
\pgfsys@transformshift{2.353203in}{0.759353in}%
\pgfsys@useobject{currentmarker}{}%
\end{pgfscope}%
\begin{pgfscope}%
\pgfsys@transformshift{2.425962in}{0.758628in}%
\pgfsys@useobject{currentmarker}{}%
\end{pgfscope}%
\begin{pgfscope}%
\pgfsys@transformshift{1.833642in}{0.764298in}%
\pgfsys@useobject{currentmarker}{}%
\end{pgfscope}%
\begin{pgfscope}%
\pgfsys@transformshift{1.924369in}{0.762537in}%
\pgfsys@useobject{currentmarker}{}%
\end{pgfscope}%
\begin{pgfscope}%
\pgfsys@transformshift{1.531804in}{0.937313in}%
\pgfsys@useobject{currentmarker}{}%
\end{pgfscope}%
\begin{pgfscope}%
\pgfsys@transformshift{1.603681in}{0.852343in}%
\pgfsys@useobject{currentmarker}{}%
\end{pgfscope}%
\begin{pgfscope}%
\pgfsys@transformshift{1.321513in}{1.573984in}%
\pgfsys@useobject{currentmarker}{}%
\end{pgfscope}%
\begin{pgfscope}%
\pgfsys@transformshift{1.812598in}{0.760365in}%
\pgfsys@useobject{currentmarker}{}%
\end{pgfscope}%
\begin{pgfscope}%
\pgfsys@transformshift{1.140456in}{1.744378in}%
\pgfsys@useobject{currentmarker}{}%
\end{pgfscope}%
\begin{pgfscope}%
\pgfsys@transformshift{2.570547in}{0.734473in}%
\pgfsys@useobject{currentmarker}{}%
\end{pgfscope}%
\begin{pgfscope}%
\pgfsys@transformshift{2.912089in}{0.722754in}%
\pgfsys@useobject{currentmarker}{}%
\end{pgfscope}%
\begin{pgfscope}%
\pgfsys@transformshift{1.329451in}{1.187355in}%
\pgfsys@useobject{currentmarker}{}%
\end{pgfscope}%
\begin{pgfscope}%
\pgfsys@transformshift{3.152914in}{0.716174in}%
\pgfsys@useobject{currentmarker}{}%
\end{pgfscope}%
\begin{pgfscope}%
\pgfsys@transformshift{1.639350in}{0.779761in}%
\pgfsys@useobject{currentmarker}{}%
\end{pgfscope}%
\begin{pgfscope}%
\pgfsys@transformshift{1.404562in}{0.786943in}%
\pgfsys@useobject{currentmarker}{}%
\end{pgfscope}%
\begin{pgfscope}%
\pgfsys@transformshift{4.171915in}{0.691592in}%
\pgfsys@useobject{currentmarker}{}%
\end{pgfscope}%
\begin{pgfscope}%
\pgfsys@transformshift{2.107711in}{0.760365in}%
\pgfsys@useobject{currentmarker}{}%
\end{pgfscope}%
\begin{pgfscope}%
\pgfsys@transformshift{1.923385in}{0.760373in}%
\pgfsys@useobject{currentmarker}{}%
\end{pgfscope}%
\begin{pgfscope}%
\pgfsys@transformshift{3.184499in}{0.717898in}%
\pgfsys@useobject{currentmarker}{}%
\end{pgfscope}%
\begin{pgfscope}%
\pgfsys@transformshift{1.401484in}{1.773192in}%
\pgfsys@useobject{currentmarker}{}%
\end{pgfscope}%
\begin{pgfscope}%
\pgfsys@transformshift{1.738464in}{0.760942in}%
\pgfsys@useobject{currentmarker}{}%
\end{pgfscope}%
\begin{pgfscope}%
\pgfsys@transformshift{1.671803in}{0.797671in}%
\pgfsys@useobject{currentmarker}{}%
\end{pgfscope}%
\begin{pgfscope}%
\pgfsys@transformshift{1.441746in}{1.075567in}%
\pgfsys@useobject{currentmarker}{}%
\end{pgfscope}%
\begin{pgfscope}%
\pgfsys@transformshift{1.477721in}{0.986750in}%
\pgfsys@useobject{currentmarker}{}%
\end{pgfscope}%
\begin{pgfscope}%
\pgfsys@transformshift{4.302261in}{0.696497in}%
\pgfsys@useobject{currentmarker}{}%
\end{pgfscope}%
\begin{pgfscope}%
\pgfsys@transformshift{4.205305in}{0.702275in}%
\pgfsys@useobject{currentmarker}{}%
\end{pgfscope}%
\begin{pgfscope}%
\pgfsys@transformshift{1.968481in}{0.760473in}%
\pgfsys@useobject{currentmarker}{}%
\end{pgfscope}%
\begin{pgfscope}%
\pgfsys@transformshift{1.185642in}{2.698592in}%
\pgfsys@useobject{currentmarker}{}%
\end{pgfscope}%
\begin{pgfscope}%
\pgfsys@transformshift{3.391600in}{0.725006in}%
\pgfsys@useobject{currentmarker}{}%
\end{pgfscope}%
\begin{pgfscope}%
\pgfsys@transformshift{1.413795in}{2.440659in}%
\pgfsys@useobject{currentmarker}{}%
\end{pgfscope}%
\begin{pgfscope}%
\pgfsys@transformshift{1.441660in}{1.919189in}%
\pgfsys@useobject{currentmarker}{}%
\end{pgfscope}%
\begin{pgfscope}%
\pgfsys@transformshift{2.294593in}{0.760365in}%
\pgfsys@useobject{currentmarker}{}%
\end{pgfscope}%
\begin{pgfscope}%
\pgfsys@transformshift{1.753833in}{0.761274in}%
\pgfsys@useobject{currentmarker}{}%
\end{pgfscope}%
\begin{pgfscope}%
\pgfsys@transformshift{1.638832in}{0.802918in}%
\pgfsys@useobject{currentmarker}{}%
\end{pgfscope}%
\begin{pgfscope}%
\pgfsys@transformshift{1.633953in}{0.828154in}%
\pgfsys@useobject{currentmarker}{}%
\end{pgfscope}%
\begin{pgfscope}%
\pgfsys@transformshift{1.635392in}{0.824412in}%
\pgfsys@useobject{currentmarker}{}%
\end{pgfscope}%
\begin{pgfscope}%
\pgfsys@transformshift{1.568445in}{0.866847in}%
\pgfsys@useobject{currentmarker}{}%
\end{pgfscope}%
\begin{pgfscope}%
\pgfsys@transformshift{2.015550in}{0.761225in}%
\pgfsys@useobject{currentmarker}{}%
\end{pgfscope}%
\begin{pgfscope}%
\pgfsys@transformshift{1.203653in}{4.536276in}%
\pgfsys@useobject{currentmarker}{}%
\end{pgfscope}%
\begin{pgfscope}%
\pgfsys@transformshift{1.322665in}{2.741489in}%
\pgfsys@useobject{currentmarker}{}%
\end{pgfscope}%
\begin{pgfscope}%
\pgfsys@transformshift{1.582505in}{0.920701in}%
\pgfsys@useobject{currentmarker}{}%
\end{pgfscope}%
\begin{pgfscope}%
\pgfsys@transformshift{1.322153in}{1.007004in}%
\pgfsys@useobject{currentmarker}{}%
\end{pgfscope}%
\begin{pgfscope}%
\pgfsys@transformshift{3.065017in}{0.714262in}%
\pgfsys@useobject{currentmarker}{}%
\end{pgfscope}%
\begin{pgfscope}%
\pgfsys@transformshift{4.262232in}{0.687783in}%
\pgfsys@useobject{currentmarker}{}%
\end{pgfscope}%
\begin{pgfscope}%
\pgfsys@transformshift{2.834176in}{0.728030in}%
\pgfsys@useobject{currentmarker}{}%
\end{pgfscope}%
\begin{pgfscope}%
\pgfsys@transformshift{1.717019in}{0.762235in}%
\pgfsys@useobject{currentmarker}{}%
\end{pgfscope}%
\begin{pgfscope}%
\pgfsys@transformshift{1.728425in}{0.761284in}%
\pgfsys@useobject{currentmarker}{}%
\end{pgfscope}%
\begin{pgfscope}%
\pgfsys@transformshift{1.474418in}{0.762599in}%
\pgfsys@useobject{currentmarker}{}%
\end{pgfscope}%
\begin{pgfscope}%
\pgfsys@transformshift{1.862433in}{0.760516in}%
\pgfsys@useobject{currentmarker}{}%
\end{pgfscope}%
\begin{pgfscope}%
\pgfsys@transformshift{1.919336in}{0.760502in}%
\pgfsys@useobject{currentmarker}{}%
\end{pgfscope}%
\begin{pgfscope}%
\pgfsys@transformshift{1.323983in}{2.025039in}%
\pgfsys@useobject{currentmarker}{}%
\end{pgfscope}%
\begin{pgfscope}%
\pgfsys@transformshift{1.366484in}{1.065376in}%
\pgfsys@useobject{currentmarker}{}%
\end{pgfscope}%
\begin{pgfscope}%
\pgfsys@transformshift{2.884539in}{0.730679in}%
\pgfsys@useobject{currentmarker}{}%
\end{pgfscope}%
\begin{pgfscope}%
\pgfsys@transformshift{0.895562in}{4.733125in}%
\pgfsys@useobject{currentmarker}{}%
\end{pgfscope}%
\begin{pgfscope}%
\pgfsys@transformshift{1.629143in}{0.766369in}%
\pgfsys@useobject{currentmarker}{}%
\end{pgfscope}%
\begin{pgfscope}%
\pgfsys@transformshift{1.559740in}{0.820799in}%
\pgfsys@useobject{currentmarker}{}%
\end{pgfscope}%
\begin{pgfscope}%
\pgfsys@transformshift{4.287578in}{0.695796in}%
\pgfsys@useobject{currentmarker}{}%
\end{pgfscope}%
\begin{pgfscope}%
\pgfsys@transformshift{4.197802in}{0.697882in}%
\pgfsys@useobject{currentmarker}{}%
\end{pgfscope}%
\begin{pgfscope}%
\pgfsys@transformshift{4.188310in}{0.703577in}%
\pgfsys@useobject{currentmarker}{}%
\end{pgfscope}%
\begin{pgfscope}%
\pgfsys@transformshift{1.945682in}{0.760438in}%
\pgfsys@useobject{currentmarker}{}%
\end{pgfscope}%
\begin{pgfscope}%
\pgfsys@transformshift{1.349376in}{2.616030in}%
\pgfsys@useobject{currentmarker}{}%
\end{pgfscope}%
\begin{pgfscope}%
\pgfsys@transformshift{3.019476in}{0.733444in}%
\pgfsys@useobject{currentmarker}{}%
\end{pgfscope}%
\begin{pgfscope}%
\pgfsys@transformshift{2.908250in}{0.733629in}%
\pgfsys@useobject{currentmarker}{}%
\end{pgfscope}%
\begin{pgfscope}%
\pgfsys@transformshift{1.512580in}{0.931349in}%
\pgfsys@useobject{currentmarker}{}%
\end{pgfscope}%
\begin{pgfscope}%
\pgfsys@transformshift{0.986252in}{4.917088in}%
\pgfsys@useobject{currentmarker}{}%
\end{pgfscope}%
\begin{pgfscope}%
\pgfsys@transformshift{1.492423in}{1.033122in}%
\pgfsys@useobject{currentmarker}{}%
\end{pgfscope}%
\begin{pgfscope}%
\pgfsys@transformshift{1.655472in}{0.766668in}%
\pgfsys@useobject{currentmarker}{}%
\end{pgfscope}%
\begin{pgfscope}%
\pgfsys@transformshift{1.886240in}{0.761043in}%
\pgfsys@useobject{currentmarker}{}%
\end{pgfscope}%
\begin{pgfscope}%
\pgfsys@transformshift{4.221668in}{0.701383in}%
\pgfsys@useobject{currentmarker}{}%
\end{pgfscope}%
\begin{pgfscope}%
\pgfsys@transformshift{2.030755in}{0.760460in}%
\pgfsys@useobject{currentmarker}{}%
\end{pgfscope}%
\begin{pgfscope}%
\pgfsys@transformshift{1.786018in}{0.761272in}%
\pgfsys@useobject{currentmarker}{}%
\end{pgfscope}%
\begin{pgfscope}%
\pgfsys@transformshift{4.880326in}{0.678874in}%
\pgfsys@useobject{currentmarker}{}%
\end{pgfscope}%
\begin{pgfscope}%
\pgfsys@transformshift{1.315197in}{1.209442in}%
\pgfsys@useobject{currentmarker}{}%
\end{pgfscope}%
\begin{pgfscope}%
\pgfsys@transformshift{3.139249in}{0.712508in}%
\pgfsys@useobject{currentmarker}{}%
\end{pgfscope}%
\begin{pgfscope}%
\pgfsys@transformshift{2.669024in}{0.729291in}%
\pgfsys@useobject{currentmarker}{}%
\end{pgfscope}%
\begin{pgfscope}%
\pgfsys@transformshift{3.718347in}{0.705834in}%
\pgfsys@useobject{currentmarker}{}%
\end{pgfscope}%
\begin{pgfscope}%
\pgfsys@transformshift{1.770335in}{0.760392in}%
\pgfsys@useobject{currentmarker}{}%
\end{pgfscope}%
\begin{pgfscope}%
\pgfsys@transformshift{1.896120in}{0.759175in}%
\pgfsys@useobject{currentmarker}{}%
\end{pgfscope}%
\begin{pgfscope}%
\pgfsys@transformshift{1.377720in}{0.810630in}%
\pgfsys@useobject{currentmarker}{}%
\end{pgfscope}%
\begin{pgfscope}%
\pgfsys@transformshift{0.915186in}{2.502347in}%
\pgfsys@useobject{currentmarker}{}%
\end{pgfscope}%
\begin{pgfscope}%
\pgfsys@transformshift{2.684618in}{0.727799in}%
\pgfsys@useobject{currentmarker}{}%
\end{pgfscope}%
\begin{pgfscope}%
\pgfsys@transformshift{2.377744in}{0.743336in}%
\pgfsys@useobject{currentmarker}{}%
\end{pgfscope}%
\begin{pgfscope}%
\pgfsys@transformshift{1.381821in}{0.778772in}%
\pgfsys@useobject{currentmarker}{}%
\end{pgfscope}%
\begin{pgfscope}%
\pgfsys@transformshift{2.935419in}{0.727542in}%
\pgfsys@useobject{currentmarker}{}%
\end{pgfscope}%
\begin{pgfscope}%
\pgfsys@transformshift{1.838820in}{0.760401in}%
\pgfsys@useobject{currentmarker}{}%
\end{pgfscope}%
\begin{pgfscope}%
\pgfsys@transformshift{1.797506in}{0.760825in}%
\pgfsys@useobject{currentmarker}{}%
\end{pgfscope}%
\begin{pgfscope}%
\pgfsys@transformshift{0.920749in}{2.667901in}%
\pgfsys@useobject{currentmarker}{}%
\end{pgfscope}%
\begin{pgfscope}%
\pgfsys@transformshift{2.724601in}{0.732946in}%
\pgfsys@useobject{currentmarker}{}%
\end{pgfscope}%
\begin{pgfscope}%
\pgfsys@transformshift{1.176558in}{2.626014in}%
\pgfsys@useobject{currentmarker}{}%
\end{pgfscope}%
\begin{pgfscope}%
\pgfsys@transformshift{1.851383in}{0.760514in}%
\pgfsys@useobject{currentmarker}{}%
\end{pgfscope}%
\begin{pgfscope}%
\pgfsys@transformshift{2.137806in}{0.760407in}%
\pgfsys@useobject{currentmarker}{}%
\end{pgfscope}%
\begin{pgfscope}%
\pgfsys@transformshift{1.950538in}{0.760408in}%
\pgfsys@useobject{currentmarker}{}%
\end{pgfscope}%
\begin{pgfscope}%
\pgfsys@transformshift{2.849150in}{0.729064in}%
\pgfsys@useobject{currentmarker}{}%
\end{pgfscope}%
\begin{pgfscope}%
\pgfsys@transformshift{2.750129in}{0.733223in}%
\pgfsys@useobject{currentmarker}{}%
\end{pgfscope}%
\begin{pgfscope}%
\pgfsys@transformshift{1.409531in}{1.019656in}%
\pgfsys@useobject{currentmarker}{}%
\end{pgfscope}%
\begin{pgfscope}%
\pgfsys@transformshift{2.051306in}{0.760421in}%
\pgfsys@useobject{currentmarker}{}%
\end{pgfscope}%
\begin{pgfscope}%
\pgfsys@transformshift{2.273979in}{0.745756in}%
\pgfsys@useobject{currentmarker}{}%
\end{pgfscope}%
\begin{pgfscope}%
\pgfsys@transformshift{4.221668in}{0.701383in}%
\pgfsys@useobject{currentmarker}{}%
\end{pgfscope}%
\begin{pgfscope}%
\pgfsys@transformshift{1.857752in}{0.760898in}%
\pgfsys@useobject{currentmarker}{}%
\end{pgfscope}%
\begin{pgfscope}%
\pgfsys@transformshift{4.302261in}{0.696497in}%
\pgfsys@useobject{currentmarker}{}%
\end{pgfscope}%
\begin{pgfscope}%
\pgfsys@transformshift{3.163510in}{0.711079in}%
\pgfsys@useobject{currentmarker}{}%
\end{pgfscope}%
\begin{pgfscope}%
\pgfsys@transformshift{1.279415in}{1.307021in}%
\pgfsys@useobject{currentmarker}{}%
\end{pgfscope}%
\begin{pgfscope}%
\pgfsys@transformshift{1.596709in}{0.760365in}%
\pgfsys@useobject{currentmarker}{}%
\end{pgfscope}%
\begin{pgfscope}%
\pgfsys@transformshift{4.040923in}{0.689645in}%
\pgfsys@useobject{currentmarker}{}%
\end{pgfscope}%
\begin{pgfscope}%
\pgfsys@transformshift{3.601069in}{0.698686in}%
\pgfsys@useobject{currentmarker}{}%
\end{pgfscope}%
\begin{pgfscope}%
\pgfsys@transformshift{2.255002in}{0.738569in}%
\pgfsys@useobject{currentmarker}{}%
\end{pgfscope}%
\begin{pgfscope}%
\pgfsys@transformshift{1.315192in}{0.954092in}%
\pgfsys@useobject{currentmarker}{}%
\end{pgfscope}%
\begin{pgfscope}%
\pgfsys@transformshift{0.899971in}{2.890112in}%
\pgfsys@useobject{currentmarker}{}%
\end{pgfscope}%
\begin{pgfscope}%
\pgfsys@transformshift{4.502344in}{0.682867in}%
\pgfsys@useobject{currentmarker}{}%
\end{pgfscope}%
\begin{pgfscope}%
\pgfsys@transformshift{4.894142in}{0.678782in}%
\pgfsys@useobject{currentmarker}{}%
\end{pgfscope}%
\begin{pgfscope}%
\pgfsys@transformshift{4.254004in}{0.686723in}%
\pgfsys@useobject{currentmarker}{}%
\end{pgfscope}%
\begin{pgfscope}%
\pgfsys@transformshift{2.141630in}{0.748290in}%
\pgfsys@useobject{currentmarker}{}%
\end{pgfscope}%
\begin{pgfscope}%
\pgfsys@transformshift{1.223553in}{2.467283in}%
\pgfsys@useobject{currentmarker}{}%
\end{pgfscope}%
\begin{pgfscope}%
\pgfsys@transformshift{1.249254in}{1.991853in}%
\pgfsys@useobject{currentmarker}{}%
\end{pgfscope}%
\begin{pgfscope}%
\pgfsys@transformshift{0.920621in}{2.690734in}%
\pgfsys@useobject{currentmarker}{}%
\end{pgfscope}%
\begin{pgfscope}%
\pgfsys@transformshift{3.194389in}{0.713290in}%
\pgfsys@useobject{currentmarker}{}%
\end{pgfscope}%
\begin{pgfscope}%
\pgfsys@transformshift{1.638671in}{0.760370in}%
\pgfsys@useobject{currentmarker}{}%
\end{pgfscope}%
\begin{pgfscope}%
\pgfsys@transformshift{4.108851in}{0.699079in}%
\pgfsys@useobject{currentmarker}{}%
\end{pgfscope}%
\begin{pgfscope}%
\pgfsys@transformshift{0.977100in}{3.705133in}%
\pgfsys@useobject{currentmarker}{}%
\end{pgfscope}%
\begin{pgfscope}%
\pgfsys@transformshift{2.725023in}{0.735014in}%
\pgfsys@useobject{currentmarker}{}%
\end{pgfscope}%
\begin{pgfscope}%
\pgfsys@transformshift{2.088125in}{0.760365in}%
\pgfsys@useobject{currentmarker}{}%
\end{pgfscope}%
\begin{pgfscope}%
\pgfsys@transformshift{1.867319in}{0.760379in}%
\pgfsys@useobject{currentmarker}{}%
\end{pgfscope}%
\begin{pgfscope}%
\pgfsys@transformshift{1.644804in}{0.761589in}%
\pgfsys@useobject{currentmarker}{}%
\end{pgfscope}%
\begin{pgfscope}%
\pgfsys@transformshift{3.964837in}{0.713240in}%
\pgfsys@useobject{currentmarker}{}%
\end{pgfscope}%
\begin{pgfscope}%
\pgfsys@transformshift{4.121794in}{0.711387in}%
\pgfsys@useobject{currentmarker}{}%
\end{pgfscope}%
\begin{pgfscope}%
\pgfsys@transformshift{5.013182in}{0.682249in}%
\pgfsys@useobject{currentmarker}{}%
\end{pgfscope}%
\begin{pgfscope}%
\pgfsys@transformshift{1.433104in}{0.977339in}%
\pgfsys@useobject{currentmarker}{}%
\end{pgfscope}%
\begin{pgfscope}%
\pgfsys@transformshift{1.525496in}{0.928227in}%
\pgfsys@useobject{currentmarker}{}%
\end{pgfscope}%
\begin{pgfscope}%
\pgfsys@transformshift{1.555305in}{0.817400in}%
\pgfsys@useobject{currentmarker}{}%
\end{pgfscope}%
\begin{pgfscope}%
\pgfsys@transformshift{4.254004in}{0.686723in}%
\pgfsys@useobject{currentmarker}{}%
\end{pgfscope}%
\begin{pgfscope}%
\pgfsys@transformshift{3.937713in}{0.692229in}%
\pgfsys@useobject{currentmarker}{}%
\end{pgfscope}%
\begin{pgfscope}%
\pgfsys@transformshift{3.347584in}{0.709943in}%
\pgfsys@useobject{currentmarker}{}%
\end{pgfscope}%
\begin{pgfscope}%
\pgfsys@transformshift{3.122072in}{0.712849in}%
\pgfsys@useobject{currentmarker}{}%
\end{pgfscope}%
\begin{pgfscope}%
\pgfsys@transformshift{1.140369in}{1.708585in}%
\pgfsys@useobject{currentmarker}{}%
\end{pgfscope}%
\begin{pgfscope}%
\pgfsys@transformshift{1.747971in}{0.759574in}%
\pgfsys@useobject{currentmarker}{}%
\end{pgfscope}%
\begin{pgfscope}%
\pgfsys@transformshift{0.859978in}{3.351553in}%
\pgfsys@useobject{currentmarker}{}%
\end{pgfscope}%
\begin{pgfscope}%
\pgfsys@transformshift{3.199086in}{0.710195in}%
\pgfsys@useobject{currentmarker}{}%
\end{pgfscope}%
\begin{pgfscope}%
\pgfsys@transformshift{2.799509in}{0.722575in}%
\pgfsys@useobject{currentmarker}{}%
\end{pgfscope}%
\begin{pgfscope}%
\pgfsys@transformshift{3.134052in}{0.712059in}%
\pgfsys@useobject{currentmarker}{}%
\end{pgfscope}%
\begin{pgfscope}%
\pgfsys@transformshift{1.197643in}{0.760887in}%
\pgfsys@useobject{currentmarker}{}%
\end{pgfscope}%
\begin{pgfscope}%
\pgfsys@transformshift{2.471424in}{0.734248in}%
\pgfsys@useobject{currentmarker}{}%
\end{pgfscope}%
\begin{pgfscope}%
\pgfsys@transformshift{3.520337in}{0.701725in}%
\pgfsys@useobject{currentmarker}{}%
\end{pgfscope}%
\begin{pgfscope}%
\pgfsys@transformshift{3.086596in}{0.718189in}%
\pgfsys@useobject{currentmarker}{}%
\end{pgfscope}%
\begin{pgfscope}%
\pgfsys@transformshift{1.672893in}{0.760365in}%
\pgfsys@useobject{currentmarker}{}%
\end{pgfscope}%
\begin{pgfscope}%
\pgfsys@transformshift{4.983932in}{0.679499in}%
\pgfsys@useobject{currentmarker}{}%
\end{pgfscope}%
\begin{pgfscope}%
\pgfsys@transformshift{1.265451in}{1.234613in}%
\pgfsys@useobject{currentmarker}{}%
\end{pgfscope}%
\begin{pgfscope}%
\pgfsys@transformshift{1.347037in}{0.769447in}%
\pgfsys@useobject{currentmarker}{}%
\end{pgfscope}%
\begin{pgfscope}%
\pgfsys@transformshift{3.656201in}{0.703709in}%
\pgfsys@useobject{currentmarker}{}%
\end{pgfscope}%
\begin{pgfscope}%
\pgfsys@transformshift{3.662630in}{0.701879in}%
\pgfsys@useobject{currentmarker}{}%
\end{pgfscope}%
\begin{pgfscope}%
\pgfsys@transformshift{0.860233in}{4.546755in}%
\pgfsys@useobject{currentmarker}{}%
\end{pgfscope}%
\begin{pgfscope}%
\pgfsys@transformshift{1.019890in}{3.007344in}%
\pgfsys@useobject{currentmarker}{}%
\end{pgfscope}%
\begin{pgfscope}%
\pgfsys@transformshift{1.697392in}{0.760365in}%
\pgfsys@useobject{currentmarker}{}%
\end{pgfscope}%
\begin{pgfscope}%
\pgfsys@transformshift{1.690915in}{0.760456in}%
\pgfsys@useobject{currentmarker}{}%
\end{pgfscope}%
\begin{pgfscope}%
\pgfsys@transformshift{5.013182in}{0.682249in}%
\pgfsys@useobject{currentmarker}{}%
\end{pgfscope}%
\begin{pgfscope}%
\pgfsys@transformshift{4.170853in}{0.699814in}%
\pgfsys@useobject{currentmarker}{}%
\end{pgfscope}%
\begin{pgfscope}%
\pgfsys@transformshift{1.037336in}{3.718638in}%
\pgfsys@useobject{currentmarker}{}%
\end{pgfscope}%
\begin{pgfscope}%
\pgfsys@transformshift{1.362717in}{1.104240in}%
\pgfsys@useobject{currentmarker}{}%
\end{pgfscope}%
\begin{pgfscope}%
\pgfsys@transformshift{1.298438in}{2.171431in}%
\pgfsys@useobject{currentmarker}{}%
\end{pgfscope}%
\begin{pgfscope}%
\pgfsys@transformshift{0.915087in}{2.507359in}%
\pgfsys@useobject{currentmarker}{}%
\end{pgfscope}%
\begin{pgfscope}%
\pgfsys@transformshift{1.139205in}{0.998436in}%
\pgfsys@useobject{currentmarker}{}%
\end{pgfscope}%
\begin{pgfscope}%
\pgfsys@transformshift{2.669018in}{0.729297in}%
\pgfsys@useobject{currentmarker}{}%
\end{pgfscope}%
\begin{pgfscope}%
\pgfsys@transformshift{3.727447in}{0.695875in}%
\pgfsys@useobject{currentmarker}{}%
\end{pgfscope}%
\begin{pgfscope}%
\pgfsys@transformshift{2.737098in}{0.726935in}%
\pgfsys@useobject{currentmarker}{}%
\end{pgfscope}%
\begin{pgfscope}%
\pgfsys@transformshift{4.759263in}{0.680544in}%
\pgfsys@useobject{currentmarker}{}%
\end{pgfscope}%
\begin{pgfscope}%
\pgfsys@transformshift{4.126067in}{0.689336in}%
\pgfsys@useobject{currentmarker}{}%
\end{pgfscope}%
\begin{pgfscope}%
\pgfsys@transformshift{2.882162in}{0.718089in}%
\pgfsys@useobject{currentmarker}{}%
\end{pgfscope}%
\begin{pgfscope}%
\pgfsys@transformshift{3.133061in}{0.712115in}%
\pgfsys@useobject{currentmarker}{}%
\end{pgfscope}%
\begin{pgfscope}%
\pgfsys@transformshift{3.221206in}{0.708911in}%
\pgfsys@useobject{currentmarker}{}%
\end{pgfscope}%
\begin{pgfscope}%
\pgfsys@transformshift{0.930726in}{1.793232in}%
\pgfsys@useobject{currentmarker}{}%
\end{pgfscope}%
\begin{pgfscope}%
\pgfsys@transformshift{5.446554in}{0.673198in}%
\pgfsys@useobject{currentmarker}{}%
\end{pgfscope}%
\begin{pgfscope}%
\pgfsys@transformshift{3.122072in}{0.712849in}%
\pgfsys@useobject{currentmarker}{}%
\end{pgfscope}%
\begin{pgfscope}%
\pgfsys@transformshift{1.643248in}{0.760365in}%
\pgfsys@useobject{currentmarker}{}%
\end{pgfscope}%
\begin{pgfscope}%
\pgfsys@transformshift{4.515501in}{0.684941in}%
\pgfsys@useobject{currentmarker}{}%
\end{pgfscope}%
\begin{pgfscope}%
\pgfsys@transformshift{4.973784in}{0.679726in}%
\pgfsys@useobject{currentmarker}{}%
\end{pgfscope}%
\begin{pgfscope}%
\pgfsys@transformshift{2.062047in}{0.760323in}%
\pgfsys@useobject{currentmarker}{}%
\end{pgfscope}%
\begin{pgfscope}%
\pgfsys@transformshift{0.871077in}{4.215304in}%
\pgfsys@useobject{currentmarker}{}%
\end{pgfscope}%
\begin{pgfscope}%
\pgfsys@transformshift{0.893687in}{3.800578in}%
\pgfsys@useobject{currentmarker}{}%
\end{pgfscope}%
\begin{pgfscope}%
\pgfsys@transformshift{2.854306in}{0.723106in}%
\pgfsys@useobject{currentmarker}{}%
\end{pgfscope}%
\begin{pgfscope}%
\pgfsys@transformshift{1.519046in}{0.762396in}%
\pgfsys@useobject{currentmarker}{}%
\end{pgfscope}%
\begin{pgfscope}%
\pgfsys@transformshift{0.917319in}{2.588728in}%
\pgfsys@useobject{currentmarker}{}%
\end{pgfscope}%
\begin{pgfscope}%
\pgfsys@transformshift{1.168385in}{1.021684in}%
\pgfsys@useobject{currentmarker}{}%
\end{pgfscope}%
\begin{pgfscope}%
\pgfsys@transformshift{4.106087in}{0.700684in}%
\pgfsys@useobject{currentmarker}{}%
\end{pgfscope}%
\begin{pgfscope}%
\pgfsys@transformshift{2.786525in}{0.732160in}%
\pgfsys@useobject{currentmarker}{}%
\end{pgfscope}%
\begin{pgfscope}%
\pgfsys@transformshift{2.916184in}{0.718127in}%
\pgfsys@useobject{currentmarker}{}%
\end{pgfscope}%
\begin{pgfscope}%
\pgfsys@transformshift{0.935878in}{2.536224in}%
\pgfsys@useobject{currentmarker}{}%
\end{pgfscope}%
\begin{pgfscope}%
\pgfsys@transformshift{2.179218in}{0.760341in}%
\pgfsys@useobject{currentmarker}{}%
\end{pgfscope}%
\begin{pgfscope}%
\pgfsys@transformshift{0.883964in}{4.463218in}%
\pgfsys@useobject{currentmarker}{}%
\end{pgfscope}%
\begin{pgfscope}%
\pgfsys@transformshift{1.646514in}{0.761234in}%
\pgfsys@useobject{currentmarker}{}%
\end{pgfscope}%
\begin{pgfscope}%
\pgfsys@transformshift{2.223726in}{0.741535in}%
\pgfsys@useobject{currentmarker}{}%
\end{pgfscope}%
\begin{pgfscope}%
\pgfsys@transformshift{4.377325in}{0.683892in}%
\pgfsys@useobject{currentmarker}{}%
\end{pgfscope}%
\begin{pgfscope}%
\pgfsys@transformshift{4.903931in}{0.678500in}%
\pgfsys@useobject{currentmarker}{}%
\end{pgfscope}%
\begin{pgfscope}%
\pgfsys@transformshift{0.859907in}{1.935935in}%
\pgfsys@useobject{currentmarker}{}%
\end{pgfscope}%
\begin{pgfscope}%
\pgfsys@transformshift{2.796830in}{0.726185in}%
\pgfsys@useobject{currentmarker}{}%
\end{pgfscope}%
\begin{pgfscope}%
\pgfsys@transformshift{4.126067in}{0.689336in}%
\pgfsys@useobject{currentmarker}{}%
\end{pgfscope}%
\begin{pgfscope}%
\pgfsys@transformshift{3.138439in}{0.710507in}%
\pgfsys@useobject{currentmarker}{}%
\end{pgfscope}%
\begin{pgfscope}%
\pgfsys@transformshift{1.842051in}{0.752182in}%
\pgfsys@useobject{currentmarker}{}%
\end{pgfscope}%
\begin{pgfscope}%
\pgfsys@transformshift{2.562660in}{0.726292in}%
\pgfsys@useobject{currentmarker}{}%
\end{pgfscope}%
\begin{pgfscope}%
\pgfsys@transformshift{2.405826in}{0.734147in}%
\pgfsys@useobject{currentmarker}{}%
\end{pgfscope}%
\begin{pgfscope}%
\pgfsys@transformshift{2.132403in}{0.750165in}%
\pgfsys@useobject{currentmarker}{}%
\end{pgfscope}%
\begin{pgfscope}%
\pgfsys@transformshift{3.743105in}{0.700268in}%
\pgfsys@useobject{currentmarker}{}%
\end{pgfscope}%
\begin{pgfscope}%
\pgfsys@transformshift{3.642594in}{0.709425in}%
\pgfsys@useobject{currentmarker}{}%
\end{pgfscope}%
\begin{pgfscope}%
\pgfsys@transformshift{1.136062in}{1.831548in}%
\pgfsys@useobject{currentmarker}{}%
\end{pgfscope}%
\begin{pgfscope}%
\pgfsys@transformshift{1.636679in}{0.760369in}%
\pgfsys@useobject{currentmarker}{}%
\end{pgfscope}%
\begin{pgfscope}%
\pgfsys@transformshift{1.886998in}{0.755603in}%
\pgfsys@useobject{currentmarker}{}%
\end{pgfscope}%
\begin{pgfscope}%
\pgfsys@transformshift{1.182843in}{1.686948in}%
\pgfsys@useobject{currentmarker}{}%
\end{pgfscope}%
\begin{pgfscope}%
\pgfsys@transformshift{1.545493in}{0.764438in}%
\pgfsys@useobject{currentmarker}{}%
\end{pgfscope}%
\begin{pgfscope}%
\pgfsys@transformshift{2.219147in}{0.740923in}%
\pgfsys@useobject{currentmarker}{}%
\end{pgfscope}%
\begin{pgfscope}%
\pgfsys@transformshift{1.138803in}{0.764032in}%
\pgfsys@useobject{currentmarker}{}%
\end{pgfscope}%
\begin{pgfscope}%
\pgfsys@transformshift{3.326095in}{0.708109in}%
\pgfsys@useobject{currentmarker}{}%
\end{pgfscope}%
\begin{pgfscope}%
\pgfsys@transformshift{1.543908in}{0.760365in}%
\pgfsys@useobject{currentmarker}{}%
\end{pgfscope}%
\begin{pgfscope}%
\pgfsys@transformshift{2.487437in}{0.727715in}%
\pgfsys@useobject{currentmarker}{}%
\end{pgfscope}%
\begin{pgfscope}%
\pgfsys@transformshift{4.708229in}{0.680931in}%
\pgfsys@useobject{currentmarker}{}%
\end{pgfscope}%
\begin{pgfscope}%
\pgfsys@transformshift{2.256172in}{0.738485in}%
\pgfsys@useobject{currentmarker}{}%
\end{pgfscope}%
\begin{pgfscope}%
\pgfsys@transformshift{2.879949in}{0.716034in}%
\pgfsys@useobject{currentmarker}{}%
\end{pgfscope}%
\begin{pgfscope}%
\pgfsys@transformshift{4.285587in}{0.685337in}%
\pgfsys@useobject{currentmarker}{}%
\end{pgfscope}%
\begin{pgfscope}%
\pgfsys@transformshift{2.252254in}{0.740105in}%
\pgfsys@useobject{currentmarker}{}%
\end{pgfscope}%
\begin{pgfscope}%
\pgfsys@transformshift{0.833732in}{2.539080in}%
\pgfsys@useobject{currentmarker}{}%
\end{pgfscope}%
\begin{pgfscope}%
\pgfsys@transformshift{2.132403in}{0.750165in}%
\pgfsys@useobject{currentmarker}{}%
\end{pgfscope}%
\begin{pgfscope}%
\pgfsys@transformshift{3.604200in}{0.700650in}%
\pgfsys@useobject{currentmarker}{}%
\end{pgfscope}%
\begin{pgfscope}%
\pgfsys@transformshift{4.068539in}{0.692512in}%
\pgfsys@useobject{currentmarker}{}%
\end{pgfscope}%
\begin{pgfscope}%
\pgfsys@transformshift{3.559976in}{0.702263in}%
\pgfsys@useobject{currentmarker}{}%
\end{pgfscope}%
\begin{pgfscope}%
\pgfsys@transformshift{5.488746in}{0.676717in}%
\pgfsys@useobject{currentmarker}{}%
\end{pgfscope}%
\begin{pgfscope}%
\pgfsys@transformshift{2.412561in}{0.736240in}%
\pgfsys@useobject{currentmarker}{}%
\end{pgfscope}%
\begin{pgfscope}%
\pgfsys@transformshift{3.033137in}{0.716656in}%
\pgfsys@useobject{currentmarker}{}%
\end{pgfscope}%
\begin{pgfscope}%
\pgfsys@transformshift{0.889312in}{2.699977in}%
\pgfsys@useobject{currentmarker}{}%
\end{pgfscope}%
\begin{pgfscope}%
\pgfsys@transformshift{4.270591in}{0.686501in}%
\pgfsys@useobject{currentmarker}{}%
\end{pgfscope}%
\begin{pgfscope}%
\pgfsys@transformshift{4.759327in}{0.680543in}%
\pgfsys@useobject{currentmarker}{}%
\end{pgfscope}%
\begin{pgfscope}%
\pgfsys@transformshift{3.684201in}{0.696522in}%
\pgfsys@useobject{currentmarker}{}%
\end{pgfscope}%
\begin{pgfscope}%
\pgfsys@transformshift{1.120809in}{0.825085in}%
\pgfsys@useobject{currentmarker}{}%
\end{pgfscope}%
\begin{pgfscope}%
\pgfsys@transformshift{3.556217in}{0.698087in}%
\pgfsys@useobject{currentmarker}{}%
\end{pgfscope}%
\begin{pgfscope}%
\pgfsys@transformshift{1.912896in}{0.751539in}%
\pgfsys@useobject{currentmarker}{}%
\end{pgfscope}%
\begin{pgfscope}%
\pgfsys@transformshift{3.972308in}{0.691110in}%
\pgfsys@useobject{currentmarker}{}%
\end{pgfscope}%
\begin{pgfscope}%
\pgfsys@transformshift{1.178355in}{0.760737in}%
\pgfsys@useobject{currentmarker}{}%
\end{pgfscope}%
\begin{pgfscope}%
\pgfsys@transformshift{2.169104in}{0.742433in}%
\pgfsys@useobject{currentmarker}{}%
\end{pgfscope}%
\begin{pgfscope}%
\pgfsys@transformshift{3.990292in}{0.689307in}%
\pgfsys@useobject{currentmarker}{}%
\end{pgfscope}%
\begin{pgfscope}%
\pgfsys@transformshift{3.490967in}{0.704883in}%
\pgfsys@useobject{currentmarker}{}%
\end{pgfscope}%
\begin{pgfscope}%
\pgfsys@transformshift{3.052096in}{0.713679in}%
\pgfsys@useobject{currentmarker}{}%
\end{pgfscope}%
\begin{pgfscope}%
\pgfsys@transformshift{2.516127in}{0.726800in}%
\pgfsys@useobject{currentmarker}{}%
\end{pgfscope}%
\begin{pgfscope}%
\pgfsys@transformshift{3.326095in}{0.708109in}%
\pgfsys@useobject{currentmarker}{}%
\end{pgfscope}%
\begin{pgfscope}%
\pgfsys@transformshift{2.976458in}{0.714740in}%
\pgfsys@useobject{currentmarker}{}%
\end{pgfscope}%
\begin{pgfscope}%
\pgfsys@transformshift{1.105178in}{1.333261in}%
\pgfsys@useobject{currentmarker}{}%
\end{pgfscope}%
\begin{pgfscope}%
\pgfsys@transformshift{2.177804in}{0.741963in}%
\pgfsys@useobject{currentmarker}{}%
\end{pgfscope}%
\begin{pgfscope}%
\pgfsys@transformshift{1.135419in}{1.827875in}%
\pgfsys@useobject{currentmarker}{}%
\end{pgfscope}%
\begin{pgfscope}%
\pgfsys@transformshift{1.055716in}{2.424695in}%
\pgfsys@useobject{currentmarker}{}%
\end{pgfscope}%
\begin{pgfscope}%
\pgfsys@transformshift{1.061686in}{2.152152in}%
\pgfsys@useobject{currentmarker}{}%
\end{pgfscope}%
\begin{pgfscope}%
\pgfsys@transformshift{0.915186in}{2.502347in}%
\pgfsys@useobject{currentmarker}{}%
\end{pgfscope}%
\begin{pgfscope}%
\pgfsys@transformshift{1.591285in}{0.760365in}%
\pgfsys@useobject{currentmarker}{}%
\end{pgfscope}%
\begin{pgfscope}%
\pgfsys@transformshift{2.134778in}{0.748489in}%
\pgfsys@useobject{currentmarker}{}%
\end{pgfscope}%
\begin{pgfscope}%
\pgfsys@transformshift{2.202349in}{0.742691in}%
\pgfsys@useobject{currentmarker}{}%
\end{pgfscope}%
\begin{pgfscope}%
\pgfsys@transformshift{2.311552in}{0.738867in}%
\pgfsys@useobject{currentmarker}{}%
\end{pgfscope}%
\begin{pgfscope}%
\pgfsys@transformshift{3.989386in}{0.695303in}%
\pgfsys@useobject{currentmarker}{}%
\end{pgfscope}%
\begin{pgfscope}%
\pgfsys@transformshift{3.113620in}{0.711330in}%
\pgfsys@useobject{currentmarker}{}%
\end{pgfscope}%
\begin{pgfscope}%
\pgfsys@transformshift{0.842921in}{1.350641in}%
\pgfsys@useobject{currentmarker}{}%
\end{pgfscope}%
\begin{pgfscope}%
\pgfsys@transformshift{4.895182in}{0.678418in}%
\pgfsys@useobject{currentmarker}{}%
\end{pgfscope}%
\begin{pgfscope}%
\pgfsys@transformshift{2.250901in}{0.738948in}%
\pgfsys@useobject{currentmarker}{}%
\end{pgfscope}%
\begin{pgfscope}%
\pgfsys@transformshift{1.120809in}{0.825085in}%
\pgfsys@useobject{currentmarker}{}%
\end{pgfscope}%
\begin{pgfscope}%
\pgfsys@transformshift{4.891763in}{0.678790in}%
\pgfsys@useobject{currentmarker}{}%
\end{pgfscope}%
\begin{pgfscope}%
\pgfsys@transformshift{2.276518in}{0.738105in}%
\pgfsys@useobject{currentmarker}{}%
\end{pgfscope}%
\begin{pgfscope}%
\pgfsys@transformshift{4.081998in}{0.688611in}%
\pgfsys@useobject{currentmarker}{}%
\end{pgfscope}%
\begin{pgfscope}%
\pgfsys@transformshift{0.771168in}{2.360982in}%
\pgfsys@useobject{currentmarker}{}%
\end{pgfscope}%
\begin{pgfscope}%
\pgfsys@transformshift{3.186899in}{0.708705in}%
\pgfsys@useobject{currentmarker}{}%
\end{pgfscope}%
\begin{pgfscope}%
\pgfsys@transformshift{3.313201in}{0.707122in}%
\pgfsys@useobject{currentmarker}{}%
\end{pgfscope}%
\begin{pgfscope}%
\pgfsys@transformshift{1.105232in}{0.994898in}%
\pgfsys@useobject{currentmarker}{}%
\end{pgfscope}%
\begin{pgfscope}%
\pgfsys@transformshift{3.096474in}{0.711989in}%
\pgfsys@useobject{currentmarker}{}%
\end{pgfscope}%
\begin{pgfscope}%
\pgfsys@transformshift{2.075773in}{0.746894in}%
\pgfsys@useobject{currentmarker}{}%
\end{pgfscope}%
\begin{pgfscope}%
\pgfsys@transformshift{2.826684in}{0.726095in}%
\pgfsys@useobject{currentmarker}{}%
\end{pgfscope}%
\begin{pgfscope}%
\pgfsys@transformshift{2.647336in}{0.729936in}%
\pgfsys@useobject{currentmarker}{}%
\end{pgfscope}%
\begin{pgfscope}%
\pgfsys@transformshift{3.052096in}{0.713679in}%
\pgfsys@useobject{currentmarker}{}%
\end{pgfscope}%
\begin{pgfscope}%
\pgfsys@transformshift{3.025226in}{0.717294in}%
\pgfsys@useobject{currentmarker}{}%
\end{pgfscope}%
\begin{pgfscope}%
\pgfsys@transformshift{2.268250in}{0.739588in}%
\pgfsys@useobject{currentmarker}{}%
\end{pgfscope}%
\begin{pgfscope}%
\pgfsys@transformshift{3.942411in}{0.694768in}%
\pgfsys@useobject{currentmarker}{}%
\end{pgfscope}%
\begin{pgfscope}%
\pgfsys@transformshift{3.578353in}{0.699157in}%
\pgfsys@useobject{currentmarker}{}%
\end{pgfscope}%
\begin{pgfscope}%
\pgfsys@transformshift{3.857829in}{0.697034in}%
\pgfsys@useobject{currentmarker}{}%
\end{pgfscope}%
\begin{pgfscope}%
\pgfsys@transformshift{4.771562in}{0.679766in}%
\pgfsys@useobject{currentmarker}{}%
\end{pgfscope}%
\begin{pgfscope}%
\pgfsys@transformshift{1.743557in}{0.759225in}%
\pgfsys@useobject{currentmarker}{}%
\end{pgfscope}%
\begin{pgfscope}%
\pgfsys@transformshift{3.937909in}{0.692193in}%
\pgfsys@useobject{currentmarker}{}%
\end{pgfscope}%
\begin{pgfscope}%
\pgfsys@transformshift{3.512277in}{0.701194in}%
\pgfsys@useobject{currentmarker}{}%
\end{pgfscope}%
\begin{pgfscope}%
\pgfsys@transformshift{2.814998in}{0.719446in}%
\pgfsys@useobject{currentmarker}{}%
\end{pgfscope}%
\begin{pgfscope}%
\pgfsys@transformshift{2.010821in}{0.740200in}%
\pgfsys@useobject{currentmarker}{}%
\end{pgfscope}%
\begin{pgfscope}%
\pgfsys@transformshift{0.928308in}{0.908323in}%
\pgfsys@useobject{currentmarker}{}%
\end{pgfscope}%
\begin{pgfscope}%
\pgfsys@transformshift{3.361602in}{0.704425in}%
\pgfsys@useobject{currentmarker}{}%
\end{pgfscope}%
\begin{pgfscope}%
\pgfsys@transformshift{3.724340in}{0.695840in}%
\pgfsys@useobject{currentmarker}{}%
\end{pgfscope}%
\begin{pgfscope}%
\pgfsys@transformshift{3.036057in}{0.712769in}%
\pgfsys@useobject{currentmarker}{}%
\end{pgfscope}%
\begin{pgfscope}%
\pgfsys@transformshift{4.229070in}{0.686312in}%
\pgfsys@useobject{currentmarker}{}%
\end{pgfscope}%
\begin{pgfscope}%
\pgfsys@transformshift{1.075531in}{1.074992in}%
\pgfsys@useobject{currentmarker}{}%
\end{pgfscope}%
\begin{pgfscope}%
\pgfsys@transformshift{1.465709in}{0.760654in}%
\pgfsys@useobject{currentmarker}{}%
\end{pgfscope}%
\begin{pgfscope}%
\pgfsys@transformshift{1.196493in}{0.760878in}%
\pgfsys@useobject{currentmarker}{}%
\end{pgfscope}%
\begin{pgfscope}%
\pgfsys@transformshift{2.178453in}{0.741902in}%
\pgfsys@useobject{currentmarker}{}%
\end{pgfscope}%
\begin{pgfscope}%
\pgfsys@transformshift{2.057231in}{0.745712in}%
\pgfsys@useobject{currentmarker}{}%
\end{pgfscope}%
\begin{pgfscope}%
\pgfsys@transformshift{0.777953in}{2.379642in}%
\pgfsys@useobject{currentmarker}{}%
\end{pgfscope}%
\begin{pgfscope}%
\pgfsys@transformshift{1.912896in}{0.751539in}%
\pgfsys@useobject{currentmarker}{}%
\end{pgfscope}%
\begin{pgfscope}%
\pgfsys@transformshift{2.487557in}{0.727725in}%
\pgfsys@useobject{currentmarker}{}%
\end{pgfscope}%
\begin{pgfscope}%
\pgfsys@transformshift{2.258183in}{0.738751in}%
\pgfsys@useobject{currentmarker}{}%
\end{pgfscope}%
\begin{pgfscope}%
\pgfsys@transformshift{3.725407in}{0.696473in}%
\pgfsys@useobject{currentmarker}{}%
\end{pgfscope}%
\begin{pgfscope}%
\pgfsys@transformshift{3.134053in}{0.712058in}%
\pgfsys@useobject{currentmarker}{}%
\end{pgfscope}%
\begin{pgfscope}%
\pgfsys@transformshift{2.464724in}{0.729239in}%
\pgfsys@useobject{currentmarker}{}%
\end{pgfscope}%
\begin{pgfscope}%
\pgfsys@transformshift{5.309409in}{0.676533in}%
\pgfsys@useobject{currentmarker}{}%
\end{pgfscope}%
\begin{pgfscope}%
\pgfsys@transformshift{3.724340in}{0.695840in}%
\pgfsys@useobject{currentmarker}{}%
\end{pgfscope}%
\begin{pgfscope}%
\pgfsys@transformshift{1.509516in}{0.760365in}%
\pgfsys@useobject{currentmarker}{}%
\end{pgfscope}%
\begin{pgfscope}%
\pgfsys@transformshift{3.685465in}{0.696137in}%
\pgfsys@useobject{currentmarker}{}%
\end{pgfscope}%
\begin{pgfscope}%
\pgfsys@transformshift{3.289954in}{0.707382in}%
\pgfsys@useobject{currentmarker}{}%
\end{pgfscope}%
\begin{pgfscope}%
\pgfsys@transformshift{2.471994in}{0.721787in}%
\pgfsys@useobject{currentmarker}{}%
\end{pgfscope}%
\begin{pgfscope}%
\pgfsys@transformshift{1.195710in}{0.764490in}%
\pgfsys@useobject{currentmarker}{}%
\end{pgfscope}%
\begin{pgfscope}%
\pgfsys@transformshift{2.525123in}{0.726495in}%
\pgfsys@useobject{currentmarker}{}%
\end{pgfscope}%
\begin{pgfscope}%
\pgfsys@transformshift{1.965773in}{0.743067in}%
\pgfsys@useobject{currentmarker}{}%
\end{pgfscope}%
\begin{pgfscope}%
\pgfsys@transformshift{1.040670in}{1.071789in}%
\pgfsys@useobject{currentmarker}{}%
\end{pgfscope}%
\begin{pgfscope}%
\pgfsys@transformshift{3.442008in}{0.707315in}%
\pgfsys@useobject{currentmarker}{}%
\end{pgfscope}%
\begin{pgfscope}%
\pgfsys@transformshift{3.163926in}{0.711474in}%
\pgfsys@useobject{currentmarker}{}%
\end{pgfscope}%
\begin{pgfscope}%
\pgfsys@transformshift{3.134053in}{0.712058in}%
\pgfsys@useobject{currentmarker}{}%
\end{pgfscope}%
\begin{pgfscope}%
\pgfsys@transformshift{4.251185in}{0.695811in}%
\pgfsys@useobject{currentmarker}{}%
\end{pgfscope}%
\begin{pgfscope}%
\pgfsys@transformshift{1.372270in}{0.758133in}%
\pgfsys@useobject{currentmarker}{}%
\end{pgfscope}%
\begin{pgfscope}%
\pgfsys@transformshift{0.814313in}{2.295710in}%
\pgfsys@useobject{currentmarker}{}%
\end{pgfscope}%
\begin{pgfscope}%
\pgfsys@transformshift{1.998497in}{0.740627in}%
\pgfsys@useobject{currentmarker}{}%
\end{pgfscope}%
\begin{pgfscope}%
\pgfsys@transformshift{1.440390in}{0.757261in}%
\pgfsys@useobject{currentmarker}{}%
\end{pgfscope}%
\begin{pgfscope}%
\pgfsys@transformshift{5.282029in}{0.676454in}%
\pgfsys@useobject{currentmarker}{}%
\end{pgfscope}%
\begin{pgfscope}%
\pgfsys@transformshift{5.399390in}{0.673534in}%
\pgfsys@useobject{currentmarker}{}%
\end{pgfscope}%
\begin{pgfscope}%
\pgfsys@transformshift{1.188952in}{0.760521in}%
\pgfsys@useobject{currentmarker}{}%
\end{pgfscope}%
\begin{pgfscope}%
\pgfsys@transformshift{4.210720in}{0.685943in}%
\pgfsys@useobject{currentmarker}{}%
\end{pgfscope}%
\begin{pgfscope}%
\pgfsys@transformshift{3.436680in}{0.699423in}%
\pgfsys@useobject{currentmarker}{}%
\end{pgfscope}%
\begin{pgfscope}%
\pgfsys@transformshift{4.632477in}{0.681580in}%
\pgfsys@useobject{currentmarker}{}%
\end{pgfscope}%
\begin{pgfscope}%
\pgfsys@transformshift{1.683153in}{0.758395in}%
\pgfsys@useobject{currentmarker}{}%
\end{pgfscope}%
\begin{pgfscope}%
\pgfsys@transformshift{2.155861in}{0.742902in}%
\pgfsys@useobject{currentmarker}{}%
\end{pgfscope}%
\begin{pgfscope}%
\pgfsys@transformshift{2.503843in}{0.722619in}%
\pgfsys@useobject{currentmarker}{}%
\end{pgfscope}%
\begin{pgfscope}%
\pgfsys@transformshift{3.253656in}{0.710746in}%
\pgfsys@useobject{currentmarker}{}%
\end{pgfscope}%
\begin{pgfscope}%
\pgfsys@transformshift{1.381448in}{0.756360in}%
\pgfsys@useobject{currentmarker}{}%
\end{pgfscope}%
\begin{pgfscope}%
\pgfsys@transformshift{3.358572in}{0.705335in}%
\pgfsys@useobject{currentmarker}{}%
\end{pgfscope}%
\begin{pgfscope}%
\pgfsys@transformshift{4.387244in}{0.683535in}%
\pgfsys@useobject{currentmarker}{}%
\end{pgfscope}%
\begin{pgfscope}%
\pgfsys@transformshift{2.095839in}{0.738050in}%
\pgfsys@useobject{currentmarker}{}%
\end{pgfscope}%
\begin{pgfscope}%
\pgfsys@transformshift{5.179582in}{0.674428in}%
\pgfsys@useobject{currentmarker}{}%
\end{pgfscope}%
\begin{pgfscope}%
\pgfsys@transformshift{2.881181in}{0.713104in}%
\pgfsys@useobject{currentmarker}{}%
\end{pgfscope}%
\begin{pgfscope}%
\pgfsys@transformshift{4.501673in}{0.682880in}%
\pgfsys@useobject{currentmarker}{}%
\end{pgfscope}%
\begin{pgfscope}%
\pgfsys@transformshift{1.372270in}{0.758133in}%
\pgfsys@useobject{currentmarker}{}%
\end{pgfscope}%
\begin{pgfscope}%
\pgfsys@transformshift{0.771089in}{2.337889in}%
\pgfsys@useobject{currentmarker}{}%
\end{pgfscope}%
\begin{pgfscope}%
\pgfsys@transformshift{1.129633in}{0.762169in}%
\pgfsys@useobject{currentmarker}{}%
\end{pgfscope}%
\begin{pgfscope}%
\pgfsys@transformshift{3.197520in}{0.708685in}%
\pgfsys@useobject{currentmarker}{}%
\end{pgfscope}%
\begin{pgfscope}%
\pgfsys@transformshift{0.886280in}{0.952894in}%
\pgfsys@useobject{currentmarker}{}%
\end{pgfscope}%
\begin{pgfscope}%
\pgfsys@transformshift{0.788167in}{1.760259in}%
\pgfsys@useobject{currentmarker}{}%
\end{pgfscope}%
\begin{pgfscope}%
\pgfsys@transformshift{0.979912in}{1.125513in}%
\pgfsys@useobject{currentmarker}{}%
\end{pgfscope}%
\begin{pgfscope}%
\pgfsys@transformshift{1.899997in}{0.753475in}%
\pgfsys@useobject{currentmarker}{}%
\end{pgfscope}%
\begin{pgfscope}%
\pgfsys@transformshift{2.503843in}{0.722619in}%
\pgfsys@useobject{currentmarker}{}%
\end{pgfscope}%
\begin{pgfscope}%
\pgfsys@transformshift{1.191039in}{0.760884in}%
\pgfsys@useobject{currentmarker}{}%
\end{pgfscope}%
\begin{pgfscope}%
\pgfsys@transformshift{4.075152in}{0.688901in}%
\pgfsys@useobject{currentmarker}{}%
\end{pgfscope}%
\begin{pgfscope}%
\pgfsys@transformshift{3.817849in}{0.695246in}%
\pgfsys@useobject{currentmarker}{}%
\end{pgfscope}%
\begin{pgfscope}%
\pgfsys@transformshift{1.430673in}{0.755190in}%
\pgfsys@useobject{currentmarker}{}%
\end{pgfscope}%
\begin{pgfscope}%
\pgfsys@transformshift{0.771089in}{2.337889in}%
\pgfsys@useobject{currentmarker}{}%
\end{pgfscope}%
\begin{pgfscope}%
\pgfsys@transformshift{3.951026in}{0.691346in}%
\pgfsys@useobject{currentmarker}{}%
\end{pgfscope}%
\begin{pgfscope}%
\pgfsys@transformshift{4.632477in}{0.681580in}%
\pgfsys@useobject{currentmarker}{}%
\end{pgfscope}%
\begin{pgfscope}%
\pgfsys@transformshift{1.119491in}{0.867587in}%
\pgfsys@useobject{currentmarker}{}%
\end{pgfscope}%
\begin{pgfscope}%
\pgfsys@transformshift{4.864926in}{0.679393in}%
\pgfsys@useobject{currentmarker}{}%
\end{pgfscope}%
\begin{pgfscope}%
\pgfsys@transformshift{2.083820in}{0.739906in}%
\pgfsys@useobject{currentmarker}{}%
\end{pgfscope}%
\begin{pgfscope}%
\pgfsys@transformshift{1.146250in}{0.760697in}%
\pgfsys@useobject{currentmarker}{}%
\end{pgfscope}%
\begin{pgfscope}%
\pgfsys@transformshift{3.358572in}{0.705335in}%
\pgfsys@useobject{currentmarker}{}%
\end{pgfscope}%
\begin{pgfscope}%
\pgfsys@transformshift{3.186899in}{0.708705in}%
\pgfsys@useobject{currentmarker}{}%
\end{pgfscope}%
\begin{pgfscope}%
\pgfsys@transformshift{0.963545in}{1.051348in}%
\pgfsys@useobject{currentmarker}{}%
\end{pgfscope}%
\begin{pgfscope}%
\pgfsys@transformshift{1.897369in}{0.753938in}%
\pgfsys@useobject{currentmarker}{}%
\end{pgfscope}%
\begin{pgfscope}%
\pgfsys@transformshift{3.008571in}{0.721199in}%
\pgfsys@useobject{currentmarker}{}%
\end{pgfscope}%
\begin{pgfscope}%
\pgfsys@transformshift{3.978411in}{0.690945in}%
\pgfsys@useobject{currentmarker}{}%
\end{pgfscope}%
\begin{pgfscope}%
\pgfsys@transformshift{1.146250in}{0.760697in}%
\pgfsys@useobject{currentmarker}{}%
\end{pgfscope}%
\begin{pgfscope}%
\pgfsys@transformshift{2.830161in}{0.718368in}%
\pgfsys@useobject{currentmarker}{}%
\end{pgfscope}%
\begin{pgfscope}%
\pgfsys@transformshift{1.372256in}{0.758131in}%
\pgfsys@useobject{currentmarker}{}%
\end{pgfscope}%
\begin{pgfscope}%
\pgfsys@transformshift{2.341631in}{0.734401in}%
\pgfsys@useobject{currentmarker}{}%
\end{pgfscope}%
\begin{pgfscope}%
\pgfsys@transformshift{2.351490in}{0.726265in}%
\pgfsys@useobject{currentmarker}{}%
\end{pgfscope}%
\begin{pgfscope}%
\pgfsys@transformshift{2.200099in}{0.735003in}%
\pgfsys@useobject{currentmarker}{}%
\end{pgfscope}%
\begin{pgfscope}%
\pgfsys@transformshift{0.821138in}{1.610371in}%
\pgfsys@useobject{currentmarker}{}%
\end{pgfscope}%
\begin{pgfscope}%
\pgfsys@transformshift{1.126935in}{0.793771in}%
\pgfsys@useobject{currentmarker}{}%
\end{pgfscope}%
\begin{pgfscope}%
\pgfsys@transformshift{1.819412in}{0.755140in}%
\pgfsys@useobject{currentmarker}{}%
\end{pgfscope}%
\begin{pgfscope}%
\pgfsys@transformshift{2.830161in}{0.718368in}%
\pgfsys@useobject{currentmarker}{}%
\end{pgfscope}%
\begin{pgfscope}%
\pgfsys@transformshift{0.827060in}{1.312739in}%
\pgfsys@useobject{currentmarker}{}%
\end{pgfscope}%
\begin{pgfscope}%
\pgfsys@transformshift{1.902775in}{0.747794in}%
\pgfsys@useobject{currentmarker}{}%
\end{pgfscope}%
\begin{pgfscope}%
\pgfsys@transformshift{3.836730in}{0.694154in}%
\pgfsys@useobject{currentmarker}{}%
\end{pgfscope}%
\begin{pgfscope}%
\pgfsys@transformshift{0.730075in}{4.466539in}%
\pgfsys@useobject{currentmarker}{}%
\end{pgfscope}%
\begin{pgfscope}%
\pgfsys@transformshift{2.317795in}{0.730772in}%
\pgfsys@useobject{currentmarker}{}%
\end{pgfscope}%
\begin{pgfscope}%
\pgfsys@transformshift{4.127106in}{0.687914in}%
\pgfsys@useobject{currentmarker}{}%
\end{pgfscope}%
\begin{pgfscope}%
\pgfsys@transformshift{4.305784in}{0.684525in}%
\pgfsys@useobject{currentmarker}{}%
\end{pgfscope}%
\begin{pgfscope}%
\pgfsys@transformshift{4.406412in}{0.683297in}%
\pgfsys@useobject{currentmarker}{}%
\end{pgfscope}%
\begin{pgfscope}%
\pgfsys@transformshift{1.029352in}{0.986899in}%
\pgfsys@useobject{currentmarker}{}%
\end{pgfscope}%
\begin{pgfscope}%
\pgfsys@transformshift{1.372270in}{0.758133in}%
\pgfsys@useobject{currentmarker}{}%
\end{pgfscope}%
\begin{pgfscope}%
\pgfsys@transformshift{1.471674in}{0.753220in}%
\pgfsys@useobject{currentmarker}{}%
\end{pgfscope}%
\begin{pgfscope}%
\pgfsys@transformshift{1.122272in}{0.760790in}%
\pgfsys@useobject{currentmarker}{}%
\end{pgfscope}%
\begin{pgfscope}%
\pgfsys@transformshift{5.280664in}{0.674268in}%
\pgfsys@useobject{currentmarker}{}%
\end{pgfscope}%
\begin{pgfscope}%
\pgfsys@transformshift{1.050681in}{0.782877in}%
\pgfsys@useobject{currentmarker}{}%
\end{pgfscope}%
\begin{pgfscope}%
\pgfsys@transformshift{0.882145in}{1.044486in}%
\pgfsys@useobject{currentmarker}{}%
\end{pgfscope}%
\begin{pgfscope}%
\pgfsys@transformshift{3.817849in}{0.695246in}%
\pgfsys@useobject{currentmarker}{}%
\end{pgfscope}%
\begin{pgfscope}%
\pgfsys@transformshift{0.879112in}{1.071879in}%
\pgfsys@useobject{currentmarker}{}%
\end{pgfscope}%
\begin{pgfscope}%
\pgfsys@transformshift{1.779853in}{0.744154in}%
\pgfsys@useobject{currentmarker}{}%
\end{pgfscope}%
\begin{pgfscope}%
\pgfsys@transformshift{3.913035in}{0.693553in}%
\pgfsys@useobject{currentmarker}{}%
\end{pgfscope}%
\begin{pgfscope}%
\pgfsys@transformshift{3.186899in}{0.708705in}%
\pgfsys@useobject{currentmarker}{}%
\end{pgfscope}%
\begin{pgfscope}%
\pgfsys@transformshift{3.045610in}{0.709972in}%
\pgfsys@useobject{currentmarker}{}%
\end{pgfscope}%
\begin{pgfscope}%
\pgfsys@transformshift{1.262185in}{0.760445in}%
\pgfsys@useobject{currentmarker}{}%
\end{pgfscope}%
\begin{pgfscope}%
\pgfsys@transformshift{1.432247in}{0.753095in}%
\pgfsys@useobject{currentmarker}{}%
\end{pgfscope}%
\begin{pgfscope}%
\pgfsys@transformshift{1.126078in}{0.806133in}%
\pgfsys@useobject{currentmarker}{}%
\end{pgfscope}%
\begin{pgfscope}%
\pgfsys@transformshift{0.760059in}{1.435212in}%
\pgfsys@useobject{currentmarker}{}%
\end{pgfscope}%
\begin{pgfscope}%
\pgfsys@transformshift{3.358336in}{0.705412in}%
\pgfsys@useobject{currentmarker}{}%
\end{pgfscope}%
\begin{pgfscope}%
\pgfsys@transformshift{0.739757in}{2.476260in}%
\pgfsys@useobject{currentmarker}{}%
\end{pgfscope}%
\begin{pgfscope}%
\pgfsys@transformshift{0.936031in}{0.906638in}%
\pgfsys@useobject{currentmarker}{}%
\end{pgfscope}%
\begin{pgfscope}%
\pgfsys@transformshift{3.825589in}{0.694054in}%
\pgfsys@useobject{currentmarker}{}%
\end{pgfscope}%
\begin{pgfscope}%
\pgfsys@transformshift{4.797491in}{0.678682in}%
\pgfsys@useobject{currentmarker}{}%
\end{pgfscope}%
\begin{pgfscope}%
\pgfsys@transformshift{4.127106in}{0.687914in}%
\pgfsys@useobject{currentmarker}{}%
\end{pgfscope}%
\begin{pgfscope}%
\pgfsys@transformshift{0.827060in}{1.312739in}%
\pgfsys@useobject{currentmarker}{}%
\end{pgfscope}%
\begin{pgfscope}%
\pgfsys@transformshift{1.373750in}{0.754998in}%
\pgfsys@useobject{currentmarker}{}%
\end{pgfscope}%
\begin{pgfscope}%
\pgfsys@transformshift{3.332423in}{0.707000in}%
\pgfsys@useobject{currentmarker}{}%
\end{pgfscope}%
\begin{pgfscope}%
\pgfsys@transformshift{4.047672in}{0.689181in}%
\pgfsys@useobject{currentmarker}{}%
\end{pgfscope}%
\begin{pgfscope}%
\pgfsys@transformshift{3.938405in}{0.691049in}%
\pgfsys@useobject{currentmarker}{}%
\end{pgfscope}%
\begin{pgfscope}%
\pgfsys@transformshift{0.996072in}{0.919191in}%
\pgfsys@useobject{currentmarker}{}%
\end{pgfscope}%
\begin{pgfscope}%
\pgfsys@transformshift{4.831787in}{0.678580in}%
\pgfsys@useobject{currentmarker}{}%
\end{pgfscope}%
\begin{pgfscope}%
\pgfsys@transformshift{5.119000in}{0.675194in}%
\pgfsys@useobject{currentmarker}{}%
\end{pgfscope}%
\begin{pgfscope}%
\pgfsys@transformshift{0.749194in}{2.052763in}%
\pgfsys@useobject{currentmarker}{}%
\end{pgfscope}%
\begin{pgfscope}%
\pgfsys@transformshift{3.825589in}{0.694054in}%
\pgfsys@useobject{currentmarker}{}%
\end{pgfscope}%
\begin{pgfscope}%
\pgfsys@transformshift{4.049966in}{0.689133in}%
\pgfsys@useobject{currentmarker}{}%
\end{pgfscope}%
\begin{pgfscope}%
\pgfsys@transformshift{4.612516in}{0.681445in}%
\pgfsys@useobject{currentmarker}{}%
\end{pgfscope}%
\begin{pgfscope}%
\pgfsys@transformshift{3.936717in}{0.691189in}%
\pgfsys@useobject{currentmarker}{}%
\end{pgfscope}%
\begin{pgfscope}%
\pgfsys@transformshift{4.892716in}{0.678467in}%
\pgfsys@useobject{currentmarker}{}%
\end{pgfscope}%
\begin{pgfscope}%
\pgfsys@transformshift{2.372108in}{0.725674in}%
\pgfsys@useobject{currentmarker}{}%
\end{pgfscope}%
\begin{pgfscope}%
\pgfsys@transformshift{2.855153in}{0.717526in}%
\pgfsys@useobject{currentmarker}{}%
\end{pgfscope}%
\begin{pgfscope}%
\pgfsys@transformshift{3.677116in}{0.698076in}%
\pgfsys@useobject{currentmarker}{}%
\end{pgfscope}%
\begin{pgfscope}%
\pgfsys@transformshift{1.373750in}{0.754998in}%
\pgfsys@useobject{currentmarker}{}%
\end{pgfscope}%
\begin{pgfscope}%
\pgfsys@transformshift{0.732997in}{1.881291in}%
\pgfsys@useobject{currentmarker}{}%
\end{pgfscope}%
\begin{pgfscope}%
\pgfsys@transformshift{4.996933in}{0.677123in}%
\pgfsys@useobject{currentmarker}{}%
\end{pgfscope}%
\begin{pgfscope}%
\pgfsys@transformshift{3.802976in}{0.694822in}%
\pgfsys@useobject{currentmarker}{}%
\end{pgfscope}%
\begin{pgfscope}%
\pgfsys@transformshift{0.723046in}{4.433103in}%
\pgfsys@useobject{currentmarker}{}%
\end{pgfscope}%
\begin{pgfscope}%
\pgfsys@transformshift{1.981638in}{0.739805in}%
\pgfsys@useobject{currentmarker}{}%
\end{pgfscope}%
\begin{pgfscope}%
\pgfsys@transformshift{3.938405in}{0.691049in}%
\pgfsys@useobject{currentmarker}{}%
\end{pgfscope}%
\begin{pgfscope}%
\pgfsys@transformshift{1.295722in}{0.770137in}%
\pgfsys@useobject{currentmarker}{}%
\end{pgfscope}%
\begin{pgfscope}%
\pgfsys@transformshift{1.824146in}{0.748042in}%
\pgfsys@useobject{currentmarker}{}%
\end{pgfscope}%
\begin{pgfscope}%
\pgfsys@transformshift{2.642292in}{0.715544in}%
\pgfsys@useobject{currentmarker}{}%
\end{pgfscope}%
\begin{pgfscope}%
\pgfsys@transformshift{3.382797in}{0.704340in}%
\pgfsys@useobject{currentmarker}{}%
\end{pgfscope}%
\begin{pgfscope}%
\pgfsys@transformshift{3.431585in}{0.702553in}%
\pgfsys@useobject{currentmarker}{}%
\end{pgfscope}%
\begin{pgfscope}%
\pgfsys@transformshift{4.263302in}{0.685111in}%
\pgfsys@useobject{currentmarker}{}%
\end{pgfscope}%
\begin{pgfscope}%
\pgfsys@transformshift{1.373160in}{0.755067in}%
\pgfsys@useobject{currentmarker}{}%
\end{pgfscope}%
\begin{pgfscope}%
\pgfsys@transformshift{3.414019in}{0.704107in}%
\pgfsys@useobject{currentmarker}{}%
\end{pgfscope}%
\begin{pgfscope}%
\pgfsys@transformshift{0.721543in}{2.335608in}%
\pgfsys@useobject{currentmarker}{}%
\end{pgfscope}%
\begin{pgfscope}%
\pgfsys@transformshift{2.900952in}{0.710826in}%
\pgfsys@useobject{currentmarker}{}%
\end{pgfscope}%
\begin{pgfscope}%
\pgfsys@transformshift{2.354657in}{0.725828in}%
\pgfsys@useobject{currentmarker}{}%
\end{pgfscope}%
\begin{pgfscope}%
\pgfsys@transformshift{5.361313in}{0.673198in}%
\pgfsys@useobject{currentmarker}{}%
\end{pgfscope}%
\begin{pgfscope}%
\pgfsys@transformshift{1.779853in}{0.744154in}%
\pgfsys@useobject{currentmarker}{}%
\end{pgfscope}%
\begin{pgfscope}%
\pgfsys@transformshift{3.668907in}{0.696774in}%
\pgfsys@useobject{currentmarker}{}%
\end{pgfscope}%
\begin{pgfscope}%
\pgfsys@transformshift{1.980309in}{0.739838in}%
\pgfsys@useobject{currentmarker}{}%
\end{pgfscope}%
\begin{pgfscope}%
\pgfsys@transformshift{1.432247in}{0.753095in}%
\pgfsys@useobject{currentmarker}{}%
\end{pgfscope}%
\begin{pgfscope}%
\pgfsys@transformshift{0.873958in}{1.529654in}%
\pgfsys@useobject{currentmarker}{}%
\end{pgfscope}%
\begin{pgfscope}%
\pgfsys@transformshift{2.727274in}{0.728282in}%
\pgfsys@useobject{currentmarker}{}%
\end{pgfscope}%
\begin{pgfscope}%
\pgfsys@transformshift{1.283410in}{0.757460in}%
\pgfsys@useobject{currentmarker}{}%
\end{pgfscope}%
\begin{pgfscope}%
\pgfsys@transformshift{1.119344in}{0.769220in}%
\pgfsys@useobject{currentmarker}{}%
\end{pgfscope}%
\begin{pgfscope}%
\pgfsys@transformshift{1.382578in}{0.754118in}%
\pgfsys@useobject{currentmarker}{}%
\end{pgfscope}%
\begin{pgfscope}%
\pgfsys@transformshift{0.826230in}{1.166889in}%
\pgfsys@useobject{currentmarker}{}%
\end{pgfscope}%
\begin{pgfscope}%
\pgfsys@transformshift{3.257339in}{0.706534in}%
\pgfsys@useobject{currentmarker}{}%
\end{pgfscope}%
\begin{pgfscope}%
\pgfsys@transformshift{0.812485in}{1.389917in}%
\pgfsys@useobject{currentmarker}{}%
\end{pgfscope}%
\begin{pgfscope}%
\pgfsys@transformshift{0.760059in}{1.435212in}%
\pgfsys@useobject{currentmarker}{}%
\end{pgfscope}%
\begin{pgfscope}%
\pgfsys@transformshift{3.428164in}{0.699029in}%
\pgfsys@useobject{currentmarker}{}%
\end{pgfscope}%
\begin{pgfscope}%
\pgfsys@transformshift{0.752123in}{2.101971in}%
\pgfsys@useobject{currentmarker}{}%
\end{pgfscope}%
\begin{pgfscope}%
\pgfsys@transformshift{0.978794in}{1.180277in}%
\pgfsys@useobject{currentmarker}{}%
\end{pgfscope}%
\begin{pgfscope}%
\pgfsys@transformshift{2.095788in}{0.738065in}%
\pgfsys@useobject{currentmarker}{}%
\end{pgfscope}%
\begin{pgfscope}%
\pgfsys@transformshift{2.053359in}{0.738550in}%
\pgfsys@useobject{currentmarker}{}%
\end{pgfscope}%
\begin{pgfscope}%
\pgfsys@transformshift{2.716671in}{0.727451in}%
\pgfsys@useobject{currentmarker}{}%
\end{pgfscope}%
\begin{pgfscope}%
\pgfsys@transformshift{3.426509in}{0.701056in}%
\pgfsys@useobject{currentmarker}{}%
\end{pgfscope}%
\begin{pgfscope}%
\pgfsys@transformshift{3.259529in}{0.705792in}%
\pgfsys@useobject{currentmarker}{}%
\end{pgfscope}%
\begin{pgfscope}%
\pgfsys@transformshift{2.339813in}{0.728718in}%
\pgfsys@useobject{currentmarker}{}%
\end{pgfscope}%
\begin{pgfscope}%
\pgfsys@transformshift{1.328129in}{0.755826in}%
\pgfsys@useobject{currentmarker}{}%
\end{pgfscope}%
\begin{pgfscope}%
\pgfsys@transformshift{3.673598in}{0.696330in}%
\pgfsys@useobject{currentmarker}{}%
\end{pgfscope}%
\begin{pgfscope}%
\pgfsys@transformshift{5.361313in}{0.673198in}%
\pgfsys@useobject{currentmarker}{}%
\end{pgfscope}%
\begin{pgfscope}%
\pgfsys@transformshift{3.428164in}{0.699029in}%
\pgfsys@useobject{currentmarker}{}%
\end{pgfscope}%
\begin{pgfscope}%
\pgfsys@transformshift{0.805773in}{1.133025in}%
\pgfsys@useobject{currentmarker}{}%
\end{pgfscope}%
\begin{pgfscope}%
\pgfsys@transformshift{4.867356in}{0.678257in}%
\pgfsys@useobject{currentmarker}{}%
\end{pgfscope}%
\begin{pgfscope}%
\pgfsys@transformshift{4.121542in}{0.688308in}%
\pgfsys@useobject{currentmarker}{}%
\end{pgfscope}%
\begin{pgfscope}%
\pgfsys@transformshift{4.329088in}{0.684166in}%
\pgfsys@useobject{currentmarker}{}%
\end{pgfscope}%
\begin{pgfscope}%
\pgfsys@transformshift{1.258342in}{0.763373in}%
\pgfsys@useobject{currentmarker}{}%
\end{pgfscope}%
\begin{pgfscope}%
\pgfsys@transformshift{2.070930in}{0.734183in}%
\pgfsys@useobject{currentmarker}{}%
\end{pgfscope}%
\begin{pgfscope}%
\pgfsys@transformshift{1.362869in}{0.756367in}%
\pgfsys@useobject{currentmarker}{}%
\end{pgfscope}%
\begin{pgfscope}%
\pgfsys@transformshift{1.095131in}{0.801595in}%
\pgfsys@useobject{currentmarker}{}%
\end{pgfscope}%
\begin{pgfscope}%
\pgfsys@transformshift{1.092804in}{0.890399in}%
\pgfsys@useobject{currentmarker}{}%
\end{pgfscope}%
\begin{pgfscope}%
\pgfsys@transformshift{3.414966in}{0.702935in}%
\pgfsys@useobject{currentmarker}{}%
\end{pgfscope}%
\begin{pgfscope}%
\pgfsys@transformshift{2.979582in}{0.707638in}%
\pgfsys@useobject{currentmarker}{}%
\end{pgfscope}%
\begin{pgfscope}%
\pgfsys@transformshift{1.333570in}{0.755420in}%
\pgfsys@useobject{currentmarker}{}%
\end{pgfscope}%
\begin{pgfscope}%
\pgfsys@transformshift{3.604949in}{0.696951in}%
\pgfsys@useobject{currentmarker}{}%
\end{pgfscope}%
\begin{pgfscope}%
\pgfsys@transformshift{1.312416in}{0.757186in}%
\pgfsys@useobject{currentmarker}{}%
\end{pgfscope}%
\begin{pgfscope}%
\pgfsys@transformshift{2.324472in}{0.729307in}%
\pgfsys@useobject{currentmarker}{}%
\end{pgfscope}%
\begin{pgfscope}%
\pgfsys@transformshift{3.668513in}{0.696457in}%
\pgfsys@useobject{currentmarker}{}%
\end{pgfscope}%
\begin{pgfscope}%
\pgfsys@transformshift{2.869189in}{0.713321in}%
\pgfsys@useobject{currentmarker}{}%
\end{pgfscope}%
\begin{pgfscope}%
\pgfsys@transformshift{0.731393in}{2.126749in}%
\pgfsys@useobject{currentmarker}{}%
\end{pgfscope}%
\begin{pgfscope}%
\pgfsys@transformshift{0.938053in}{0.903613in}%
\pgfsys@useobject{currentmarker}{}%
\end{pgfscope}%
\begin{pgfscope}%
\pgfsys@transformshift{1.182438in}{0.760618in}%
\pgfsys@useobject{currentmarker}{}%
\end{pgfscope}%
\begin{pgfscope}%
\pgfsys@transformshift{0.930837in}{1.062002in}%
\pgfsys@useobject{currentmarker}{}%
\end{pgfscope}%
\begin{pgfscope}%
\pgfsys@transformshift{0.812485in}{1.389917in}%
\pgfsys@useobject{currentmarker}{}%
\end{pgfscope}%
\begin{pgfscope}%
\pgfsys@transformshift{1.098507in}{0.762751in}%
\pgfsys@useobject{currentmarker}{}%
\end{pgfscope}%
\begin{pgfscope}%
\pgfsys@transformshift{2.795879in}{0.714621in}%
\pgfsys@useobject{currentmarker}{}%
\end{pgfscope}%
\begin{pgfscope}%
\pgfsys@transformshift{0.805773in}{1.133025in}%
\pgfsys@useobject{currentmarker}{}%
\end{pgfscope}%
\begin{pgfscope}%
\pgfsys@transformshift{3.422862in}{0.700202in}%
\pgfsys@useobject{currentmarker}{}%
\end{pgfscope}%
\begin{pgfscope}%
\pgfsys@transformshift{2.385348in}{0.724281in}%
\pgfsys@useobject{currentmarker}{}%
\end{pgfscope}%
\begin{pgfscope}%
\pgfsys@transformshift{4.361802in}{0.683996in}%
\pgfsys@useobject{currentmarker}{}%
\end{pgfscope}%
\begin{pgfscope}%
\pgfsys@transformshift{3.351815in}{0.705738in}%
\pgfsys@useobject{currentmarker}{}%
\end{pgfscope}%
\begin{pgfscope}%
\pgfsys@transformshift{4.837497in}{0.678433in}%
\pgfsys@useobject{currentmarker}{}%
\end{pgfscope}%
\begin{pgfscope}%
\pgfsys@transformshift{4.837069in}{0.678443in}%
\pgfsys@useobject{currentmarker}{}%
\end{pgfscope}%
\begin{pgfscope}%
\pgfsys@transformshift{2.324472in}{0.729307in}%
\pgfsys@useobject{currentmarker}{}%
\end{pgfscope}%
\begin{pgfscope}%
\pgfsys@transformshift{3.663703in}{0.696271in}%
\pgfsys@useobject{currentmarker}{}%
\end{pgfscope}%
\begin{pgfscope}%
\pgfsys@transformshift{3.329899in}{0.701783in}%
\pgfsys@useobject{currentmarker}{}%
\end{pgfscope}%
\begin{pgfscope}%
\pgfsys@transformshift{4.832220in}{0.678568in}%
\pgfsys@useobject{currentmarker}{}%
\end{pgfscope}%
\begin{pgfscope}%
\pgfsys@transformshift{0.811790in}{1.074213in}%
\pgfsys@useobject{currentmarker}{}%
\end{pgfscope}%
\begin{pgfscope}%
\pgfsys@transformshift{1.033734in}{0.803801in}%
\pgfsys@useobject{currentmarker}{}%
\end{pgfscope}%
\begin{pgfscope}%
\pgfsys@transformshift{4.127106in}{0.687914in}%
\pgfsys@useobject{currentmarker}{}%
\end{pgfscope}%
\begin{pgfscope}%
\pgfsys@transformshift{1.153722in}{0.760516in}%
\pgfsys@useobject{currentmarker}{}%
\end{pgfscope}%
\begin{pgfscope}%
\pgfsys@transformshift{4.062461in}{0.694741in}%
\pgfsys@useobject{currentmarker}{}%
\end{pgfscope}%
\begin{pgfscope}%
\pgfsys@transformshift{5.031407in}{0.676882in}%
\pgfsys@useobject{currentmarker}{}%
\end{pgfscope}%
\begin{pgfscope}%
\pgfsys@transformshift{2.300667in}{0.727941in}%
\pgfsys@useobject{currentmarker}{}%
\end{pgfscope}%
\begin{pgfscope}%
\pgfsys@transformshift{2.207921in}{0.733621in}%
\pgfsys@useobject{currentmarker}{}%
\end{pgfscope}%
\begin{pgfscope}%
\pgfsys@transformshift{0.837266in}{1.050945in}%
\pgfsys@useobject{currentmarker}{}%
\end{pgfscope}%
\begin{pgfscope}%
\pgfsys@transformshift{0.811790in}{1.074213in}%
\pgfsys@useobject{currentmarker}{}%
\end{pgfscope}%
\begin{pgfscope}%
\pgfsys@transformshift{2.986378in}{0.707559in}%
\pgfsys@useobject{currentmarker}{}%
\end{pgfscope}%
\begin{pgfscope}%
\pgfsys@transformshift{2.795879in}{0.714621in}%
\pgfsys@useobject{currentmarker}{}%
\end{pgfscope}%
\begin{pgfscope}%
\pgfsys@transformshift{2.371871in}{0.722863in}%
\pgfsys@useobject{currentmarker}{}%
\end{pgfscope}%
\begin{pgfscope}%
\pgfsys@transformshift{1.121077in}{0.761098in}%
\pgfsys@useobject{currentmarker}{}%
\end{pgfscope}%
\begin{pgfscope}%
\pgfsys@transformshift{2.237687in}{0.731067in}%
\pgfsys@useobject{currentmarker}{}%
\end{pgfscope}%
\begin{pgfscope}%
\pgfsys@transformshift{4.303498in}{0.684586in}%
\pgfsys@useobject{currentmarker}{}%
\end{pgfscope}%
\begin{pgfscope}%
\pgfsys@transformshift{1.377391in}{0.754587in}%
\pgfsys@useobject{currentmarker}{}%
\end{pgfscope}%
\begin{pgfscope}%
\pgfsys@transformshift{3.686836in}{0.695848in}%
\pgfsys@useobject{currentmarker}{}%
\end{pgfscope}%
\begin{pgfscope}%
\pgfsys@transformshift{3.329899in}{0.701783in}%
\pgfsys@useobject{currentmarker}{}%
\end{pgfscope}%
\begin{pgfscope}%
\pgfsys@transformshift{5.031407in}{0.676882in}%
\pgfsys@useobject{currentmarker}{}%
\end{pgfscope}%
\begin{pgfscope}%
\pgfsys@transformshift{1.361052in}{0.753154in}%
\pgfsys@useobject{currentmarker}{}%
\end{pgfscope}%
\begin{pgfscope}%
\pgfsys@transformshift{3.388033in}{0.700409in}%
\pgfsys@useobject{currentmarker}{}%
\end{pgfscope}%
\begin{pgfscope}%
\pgfsys@transformshift{3.251221in}{0.705446in}%
\pgfsys@useobject{currentmarker}{}%
\end{pgfscope}%
\begin{pgfscope}%
\pgfsys@transformshift{3.773526in}{0.693714in}%
\pgfsys@useobject{currentmarker}{}%
\end{pgfscope}%
\begin{pgfscope}%
\pgfsys@transformshift{1.781575in}{0.742215in}%
\pgfsys@useobject{currentmarker}{}%
\end{pgfscope}%
\begin{pgfscope}%
\pgfsys@transformshift{2.416228in}{0.721021in}%
\pgfsys@useobject{currentmarker}{}%
\end{pgfscope}%
\begin{pgfscope}%
\pgfsys@transformshift{0.995536in}{0.934475in}%
\pgfsys@useobject{currentmarker}{}%
\end{pgfscope}%
\begin{pgfscope}%
\pgfsys@transformshift{0.864259in}{1.054629in}%
\pgfsys@useobject{currentmarker}{}%
\end{pgfscope}%
\begin{pgfscope}%
\pgfsys@transformshift{2.905054in}{0.710626in}%
\pgfsys@useobject{currentmarker}{}%
\end{pgfscope}%
\begin{pgfscope}%
\pgfsys@transformshift{3.251221in}{0.705446in}%
\pgfsys@useobject{currentmarker}{}%
\end{pgfscope}%
\begin{pgfscope}%
\pgfsys@transformshift{2.868522in}{0.714215in}%
\pgfsys@useobject{currentmarker}{}%
\end{pgfscope}%
\begin{pgfscope}%
\pgfsys@transformshift{1.170982in}{0.760497in}%
\pgfsys@useobject{currentmarker}{}%
\end{pgfscope}%
\begin{pgfscope}%
\pgfsys@transformshift{2.300667in}{0.727941in}%
\pgfsys@useobject{currentmarker}{}%
\end{pgfscope}%
\begin{pgfscope}%
\pgfsys@transformshift{1.957319in}{0.742293in}%
\pgfsys@useobject{currentmarker}{}%
\end{pgfscope}%
\begin{pgfscope}%
\pgfsys@transformshift{1.137094in}{0.801875in}%
\pgfsys@useobject{currentmarker}{}%
\end{pgfscope}%
\begin{pgfscope}%
\pgfsys@transformshift{1.699087in}{0.750586in}%
\pgfsys@useobject{currentmarker}{}%
\end{pgfscope}%
\begin{pgfscope}%
\pgfsys@transformshift{1.095304in}{0.883521in}%
\pgfsys@useobject{currentmarker}{}%
\end{pgfscope}%
\begin{pgfscope}%
\pgfsys@transformshift{4.683669in}{0.680143in}%
\pgfsys@useobject{currentmarker}{}%
\end{pgfscope}%
\begin{pgfscope}%
\pgfsys@transformshift{4.121499in}{0.688309in}%
\pgfsys@useobject{currentmarker}{}%
\end{pgfscope}%
\begin{pgfscope}%
\pgfsys@transformshift{1.823383in}{0.739501in}%
\pgfsys@useobject{currentmarker}{}%
\end{pgfscope}%
\begin{pgfscope}%
\pgfsys@transformshift{4.304771in}{0.684579in}%
\pgfsys@useobject{currentmarker}{}%
\end{pgfscope}%
\begin{pgfscope}%
\pgfsys@transformshift{2.859938in}{0.713419in}%
\pgfsys@useobject{currentmarker}{}%
\end{pgfscope}%
\begin{pgfscope}%
\pgfsys@transformshift{5.718003in}{0.672034in}%
\pgfsys@useobject{currentmarker}{}%
\end{pgfscope}%
\begin{pgfscope}%
\pgfsys@transformshift{2.261777in}{0.729497in}%
\pgfsys@useobject{currentmarker}{}%
\end{pgfscope}%
\begin{pgfscope}%
\pgfsys@transformshift{2.144108in}{0.740099in}%
\pgfsys@useobject{currentmarker}{}%
\end{pgfscope}%
\begin{pgfscope}%
\pgfsys@transformshift{0.864259in}{1.054629in}%
\pgfsys@useobject{currentmarker}{}%
\end{pgfscope}%
\begin{pgfscope}%
\pgfsys@transformshift{0.887344in}{0.931771in}%
\pgfsys@useobject{currentmarker}{}%
\end{pgfscope}%
\begin{pgfscope}%
\pgfsys@transformshift{1.581199in}{0.749564in}%
\pgfsys@useobject{currentmarker}{}%
\end{pgfscope}%
\begin{pgfscope}%
\pgfsys@transformshift{1.058140in}{0.776203in}%
\pgfsys@useobject{currentmarker}{}%
\end{pgfscope}%
\begin{pgfscope}%
\pgfsys@transformshift{1.699422in}{0.744529in}%
\pgfsys@useobject{currentmarker}{}%
\end{pgfscope}%
\begin{pgfscope}%
\pgfsys@transformshift{2.243656in}{0.728890in}%
\pgfsys@useobject{currentmarker}{}%
\end{pgfscope}%
\begin{pgfscope}%
\pgfsys@transformshift{0.726542in}{2.093137in}%
\pgfsys@useobject{currentmarker}{}%
\end{pgfscope}%
\begin{pgfscope}%
\pgfsys@transformshift{1.120003in}{0.761574in}%
\pgfsys@useobject{currentmarker}{}%
\end{pgfscope}%
\begin{pgfscope}%
\pgfsys@transformshift{4.118614in}{0.688354in}%
\pgfsys@useobject{currentmarker}{}%
\end{pgfscope}%
\begin{pgfscope}%
\pgfsys@transformshift{2.199954in}{0.730944in}%
\pgfsys@useobject{currentmarker}{}%
\end{pgfscope}%
\begin{pgfscope}%
\pgfsys@transformshift{3.846806in}{0.689688in}%
\pgfsys@useobject{currentmarker}{}%
\end{pgfscope}%
\begin{pgfscope}%
\pgfsys@transformshift{2.875974in}{0.712689in}%
\pgfsys@useobject{currentmarker}{}%
\end{pgfscope}%
\begin{pgfscope}%
\pgfsys@transformshift{2.365293in}{0.723451in}%
\pgfsys@useobject{currentmarker}{}%
\end{pgfscope}%
\begin{pgfscope}%
\pgfsys@transformshift{3.854410in}{0.689345in}%
\pgfsys@useobject{currentmarker}{}%
\end{pgfscope}%
\begin{pgfscope}%
\pgfsys@transformshift{0.723278in}{2.234686in}%
\pgfsys@useobject{currentmarker}{}%
\end{pgfscope}%
\begin{pgfscope}%
\pgfsys@transformshift{4.310365in}{0.684473in}%
\pgfsys@useobject{currentmarker}{}%
\end{pgfscope}%
\begin{pgfscope}%
\pgfsys@transformshift{2.253711in}{0.728381in}%
\pgfsys@useobject{currentmarker}{}%
\end{pgfscope}%
\begin{pgfscope}%
\pgfsys@transformshift{2.959661in}{0.709729in}%
\pgfsys@useobject{currentmarker}{}%
\end{pgfscope}%
\begin{pgfscope}%
\pgfsys@transformshift{4.107377in}{0.688677in}%
\pgfsys@useobject{currentmarker}{}%
\end{pgfscope}%
\begin{pgfscope}%
\pgfsys@transformshift{4.124248in}{0.687952in}%
\pgfsys@useobject{currentmarker}{}%
\end{pgfscope}%
\begin{pgfscope}%
\pgfsys@transformshift{2.316257in}{0.726716in}%
\pgfsys@useobject{currentmarker}{}%
\end{pgfscope}%
\begin{pgfscope}%
\pgfsys@transformshift{2.178291in}{0.733071in}%
\pgfsys@useobject{currentmarker}{}%
\end{pgfscope}%
\begin{pgfscope}%
\pgfsys@transformshift{0.757701in}{1.444369in}%
\pgfsys@useobject{currentmarker}{}%
\end{pgfscope}%
\begin{pgfscope}%
\pgfsys@transformshift{1.831046in}{0.738746in}%
\pgfsys@useobject{currentmarker}{}%
\end{pgfscope}%
\begin{pgfscope}%
\pgfsys@transformshift{3.854410in}{0.689345in}%
\pgfsys@useobject{currentmarker}{}%
\end{pgfscope}%
\begin{pgfscope}%
\pgfsys@transformshift{4.848450in}{0.678233in}%
\pgfsys@useobject{currentmarker}{}%
\end{pgfscope}%
\begin{pgfscope}%
\pgfsys@transformshift{0.729062in}{2.061187in}%
\pgfsys@useobject{currentmarker}{}%
\end{pgfscope}%
\begin{pgfscope}%
\pgfsys@transformshift{2.846360in}{0.714468in}%
\pgfsys@useobject{currentmarker}{}%
\end{pgfscope}%
\begin{pgfscope}%
\pgfsys@transformshift{4.108122in}{0.688661in}%
\pgfsys@useobject{currentmarker}{}%
\end{pgfscope}%
\begin{pgfscope}%
\pgfsys@transformshift{4.124248in}{0.687952in}%
\pgfsys@useobject{currentmarker}{}%
\end{pgfscope}%
\begin{pgfscope}%
\pgfsys@transformshift{4.136586in}{0.687753in}%
\pgfsys@useobject{currentmarker}{}%
\end{pgfscope}%
\begin{pgfscope}%
\pgfsys@transformshift{3.120443in}{0.707402in}%
\pgfsys@useobject{currentmarker}{}%
\end{pgfscope}%
\begin{pgfscope}%
\pgfsys@transformshift{5.859381in}{0.671999in}%
\pgfsys@useobject{currentmarker}{}%
\end{pgfscope}%
\begin{pgfscope}%
\pgfsys@transformshift{1.582315in}{0.749096in}%
\pgfsys@useobject{currentmarker}{}%
\end{pgfscope}%
\begin{pgfscope}%
\pgfsys@transformshift{0.750732in}{1.600285in}%
\pgfsys@useobject{currentmarker}{}%
\end{pgfscope}%
\begin{pgfscope}%
\pgfsys@transformshift{0.870738in}{1.029103in}%
\pgfsys@useobject{currentmarker}{}%
\end{pgfscope}%
\begin{pgfscope}%
\pgfsys@transformshift{2.199488in}{0.732818in}%
\pgfsys@useobject{currentmarker}{}%
\end{pgfscope}%
\begin{pgfscope}%
\pgfsys@transformshift{2.900819in}{0.710792in}%
\pgfsys@useobject{currentmarker}{}%
\end{pgfscope}%
\begin{pgfscope}%
\pgfsys@transformshift{0.729062in}{2.061187in}%
\pgfsys@useobject{currentmarker}{}%
\end{pgfscope}%
\begin{pgfscope}%
\pgfsys@transformshift{0.757652in}{1.446648in}%
\pgfsys@useobject{currentmarker}{}%
\end{pgfscope}%
\begin{pgfscope}%
\pgfsys@transformshift{0.871713in}{1.029070in}%
\pgfsys@useobject{currentmarker}{}%
\end{pgfscope}%
\begin{pgfscope}%
\pgfsys@transformshift{0.774650in}{1.239856in}%
\pgfsys@useobject{currentmarker}{}%
\end{pgfscope}%
\begin{pgfscope}%
\pgfsys@transformshift{5.713447in}{0.671993in}%
\pgfsys@useobject{currentmarker}{}%
\end{pgfscope}%
\begin{pgfscope}%
\pgfsys@transformshift{1.114966in}{0.761591in}%
\pgfsys@useobject{currentmarker}{}%
\end{pgfscope}%
\begin{pgfscope}%
\pgfsys@transformshift{0.757462in}{1.448292in}%
\pgfsys@useobject{currentmarker}{}%
\end{pgfscope}%
\begin{pgfscope}%
\pgfsys@transformshift{2.253711in}{0.728381in}%
\pgfsys@useobject{currentmarker}{}%
\end{pgfscope}%
\begin{pgfscope}%
\pgfsys@transformshift{0.774504in}{1.382015in}%
\pgfsys@useobject{currentmarker}{}%
\end{pgfscope}%
\begin{pgfscope}%
\pgfsys@transformshift{0.711064in}{3.586748in}%
\pgfsys@useobject{currentmarker}{}%
\end{pgfscope}%
\begin{pgfscope}%
\pgfsys@transformshift{0.747618in}{1.656351in}%
\pgfsys@useobject{currentmarker}{}%
\end{pgfscope}%
\begin{pgfscope}%
\pgfsys@transformshift{2.897206in}{0.712688in}%
\pgfsys@useobject{currentmarker}{}%
\end{pgfscope}%
\begin{pgfscope}%
\pgfsys@transformshift{2.220698in}{0.730122in}%
\pgfsys@useobject{currentmarker}{}%
\end{pgfscope}%
\begin{pgfscope}%
\pgfsys@transformshift{4.279999in}{0.685635in}%
\pgfsys@useobject{currentmarker}{}%
\end{pgfscope}%
\begin{pgfscope}%
\pgfsys@transformshift{2.814621in}{0.712998in}%
\pgfsys@useobject{currentmarker}{}%
\end{pgfscope}%
\begin{pgfscope}%
\pgfsys@transformshift{4.256090in}{0.686304in}%
\pgfsys@useobject{currentmarker}{}%
\end{pgfscope}%
\begin{pgfscope}%
\pgfsys@transformshift{0.708512in}{3.949312in}%
\pgfsys@useobject{currentmarker}{}%
\end{pgfscope}%
\begin{pgfscope}%
\pgfsys@transformshift{1.050911in}{0.780778in}%
\pgfsys@useobject{currentmarker}{}%
\end{pgfscope}%
\begin{pgfscope}%
\pgfsys@transformshift{2.313887in}{0.726899in}%
\pgfsys@useobject{currentmarker}{}%
\end{pgfscope}%
\begin{pgfscope}%
\pgfsys@transformshift{0.734643in}{1.835038in}%
\pgfsys@useobject{currentmarker}{}%
\end{pgfscope}%
\begin{pgfscope}%
\pgfsys@transformshift{1.000199in}{0.889293in}%
\pgfsys@useobject{currentmarker}{}%
\end{pgfscope}%
\begin{pgfscope}%
\pgfsys@transformshift{0.783599in}{1.171104in}%
\pgfsys@useobject{currentmarker}{}%
\end{pgfscope}%
\begin{pgfscope}%
\pgfsys@transformshift{1.098141in}{0.763527in}%
\pgfsys@useobject{currentmarker}{}%
\end{pgfscope}%
\begin{pgfscope}%
\pgfsys@transformshift{1.794970in}{0.741995in}%
\pgfsys@useobject{currentmarker}{}%
\end{pgfscope}%
\begin{pgfscope}%
\pgfsys@transformshift{1.869152in}{0.737136in}%
\pgfsys@useobject{currentmarker}{}%
\end{pgfscope}%
\begin{pgfscope}%
\pgfsys@transformshift{1.831046in}{0.738746in}%
\pgfsys@useobject{currentmarker}{}%
\end{pgfscope}%
\begin{pgfscope}%
\pgfsys@transformshift{2.178578in}{0.733058in}%
\pgfsys@useobject{currentmarker}{}%
\end{pgfscope}%
\begin{pgfscope}%
\pgfsys@transformshift{0.725895in}{2.126939in}%
\pgfsys@useobject{currentmarker}{}%
\end{pgfscope}%
\begin{pgfscope}%
\pgfsys@transformshift{3.212090in}{0.704793in}%
\pgfsys@useobject{currentmarker}{}%
\end{pgfscope}%
\begin{pgfscope}%
\pgfsys@transformshift{3.849406in}{0.689570in}%
\pgfsys@useobject{currentmarker}{}%
\end{pgfscope}%
\begin{pgfscope}%
\pgfsys@transformshift{1.085895in}{0.773659in}%
\pgfsys@useobject{currentmarker}{}%
\end{pgfscope}%
\begin{pgfscope}%
\pgfsys@transformshift{1.579966in}{0.748708in}%
\pgfsys@useobject{currentmarker}{}%
\end{pgfscope}%
\begin{pgfscope}%
\pgfsys@transformshift{0.757541in}{1.447062in}%
\pgfsys@useobject{currentmarker}{}%
\end{pgfscope}%
\begin{pgfscope}%
\pgfsys@transformshift{0.757622in}{1.445599in}%
\pgfsys@useobject{currentmarker}{}%
\end{pgfscope}%
\begin{pgfscope}%
\pgfsys@transformshift{2.193898in}{0.730612in}%
\pgfsys@useobject{currentmarker}{}%
\end{pgfscope}%
\begin{pgfscope}%
\pgfsys@transformshift{1.706461in}{0.743411in}%
\pgfsys@useobject{currentmarker}{}%
\end{pgfscope}%
\begin{pgfscope}%
\pgfsys@transformshift{3.868610in}{0.688845in}%
\pgfsys@useobject{currentmarker}{}%
\end{pgfscope}%
\begin{pgfscope}%
\pgfsys@transformshift{0.711105in}{3.569623in}%
\pgfsys@useobject{currentmarker}{}%
\end{pgfscope}%
\begin{pgfscope}%
\pgfsys@transformshift{2.131801in}{0.733311in}%
\pgfsys@useobject{currentmarker}{}%
\end{pgfscope}%
\begin{pgfscope}%
\pgfsys@transformshift{1.811959in}{0.739776in}%
\pgfsys@useobject{currentmarker}{}%
\end{pgfscope}%
\begin{pgfscope}%
\pgfsys@transformshift{1.035508in}{0.800095in}%
\pgfsys@useobject{currentmarker}{}%
\end{pgfscope}%
\begin{pgfscope}%
\pgfsys@transformshift{2.372554in}{0.722665in}%
\pgfsys@useobject{currentmarker}{}%
\end{pgfscope}%
\begin{pgfscope}%
\pgfsys@transformshift{1.018098in}{0.803363in}%
\pgfsys@useobject{currentmarker}{}%
\end{pgfscope}%
\begin{pgfscope}%
\pgfsys@transformshift{1.104345in}{0.762273in}%
\pgfsys@useobject{currentmarker}{}%
\end{pgfscope}%
\begin{pgfscope}%
\pgfsys@transformshift{1.331450in}{0.754282in}%
\pgfsys@useobject{currentmarker}{}%
\end{pgfscope}%
\begin{pgfscope}%
\pgfsys@transformshift{0.744591in}{1.695043in}%
\pgfsys@useobject{currentmarker}{}%
\end{pgfscope}%
\begin{pgfscope}%
\pgfsys@transformshift{0.710076in}{3.467627in}%
\pgfsys@useobject{currentmarker}{}%
\end{pgfscope}%
\begin{pgfscope}%
\pgfsys@transformshift{2.938944in}{0.707079in}%
\pgfsys@useobject{currentmarker}{}%
\end{pgfscope}%
\begin{pgfscope}%
\pgfsys@transformshift{1.842127in}{0.737834in}%
\pgfsys@useobject{currentmarker}{}%
\end{pgfscope}%
\begin{pgfscope}%
\pgfsys@transformshift{0.759918in}{1.437523in}%
\pgfsys@useobject{currentmarker}{}%
\end{pgfscope}%
\begin{pgfscope}%
\pgfsys@transformshift{0.754110in}{1.510704in}%
\pgfsys@useobject{currentmarker}{}%
\end{pgfscope}%
\begin{pgfscope}%
\pgfsys@transformshift{0.715473in}{2.770625in}%
\pgfsys@useobject{currentmarker}{}%
\end{pgfscope}%
\begin{pgfscope}%
\pgfsys@transformshift{2.297934in}{0.727444in}%
\pgfsys@useobject{currentmarker}{}%
\end{pgfscope}%
\begin{pgfscope}%
\pgfsys@transformshift{0.795350in}{1.153728in}%
\pgfsys@useobject{currentmarker}{}%
\end{pgfscope}%
\begin{pgfscope}%
\pgfsys@transformshift{0.766503in}{1.363338in}%
\pgfsys@useobject{currentmarker}{}%
\end{pgfscope}%
\begin{pgfscope}%
\pgfsys@transformshift{0.731254in}{1.946189in}%
\pgfsys@useobject{currentmarker}{}%
\end{pgfscope}%
\begin{pgfscope}%
\pgfsys@transformshift{0.760425in}{1.420282in}%
\pgfsys@useobject{currentmarker}{}%
\end{pgfscope}%
\begin{pgfscope}%
\pgfsys@transformshift{1.465829in}{0.749310in}%
\pgfsys@useobject{currentmarker}{}%
\end{pgfscope}%
\begin{pgfscope}%
\pgfsys@transformshift{0.841100in}{1.041658in}%
\pgfsys@useobject{currentmarker}{}%
\end{pgfscope}%
\begin{pgfscope}%
\pgfsys@transformshift{1.579043in}{0.748572in}%
\pgfsys@useobject{currentmarker}{}%
\end{pgfscope}%
\begin{pgfscope}%
\pgfsys@transformshift{0.745609in}{1.595483in}%
\pgfsys@useobject{currentmarker}{}%
\end{pgfscope}%
\begin{pgfscope}%
\pgfsys@transformshift{0.715071in}{2.758869in}%
\pgfsys@useobject{currentmarker}{}%
\end{pgfscope}%
\begin{pgfscope}%
\pgfsys@transformshift{0.703027in}{5.172624in}%
\pgfsys@useobject{currentmarker}{}%
\end{pgfscope}%
\begin{pgfscope}%
\pgfsys@transformshift{0.743747in}{1.743507in}%
\pgfsys@useobject{currentmarker}{}%
\end{pgfscope}%
\begin{pgfscope}%
\pgfsys@transformshift{0.709222in}{3.235282in}%
\pgfsys@useobject{currentmarker}{}%
\end{pgfscope}%
\begin{pgfscope}%
\pgfsys@transformshift{0.732410in}{1.890516in}%
\pgfsys@useobject{currentmarker}{}%
\end{pgfscope}%
\begin{pgfscope}%
\pgfsys@transformshift{0.750011in}{1.567536in}%
\pgfsys@useobject{currentmarker}{}%
\end{pgfscope}%
\begin{pgfscope}%
\pgfsys@transformshift{0.783190in}{1.169845in}%
\pgfsys@useobject{currentmarker}{}%
\end{pgfscope}%
\begin{pgfscope}%
\pgfsys@transformshift{0.739505in}{1.787144in}%
\pgfsys@useobject{currentmarker}{}%
\end{pgfscope}%
\begin{pgfscope}%
\pgfsys@transformshift{0.752421in}{1.414683in}%
\pgfsys@useobject{currentmarker}{}%
\end{pgfscope}%
\begin{pgfscope}%
\pgfsys@transformshift{0.807546in}{1.083680in}%
\pgfsys@useobject{currentmarker}{}%
\end{pgfscope}%
\begin{pgfscope}%
\pgfsys@transformshift{0.727085in}{1.902374in}%
\pgfsys@useobject{currentmarker}{}%
\end{pgfscope}%
\begin{pgfscope}%
\pgfsys@transformshift{0.723003in}{2.114260in}%
\pgfsys@useobject{currentmarker}{}%
\end{pgfscope}%
\begin{pgfscope}%
\pgfsys@transformshift{0.732661in}{1.882602in}%
\pgfsys@useobject{currentmarker}{}%
\end{pgfscope}%
\begin{pgfscope}%
\pgfsys@transformshift{2.233434in}{0.729073in}%
\pgfsys@useobject{currentmarker}{}%
\end{pgfscope}%
\begin{pgfscope}%
\pgfsys@transformshift{0.741455in}{1.772160in}%
\pgfsys@useobject{currentmarker}{}%
\end{pgfscope}%
\begin{pgfscope}%
\pgfsys@transformshift{5.516657in}{0.673454in}%
\pgfsys@useobject{currentmarker}{}%
\end{pgfscope}%
\begin{pgfscope}%
\pgfsys@transformshift{1.332663in}{0.754165in}%
\pgfsys@useobject{currentmarker}{}%
\end{pgfscope}%
\begin{pgfscope}%
\pgfsys@transformshift{5.436411in}{0.673461in}%
\pgfsys@useobject{currentmarker}{}%
\end{pgfscope}%
\begin{pgfscope}%
\pgfsys@transformshift{0.698207in}{3.915342in}%
\pgfsys@useobject{currentmarker}{}%
\end{pgfscope}%
\begin{pgfscope}%
\pgfsys@transformshift{0.752345in}{1.350208in}%
\pgfsys@useobject{currentmarker}{}%
\end{pgfscope}%
\begin{pgfscope}%
\pgfsys@transformshift{3.596288in}{0.692444in}%
\pgfsys@useobject{currentmarker}{}%
\end{pgfscope}%
\begin{pgfscope}%
\pgfsys@transformshift{0.750171in}{1.558245in}%
\pgfsys@useobject{currentmarker}{}%
\end{pgfscope}%
\begin{pgfscope}%
\pgfsys@transformshift{0.844482in}{0.999371in}%
\pgfsys@useobject{currentmarker}{}%
\end{pgfscope}%
\begin{pgfscope}%
\pgfsys@transformshift{0.877749in}{0.973866in}%
\pgfsys@useobject{currentmarker}{}%
\end{pgfscope}%
\begin{pgfscope}%
\pgfsys@transformshift{0.781838in}{1.219120in}%
\pgfsys@useobject{currentmarker}{}%
\end{pgfscope}%
\begin{pgfscope}%
\pgfsys@transformshift{0.721159in}{2.355097in}%
\pgfsys@useobject{currentmarker}{}%
\end{pgfscope}%
\begin{pgfscope}%
\pgfsys@transformshift{4.115327in}{0.687449in}%
\pgfsys@useobject{currentmarker}{}%
\end{pgfscope}%
\begin{pgfscope}%
\pgfsys@transformshift{0.715071in}{2.758869in}%
\pgfsys@useobject{currentmarker}{}%
\end{pgfscope}%
\begin{pgfscope}%
\pgfsys@transformshift{1.117508in}{0.761549in}%
\pgfsys@useobject{currentmarker}{}%
\end{pgfscope}%
\begin{pgfscope}%
\pgfsys@transformshift{1.331618in}{0.754264in}%
\pgfsys@useobject{currentmarker}{}%
\end{pgfscope}%
\begin{pgfscope}%
\pgfsys@transformshift{0.719105in}{2.488469in}%
\pgfsys@useobject{currentmarker}{}%
\end{pgfscope}%
\begin{pgfscope}%
\pgfsys@transformshift{1.071211in}{0.766943in}%
\pgfsys@useobject{currentmarker}{}%
\end{pgfscope}%
\begin{pgfscope}%
\pgfsys@transformshift{0.771866in}{1.335450in}%
\pgfsys@useobject{currentmarker}{}%
\end{pgfscope}%
\begin{pgfscope}%
\pgfsys@transformshift{0.930066in}{0.888254in}%
\pgfsys@useobject{currentmarker}{}%
\end{pgfscope}%
\begin{pgfscope}%
\pgfsys@transformshift{0.877065in}{0.997446in}%
\pgfsys@useobject{currentmarker}{}%
\end{pgfscope}%
\begin{pgfscope}%
\pgfsys@transformshift{0.781369in}{1.222616in}%
\pgfsys@useobject{currentmarker}{}%
\end{pgfscope}%
\begin{pgfscope}%
\pgfsys@transformshift{4.113921in}{0.687564in}%
\pgfsys@useobject{currentmarker}{}%
\end{pgfscope}%
\begin{pgfscope}%
\pgfsys@transformshift{1.332492in}{0.754183in}%
\pgfsys@useobject{currentmarker}{}%
\end{pgfscope}%
\begin{pgfscope}%
\pgfsys@transformshift{0.740035in}{1.631609in}%
\pgfsys@useobject{currentmarker}{}%
\end{pgfscope}%
\begin{pgfscope}%
\pgfsys@transformshift{0.867031in}{0.998078in}%
\pgfsys@useobject{currentmarker}{}%
\end{pgfscope}%
\begin{pgfscope}%
\pgfsys@transformshift{1.016164in}{0.852402in}%
\pgfsys@useobject{currentmarker}{}%
\end{pgfscope}%
\begin{pgfscope}%
\pgfsys@transformshift{0.885952in}{0.914664in}%
\pgfsys@useobject{currentmarker}{}%
\end{pgfscope}%
\begin{pgfscope}%
\pgfsys@transformshift{0.693161in}{4.369143in}%
\pgfsys@useobject{currentmarker}{}%
\end{pgfscope}%
\begin{pgfscope}%
\pgfsys@transformshift{1.341785in}{0.753847in}%
\pgfsys@useobject{currentmarker}{}%
\end{pgfscope}%
\begin{pgfscope}%
\pgfsys@transformshift{1.117032in}{0.761543in}%
\pgfsys@useobject{currentmarker}{}%
\end{pgfscope}%
\begin{pgfscope}%
\pgfsys@transformshift{0.764486in}{1.266983in}%
\pgfsys@useobject{currentmarker}{}%
\end{pgfscope}%
\begin{pgfscope}%
\pgfsys@transformshift{0.878473in}{0.973773in}%
\pgfsys@useobject{currentmarker}{}%
\end{pgfscope}%
\begin{pgfscope}%
\pgfsys@transformshift{0.697211in}{4.143571in}%
\pgfsys@useobject{currentmarker}{}%
\end{pgfscope}%
\begin{pgfscope}%
\pgfsys@transformshift{0.707681in}{3.425414in}%
\pgfsys@useobject{currentmarker}{}%
\end{pgfscope}%
\begin{pgfscope}%
\pgfsys@transformshift{1.271885in}{0.758240in}%
\pgfsys@useobject{currentmarker}{}%
\end{pgfscope}%
\begin{pgfscope}%
\pgfsys@transformshift{0.707456in}{3.522902in}%
\pgfsys@useobject{currentmarker}{}%
\end{pgfscope}%
\begin{pgfscope}%
\pgfsys@transformshift{3.691441in}{0.698225in}%
\pgfsys@useobject{currentmarker}{}%
\end{pgfscope}%
\begin{pgfscope}%
\pgfsys@transformshift{1.728224in}{0.740423in}%
\pgfsys@useobject{currentmarker}{}%
\end{pgfscope}%
\begin{pgfscope}%
\pgfsys@transformshift{5.449399in}{0.678586in}%
\pgfsys@useobject{currentmarker}{}%
\end{pgfscope}%
\begin{pgfscope}%
\pgfsys@transformshift{0.742517in}{1.541208in}%
\pgfsys@useobject{currentmarker}{}%
\end{pgfscope}%
\begin{pgfscope}%
\pgfsys@transformshift{0.734651in}{1.834867in}%
\pgfsys@useobject{currentmarker}{}%
\end{pgfscope}%
\begin{pgfscope}%
\pgfsys@transformshift{0.792193in}{1.084317in}%
\pgfsys@useobject{currentmarker}{}%
\end{pgfscope}%
\begin{pgfscope}%
\pgfsys@transformshift{0.844482in}{0.999371in}%
\pgfsys@useobject{currentmarker}{}%
\end{pgfscope}%
\begin{pgfscope}%
\pgfsys@transformshift{3.691441in}{0.698225in}%
\pgfsys@useobject{currentmarker}{}%
\end{pgfscope}%
\begin{pgfscope}%
\pgfsys@transformshift{0.709097in}{3.044923in}%
\pgfsys@useobject{currentmarker}{}%
\end{pgfscope}%
\begin{pgfscope}%
\pgfsys@transformshift{0.718884in}{2.307843in}%
\pgfsys@useobject{currentmarker}{}%
\end{pgfscope}%
\begin{pgfscope}%
\pgfsys@transformshift{0.841046in}{1.042660in}%
\pgfsys@useobject{currentmarker}{}%
\end{pgfscope}%
\begin{pgfscope}%
\pgfsys@transformshift{2.632031in}{0.713187in}%
\pgfsys@useobject{currentmarker}{}%
\end{pgfscope}%
\begin{pgfscope}%
\pgfsys@transformshift{0.730100in}{1.803435in}%
\pgfsys@useobject{currentmarker}{}%
\end{pgfscope}%
\begin{pgfscope}%
\pgfsys@transformshift{0.763926in}{1.272758in}%
\pgfsys@useobject{currentmarker}{}%
\end{pgfscope}%
\begin{pgfscope}%
\pgfsys@transformshift{0.811698in}{1.083530in}%
\pgfsys@useobject{currentmarker}{}%
\end{pgfscope}%
\begin{pgfscope}%
\pgfsys@transformshift{0.829996in}{1.038066in}%
\pgfsys@useobject{currentmarker}{}%
\end{pgfscope}%
\begin{pgfscope}%
\pgfsys@transformshift{0.740747in}{1.573135in}%
\pgfsys@useobject{currentmarker}{}%
\end{pgfscope}%
\begin{pgfscope}%
\pgfsys@transformshift{0.778642in}{1.235540in}%
\pgfsys@useobject{currentmarker}{}%
\end{pgfscope}%
\begin{pgfscope}%
\pgfsys@transformshift{2.486053in}{0.718875in}%
\pgfsys@useobject{currentmarker}{}%
\end{pgfscope}%
\begin{pgfscope}%
\pgfsys@transformshift{0.807836in}{1.080891in}%
\pgfsys@useobject{currentmarker}{}%
\end{pgfscope}%
\begin{pgfscope}%
\pgfsys@transformshift{1.761629in}{0.739916in}%
\pgfsys@useobject{currentmarker}{}%
\end{pgfscope}%
\begin{pgfscope}%
\pgfsys@transformshift{0.877749in}{0.973866in}%
\pgfsys@useobject{currentmarker}{}%
\end{pgfscope}%
\begin{pgfscope}%
\pgfsys@transformshift{0.730718in}{1.787783in}%
\pgfsys@useobject{currentmarker}{}%
\end{pgfscope}%
\begin{pgfscope}%
\pgfsys@transformshift{2.943992in}{0.707474in}%
\pgfsys@useobject{currentmarker}{}%
\end{pgfscope}%
\begin{pgfscope}%
\pgfsys@transformshift{0.740123in}{1.629871in}%
\pgfsys@useobject{currentmarker}{}%
\end{pgfscope}%
\begin{pgfscope}%
\pgfsys@transformshift{0.809594in}{1.075017in}%
\pgfsys@useobject{currentmarker}{}%
\end{pgfscope}%
\begin{pgfscope}%
\pgfsys@transformshift{0.832252in}{1.037367in}%
\pgfsys@useobject{currentmarker}{}%
\end{pgfscope}%
\begin{pgfscope}%
\pgfsys@transformshift{0.869014in}{0.980635in}%
\pgfsys@useobject{currentmarker}{}%
\end{pgfscope}%
\begin{pgfscope}%
\pgfsys@transformshift{1.160820in}{0.760431in}%
\pgfsys@useobject{currentmarker}{}%
\end{pgfscope}%
\begin{pgfscope}%
\pgfsys@transformshift{0.782005in}{1.139153in}%
\pgfsys@useobject{currentmarker}{}%
\end{pgfscope}%
\begin{pgfscope}%
\pgfsys@transformshift{0.745393in}{1.514732in}%
\pgfsys@useobject{currentmarker}{}%
\end{pgfscope}%
\begin{pgfscope}%
\pgfsys@transformshift{0.807474in}{1.083664in}%
\pgfsys@useobject{currentmarker}{}%
\end{pgfscope}%
\begin{pgfscope}%
\pgfsys@transformshift{0.778642in}{1.235540in}%
\pgfsys@useobject{currentmarker}{}%
\end{pgfscope}%
\begin{pgfscope}%
\pgfsys@transformshift{0.718884in}{2.307843in}%
\pgfsys@useobject{currentmarker}{}%
\end{pgfscope}%
\begin{pgfscope}%
\pgfsys@transformshift{4.091845in}{0.686381in}%
\pgfsys@useobject{currentmarker}{}%
\end{pgfscope}%
\begin{pgfscope}%
\pgfsys@transformshift{0.715397in}{2.625368in}%
\pgfsys@useobject{currentmarker}{}%
\end{pgfscope}%
\begin{pgfscope}%
\pgfsys@transformshift{2.016064in}{0.731737in}%
\pgfsys@useobject{currentmarker}{}%
\end{pgfscope}%
\begin{pgfscope}%
\pgfsys@transformshift{0.712353in}{2.840169in}%
\pgfsys@useobject{currentmarker}{}%
\end{pgfscope}%
\begin{pgfscope}%
\pgfsys@transformshift{0.773376in}{1.210624in}%
\pgfsys@useobject{currentmarker}{}%
\end{pgfscope}%
\begin{pgfscope}%
\pgfsys@transformshift{1.418191in}{0.752115in}%
\pgfsys@useobject{currentmarker}{}%
\end{pgfscope}%
\begin{pgfscope}%
\pgfsys@transformshift{0.746592in}{1.394000in}%
\pgfsys@useobject{currentmarker}{}%
\end{pgfscope}%
\begin{pgfscope}%
\pgfsys@transformshift{0.701345in}{3.558576in}%
\pgfsys@useobject{currentmarker}{}%
\end{pgfscope}%
\begin{pgfscope}%
\pgfsys@transformshift{1.288095in}{0.756782in}%
\pgfsys@useobject{currentmarker}{}%
\end{pgfscope}%
\begin{pgfscope}%
\pgfsys@transformshift{0.740008in}{1.632327in}%
\pgfsys@useobject{currentmarker}{}%
\end{pgfscope}%
\begin{pgfscope}%
\pgfsys@transformshift{0.742140in}{1.550546in}%
\pgfsys@useobject{currentmarker}{}%
\end{pgfscope}%
\begin{pgfscope}%
\pgfsys@transformshift{0.744933in}{1.420451in}%
\pgfsys@useobject{currentmarker}{}%
\end{pgfscope}%
\begin{pgfscope}%
\pgfsys@transformshift{0.805424in}{1.070407in}%
\pgfsys@useobject{currentmarker}{}%
\end{pgfscope}%
\begin{pgfscope}%
\pgfsys@transformshift{0.735486in}{1.652783in}%
\pgfsys@useobject{currentmarker}{}%
\end{pgfscope}%
\begin{pgfscope}%
\pgfsys@transformshift{0.739994in}{1.588559in}%
\pgfsys@useobject{currentmarker}{}%
\end{pgfscope}%
\begin{pgfscope}%
\pgfsys@transformshift{2.454263in}{0.719125in}%
\pgfsys@useobject{currentmarker}{}%
\end{pgfscope}%
\begin{pgfscope}%
\pgfsys@transformshift{0.717795in}{2.637385in}%
\pgfsys@useobject{currentmarker}{}%
\end{pgfscope}%
\begin{pgfscope}%
\pgfsys@transformshift{0.744933in}{1.420451in}%
\pgfsys@useobject{currentmarker}{}%
\end{pgfscope}%
\begin{pgfscope}%
\pgfsys@transformshift{0.832011in}{0.993452in}%
\pgfsys@useobject{currentmarker}{}%
\end{pgfscope}%
\begin{pgfscope}%
\pgfsys@transformshift{0.885524in}{0.921106in}%
\pgfsys@useobject{currentmarker}{}%
\end{pgfscope}%
\begin{pgfscope}%
\pgfsys@transformshift{0.734965in}{1.657799in}%
\pgfsys@useobject{currentmarker}{}%
\end{pgfscope}%
\begin{pgfscope}%
\pgfsys@transformshift{0.740126in}{1.588036in}%
\pgfsys@useobject{currentmarker}{}%
\end{pgfscope}%
\begin{pgfscope}%
\pgfsys@transformshift{3.695036in}{0.698033in}%
\pgfsys@useobject{currentmarker}{}%
\end{pgfscope}%
\begin{pgfscope}%
\pgfsys@transformshift{0.764859in}{1.263004in}%
\pgfsys@useobject{currentmarker}{}%
\end{pgfscope}%
\begin{pgfscope}%
\pgfsys@transformshift{2.423695in}{0.721311in}%
\pgfsys@useobject{currentmarker}{}%
\end{pgfscope}%
\begin{pgfscope}%
\pgfsys@transformshift{2.454263in}{0.719125in}%
\pgfsys@useobject{currentmarker}{}%
\end{pgfscope}%
\begin{pgfscope}%
\pgfsys@transformshift{3.448449in}{0.701321in}%
\pgfsys@useobject{currentmarker}{}%
\end{pgfscope}%
\begin{pgfscope}%
\pgfsys@transformshift{0.769210in}{1.220963in}%
\pgfsys@useobject{currentmarker}{}%
\end{pgfscope}%
\begin{pgfscope}%
\pgfsys@transformshift{0.772737in}{1.163625in}%
\pgfsys@useobject{currentmarker}{}%
\end{pgfscope}%
\begin{pgfscope}%
\pgfsys@transformshift{0.960727in}{0.854022in}%
\pgfsys@useobject{currentmarker}{}%
\end{pgfscope}%
\begin{pgfscope}%
\pgfsys@transformshift{0.782559in}{1.113067in}%
\pgfsys@useobject{currentmarker}{}%
\end{pgfscope}%
\begin{pgfscope}%
\pgfsys@transformshift{0.740120in}{1.578168in}%
\pgfsys@useobject{currentmarker}{}%
\end{pgfscope}%
\begin{pgfscope}%
\pgfsys@transformshift{1.018163in}{0.803247in}%
\pgfsys@useobject{currentmarker}{}%
\end{pgfscope}%
\begin{pgfscope}%
\pgfsys@transformshift{0.737495in}{1.636902in}%
\pgfsys@useobject{currentmarker}{}%
\end{pgfscope}%
\begin{pgfscope}%
\pgfsys@transformshift{0.718891in}{2.307604in}%
\pgfsys@useobject{currentmarker}{}%
\end{pgfscope}%
\begin{pgfscope}%
\pgfsys@transformshift{0.717795in}{2.637385in}%
\pgfsys@useobject{currentmarker}{}%
\end{pgfscope}%
\begin{pgfscope}%
\pgfsys@transformshift{0.751698in}{1.359672in}%
\pgfsys@useobject{currentmarker}{}%
\end{pgfscope}%
\begin{pgfscope}%
\pgfsys@transformshift{3.832567in}{0.688500in}%
\pgfsys@useobject{currentmarker}{}%
\end{pgfscope}%
\begin{pgfscope}%
\pgfsys@transformshift{0.752565in}{1.347576in}%
\pgfsys@useobject{currentmarker}{}%
\end{pgfscope}%
\begin{pgfscope}%
\pgfsys@transformshift{0.799897in}{1.065955in}%
\pgfsys@useobject{currentmarker}{}%
\end{pgfscope}%
\begin{pgfscope}%
\pgfsys@transformshift{0.869749in}{0.972392in}%
\pgfsys@useobject{currentmarker}{}%
\end{pgfscope}%
\begin{pgfscope}%
\pgfsys@transformshift{0.976203in}{0.852287in}%
\pgfsys@useobject{currentmarker}{}%
\end{pgfscope}%
\begin{pgfscope}%
\pgfsys@transformshift{0.730067in}{1.803444in}%
\pgfsys@useobject{currentmarker}{}%
\end{pgfscope}%
\begin{pgfscope}%
\pgfsys@transformshift{3.237089in}{0.706760in}%
\pgfsys@useobject{currentmarker}{}%
\end{pgfscope}%
\begin{pgfscope}%
\pgfsys@transformshift{0.734104in}{1.703731in}%
\pgfsys@useobject{currentmarker}{}%
\end{pgfscope}%
\begin{pgfscope}%
\pgfsys@transformshift{0.763777in}{1.274304in}%
\pgfsys@useobject{currentmarker}{}%
\end{pgfscope}%
\begin{pgfscope}%
\pgfsys@transformshift{1.186083in}{0.760404in}%
\pgfsys@useobject{currentmarker}{}%
\end{pgfscope}%
\begin{pgfscope}%
\pgfsys@transformshift{0.698224in}{3.764284in}%
\pgfsys@useobject{currentmarker}{}%
\end{pgfscope}%
\begin{pgfscope}%
\pgfsys@transformshift{0.794429in}{1.069044in}%
\pgfsys@useobject{currentmarker}{}%
\end{pgfscope}%
\begin{pgfscope}%
\pgfsys@transformshift{0.717795in}{2.637385in}%
\pgfsys@useobject{currentmarker}{}%
\end{pgfscope}%
\begin{pgfscope}%
\pgfsys@transformshift{2.202944in}{0.726283in}%
\pgfsys@useobject{currentmarker}{}%
\end{pgfscope}%
\begin{pgfscope}%
\pgfsys@transformshift{0.729875in}{1.721374in}%
\pgfsys@useobject{currentmarker}{}%
\end{pgfscope}%
\begin{pgfscope}%
\pgfsys@transformshift{0.767264in}{1.216158in}%
\pgfsys@useobject{currentmarker}{}%
\end{pgfscope}%
\begin{pgfscope}%
\pgfsys@transformshift{0.767383in}{1.202379in}%
\pgfsys@useobject{currentmarker}{}%
\end{pgfscope}%
\begin{pgfscope}%
\pgfsys@transformshift{0.704059in}{3.640385in}%
\pgfsys@useobject{currentmarker}{}%
\end{pgfscope}%
\begin{pgfscope}%
\pgfsys@transformshift{0.745963in}{1.403981in}%
\pgfsys@useobject{currentmarker}{}%
\end{pgfscope}%
\begin{pgfscope}%
\pgfsys@transformshift{0.772666in}{1.163577in}%
\pgfsys@useobject{currentmarker}{}%
\end{pgfscope}%
\begin{pgfscope}%
\pgfsys@transformshift{0.755119in}{1.302275in}%
\pgfsys@useobject{currentmarker}{}%
\end{pgfscope}%
\begin{pgfscope}%
\pgfsys@transformshift{0.728286in}{1.815927in}%
\pgfsys@useobject{currentmarker}{}%
\end{pgfscope}%
\begin{pgfscope}%
\pgfsys@transformshift{0.747500in}{1.381361in}%
\pgfsys@useobject{currentmarker}{}%
\end{pgfscope}%
\begin{pgfscope}%
\pgfsys@transformshift{0.717725in}{2.564130in}%
\pgfsys@useobject{currentmarker}{}%
\end{pgfscope}%
\begin{pgfscope}%
\pgfsys@transformshift{0.740008in}{1.584655in}%
\pgfsys@useobject{currentmarker}{}%
\end{pgfscope}%
\begin{pgfscope}%
\pgfsys@transformshift{3.691063in}{0.698193in}%
\pgfsys@useobject{currentmarker}{}%
\end{pgfscope}%
\begin{pgfscope}%
\pgfsys@transformshift{1.097731in}{0.762742in}%
\pgfsys@useobject{currentmarker}{}%
\end{pgfscope}%
\begin{pgfscope}%
\pgfsys@transformshift{0.740797in}{1.572352in}%
\pgfsys@useobject{currentmarker}{}%
\end{pgfscope}%
\begin{pgfscope}%
\pgfsys@transformshift{0.858076in}{0.976844in}%
\pgfsys@useobject{currentmarker}{}%
\end{pgfscope}%
\begin{pgfscope}%
\pgfsys@transformshift{0.717721in}{2.552563in}%
\pgfsys@useobject{currentmarker}{}%
\end{pgfscope}%
\begin{pgfscope}%
\pgfsys@transformshift{0.703669in}{3.580375in}%
\pgfsys@useobject{currentmarker}{}%
\end{pgfscope}%
\begin{pgfscope}%
\pgfsys@transformshift{1.071142in}{0.766933in}%
\pgfsys@useobject{currentmarker}{}%
\end{pgfscope}%
\begin{pgfscope}%
\pgfsys@transformshift{0.718712in}{2.318612in}%
\pgfsys@useobject{currentmarker}{}%
\end{pgfscope}%
\begin{pgfscope}%
\pgfsys@transformshift{3.717607in}{0.694013in}%
\pgfsys@useobject{currentmarker}{}%
\end{pgfscope}%
\begin{pgfscope}%
\pgfsys@transformshift{0.761956in}{1.275794in}%
\pgfsys@useobject{currentmarker}{}%
\end{pgfscope}%
\begin{pgfscope}%
\pgfsys@transformshift{1.577338in}{0.748397in}%
\pgfsys@useobject{currentmarker}{}%
\end{pgfscope}%
\begin{pgfscope}%
\pgfsys@transformshift{0.829378in}{1.058593in}%
\pgfsys@useobject{currentmarker}{}%
\end{pgfscope}%
\begin{pgfscope}%
\pgfsys@transformshift{0.760939in}{1.292936in}%
\pgfsys@useobject{currentmarker}{}%
\end{pgfscope}%
\begin{pgfscope}%
\pgfsys@transformshift{0.733582in}{1.692556in}%
\pgfsys@useobject{currentmarker}{}%
\end{pgfscope}%
\begin{pgfscope}%
\pgfsys@transformshift{0.770952in}{1.165703in}%
\pgfsys@useobject{currentmarker}{}%
\end{pgfscope}%
\begin{pgfscope}%
\pgfsys@transformshift{0.755119in}{1.302275in}%
\pgfsys@useobject{currentmarker}{}%
\end{pgfscope}%
\begin{pgfscope}%
\pgfsys@transformshift{0.829378in}{1.058593in}%
\pgfsys@useobject{currentmarker}{}%
\end{pgfscope}%
\begin{pgfscope}%
\pgfsys@transformshift{0.738103in}{1.566063in}%
\pgfsys@useobject{currentmarker}{}%
\end{pgfscope}%
\begin{pgfscope}%
\pgfsys@transformshift{0.757516in}{1.286210in}%
\pgfsys@useobject{currentmarker}{}%
\end{pgfscope}%
\begin{pgfscope}%
\pgfsys@transformshift{0.764491in}{1.230864in}%
\pgfsys@useobject{currentmarker}{}%
\end{pgfscope}%
\begin{pgfscope}%
\pgfsys@transformshift{1.322880in}{0.754149in}%
\pgfsys@useobject{currentmarker}{}%
\end{pgfscope}%
\begin{pgfscope}%
\pgfsys@transformshift{3.637762in}{0.701293in}%
\pgfsys@useobject{currentmarker}{}%
\end{pgfscope}%
\begin{pgfscope}%
\pgfsys@transformshift{0.816587in}{1.061960in}%
\pgfsys@useobject{currentmarker}{}%
\end{pgfscope}%
\begin{pgfscope}%
\pgfsys@transformshift{0.831546in}{0.995642in}%
\pgfsys@useobject{currentmarker}{}%
\end{pgfscope}%
\begin{pgfscope}%
\pgfsys@transformshift{1.158312in}{0.760421in}%
\pgfsys@useobject{currentmarker}{}%
\end{pgfscope}%
\begin{pgfscope}%
\pgfsys@transformshift{0.724158in}{1.873168in}%
\pgfsys@useobject{currentmarker}{}%
\end{pgfscope}%
\begin{pgfscope}%
\pgfsys@transformshift{0.746004in}{1.403857in}%
\pgfsys@useobject{currentmarker}{}%
\end{pgfscope}%
\begin{pgfscope}%
\pgfsys@transformshift{0.763584in}{1.235519in}%
\pgfsys@useobject{currentmarker}{}%
\end{pgfscope}%
\begin{pgfscope}%
\pgfsys@transformshift{0.745963in}{1.403981in}%
\pgfsys@useobject{currentmarker}{}%
\end{pgfscope}%
\begin{pgfscope}%
\pgfsys@transformshift{0.746552in}{1.394128in}%
\pgfsys@useobject{currentmarker}{}%
\end{pgfscope}%
\begin{pgfscope}%
\pgfsys@transformshift{3.310768in}{0.700522in}%
\pgfsys@useobject{currentmarker}{}%
\end{pgfscope}%
\begin{pgfscope}%
\pgfsys@transformshift{0.761177in}{1.264476in}%
\pgfsys@useobject{currentmarker}{}%
\end{pgfscope}%
\begin{pgfscope}%
\pgfsys@transformshift{0.779454in}{1.138820in}%
\pgfsys@useobject{currentmarker}{}%
\end{pgfscope}%
\begin{pgfscope}%
\pgfsys@transformshift{0.765536in}{1.212186in}%
\pgfsys@useobject{currentmarker}{}%
\end{pgfscope}%
\begin{pgfscope}%
\pgfsys@transformshift{0.770575in}{1.148701in}%
\pgfsys@useobject{currentmarker}{}%
\end{pgfscope}%
\begin{pgfscope}%
\pgfsys@transformshift{0.740471in}{1.540765in}%
\pgfsys@useobject{currentmarker}{}%
\end{pgfscope}%
\begin{pgfscope}%
\pgfsys@transformshift{0.742644in}{1.390871in}%
\pgfsys@useobject{currentmarker}{}%
\end{pgfscope}%
\begin{pgfscope}%
\pgfsys@transformshift{0.755402in}{1.299389in}%
\pgfsys@useobject{currentmarker}{}%
\end{pgfscope}%
\begin{pgfscope}%
\pgfsys@transformshift{3.236450in}{0.706811in}%
\pgfsys@useobject{currentmarker}{}%
\end{pgfscope}%
\begin{pgfscope}%
\pgfsys@transformshift{0.731615in}{1.718057in}%
\pgfsys@useobject{currentmarker}{}%
\end{pgfscope}%
\begin{pgfscope}%
\pgfsys@transformshift{0.750721in}{1.377509in}%
\pgfsys@useobject{currentmarker}{}%
\end{pgfscope}%
\begin{pgfscope}%
\pgfsys@transformshift{0.719947in}{2.056985in}%
\pgfsys@useobject{currentmarker}{}%
\end{pgfscope}%
\begin{pgfscope}%
\pgfsys@transformshift{1.144138in}{0.760401in}%
\pgfsys@useobject{currentmarker}{}%
\end{pgfscope}%
\begin{pgfscope}%
\pgfsys@transformshift{3.957078in}{0.690206in}%
\pgfsys@useobject{currentmarker}{}%
\end{pgfscope}%
\begin{pgfscope}%
\pgfsys@transformshift{0.709224in}{3.321499in}%
\pgfsys@useobject{currentmarker}{}%
\end{pgfscope}%
\begin{pgfscope}%
\pgfsys@transformshift{0.789313in}{1.090124in}%
\pgfsys@useobject{currentmarker}{}%
\end{pgfscope}%
\begin{pgfscope}%
\pgfsys@transformshift{0.740515in}{1.540129in}%
\pgfsys@useobject{currentmarker}{}%
\end{pgfscope}%
\begin{pgfscope}%
\pgfsys@transformshift{0.730038in}{1.775867in}%
\pgfsys@useobject{currentmarker}{}%
\end{pgfscope}%
\begin{pgfscope}%
\pgfsys@transformshift{1.096837in}{0.762701in}%
\pgfsys@useobject{currentmarker}{}%
\end{pgfscope}%
\begin{pgfscope}%
\pgfsys@transformshift{0.741102in}{1.405068in}%
\pgfsys@useobject{currentmarker}{}%
\end{pgfscope}%
\begin{pgfscope}%
\pgfsys@transformshift{2.376069in}{0.721870in}%
\pgfsys@useobject{currentmarker}{}%
\end{pgfscope}%
\begin{pgfscope}%
\pgfsys@transformshift{0.735574in}{1.634971in}%
\pgfsys@useobject{currentmarker}{}%
\end{pgfscope}%
\begin{pgfscope}%
\pgfsys@transformshift{0.760043in}{1.278793in}%
\pgfsys@useobject{currentmarker}{}%
\end{pgfscope}%
\begin{pgfscope}%
\pgfsys@transformshift{0.754784in}{1.317117in}%
\pgfsys@useobject{currentmarker}{}%
\end{pgfscope}%
\begin{pgfscope}%
\pgfsys@transformshift{0.730179in}{1.772193in}%
\pgfsys@useobject{currentmarker}{}%
\end{pgfscope}%
\begin{pgfscope}%
\pgfsys@transformshift{0.770575in}{1.148701in}%
\pgfsys@useobject{currentmarker}{}%
\end{pgfscope}%
\begin{pgfscope}%
\pgfsys@transformshift{1.018091in}{0.803361in}%
\pgfsys@useobject{currentmarker}{}%
\end{pgfscope}%
\begin{pgfscope}%
\pgfsys@transformshift{2.203012in}{0.726277in}%
\pgfsys@useobject{currentmarker}{}%
\end{pgfscope}%
\begin{pgfscope}%
\pgfsys@transformshift{0.763855in}{1.240586in}%
\pgfsys@useobject{currentmarker}{}%
\end{pgfscope}%
\begin{pgfscope}%
\pgfsys@transformshift{0.766842in}{1.200623in}%
\pgfsys@useobject{currentmarker}{}%
\end{pgfscope}%
\begin{pgfscope}%
\pgfsys@transformshift{0.708182in}{2.516637in}%
\pgfsys@useobject{currentmarker}{}%
\end{pgfscope}%
\begin{pgfscope}%
\pgfsys@transformshift{0.770575in}{1.148701in}%
\pgfsys@useobject{currentmarker}{}%
\end{pgfscope}%
\begin{pgfscope}%
\pgfsys@transformshift{0.764690in}{1.221460in}%
\pgfsys@useobject{currentmarker}{}%
\end{pgfscope}%
\begin{pgfscope}%
\pgfsys@transformshift{3.233861in}{0.706148in}%
\pgfsys@useobject{currentmarker}{}%
\end{pgfscope}%
\begin{pgfscope}%
\pgfsys@transformshift{0.729305in}{1.709215in}%
\pgfsys@useobject{currentmarker}{}%
\end{pgfscope}%
\begin{pgfscope}%
\pgfsys@transformshift{0.735880in}{1.634812in}%
\pgfsys@useobject{currentmarker}{}%
\end{pgfscope}%
\begin{pgfscope}%
\pgfsys@transformshift{0.698910in}{3.661788in}%
\pgfsys@useobject{currentmarker}{}%
\end{pgfscope}%
\begin{pgfscope}%
\pgfsys@transformshift{1.052902in}{0.777195in}%
\pgfsys@useobject{currentmarker}{}%
\end{pgfscope}%
\begin{pgfscope}%
\pgfsys@transformshift{0.717159in}{2.440212in}%
\pgfsys@useobject{currentmarker}{}%
\end{pgfscope}%
\begin{pgfscope}%
\pgfsys@transformshift{3.729259in}{0.696128in}%
\pgfsys@useobject{currentmarker}{}%
\end{pgfscope}%
\begin{pgfscope}%
\pgfsys@transformshift{3.308543in}{0.700863in}%
\pgfsys@useobject{currentmarker}{}%
\end{pgfscope}%
\begin{pgfscope}%
\pgfsys@transformshift{0.708182in}{2.516637in}%
\pgfsys@useobject{currentmarker}{}%
\end{pgfscope}%
\begin{pgfscope}%
\pgfsys@transformshift{3.606561in}{0.697452in}%
\pgfsys@useobject{currentmarker}{}%
\end{pgfscope}%
\begin{pgfscope}%
\pgfsys@transformshift{0.754748in}{1.317542in}%
\pgfsys@useobject{currentmarker}{}%
\end{pgfscope}%
\begin{pgfscope}%
\pgfsys@transformshift{0.747497in}{1.381404in}%
\pgfsys@useobject{currentmarker}{}%
\end{pgfscope}%
\begin{pgfscope}%
\pgfsys@transformshift{3.592753in}{0.700818in}%
\pgfsys@useobject{currentmarker}{}%
\end{pgfscope}%
\begin{pgfscope}%
\pgfsys@transformshift{0.726743in}{1.698359in}%
\pgfsys@useobject{currentmarker}{}%
\end{pgfscope}%
\begin{pgfscope}%
\pgfsys@transformshift{5.102941in}{0.679391in}%
\pgfsys@useobject{currentmarker}{}%
\end{pgfscope}%
\end{pgfscope}%
\begin{pgfscope}%
\pgfpathrectangle{\pgfqpoint{0.688192in}{0.670138in}}{\pgfqpoint{7.111808in}{5.129862in}}%
\pgfusepath{clip}%
\pgfsetbuttcap%
\pgfsetroundjoin%
\definecolor{currentfill}{rgb}{0.839216,0.152941,0.156863}%
\pgfsetfillcolor{currentfill}%
\pgfsetlinewidth{1.003750pt}%
\definecolor{currentstroke}{rgb}{0.839216,0.152941,0.156863}%
\pgfsetstrokecolor{currentstroke}%
\pgfsetdash{}{0pt}%
\pgfsys@defobject{currentmarker}{\pgfqpoint{-0.041084in}{-0.041084in}}{\pgfqpoint{0.041084in}{0.041084in}}{%
\pgfpathmoveto{\pgfqpoint{0.000000in}{-0.041084in}}%
\pgfpathcurveto{\pgfqpoint{0.010896in}{-0.041084in}}{\pgfqpoint{0.021346in}{-0.036755in}}{\pgfqpoint{0.029051in}{-0.029051in}}%
\pgfpathcurveto{\pgfqpoint{0.036755in}{-0.021346in}}{\pgfqpoint{0.041084in}{-0.010896in}}{\pgfqpoint{0.041084in}{0.000000in}}%
\pgfpathcurveto{\pgfqpoint{0.041084in}{0.010896in}}{\pgfqpoint{0.036755in}{0.021346in}}{\pgfqpoint{0.029051in}{0.029051in}}%
\pgfpathcurveto{\pgfqpoint{0.021346in}{0.036755in}}{\pgfqpoint{0.010896in}{0.041084in}}{\pgfqpoint{0.000000in}{0.041084in}}%
\pgfpathcurveto{\pgfqpoint{-0.010896in}{0.041084in}}{\pgfqpoint{-0.021346in}{0.036755in}}{\pgfqpoint{-0.029051in}{0.029051in}}%
\pgfpathcurveto{\pgfqpoint{-0.036755in}{0.021346in}}{\pgfqpoint{-0.041084in}{0.010896in}}{\pgfqpoint{-0.041084in}{0.000000in}}%
\pgfpathcurveto{\pgfqpoint{-0.041084in}{-0.010896in}}{\pgfqpoint{-0.036755in}{-0.021346in}}{\pgfqpoint{-0.029051in}{-0.029051in}}%
\pgfpathcurveto{\pgfqpoint{-0.021346in}{-0.036755in}}{\pgfqpoint{-0.010896in}{-0.041084in}}{\pgfqpoint{0.000000in}{-0.041084in}}%
\pgfpathlineto{\pgfqpoint{0.000000in}{-0.041084in}}%
\pgfpathclose%
\pgfusepath{stroke,fill}%
}%
\begin{pgfscope}%
\pgfsys@transformshift{1.050681in}{0.782877in}%
\pgfsys@useobject{currentmarker}{}%
\end{pgfscope}%
\begin{pgfscope}%
\pgfsys@transformshift{1.329451in}{1.187355in}%
\pgfsys@useobject{currentmarker}{}%
\end{pgfscope}%
\begin{pgfscope}%
\pgfsys@transformshift{3.617724in}{0.808912in}%
\pgfsys@useobject{currentmarker}{}%
\end{pgfscope}%
\begin{pgfscope}%
\pgfsys@transformshift{1.509516in}{0.760365in}%
\pgfsys@useobject{currentmarker}{}%
\end{pgfscope}%
\begin{pgfscope}%
\pgfsys@transformshift{2.927418in}{0.758622in}%
\pgfsys@useobject{currentmarker}{}%
\end{pgfscope}%
\begin{pgfscope}%
\pgfsys@transformshift{0.708182in}{2.516637in}%
\pgfsys@useobject{currentmarker}{}%
\end{pgfscope}%
\begin{pgfscope}%
\pgfsys@transformshift{2.294593in}{0.760365in}%
\pgfsys@useobject{currentmarker}{}%
\end{pgfscope}%
\begin{pgfscope}%
\pgfsys@transformshift{0.708182in}{2.516637in}%
\pgfsys@useobject{currentmarker}{}%
\end{pgfscope}%
\begin{pgfscope}%
\pgfsys@transformshift{4.205305in}{0.702275in}%
\pgfsys@useobject{currentmarker}{}%
\end{pgfscope}%
\begin{pgfscope}%
\pgfsys@transformshift{1.521082in}{1.181628in}%
\pgfsys@useobject{currentmarker}{}%
\end{pgfscope}%
\begin{pgfscope}%
\pgfsys@transformshift{1.543908in}{0.760365in}%
\pgfsys@useobject{currentmarker}{}%
\end{pgfscope}%
\begin{pgfscope}%
\pgfsys@transformshift{2.761865in}{0.761944in}%
\pgfsys@useobject{currentmarker}{}%
\end{pgfscope}%
\begin{pgfscope}%
\pgfsys@transformshift{0.751698in}{1.359672in}%
\pgfsys@useobject{currentmarker}{}%
\end{pgfscope}%
\begin{pgfscope}%
\pgfsys@transformshift{1.892499in}{4.580764in}%
\pgfsys@useobject{currentmarker}{}%
\end{pgfscope}%
\begin{pgfscope}%
\pgfsys@transformshift{1.596709in}{0.760365in}%
\pgfsys@useobject{currentmarker}{}%
\end{pgfscope}%
\begin{pgfscope}%
\pgfsys@transformshift{1.375497in}{1.141330in}%
\pgfsys@useobject{currentmarker}{}%
\end{pgfscope}%
\begin{pgfscope}%
\pgfsys@transformshift{2.895425in}{0.757664in}%
\pgfsys@useobject{currentmarker}{}%
\end{pgfscope}%
\begin{pgfscope}%
\pgfsys@transformshift{0.723046in}{4.433103in}%
\pgfsys@useobject{currentmarker}{}%
\end{pgfscope}%
\begin{pgfscope}%
\pgfsys@transformshift{3.691441in}{0.698225in}%
\pgfsys@useobject{currentmarker}{}%
\end{pgfscope}%
\begin{pgfscope}%
\pgfsys@transformshift{2.373971in}{0.770351in}%
\pgfsys@useobject{currentmarker}{}%
\end{pgfscope}%
\begin{pgfscope}%
\pgfsys@transformshift{0.719947in}{2.056985in}%
\pgfsys@useobject{currentmarker}{}%
\end{pgfscope}%
\begin{pgfscope}%
\pgfsys@transformshift{1.591285in}{0.760365in}%
\pgfsys@useobject{currentmarker}{}%
\end{pgfscope}%
\begin{pgfscope}%
\pgfsys@transformshift{0.871077in}{4.215304in}%
\pgfsys@useobject{currentmarker}{}%
\end{pgfscope}%
\begin{pgfscope}%
\pgfsys@transformshift{3.811388in}{0.750642in}%
\pgfsys@useobject{currentmarker}{}%
\end{pgfscope}%
\begin{pgfscope}%
\pgfsys@transformshift{0.737495in}{1.636902in}%
\pgfsys@useobject{currentmarker}{}%
\end{pgfscope}%
\begin{pgfscope}%
\pgfsys@transformshift{1.881814in}{0.790413in}%
\pgfsys@useobject{currentmarker}{}%
\end{pgfscope}%
\begin{pgfscope}%
\pgfsys@transformshift{1.576114in}{1.159187in}%
\pgfsys@useobject{currentmarker}{}%
\end{pgfscope}%
\begin{pgfscope}%
\pgfsys@transformshift{2.194335in}{0.760491in}%
\pgfsys@useobject{currentmarker}{}%
\end{pgfscope}%
\begin{pgfscope}%
\pgfsys@transformshift{4.188310in}{0.703577in}%
\pgfsys@useobject{currentmarker}{}%
\end{pgfscope}%
\begin{pgfscope}%
\pgfsys@transformshift{0.726743in}{1.698359in}%
\pgfsys@useobject{currentmarker}{}%
\end{pgfscope}%
\begin{pgfscope}%
\pgfsys@transformshift{1.697392in}{0.760365in}%
\pgfsys@useobject{currentmarker}{}%
\end{pgfscope}%
\end{pgfscope}%
\begin{pgfscope}%
\pgfpathrectangle{\pgfqpoint{0.688192in}{0.670138in}}{\pgfqpoint{7.111808in}{5.129862in}}%
\pgfusepath{clip}%
\pgfsetbuttcap%
\pgfsetmiterjoin%
\definecolor{currentfill}{rgb}{0.501961,0.501961,0.501961}%
\pgfsetfillcolor{currentfill}%
\pgfsetfillopacity{0.500000}%
\pgfsetlinewidth{1.003750pt}%
\definecolor{currentstroke}{rgb}{0.501961,0.501961,0.501961}%
\pgfsetstrokecolor{currentstroke}%
\pgfsetstrokeopacity{0.500000}%
\pgfsetdash{}{0pt}%
\pgfpathmoveto{\pgfqpoint{0.688192in}{5.255087in}}%
\pgfpathlineto{\pgfqpoint{0.691952in}{4.116925in}}%
\pgfpathlineto{\pgfqpoint{0.693069in}{3.535014in}}%
\pgfpathlineto{\pgfqpoint{0.699877in}{3.427816in}}%
\pgfpathlineto{\pgfqpoint{0.703696in}{2.711585in}}%
\pgfpathlineto{\pgfqpoint{0.708182in}{2.516637in}}%
\pgfpathlineto{\pgfqpoint{0.710424in}{2.515490in}}%
\pgfpathlineto{\pgfqpoint{0.715926in}{2.277179in}}%
\pgfpathlineto{\pgfqpoint{0.716194in}{2.143247in}}%
\pgfpathlineto{\pgfqpoint{0.719167in}{1.998136in}}%
\pgfpathlineto{\pgfqpoint{0.721104in}{1.962980in}}%
\pgfpathlineto{\pgfqpoint{0.722061in}{1.931704in}}%
\pgfpathlineto{\pgfqpoint{0.722141in}{1.928466in}}%
\pgfpathlineto{\pgfqpoint{0.722511in}{1.733351in}}%
\pgfpathlineto{\pgfqpoint{0.722511in}{1.733351in}}%
\pgfpathlineto{\pgfqpoint{0.726743in}{1.698359in}}%
\pgfpathlineto{\pgfqpoint{0.730789in}{1.657616in}}%
\pgfpathlineto{\pgfqpoint{0.732621in}{1.495185in}}%
\pgfpathlineto{\pgfqpoint{0.732629in}{1.494803in}}%
\pgfpathlineto{\pgfqpoint{0.735321in}{1.444486in}}%
\pgfpathlineto{\pgfqpoint{0.735501in}{1.401992in}}%
\pgfpathlineto{\pgfqpoint{0.740362in}{1.381421in}}%
\pgfpathlineto{\pgfqpoint{0.747329in}{1.376655in}}%
\pgfpathlineto{\pgfqpoint{0.749794in}{1.239967in}}%
\pgfpathlineto{\pgfqpoint{0.755774in}{1.206744in}}%
\pgfpathlineto{\pgfqpoint{0.766804in}{1.200574in}}%
\pgfpathlineto{\pgfqpoint{0.770575in}{1.148701in}}%
\pgfpathlineto{\pgfqpoint{0.774443in}{1.139626in}}%
\pgfpathlineto{\pgfqpoint{0.777060in}{1.132809in}}%
\pgfpathlineto{\pgfqpoint{0.782053in}{1.114816in}}%
\pgfpathlineto{\pgfqpoint{0.782443in}{1.113980in}}%
\pgfpathlineto{\pgfqpoint{0.782559in}{1.113067in}}%
\pgfpathlineto{\pgfqpoint{0.789241in}{1.084530in}}%
\pgfpathlineto{\pgfqpoint{0.789763in}{1.079638in}}%
\pgfpathlineto{\pgfqpoint{0.790083in}{1.073959in}}%
\pgfpathlineto{\pgfqpoint{0.794194in}{1.063688in}}%
\pgfpathlineto{\pgfqpoint{0.812030in}{1.034334in}}%
\pgfpathlineto{\pgfqpoint{0.830999in}{0.993434in}}%
\pgfpathlineto{\pgfqpoint{0.856742in}{0.964253in}}%
\pgfpathlineto{\pgfqpoint{0.880316in}{0.954452in}}%
\pgfpathlineto{\pgfqpoint{0.881277in}{0.941511in}}%
\pgfpathlineto{\pgfqpoint{0.882669in}{0.918939in}}%
\pgfpathlineto{\pgfqpoint{0.885952in}{0.914664in}}%
\pgfpathlineto{\pgfqpoint{0.886403in}{0.910894in}}%
\pgfpathlineto{\pgfqpoint{0.886861in}{0.910810in}}%
\pgfpathlineto{\pgfqpoint{0.895312in}{0.907258in}}%
\pgfpathlineto{\pgfqpoint{0.899610in}{0.900126in}}%
\pgfpathlineto{\pgfqpoint{0.901363in}{0.897632in}}%
\pgfpathlineto{\pgfqpoint{0.928243in}{0.882681in}}%
\pgfpathlineto{\pgfqpoint{0.931763in}{0.874747in}}%
\pgfpathlineto{\pgfqpoint{0.931867in}{0.866217in}}%
\pgfpathlineto{\pgfqpoint{0.937645in}{0.864808in}}%
\pgfpathlineto{\pgfqpoint{0.951297in}{0.850481in}}%
\pgfpathlineto{\pgfqpoint{0.952302in}{0.849081in}}%
\pgfpathlineto{\pgfqpoint{0.957757in}{0.849024in}}%
\pgfpathlineto{\pgfqpoint{0.958756in}{0.843680in}}%
\pgfpathlineto{\pgfqpoint{1.012547in}{0.829947in}}%
\pgfpathlineto{\pgfqpoint{1.016759in}{0.803132in}}%
\pgfpathlineto{\pgfqpoint{1.019140in}{0.801064in}}%
\pgfpathlineto{\pgfqpoint{1.019151in}{0.801064in}}%
\pgfpathlineto{\pgfqpoint{1.026002in}{0.799177in}}%
\pgfpathlineto{\pgfqpoint{1.027669in}{0.797472in}}%
\pgfpathlineto{\pgfqpoint{1.044526in}{0.783805in}}%
\pgfpathlineto{\pgfqpoint{1.045306in}{0.782620in}}%
\pgfpathlineto{\pgfqpoint{1.050628in}{0.780895in}}%
\pgfpathlineto{\pgfqpoint{1.050667in}{0.780323in}}%
\pgfpathlineto{\pgfqpoint{1.052060in}{0.777139in}}%
\pgfpathlineto{\pgfqpoint{1.056340in}{0.775974in}}%
\pgfpathlineto{\pgfqpoint{1.058884in}{0.775846in}}%
\pgfpathlineto{\pgfqpoint{1.069116in}{0.770892in}}%
\pgfpathlineto{\pgfqpoint{1.070830in}{0.766970in}}%
\pgfpathlineto{\pgfqpoint{1.071092in}{0.766931in}}%
\pgfpathlineto{\pgfqpoint{1.090088in}{0.766654in}}%
\pgfpathlineto{\pgfqpoint{1.091394in}{0.763336in}}%
\pgfpathlineto{\pgfqpoint{1.094048in}{0.762662in}}%
\pgfpathlineto{\pgfqpoint{1.101880in}{0.761761in}}%
\pgfpathlineto{\pgfqpoint{1.106601in}{0.761527in}}%
\pgfpathlineto{\pgfqpoint{1.120660in}{0.760704in}}%
\pgfpathlineto{\pgfqpoint{1.139895in}{0.760442in}}%
\pgfpathlineto{\pgfqpoint{1.141797in}{0.760441in}}%
\pgfpathlineto{\pgfqpoint{1.142999in}{0.760397in}}%
\pgfpathlineto{\pgfqpoint{1.163954in}{0.760365in}}%
\pgfpathlineto{\pgfqpoint{1.259457in}{0.757136in}}%
\pgfpathlineto{\pgfqpoint{1.278130in}{0.755206in}}%
\pgfpathlineto{\pgfqpoint{1.307540in}{0.754621in}}%
\pgfpathlineto{\pgfqpoint{1.308402in}{0.754533in}}%
\pgfpathlineto{\pgfqpoint{1.320633in}{0.753865in}}%
\pgfpathlineto{\pgfqpoint{1.341138in}{0.753397in}}%
\pgfpathlineto{\pgfqpoint{1.344495in}{0.753033in}}%
\pgfpathlineto{\pgfqpoint{1.351667in}{0.752560in}}%
\pgfpathlineto{\pgfqpoint{1.389197in}{0.752350in}}%
\pgfpathlineto{\pgfqpoint{1.400715in}{0.752063in}}%
\pgfpathlineto{\pgfqpoint{1.414589in}{0.751918in}}%
\pgfpathlineto{\pgfqpoint{1.417824in}{0.751904in}}%
\pgfpathlineto{\pgfqpoint{1.418761in}{0.751837in}}%
\pgfpathlineto{\pgfqpoint{1.419139in}{0.751796in}}%
\pgfpathlineto{\pgfqpoint{1.419462in}{0.751760in}}%
\pgfpathlineto{\pgfqpoint{1.421141in}{0.751622in}}%
\pgfpathlineto{\pgfqpoint{1.422574in}{0.751405in}}%
\pgfpathlineto{\pgfqpoint{1.441321in}{0.749466in}}%
\pgfpathlineto{\pgfqpoint{1.448644in}{0.748886in}}%
\pgfpathlineto{\pgfqpoint{1.528667in}{0.747888in}}%
\pgfpathlineto{\pgfqpoint{1.605641in}{0.747867in}}%
\pgfpathlineto{\pgfqpoint{1.606745in}{0.746981in}}%
\pgfpathlineto{\pgfqpoint{1.610058in}{0.746614in}}%
\pgfpathlineto{\pgfqpoint{1.622014in}{0.746220in}}%
\pgfpathlineto{\pgfqpoint{1.625190in}{0.745498in}}%
\pgfpathlineto{\pgfqpoint{1.628853in}{0.745202in}}%
\pgfpathlineto{\pgfqpoint{1.631842in}{0.744948in}}%
\pgfpathlineto{\pgfqpoint{1.666551in}{0.744594in}}%
\pgfpathlineto{\pgfqpoint{1.678503in}{0.743559in}}%
\pgfpathlineto{\pgfqpoint{1.678976in}{0.743508in}}%
\pgfpathlineto{\pgfqpoint{1.678976in}{0.743508in}}%
\pgfpathlineto{\pgfqpoint{1.679156in}{0.743491in}}%
\pgfpathlineto{\pgfqpoint{1.679564in}{0.743446in}}%
\pgfpathlineto{\pgfqpoint{1.685511in}{0.742202in}}%
\pgfpathlineto{\pgfqpoint{1.685511in}{0.742202in}}%
\pgfpathlineto{\pgfqpoint{1.702822in}{0.741854in}}%
\pgfpathlineto{\pgfqpoint{1.707667in}{0.740400in}}%
\pgfpathlineto{\pgfqpoint{1.708990in}{0.740247in}}%
\pgfpathlineto{\pgfqpoint{1.713082in}{0.739976in}}%
\pgfpathlineto{\pgfqpoint{1.759569in}{0.739880in}}%
\pgfpathlineto{\pgfqpoint{1.759569in}{0.739880in}}%
\pgfpathlineto{\pgfqpoint{1.763752in}{0.739729in}}%
\pgfpathlineto{\pgfqpoint{1.775222in}{0.738819in}}%
\pgfpathlineto{\pgfqpoint{1.800509in}{0.738014in}}%
\pgfpathlineto{\pgfqpoint{1.840914in}{0.737153in}}%
\pgfpathlineto{\pgfqpoint{1.852165in}{0.736068in}}%
\pgfpathlineto{\pgfqpoint{1.877664in}{0.735418in}}%
\pgfpathlineto{\pgfqpoint{1.916797in}{0.734885in}}%
\pgfpathlineto{\pgfqpoint{1.920423in}{0.733745in}}%
\pgfpathlineto{\pgfqpoint{2.013663in}{0.731531in}}%
\pgfpathlineto{\pgfqpoint{2.176728in}{0.728600in}}%
\pgfpathlineto{\pgfqpoint{2.187729in}{0.726449in}}%
\pgfpathlineto{\pgfqpoint{2.188963in}{0.726384in}}%
\pgfpathlineto{\pgfqpoint{2.195614in}{0.726305in}}%
\pgfpathlineto{\pgfqpoint{2.196830in}{0.726204in}}%
\pgfpathlineto{\pgfqpoint{2.217170in}{0.725864in}}%
\pgfpathlineto{\pgfqpoint{2.331314in}{0.722875in}}%
\pgfpathlineto{\pgfqpoint{2.346054in}{0.721682in}}%
\pgfpathlineto{\pgfqpoint{2.382807in}{0.719754in}}%
\pgfpathlineto{\pgfqpoint{2.406189in}{0.719470in}}%
\pgfpathlineto{\pgfqpoint{2.443271in}{0.719418in}}%
\pgfpathlineto{\pgfqpoint{2.448194in}{0.718998in}}%
\pgfpathlineto{\pgfqpoint{2.464002in}{0.717470in}}%
\pgfpathlineto{\pgfqpoint{2.464440in}{0.717412in}}%
\pgfpathlineto{\pgfqpoint{2.497167in}{0.716312in}}%
\pgfpathlineto{\pgfqpoint{2.563540in}{0.715743in}}%
\pgfpathlineto{\pgfqpoint{2.628270in}{0.713086in}}%
\pgfpathlineto{\pgfqpoint{2.632948in}{0.712804in}}%
\pgfpathlineto{\pgfqpoint{2.869035in}{0.710469in}}%
\pgfpathlineto{\pgfqpoint{2.890311in}{0.707406in}}%
\pgfpathlineto{\pgfqpoint{2.934581in}{0.705830in}}%
\pgfpathlineto{\pgfqpoint{2.983194in}{0.704605in}}%
\pgfpathlineto{\pgfqpoint{3.241809in}{0.699660in}}%
\pgfpathlineto{\pgfqpoint{3.252817in}{0.698436in}}%
\pgfpathlineto{\pgfqpoint{3.523301in}{0.692124in}}%
\pgfpathlineto{\pgfqpoint{3.755809in}{0.689503in}}%
\pgfpathlineto{\pgfqpoint{3.791217in}{0.688607in}}%
\pgfpathlineto{\pgfqpoint{3.829576in}{0.687811in}}%
\pgfpathlineto{\pgfqpoint{3.953911in}{0.685576in}}%
\pgfpathlineto{\pgfqpoint{3.963043in}{0.685366in}}%
\pgfpathlineto{\pgfqpoint{4.123628in}{0.683245in}}%
\pgfpathlineto{\pgfqpoint{4.176025in}{0.682823in}}%
\pgfpathlineto{\pgfqpoint{4.182175in}{0.682726in}}%
\pgfpathlineto{\pgfqpoint{4.391441in}{0.682670in}}%
\pgfpathlineto{\pgfqpoint{4.391763in}{0.682661in}}%
\pgfpathlineto{\pgfqpoint{4.462854in}{0.682464in}}%
\pgfpathlineto{\pgfqpoint{4.549016in}{0.682007in}}%
\pgfpathlineto{\pgfqpoint{4.586127in}{0.681928in}}%
\pgfpathlineto{\pgfqpoint{4.603680in}{0.680832in}}%
\pgfpathlineto{\pgfqpoint{4.634758in}{0.680738in}}%
\pgfpathlineto{\pgfqpoint{4.661601in}{0.679515in}}%
\pgfpathlineto{\pgfqpoint{4.700040in}{0.679144in}}%
\pgfpathlineto{\pgfqpoint{4.708661in}{0.678566in}}%
\pgfpathlineto{\pgfqpoint{4.724535in}{0.678336in}}%
\pgfpathlineto{\pgfqpoint{4.737940in}{0.678102in}}%
\pgfpathlineto{\pgfqpoint{4.889518in}{0.677863in}}%
\pgfpathlineto{\pgfqpoint{4.899791in}{0.677682in}}%
\pgfpathlineto{\pgfqpoint{4.914884in}{0.677644in}}%
\pgfpathlineto{\pgfqpoint{4.940575in}{0.677405in}}%
\pgfpathlineto{\pgfqpoint{4.954913in}{0.677128in}}%
\pgfpathlineto{\pgfqpoint{4.970414in}{0.676799in}}%
\pgfpathlineto{\pgfqpoint{5.061952in}{0.676153in}}%
\pgfpathlineto{\pgfqpoint{5.112107in}{0.675923in}}%
\pgfpathlineto{\pgfqpoint{5.117964in}{0.675193in}}%
\pgfpathlineto{\pgfqpoint{5.136924in}{0.674382in}}%
\pgfpathlineto{\pgfqpoint{5.252737in}{0.674245in}}%
\pgfpathlineto{\pgfqpoint{5.252737in}{0.674245in}}%
\pgfpathlineto{\pgfqpoint{5.261268in}{0.673970in}}%
\pgfpathlineto{\pgfqpoint{5.338744in}{0.673524in}}%
\pgfpathlineto{\pgfqpoint{5.349847in}{0.673027in}}%
\pgfpathlineto{\pgfqpoint{5.507614in}{0.672879in}}%
\pgfpathlineto{\pgfqpoint{5.537814in}{0.672431in}}%
\pgfpathlineto{\pgfqpoint{5.595362in}{0.672311in}}%
\pgfpathlineto{\pgfqpoint{5.638666in}{0.671980in}}%
\pgfpathlineto{\pgfqpoint{5.696293in}{0.671853in}}%
\pgfpathlineto{\pgfqpoint{5.823193in}{0.670783in}}%
\pgfpathlineto{\pgfqpoint{5.988842in}{0.670138in}}%
\pgfpathlineto{\pgfqpoint{7.800000in}{0.756556in}}%
\pgfpathlineto{\pgfqpoint{7.617785in}{0.757265in}}%
\pgfpathlineto{\pgfqpoint{7.478196in}{0.758442in}}%
\pgfpathlineto{\pgfqpoint{7.414806in}{0.758582in}}%
\pgfpathlineto{\pgfqpoint{7.367172in}{0.758946in}}%
\pgfpathlineto{\pgfqpoint{7.303869in}{0.759078in}}%
\pgfpathlineto{\pgfqpoint{7.270649in}{0.759571in}}%
\pgfpathlineto{\pgfqpoint{7.097105in}{0.759733in}}%
\pgfpathlineto{\pgfqpoint{7.084892in}{0.760280in}}%
\pgfpathlineto{\pgfqpoint{6.999668in}{0.760771in}}%
\pgfpathlineto{\pgfqpoint{6.990284in}{0.761073in}}%
\pgfpathlineto{\pgfqpoint{6.990284in}{0.761073in}}%
\pgfpathlineto{\pgfqpoint{6.862890in}{0.761224in}}%
\pgfpathlineto{\pgfqpoint{6.842034in}{0.762116in}}%
\pgfpathlineto{\pgfqpoint{6.835591in}{0.762919in}}%
\pgfpathlineto{\pgfqpoint{6.780420in}{0.763173in}}%
\pgfpathlineto{\pgfqpoint{6.679729in}{0.763883in}}%
\pgfpathlineto{\pgfqpoint{6.662678in}{0.764245in}}%
\pgfpathlineto{\pgfqpoint{6.646906in}{0.764550in}}%
\pgfpathlineto{\pgfqpoint{6.618646in}{0.764812in}}%
\pgfpathlineto{\pgfqpoint{6.602044in}{0.764854in}}%
\pgfpathlineto{\pgfqpoint{6.590743in}{0.765053in}}%
\pgfpathlineto{\pgfqpoint{6.424008in}{0.765316in}}%
\pgfpathlineto{\pgfqpoint{6.409262in}{0.765573in}}%
\pgfpathlineto{\pgfqpoint{6.391800in}{0.765827in}}%
\pgfpathlineto{\pgfqpoint{6.382318in}{0.766462in}}%
\pgfpathlineto{\pgfqpoint{6.340034in}{0.766870in}}%
\pgfpathlineto{\pgfqpoint{6.310507in}{0.768216in}}%
\pgfpathlineto{\pgfqpoint{6.276321in}{0.768320in}}%
\pgfpathlineto{\pgfqpoint{6.257013in}{0.769525in}}%
\pgfpathlineto{\pgfqpoint{6.216191in}{0.769611in}}%
\pgfpathlineto{\pgfqpoint{6.121412in}{0.770115in}}%
\pgfpathlineto{\pgfqpoint{6.043212in}{0.770331in}}%
\pgfpathlineto{\pgfqpoint{6.042858in}{0.770340in}}%
\pgfpathlineto{\pgfqpoint{5.812666in}{0.770402in}}%
\pgfpathlineto{\pgfqpoint{5.805901in}{0.770509in}}%
\pgfpathlineto{\pgfqpoint{5.748264in}{0.770973in}}%
\pgfpathlineto{\pgfqpoint{5.571620in}{0.773307in}}%
\pgfpathlineto{\pgfqpoint{5.561575in}{0.773537in}}%
\pgfpathlineto{\pgfqpoint{5.424807in}{0.775996in}}%
\pgfpathlineto{\pgfqpoint{5.382612in}{0.776872in}}%
\pgfpathlineto{\pgfqpoint{5.343664in}{0.777857in}}%
\pgfpathlineto{\pgfqpoint{5.087904in}{0.780740in}}%
\pgfpathlineto{\pgfqpoint{4.790372in}{0.787683in}}%
\pgfpathlineto{\pgfqpoint{4.778264in}{0.789030in}}%
\pgfpathlineto{\pgfqpoint{4.493787in}{0.794470in}}%
\pgfpathlineto{\pgfqpoint{4.440312in}{0.795817in}}%
\pgfpathlineto{\pgfqpoint{4.391616in}{0.797551in}}%
\pgfpathlineto{\pgfqpoint{4.368212in}{0.800920in}}%
\pgfpathlineto{\pgfqpoint{4.108516in}{0.803488in}}%
\pgfpathlineto{\pgfqpoint{4.103370in}{0.803798in}}%
\pgfpathlineto{\pgfqpoint{4.032167in}{0.806721in}}%
\pgfpathlineto{\pgfqpoint{3.959157in}{0.807348in}}%
\pgfpathlineto{\pgfqpoint{3.923158in}{0.808557in}}%
\pgfpathlineto{\pgfqpoint{3.922676in}{0.808621in}}%
\pgfpathlineto{\pgfqpoint{3.905287in}{0.810301in}}%
\pgfpathlineto{\pgfqpoint{3.899872in}{0.810764in}}%
\pgfpathlineto{\pgfqpoint{3.859081in}{0.810821in}}%
\pgfpathlineto{\pgfqpoint{3.833361in}{0.811133in}}%
\pgfpathlineto{\pgfqpoint{3.792932in}{0.813254in}}%
\pgfpathlineto{\pgfqpoint{3.776719in}{0.814566in}}%
\pgfpathlineto{\pgfqpoint{3.651161in}{0.817855in}}%
\pgfpathlineto{\pgfqpoint{3.628787in}{0.818228in}}%
\pgfpathlineto{\pgfqpoint{3.627448in}{0.818339in}}%
\pgfpathlineto{\pgfqpoint{3.620132in}{0.818426in}}%
\pgfpathlineto{\pgfqpoint{3.618775in}{0.818497in}}%
\pgfpathlineto{\pgfqpoint{3.606674in}{0.820864in}}%
\pgfpathlineto{\pgfqpoint{3.427303in}{0.824088in}}%
\pgfpathlineto{\pgfqpoint{3.324738in}{0.826524in}}%
\pgfpathlineto{\pgfqpoint{3.320750in}{0.827778in}}%
\pgfpathlineto{\pgfqpoint{3.277703in}{0.828363in}}%
\pgfpathlineto{\pgfqpoint{3.249654in}{0.829079in}}%
\pgfpathlineto{\pgfqpoint{3.237278in}{0.830273in}}%
\pgfpathlineto{\pgfqpoint{3.192834in}{0.831220in}}%
\pgfpathlineto{\pgfqpoint{3.165018in}{0.832105in}}%
\pgfpathlineto{\pgfqpoint{3.152400in}{0.833106in}}%
\pgfpathlineto{\pgfqpoint{3.147799in}{0.833272in}}%
\pgfpathlineto{\pgfqpoint{3.147799in}{0.833272in}}%
\pgfpathlineto{\pgfqpoint{3.096664in}{0.833378in}}%
\pgfpathlineto{\pgfqpoint{3.092163in}{0.833675in}}%
\pgfpathlineto{\pgfqpoint{3.090707in}{0.833844in}}%
\pgfpathlineto{\pgfqpoint{3.085378in}{0.835443in}}%
\pgfpathlineto{\pgfqpoint{3.066335in}{0.835826in}}%
\pgfpathlineto{\pgfqpoint{3.066335in}{0.835826in}}%
\pgfpathlineto{\pgfqpoint{3.059794in}{0.837194in}}%
\pgfpathlineto{\pgfqpoint{3.059344in}{0.837244in}}%
\pgfpathlineto{\pgfqpoint{3.059147in}{0.837263in}}%
\pgfpathlineto{\pgfqpoint{3.059147in}{0.837263in}}%
\pgfpathlineto{\pgfqpoint{3.058627in}{0.837319in}}%
\pgfpathlineto{\pgfqpoint{3.045479in}{0.838457in}}%
\pgfpathlineto{\pgfqpoint{3.007299in}{0.838846in}}%
\pgfpathlineto{\pgfqpoint{3.004012in}{0.839126in}}%
\pgfpathlineto{\pgfqpoint{2.999983in}{0.839451in}}%
\pgfpathlineto{\pgfqpoint{2.996489in}{0.840246in}}%
\pgfpathlineto{\pgfqpoint{2.983337in}{0.840679in}}%
\pgfpathlineto{\pgfqpoint{2.979693in}{0.841083in}}%
\pgfpathlineto{\pgfqpoint{2.978478in}{0.842058in}}%
\pgfpathlineto{\pgfqpoint{2.893807in}{0.842080in}}%
\pgfpathlineto{\pgfqpoint{2.805781in}{0.843178in}}%
\pgfpathlineto{\pgfqpoint{2.797726in}{0.843817in}}%
\pgfpathlineto{\pgfqpoint{2.777105in}{0.845949in}}%
\pgfpathlineto{\pgfqpoint{2.775528in}{0.846189in}}%
\pgfpathlineto{\pgfqpoint{2.773681in}{0.846340in}}%
\pgfpathlineto{\pgfqpoint{2.773326in}{0.846379in}}%
\pgfpathlineto{\pgfqpoint{2.772911in}{0.846425in}}%
\pgfpathlineto{\pgfqpoint{2.771880in}{0.846499in}}%
\pgfpathlineto{\pgfqpoint{2.768321in}{0.846514in}}%
\pgfpathlineto{\pgfqpoint{2.753060in}{0.846673in}}%
\pgfpathlineto{\pgfqpoint{2.740390in}{0.846989in}}%
\pgfpathlineto{\pgfqpoint{2.699107in}{0.847220in}}%
\pgfpathlineto{\pgfqpoint{2.691218in}{0.847740in}}%
\pgfpathlineto{\pgfqpoint{2.687525in}{0.848141in}}%
\pgfpathlineto{\pgfqpoint{2.664970in}{0.848655in}}%
\pgfpathlineto{\pgfqpoint{2.651515in}{0.849390in}}%
\pgfpathlineto{\pgfqpoint{2.650567in}{0.849487in}}%
\pgfpathlineto{\pgfqpoint{2.618217in}{0.850131in}}%
\pgfpathlineto{\pgfqpoint{2.597676in}{0.852253in}}%
\pgfpathlineto{\pgfqpoint{2.492622in}{0.855806in}}%
\pgfpathlineto{\pgfqpoint{2.469573in}{0.855840in}}%
\pgfpathlineto{\pgfqpoint{2.468250in}{0.855889in}}%
\pgfpathlineto{\pgfqpoint{2.466157in}{0.855891in}}%
\pgfpathlineto{\pgfqpoint{2.444999in}{0.856179in}}%
\pgfpathlineto{\pgfqpoint{2.429535in}{0.857083in}}%
\pgfpathlineto{\pgfqpoint{2.424341in}{0.857341in}}%
\pgfpathlineto{\pgfqpoint{2.415726in}{0.858332in}}%
\pgfpathlineto{\pgfqpoint{2.412807in}{0.859073in}}%
\pgfpathlineto{\pgfqpoint{2.411370in}{0.862723in}}%
\pgfpathlineto{\pgfqpoint{2.390474in}{0.863028in}}%
\pgfpathlineto{\pgfqpoint{2.390186in}{0.863071in}}%
\pgfpathlineto{\pgfqpoint{2.388301in}{0.867385in}}%
\pgfpathlineto{\pgfqpoint{2.377046in}{0.872835in}}%
\pgfpathlineto{\pgfqpoint{2.374247in}{0.872975in}}%
\pgfpathlineto{\pgfqpoint{2.369539in}{0.874257in}}%
\pgfpathlineto{\pgfqpoint{2.368007in}{0.877760in}}%
\pgfpathlineto{\pgfqpoint{2.367964in}{0.878388in}}%
\pgfpathlineto{\pgfqpoint{2.362110in}{0.880286in}}%
\pgfpathlineto{\pgfqpoint{2.361252in}{0.881589in}}%
\pgfpathlineto{\pgfqpoint{2.342709in}{0.896623in}}%
\pgfpathlineto{\pgfqpoint{2.340875in}{0.898499in}}%
\pgfpathlineto{\pgfqpoint{2.333340in}{0.900575in}}%
\pgfpathlineto{\pgfqpoint{2.333327in}{0.900575in}}%
\pgfpathlineto{\pgfqpoint{2.330708in}{0.902849in}}%
\pgfpathlineto{\pgfqpoint{2.326075in}{0.932346in}}%
\pgfpathlineto{\pgfqpoint{2.266905in}{0.947452in}}%
\pgfpathlineto{\pgfqpoint{2.265806in}{0.953331in}}%
\pgfpathlineto{\pgfqpoint{2.259805in}{0.953394in}}%
\pgfpathlineto{\pgfqpoint{2.258700in}{0.954933in}}%
\pgfpathlineto{\pgfqpoint{2.243682in}{0.970693in}}%
\pgfpathlineto{\pgfqpoint{2.237327in}{0.972243in}}%
\pgfpathlineto{\pgfqpoint{2.237213in}{0.981626in}}%
\pgfpathlineto{\pgfqpoint{2.233341in}{0.990353in}}%
\pgfpathlineto{\pgfqpoint{2.203773in}{1.006799in}}%
\pgfpathlineto{\pgfqpoint{2.201845in}{1.009543in}}%
\pgfpathlineto{\pgfqpoint{2.197116in}{1.017387in}}%
\pgfpathlineto{\pgfqpoint{2.187820in}{1.021295in}}%
\pgfpathlineto{\pgfqpoint{2.187316in}{1.021388in}}%
\pgfpathlineto{\pgfqpoint{2.186821in}{1.025534in}}%
\pgfpathlineto{\pgfqpoint{2.183210in}{1.030237in}}%
\pgfpathlineto{\pgfqpoint{2.181678in}{1.055066in}}%
\pgfpathlineto{\pgfqpoint{2.180621in}{1.069301in}}%
\pgfpathlineto{\pgfqpoint{2.154690in}{1.080082in}}%
\pgfpathlineto{\pgfqpoint{2.126372in}{1.112181in}}%
\pgfpathlineto{\pgfqpoint{2.105507in}{1.157172in}}%
\pgfpathlineto{\pgfqpoint{2.085887in}{1.189461in}}%
\pgfpathlineto{\pgfqpoint{2.081364in}{1.200759in}}%
\pgfpathlineto{\pgfqpoint{2.081013in}{1.207005in}}%
\pgfpathlineto{\pgfqpoint{2.080438in}{1.212387in}}%
\pgfpathlineto{\pgfqpoint{2.073088in}{1.243778in}}%
\pgfpathlineto{\pgfqpoint{2.072961in}{1.244782in}}%
\pgfpathlineto{\pgfqpoint{2.072532in}{1.245701in}}%
\pgfpathlineto{\pgfqpoint{2.067039in}{1.265494in}}%
\pgfpathlineto{\pgfqpoint{2.064160in}{1.272993in}}%
\pgfpathlineto{\pgfqpoint{2.059905in}{1.282975in}}%
\pgfpathlineto{\pgfqpoint{2.055757in}{1.340035in}}%
\pgfpathlineto{\pgfqpoint{2.043625in}{1.346822in}}%
\pgfpathlineto{\pgfqpoint{2.037047in}{1.383368in}}%
\pgfpathlineto{\pgfqpoint{2.034335in}{1.533724in}}%
\pgfpathlineto{\pgfqpoint{2.026671in}{1.538967in}}%
\pgfpathlineto{\pgfqpoint{2.021324in}{1.561595in}}%
\pgfpathlineto{\pgfqpoint{2.021126in}{1.608339in}}%
\pgfpathlineto{\pgfqpoint{2.018165in}{1.663687in}}%
\pgfpathlineto{\pgfqpoint{2.018157in}{1.664108in}}%
\pgfpathlineto{\pgfqpoint{2.016142in}{1.842782in}}%
\pgfpathlineto{\pgfqpoint{2.011691in}{1.887598in}}%
\pgfpathlineto{\pgfqpoint{2.007036in}{1.926090in}}%
\pgfpathlineto{\pgfqpoint{2.007036in}{1.926090in}}%
\pgfpathlineto{\pgfqpoint{2.006628in}{2.140716in}}%
\pgfpathlineto{\pgfqpoint{2.006541in}{2.144278in}}%
\pgfpathlineto{\pgfqpoint{2.005488in}{2.178682in}}%
\pgfpathlineto{\pgfqpoint{2.003357in}{2.217353in}}%
\pgfpathlineto{\pgfqpoint{2.000087in}{2.376976in}}%
\pgfpathlineto{\pgfqpoint{1.999791in}{2.524301in}}%
\pgfpathlineto{\pgfqpoint{1.993739in}{2.786443in}}%
\pgfpathlineto{\pgfqpoint{1.991273in}{2.787704in}}%
\pgfpathlineto{\pgfqpoint{1.986338in}{3.002147in}}%
\pgfpathlineto{\pgfqpoint{1.982138in}{3.790001in}}%
\pgfpathlineto{\pgfqpoint{1.974650in}{3.907920in}}%
\pgfpathlineto{\pgfqpoint{1.973421in}{4.548021in}}%
\pgfpathlineto{\pgfqpoint{1.969284in}{5.800000in}}%
\pgfpathlineto{\pgfqpoint{0.688192in}{5.255087in}}%
\pgfpathclose%
\pgfusepath{stroke,fill}%
\end{pgfscope}%
\begin{pgfscope}%
\pgfpathrectangle{\pgfqpoint{0.688192in}{0.670138in}}{\pgfqpoint{7.111808in}{5.129862in}}%
\pgfusepath{clip}%
\pgfsetrectcap%
\pgfsetroundjoin%
\pgfsetlinewidth{0.803000pt}%
\definecolor{currentstroke}{rgb}{0.690196,0.690196,0.690196}%
\pgfsetstrokecolor{currentstroke}%
\pgfsetdash{}{0pt}%
\pgfpathmoveto{\pgfqpoint{1.129065in}{0.670138in}}%
\pgfpathlineto{\pgfqpoint{1.129065in}{5.800000in}}%
\pgfusepath{stroke}%
\end{pgfscope}%
\begin{pgfscope}%
\pgfsetbuttcap%
\pgfsetroundjoin%
\definecolor{currentfill}{rgb}{0.000000,0.000000,0.000000}%
\pgfsetfillcolor{currentfill}%
\pgfsetlinewidth{0.803000pt}%
\definecolor{currentstroke}{rgb}{0.000000,0.000000,0.000000}%
\pgfsetstrokecolor{currentstroke}%
\pgfsetdash{}{0pt}%
\pgfsys@defobject{currentmarker}{\pgfqpoint{0.000000in}{-0.048611in}}{\pgfqpoint{0.000000in}{0.000000in}}{%
\pgfpathmoveto{\pgfqpoint{0.000000in}{0.000000in}}%
\pgfpathlineto{\pgfqpoint{0.000000in}{-0.048611in}}%
\pgfusepath{stroke,fill}%
}%
\begin{pgfscope}%
\pgfsys@transformshift{1.129065in}{0.670138in}%
\pgfsys@useobject{currentmarker}{}%
\end{pgfscope}%
\end{pgfscope}%
\begin{pgfscope}%
\definecolor{textcolor}{rgb}{0.000000,0.000000,0.000000}%
\pgfsetstrokecolor{textcolor}%
\pgfsetfillcolor{textcolor}%
\pgftext[x=1.129065in,y=0.572916in,,top]{\color{textcolor}{\rmfamily\fontsize{14.000000}{16.800000}\selectfont\catcode`\^=\active\def^{\ifmmode\sp\else\^{}\fi}\catcode`\%=\active\def%{\%}$\mathdefault{5500}$}}%
\end{pgfscope}%
\begin{pgfscope}%
\pgfpathrectangle{\pgfqpoint{0.688192in}{0.670138in}}{\pgfqpoint{7.111808in}{5.129862in}}%
\pgfusepath{clip}%
\pgfsetrectcap%
\pgfsetroundjoin%
\pgfsetlinewidth{0.803000pt}%
\definecolor{currentstroke}{rgb}{0.690196,0.690196,0.690196}%
\pgfsetstrokecolor{currentstroke}%
\pgfsetdash{}{0pt}%
\pgfpathmoveto{\pgfqpoint{2.333774in}{0.670138in}}%
\pgfpathlineto{\pgfqpoint{2.333774in}{5.800000in}}%
\pgfusepath{stroke}%
\end{pgfscope}%
\begin{pgfscope}%
\pgfsetbuttcap%
\pgfsetroundjoin%
\definecolor{currentfill}{rgb}{0.000000,0.000000,0.000000}%
\pgfsetfillcolor{currentfill}%
\pgfsetlinewidth{0.803000pt}%
\definecolor{currentstroke}{rgb}{0.000000,0.000000,0.000000}%
\pgfsetstrokecolor{currentstroke}%
\pgfsetdash{}{0pt}%
\pgfsys@defobject{currentmarker}{\pgfqpoint{0.000000in}{-0.048611in}}{\pgfqpoint{0.000000in}{0.000000in}}{%
\pgfpathmoveto{\pgfqpoint{0.000000in}{0.000000in}}%
\pgfpathlineto{\pgfqpoint{0.000000in}{-0.048611in}}%
\pgfusepath{stroke,fill}%
}%
\begin{pgfscope}%
\pgfsys@transformshift{2.333774in}{0.670138in}%
\pgfsys@useobject{currentmarker}{}%
\end{pgfscope}%
\end{pgfscope}%
\begin{pgfscope}%
\definecolor{textcolor}{rgb}{0.000000,0.000000,0.000000}%
\pgfsetstrokecolor{textcolor}%
\pgfsetfillcolor{textcolor}%
\pgftext[x=2.333774in,y=0.572916in,,top]{\color{textcolor}{\rmfamily\fontsize{14.000000}{16.800000}\selectfont\catcode`\^=\active\def^{\ifmmode\sp\else\^{}\fi}\catcode`\%=\active\def%{\%}$\mathdefault{6000}$}}%
\end{pgfscope}%
\begin{pgfscope}%
\pgfpathrectangle{\pgfqpoint{0.688192in}{0.670138in}}{\pgfqpoint{7.111808in}{5.129862in}}%
\pgfusepath{clip}%
\pgfsetrectcap%
\pgfsetroundjoin%
\pgfsetlinewidth{0.803000pt}%
\definecolor{currentstroke}{rgb}{0.690196,0.690196,0.690196}%
\pgfsetstrokecolor{currentstroke}%
\pgfsetdash{}{0pt}%
\pgfpathmoveto{\pgfqpoint{3.538483in}{0.670138in}}%
\pgfpathlineto{\pgfqpoint{3.538483in}{5.800000in}}%
\pgfusepath{stroke}%
\end{pgfscope}%
\begin{pgfscope}%
\pgfsetbuttcap%
\pgfsetroundjoin%
\definecolor{currentfill}{rgb}{0.000000,0.000000,0.000000}%
\pgfsetfillcolor{currentfill}%
\pgfsetlinewidth{0.803000pt}%
\definecolor{currentstroke}{rgb}{0.000000,0.000000,0.000000}%
\pgfsetstrokecolor{currentstroke}%
\pgfsetdash{}{0pt}%
\pgfsys@defobject{currentmarker}{\pgfqpoint{0.000000in}{-0.048611in}}{\pgfqpoint{0.000000in}{0.000000in}}{%
\pgfpathmoveto{\pgfqpoint{0.000000in}{0.000000in}}%
\pgfpathlineto{\pgfqpoint{0.000000in}{-0.048611in}}%
\pgfusepath{stroke,fill}%
}%
\begin{pgfscope}%
\pgfsys@transformshift{3.538483in}{0.670138in}%
\pgfsys@useobject{currentmarker}{}%
\end{pgfscope}%
\end{pgfscope}%
\begin{pgfscope}%
\definecolor{textcolor}{rgb}{0.000000,0.000000,0.000000}%
\pgfsetstrokecolor{textcolor}%
\pgfsetfillcolor{textcolor}%
\pgftext[x=3.538483in,y=0.572916in,,top]{\color{textcolor}{\rmfamily\fontsize{14.000000}{16.800000}\selectfont\catcode`\^=\active\def^{\ifmmode\sp\else\^{}\fi}\catcode`\%=\active\def%{\%}$\mathdefault{6500}$}}%
\end{pgfscope}%
\begin{pgfscope}%
\pgfpathrectangle{\pgfqpoint{0.688192in}{0.670138in}}{\pgfqpoint{7.111808in}{5.129862in}}%
\pgfusepath{clip}%
\pgfsetrectcap%
\pgfsetroundjoin%
\pgfsetlinewidth{0.803000pt}%
\definecolor{currentstroke}{rgb}{0.690196,0.690196,0.690196}%
\pgfsetstrokecolor{currentstroke}%
\pgfsetdash{}{0pt}%
\pgfpathmoveto{\pgfqpoint{4.743192in}{0.670138in}}%
\pgfpathlineto{\pgfqpoint{4.743192in}{5.800000in}}%
\pgfusepath{stroke}%
\end{pgfscope}%
\begin{pgfscope}%
\pgfsetbuttcap%
\pgfsetroundjoin%
\definecolor{currentfill}{rgb}{0.000000,0.000000,0.000000}%
\pgfsetfillcolor{currentfill}%
\pgfsetlinewidth{0.803000pt}%
\definecolor{currentstroke}{rgb}{0.000000,0.000000,0.000000}%
\pgfsetstrokecolor{currentstroke}%
\pgfsetdash{}{0pt}%
\pgfsys@defobject{currentmarker}{\pgfqpoint{0.000000in}{-0.048611in}}{\pgfqpoint{0.000000in}{0.000000in}}{%
\pgfpathmoveto{\pgfqpoint{0.000000in}{0.000000in}}%
\pgfpathlineto{\pgfqpoint{0.000000in}{-0.048611in}}%
\pgfusepath{stroke,fill}%
}%
\begin{pgfscope}%
\pgfsys@transformshift{4.743192in}{0.670138in}%
\pgfsys@useobject{currentmarker}{}%
\end{pgfscope}%
\end{pgfscope}%
\begin{pgfscope}%
\definecolor{textcolor}{rgb}{0.000000,0.000000,0.000000}%
\pgfsetstrokecolor{textcolor}%
\pgfsetfillcolor{textcolor}%
\pgftext[x=4.743192in,y=0.572916in,,top]{\color{textcolor}{\rmfamily\fontsize{14.000000}{16.800000}\selectfont\catcode`\^=\active\def^{\ifmmode\sp\else\^{}\fi}\catcode`\%=\active\def%{\%}$\mathdefault{7000}$}}%
\end{pgfscope}%
\begin{pgfscope}%
\pgfpathrectangle{\pgfqpoint{0.688192in}{0.670138in}}{\pgfqpoint{7.111808in}{5.129862in}}%
\pgfusepath{clip}%
\pgfsetrectcap%
\pgfsetroundjoin%
\pgfsetlinewidth{0.803000pt}%
\definecolor{currentstroke}{rgb}{0.690196,0.690196,0.690196}%
\pgfsetstrokecolor{currentstroke}%
\pgfsetdash{}{0pt}%
\pgfpathmoveto{\pgfqpoint{5.947900in}{0.670138in}}%
\pgfpathlineto{\pgfqpoint{5.947900in}{5.800000in}}%
\pgfusepath{stroke}%
\end{pgfscope}%
\begin{pgfscope}%
\pgfsetbuttcap%
\pgfsetroundjoin%
\definecolor{currentfill}{rgb}{0.000000,0.000000,0.000000}%
\pgfsetfillcolor{currentfill}%
\pgfsetlinewidth{0.803000pt}%
\definecolor{currentstroke}{rgb}{0.000000,0.000000,0.000000}%
\pgfsetstrokecolor{currentstroke}%
\pgfsetdash{}{0pt}%
\pgfsys@defobject{currentmarker}{\pgfqpoint{0.000000in}{-0.048611in}}{\pgfqpoint{0.000000in}{0.000000in}}{%
\pgfpathmoveto{\pgfqpoint{0.000000in}{0.000000in}}%
\pgfpathlineto{\pgfqpoint{0.000000in}{-0.048611in}}%
\pgfusepath{stroke,fill}%
}%
\begin{pgfscope}%
\pgfsys@transformshift{5.947900in}{0.670138in}%
\pgfsys@useobject{currentmarker}{}%
\end{pgfscope}%
\end{pgfscope}%
\begin{pgfscope}%
\definecolor{textcolor}{rgb}{0.000000,0.000000,0.000000}%
\pgfsetstrokecolor{textcolor}%
\pgfsetfillcolor{textcolor}%
\pgftext[x=5.947900in,y=0.572916in,,top]{\color{textcolor}{\rmfamily\fontsize{14.000000}{16.800000}\selectfont\catcode`\^=\active\def^{\ifmmode\sp\else\^{}\fi}\catcode`\%=\active\def%{\%}$\mathdefault{7500}$}}%
\end{pgfscope}%
\begin{pgfscope}%
\pgfpathrectangle{\pgfqpoint{0.688192in}{0.670138in}}{\pgfqpoint{7.111808in}{5.129862in}}%
\pgfusepath{clip}%
\pgfsetrectcap%
\pgfsetroundjoin%
\pgfsetlinewidth{0.803000pt}%
\definecolor{currentstroke}{rgb}{0.690196,0.690196,0.690196}%
\pgfsetstrokecolor{currentstroke}%
\pgfsetdash{}{0pt}%
\pgfpathmoveto{\pgfqpoint{7.152609in}{0.670138in}}%
\pgfpathlineto{\pgfqpoint{7.152609in}{5.800000in}}%
\pgfusepath{stroke}%
\end{pgfscope}%
\begin{pgfscope}%
\pgfsetbuttcap%
\pgfsetroundjoin%
\definecolor{currentfill}{rgb}{0.000000,0.000000,0.000000}%
\pgfsetfillcolor{currentfill}%
\pgfsetlinewidth{0.803000pt}%
\definecolor{currentstroke}{rgb}{0.000000,0.000000,0.000000}%
\pgfsetstrokecolor{currentstroke}%
\pgfsetdash{}{0pt}%
\pgfsys@defobject{currentmarker}{\pgfqpoint{0.000000in}{-0.048611in}}{\pgfqpoint{0.000000in}{0.000000in}}{%
\pgfpathmoveto{\pgfqpoint{0.000000in}{0.000000in}}%
\pgfpathlineto{\pgfqpoint{0.000000in}{-0.048611in}}%
\pgfusepath{stroke,fill}%
}%
\begin{pgfscope}%
\pgfsys@transformshift{7.152609in}{0.670138in}%
\pgfsys@useobject{currentmarker}{}%
\end{pgfscope}%
\end{pgfscope}%
\begin{pgfscope}%
\definecolor{textcolor}{rgb}{0.000000,0.000000,0.000000}%
\pgfsetstrokecolor{textcolor}%
\pgfsetfillcolor{textcolor}%
\pgftext[x=7.152609in,y=0.572916in,,top]{\color{textcolor}{\rmfamily\fontsize{14.000000}{16.800000}\selectfont\catcode`\^=\active\def^{\ifmmode\sp\else\^{}\fi}\catcode`\%=\active\def%{\%}$\mathdefault{8000}$}}%
\end{pgfscope}%
\begin{pgfscope}%
\definecolor{textcolor}{rgb}{0.000000,0.000000,0.000000}%
\pgfsetstrokecolor{textcolor}%
\pgfsetfillcolor{textcolor}%
\pgftext[x=4.244096in,y=0.339583in,,top]{\color{textcolor}{\rmfamily\fontsize{18.000000}{21.600000}\selectfont\catcode`\^=\active\def^{\ifmmode\sp\else\^{}\fi}\catcode`\%=\active\def%{\%}Total Cost [M\$]}}%
\end{pgfscope}%
\begin{pgfscope}%
\pgfpathrectangle{\pgfqpoint{0.688192in}{0.670138in}}{\pgfqpoint{7.111808in}{5.129862in}}%
\pgfusepath{clip}%
\pgfsetrectcap%
\pgfsetroundjoin%
\pgfsetlinewidth{0.803000pt}%
\definecolor{currentstroke}{rgb}{0.690196,0.690196,0.690196}%
\pgfsetstrokecolor{currentstroke}%
\pgfsetdash{}{0pt}%
\pgfpathmoveto{\pgfqpoint{0.688192in}{1.346004in}}%
\pgfpathlineto{\pgfqpoint{7.800000in}{1.346004in}}%
\pgfusepath{stroke}%
\end{pgfscope}%
\begin{pgfscope}%
\pgfsetbuttcap%
\pgfsetroundjoin%
\definecolor{currentfill}{rgb}{0.000000,0.000000,0.000000}%
\pgfsetfillcolor{currentfill}%
\pgfsetlinewidth{0.803000pt}%
\definecolor{currentstroke}{rgb}{0.000000,0.000000,0.000000}%
\pgfsetstrokecolor{currentstroke}%
\pgfsetdash{}{0pt}%
\pgfsys@defobject{currentmarker}{\pgfqpoint{-0.048611in}{0.000000in}}{\pgfqpoint{-0.000000in}{0.000000in}}{%
\pgfpathmoveto{\pgfqpoint{-0.000000in}{0.000000in}}%
\pgfpathlineto{\pgfqpoint{-0.048611in}{0.000000in}}%
\pgfusepath{stroke,fill}%
}%
\begin{pgfscope}%
\pgfsys@transformshift{0.688192in}{1.346004in}%
\pgfsys@useobject{currentmarker}{}%
\end{pgfscope}%
\end{pgfscope}%
\begin{pgfscope}%
\definecolor{textcolor}{rgb}{0.000000,0.000000,0.000000}%
\pgfsetstrokecolor{textcolor}%
\pgfsetfillcolor{textcolor}%
\pgftext[x=0.493054in, y=1.276560in, left, base]{\color{textcolor}{\rmfamily\fontsize{14.000000}{16.800000}\selectfont\catcode`\^=\active\def^{\ifmmode\sp\else\^{}\fi}\catcode`\%=\active\def%{\%}$\mathdefault{4}$}}%
\end{pgfscope}%
\begin{pgfscope}%
\pgfpathrectangle{\pgfqpoint{0.688192in}{0.670138in}}{\pgfqpoint{7.111808in}{5.129862in}}%
\pgfusepath{clip}%
\pgfsetrectcap%
\pgfsetroundjoin%
\pgfsetlinewidth{0.803000pt}%
\definecolor{currentstroke}{rgb}{0.690196,0.690196,0.690196}%
\pgfsetstrokecolor{currentstroke}%
\pgfsetdash{}{0pt}%
\pgfpathmoveto{\pgfqpoint{0.688192in}{2.116026in}}%
\pgfpathlineto{\pgfqpoint{7.800000in}{2.116026in}}%
\pgfusepath{stroke}%
\end{pgfscope}%
\begin{pgfscope}%
\pgfsetbuttcap%
\pgfsetroundjoin%
\definecolor{currentfill}{rgb}{0.000000,0.000000,0.000000}%
\pgfsetfillcolor{currentfill}%
\pgfsetlinewidth{0.803000pt}%
\definecolor{currentstroke}{rgb}{0.000000,0.000000,0.000000}%
\pgfsetstrokecolor{currentstroke}%
\pgfsetdash{}{0pt}%
\pgfsys@defobject{currentmarker}{\pgfqpoint{-0.048611in}{0.000000in}}{\pgfqpoint{-0.000000in}{0.000000in}}{%
\pgfpathmoveto{\pgfqpoint{-0.000000in}{0.000000in}}%
\pgfpathlineto{\pgfqpoint{-0.048611in}{0.000000in}}%
\pgfusepath{stroke,fill}%
}%
\begin{pgfscope}%
\pgfsys@transformshift{0.688192in}{2.116026in}%
\pgfsys@useobject{currentmarker}{}%
\end{pgfscope}%
\end{pgfscope}%
\begin{pgfscope}%
\definecolor{textcolor}{rgb}{0.000000,0.000000,0.000000}%
\pgfsetstrokecolor{textcolor}%
\pgfsetfillcolor{textcolor}%
\pgftext[x=0.493054in, y=2.046582in, left, base]{\color{textcolor}{\rmfamily\fontsize{14.000000}{16.800000}\selectfont\catcode`\^=\active\def^{\ifmmode\sp\else\^{}\fi}\catcode`\%=\active\def%{\%}$\mathdefault{6}$}}%
\end{pgfscope}%
\begin{pgfscope}%
\pgfpathrectangle{\pgfqpoint{0.688192in}{0.670138in}}{\pgfqpoint{7.111808in}{5.129862in}}%
\pgfusepath{clip}%
\pgfsetrectcap%
\pgfsetroundjoin%
\pgfsetlinewidth{0.803000pt}%
\definecolor{currentstroke}{rgb}{0.690196,0.690196,0.690196}%
\pgfsetstrokecolor{currentstroke}%
\pgfsetdash{}{0pt}%
\pgfpathmoveto{\pgfqpoint{0.688192in}{2.886048in}}%
\pgfpathlineto{\pgfqpoint{7.800000in}{2.886048in}}%
\pgfusepath{stroke}%
\end{pgfscope}%
\begin{pgfscope}%
\pgfsetbuttcap%
\pgfsetroundjoin%
\definecolor{currentfill}{rgb}{0.000000,0.000000,0.000000}%
\pgfsetfillcolor{currentfill}%
\pgfsetlinewidth{0.803000pt}%
\definecolor{currentstroke}{rgb}{0.000000,0.000000,0.000000}%
\pgfsetstrokecolor{currentstroke}%
\pgfsetdash{}{0pt}%
\pgfsys@defobject{currentmarker}{\pgfqpoint{-0.048611in}{0.000000in}}{\pgfqpoint{-0.000000in}{0.000000in}}{%
\pgfpathmoveto{\pgfqpoint{-0.000000in}{0.000000in}}%
\pgfpathlineto{\pgfqpoint{-0.048611in}{0.000000in}}%
\pgfusepath{stroke,fill}%
}%
\begin{pgfscope}%
\pgfsys@transformshift{0.688192in}{2.886048in}%
\pgfsys@useobject{currentmarker}{}%
\end{pgfscope}%
\end{pgfscope}%
\begin{pgfscope}%
\definecolor{textcolor}{rgb}{0.000000,0.000000,0.000000}%
\pgfsetstrokecolor{textcolor}%
\pgfsetfillcolor{textcolor}%
\pgftext[x=0.493054in, y=2.816603in, left, base]{\color{textcolor}{\rmfamily\fontsize{14.000000}{16.800000}\selectfont\catcode`\^=\active\def^{\ifmmode\sp\else\^{}\fi}\catcode`\%=\active\def%{\%}$\mathdefault{8}$}}%
\end{pgfscope}%
\begin{pgfscope}%
\pgfpathrectangle{\pgfqpoint{0.688192in}{0.670138in}}{\pgfqpoint{7.111808in}{5.129862in}}%
\pgfusepath{clip}%
\pgfsetrectcap%
\pgfsetroundjoin%
\pgfsetlinewidth{0.803000pt}%
\definecolor{currentstroke}{rgb}{0.690196,0.690196,0.690196}%
\pgfsetstrokecolor{currentstroke}%
\pgfsetdash{}{0pt}%
\pgfpathmoveto{\pgfqpoint{0.688192in}{3.656070in}}%
\pgfpathlineto{\pgfqpoint{7.800000in}{3.656070in}}%
\pgfusepath{stroke}%
\end{pgfscope}%
\begin{pgfscope}%
\pgfsetbuttcap%
\pgfsetroundjoin%
\definecolor{currentfill}{rgb}{0.000000,0.000000,0.000000}%
\pgfsetfillcolor{currentfill}%
\pgfsetlinewidth{0.803000pt}%
\definecolor{currentstroke}{rgb}{0.000000,0.000000,0.000000}%
\pgfsetstrokecolor{currentstroke}%
\pgfsetdash{}{0pt}%
\pgfsys@defobject{currentmarker}{\pgfqpoint{-0.048611in}{0.000000in}}{\pgfqpoint{-0.000000in}{0.000000in}}{%
\pgfpathmoveto{\pgfqpoint{-0.000000in}{0.000000in}}%
\pgfpathlineto{\pgfqpoint{-0.048611in}{0.000000in}}%
\pgfusepath{stroke,fill}%
}%
\begin{pgfscope}%
\pgfsys@transformshift{0.688192in}{3.656070in}%
\pgfsys@useobject{currentmarker}{}%
\end{pgfscope}%
\end{pgfscope}%
\begin{pgfscope}%
\definecolor{textcolor}{rgb}{0.000000,0.000000,0.000000}%
\pgfsetstrokecolor{textcolor}%
\pgfsetfillcolor{textcolor}%
\pgftext[x=0.395138in, y=3.586625in, left, base]{\color{textcolor}{\rmfamily\fontsize{14.000000}{16.800000}\selectfont\catcode`\^=\active\def^{\ifmmode\sp\else\^{}\fi}\catcode`\%=\active\def%{\%}$\mathdefault{10}$}}%
\end{pgfscope}%
\begin{pgfscope}%
\pgfpathrectangle{\pgfqpoint{0.688192in}{0.670138in}}{\pgfqpoint{7.111808in}{5.129862in}}%
\pgfusepath{clip}%
\pgfsetrectcap%
\pgfsetroundjoin%
\pgfsetlinewidth{0.803000pt}%
\definecolor{currentstroke}{rgb}{0.690196,0.690196,0.690196}%
\pgfsetstrokecolor{currentstroke}%
\pgfsetdash{}{0pt}%
\pgfpathmoveto{\pgfqpoint{0.688192in}{4.426091in}}%
\pgfpathlineto{\pgfqpoint{7.800000in}{4.426091in}}%
\pgfusepath{stroke}%
\end{pgfscope}%
\begin{pgfscope}%
\pgfsetbuttcap%
\pgfsetroundjoin%
\definecolor{currentfill}{rgb}{0.000000,0.000000,0.000000}%
\pgfsetfillcolor{currentfill}%
\pgfsetlinewidth{0.803000pt}%
\definecolor{currentstroke}{rgb}{0.000000,0.000000,0.000000}%
\pgfsetstrokecolor{currentstroke}%
\pgfsetdash{}{0pt}%
\pgfsys@defobject{currentmarker}{\pgfqpoint{-0.048611in}{0.000000in}}{\pgfqpoint{-0.000000in}{0.000000in}}{%
\pgfpathmoveto{\pgfqpoint{-0.000000in}{0.000000in}}%
\pgfpathlineto{\pgfqpoint{-0.048611in}{0.000000in}}%
\pgfusepath{stroke,fill}%
}%
\begin{pgfscope}%
\pgfsys@transformshift{0.688192in}{4.426091in}%
\pgfsys@useobject{currentmarker}{}%
\end{pgfscope}%
\end{pgfscope}%
\begin{pgfscope}%
\definecolor{textcolor}{rgb}{0.000000,0.000000,0.000000}%
\pgfsetstrokecolor{textcolor}%
\pgfsetfillcolor{textcolor}%
\pgftext[x=0.395138in, y=4.356647in, left, base]{\color{textcolor}{\rmfamily\fontsize{14.000000}{16.800000}\selectfont\catcode`\^=\active\def^{\ifmmode\sp\else\^{}\fi}\catcode`\%=\active\def%{\%}$\mathdefault{12}$}}%
\end{pgfscope}%
\begin{pgfscope}%
\pgfpathrectangle{\pgfqpoint{0.688192in}{0.670138in}}{\pgfqpoint{7.111808in}{5.129862in}}%
\pgfusepath{clip}%
\pgfsetrectcap%
\pgfsetroundjoin%
\pgfsetlinewidth{0.803000pt}%
\definecolor{currentstroke}{rgb}{0.690196,0.690196,0.690196}%
\pgfsetstrokecolor{currentstroke}%
\pgfsetdash{}{0pt}%
\pgfpathmoveto{\pgfqpoint{0.688192in}{5.196113in}}%
\pgfpathlineto{\pgfqpoint{7.800000in}{5.196113in}}%
\pgfusepath{stroke}%
\end{pgfscope}%
\begin{pgfscope}%
\pgfsetbuttcap%
\pgfsetroundjoin%
\definecolor{currentfill}{rgb}{0.000000,0.000000,0.000000}%
\pgfsetfillcolor{currentfill}%
\pgfsetlinewidth{0.803000pt}%
\definecolor{currentstroke}{rgb}{0.000000,0.000000,0.000000}%
\pgfsetstrokecolor{currentstroke}%
\pgfsetdash{}{0pt}%
\pgfsys@defobject{currentmarker}{\pgfqpoint{-0.048611in}{0.000000in}}{\pgfqpoint{-0.000000in}{0.000000in}}{%
\pgfpathmoveto{\pgfqpoint{-0.000000in}{0.000000in}}%
\pgfpathlineto{\pgfqpoint{-0.048611in}{0.000000in}}%
\pgfusepath{stroke,fill}%
}%
\begin{pgfscope}%
\pgfsys@transformshift{0.688192in}{5.196113in}%
\pgfsys@useobject{currentmarker}{}%
\end{pgfscope}%
\end{pgfscope}%
\begin{pgfscope}%
\definecolor{textcolor}{rgb}{0.000000,0.000000,0.000000}%
\pgfsetstrokecolor{textcolor}%
\pgfsetfillcolor{textcolor}%
\pgftext[x=0.395138in, y=5.126669in, left, base]{\color{textcolor}{\rmfamily\fontsize{14.000000}{16.800000}\selectfont\catcode`\^=\active\def^{\ifmmode\sp\else\^{}\fi}\catcode`\%=\active\def%{\%}$\mathdefault{14}$}}%
\end{pgfscope}%
\begin{pgfscope}%
\definecolor{textcolor}{rgb}{0.000000,0.000000,0.000000}%
\pgfsetstrokecolor{textcolor}%
\pgfsetfillcolor{textcolor}%
\pgftext[x=0.339583in,y=3.235069in,,bottom,rotate=90.000000]{\color{textcolor}{\rmfamily\fontsize{18.000000}{21.600000}\selectfont\catcode`\^=\active\def^{\ifmmode\sp\else\^{}\fi}\catcode`\%=\active\def%{\%}Emissions [MT CO$_2$eq]}}%
\end{pgfscope}%
\begin{pgfscope}%
\pgfpathrectangle{\pgfqpoint{0.688192in}{0.670138in}}{\pgfqpoint{7.111808in}{5.129862in}}%
\pgfusepath{clip}%
\pgfsetrectcap%
\pgfsetroundjoin%
\pgfsetlinewidth{2.509375pt}%
\definecolor{currentstroke}{rgb}{0.000000,0.000000,0.000000}%
\pgfsetstrokecolor{currentstroke}%
\pgfsetdash{}{0pt}%
\pgfpathmoveto{\pgfqpoint{0.688192in}{5.255087in}}%
\pgfpathlineto{\pgfqpoint{0.691952in}{4.116925in}}%
\pgfpathlineto{\pgfqpoint{0.693069in}{3.535014in}}%
\pgfpathlineto{\pgfqpoint{0.699877in}{3.427816in}}%
\pgfpathlineto{\pgfqpoint{0.703696in}{2.711585in}}%
\pgfpathlineto{\pgfqpoint{0.708182in}{2.516637in}}%
\pgfpathlineto{\pgfqpoint{0.710424in}{2.515490in}}%
\pgfpathlineto{\pgfqpoint{0.715926in}{2.277179in}}%
\pgfpathlineto{\pgfqpoint{0.716194in}{2.143247in}}%
\pgfpathlineto{\pgfqpoint{0.719167in}{1.998136in}}%
\pgfpathlineto{\pgfqpoint{0.721104in}{1.962980in}}%
\pgfpathlineto{\pgfqpoint{0.722141in}{1.928466in}}%
\pgfpathlineto{\pgfqpoint{0.722511in}{1.733351in}}%
\pgfpathlineto{\pgfqpoint{0.726743in}{1.698359in}}%
\pgfpathlineto{\pgfqpoint{0.730789in}{1.657616in}}%
\pgfpathlineto{\pgfqpoint{0.732629in}{1.494803in}}%
\pgfpathlineto{\pgfqpoint{0.735321in}{1.444486in}}%
\pgfpathlineto{\pgfqpoint{0.735501in}{1.401992in}}%
\pgfpathlineto{\pgfqpoint{0.740362in}{1.381421in}}%
\pgfpathlineto{\pgfqpoint{0.747329in}{1.376655in}}%
\pgfpathlineto{\pgfqpoint{0.749794in}{1.239967in}}%
\pgfpathlineto{\pgfqpoint{0.755774in}{1.206744in}}%
\pgfpathlineto{\pgfqpoint{0.766804in}{1.200574in}}%
\pgfpathlineto{\pgfqpoint{0.770575in}{1.148701in}}%
\pgfpathlineto{\pgfqpoint{0.777060in}{1.132809in}}%
\pgfpathlineto{\pgfqpoint{0.782559in}{1.113067in}}%
\pgfpathlineto{\pgfqpoint{0.789241in}{1.084530in}}%
\pgfpathlineto{\pgfqpoint{0.789763in}{1.079638in}}%
\pgfpathlineto{\pgfqpoint{0.790083in}{1.073959in}}%
\pgfpathlineto{\pgfqpoint{0.794194in}{1.063688in}}%
\pgfpathlineto{\pgfqpoint{0.812030in}{1.034334in}}%
\pgfpathlineto{\pgfqpoint{0.830999in}{0.993434in}}%
\pgfpathlineto{\pgfqpoint{0.856742in}{0.964253in}}%
\pgfpathlineto{\pgfqpoint{0.880316in}{0.954452in}}%
\pgfpathlineto{\pgfqpoint{0.881277in}{0.941511in}}%
\pgfpathlineto{\pgfqpoint{0.882669in}{0.918939in}}%
\pgfpathlineto{\pgfqpoint{0.885952in}{0.914664in}}%
\pgfpathlineto{\pgfqpoint{0.886403in}{0.910894in}}%
\pgfpathlineto{\pgfqpoint{0.886861in}{0.910810in}}%
\pgfpathlineto{\pgfqpoint{0.895312in}{0.907258in}}%
\pgfpathlineto{\pgfqpoint{0.901363in}{0.897632in}}%
\pgfpathlineto{\pgfqpoint{0.928243in}{0.882681in}}%
\pgfpathlineto{\pgfqpoint{0.931763in}{0.874747in}}%
\pgfpathlineto{\pgfqpoint{0.931867in}{0.866217in}}%
\pgfpathlineto{\pgfqpoint{0.937645in}{0.864808in}}%
\pgfpathlineto{\pgfqpoint{0.952302in}{0.849081in}}%
\pgfpathlineto{\pgfqpoint{0.957757in}{0.849024in}}%
\pgfpathlineto{\pgfqpoint{0.958756in}{0.843680in}}%
\pgfpathlineto{\pgfqpoint{1.012547in}{0.829947in}}%
\pgfpathlineto{\pgfqpoint{1.016759in}{0.803132in}}%
\pgfpathlineto{\pgfqpoint{1.019151in}{0.801064in}}%
\pgfpathlineto{\pgfqpoint{1.026002in}{0.799177in}}%
\pgfpathlineto{\pgfqpoint{1.027669in}{0.797472in}}%
\pgfpathlineto{\pgfqpoint{1.044526in}{0.783805in}}%
\pgfpathlineto{\pgfqpoint{1.045306in}{0.782620in}}%
\pgfpathlineto{\pgfqpoint{1.050628in}{0.780895in}}%
\pgfpathlineto{\pgfqpoint{1.050667in}{0.780323in}}%
\pgfpathlineto{\pgfqpoint{1.052060in}{0.777139in}}%
\pgfpathlineto{\pgfqpoint{1.056340in}{0.775974in}}%
\pgfpathlineto{\pgfqpoint{1.058884in}{0.775846in}}%
\pgfpathlineto{\pgfqpoint{1.069116in}{0.770892in}}%
\pgfpathlineto{\pgfqpoint{1.071092in}{0.766931in}}%
\pgfpathlineto{\pgfqpoint{1.090088in}{0.766654in}}%
\pgfpathlineto{\pgfqpoint{1.091394in}{0.763336in}}%
\pgfpathlineto{\pgfqpoint{1.094048in}{0.762662in}}%
\pgfpathlineto{\pgfqpoint{1.101880in}{0.761761in}}%
\pgfpathlineto{\pgfqpoint{1.120660in}{0.760704in}}%
\pgfpathlineto{\pgfqpoint{1.163954in}{0.760365in}}%
\pgfpathlineto{\pgfqpoint{1.259457in}{0.757136in}}%
\pgfpathlineto{\pgfqpoint{1.278130in}{0.755206in}}%
\pgfpathlineto{\pgfqpoint{1.308402in}{0.754533in}}%
\pgfpathlineto{\pgfqpoint{1.320633in}{0.753865in}}%
\pgfpathlineto{\pgfqpoint{1.341138in}{0.753397in}}%
\pgfpathlineto{\pgfqpoint{1.344495in}{0.753033in}}%
\pgfpathlineto{\pgfqpoint{1.351667in}{0.752560in}}%
\pgfpathlineto{\pgfqpoint{1.400715in}{0.752063in}}%
\pgfpathlineto{\pgfqpoint{1.421141in}{0.751622in}}%
\pgfpathlineto{\pgfqpoint{1.422574in}{0.751405in}}%
\pgfpathlineto{\pgfqpoint{1.448644in}{0.748886in}}%
\pgfpathlineto{\pgfqpoint{1.528667in}{0.747888in}}%
\pgfpathlineto{\pgfqpoint{1.605641in}{0.747867in}}%
\pgfpathlineto{\pgfqpoint{1.606745in}{0.746981in}}%
\pgfpathlineto{\pgfqpoint{1.610058in}{0.746614in}}%
\pgfpathlineto{\pgfqpoint{1.622014in}{0.746220in}}%
\pgfpathlineto{\pgfqpoint{1.625190in}{0.745498in}}%
\pgfpathlineto{\pgfqpoint{1.631842in}{0.744948in}}%
\pgfpathlineto{\pgfqpoint{1.666551in}{0.744594in}}%
\pgfpathlineto{\pgfqpoint{1.679564in}{0.743446in}}%
\pgfpathlineto{\pgfqpoint{1.685511in}{0.742202in}}%
\pgfpathlineto{\pgfqpoint{1.702822in}{0.741854in}}%
\pgfpathlineto{\pgfqpoint{1.708990in}{0.740247in}}%
\pgfpathlineto{\pgfqpoint{1.713082in}{0.739976in}}%
\pgfpathlineto{\pgfqpoint{1.763752in}{0.739729in}}%
\pgfpathlineto{\pgfqpoint{1.775222in}{0.738819in}}%
\pgfpathlineto{\pgfqpoint{1.800509in}{0.738014in}}%
\pgfpathlineto{\pgfqpoint{1.840914in}{0.737153in}}%
\pgfpathlineto{\pgfqpoint{1.852165in}{0.736068in}}%
\pgfpathlineto{\pgfqpoint{1.877664in}{0.735418in}}%
\pgfpathlineto{\pgfqpoint{1.916797in}{0.734885in}}%
\pgfpathlineto{\pgfqpoint{1.920423in}{0.733745in}}%
\pgfpathlineto{\pgfqpoint{2.013663in}{0.731531in}}%
\pgfpathlineto{\pgfqpoint{2.176728in}{0.728600in}}%
\pgfpathlineto{\pgfqpoint{2.188963in}{0.726384in}}%
\pgfpathlineto{\pgfqpoint{2.217170in}{0.725864in}}%
\pgfpathlineto{\pgfqpoint{2.331314in}{0.722875in}}%
\pgfpathlineto{\pgfqpoint{2.346054in}{0.721682in}}%
\pgfpathlineto{\pgfqpoint{2.382807in}{0.719754in}}%
\pgfpathlineto{\pgfqpoint{2.406189in}{0.719470in}}%
\pgfpathlineto{\pgfqpoint{2.443271in}{0.719418in}}%
\pgfpathlineto{\pgfqpoint{2.464440in}{0.717412in}}%
\pgfpathlineto{\pgfqpoint{2.497167in}{0.716312in}}%
\pgfpathlineto{\pgfqpoint{2.563540in}{0.715743in}}%
\pgfpathlineto{\pgfqpoint{2.632948in}{0.712804in}}%
\pgfpathlineto{\pgfqpoint{2.869035in}{0.710469in}}%
\pgfpathlineto{\pgfqpoint{2.890311in}{0.707406in}}%
\pgfpathlineto{\pgfqpoint{2.934581in}{0.705830in}}%
\pgfpathlineto{\pgfqpoint{2.983194in}{0.704605in}}%
\pgfpathlineto{\pgfqpoint{3.241809in}{0.699660in}}%
\pgfpathlineto{\pgfqpoint{3.252817in}{0.698436in}}%
\pgfpathlineto{\pgfqpoint{3.523301in}{0.692124in}}%
\pgfpathlineto{\pgfqpoint{3.755809in}{0.689503in}}%
\pgfpathlineto{\pgfqpoint{3.829576in}{0.687811in}}%
\pgfpathlineto{\pgfqpoint{3.963043in}{0.685366in}}%
\pgfpathlineto{\pgfqpoint{4.182175in}{0.682726in}}%
\pgfpathlineto{\pgfqpoint{4.462854in}{0.682464in}}%
\pgfpathlineto{\pgfqpoint{4.586127in}{0.681928in}}%
\pgfpathlineto{\pgfqpoint{4.603680in}{0.680832in}}%
\pgfpathlineto{\pgfqpoint{4.634758in}{0.680738in}}%
\pgfpathlineto{\pgfqpoint{4.661601in}{0.679515in}}%
\pgfpathlineto{\pgfqpoint{4.700040in}{0.679144in}}%
\pgfpathlineto{\pgfqpoint{4.708661in}{0.678566in}}%
\pgfpathlineto{\pgfqpoint{4.737940in}{0.678102in}}%
\pgfpathlineto{\pgfqpoint{4.914884in}{0.677644in}}%
\pgfpathlineto{\pgfqpoint{4.954913in}{0.677128in}}%
\pgfpathlineto{\pgfqpoint{4.970414in}{0.676799in}}%
\pgfpathlineto{\pgfqpoint{5.112107in}{0.675923in}}%
\pgfpathlineto{\pgfqpoint{5.117964in}{0.675193in}}%
\pgfpathlineto{\pgfqpoint{5.136924in}{0.674382in}}%
\pgfpathlineto{\pgfqpoint{5.261268in}{0.673970in}}%
\pgfpathlineto{\pgfqpoint{5.338744in}{0.673524in}}%
\pgfpathlineto{\pgfqpoint{5.349847in}{0.673027in}}%
\pgfpathlineto{\pgfqpoint{5.507614in}{0.672879in}}%
\pgfpathlineto{\pgfqpoint{5.537814in}{0.672431in}}%
\pgfpathlineto{\pgfqpoint{5.696293in}{0.671853in}}%
\pgfpathlineto{\pgfqpoint{5.823193in}{0.670783in}}%
\pgfpathlineto{\pgfqpoint{5.988842in}{0.670138in}}%
\pgfpathlineto{\pgfqpoint{5.988842in}{0.670138in}}%
\pgfusepath{stroke}%
\end{pgfscope}%
\begin{pgfscope}%
\pgfpathrectangle{\pgfqpoint{0.688192in}{0.670138in}}{\pgfqpoint{7.111808in}{5.129862in}}%
\pgfusepath{clip}%
\pgfsetbuttcap%
\pgfsetroundjoin%
\definecolor{currentfill}{rgb}{0.000000,0.000000,0.000000}%
\pgfsetfillcolor{currentfill}%
\pgfsetlinewidth{1.003750pt}%
\definecolor{currentstroke}{rgb}{0.000000,0.000000,0.000000}%
\pgfsetstrokecolor{currentstroke}%
\pgfsetdash{}{0pt}%
\pgfsys@defobject{currentmarker}{\pgfqpoint{-0.020833in}{-0.020833in}}{\pgfqpoint{0.020833in}{0.020833in}}{%
\pgfpathmoveto{\pgfqpoint{0.000000in}{-0.020833in}}%
\pgfpathcurveto{\pgfqpoint{0.005525in}{-0.020833in}}{\pgfqpoint{0.010825in}{-0.018638in}}{\pgfqpoint{0.014731in}{-0.014731in}}%
\pgfpathcurveto{\pgfqpoint{0.018638in}{-0.010825in}}{\pgfqpoint{0.020833in}{-0.005525in}}{\pgfqpoint{0.020833in}{0.000000in}}%
\pgfpathcurveto{\pgfqpoint{0.020833in}{0.005525in}}{\pgfqpoint{0.018638in}{0.010825in}}{\pgfqpoint{0.014731in}{0.014731in}}%
\pgfpathcurveto{\pgfqpoint{0.010825in}{0.018638in}}{\pgfqpoint{0.005525in}{0.020833in}}{\pgfqpoint{0.000000in}{0.020833in}}%
\pgfpathcurveto{\pgfqpoint{-0.005525in}{0.020833in}}{\pgfqpoint{-0.010825in}{0.018638in}}{\pgfqpoint{-0.014731in}{0.014731in}}%
\pgfpathcurveto{\pgfqpoint{-0.018638in}{0.010825in}}{\pgfqpoint{-0.020833in}{0.005525in}}{\pgfqpoint{-0.020833in}{0.000000in}}%
\pgfpathcurveto{\pgfqpoint{-0.020833in}{-0.005525in}}{\pgfqpoint{-0.018638in}{-0.010825in}}{\pgfqpoint{-0.014731in}{-0.014731in}}%
\pgfpathcurveto{\pgfqpoint{-0.010825in}{-0.018638in}}{\pgfqpoint{-0.005525in}{-0.020833in}}{\pgfqpoint{0.000000in}{-0.020833in}}%
\pgfpathlineto{\pgfqpoint{0.000000in}{-0.020833in}}%
\pgfpathclose%
\pgfusepath{stroke,fill}%
}%
\begin{pgfscope}%
\pgfsys@transformshift{0.688192in}{5.255087in}%
\pgfsys@useobject{currentmarker}{}%
\end{pgfscope}%
\begin{pgfscope}%
\pgfsys@transformshift{0.691952in}{4.116925in}%
\pgfsys@useobject{currentmarker}{}%
\end{pgfscope}%
\begin{pgfscope}%
\pgfsys@transformshift{0.693069in}{3.535014in}%
\pgfsys@useobject{currentmarker}{}%
\end{pgfscope}%
\begin{pgfscope}%
\pgfsys@transformshift{0.699877in}{3.427816in}%
\pgfsys@useobject{currentmarker}{}%
\end{pgfscope}%
\begin{pgfscope}%
\pgfsys@transformshift{0.703696in}{2.711585in}%
\pgfsys@useobject{currentmarker}{}%
\end{pgfscope}%
\begin{pgfscope}%
\pgfsys@transformshift{0.708182in}{2.516637in}%
\pgfsys@useobject{currentmarker}{}%
\end{pgfscope}%
\begin{pgfscope}%
\pgfsys@transformshift{0.710424in}{2.515490in}%
\pgfsys@useobject{currentmarker}{}%
\end{pgfscope}%
\begin{pgfscope}%
\pgfsys@transformshift{0.715926in}{2.277179in}%
\pgfsys@useobject{currentmarker}{}%
\end{pgfscope}%
\begin{pgfscope}%
\pgfsys@transformshift{0.716194in}{2.143247in}%
\pgfsys@useobject{currentmarker}{}%
\end{pgfscope}%
\begin{pgfscope}%
\pgfsys@transformshift{0.719167in}{1.998136in}%
\pgfsys@useobject{currentmarker}{}%
\end{pgfscope}%
\begin{pgfscope}%
\pgfsys@transformshift{0.721104in}{1.962980in}%
\pgfsys@useobject{currentmarker}{}%
\end{pgfscope}%
\begin{pgfscope}%
\pgfsys@transformshift{0.722061in}{1.931704in}%
\pgfsys@useobject{currentmarker}{}%
\end{pgfscope}%
\begin{pgfscope}%
\pgfsys@transformshift{0.722141in}{1.928466in}%
\pgfsys@useobject{currentmarker}{}%
\end{pgfscope}%
\begin{pgfscope}%
\pgfsys@transformshift{0.722511in}{1.733351in}%
\pgfsys@useobject{currentmarker}{}%
\end{pgfscope}%
\begin{pgfscope}%
\pgfsys@transformshift{0.722511in}{1.733351in}%
\pgfsys@useobject{currentmarker}{}%
\end{pgfscope}%
\begin{pgfscope}%
\pgfsys@transformshift{0.726743in}{1.698359in}%
\pgfsys@useobject{currentmarker}{}%
\end{pgfscope}%
\begin{pgfscope}%
\pgfsys@transformshift{0.730789in}{1.657616in}%
\pgfsys@useobject{currentmarker}{}%
\end{pgfscope}%
\begin{pgfscope}%
\pgfsys@transformshift{0.732621in}{1.495185in}%
\pgfsys@useobject{currentmarker}{}%
\end{pgfscope}%
\begin{pgfscope}%
\pgfsys@transformshift{0.732629in}{1.494803in}%
\pgfsys@useobject{currentmarker}{}%
\end{pgfscope}%
\begin{pgfscope}%
\pgfsys@transformshift{0.735321in}{1.444486in}%
\pgfsys@useobject{currentmarker}{}%
\end{pgfscope}%
\begin{pgfscope}%
\pgfsys@transformshift{0.735501in}{1.401992in}%
\pgfsys@useobject{currentmarker}{}%
\end{pgfscope}%
\begin{pgfscope}%
\pgfsys@transformshift{0.740362in}{1.381421in}%
\pgfsys@useobject{currentmarker}{}%
\end{pgfscope}%
\begin{pgfscope}%
\pgfsys@transformshift{0.747329in}{1.376655in}%
\pgfsys@useobject{currentmarker}{}%
\end{pgfscope}%
\begin{pgfscope}%
\pgfsys@transformshift{0.749794in}{1.239967in}%
\pgfsys@useobject{currentmarker}{}%
\end{pgfscope}%
\begin{pgfscope}%
\pgfsys@transformshift{0.755774in}{1.206744in}%
\pgfsys@useobject{currentmarker}{}%
\end{pgfscope}%
\begin{pgfscope}%
\pgfsys@transformshift{0.766804in}{1.200574in}%
\pgfsys@useobject{currentmarker}{}%
\end{pgfscope}%
\begin{pgfscope}%
\pgfsys@transformshift{0.770575in}{1.148701in}%
\pgfsys@useobject{currentmarker}{}%
\end{pgfscope}%
\begin{pgfscope}%
\pgfsys@transformshift{0.774443in}{1.139626in}%
\pgfsys@useobject{currentmarker}{}%
\end{pgfscope}%
\begin{pgfscope}%
\pgfsys@transformshift{0.777060in}{1.132809in}%
\pgfsys@useobject{currentmarker}{}%
\end{pgfscope}%
\begin{pgfscope}%
\pgfsys@transformshift{0.782053in}{1.114816in}%
\pgfsys@useobject{currentmarker}{}%
\end{pgfscope}%
\begin{pgfscope}%
\pgfsys@transformshift{0.782443in}{1.113980in}%
\pgfsys@useobject{currentmarker}{}%
\end{pgfscope}%
\begin{pgfscope}%
\pgfsys@transformshift{0.782559in}{1.113067in}%
\pgfsys@useobject{currentmarker}{}%
\end{pgfscope}%
\begin{pgfscope}%
\pgfsys@transformshift{0.789241in}{1.084530in}%
\pgfsys@useobject{currentmarker}{}%
\end{pgfscope}%
\begin{pgfscope}%
\pgfsys@transformshift{0.789763in}{1.079638in}%
\pgfsys@useobject{currentmarker}{}%
\end{pgfscope}%
\begin{pgfscope}%
\pgfsys@transformshift{0.790083in}{1.073959in}%
\pgfsys@useobject{currentmarker}{}%
\end{pgfscope}%
\begin{pgfscope}%
\pgfsys@transformshift{0.794194in}{1.063688in}%
\pgfsys@useobject{currentmarker}{}%
\end{pgfscope}%
\begin{pgfscope}%
\pgfsys@transformshift{0.812030in}{1.034334in}%
\pgfsys@useobject{currentmarker}{}%
\end{pgfscope}%
\begin{pgfscope}%
\pgfsys@transformshift{0.830999in}{0.993434in}%
\pgfsys@useobject{currentmarker}{}%
\end{pgfscope}%
\begin{pgfscope}%
\pgfsys@transformshift{0.856742in}{0.964253in}%
\pgfsys@useobject{currentmarker}{}%
\end{pgfscope}%
\begin{pgfscope}%
\pgfsys@transformshift{0.880316in}{0.954452in}%
\pgfsys@useobject{currentmarker}{}%
\end{pgfscope}%
\begin{pgfscope}%
\pgfsys@transformshift{0.881277in}{0.941511in}%
\pgfsys@useobject{currentmarker}{}%
\end{pgfscope}%
\begin{pgfscope}%
\pgfsys@transformshift{0.882669in}{0.918939in}%
\pgfsys@useobject{currentmarker}{}%
\end{pgfscope}%
\begin{pgfscope}%
\pgfsys@transformshift{0.885952in}{0.914664in}%
\pgfsys@useobject{currentmarker}{}%
\end{pgfscope}%
\begin{pgfscope}%
\pgfsys@transformshift{0.886403in}{0.910894in}%
\pgfsys@useobject{currentmarker}{}%
\end{pgfscope}%
\begin{pgfscope}%
\pgfsys@transformshift{0.886861in}{0.910810in}%
\pgfsys@useobject{currentmarker}{}%
\end{pgfscope}%
\begin{pgfscope}%
\pgfsys@transformshift{0.895312in}{0.907258in}%
\pgfsys@useobject{currentmarker}{}%
\end{pgfscope}%
\begin{pgfscope}%
\pgfsys@transformshift{0.899610in}{0.900126in}%
\pgfsys@useobject{currentmarker}{}%
\end{pgfscope}%
\begin{pgfscope}%
\pgfsys@transformshift{0.901363in}{0.897632in}%
\pgfsys@useobject{currentmarker}{}%
\end{pgfscope}%
\begin{pgfscope}%
\pgfsys@transformshift{0.928243in}{0.882681in}%
\pgfsys@useobject{currentmarker}{}%
\end{pgfscope}%
\begin{pgfscope}%
\pgfsys@transformshift{0.931763in}{0.874747in}%
\pgfsys@useobject{currentmarker}{}%
\end{pgfscope}%
\begin{pgfscope}%
\pgfsys@transformshift{0.931867in}{0.866217in}%
\pgfsys@useobject{currentmarker}{}%
\end{pgfscope}%
\begin{pgfscope}%
\pgfsys@transformshift{0.937645in}{0.864808in}%
\pgfsys@useobject{currentmarker}{}%
\end{pgfscope}%
\begin{pgfscope}%
\pgfsys@transformshift{0.951297in}{0.850481in}%
\pgfsys@useobject{currentmarker}{}%
\end{pgfscope}%
\begin{pgfscope}%
\pgfsys@transformshift{0.952302in}{0.849081in}%
\pgfsys@useobject{currentmarker}{}%
\end{pgfscope}%
\begin{pgfscope}%
\pgfsys@transformshift{0.957757in}{0.849024in}%
\pgfsys@useobject{currentmarker}{}%
\end{pgfscope}%
\begin{pgfscope}%
\pgfsys@transformshift{0.958756in}{0.843680in}%
\pgfsys@useobject{currentmarker}{}%
\end{pgfscope}%
\begin{pgfscope}%
\pgfsys@transformshift{1.012547in}{0.829947in}%
\pgfsys@useobject{currentmarker}{}%
\end{pgfscope}%
\begin{pgfscope}%
\pgfsys@transformshift{1.016759in}{0.803132in}%
\pgfsys@useobject{currentmarker}{}%
\end{pgfscope}%
\begin{pgfscope}%
\pgfsys@transformshift{1.019140in}{0.801064in}%
\pgfsys@useobject{currentmarker}{}%
\end{pgfscope}%
\begin{pgfscope}%
\pgfsys@transformshift{1.019151in}{0.801064in}%
\pgfsys@useobject{currentmarker}{}%
\end{pgfscope}%
\begin{pgfscope}%
\pgfsys@transformshift{1.026002in}{0.799177in}%
\pgfsys@useobject{currentmarker}{}%
\end{pgfscope}%
\begin{pgfscope}%
\pgfsys@transformshift{1.027669in}{0.797472in}%
\pgfsys@useobject{currentmarker}{}%
\end{pgfscope}%
\begin{pgfscope}%
\pgfsys@transformshift{1.044526in}{0.783805in}%
\pgfsys@useobject{currentmarker}{}%
\end{pgfscope}%
\begin{pgfscope}%
\pgfsys@transformshift{1.045306in}{0.782620in}%
\pgfsys@useobject{currentmarker}{}%
\end{pgfscope}%
\begin{pgfscope}%
\pgfsys@transformshift{1.050628in}{0.780895in}%
\pgfsys@useobject{currentmarker}{}%
\end{pgfscope}%
\begin{pgfscope}%
\pgfsys@transformshift{1.050667in}{0.780323in}%
\pgfsys@useobject{currentmarker}{}%
\end{pgfscope}%
\begin{pgfscope}%
\pgfsys@transformshift{1.052060in}{0.777139in}%
\pgfsys@useobject{currentmarker}{}%
\end{pgfscope}%
\begin{pgfscope}%
\pgfsys@transformshift{1.056340in}{0.775974in}%
\pgfsys@useobject{currentmarker}{}%
\end{pgfscope}%
\begin{pgfscope}%
\pgfsys@transformshift{1.058884in}{0.775846in}%
\pgfsys@useobject{currentmarker}{}%
\end{pgfscope}%
\begin{pgfscope}%
\pgfsys@transformshift{1.069116in}{0.770892in}%
\pgfsys@useobject{currentmarker}{}%
\end{pgfscope}%
\begin{pgfscope}%
\pgfsys@transformshift{1.070830in}{0.766970in}%
\pgfsys@useobject{currentmarker}{}%
\end{pgfscope}%
\begin{pgfscope}%
\pgfsys@transformshift{1.071092in}{0.766931in}%
\pgfsys@useobject{currentmarker}{}%
\end{pgfscope}%
\begin{pgfscope}%
\pgfsys@transformshift{1.090088in}{0.766654in}%
\pgfsys@useobject{currentmarker}{}%
\end{pgfscope}%
\begin{pgfscope}%
\pgfsys@transformshift{1.091394in}{0.763336in}%
\pgfsys@useobject{currentmarker}{}%
\end{pgfscope}%
\begin{pgfscope}%
\pgfsys@transformshift{1.094048in}{0.762662in}%
\pgfsys@useobject{currentmarker}{}%
\end{pgfscope}%
\begin{pgfscope}%
\pgfsys@transformshift{1.101880in}{0.761761in}%
\pgfsys@useobject{currentmarker}{}%
\end{pgfscope}%
\begin{pgfscope}%
\pgfsys@transformshift{1.106601in}{0.761527in}%
\pgfsys@useobject{currentmarker}{}%
\end{pgfscope}%
\begin{pgfscope}%
\pgfsys@transformshift{1.120660in}{0.760704in}%
\pgfsys@useobject{currentmarker}{}%
\end{pgfscope}%
\begin{pgfscope}%
\pgfsys@transformshift{1.139895in}{0.760442in}%
\pgfsys@useobject{currentmarker}{}%
\end{pgfscope}%
\begin{pgfscope}%
\pgfsys@transformshift{1.141797in}{0.760441in}%
\pgfsys@useobject{currentmarker}{}%
\end{pgfscope}%
\begin{pgfscope}%
\pgfsys@transformshift{1.142999in}{0.760397in}%
\pgfsys@useobject{currentmarker}{}%
\end{pgfscope}%
\begin{pgfscope}%
\pgfsys@transformshift{1.163954in}{0.760365in}%
\pgfsys@useobject{currentmarker}{}%
\end{pgfscope}%
\begin{pgfscope}%
\pgfsys@transformshift{1.259457in}{0.757136in}%
\pgfsys@useobject{currentmarker}{}%
\end{pgfscope}%
\begin{pgfscope}%
\pgfsys@transformshift{1.278130in}{0.755206in}%
\pgfsys@useobject{currentmarker}{}%
\end{pgfscope}%
\begin{pgfscope}%
\pgfsys@transformshift{1.307540in}{0.754621in}%
\pgfsys@useobject{currentmarker}{}%
\end{pgfscope}%
\begin{pgfscope}%
\pgfsys@transformshift{1.308402in}{0.754533in}%
\pgfsys@useobject{currentmarker}{}%
\end{pgfscope}%
\begin{pgfscope}%
\pgfsys@transformshift{1.320633in}{0.753865in}%
\pgfsys@useobject{currentmarker}{}%
\end{pgfscope}%
\begin{pgfscope}%
\pgfsys@transformshift{1.341138in}{0.753397in}%
\pgfsys@useobject{currentmarker}{}%
\end{pgfscope}%
\begin{pgfscope}%
\pgfsys@transformshift{1.344495in}{0.753033in}%
\pgfsys@useobject{currentmarker}{}%
\end{pgfscope}%
\begin{pgfscope}%
\pgfsys@transformshift{1.351667in}{0.752560in}%
\pgfsys@useobject{currentmarker}{}%
\end{pgfscope}%
\begin{pgfscope}%
\pgfsys@transformshift{1.389197in}{0.752350in}%
\pgfsys@useobject{currentmarker}{}%
\end{pgfscope}%
\begin{pgfscope}%
\pgfsys@transformshift{1.400715in}{0.752063in}%
\pgfsys@useobject{currentmarker}{}%
\end{pgfscope}%
\begin{pgfscope}%
\pgfsys@transformshift{1.414589in}{0.751918in}%
\pgfsys@useobject{currentmarker}{}%
\end{pgfscope}%
\begin{pgfscope}%
\pgfsys@transformshift{1.417824in}{0.751904in}%
\pgfsys@useobject{currentmarker}{}%
\end{pgfscope}%
\begin{pgfscope}%
\pgfsys@transformshift{1.418761in}{0.751837in}%
\pgfsys@useobject{currentmarker}{}%
\end{pgfscope}%
\begin{pgfscope}%
\pgfsys@transformshift{1.419139in}{0.751796in}%
\pgfsys@useobject{currentmarker}{}%
\end{pgfscope}%
\begin{pgfscope}%
\pgfsys@transformshift{1.419462in}{0.751760in}%
\pgfsys@useobject{currentmarker}{}%
\end{pgfscope}%
\begin{pgfscope}%
\pgfsys@transformshift{1.421141in}{0.751622in}%
\pgfsys@useobject{currentmarker}{}%
\end{pgfscope}%
\begin{pgfscope}%
\pgfsys@transformshift{1.422574in}{0.751405in}%
\pgfsys@useobject{currentmarker}{}%
\end{pgfscope}%
\begin{pgfscope}%
\pgfsys@transformshift{1.441321in}{0.749466in}%
\pgfsys@useobject{currentmarker}{}%
\end{pgfscope}%
\begin{pgfscope}%
\pgfsys@transformshift{1.448644in}{0.748886in}%
\pgfsys@useobject{currentmarker}{}%
\end{pgfscope}%
\begin{pgfscope}%
\pgfsys@transformshift{1.528667in}{0.747888in}%
\pgfsys@useobject{currentmarker}{}%
\end{pgfscope}%
\begin{pgfscope}%
\pgfsys@transformshift{1.605641in}{0.747867in}%
\pgfsys@useobject{currentmarker}{}%
\end{pgfscope}%
\begin{pgfscope}%
\pgfsys@transformshift{1.606745in}{0.746981in}%
\pgfsys@useobject{currentmarker}{}%
\end{pgfscope}%
\begin{pgfscope}%
\pgfsys@transformshift{1.610058in}{0.746614in}%
\pgfsys@useobject{currentmarker}{}%
\end{pgfscope}%
\begin{pgfscope}%
\pgfsys@transformshift{1.622014in}{0.746220in}%
\pgfsys@useobject{currentmarker}{}%
\end{pgfscope}%
\begin{pgfscope}%
\pgfsys@transformshift{1.625190in}{0.745498in}%
\pgfsys@useobject{currentmarker}{}%
\end{pgfscope}%
\begin{pgfscope}%
\pgfsys@transformshift{1.628853in}{0.745202in}%
\pgfsys@useobject{currentmarker}{}%
\end{pgfscope}%
\begin{pgfscope}%
\pgfsys@transformshift{1.631842in}{0.744948in}%
\pgfsys@useobject{currentmarker}{}%
\end{pgfscope}%
\begin{pgfscope}%
\pgfsys@transformshift{1.666551in}{0.744594in}%
\pgfsys@useobject{currentmarker}{}%
\end{pgfscope}%
\begin{pgfscope}%
\pgfsys@transformshift{1.678503in}{0.743559in}%
\pgfsys@useobject{currentmarker}{}%
\end{pgfscope}%
\begin{pgfscope}%
\pgfsys@transformshift{1.678976in}{0.743508in}%
\pgfsys@useobject{currentmarker}{}%
\end{pgfscope}%
\begin{pgfscope}%
\pgfsys@transformshift{1.678976in}{0.743508in}%
\pgfsys@useobject{currentmarker}{}%
\end{pgfscope}%
\begin{pgfscope}%
\pgfsys@transformshift{1.679156in}{0.743491in}%
\pgfsys@useobject{currentmarker}{}%
\end{pgfscope}%
\begin{pgfscope}%
\pgfsys@transformshift{1.679564in}{0.743446in}%
\pgfsys@useobject{currentmarker}{}%
\end{pgfscope}%
\begin{pgfscope}%
\pgfsys@transformshift{1.685511in}{0.742202in}%
\pgfsys@useobject{currentmarker}{}%
\end{pgfscope}%
\begin{pgfscope}%
\pgfsys@transformshift{1.685511in}{0.742202in}%
\pgfsys@useobject{currentmarker}{}%
\end{pgfscope}%
\begin{pgfscope}%
\pgfsys@transformshift{1.702822in}{0.741854in}%
\pgfsys@useobject{currentmarker}{}%
\end{pgfscope}%
\begin{pgfscope}%
\pgfsys@transformshift{1.707667in}{0.740400in}%
\pgfsys@useobject{currentmarker}{}%
\end{pgfscope}%
\begin{pgfscope}%
\pgfsys@transformshift{1.708990in}{0.740247in}%
\pgfsys@useobject{currentmarker}{}%
\end{pgfscope}%
\begin{pgfscope}%
\pgfsys@transformshift{1.713082in}{0.739976in}%
\pgfsys@useobject{currentmarker}{}%
\end{pgfscope}%
\begin{pgfscope}%
\pgfsys@transformshift{1.759569in}{0.739880in}%
\pgfsys@useobject{currentmarker}{}%
\end{pgfscope}%
\begin{pgfscope}%
\pgfsys@transformshift{1.759569in}{0.739880in}%
\pgfsys@useobject{currentmarker}{}%
\end{pgfscope}%
\begin{pgfscope}%
\pgfsys@transformshift{1.763752in}{0.739729in}%
\pgfsys@useobject{currentmarker}{}%
\end{pgfscope}%
\begin{pgfscope}%
\pgfsys@transformshift{1.775222in}{0.738819in}%
\pgfsys@useobject{currentmarker}{}%
\end{pgfscope}%
\begin{pgfscope}%
\pgfsys@transformshift{1.800509in}{0.738014in}%
\pgfsys@useobject{currentmarker}{}%
\end{pgfscope}%
\begin{pgfscope}%
\pgfsys@transformshift{1.840914in}{0.737153in}%
\pgfsys@useobject{currentmarker}{}%
\end{pgfscope}%
\begin{pgfscope}%
\pgfsys@transformshift{1.852165in}{0.736068in}%
\pgfsys@useobject{currentmarker}{}%
\end{pgfscope}%
\begin{pgfscope}%
\pgfsys@transformshift{1.877664in}{0.735418in}%
\pgfsys@useobject{currentmarker}{}%
\end{pgfscope}%
\begin{pgfscope}%
\pgfsys@transformshift{1.916797in}{0.734885in}%
\pgfsys@useobject{currentmarker}{}%
\end{pgfscope}%
\begin{pgfscope}%
\pgfsys@transformshift{1.920423in}{0.733745in}%
\pgfsys@useobject{currentmarker}{}%
\end{pgfscope}%
\begin{pgfscope}%
\pgfsys@transformshift{2.013663in}{0.731531in}%
\pgfsys@useobject{currentmarker}{}%
\end{pgfscope}%
\begin{pgfscope}%
\pgfsys@transformshift{2.176728in}{0.728600in}%
\pgfsys@useobject{currentmarker}{}%
\end{pgfscope}%
\begin{pgfscope}%
\pgfsys@transformshift{2.187729in}{0.726449in}%
\pgfsys@useobject{currentmarker}{}%
\end{pgfscope}%
\begin{pgfscope}%
\pgfsys@transformshift{2.188963in}{0.726384in}%
\pgfsys@useobject{currentmarker}{}%
\end{pgfscope}%
\begin{pgfscope}%
\pgfsys@transformshift{2.195614in}{0.726305in}%
\pgfsys@useobject{currentmarker}{}%
\end{pgfscope}%
\begin{pgfscope}%
\pgfsys@transformshift{2.196830in}{0.726204in}%
\pgfsys@useobject{currentmarker}{}%
\end{pgfscope}%
\begin{pgfscope}%
\pgfsys@transformshift{2.217170in}{0.725864in}%
\pgfsys@useobject{currentmarker}{}%
\end{pgfscope}%
\begin{pgfscope}%
\pgfsys@transformshift{2.331314in}{0.722875in}%
\pgfsys@useobject{currentmarker}{}%
\end{pgfscope}%
\begin{pgfscope}%
\pgfsys@transformshift{2.346054in}{0.721682in}%
\pgfsys@useobject{currentmarker}{}%
\end{pgfscope}%
\begin{pgfscope}%
\pgfsys@transformshift{2.382807in}{0.719754in}%
\pgfsys@useobject{currentmarker}{}%
\end{pgfscope}%
\begin{pgfscope}%
\pgfsys@transformshift{2.406189in}{0.719470in}%
\pgfsys@useobject{currentmarker}{}%
\end{pgfscope}%
\begin{pgfscope}%
\pgfsys@transformshift{2.443271in}{0.719418in}%
\pgfsys@useobject{currentmarker}{}%
\end{pgfscope}%
\begin{pgfscope}%
\pgfsys@transformshift{2.448194in}{0.718998in}%
\pgfsys@useobject{currentmarker}{}%
\end{pgfscope}%
\begin{pgfscope}%
\pgfsys@transformshift{2.464002in}{0.717470in}%
\pgfsys@useobject{currentmarker}{}%
\end{pgfscope}%
\begin{pgfscope}%
\pgfsys@transformshift{2.464440in}{0.717412in}%
\pgfsys@useobject{currentmarker}{}%
\end{pgfscope}%
\begin{pgfscope}%
\pgfsys@transformshift{2.497167in}{0.716312in}%
\pgfsys@useobject{currentmarker}{}%
\end{pgfscope}%
\begin{pgfscope}%
\pgfsys@transformshift{2.563540in}{0.715743in}%
\pgfsys@useobject{currentmarker}{}%
\end{pgfscope}%
\begin{pgfscope}%
\pgfsys@transformshift{2.628270in}{0.713086in}%
\pgfsys@useobject{currentmarker}{}%
\end{pgfscope}%
\begin{pgfscope}%
\pgfsys@transformshift{2.632948in}{0.712804in}%
\pgfsys@useobject{currentmarker}{}%
\end{pgfscope}%
\begin{pgfscope}%
\pgfsys@transformshift{2.869035in}{0.710469in}%
\pgfsys@useobject{currentmarker}{}%
\end{pgfscope}%
\begin{pgfscope}%
\pgfsys@transformshift{2.890311in}{0.707406in}%
\pgfsys@useobject{currentmarker}{}%
\end{pgfscope}%
\begin{pgfscope}%
\pgfsys@transformshift{2.934581in}{0.705830in}%
\pgfsys@useobject{currentmarker}{}%
\end{pgfscope}%
\begin{pgfscope}%
\pgfsys@transformshift{2.983194in}{0.704605in}%
\pgfsys@useobject{currentmarker}{}%
\end{pgfscope}%
\begin{pgfscope}%
\pgfsys@transformshift{3.241809in}{0.699660in}%
\pgfsys@useobject{currentmarker}{}%
\end{pgfscope}%
\begin{pgfscope}%
\pgfsys@transformshift{3.252817in}{0.698436in}%
\pgfsys@useobject{currentmarker}{}%
\end{pgfscope}%
\begin{pgfscope}%
\pgfsys@transformshift{3.523301in}{0.692124in}%
\pgfsys@useobject{currentmarker}{}%
\end{pgfscope}%
\begin{pgfscope}%
\pgfsys@transformshift{3.755809in}{0.689503in}%
\pgfsys@useobject{currentmarker}{}%
\end{pgfscope}%
\begin{pgfscope}%
\pgfsys@transformshift{3.791217in}{0.688607in}%
\pgfsys@useobject{currentmarker}{}%
\end{pgfscope}%
\begin{pgfscope}%
\pgfsys@transformshift{3.829576in}{0.687811in}%
\pgfsys@useobject{currentmarker}{}%
\end{pgfscope}%
\begin{pgfscope}%
\pgfsys@transformshift{3.953911in}{0.685576in}%
\pgfsys@useobject{currentmarker}{}%
\end{pgfscope}%
\begin{pgfscope}%
\pgfsys@transformshift{3.963043in}{0.685366in}%
\pgfsys@useobject{currentmarker}{}%
\end{pgfscope}%
\begin{pgfscope}%
\pgfsys@transformshift{4.123628in}{0.683245in}%
\pgfsys@useobject{currentmarker}{}%
\end{pgfscope}%
\begin{pgfscope}%
\pgfsys@transformshift{4.176025in}{0.682823in}%
\pgfsys@useobject{currentmarker}{}%
\end{pgfscope}%
\begin{pgfscope}%
\pgfsys@transformshift{4.182175in}{0.682726in}%
\pgfsys@useobject{currentmarker}{}%
\end{pgfscope}%
\begin{pgfscope}%
\pgfsys@transformshift{4.391441in}{0.682670in}%
\pgfsys@useobject{currentmarker}{}%
\end{pgfscope}%
\begin{pgfscope}%
\pgfsys@transformshift{4.391763in}{0.682661in}%
\pgfsys@useobject{currentmarker}{}%
\end{pgfscope}%
\begin{pgfscope}%
\pgfsys@transformshift{4.462854in}{0.682464in}%
\pgfsys@useobject{currentmarker}{}%
\end{pgfscope}%
\begin{pgfscope}%
\pgfsys@transformshift{4.549016in}{0.682007in}%
\pgfsys@useobject{currentmarker}{}%
\end{pgfscope}%
\begin{pgfscope}%
\pgfsys@transformshift{4.586127in}{0.681928in}%
\pgfsys@useobject{currentmarker}{}%
\end{pgfscope}%
\begin{pgfscope}%
\pgfsys@transformshift{4.603680in}{0.680832in}%
\pgfsys@useobject{currentmarker}{}%
\end{pgfscope}%
\begin{pgfscope}%
\pgfsys@transformshift{4.634758in}{0.680738in}%
\pgfsys@useobject{currentmarker}{}%
\end{pgfscope}%
\begin{pgfscope}%
\pgfsys@transformshift{4.661601in}{0.679515in}%
\pgfsys@useobject{currentmarker}{}%
\end{pgfscope}%
\begin{pgfscope}%
\pgfsys@transformshift{4.700040in}{0.679144in}%
\pgfsys@useobject{currentmarker}{}%
\end{pgfscope}%
\begin{pgfscope}%
\pgfsys@transformshift{4.708661in}{0.678566in}%
\pgfsys@useobject{currentmarker}{}%
\end{pgfscope}%
\begin{pgfscope}%
\pgfsys@transformshift{4.724535in}{0.678336in}%
\pgfsys@useobject{currentmarker}{}%
\end{pgfscope}%
\begin{pgfscope}%
\pgfsys@transformshift{4.737940in}{0.678102in}%
\pgfsys@useobject{currentmarker}{}%
\end{pgfscope}%
\begin{pgfscope}%
\pgfsys@transformshift{4.889518in}{0.677863in}%
\pgfsys@useobject{currentmarker}{}%
\end{pgfscope}%
\begin{pgfscope}%
\pgfsys@transformshift{4.899791in}{0.677682in}%
\pgfsys@useobject{currentmarker}{}%
\end{pgfscope}%
\begin{pgfscope}%
\pgfsys@transformshift{4.914884in}{0.677644in}%
\pgfsys@useobject{currentmarker}{}%
\end{pgfscope}%
\begin{pgfscope}%
\pgfsys@transformshift{4.940575in}{0.677405in}%
\pgfsys@useobject{currentmarker}{}%
\end{pgfscope}%
\begin{pgfscope}%
\pgfsys@transformshift{4.954913in}{0.677128in}%
\pgfsys@useobject{currentmarker}{}%
\end{pgfscope}%
\begin{pgfscope}%
\pgfsys@transformshift{4.970414in}{0.676799in}%
\pgfsys@useobject{currentmarker}{}%
\end{pgfscope}%
\begin{pgfscope}%
\pgfsys@transformshift{5.061952in}{0.676153in}%
\pgfsys@useobject{currentmarker}{}%
\end{pgfscope}%
\begin{pgfscope}%
\pgfsys@transformshift{5.112107in}{0.675923in}%
\pgfsys@useobject{currentmarker}{}%
\end{pgfscope}%
\begin{pgfscope}%
\pgfsys@transformshift{5.117964in}{0.675193in}%
\pgfsys@useobject{currentmarker}{}%
\end{pgfscope}%
\begin{pgfscope}%
\pgfsys@transformshift{5.136924in}{0.674382in}%
\pgfsys@useobject{currentmarker}{}%
\end{pgfscope}%
\begin{pgfscope}%
\pgfsys@transformshift{5.252737in}{0.674245in}%
\pgfsys@useobject{currentmarker}{}%
\end{pgfscope}%
\begin{pgfscope}%
\pgfsys@transformshift{5.252737in}{0.674245in}%
\pgfsys@useobject{currentmarker}{}%
\end{pgfscope}%
\begin{pgfscope}%
\pgfsys@transformshift{5.261268in}{0.673970in}%
\pgfsys@useobject{currentmarker}{}%
\end{pgfscope}%
\begin{pgfscope}%
\pgfsys@transformshift{5.338744in}{0.673524in}%
\pgfsys@useobject{currentmarker}{}%
\end{pgfscope}%
\begin{pgfscope}%
\pgfsys@transformshift{5.349847in}{0.673027in}%
\pgfsys@useobject{currentmarker}{}%
\end{pgfscope}%
\begin{pgfscope}%
\pgfsys@transformshift{5.507614in}{0.672879in}%
\pgfsys@useobject{currentmarker}{}%
\end{pgfscope}%
\begin{pgfscope}%
\pgfsys@transformshift{5.537814in}{0.672431in}%
\pgfsys@useobject{currentmarker}{}%
\end{pgfscope}%
\begin{pgfscope}%
\pgfsys@transformshift{5.595362in}{0.672311in}%
\pgfsys@useobject{currentmarker}{}%
\end{pgfscope}%
\begin{pgfscope}%
\pgfsys@transformshift{5.638666in}{0.671980in}%
\pgfsys@useobject{currentmarker}{}%
\end{pgfscope}%
\begin{pgfscope}%
\pgfsys@transformshift{5.696293in}{0.671853in}%
\pgfsys@useobject{currentmarker}{}%
\end{pgfscope}%
\begin{pgfscope}%
\pgfsys@transformshift{5.823193in}{0.670783in}%
\pgfsys@useobject{currentmarker}{}%
\end{pgfscope}%
\begin{pgfscope}%
\pgfsys@transformshift{5.988842in}{0.670138in}%
\pgfsys@useobject{currentmarker}{}%
\end{pgfscope}%
\end{pgfscope}%
\begin{pgfscope}%
\pgfpathrectangle{\pgfqpoint{0.688192in}{0.670138in}}{\pgfqpoint{7.111808in}{5.129862in}}%
\pgfusepath{clip}%
\pgfsetrectcap%
\pgfsetroundjoin%
\pgfsetlinewidth{1.505625pt}%
\definecolor{currentstroke}{rgb}{0.501961,0.501961,0.501961}%
\pgfsetstrokecolor{currentstroke}%
\pgfsetstrokeopacity{0.500000}%
\pgfsetdash{}{0pt}%
\pgfpathmoveto{\pgfqpoint{1.969284in}{5.800000in}}%
\pgfpathlineto{\pgfqpoint{1.973421in}{4.548021in}}%
\pgfpathlineto{\pgfqpoint{1.974650in}{3.907920in}}%
\pgfpathlineto{\pgfqpoint{1.982138in}{3.790001in}}%
\pgfpathlineto{\pgfqpoint{1.986338in}{3.002147in}}%
\pgfpathlineto{\pgfqpoint{1.991273in}{2.787704in}}%
\pgfpathlineto{\pgfqpoint{1.993739in}{2.786443in}}%
\pgfpathlineto{\pgfqpoint{1.999791in}{2.524301in}}%
\pgfpathlineto{\pgfqpoint{2.000087in}{2.376976in}}%
\pgfpathlineto{\pgfqpoint{2.003357in}{2.217353in}}%
\pgfpathlineto{\pgfqpoint{2.005488in}{2.178682in}}%
\pgfpathlineto{\pgfqpoint{2.006628in}{2.140716in}}%
\pgfpathlineto{\pgfqpoint{2.007036in}{1.926090in}}%
\pgfpathlineto{\pgfqpoint{2.011691in}{1.887598in}}%
\pgfpathlineto{\pgfqpoint{2.016142in}{1.842782in}}%
\pgfpathlineto{\pgfqpoint{2.018165in}{1.663687in}}%
\pgfpathlineto{\pgfqpoint{2.021126in}{1.608339in}}%
\pgfpathlineto{\pgfqpoint{2.021324in}{1.561595in}}%
\pgfpathlineto{\pgfqpoint{2.026671in}{1.538967in}}%
\pgfpathlineto{\pgfqpoint{2.034335in}{1.533724in}}%
\pgfpathlineto{\pgfqpoint{2.037047in}{1.383368in}}%
\pgfpathlineto{\pgfqpoint{2.043625in}{1.346822in}}%
\pgfpathlineto{\pgfqpoint{2.055757in}{1.340035in}}%
\pgfpathlineto{\pgfqpoint{2.059905in}{1.282975in}}%
\pgfpathlineto{\pgfqpoint{2.064160in}{1.272993in}}%
\pgfpathlineto{\pgfqpoint{2.067039in}{1.265494in}}%
\pgfpathlineto{\pgfqpoint{2.073088in}{1.243778in}}%
\pgfpathlineto{\pgfqpoint{2.080438in}{1.212387in}}%
\pgfpathlineto{\pgfqpoint{2.081013in}{1.207005in}}%
\pgfpathlineto{\pgfqpoint{2.081364in}{1.200759in}}%
\pgfpathlineto{\pgfqpoint{2.085887in}{1.189461in}}%
\pgfpathlineto{\pgfqpoint{2.105507in}{1.157172in}}%
\pgfpathlineto{\pgfqpoint{2.126372in}{1.112181in}}%
\pgfpathlineto{\pgfqpoint{2.154690in}{1.080082in}}%
\pgfpathlineto{\pgfqpoint{2.180621in}{1.069301in}}%
\pgfpathlineto{\pgfqpoint{2.181678in}{1.055066in}}%
\pgfpathlineto{\pgfqpoint{2.183210in}{1.030237in}}%
\pgfpathlineto{\pgfqpoint{2.186821in}{1.025534in}}%
\pgfpathlineto{\pgfqpoint{2.187316in}{1.021388in}}%
\pgfpathlineto{\pgfqpoint{2.187820in}{1.021295in}}%
\pgfpathlineto{\pgfqpoint{2.197116in}{1.017387in}}%
\pgfpathlineto{\pgfqpoint{2.203773in}{1.006799in}}%
\pgfpathlineto{\pgfqpoint{2.233341in}{0.990353in}}%
\pgfpathlineto{\pgfqpoint{2.237213in}{0.981626in}}%
\pgfpathlineto{\pgfqpoint{2.237327in}{0.972243in}}%
\pgfpathlineto{\pgfqpoint{2.243682in}{0.970693in}}%
\pgfpathlineto{\pgfqpoint{2.259805in}{0.953394in}}%
\pgfpathlineto{\pgfqpoint{2.265806in}{0.953331in}}%
\pgfpathlineto{\pgfqpoint{2.266905in}{0.947452in}}%
\pgfpathlineto{\pgfqpoint{2.326075in}{0.932346in}}%
\pgfpathlineto{\pgfqpoint{2.330708in}{0.902849in}}%
\pgfpathlineto{\pgfqpoint{2.333340in}{0.900575in}}%
\pgfpathlineto{\pgfqpoint{2.340875in}{0.898499in}}%
\pgfpathlineto{\pgfqpoint{2.342709in}{0.896623in}}%
\pgfpathlineto{\pgfqpoint{2.361252in}{0.881589in}}%
\pgfpathlineto{\pgfqpoint{2.362110in}{0.880286in}}%
\pgfpathlineto{\pgfqpoint{2.367964in}{0.878388in}}%
\pgfpathlineto{\pgfqpoint{2.368007in}{0.877760in}}%
\pgfpathlineto{\pgfqpoint{2.369539in}{0.874257in}}%
\pgfpathlineto{\pgfqpoint{2.374247in}{0.872975in}}%
\pgfpathlineto{\pgfqpoint{2.377046in}{0.872835in}}%
\pgfpathlineto{\pgfqpoint{2.388301in}{0.867385in}}%
\pgfpathlineto{\pgfqpoint{2.390474in}{0.863028in}}%
\pgfpathlineto{\pgfqpoint{2.411370in}{0.862723in}}%
\pgfpathlineto{\pgfqpoint{2.412807in}{0.859073in}}%
\pgfpathlineto{\pgfqpoint{2.415726in}{0.858332in}}%
\pgfpathlineto{\pgfqpoint{2.424341in}{0.857341in}}%
\pgfpathlineto{\pgfqpoint{2.444999in}{0.856179in}}%
\pgfpathlineto{\pgfqpoint{2.492622in}{0.855806in}}%
\pgfpathlineto{\pgfqpoint{2.597676in}{0.852253in}}%
\pgfpathlineto{\pgfqpoint{2.618217in}{0.850131in}}%
\pgfpathlineto{\pgfqpoint{2.651515in}{0.849390in}}%
\pgfpathlineto{\pgfqpoint{2.664970in}{0.848655in}}%
\pgfpathlineto{\pgfqpoint{2.687525in}{0.848141in}}%
\pgfpathlineto{\pgfqpoint{2.691218in}{0.847740in}}%
\pgfpathlineto{\pgfqpoint{2.699107in}{0.847220in}}%
\pgfpathlineto{\pgfqpoint{2.753060in}{0.846673in}}%
\pgfpathlineto{\pgfqpoint{2.775528in}{0.846189in}}%
\pgfpathlineto{\pgfqpoint{2.777105in}{0.845949in}}%
\pgfpathlineto{\pgfqpoint{2.805781in}{0.843178in}}%
\pgfpathlineto{\pgfqpoint{2.893807in}{0.842080in}}%
\pgfpathlineto{\pgfqpoint{2.978478in}{0.842058in}}%
\pgfpathlineto{\pgfqpoint{2.979693in}{0.841083in}}%
\pgfpathlineto{\pgfqpoint{2.983337in}{0.840679in}}%
\pgfpathlineto{\pgfqpoint{2.996489in}{0.840246in}}%
\pgfpathlineto{\pgfqpoint{2.999983in}{0.839451in}}%
\pgfpathlineto{\pgfqpoint{3.007299in}{0.838846in}}%
\pgfpathlineto{\pgfqpoint{3.045479in}{0.838457in}}%
\pgfpathlineto{\pgfqpoint{3.059794in}{0.837194in}}%
\pgfpathlineto{\pgfqpoint{3.066335in}{0.835826in}}%
\pgfpathlineto{\pgfqpoint{3.085378in}{0.835443in}}%
\pgfpathlineto{\pgfqpoint{3.092163in}{0.833675in}}%
\pgfpathlineto{\pgfqpoint{3.096664in}{0.833378in}}%
\pgfpathlineto{\pgfqpoint{3.152400in}{0.833106in}}%
\pgfpathlineto{\pgfqpoint{3.165018in}{0.832105in}}%
\pgfpathlineto{\pgfqpoint{3.192834in}{0.831220in}}%
\pgfpathlineto{\pgfqpoint{3.237278in}{0.830273in}}%
\pgfpathlineto{\pgfqpoint{3.249654in}{0.829079in}}%
\pgfpathlineto{\pgfqpoint{3.277703in}{0.828363in}}%
\pgfpathlineto{\pgfqpoint{3.320750in}{0.827778in}}%
\pgfpathlineto{\pgfqpoint{3.324738in}{0.826524in}}%
\pgfpathlineto{\pgfqpoint{3.427303in}{0.824088in}}%
\pgfpathlineto{\pgfqpoint{3.606674in}{0.820864in}}%
\pgfpathlineto{\pgfqpoint{3.620132in}{0.818426in}}%
\pgfpathlineto{\pgfqpoint{3.651161in}{0.817855in}}%
\pgfpathlineto{\pgfqpoint{3.776719in}{0.814566in}}%
\pgfpathlineto{\pgfqpoint{3.792932in}{0.813254in}}%
\pgfpathlineto{\pgfqpoint{3.833361in}{0.811133in}}%
\pgfpathlineto{\pgfqpoint{3.859081in}{0.810821in}}%
\pgfpathlineto{\pgfqpoint{3.899872in}{0.810764in}}%
\pgfpathlineto{\pgfqpoint{3.923158in}{0.808557in}}%
\pgfpathlineto{\pgfqpoint{3.959157in}{0.807348in}}%
\pgfpathlineto{\pgfqpoint{4.032167in}{0.806721in}}%
\pgfpathlineto{\pgfqpoint{4.108516in}{0.803488in}}%
\pgfpathlineto{\pgfqpoint{4.368212in}{0.800920in}}%
\pgfpathlineto{\pgfqpoint{4.391616in}{0.797551in}}%
\pgfpathlineto{\pgfqpoint{4.440312in}{0.795817in}}%
\pgfpathlineto{\pgfqpoint{4.493787in}{0.794470in}}%
\pgfpathlineto{\pgfqpoint{4.778264in}{0.789030in}}%
\pgfpathlineto{\pgfqpoint{4.790372in}{0.787683in}}%
\pgfpathlineto{\pgfqpoint{5.087904in}{0.780740in}}%
\pgfpathlineto{\pgfqpoint{5.343664in}{0.777857in}}%
\pgfpathlineto{\pgfqpoint{5.424807in}{0.775996in}}%
\pgfpathlineto{\pgfqpoint{5.571620in}{0.773307in}}%
\pgfpathlineto{\pgfqpoint{5.748264in}{0.770973in}}%
\pgfpathlineto{\pgfqpoint{5.812666in}{0.770402in}}%
\pgfpathlineto{\pgfqpoint{6.121412in}{0.770115in}}%
\pgfpathlineto{\pgfqpoint{6.257013in}{0.769525in}}%
\pgfpathlineto{\pgfqpoint{6.276321in}{0.768320in}}%
\pgfpathlineto{\pgfqpoint{6.310507in}{0.768216in}}%
\pgfpathlineto{\pgfqpoint{6.340034in}{0.766870in}}%
\pgfpathlineto{\pgfqpoint{6.382318in}{0.766462in}}%
\pgfpathlineto{\pgfqpoint{6.391800in}{0.765827in}}%
\pgfpathlineto{\pgfqpoint{6.424008in}{0.765316in}}%
\pgfpathlineto{\pgfqpoint{6.618646in}{0.764812in}}%
\pgfpathlineto{\pgfqpoint{6.662678in}{0.764245in}}%
\pgfpathlineto{\pgfqpoint{6.679729in}{0.763883in}}%
\pgfpathlineto{\pgfqpoint{6.835591in}{0.762919in}}%
\pgfpathlineto{\pgfqpoint{6.842034in}{0.762116in}}%
\pgfpathlineto{\pgfqpoint{6.862890in}{0.761224in}}%
\pgfpathlineto{\pgfqpoint{6.990284in}{0.761073in}}%
\pgfpathlineto{\pgfqpoint{6.999668in}{0.760771in}}%
\pgfpathlineto{\pgfqpoint{7.084892in}{0.760280in}}%
\pgfpathlineto{\pgfqpoint{7.097105in}{0.759733in}}%
\pgfpathlineto{\pgfqpoint{7.270649in}{0.759571in}}%
\pgfpathlineto{\pgfqpoint{7.303869in}{0.759078in}}%
\pgfpathlineto{\pgfqpoint{7.478196in}{0.758442in}}%
\pgfpathlineto{\pgfqpoint{7.617785in}{0.757265in}}%
\pgfpathlineto{\pgfqpoint{7.800000in}{0.756556in}}%
\pgfpathlineto{\pgfqpoint{7.800000in}{0.756556in}}%
\pgfusepath{stroke}%
\end{pgfscope}%
\begin{pgfscope}%
\pgfsetrectcap%
\pgfsetmiterjoin%
\pgfsetlinewidth{0.803000pt}%
\definecolor{currentstroke}{rgb}{0.000000,0.000000,0.000000}%
\pgfsetstrokecolor{currentstroke}%
\pgfsetdash{}{0pt}%
\pgfpathmoveto{\pgfqpoint{0.688192in}{0.670138in}}%
\pgfpathlineto{\pgfqpoint{0.688192in}{5.800000in}}%
\pgfusepath{stroke}%
\end{pgfscope}%
\begin{pgfscope}%
\pgfsetrectcap%
\pgfsetmiterjoin%
\pgfsetlinewidth{0.803000pt}%
\definecolor{currentstroke}{rgb}{0.000000,0.000000,0.000000}%
\pgfsetstrokecolor{currentstroke}%
\pgfsetdash{}{0pt}%
\pgfpathmoveto{\pgfqpoint{7.800000in}{0.670138in}}%
\pgfpathlineto{\pgfqpoint{7.800000in}{5.800000in}}%
\pgfusepath{stroke}%
\end{pgfscope}%
\begin{pgfscope}%
\pgfsetrectcap%
\pgfsetmiterjoin%
\pgfsetlinewidth{0.803000pt}%
\definecolor{currentstroke}{rgb}{0.000000,0.000000,0.000000}%
\pgfsetstrokecolor{currentstroke}%
\pgfsetdash{}{0pt}%
\pgfpathmoveto{\pgfqpoint{0.688192in}{0.670138in}}%
\pgfpathlineto{\pgfqpoint{7.800000in}{0.670138in}}%
\pgfusepath{stroke}%
\end{pgfscope}%
\begin{pgfscope}%
\pgfsetrectcap%
\pgfsetmiterjoin%
\pgfsetlinewidth{0.803000pt}%
\definecolor{currentstroke}{rgb}{0.000000,0.000000,0.000000}%
\pgfsetstrokecolor{currentstroke}%
\pgfsetdash{}{0pt}%
\pgfpathmoveto{\pgfqpoint{0.688192in}{5.800000in}}%
\pgfpathlineto{\pgfqpoint{7.800000in}{5.800000in}}%
\pgfusepath{stroke}%
\end{pgfscope}%
\begin{pgfscope}%
\pgfsetbuttcap%
\pgfsetmiterjoin%
\definecolor{currentfill}{rgb}{1.000000,1.000000,1.000000}%
\pgfsetfillcolor{currentfill}%
\pgfsetlinewidth{1.003750pt}%
\definecolor{currentstroke}{rgb}{0.800000,0.800000,0.800000}%
\pgfsetstrokecolor{currentstroke}%
\pgfsetdash{}{0pt}%
\pgfpathmoveto{\pgfqpoint{5.082600in}{3.999924in}}%
\pgfpathlineto{\pgfqpoint{7.644444in}{3.999924in}}%
\pgfpathquadraticcurveto{\pgfqpoint{7.688889in}{3.999924in}}{\pgfqpoint{7.688889in}{4.044368in}}%
\pgfpathlineto{\pgfqpoint{7.688889in}{5.644444in}}%
\pgfpathquadraticcurveto{\pgfqpoint{7.688889in}{5.688889in}}{\pgfqpoint{7.644444in}{5.688889in}}%
\pgfpathlineto{\pgfqpoint{5.082600in}{5.688889in}}%
\pgfpathquadraticcurveto{\pgfqpoint{5.038156in}{5.688889in}}{\pgfqpoint{5.038156in}{5.644444in}}%
\pgfpathlineto{\pgfqpoint{5.038156in}{4.044368in}}%
\pgfpathquadraticcurveto{\pgfqpoint{5.038156in}{3.999924in}}{\pgfqpoint{5.082600in}{3.999924in}}%
\pgfpathlineto{\pgfqpoint{5.082600in}{3.999924in}}%
\pgfpathclose%
\pgfusepath{stroke,fill}%
\end{pgfscope}%
\begin{pgfscope}%
\pgfsetbuttcap%
\pgfsetroundjoin%
\pgfsetlinewidth{1.003750pt}%
\definecolor{currentstroke}{rgb}{0.000000,0.000000,0.000000}%
\pgfsetstrokecolor{currentstroke}%
\pgfsetdash{}{0pt}%
\pgfpathmoveto{\pgfqpoint{5.349267in}{5.450583in}}%
\pgfpathcurveto{\pgfqpoint{5.360162in}{5.450583in}}{\pgfqpoint{5.370613in}{5.454912in}}{\pgfqpoint{5.378318in}{5.462616in}}%
\pgfpathcurveto{\pgfqpoint{5.386022in}{5.470320in}}{\pgfqpoint{5.390351in}{5.480771in}}{\pgfqpoint{5.390351in}{5.491667in}}%
\pgfpathcurveto{\pgfqpoint{5.390351in}{5.502562in}}{\pgfqpoint{5.386022in}{5.513013in}}{\pgfqpoint{5.378318in}{5.520717in}}%
\pgfpathcurveto{\pgfqpoint{5.370613in}{5.528422in}}{\pgfqpoint{5.360162in}{5.532751in}}{\pgfqpoint{5.349267in}{5.532751in}}%
\pgfpathcurveto{\pgfqpoint{5.338371in}{5.532751in}}{\pgfqpoint{5.327920in}{5.528422in}}{\pgfqpoint{5.320216in}{5.520717in}}%
\pgfpathcurveto{\pgfqpoint{5.312512in}{5.513013in}}{\pgfqpoint{5.308183in}{5.502562in}}{\pgfqpoint{5.308183in}{5.491667in}}%
\pgfpathcurveto{\pgfqpoint{5.308183in}{5.480771in}}{\pgfqpoint{5.312512in}{5.470320in}}{\pgfqpoint{5.320216in}{5.462616in}}%
\pgfpathcurveto{\pgfqpoint{5.327920in}{5.454912in}}{\pgfqpoint{5.338371in}{5.450583in}}{\pgfqpoint{5.349267in}{5.450583in}}%
\pgfpathlineto{\pgfqpoint{5.349267in}{5.450583in}}%
\pgfpathclose%
\pgfusepath{stroke}%
\end{pgfscope}%
\begin{pgfscope}%
\definecolor{textcolor}{rgb}{0.000000,0.000000,0.000000}%
\pgfsetstrokecolor{textcolor}%
\pgfsetfillcolor{textcolor}%
\pgftext[x=5.749267in,y=5.433333in,left,base]{\color{textcolor}{\rmfamily\fontsize{16.000000}{19.200000}\selectfont\catcode`\^=\active\def^{\ifmmode\sp\else\^{}\fi}\catcode`\%=\active\def%{\%}tested points}}%
\end{pgfscope}%
\begin{pgfscope}%
\pgfsetbuttcap%
\pgfsetroundjoin%
\definecolor{currentfill}{rgb}{0.172549,0.627451,0.172549}%
\pgfsetfillcolor{currentfill}%
\pgfsetlinewidth{1.003750pt}%
\definecolor{currentstroke}{rgb}{0.172549,0.627451,0.172549}%
\pgfsetstrokecolor{currentstroke}%
\pgfsetdash{}{0pt}%
\pgfsys@defobject{currentmarker}{\pgfqpoint{-0.041084in}{-0.041084in}}{\pgfqpoint{0.041084in}{0.041084in}}{%
\pgfpathmoveto{\pgfqpoint{0.000000in}{-0.041084in}}%
\pgfpathcurveto{\pgfqpoint{0.010896in}{-0.041084in}}{\pgfqpoint{0.021346in}{-0.036755in}}{\pgfqpoint{0.029051in}{-0.029051in}}%
\pgfpathcurveto{\pgfqpoint{0.036755in}{-0.021346in}}{\pgfqpoint{0.041084in}{-0.010896in}}{\pgfqpoint{0.041084in}{0.000000in}}%
\pgfpathcurveto{\pgfqpoint{0.041084in}{0.010896in}}{\pgfqpoint{0.036755in}{0.021346in}}{\pgfqpoint{0.029051in}{0.029051in}}%
\pgfpathcurveto{\pgfqpoint{0.021346in}{0.036755in}}{\pgfqpoint{0.010896in}{0.041084in}}{\pgfqpoint{0.000000in}{0.041084in}}%
\pgfpathcurveto{\pgfqpoint{-0.010896in}{0.041084in}}{\pgfqpoint{-0.021346in}{0.036755in}}{\pgfqpoint{-0.029051in}{0.029051in}}%
\pgfpathcurveto{\pgfqpoint{-0.036755in}{0.021346in}}{\pgfqpoint{-0.041084in}{0.010896in}}{\pgfqpoint{-0.041084in}{0.000000in}}%
\pgfpathcurveto{\pgfqpoint{-0.041084in}{-0.010896in}}{\pgfqpoint{-0.036755in}{-0.021346in}}{\pgfqpoint{-0.029051in}{-0.029051in}}%
\pgfpathcurveto{\pgfqpoint{-0.021346in}{-0.036755in}}{\pgfqpoint{-0.010896in}{-0.041084in}}{\pgfqpoint{0.000000in}{-0.041084in}}%
\pgfpathlineto{\pgfqpoint{0.000000in}{-0.041084in}}%
\pgfpathclose%
\pgfusepath{stroke,fill}%
}%
\begin{pgfscope}%
\pgfsys@transformshift{5.349267in}{5.167207in}%
\pgfsys@useobject{currentmarker}{}%
\end{pgfscope}%
\end{pgfscope}%
\begin{pgfscope}%
\definecolor{textcolor}{rgb}{0.000000,0.000000,0.000000}%
\pgfsetstrokecolor{textcolor}%
\pgfsetfillcolor{textcolor}%
\pgftext[x=5.749267in,y=5.108874in,left,base]{\color{textcolor}{\rmfamily\fontsize{16.000000}{19.200000}\selectfont\catcode`\^=\active\def^{\ifmmode\sp\else\^{}\fi}\catcode`\%=\active\def%{\%}sub-optimal}}%
\end{pgfscope}%
\begin{pgfscope}%
\pgfsetbuttcap%
\pgfsetroundjoin%
\definecolor{currentfill}{rgb}{0.839216,0.152941,0.156863}%
\pgfsetfillcolor{currentfill}%
\pgfsetlinewidth{1.003750pt}%
\definecolor{currentstroke}{rgb}{0.839216,0.152941,0.156863}%
\pgfsetstrokecolor{currentstroke}%
\pgfsetdash{}{0pt}%
\pgfsys@defobject{currentmarker}{\pgfqpoint{-0.041084in}{-0.041084in}}{\pgfqpoint{0.041084in}{0.041084in}}{%
\pgfpathmoveto{\pgfqpoint{0.000000in}{-0.041084in}}%
\pgfpathcurveto{\pgfqpoint{0.010896in}{-0.041084in}}{\pgfqpoint{0.021346in}{-0.036755in}}{\pgfqpoint{0.029051in}{-0.029051in}}%
\pgfpathcurveto{\pgfqpoint{0.036755in}{-0.021346in}}{\pgfqpoint{0.041084in}{-0.010896in}}{\pgfqpoint{0.041084in}{0.000000in}}%
\pgfpathcurveto{\pgfqpoint{0.041084in}{0.010896in}}{\pgfqpoint{0.036755in}{0.021346in}}{\pgfqpoint{0.029051in}{0.029051in}}%
\pgfpathcurveto{\pgfqpoint{0.021346in}{0.036755in}}{\pgfqpoint{0.010896in}{0.041084in}}{\pgfqpoint{0.000000in}{0.041084in}}%
\pgfpathcurveto{\pgfqpoint{-0.010896in}{0.041084in}}{\pgfqpoint{-0.021346in}{0.036755in}}{\pgfqpoint{-0.029051in}{0.029051in}}%
\pgfpathcurveto{\pgfqpoint{-0.036755in}{0.021346in}}{\pgfqpoint{-0.041084in}{0.010896in}}{\pgfqpoint{-0.041084in}{0.000000in}}%
\pgfpathcurveto{\pgfqpoint{-0.041084in}{-0.010896in}}{\pgfqpoint{-0.036755in}{-0.021346in}}{\pgfqpoint{-0.029051in}{-0.029051in}}%
\pgfpathcurveto{\pgfqpoint{-0.021346in}{-0.036755in}}{\pgfqpoint{-0.010896in}{-0.041084in}}{\pgfqpoint{0.000000in}{-0.041084in}}%
\pgfpathlineto{\pgfqpoint{0.000000in}{-0.041084in}}%
\pgfpathclose%
\pgfusepath{stroke,fill}%
}%
\begin{pgfscope}%
\pgfsys@transformshift{5.349267in}{4.842747in}%
\pgfsys@useobject{currentmarker}{}%
\end{pgfscope}%
\end{pgfscope}%
\begin{pgfscope}%
\definecolor{textcolor}{rgb}{0.000000,0.000000,0.000000}%
\pgfsetstrokecolor{textcolor}%
\pgfsetfillcolor{textcolor}%
\pgftext[x=5.749267in,y=4.784414in,left,base]{\color{textcolor}{\rmfamily\fontsize{16.000000}{19.200000}\selectfont\catcode`\^=\active\def^{\ifmmode\sp\else\^{}\fi}\catcode`\%=\active\def%{\%}selected points}}%
\end{pgfscope}%
\begin{pgfscope}%
\pgfsetrectcap%
\pgfsetroundjoin%
\pgfsetlinewidth{2.509375pt}%
\definecolor{currentstroke}{rgb}{0.000000,0.000000,0.000000}%
\pgfsetstrokecolor{currentstroke}%
\pgfsetdash{}{0pt}%
\pgfpathmoveto{\pgfqpoint{5.127045in}{4.537732in}}%
\pgfpathlineto{\pgfqpoint{5.349267in}{4.537732in}}%
\pgfpathlineto{\pgfqpoint{5.571489in}{4.537732in}}%
\pgfusepath{stroke}%
\end{pgfscope}%
\begin{pgfscope}%
\pgfsetbuttcap%
\pgfsetroundjoin%
\definecolor{currentfill}{rgb}{0.000000,0.000000,0.000000}%
\pgfsetfillcolor{currentfill}%
\pgfsetlinewidth{1.003750pt}%
\definecolor{currentstroke}{rgb}{0.000000,0.000000,0.000000}%
\pgfsetstrokecolor{currentstroke}%
\pgfsetdash{}{0pt}%
\pgfsys@defobject{currentmarker}{\pgfqpoint{-0.020833in}{-0.020833in}}{\pgfqpoint{0.020833in}{0.020833in}}{%
\pgfpathmoveto{\pgfqpoint{0.000000in}{-0.020833in}}%
\pgfpathcurveto{\pgfqpoint{0.005525in}{-0.020833in}}{\pgfqpoint{0.010825in}{-0.018638in}}{\pgfqpoint{0.014731in}{-0.014731in}}%
\pgfpathcurveto{\pgfqpoint{0.018638in}{-0.010825in}}{\pgfqpoint{0.020833in}{-0.005525in}}{\pgfqpoint{0.020833in}{0.000000in}}%
\pgfpathcurveto{\pgfqpoint{0.020833in}{0.005525in}}{\pgfqpoint{0.018638in}{0.010825in}}{\pgfqpoint{0.014731in}{0.014731in}}%
\pgfpathcurveto{\pgfqpoint{0.010825in}{0.018638in}}{\pgfqpoint{0.005525in}{0.020833in}}{\pgfqpoint{0.000000in}{0.020833in}}%
\pgfpathcurveto{\pgfqpoint{-0.005525in}{0.020833in}}{\pgfqpoint{-0.010825in}{0.018638in}}{\pgfqpoint{-0.014731in}{0.014731in}}%
\pgfpathcurveto{\pgfqpoint{-0.018638in}{0.010825in}}{\pgfqpoint{-0.020833in}{0.005525in}}{\pgfqpoint{-0.020833in}{0.000000in}}%
\pgfpathcurveto{\pgfqpoint{-0.020833in}{-0.005525in}}{\pgfqpoint{-0.018638in}{-0.010825in}}{\pgfqpoint{-0.014731in}{-0.014731in}}%
\pgfpathcurveto{\pgfqpoint{-0.010825in}{-0.018638in}}{\pgfqpoint{-0.005525in}{-0.020833in}}{\pgfqpoint{0.000000in}{-0.020833in}}%
\pgfpathlineto{\pgfqpoint{0.000000in}{-0.020833in}}%
\pgfpathclose%
\pgfusepath{stroke,fill}%
}%
\begin{pgfscope}%
\pgfsys@transformshift{5.349267in}{4.537732in}%
\pgfsys@useobject{currentmarker}{}%
\end{pgfscope}%
\end{pgfscope}%
\begin{pgfscope}%
\definecolor{textcolor}{rgb}{0.000000,0.000000,0.000000}%
\pgfsetstrokecolor{textcolor}%
\pgfsetfillcolor{textcolor}%
\pgftext[x=5.749267in,y=4.459954in,left,base]{\color{textcolor}{\rmfamily\fontsize{16.000000}{19.200000}\selectfont\catcode`\^=\active\def^{\ifmmode\sp\else\^{}\fi}\catcode`\%=\active\def%{\%}Pareto-front}}%
\end{pgfscope}%
\begin{pgfscope}%
\pgfsetrectcap%
\pgfsetroundjoin%
\pgfsetlinewidth{1.505625pt}%
\definecolor{currentstroke}{rgb}{0.501961,0.501961,0.501961}%
\pgfsetstrokecolor{currentstroke}%
\pgfsetstrokeopacity{0.500000}%
\pgfsetdash{}{0pt}%
\pgfpathmoveto{\pgfqpoint{5.127045in}{4.213272in}}%
\pgfpathlineto{\pgfqpoint{5.349267in}{4.213272in}}%
\pgfpathlineto{\pgfqpoint{5.571489in}{4.213272in}}%
\pgfusepath{stroke}%
\end{pgfscope}%
\begin{pgfscope}%
\definecolor{textcolor}{rgb}{0.000000,0.000000,0.000000}%
\pgfsetstrokecolor{textcolor}%
\pgfsetfillcolor{textcolor}%
\pgftext[x=5.749267in,y=4.135494in,left,base]{\color{textcolor}{\rmfamily\fontsize{16.000000}{19.200000}\selectfont\catcode`\^=\active\def^{\ifmmode\sp\else\^{}\fi}\catcode`\%=\active\def%{\%}Near-optimal space}}%
\end{pgfscope}%
\end{pgfpicture}%
\makeatother%
\endgroup%
}
        \caption{The near-optimal objective space around a Pareto front generated by
        \gls{osier}, within 10\% of the minimum for each objective axis. Note:
        The number of tested points shown in this figure is reduced due to
        display constraints.}
        \label{fig:osier-near-optimal}
    \end{center}
\end{figure}

Figure \ref{fig:temoa-benchmark-03} presents the spread of results in
the decision space for each model. Figure \ref{fig:temoa-benchmark-03}a shows
the spread of each technology present in \gls{osier}'s Pareto front. Figure
\ref{fig:temoa-benchmark-03}b shows the same, but also includes the
selected points from \gls{osier}'s near-optimal space. Lastly, Figure
\ref{fig:temoa-benchmark-03}c shows the same kind of distribution for
\gls{temoa}'s \gls{mga} solutions.

\begin{figure}[ht!]
  \centering
  \resizebox{\columnwidth}{!}{%% Creator: Matplotlib, PGF backend
%%
%% To include the figure in your LaTeX document, write
%%   \input{<filename>.pgf}
%%
%% Make sure the required packages are loaded in your preamble
%%   \usepackage{pgf}
%%
%% Also ensure that all the required font packages are loaded; for instance,
%% the lmodern package is sometimes necessary when using math font.
%%   \usepackage{lmodern}
%%
%% Figures using additional raster images can only be included by \input if
%% they are in the same directory as the main LaTeX file. For loading figures
%% from other directories you can use the `import` package
%%   \usepackage{import}
%%
%% and then include the figures with
%%   \import{<path to file>}{<filename>.pgf}
%%
%% Matplotlib used the following preamble
%%   \def\mathdefault#1{#1}
%%   \everymath=\expandafter{\the\everymath\displaystyle}
%%   \IfFileExists{scrextend.sty}{
%%     \usepackage[fontsize=10.000000pt]{scrextend}
%%   }{
%%     \renewcommand{\normalsize}{\fontsize{10.000000}{12.000000}\selectfont}
%%     \normalsize
%%   }
%%   
%%   \makeatletter\@ifpackageloaded{underscore}{}{\usepackage[strings]{underscore}}\makeatother
%%
\begingroup%
\makeatletter%
\begin{pgfpicture}%
\pgfpathrectangle{\pgfpointorigin}{\pgfqpoint{11.884299in}{13.900000in}}%
\pgfusepath{use as bounding box, clip}%
\begin{pgfscope}%
\pgfsetbuttcap%
\pgfsetmiterjoin%
\definecolor{currentfill}{rgb}{1.000000,1.000000,1.000000}%
\pgfsetfillcolor{currentfill}%
\pgfsetlinewidth{0.000000pt}%
\definecolor{currentstroke}{rgb}{0.000000,0.000000,0.000000}%
\pgfsetstrokecolor{currentstroke}%
\pgfsetdash{}{0pt}%
\pgfpathmoveto{\pgfqpoint{0.000000in}{0.000000in}}%
\pgfpathlineto{\pgfqpoint{11.884299in}{0.000000in}}%
\pgfpathlineto{\pgfqpoint{11.884299in}{13.900000in}}%
\pgfpathlineto{\pgfqpoint{0.000000in}{13.900000in}}%
\pgfpathlineto{\pgfqpoint{0.000000in}{0.000000in}}%
\pgfpathclose%
\pgfusepath{fill}%
\end{pgfscope}%
\begin{pgfscope}%
\pgfsetbuttcap%
\pgfsetmiterjoin%
\definecolor{currentfill}{rgb}{1.000000,1.000000,1.000000}%
\pgfsetfillcolor{currentfill}%
\pgfsetlinewidth{0.000000pt}%
\definecolor{currentstroke}{rgb}{0.000000,0.000000,0.000000}%
\pgfsetstrokecolor{currentstroke}%
\pgfsetstrokeopacity{0.000000}%
\pgfsetdash{}{0pt}%
\pgfpathmoveto{\pgfqpoint{0.688192in}{9.576569in}}%
\pgfpathlineto{\pgfqpoint{11.784299in}{9.576569in}}%
\pgfpathlineto{\pgfqpoint{11.784299in}{13.800000in}}%
\pgfpathlineto{\pgfqpoint{0.688192in}{13.800000in}}%
\pgfpathlineto{\pgfqpoint{0.688192in}{9.576569in}}%
\pgfpathclose%
\pgfusepath{fill}%
\end{pgfscope}%
\begin{pgfscope}%
\pgfpathrectangle{\pgfqpoint{0.688192in}{9.576569in}}{\pgfqpoint{11.096108in}{4.223431in}}%
\pgfusepath{clip}%
\pgfsetbuttcap%
\pgfsetroundjoin%
\definecolor{currentfill}{rgb}{0.469005,0.634306,0.749958}%
\pgfsetfillcolor{currentfill}%
\pgfsetlinewidth{0.752812pt}%
\definecolor{currentstroke}{rgb}{0.240000,0.240000,0.240000}%
\pgfsetstrokecolor{currentstroke}%
\pgfsetdash{}{0pt}%
\pgfpathmoveto{\pgfqpoint{1.132036in}{12.433332in}}%
\pgfpathlineto{\pgfqpoint{1.353958in}{12.433332in}}%
\pgfpathlineto{\pgfqpoint{1.353958in}{12.478266in}}%
\pgfpathlineto{\pgfqpoint{1.132036in}{12.478266in}}%
\pgfpathlineto{\pgfqpoint{1.132036in}{12.433332in}}%
\pgfpathclose%
\pgfusepath{stroke,fill}%
\end{pgfscope}%
\begin{pgfscope}%
\pgfpathrectangle{\pgfqpoint{0.688192in}{9.576569in}}{\pgfqpoint{11.096108in}{4.223431in}}%
\pgfusepath{clip}%
\pgfsetbuttcap%
\pgfsetroundjoin%
\definecolor{currentfill}{rgb}{0.346402,0.553490,0.697630}%
\pgfsetfillcolor{currentfill}%
\pgfsetlinewidth{0.752812pt}%
\definecolor{currentstroke}{rgb}{0.240000,0.240000,0.240000}%
\pgfsetstrokecolor{currentstroke}%
\pgfsetdash{}{0pt}%
\pgfpathmoveto{\pgfqpoint{1.021075in}{12.478266in}}%
\pgfpathlineto{\pgfqpoint{1.464919in}{12.478266in}}%
\pgfpathlineto{\pgfqpoint{1.464919in}{12.594349in}}%
\pgfpathlineto{\pgfqpoint{1.021075in}{12.594349in}}%
\pgfpathlineto{\pgfqpoint{1.021075in}{12.478266in}}%
\pgfpathclose%
\pgfusepath{stroke,fill}%
\end{pgfscope}%
\begin{pgfscope}%
\pgfpathrectangle{\pgfqpoint{0.688192in}{9.576569in}}{\pgfqpoint{11.096108in}{4.223431in}}%
\pgfusepath{clip}%
\pgfsetbuttcap%
\pgfsetroundjoin%
\definecolor{currentfill}{rgb}{0.194608,0.453431,0.632843}%
\pgfsetfillcolor{currentfill}%
\pgfsetlinewidth{0.752812pt}%
\definecolor{currentstroke}{rgb}{0.240000,0.240000,0.240000}%
\pgfsetstrokecolor{currentstroke}%
\pgfsetdash{}{0pt}%
\pgfpathmoveto{\pgfqpoint{0.799153in}{12.594349in}}%
\pgfpathlineto{\pgfqpoint{1.686841in}{12.594349in}}%
\pgfpathlineto{\pgfqpoint{1.686841in}{12.926942in}}%
\pgfpathlineto{\pgfqpoint{0.799153in}{12.926942in}}%
\pgfpathlineto{\pgfqpoint{0.799153in}{12.594349in}}%
\pgfpathclose%
\pgfusepath{stroke,fill}%
\end{pgfscope}%
\begin{pgfscope}%
\pgfpathrectangle{\pgfqpoint{0.688192in}{9.576569in}}{\pgfqpoint{11.096108in}{4.223431in}}%
\pgfusepath{clip}%
\pgfsetbuttcap%
\pgfsetroundjoin%
\definecolor{currentfill}{rgb}{0.346402,0.553490,0.697630}%
\pgfsetfillcolor{currentfill}%
\pgfsetlinewidth{0.752812pt}%
\definecolor{currentstroke}{rgb}{0.240000,0.240000,0.240000}%
\pgfsetstrokecolor{currentstroke}%
\pgfsetdash{}{0pt}%
\pgfpathmoveto{\pgfqpoint{1.021075in}{12.926942in}}%
\pgfpathlineto{\pgfqpoint{1.464919in}{12.926942in}}%
\pgfpathlineto{\pgfqpoint{1.464919in}{12.959676in}}%
\pgfpathlineto{\pgfqpoint{1.021075in}{12.959676in}}%
\pgfpathlineto{\pgfqpoint{1.021075in}{12.926942in}}%
\pgfpathclose%
\pgfusepath{stroke,fill}%
\end{pgfscope}%
\begin{pgfscope}%
\pgfpathrectangle{\pgfqpoint{0.688192in}{9.576569in}}{\pgfqpoint{11.096108in}{4.223431in}}%
\pgfusepath{clip}%
\pgfsetbuttcap%
\pgfsetroundjoin%
\definecolor{currentfill}{rgb}{0.469005,0.634306,0.749958}%
\pgfsetfillcolor{currentfill}%
\pgfsetlinewidth{0.752812pt}%
\definecolor{currentstroke}{rgb}{0.240000,0.240000,0.240000}%
\pgfsetstrokecolor{currentstroke}%
\pgfsetdash{}{0pt}%
\pgfpathmoveto{\pgfqpoint{1.132036in}{12.959676in}}%
\pgfpathlineto{\pgfqpoint{1.353958in}{12.959676in}}%
\pgfpathlineto{\pgfqpoint{1.353958in}{12.980754in}}%
\pgfpathlineto{\pgfqpoint{1.132036in}{12.980754in}}%
\pgfpathlineto{\pgfqpoint{1.132036in}{12.959676in}}%
\pgfpathclose%
\pgfusepath{stroke,fill}%
\end{pgfscope}%
\begin{pgfscope}%
\pgfpathrectangle{\pgfqpoint{0.688192in}{9.576569in}}{\pgfqpoint{11.096108in}{4.223431in}}%
\pgfusepath{clip}%
\pgfsetbuttcap%
\pgfsetroundjoin%
\pgfsetlinewidth{1.003750pt}%
\definecolor{currentstroke}{rgb}{0.450000,0.450000,0.450000}%
\pgfsetstrokecolor{currentstroke}%
\pgfsetdash{}{0pt}%
\pgfpathmoveto{\pgfqpoint{1.242997in}{12.179672in}}%
\pgfpathcurveto{\pgfqpoint{1.252205in}{12.179672in}}{\pgfqpoint{1.261038in}{12.183330in}}{\pgfqpoint{1.267549in}{12.189842in}}%
\pgfpathcurveto{\pgfqpoint{1.274061in}{12.196353in}}{\pgfqpoint{1.277719in}{12.205185in}}{\pgfqpoint{1.277719in}{12.214394in}}%
\pgfpathcurveto{\pgfqpoint{1.277719in}{12.223602in}}{\pgfqpoint{1.274061in}{12.232435in}}{\pgfqpoint{1.267549in}{12.238946in}}%
\pgfpathcurveto{\pgfqpoint{1.261038in}{12.245458in}}{\pgfqpoint{1.252205in}{12.249116in}}{\pgfqpoint{1.242997in}{12.249116in}}%
\pgfpathcurveto{\pgfqpoint{1.233788in}{12.249116in}}{\pgfqpoint{1.224956in}{12.245458in}}{\pgfqpoint{1.218445in}{12.238946in}}%
\pgfpathcurveto{\pgfqpoint{1.211933in}{12.232435in}}{\pgfqpoint{1.208275in}{12.223602in}}{\pgfqpoint{1.208275in}{12.214394in}}%
\pgfpathcurveto{\pgfqpoint{1.208275in}{12.205185in}}{\pgfqpoint{1.211933in}{12.196353in}}{\pgfqpoint{1.218445in}{12.189842in}}%
\pgfpathcurveto{\pgfqpoint{1.224956in}{12.183330in}}{\pgfqpoint{1.233788in}{12.179672in}}{\pgfqpoint{1.242997in}{12.179672in}}%
\pgfpathlineto{\pgfqpoint{1.242997in}{12.179672in}}%
\pgfpathclose%
\pgfusepath{stroke}%
\end{pgfscope}%
\begin{pgfscope}%
\pgfpathrectangle{\pgfqpoint{0.688192in}{9.576569in}}{\pgfqpoint{11.096108in}{4.223431in}}%
\pgfusepath{clip}%
\pgfsetbuttcap%
\pgfsetroundjoin%
\pgfsetlinewidth{1.003750pt}%
\definecolor{currentstroke}{rgb}{0.450000,0.450000,0.450000}%
\pgfsetstrokecolor{currentstroke}%
\pgfsetdash{}{0pt}%
\pgfpathmoveto{\pgfqpoint{1.242997in}{12.276080in}}%
\pgfpathcurveto{\pgfqpoint{1.252205in}{12.276080in}}{\pgfqpoint{1.261038in}{12.279739in}}{\pgfqpoint{1.267549in}{12.286250in}}%
\pgfpathcurveto{\pgfqpoint{1.274061in}{12.292762in}}{\pgfqpoint{1.277719in}{12.301594in}}{\pgfqpoint{1.277719in}{12.310803in}}%
\pgfpathcurveto{\pgfqpoint{1.277719in}{12.320011in}}{\pgfqpoint{1.274061in}{12.328844in}}{\pgfqpoint{1.267549in}{12.335355in}}%
\pgfpathcurveto{\pgfqpoint{1.261038in}{12.341866in}}{\pgfqpoint{1.252205in}{12.345525in}}{\pgfqpoint{1.242997in}{12.345525in}}%
\pgfpathcurveto{\pgfqpoint{1.233788in}{12.345525in}}{\pgfqpoint{1.224956in}{12.341866in}}{\pgfqpoint{1.218445in}{12.335355in}}%
\pgfpathcurveto{\pgfqpoint{1.211933in}{12.328844in}}{\pgfqpoint{1.208275in}{12.320011in}}{\pgfqpoint{1.208275in}{12.310803in}}%
\pgfpathcurveto{\pgfqpoint{1.208275in}{12.301594in}}{\pgfqpoint{1.211933in}{12.292762in}}{\pgfqpoint{1.218445in}{12.286250in}}%
\pgfpathcurveto{\pgfqpoint{1.224956in}{12.279739in}}{\pgfqpoint{1.233788in}{12.276080in}}{\pgfqpoint{1.242997in}{12.276080in}}%
\pgfpathlineto{\pgfqpoint{1.242997in}{12.276080in}}%
\pgfpathclose%
\pgfusepath{stroke}%
\end{pgfscope}%
\begin{pgfscope}%
\pgfpathrectangle{\pgfqpoint{0.688192in}{9.576569in}}{\pgfqpoint{11.096108in}{4.223431in}}%
\pgfusepath{clip}%
\pgfsetbuttcap%
\pgfsetroundjoin%
\pgfsetlinewidth{1.003750pt}%
\definecolor{currentstroke}{rgb}{0.450000,0.450000,0.450000}%
\pgfsetstrokecolor{currentstroke}%
\pgfsetdash{}{0pt}%
\pgfpathmoveto{\pgfqpoint{1.242997in}{12.046512in}}%
\pgfpathcurveto{\pgfqpoint{1.252205in}{12.046512in}}{\pgfqpoint{1.261038in}{12.050171in}}{\pgfqpoint{1.267549in}{12.056682in}}%
\pgfpathcurveto{\pgfqpoint{1.274061in}{12.063194in}}{\pgfqpoint{1.277719in}{12.072026in}}{\pgfqpoint{1.277719in}{12.081235in}}%
\pgfpathcurveto{\pgfqpoint{1.277719in}{12.090443in}}{\pgfqpoint{1.274061in}{12.099276in}}{\pgfqpoint{1.267549in}{12.105787in}}%
\pgfpathcurveto{\pgfqpoint{1.261038in}{12.112298in}}{\pgfqpoint{1.252205in}{12.115957in}}{\pgfqpoint{1.242997in}{12.115957in}}%
\pgfpathcurveto{\pgfqpoint{1.233788in}{12.115957in}}{\pgfqpoint{1.224956in}{12.112298in}}{\pgfqpoint{1.218445in}{12.105787in}}%
\pgfpathcurveto{\pgfqpoint{1.211933in}{12.099276in}}{\pgfqpoint{1.208275in}{12.090443in}}{\pgfqpoint{1.208275in}{12.081235in}}%
\pgfpathcurveto{\pgfqpoint{1.208275in}{12.072026in}}{\pgfqpoint{1.211933in}{12.063194in}}{\pgfqpoint{1.218445in}{12.056682in}}%
\pgfpathcurveto{\pgfqpoint{1.224956in}{12.050171in}}{\pgfqpoint{1.233788in}{12.046512in}}{\pgfqpoint{1.242997in}{12.046512in}}%
\pgfpathlineto{\pgfqpoint{1.242997in}{12.046512in}}%
\pgfpathclose%
\pgfusepath{stroke}%
\end{pgfscope}%
\begin{pgfscope}%
\pgfpathrectangle{\pgfqpoint{0.688192in}{9.576569in}}{\pgfqpoint{11.096108in}{4.223431in}}%
\pgfusepath{clip}%
\pgfsetbuttcap%
\pgfsetroundjoin%
\pgfsetlinewidth{1.003750pt}%
\definecolor{currentstroke}{rgb}{0.450000,0.450000,0.450000}%
\pgfsetstrokecolor{currentstroke}%
\pgfsetdash{}{0pt}%
\pgfpathmoveto{\pgfqpoint{1.242997in}{12.956844in}}%
\pgfpathcurveto{\pgfqpoint{1.252205in}{12.956844in}}{\pgfqpoint{1.261038in}{12.960502in}}{\pgfqpoint{1.267549in}{12.967014in}}%
\pgfpathcurveto{\pgfqpoint{1.274061in}{12.973525in}}{\pgfqpoint{1.277719in}{12.982357in}}{\pgfqpoint{1.277719in}{12.991566in}}%
\pgfpathcurveto{\pgfqpoint{1.277719in}{13.000774in}}{\pgfqpoint{1.274061in}{13.009607in}}{\pgfqpoint{1.267549in}{13.016118in}}%
\pgfpathcurveto{\pgfqpoint{1.261038in}{13.022630in}}{\pgfqpoint{1.252205in}{13.026288in}}{\pgfqpoint{1.242997in}{13.026288in}}%
\pgfpathcurveto{\pgfqpoint{1.233788in}{13.026288in}}{\pgfqpoint{1.224956in}{13.022630in}}{\pgfqpoint{1.218445in}{13.016118in}}%
\pgfpathcurveto{\pgfqpoint{1.211933in}{13.009607in}}{\pgfqpoint{1.208275in}{13.000774in}}{\pgfqpoint{1.208275in}{12.991566in}}%
\pgfpathcurveto{\pgfqpoint{1.208275in}{12.982357in}}{\pgfqpoint{1.211933in}{12.973525in}}{\pgfqpoint{1.218445in}{12.967014in}}%
\pgfpathcurveto{\pgfqpoint{1.224956in}{12.960502in}}{\pgfqpoint{1.233788in}{12.956844in}}{\pgfqpoint{1.242997in}{12.956844in}}%
\pgfpathlineto{\pgfqpoint{1.242997in}{12.956844in}}%
\pgfpathclose%
\pgfusepath{stroke}%
\end{pgfscope}%
\begin{pgfscope}%
\pgfpathrectangle{\pgfqpoint{0.688192in}{9.576569in}}{\pgfqpoint{11.096108in}{4.223431in}}%
\pgfusepath{clip}%
\pgfsetbuttcap%
\pgfsetroundjoin%
\pgfsetlinewidth{1.003750pt}%
\definecolor{currentstroke}{rgb}{0.450000,0.450000,0.450000}%
\pgfsetstrokecolor{currentstroke}%
\pgfsetdash{}{0pt}%
\pgfpathmoveto{\pgfqpoint{1.242997in}{12.949619in}}%
\pgfpathcurveto{\pgfqpoint{1.252205in}{12.949619in}}{\pgfqpoint{1.261038in}{12.953278in}}{\pgfqpoint{1.267549in}{12.959789in}}%
\pgfpathcurveto{\pgfqpoint{1.274061in}{12.966300in}}{\pgfqpoint{1.277719in}{12.975133in}}{\pgfqpoint{1.277719in}{12.984341in}}%
\pgfpathcurveto{\pgfqpoint{1.277719in}{12.993550in}}{\pgfqpoint{1.274061in}{13.002382in}}{\pgfqpoint{1.267549in}{13.008894in}}%
\pgfpathcurveto{\pgfqpoint{1.261038in}{13.015405in}}{\pgfqpoint{1.252205in}{13.019063in}}{\pgfqpoint{1.242997in}{13.019063in}}%
\pgfpathcurveto{\pgfqpoint{1.233788in}{13.019063in}}{\pgfqpoint{1.224956in}{13.015405in}}{\pgfqpoint{1.218445in}{13.008894in}}%
\pgfpathcurveto{\pgfqpoint{1.211933in}{13.002382in}}{\pgfqpoint{1.208275in}{12.993550in}}{\pgfqpoint{1.208275in}{12.984341in}}%
\pgfpathcurveto{\pgfqpoint{1.208275in}{12.975133in}}{\pgfqpoint{1.211933in}{12.966300in}}{\pgfqpoint{1.218445in}{12.959789in}}%
\pgfpathcurveto{\pgfqpoint{1.224956in}{12.953278in}}{\pgfqpoint{1.233788in}{12.949619in}}{\pgfqpoint{1.242997in}{12.949619in}}%
\pgfpathlineto{\pgfqpoint{1.242997in}{12.949619in}}%
\pgfpathclose%
\pgfusepath{stroke}%
\end{pgfscope}%
\begin{pgfscope}%
\pgfpathrectangle{\pgfqpoint{0.688192in}{9.576569in}}{\pgfqpoint{11.096108in}{4.223431in}}%
\pgfusepath{clip}%
\pgfsetbuttcap%
\pgfsetroundjoin%
\pgfsetlinewidth{1.003750pt}%
\definecolor{currentstroke}{rgb}{0.450000,0.450000,0.450000}%
\pgfsetstrokecolor{currentstroke}%
\pgfsetdash{}{0pt}%
\pgfpathmoveto{\pgfqpoint{1.242997in}{12.948832in}}%
\pgfpathcurveto{\pgfqpoint{1.252205in}{12.948832in}}{\pgfqpoint{1.261038in}{12.952491in}}{\pgfqpoint{1.267549in}{12.959002in}}%
\pgfpathcurveto{\pgfqpoint{1.274061in}{12.965513in}}{\pgfqpoint{1.277719in}{12.974346in}}{\pgfqpoint{1.277719in}{12.983554in}}%
\pgfpathcurveto{\pgfqpoint{1.277719in}{12.992763in}}{\pgfqpoint{1.274061in}{13.001595in}}{\pgfqpoint{1.267549in}{13.008107in}}%
\pgfpathcurveto{\pgfqpoint{1.261038in}{13.014618in}}{\pgfqpoint{1.252205in}{13.018276in}}{\pgfqpoint{1.242997in}{13.018276in}}%
\pgfpathcurveto{\pgfqpoint{1.233788in}{13.018276in}}{\pgfqpoint{1.224956in}{13.014618in}}{\pgfqpoint{1.218445in}{13.008107in}}%
\pgfpathcurveto{\pgfqpoint{1.211933in}{13.001595in}}{\pgfqpoint{1.208275in}{12.992763in}}{\pgfqpoint{1.208275in}{12.983554in}}%
\pgfpathcurveto{\pgfqpoint{1.208275in}{12.974346in}}{\pgfqpoint{1.211933in}{12.965513in}}{\pgfqpoint{1.218445in}{12.959002in}}%
\pgfpathcurveto{\pgfqpoint{1.224956in}{12.952491in}}{\pgfqpoint{1.233788in}{12.948832in}}{\pgfqpoint{1.242997in}{12.948832in}}%
\pgfpathlineto{\pgfqpoint{1.242997in}{12.948832in}}%
\pgfpathclose%
\pgfusepath{stroke}%
\end{pgfscope}%
\begin{pgfscope}%
\pgfpathrectangle{\pgfqpoint{0.688192in}{9.576569in}}{\pgfqpoint{11.096108in}{4.223431in}}%
\pgfusepath{clip}%
\pgfsetbuttcap%
\pgfsetroundjoin%
\pgfsetlinewidth{1.003750pt}%
\definecolor{currentstroke}{rgb}{0.450000,0.450000,0.450000}%
\pgfsetstrokecolor{currentstroke}%
\pgfsetdash{}{0pt}%
\pgfpathmoveto{\pgfqpoint{1.242997in}{12.946144in}}%
\pgfpathcurveto{\pgfqpoint{1.252205in}{12.946144in}}{\pgfqpoint{1.261038in}{12.949803in}}{\pgfqpoint{1.267549in}{12.956314in}}%
\pgfpathcurveto{\pgfqpoint{1.274061in}{12.962825in}}{\pgfqpoint{1.277719in}{12.971658in}}{\pgfqpoint{1.277719in}{12.980866in}}%
\pgfpathcurveto{\pgfqpoint{1.277719in}{12.990075in}}{\pgfqpoint{1.274061in}{12.998907in}}{\pgfqpoint{1.267549in}{13.005419in}}%
\pgfpathcurveto{\pgfqpoint{1.261038in}{13.011930in}}{\pgfqpoint{1.252205in}{13.015589in}}{\pgfqpoint{1.242997in}{13.015589in}}%
\pgfpathcurveto{\pgfqpoint{1.233788in}{13.015589in}}{\pgfqpoint{1.224956in}{13.011930in}}{\pgfqpoint{1.218445in}{13.005419in}}%
\pgfpathcurveto{\pgfqpoint{1.211933in}{12.998907in}}{\pgfqpoint{1.208275in}{12.990075in}}{\pgfqpoint{1.208275in}{12.980866in}}%
\pgfpathcurveto{\pgfqpoint{1.208275in}{12.971658in}}{\pgfqpoint{1.211933in}{12.962825in}}{\pgfqpoint{1.218445in}{12.956314in}}%
\pgfpathcurveto{\pgfqpoint{1.224956in}{12.949803in}}{\pgfqpoint{1.233788in}{12.946144in}}{\pgfqpoint{1.242997in}{12.946144in}}%
\pgfpathlineto{\pgfqpoint{1.242997in}{12.946144in}}%
\pgfpathclose%
\pgfusepath{stroke}%
\end{pgfscope}%
\begin{pgfscope}%
\pgfpathrectangle{\pgfqpoint{0.688192in}{9.576569in}}{\pgfqpoint{11.096108in}{4.223431in}}%
\pgfusepath{clip}%
\pgfsetbuttcap%
\pgfsetroundjoin%
\pgfsetlinewidth{1.003750pt}%
\definecolor{currentstroke}{rgb}{0.450000,0.450000,0.450000}%
\pgfsetstrokecolor{currentstroke}%
\pgfsetdash{}{0pt}%
\pgfpathmoveto{\pgfqpoint{1.242997in}{12.368925in}}%
\pgfpathcurveto{\pgfqpoint{1.252205in}{12.368925in}}{\pgfqpoint{1.261038in}{12.372584in}}{\pgfqpoint{1.267549in}{12.379095in}}%
\pgfpathcurveto{\pgfqpoint{1.274061in}{12.385606in}}{\pgfqpoint{1.277719in}{12.394439in}}{\pgfqpoint{1.277719in}{12.403647in}}%
\pgfpathcurveto{\pgfqpoint{1.277719in}{12.412856in}}{\pgfqpoint{1.274061in}{12.421688in}}{\pgfqpoint{1.267549in}{12.428200in}}%
\pgfpathcurveto{\pgfqpoint{1.261038in}{12.434711in}}{\pgfqpoint{1.252205in}{12.438369in}}{\pgfqpoint{1.242997in}{12.438369in}}%
\pgfpathcurveto{\pgfqpoint{1.233788in}{12.438369in}}{\pgfqpoint{1.224956in}{12.434711in}}{\pgfqpoint{1.218445in}{12.428200in}}%
\pgfpathcurveto{\pgfqpoint{1.211933in}{12.421688in}}{\pgfqpoint{1.208275in}{12.412856in}}{\pgfqpoint{1.208275in}{12.403647in}}%
\pgfpathcurveto{\pgfqpoint{1.208275in}{12.394439in}}{\pgfqpoint{1.211933in}{12.385606in}}{\pgfqpoint{1.218445in}{12.379095in}}%
\pgfpathcurveto{\pgfqpoint{1.224956in}{12.372584in}}{\pgfqpoint{1.233788in}{12.368925in}}{\pgfqpoint{1.242997in}{12.368925in}}%
\pgfpathlineto{\pgfqpoint{1.242997in}{12.368925in}}%
\pgfpathclose%
\pgfusepath{stroke}%
\end{pgfscope}%
\begin{pgfscope}%
\pgfpathrectangle{\pgfqpoint{0.688192in}{9.576569in}}{\pgfqpoint{11.096108in}{4.223431in}}%
\pgfusepath{clip}%
\pgfsetbuttcap%
\pgfsetroundjoin%
\definecolor{currentfill}{rgb}{0.907545,0.666205,0.454979}%
\pgfsetfillcolor{currentfill}%
\pgfsetlinewidth{0.752812pt}%
\definecolor{currentstroke}{rgb}{0.240000,0.240000,0.240000}%
\pgfsetstrokecolor{currentstroke}%
\pgfsetdash{}{0pt}%
\pgfpathmoveto{\pgfqpoint{2.241647in}{9.576569in}}%
\pgfpathlineto{\pgfqpoint{2.463569in}{9.576569in}}%
\pgfpathlineto{\pgfqpoint{2.463569in}{9.576569in}}%
\pgfpathlineto{\pgfqpoint{2.241647in}{9.576569in}}%
\pgfpathlineto{\pgfqpoint{2.241647in}{9.576569in}}%
\pgfpathclose%
\pgfusepath{stroke,fill}%
\end{pgfscope}%
\begin{pgfscope}%
\pgfpathrectangle{\pgfqpoint{0.688192in}{9.576569in}}{\pgfqpoint{11.096108in}{4.223431in}}%
\pgfusepath{clip}%
\pgfsetbuttcap%
\pgfsetroundjoin%
\definecolor{currentfill}{rgb}{0.896070,0.594353,0.329006}%
\pgfsetfillcolor{currentfill}%
\pgfsetlinewidth{0.752812pt}%
\definecolor{currentstroke}{rgb}{0.240000,0.240000,0.240000}%
\pgfsetstrokecolor{currentstroke}%
\pgfsetdash{}{0pt}%
\pgfpathmoveto{\pgfqpoint{2.130686in}{9.576569in}}%
\pgfpathlineto{\pgfqpoint{2.574530in}{9.576569in}}%
\pgfpathlineto{\pgfqpoint{2.574530in}{9.576569in}}%
\pgfpathlineto{\pgfqpoint{2.130686in}{9.576569in}}%
\pgfpathlineto{\pgfqpoint{2.130686in}{9.576569in}}%
\pgfpathclose%
\pgfusepath{stroke,fill}%
\end{pgfscope}%
\begin{pgfscope}%
\pgfpathrectangle{\pgfqpoint{0.688192in}{9.576569in}}{\pgfqpoint{11.096108in}{4.223431in}}%
\pgfusepath{clip}%
\pgfsetbuttcap%
\pgfsetroundjoin%
\definecolor{currentfill}{rgb}{0.881863,0.505392,0.173039}%
\pgfsetfillcolor{currentfill}%
\pgfsetlinewidth{0.752812pt}%
\definecolor{currentstroke}{rgb}{0.240000,0.240000,0.240000}%
\pgfsetstrokecolor{currentstroke}%
\pgfsetdash{}{0pt}%
\pgfpathmoveto{\pgfqpoint{1.908763in}{9.576569in}}%
\pgfpathlineto{\pgfqpoint{2.796452in}{9.576569in}}%
\pgfpathlineto{\pgfqpoint{2.796452in}{9.591694in}}%
\pgfpathlineto{\pgfqpoint{1.908763in}{9.591694in}}%
\pgfpathlineto{\pgfqpoint{1.908763in}{9.576569in}}%
\pgfpathclose%
\pgfusepath{stroke,fill}%
\end{pgfscope}%
\begin{pgfscope}%
\pgfpathrectangle{\pgfqpoint{0.688192in}{9.576569in}}{\pgfqpoint{11.096108in}{4.223431in}}%
\pgfusepath{clip}%
\pgfsetbuttcap%
\pgfsetroundjoin%
\definecolor{currentfill}{rgb}{0.896070,0.594353,0.329006}%
\pgfsetfillcolor{currentfill}%
\pgfsetlinewidth{0.752812pt}%
\definecolor{currentstroke}{rgb}{0.240000,0.240000,0.240000}%
\pgfsetstrokecolor{currentstroke}%
\pgfsetdash{}{0pt}%
\pgfpathmoveto{\pgfqpoint{2.130686in}{9.591694in}}%
\pgfpathlineto{\pgfqpoint{2.574530in}{9.591694in}}%
\pgfpathlineto{\pgfqpoint{2.574530in}{9.771408in}}%
\pgfpathlineto{\pgfqpoint{2.130686in}{9.771408in}}%
\pgfpathlineto{\pgfqpoint{2.130686in}{9.591694in}}%
\pgfpathclose%
\pgfusepath{stroke,fill}%
\end{pgfscope}%
\begin{pgfscope}%
\pgfpathrectangle{\pgfqpoint{0.688192in}{9.576569in}}{\pgfqpoint{11.096108in}{4.223431in}}%
\pgfusepath{clip}%
\pgfsetbuttcap%
\pgfsetroundjoin%
\definecolor{currentfill}{rgb}{0.907545,0.666205,0.454979}%
\pgfsetfillcolor{currentfill}%
\pgfsetlinewidth{0.752812pt}%
\definecolor{currentstroke}{rgb}{0.240000,0.240000,0.240000}%
\pgfsetstrokecolor{currentstroke}%
\pgfsetdash{}{0pt}%
\pgfpathmoveto{\pgfqpoint{2.241647in}{9.771408in}}%
\pgfpathlineto{\pgfqpoint{2.463569in}{9.771408in}}%
\pgfpathlineto{\pgfqpoint{2.463569in}{9.953718in}}%
\pgfpathlineto{\pgfqpoint{2.241647in}{9.953718in}}%
\pgfpathlineto{\pgfqpoint{2.241647in}{9.771408in}}%
\pgfpathclose%
\pgfusepath{stroke,fill}%
\end{pgfscope}%
\begin{pgfscope}%
\pgfpathrectangle{\pgfqpoint{0.688192in}{9.576569in}}{\pgfqpoint{11.096108in}{4.223431in}}%
\pgfusepath{clip}%
\pgfsetbuttcap%
\pgfsetroundjoin%
\pgfsetlinewidth{1.003750pt}%
\definecolor{currentstroke}{rgb}{0.450000,0.450000,0.450000}%
\pgfsetstrokecolor{currentstroke}%
\pgfsetdash{}{0pt}%
\pgfpathmoveto{\pgfqpoint{2.352608in}{10.096539in}}%
\pgfpathcurveto{\pgfqpoint{2.361816in}{10.096539in}}{\pgfqpoint{2.370649in}{10.100198in}}{\pgfqpoint{2.377160in}{10.106709in}}%
\pgfpathcurveto{\pgfqpoint{2.383671in}{10.113221in}}{\pgfqpoint{2.387330in}{10.122053in}}{\pgfqpoint{2.387330in}{10.131262in}}%
\pgfpathcurveto{\pgfqpoint{2.387330in}{10.140470in}}{\pgfqpoint{2.383671in}{10.149303in}}{\pgfqpoint{2.377160in}{10.155814in}}%
\pgfpathcurveto{\pgfqpoint{2.370649in}{10.162325in}}{\pgfqpoint{2.361816in}{10.165984in}}{\pgfqpoint{2.352608in}{10.165984in}}%
\pgfpathcurveto{\pgfqpoint{2.343399in}{10.165984in}}{\pgfqpoint{2.334567in}{10.162325in}}{\pgfqpoint{2.328055in}{10.155814in}}%
\pgfpathcurveto{\pgfqpoint{2.321544in}{10.149303in}}{\pgfqpoint{2.317885in}{10.140470in}}{\pgfqpoint{2.317885in}{10.131262in}}%
\pgfpathcurveto{\pgfqpoint{2.317885in}{10.122053in}}{\pgfqpoint{2.321544in}{10.113221in}}{\pgfqpoint{2.328055in}{10.106709in}}%
\pgfpathcurveto{\pgfqpoint{2.334567in}{10.100198in}}{\pgfqpoint{2.343399in}{10.096539in}}{\pgfqpoint{2.352608in}{10.096539in}}%
\pgfpathlineto{\pgfqpoint{2.352608in}{10.096539in}}%
\pgfpathclose%
\pgfusepath{stroke}%
\end{pgfscope}%
\begin{pgfscope}%
\pgfpathrectangle{\pgfqpoint{0.688192in}{9.576569in}}{\pgfqpoint{11.096108in}{4.223431in}}%
\pgfusepath{clip}%
\pgfsetbuttcap%
\pgfsetroundjoin%
\pgfsetlinewidth{1.003750pt}%
\definecolor{currentstroke}{rgb}{0.450000,0.450000,0.450000}%
\pgfsetstrokecolor{currentstroke}%
\pgfsetdash{}{0pt}%
\pgfpathmoveto{\pgfqpoint{2.352608in}{11.041168in}}%
\pgfpathcurveto{\pgfqpoint{2.361816in}{11.041168in}}{\pgfqpoint{2.370649in}{11.044827in}}{\pgfqpoint{2.377160in}{11.051338in}}%
\pgfpathcurveto{\pgfqpoint{2.383671in}{11.057850in}}{\pgfqpoint{2.387330in}{11.066682in}}{\pgfqpoint{2.387330in}{11.075891in}}%
\pgfpathcurveto{\pgfqpoint{2.387330in}{11.085099in}}{\pgfqpoint{2.383671in}{11.093932in}}{\pgfqpoint{2.377160in}{11.100443in}}%
\pgfpathcurveto{\pgfqpoint{2.370649in}{11.106954in}}{\pgfqpoint{2.361816in}{11.110613in}}{\pgfqpoint{2.352608in}{11.110613in}}%
\pgfpathcurveto{\pgfqpoint{2.343399in}{11.110613in}}{\pgfqpoint{2.334567in}{11.106954in}}{\pgfqpoint{2.328055in}{11.100443in}}%
\pgfpathcurveto{\pgfqpoint{2.321544in}{11.093932in}}{\pgfqpoint{2.317885in}{11.085099in}}{\pgfqpoint{2.317885in}{11.075891in}}%
\pgfpathcurveto{\pgfqpoint{2.317885in}{11.066682in}}{\pgfqpoint{2.321544in}{11.057850in}}{\pgfqpoint{2.328055in}{11.051338in}}%
\pgfpathcurveto{\pgfqpoint{2.334567in}{11.044827in}}{\pgfqpoint{2.343399in}{11.041168in}}{\pgfqpoint{2.352608in}{11.041168in}}%
\pgfpathlineto{\pgfqpoint{2.352608in}{11.041168in}}%
\pgfpathclose%
\pgfusepath{stroke}%
\end{pgfscope}%
\begin{pgfscope}%
\pgfpathrectangle{\pgfqpoint{0.688192in}{9.576569in}}{\pgfqpoint{11.096108in}{4.223431in}}%
\pgfusepath{clip}%
\pgfsetbuttcap%
\pgfsetroundjoin%
\pgfsetlinewidth{1.003750pt}%
\definecolor{currentstroke}{rgb}{0.450000,0.450000,0.450000}%
\pgfsetstrokecolor{currentstroke}%
\pgfsetdash{}{0pt}%
\pgfpathmoveto{\pgfqpoint{2.352608in}{9.966121in}}%
\pgfpathcurveto{\pgfqpoint{2.361816in}{9.966121in}}{\pgfqpoint{2.370649in}{9.969779in}}{\pgfqpoint{2.377160in}{9.976291in}}%
\pgfpathcurveto{\pgfqpoint{2.383671in}{9.982802in}}{\pgfqpoint{2.387330in}{9.991635in}}{\pgfqpoint{2.387330in}{10.000843in}}%
\pgfpathcurveto{\pgfqpoint{2.387330in}{10.010051in}}{\pgfqpoint{2.383671in}{10.018884in}}{\pgfqpoint{2.377160in}{10.025395in}}%
\pgfpathcurveto{\pgfqpoint{2.370649in}{10.031907in}}{\pgfqpoint{2.361816in}{10.035565in}}{\pgfqpoint{2.352608in}{10.035565in}}%
\pgfpathcurveto{\pgfqpoint{2.343399in}{10.035565in}}{\pgfqpoint{2.334567in}{10.031907in}}{\pgfqpoint{2.328055in}{10.025395in}}%
\pgfpathcurveto{\pgfqpoint{2.321544in}{10.018884in}}{\pgfqpoint{2.317885in}{10.010051in}}{\pgfqpoint{2.317885in}{10.000843in}}%
\pgfpathcurveto{\pgfqpoint{2.317885in}{9.991635in}}{\pgfqpoint{2.321544in}{9.982802in}}{\pgfqpoint{2.328055in}{9.976291in}}%
\pgfpathcurveto{\pgfqpoint{2.334567in}{9.969779in}}{\pgfqpoint{2.343399in}{9.966121in}}{\pgfqpoint{2.352608in}{9.966121in}}%
\pgfpathlineto{\pgfqpoint{2.352608in}{9.966121in}}%
\pgfpathclose%
\pgfusepath{stroke}%
\end{pgfscope}%
\begin{pgfscope}%
\pgfpathrectangle{\pgfqpoint{0.688192in}{9.576569in}}{\pgfqpoint{11.096108in}{4.223431in}}%
\pgfusepath{clip}%
\pgfsetbuttcap%
\pgfsetroundjoin%
\pgfsetlinewidth{1.003750pt}%
\definecolor{currentstroke}{rgb}{0.450000,0.450000,0.450000}%
\pgfsetstrokecolor{currentstroke}%
\pgfsetdash{}{0pt}%
\pgfpathmoveto{\pgfqpoint{2.352608in}{10.322779in}}%
\pgfpathcurveto{\pgfqpoint{2.361816in}{10.322779in}}{\pgfqpoint{2.370649in}{10.326437in}}{\pgfqpoint{2.377160in}{10.332949in}}%
\pgfpathcurveto{\pgfqpoint{2.383671in}{10.339460in}}{\pgfqpoint{2.387330in}{10.348293in}}{\pgfqpoint{2.387330in}{10.357501in}}%
\pgfpathcurveto{\pgfqpoint{2.387330in}{10.366710in}}{\pgfqpoint{2.383671in}{10.375542in}}{\pgfqpoint{2.377160in}{10.382053in}}%
\pgfpathcurveto{\pgfqpoint{2.370649in}{10.388565in}}{\pgfqpoint{2.361816in}{10.392223in}}{\pgfqpoint{2.352608in}{10.392223in}}%
\pgfpathcurveto{\pgfqpoint{2.343399in}{10.392223in}}{\pgfqpoint{2.334567in}{10.388565in}}{\pgfqpoint{2.328055in}{10.382053in}}%
\pgfpathcurveto{\pgfqpoint{2.321544in}{10.375542in}}{\pgfqpoint{2.317885in}{10.366710in}}{\pgfqpoint{2.317885in}{10.357501in}}%
\pgfpathcurveto{\pgfqpoint{2.317885in}{10.348293in}}{\pgfqpoint{2.321544in}{10.339460in}}{\pgfqpoint{2.328055in}{10.332949in}}%
\pgfpathcurveto{\pgfqpoint{2.334567in}{10.326437in}}{\pgfqpoint{2.343399in}{10.322779in}}{\pgfqpoint{2.352608in}{10.322779in}}%
\pgfpathlineto{\pgfqpoint{2.352608in}{10.322779in}}%
\pgfpathclose%
\pgfusepath{stroke}%
\end{pgfscope}%
\begin{pgfscope}%
\pgfpathrectangle{\pgfqpoint{0.688192in}{9.576569in}}{\pgfqpoint{11.096108in}{4.223431in}}%
\pgfusepath{clip}%
\pgfsetbuttcap%
\pgfsetroundjoin%
\definecolor{currentfill}{rgb}{0.485914,0.710682,0.485881}%
\pgfsetfillcolor{currentfill}%
\pgfsetlinewidth{0.752812pt}%
\definecolor{currentstroke}{rgb}{0.240000,0.240000,0.240000}%
\pgfsetstrokecolor{currentstroke}%
\pgfsetdash{}{0pt}%
\pgfpathmoveto{\pgfqpoint{3.351257in}{10.085326in}}%
\pgfpathlineto{\pgfqpoint{3.573180in}{10.085326in}}%
\pgfpathlineto{\pgfqpoint{3.573180in}{10.148379in}}%
\pgfpathlineto{\pgfqpoint{3.351257in}{10.148379in}}%
\pgfpathlineto{\pgfqpoint{3.351257in}{10.085326in}}%
\pgfpathclose%
\pgfusepath{stroke,fill}%
\end{pgfscope}%
\begin{pgfscope}%
\pgfpathrectangle{\pgfqpoint{0.688192in}{9.576569in}}{\pgfqpoint{11.096108in}{4.223431in}}%
\pgfusepath{clip}%
\pgfsetbuttcap%
\pgfsetroundjoin%
\definecolor{currentfill}{rgb}{0.371306,0.648087,0.371288}%
\pgfsetfillcolor{currentfill}%
\pgfsetlinewidth{0.752812pt}%
\definecolor{currentstroke}{rgb}{0.240000,0.240000,0.240000}%
\pgfsetstrokecolor{currentstroke}%
\pgfsetdash{}{0pt}%
\pgfpathmoveto{\pgfqpoint{3.240296in}{10.148379in}}%
\pgfpathlineto{\pgfqpoint{3.684141in}{10.148379in}}%
\pgfpathlineto{\pgfqpoint{3.684141in}{10.208221in}}%
\pgfpathlineto{\pgfqpoint{3.240296in}{10.208221in}}%
\pgfpathlineto{\pgfqpoint{3.240296in}{10.148379in}}%
\pgfpathclose%
\pgfusepath{stroke,fill}%
\end{pgfscope}%
\begin{pgfscope}%
\pgfpathrectangle{\pgfqpoint{0.688192in}{9.576569in}}{\pgfqpoint{11.096108in}{4.223431in}}%
\pgfusepath{clip}%
\pgfsetbuttcap%
\pgfsetroundjoin%
\definecolor{currentfill}{rgb}{0.229412,0.570588,0.229412}%
\pgfsetfillcolor{currentfill}%
\pgfsetlinewidth{0.752812pt}%
\definecolor{currentstroke}{rgb}{0.240000,0.240000,0.240000}%
\pgfsetstrokecolor{currentstroke}%
\pgfsetdash{}{0pt}%
\pgfpathmoveto{\pgfqpoint{3.018374in}{10.208221in}}%
\pgfpathlineto{\pgfqpoint{3.906063in}{10.208221in}}%
\pgfpathlineto{\pgfqpoint{3.906063in}{10.357089in}}%
\pgfpathlineto{\pgfqpoint{3.018374in}{10.357089in}}%
\pgfpathlineto{\pgfqpoint{3.018374in}{10.208221in}}%
\pgfpathclose%
\pgfusepath{stroke,fill}%
\end{pgfscope}%
\begin{pgfscope}%
\pgfpathrectangle{\pgfqpoint{0.688192in}{9.576569in}}{\pgfqpoint{11.096108in}{4.223431in}}%
\pgfusepath{clip}%
\pgfsetbuttcap%
\pgfsetroundjoin%
\definecolor{currentfill}{rgb}{0.371306,0.648087,0.371288}%
\pgfsetfillcolor{currentfill}%
\pgfsetlinewidth{0.752812pt}%
\definecolor{currentstroke}{rgb}{0.240000,0.240000,0.240000}%
\pgfsetstrokecolor{currentstroke}%
\pgfsetdash{}{0pt}%
\pgfpathmoveto{\pgfqpoint{3.240296in}{10.357089in}}%
\pgfpathlineto{\pgfqpoint{3.684141in}{10.357089in}}%
\pgfpathlineto{\pgfqpoint{3.684141in}{10.366717in}}%
\pgfpathlineto{\pgfqpoint{3.240296in}{10.366717in}}%
\pgfpathlineto{\pgfqpoint{3.240296in}{10.357089in}}%
\pgfpathclose%
\pgfusepath{stroke,fill}%
\end{pgfscope}%
\begin{pgfscope}%
\pgfpathrectangle{\pgfqpoint{0.688192in}{9.576569in}}{\pgfqpoint{11.096108in}{4.223431in}}%
\pgfusepath{clip}%
\pgfsetbuttcap%
\pgfsetroundjoin%
\definecolor{currentfill}{rgb}{0.485914,0.710682,0.485881}%
\pgfsetfillcolor{currentfill}%
\pgfsetlinewidth{0.752812pt}%
\definecolor{currentstroke}{rgb}{0.240000,0.240000,0.240000}%
\pgfsetstrokecolor{currentstroke}%
\pgfsetdash{}{0pt}%
\pgfpathmoveto{\pgfqpoint{3.351257in}{10.366717in}}%
\pgfpathlineto{\pgfqpoint{3.573180in}{10.366717in}}%
\pgfpathlineto{\pgfqpoint{3.573180in}{10.376075in}}%
\pgfpathlineto{\pgfqpoint{3.351257in}{10.376075in}}%
\pgfpathlineto{\pgfqpoint{3.351257in}{10.366717in}}%
\pgfpathclose%
\pgfusepath{stroke,fill}%
\end{pgfscope}%
\begin{pgfscope}%
\pgfpathrectangle{\pgfqpoint{0.688192in}{9.576569in}}{\pgfqpoint{11.096108in}{4.223431in}}%
\pgfusepath{clip}%
\pgfsetbuttcap%
\pgfsetroundjoin%
\pgfsetlinewidth{1.003750pt}%
\definecolor{currentstroke}{rgb}{0.450000,0.450000,0.450000}%
\pgfsetstrokecolor{currentstroke}%
\pgfsetdash{}{0pt}%
\pgfpathmoveto{\pgfqpoint{3.462219in}{10.005680in}}%
\pgfpathcurveto{\pgfqpoint{3.471427in}{10.005680in}}{\pgfqpoint{3.480259in}{10.009338in}}{\pgfqpoint{3.486771in}{10.015850in}}%
\pgfpathcurveto{\pgfqpoint{3.493282in}{10.022361in}}{\pgfqpoint{3.496941in}{10.031194in}}{\pgfqpoint{3.496941in}{10.040402in}}%
\pgfpathcurveto{\pgfqpoint{3.496941in}{10.049611in}}{\pgfqpoint{3.493282in}{10.058443in}}{\pgfqpoint{3.486771in}{10.064954in}}%
\pgfpathcurveto{\pgfqpoint{3.480259in}{10.071466in}}{\pgfqpoint{3.471427in}{10.075124in}}{\pgfqpoint{3.462219in}{10.075124in}}%
\pgfpathcurveto{\pgfqpoint{3.453010in}{10.075124in}}{\pgfqpoint{3.444178in}{10.071466in}}{\pgfqpoint{3.437666in}{10.064954in}}%
\pgfpathcurveto{\pgfqpoint{3.431155in}{10.058443in}}{\pgfqpoint{3.427496in}{10.049611in}}{\pgfqpoint{3.427496in}{10.040402in}}%
\pgfpathcurveto{\pgfqpoint{3.427496in}{10.031194in}}{\pgfqpoint{3.431155in}{10.022361in}}{\pgfqpoint{3.437666in}{10.015850in}}%
\pgfpathcurveto{\pgfqpoint{3.444178in}{10.009338in}}{\pgfqpoint{3.453010in}{10.005680in}}{\pgfqpoint{3.462219in}{10.005680in}}%
\pgfpathlineto{\pgfqpoint{3.462219in}{10.005680in}}%
\pgfpathclose%
\pgfusepath{stroke}%
\end{pgfscope}%
\begin{pgfscope}%
\pgfpathrectangle{\pgfqpoint{0.688192in}{9.576569in}}{\pgfqpoint{11.096108in}{4.223431in}}%
\pgfusepath{clip}%
\pgfsetbuttcap%
\pgfsetroundjoin%
\pgfsetlinewidth{1.003750pt}%
\definecolor{currentstroke}{rgb}{0.450000,0.450000,0.450000}%
\pgfsetstrokecolor{currentstroke}%
\pgfsetdash{}{0pt}%
\pgfpathmoveto{\pgfqpoint{3.462219in}{9.600749in}}%
\pgfpathcurveto{\pgfqpoint{3.471427in}{9.600749in}}{\pgfqpoint{3.480259in}{9.604407in}}{\pgfqpoint{3.486771in}{9.610918in}}%
\pgfpathcurveto{\pgfqpoint{3.493282in}{9.617430in}}{\pgfqpoint{3.496941in}{9.626262in}}{\pgfqpoint{3.496941in}{9.635471in}}%
\pgfpathcurveto{\pgfqpoint{3.496941in}{9.644679in}}{\pgfqpoint{3.493282in}{9.653512in}}{\pgfqpoint{3.486771in}{9.660023in}}%
\pgfpathcurveto{\pgfqpoint{3.480259in}{9.666534in}}{\pgfqpoint{3.471427in}{9.670193in}}{\pgfqpoint{3.462219in}{9.670193in}}%
\pgfpathcurveto{\pgfqpoint{3.453010in}{9.670193in}}{\pgfqpoint{3.444178in}{9.666534in}}{\pgfqpoint{3.437666in}{9.660023in}}%
\pgfpathcurveto{\pgfqpoint{3.431155in}{9.653512in}}{\pgfqpoint{3.427496in}{9.644679in}}{\pgfqpoint{3.427496in}{9.635471in}}%
\pgfpathcurveto{\pgfqpoint{3.427496in}{9.626262in}}{\pgfqpoint{3.431155in}{9.617430in}}{\pgfqpoint{3.437666in}{9.610918in}}%
\pgfpathcurveto{\pgfqpoint{3.444178in}{9.604407in}}{\pgfqpoint{3.453010in}{9.600749in}}{\pgfqpoint{3.462219in}{9.600749in}}%
\pgfpathlineto{\pgfqpoint{3.462219in}{9.600749in}}%
\pgfpathclose%
\pgfusepath{stroke}%
\end{pgfscope}%
\begin{pgfscope}%
\pgfpathrectangle{\pgfqpoint{0.688192in}{9.576569in}}{\pgfqpoint{11.096108in}{4.223431in}}%
\pgfusepath{clip}%
\pgfsetbuttcap%
\pgfsetroundjoin%
\pgfsetlinewidth{1.003750pt}%
\definecolor{currentstroke}{rgb}{0.450000,0.450000,0.450000}%
\pgfsetstrokecolor{currentstroke}%
\pgfsetdash{}{0pt}%
\pgfpathmoveto{\pgfqpoint{3.462219in}{10.360632in}}%
\pgfpathcurveto{\pgfqpoint{3.471427in}{10.360632in}}{\pgfqpoint{3.480259in}{10.364291in}}{\pgfqpoint{3.486771in}{10.370802in}}%
\pgfpathcurveto{\pgfqpoint{3.493282in}{10.377314in}}{\pgfqpoint{3.496941in}{10.386146in}}{\pgfqpoint{3.496941in}{10.395355in}}%
\pgfpathcurveto{\pgfqpoint{3.496941in}{10.404563in}}{\pgfqpoint{3.493282in}{10.413396in}}{\pgfqpoint{3.486771in}{10.419907in}}%
\pgfpathcurveto{\pgfqpoint{3.480259in}{10.426418in}}{\pgfqpoint{3.471427in}{10.430077in}}{\pgfqpoint{3.462219in}{10.430077in}}%
\pgfpathcurveto{\pgfqpoint{3.453010in}{10.430077in}}{\pgfqpoint{3.444178in}{10.426418in}}{\pgfqpoint{3.437666in}{10.419907in}}%
\pgfpathcurveto{\pgfqpoint{3.431155in}{10.413396in}}{\pgfqpoint{3.427496in}{10.404563in}}{\pgfqpoint{3.427496in}{10.395355in}}%
\pgfpathcurveto{\pgfqpoint{3.427496in}{10.386146in}}{\pgfqpoint{3.431155in}{10.377314in}}{\pgfqpoint{3.437666in}{10.370802in}}%
\pgfpathcurveto{\pgfqpoint{3.444178in}{10.364291in}}{\pgfqpoint{3.453010in}{10.360632in}}{\pgfqpoint{3.462219in}{10.360632in}}%
\pgfpathlineto{\pgfqpoint{3.462219in}{10.360632in}}%
\pgfpathclose%
\pgfusepath{stroke}%
\end{pgfscope}%
\begin{pgfscope}%
\pgfpathrectangle{\pgfqpoint{0.688192in}{9.576569in}}{\pgfqpoint{11.096108in}{4.223431in}}%
\pgfusepath{clip}%
\pgfsetbuttcap%
\pgfsetroundjoin%
\pgfsetlinewidth{1.003750pt}%
\definecolor{currentstroke}{rgb}{0.450000,0.450000,0.450000}%
\pgfsetstrokecolor{currentstroke}%
\pgfsetdash{}{0pt}%
\pgfpathmoveto{\pgfqpoint{3.462219in}{10.043403in}}%
\pgfpathcurveto{\pgfqpoint{3.471427in}{10.043403in}}{\pgfqpoint{3.480259in}{10.047062in}}{\pgfqpoint{3.486771in}{10.053573in}}%
\pgfpathcurveto{\pgfqpoint{3.493282in}{10.060085in}}{\pgfqpoint{3.496941in}{10.068917in}}{\pgfqpoint{3.496941in}{10.078126in}}%
\pgfpathcurveto{\pgfqpoint{3.496941in}{10.087334in}}{\pgfqpoint{3.493282in}{10.096167in}}{\pgfqpoint{3.486771in}{10.102678in}}%
\pgfpathcurveto{\pgfqpoint{3.480259in}{10.109189in}}{\pgfqpoint{3.471427in}{10.112848in}}{\pgfqpoint{3.462219in}{10.112848in}}%
\pgfpathcurveto{\pgfqpoint{3.453010in}{10.112848in}}{\pgfqpoint{3.444178in}{10.109189in}}{\pgfqpoint{3.437666in}{10.102678in}}%
\pgfpathcurveto{\pgfqpoint{3.431155in}{10.096167in}}{\pgfqpoint{3.427496in}{10.087334in}}{\pgfqpoint{3.427496in}{10.078126in}}%
\pgfpathcurveto{\pgfqpoint{3.427496in}{10.068917in}}{\pgfqpoint{3.431155in}{10.060085in}}{\pgfqpoint{3.437666in}{10.053573in}}%
\pgfpathcurveto{\pgfqpoint{3.444178in}{10.047062in}}{\pgfqpoint{3.453010in}{10.043403in}}{\pgfqpoint{3.462219in}{10.043403in}}%
\pgfpathlineto{\pgfqpoint{3.462219in}{10.043403in}}%
\pgfpathclose%
\pgfusepath{stroke}%
\end{pgfscope}%
\begin{pgfscope}%
\pgfpathrectangle{\pgfqpoint{0.688192in}{9.576569in}}{\pgfqpoint{11.096108in}{4.223431in}}%
\pgfusepath{clip}%
\pgfsetbuttcap%
\pgfsetroundjoin%
\pgfsetlinewidth{1.003750pt}%
\definecolor{currentstroke}{rgb}{0.450000,0.450000,0.450000}%
\pgfsetstrokecolor{currentstroke}%
\pgfsetdash{}{0pt}%
\pgfpathmoveto{\pgfqpoint{3.462219in}{10.341565in}}%
\pgfpathcurveto{\pgfqpoint{3.471427in}{10.341565in}}{\pgfqpoint{3.480259in}{10.345224in}}{\pgfqpoint{3.486771in}{10.351735in}}%
\pgfpathcurveto{\pgfqpoint{3.493282in}{10.358247in}}{\pgfqpoint{3.496941in}{10.367079in}}{\pgfqpoint{3.496941in}{10.376288in}}%
\pgfpathcurveto{\pgfqpoint{3.496941in}{10.385496in}}{\pgfqpoint{3.493282in}{10.394329in}}{\pgfqpoint{3.486771in}{10.400840in}}%
\pgfpathcurveto{\pgfqpoint{3.480259in}{10.407351in}}{\pgfqpoint{3.471427in}{10.411010in}}{\pgfqpoint{3.462219in}{10.411010in}}%
\pgfpathcurveto{\pgfqpoint{3.453010in}{10.411010in}}{\pgfqpoint{3.444178in}{10.407351in}}{\pgfqpoint{3.437666in}{10.400840in}}%
\pgfpathcurveto{\pgfqpoint{3.431155in}{10.394329in}}{\pgfqpoint{3.427496in}{10.385496in}}{\pgfqpoint{3.427496in}{10.376288in}}%
\pgfpathcurveto{\pgfqpoint{3.427496in}{10.367079in}}{\pgfqpoint{3.431155in}{10.358247in}}{\pgfqpoint{3.437666in}{10.351735in}}%
\pgfpathcurveto{\pgfqpoint{3.444178in}{10.345224in}}{\pgfqpoint{3.453010in}{10.341565in}}{\pgfqpoint{3.462219in}{10.341565in}}%
\pgfpathlineto{\pgfqpoint{3.462219in}{10.341565in}}%
\pgfpathclose%
\pgfusepath{stroke}%
\end{pgfscope}%
\begin{pgfscope}%
\pgfpathrectangle{\pgfqpoint{0.688192in}{9.576569in}}{\pgfqpoint{11.096108in}{4.223431in}}%
\pgfusepath{clip}%
\pgfsetbuttcap%
\pgfsetroundjoin%
\pgfsetlinewidth{1.003750pt}%
\definecolor{currentstroke}{rgb}{0.450000,0.450000,0.450000}%
\pgfsetstrokecolor{currentstroke}%
\pgfsetdash{}{0pt}%
\pgfpathmoveto{\pgfqpoint{3.462219in}{10.807667in}}%
\pgfpathcurveto{\pgfqpoint{3.471427in}{10.807667in}}{\pgfqpoint{3.480259in}{10.811325in}}{\pgfqpoint{3.486771in}{10.817837in}}%
\pgfpathcurveto{\pgfqpoint{3.493282in}{10.824348in}}{\pgfqpoint{3.496941in}{10.833181in}}{\pgfqpoint{3.496941in}{10.842389in}}%
\pgfpathcurveto{\pgfqpoint{3.496941in}{10.851598in}}{\pgfqpoint{3.493282in}{10.860430in}}{\pgfqpoint{3.486771in}{10.866941in}}%
\pgfpathcurveto{\pgfqpoint{3.480259in}{10.873453in}}{\pgfqpoint{3.471427in}{10.877111in}}{\pgfqpoint{3.462219in}{10.877111in}}%
\pgfpathcurveto{\pgfqpoint{3.453010in}{10.877111in}}{\pgfqpoint{3.444178in}{10.873453in}}{\pgfqpoint{3.437666in}{10.866941in}}%
\pgfpathcurveto{\pgfqpoint{3.431155in}{10.860430in}}{\pgfqpoint{3.427496in}{10.851598in}}{\pgfqpoint{3.427496in}{10.842389in}}%
\pgfpathcurveto{\pgfqpoint{3.427496in}{10.833181in}}{\pgfqpoint{3.431155in}{10.824348in}}{\pgfqpoint{3.437666in}{10.817837in}}%
\pgfpathcurveto{\pgfqpoint{3.444178in}{10.811325in}}{\pgfqpoint{3.453010in}{10.807667in}}{\pgfqpoint{3.462219in}{10.807667in}}%
\pgfpathlineto{\pgfqpoint{3.462219in}{10.807667in}}%
\pgfpathclose%
\pgfusepath{stroke}%
\end{pgfscope}%
\begin{pgfscope}%
\pgfpathrectangle{\pgfqpoint{0.688192in}{9.576569in}}{\pgfqpoint{11.096108in}{4.223431in}}%
\pgfusepath{clip}%
\pgfsetbuttcap%
\pgfsetroundjoin%
\pgfsetlinewidth{1.003750pt}%
\definecolor{currentstroke}{rgb}{0.450000,0.450000,0.450000}%
\pgfsetstrokecolor{currentstroke}%
\pgfsetdash{}{0pt}%
\pgfpathmoveto{\pgfqpoint{3.462219in}{9.970896in}}%
\pgfpathcurveto{\pgfqpoint{3.471427in}{9.970896in}}{\pgfqpoint{3.480259in}{9.974554in}}{\pgfqpoint{3.486771in}{9.981065in}}%
\pgfpathcurveto{\pgfqpoint{3.493282in}{9.987577in}}{\pgfqpoint{3.496941in}{9.996409in}}{\pgfqpoint{3.496941in}{10.005618in}}%
\pgfpathcurveto{\pgfqpoint{3.496941in}{10.014826in}}{\pgfqpoint{3.493282in}{10.023659in}}{\pgfqpoint{3.486771in}{10.030170in}}%
\pgfpathcurveto{\pgfqpoint{3.480259in}{10.036681in}}{\pgfqpoint{3.471427in}{10.040340in}}{\pgfqpoint{3.462219in}{10.040340in}}%
\pgfpathcurveto{\pgfqpoint{3.453010in}{10.040340in}}{\pgfqpoint{3.444178in}{10.036681in}}{\pgfqpoint{3.437666in}{10.030170in}}%
\pgfpathcurveto{\pgfqpoint{3.431155in}{10.023659in}}{\pgfqpoint{3.427496in}{10.014826in}}{\pgfqpoint{3.427496in}{10.005618in}}%
\pgfpathcurveto{\pgfqpoint{3.427496in}{9.996409in}}{\pgfqpoint{3.431155in}{9.987577in}}{\pgfqpoint{3.437666in}{9.981065in}}%
\pgfpathcurveto{\pgfqpoint{3.444178in}{9.974554in}}{\pgfqpoint{3.453010in}{9.970896in}}{\pgfqpoint{3.462219in}{9.970896in}}%
\pgfpathlineto{\pgfqpoint{3.462219in}{9.970896in}}%
\pgfpathclose%
\pgfusepath{stroke}%
\end{pgfscope}%
\begin{pgfscope}%
\pgfpathrectangle{\pgfqpoint{0.688192in}{9.576569in}}{\pgfqpoint{11.096108in}{4.223431in}}%
\pgfusepath{clip}%
\pgfsetbuttcap%
\pgfsetroundjoin%
\pgfsetlinewidth{1.003750pt}%
\definecolor{currentstroke}{rgb}{0.450000,0.450000,0.450000}%
\pgfsetstrokecolor{currentstroke}%
\pgfsetdash{}{0pt}%
\pgfpathmoveto{\pgfqpoint{3.462219in}{10.342480in}}%
\pgfpathcurveto{\pgfqpoint{3.471427in}{10.342480in}}{\pgfqpoint{3.480259in}{10.346139in}}{\pgfqpoint{3.486771in}{10.352650in}}%
\pgfpathcurveto{\pgfqpoint{3.493282in}{10.359162in}}{\pgfqpoint{3.496941in}{10.367994in}}{\pgfqpoint{3.496941in}{10.377203in}}%
\pgfpathcurveto{\pgfqpoint{3.496941in}{10.386411in}}{\pgfqpoint{3.493282in}{10.395244in}}{\pgfqpoint{3.486771in}{10.401755in}}%
\pgfpathcurveto{\pgfqpoint{3.480259in}{10.408266in}}{\pgfqpoint{3.471427in}{10.411925in}}{\pgfqpoint{3.462219in}{10.411925in}}%
\pgfpathcurveto{\pgfqpoint{3.453010in}{10.411925in}}{\pgfqpoint{3.444178in}{10.408266in}}{\pgfqpoint{3.437666in}{10.401755in}}%
\pgfpathcurveto{\pgfqpoint{3.431155in}{10.395244in}}{\pgfqpoint{3.427496in}{10.386411in}}{\pgfqpoint{3.427496in}{10.377203in}}%
\pgfpathcurveto{\pgfqpoint{3.427496in}{10.367994in}}{\pgfqpoint{3.431155in}{10.359162in}}{\pgfqpoint{3.437666in}{10.352650in}}%
\pgfpathcurveto{\pgfqpoint{3.444178in}{10.346139in}}{\pgfqpoint{3.453010in}{10.342480in}}{\pgfqpoint{3.462219in}{10.342480in}}%
\pgfpathlineto{\pgfqpoint{3.462219in}{10.342480in}}%
\pgfpathclose%
\pgfusepath{stroke}%
\end{pgfscope}%
\begin{pgfscope}%
\pgfpathrectangle{\pgfqpoint{0.688192in}{9.576569in}}{\pgfqpoint{11.096108in}{4.223431in}}%
\pgfusepath{clip}%
\pgfsetbuttcap%
\pgfsetroundjoin%
\definecolor{currentfill}{rgb}{0.825642,0.497939,0.499757}%
\pgfsetfillcolor{currentfill}%
\pgfsetlinewidth{0.752812pt}%
\definecolor{currentstroke}{rgb}{0.240000,0.240000,0.240000}%
\pgfsetstrokecolor{currentstroke}%
\pgfsetdash{}{0pt}%
\pgfpathmoveto{\pgfqpoint{4.460868in}{9.576569in}}%
\pgfpathlineto{\pgfqpoint{4.682790in}{9.576569in}}%
\pgfpathlineto{\pgfqpoint{4.682790in}{9.576569in}}%
\pgfpathlineto{\pgfqpoint{4.460868in}{9.576569in}}%
\pgfpathlineto{\pgfqpoint{4.460868in}{9.576569in}}%
\pgfpathclose%
\pgfusepath{stroke,fill}%
\end{pgfscope}%
\begin{pgfscope}%
\pgfpathrectangle{\pgfqpoint{0.688192in}{9.576569in}}{\pgfqpoint{11.096108in}{4.223431in}}%
\pgfusepath{clip}%
\pgfsetbuttcap%
\pgfsetroundjoin%
\definecolor{currentfill}{rgb}{0.793378,0.382120,0.384440}%
\pgfsetfillcolor{currentfill}%
\pgfsetlinewidth{0.752812pt}%
\definecolor{currentstroke}{rgb}{0.240000,0.240000,0.240000}%
\pgfsetstrokecolor{currentstroke}%
\pgfsetdash{}{0pt}%
\pgfpathmoveto{\pgfqpoint{4.349907in}{9.576569in}}%
\pgfpathlineto{\pgfqpoint{4.793751in}{9.576569in}}%
\pgfpathlineto{\pgfqpoint{4.793751in}{9.576569in}}%
\pgfpathlineto{\pgfqpoint{4.349907in}{9.576569in}}%
\pgfpathlineto{\pgfqpoint{4.349907in}{9.576569in}}%
\pgfpathclose%
\pgfusepath{stroke,fill}%
\end{pgfscope}%
\begin{pgfscope}%
\pgfpathrectangle{\pgfqpoint{0.688192in}{9.576569in}}{\pgfqpoint{11.096108in}{4.223431in}}%
\pgfusepath{clip}%
\pgfsetbuttcap%
\pgfsetroundjoin%
\definecolor{currentfill}{rgb}{0.753431,0.238725,0.241667}%
\pgfsetfillcolor{currentfill}%
\pgfsetlinewidth{0.752812pt}%
\definecolor{currentstroke}{rgb}{0.240000,0.240000,0.240000}%
\pgfsetstrokecolor{currentstroke}%
\pgfsetdash{}{0pt}%
\pgfpathmoveto{\pgfqpoint{4.127985in}{9.576569in}}%
\pgfpathlineto{\pgfqpoint{5.015674in}{9.576569in}}%
\pgfpathlineto{\pgfqpoint{5.015674in}{9.576569in}}%
\pgfpathlineto{\pgfqpoint{4.127985in}{9.576569in}}%
\pgfpathlineto{\pgfqpoint{4.127985in}{9.576569in}}%
\pgfpathclose%
\pgfusepath{stroke,fill}%
\end{pgfscope}%
\begin{pgfscope}%
\pgfpathrectangle{\pgfqpoint{0.688192in}{9.576569in}}{\pgfqpoint{11.096108in}{4.223431in}}%
\pgfusepath{clip}%
\pgfsetbuttcap%
\pgfsetroundjoin%
\definecolor{currentfill}{rgb}{0.793378,0.382120,0.384440}%
\pgfsetfillcolor{currentfill}%
\pgfsetlinewidth{0.752812pt}%
\definecolor{currentstroke}{rgb}{0.240000,0.240000,0.240000}%
\pgfsetstrokecolor{currentstroke}%
\pgfsetdash{}{0pt}%
\pgfpathmoveto{\pgfqpoint{4.349907in}{9.576569in}}%
\pgfpathlineto{\pgfqpoint{4.793751in}{9.576569in}}%
\pgfpathlineto{\pgfqpoint{4.793751in}{9.576569in}}%
\pgfpathlineto{\pgfqpoint{4.349907in}{9.576569in}}%
\pgfpathlineto{\pgfqpoint{4.349907in}{9.576569in}}%
\pgfpathclose%
\pgfusepath{stroke,fill}%
\end{pgfscope}%
\begin{pgfscope}%
\pgfpathrectangle{\pgfqpoint{0.688192in}{9.576569in}}{\pgfqpoint{11.096108in}{4.223431in}}%
\pgfusepath{clip}%
\pgfsetbuttcap%
\pgfsetroundjoin%
\definecolor{currentfill}{rgb}{0.825642,0.497939,0.499757}%
\pgfsetfillcolor{currentfill}%
\pgfsetlinewidth{0.752812pt}%
\definecolor{currentstroke}{rgb}{0.240000,0.240000,0.240000}%
\pgfsetstrokecolor{currentstroke}%
\pgfsetdash{}{0pt}%
\pgfpathmoveto{\pgfqpoint{4.460868in}{9.576569in}}%
\pgfpathlineto{\pgfqpoint{4.682790in}{9.576569in}}%
\pgfpathlineto{\pgfqpoint{4.682790in}{9.576569in}}%
\pgfpathlineto{\pgfqpoint{4.460868in}{9.576569in}}%
\pgfpathlineto{\pgfqpoint{4.460868in}{9.576569in}}%
\pgfpathclose%
\pgfusepath{stroke,fill}%
\end{pgfscope}%
\begin{pgfscope}%
\pgfpathrectangle{\pgfqpoint{0.688192in}{9.576569in}}{\pgfqpoint{11.096108in}{4.223431in}}%
\pgfusepath{clip}%
\pgfsetbuttcap%
\pgfsetroundjoin%
\pgfsetlinewidth{1.003750pt}%
\definecolor{currentstroke}{rgb}{0.450000,0.450000,0.450000}%
\pgfsetstrokecolor{currentstroke}%
\pgfsetdash{}{0pt}%
\pgfpathmoveto{\pgfqpoint{0.000000in}{-0.034722in}}%
\pgfpathcurveto{\pgfqpoint{0.009208in}{-0.034722in}}{\pgfqpoint{0.018041in}{-0.031064in}}{\pgfqpoint{0.024552in}{-0.024552in}}%
\pgfpathcurveto{\pgfqpoint{0.031064in}{-0.018041in}}{\pgfqpoint{0.034722in}{-0.009208in}}{\pgfqpoint{0.034722in}{0.000000in}}%
\pgfpathcurveto{\pgfqpoint{0.034722in}{0.009208in}}{\pgfqpoint{0.031064in}{0.018041in}}{\pgfqpoint{0.024552in}{0.024552in}}%
\pgfpathcurveto{\pgfqpoint{0.018041in}{0.031064in}}{\pgfqpoint{0.009208in}{0.034722in}}{\pgfqpoint{0.000000in}{0.034722in}}%
\pgfpathcurveto{\pgfqpoint{-0.009208in}{0.034722in}}{\pgfqpoint{-0.018041in}{0.031064in}}{\pgfqpoint{-0.024552in}{0.024552in}}%
\pgfpathcurveto{\pgfqpoint{-0.031064in}{0.018041in}}{\pgfqpoint{-0.034722in}{0.009208in}}{\pgfqpoint{-0.034722in}{0.000000in}}%
\pgfpathcurveto{\pgfqpoint{-0.034722in}{-0.009208in}}{\pgfqpoint{-0.031064in}{-0.018041in}}{\pgfqpoint{-0.024552in}{-0.024552in}}%
\pgfpathcurveto{\pgfqpoint{-0.018041in}{-0.031064in}}{\pgfqpoint{-0.009208in}{-0.034722in}}{\pgfqpoint{0.000000in}{-0.034722in}}%
\pgfusepath{stroke}%
\end{pgfscope}%
\begin{pgfscope}%
\pgfpathrectangle{\pgfqpoint{0.688192in}{9.576569in}}{\pgfqpoint{11.096108in}{4.223431in}}%
\pgfusepath{clip}%
\pgfsetbuttcap%
\pgfsetroundjoin%
\definecolor{currentfill}{rgb}{0.713429,0.629184,0.790364}%
\pgfsetfillcolor{currentfill}%
\pgfsetlinewidth{0.752812pt}%
\definecolor{currentstroke}{rgb}{0.240000,0.240000,0.240000}%
\pgfsetstrokecolor{currentstroke}%
\pgfsetdash{}{0pt}%
\pgfpathmoveto{\pgfqpoint{5.570479in}{9.576569in}}%
\pgfpathlineto{\pgfqpoint{5.792401in}{9.576569in}}%
\pgfpathlineto{\pgfqpoint{5.792401in}{9.576569in}}%
\pgfpathlineto{\pgfqpoint{5.570479in}{9.576569in}}%
\pgfpathlineto{\pgfqpoint{5.570479in}{9.576569in}}%
\pgfpathclose%
\pgfusepath{stroke,fill}%
\end{pgfscope}%
\begin{pgfscope}%
\pgfpathrectangle{\pgfqpoint{0.688192in}{9.576569in}}{\pgfqpoint{11.096108in}{4.223431in}}%
\pgfusepath{clip}%
\pgfsetbuttcap%
\pgfsetroundjoin%
\definecolor{currentfill}{rgb}{0.653111,0.547371,0.749551}%
\pgfsetfillcolor{currentfill}%
\pgfsetlinewidth{0.752812pt}%
\definecolor{currentstroke}{rgb}{0.240000,0.240000,0.240000}%
\pgfsetstrokecolor{currentstroke}%
\pgfsetdash{}{0pt}%
\pgfpathmoveto{\pgfqpoint{5.459518in}{9.576569in}}%
\pgfpathlineto{\pgfqpoint{5.903362in}{9.576569in}}%
\pgfpathlineto{\pgfqpoint{5.903362in}{9.576569in}}%
\pgfpathlineto{\pgfqpoint{5.459518in}{9.576569in}}%
\pgfpathlineto{\pgfqpoint{5.459518in}{9.576569in}}%
\pgfpathclose%
\pgfusepath{stroke,fill}%
\end{pgfscope}%
\begin{pgfscope}%
\pgfpathrectangle{\pgfqpoint{0.688192in}{9.576569in}}{\pgfqpoint{11.096108in}{4.223431in}}%
\pgfusepath{clip}%
\pgfsetbuttcap%
\pgfsetroundjoin%
\definecolor{currentfill}{rgb}{0.578431,0.446078,0.699020}%
\pgfsetfillcolor{currentfill}%
\pgfsetlinewidth{0.752812pt}%
\definecolor{currentstroke}{rgb}{0.240000,0.240000,0.240000}%
\pgfsetstrokecolor{currentstroke}%
\pgfsetdash{}{0pt}%
\pgfpathmoveto{\pgfqpoint{5.237596in}{9.576569in}}%
\pgfpathlineto{\pgfqpoint{6.125284in}{9.576569in}}%
\pgfpathlineto{\pgfqpoint{6.125284in}{9.576569in}}%
\pgfpathlineto{\pgfqpoint{5.237596in}{9.576569in}}%
\pgfpathlineto{\pgfqpoint{5.237596in}{9.576569in}}%
\pgfpathclose%
\pgfusepath{stroke,fill}%
\end{pgfscope}%
\begin{pgfscope}%
\pgfpathrectangle{\pgfqpoint{0.688192in}{9.576569in}}{\pgfqpoint{11.096108in}{4.223431in}}%
\pgfusepath{clip}%
\pgfsetbuttcap%
\pgfsetroundjoin%
\definecolor{currentfill}{rgb}{0.653111,0.547371,0.749551}%
\pgfsetfillcolor{currentfill}%
\pgfsetlinewidth{0.752812pt}%
\definecolor{currentstroke}{rgb}{0.240000,0.240000,0.240000}%
\pgfsetstrokecolor{currentstroke}%
\pgfsetdash{}{0pt}%
\pgfpathmoveto{\pgfqpoint{5.459518in}{9.576569in}}%
\pgfpathlineto{\pgfqpoint{5.903362in}{9.576569in}}%
\pgfpathlineto{\pgfqpoint{5.903362in}{9.576569in}}%
\pgfpathlineto{\pgfqpoint{5.459518in}{9.576569in}}%
\pgfpathlineto{\pgfqpoint{5.459518in}{9.576569in}}%
\pgfpathclose%
\pgfusepath{stroke,fill}%
\end{pgfscope}%
\begin{pgfscope}%
\pgfpathrectangle{\pgfqpoint{0.688192in}{9.576569in}}{\pgfqpoint{11.096108in}{4.223431in}}%
\pgfusepath{clip}%
\pgfsetbuttcap%
\pgfsetroundjoin%
\definecolor{currentfill}{rgb}{0.713429,0.629184,0.790364}%
\pgfsetfillcolor{currentfill}%
\pgfsetlinewidth{0.752812pt}%
\definecolor{currentstroke}{rgb}{0.240000,0.240000,0.240000}%
\pgfsetstrokecolor{currentstroke}%
\pgfsetdash{}{0pt}%
\pgfpathmoveto{\pgfqpoint{5.570479in}{9.576569in}}%
\pgfpathlineto{\pgfqpoint{5.792401in}{9.576569in}}%
\pgfpathlineto{\pgfqpoint{5.792401in}{9.576569in}}%
\pgfpathlineto{\pgfqpoint{5.570479in}{9.576569in}}%
\pgfpathlineto{\pgfqpoint{5.570479in}{9.576569in}}%
\pgfpathclose%
\pgfusepath{stroke,fill}%
\end{pgfscope}%
\begin{pgfscope}%
\pgfpathrectangle{\pgfqpoint{0.688192in}{9.576569in}}{\pgfqpoint{11.096108in}{4.223431in}}%
\pgfusepath{clip}%
\pgfsetbuttcap%
\pgfsetroundjoin%
\pgfsetlinewidth{1.003750pt}%
\definecolor{currentstroke}{rgb}{0.450000,0.450000,0.450000}%
\pgfsetstrokecolor{currentstroke}%
\pgfsetdash{}{0pt}%
\pgfpathmoveto{\pgfqpoint{0.000000in}{-0.034722in}}%
\pgfpathcurveto{\pgfqpoint{0.009208in}{-0.034722in}}{\pgfqpoint{0.018041in}{-0.031064in}}{\pgfqpoint{0.024552in}{-0.024552in}}%
\pgfpathcurveto{\pgfqpoint{0.031064in}{-0.018041in}}{\pgfqpoint{0.034722in}{-0.009208in}}{\pgfqpoint{0.034722in}{0.000000in}}%
\pgfpathcurveto{\pgfqpoint{0.034722in}{0.009208in}}{\pgfqpoint{0.031064in}{0.018041in}}{\pgfqpoint{0.024552in}{0.024552in}}%
\pgfpathcurveto{\pgfqpoint{0.018041in}{0.031064in}}{\pgfqpoint{0.009208in}{0.034722in}}{\pgfqpoint{0.000000in}{0.034722in}}%
\pgfpathcurveto{\pgfqpoint{-0.009208in}{0.034722in}}{\pgfqpoint{-0.018041in}{0.031064in}}{\pgfqpoint{-0.024552in}{0.024552in}}%
\pgfpathcurveto{\pgfqpoint{-0.031064in}{0.018041in}}{\pgfqpoint{-0.034722in}{0.009208in}}{\pgfqpoint{-0.034722in}{0.000000in}}%
\pgfpathcurveto{\pgfqpoint{-0.034722in}{-0.009208in}}{\pgfqpoint{-0.031064in}{-0.018041in}}{\pgfqpoint{-0.024552in}{-0.024552in}}%
\pgfpathcurveto{\pgfqpoint{-0.018041in}{-0.031064in}}{\pgfqpoint{-0.009208in}{-0.034722in}}{\pgfqpoint{0.000000in}{-0.034722in}}%
\pgfusepath{stroke}%
\end{pgfscope}%
\begin{pgfscope}%
\pgfpathrectangle{\pgfqpoint{0.688192in}{9.576569in}}{\pgfqpoint{11.096108in}{4.223431in}}%
\pgfusepath{clip}%
\pgfsetbuttcap%
\pgfsetroundjoin%
\definecolor{currentfill}{rgb}{0.675311,0.573788,0.553126}%
\pgfsetfillcolor{currentfill}%
\pgfsetlinewidth{0.752812pt}%
\definecolor{currentstroke}{rgb}{0.240000,0.240000,0.240000}%
\pgfsetstrokecolor{currentstroke}%
\pgfsetdash{}{0pt}%
\pgfpathmoveto{\pgfqpoint{6.680090in}{9.576569in}}%
\pgfpathlineto{\pgfqpoint{6.902012in}{9.576569in}}%
\pgfpathlineto{\pgfqpoint{6.902012in}{9.576569in}}%
\pgfpathlineto{\pgfqpoint{6.680090in}{9.576569in}}%
\pgfpathlineto{\pgfqpoint{6.680090in}{9.576569in}}%
\pgfpathclose%
\pgfusepath{stroke,fill}%
\end{pgfscope}%
\begin{pgfscope}%
\pgfpathrectangle{\pgfqpoint{0.688192in}{9.576569in}}{\pgfqpoint{11.096108in}{4.223431in}}%
\pgfusepath{clip}%
\pgfsetbuttcap%
\pgfsetroundjoin%
\definecolor{currentfill}{rgb}{0.604646,0.477521,0.451635}%
\pgfsetfillcolor{currentfill}%
\pgfsetlinewidth{0.752812pt}%
\definecolor{currentstroke}{rgb}{0.240000,0.240000,0.240000}%
\pgfsetstrokecolor{currentstroke}%
\pgfsetdash{}{0pt}%
\pgfpathmoveto{\pgfqpoint{6.569129in}{9.576569in}}%
\pgfpathlineto{\pgfqpoint{7.012973in}{9.576569in}}%
\pgfpathlineto{\pgfqpoint{7.012973in}{9.576569in}}%
\pgfpathlineto{\pgfqpoint{6.569129in}{9.576569in}}%
\pgfpathlineto{\pgfqpoint{6.569129in}{9.576569in}}%
\pgfpathclose%
\pgfusepath{stroke,fill}%
\end{pgfscope}%
\begin{pgfscope}%
\pgfpathrectangle{\pgfqpoint{0.688192in}{9.576569in}}{\pgfqpoint{11.096108in}{4.223431in}}%
\pgfusepath{clip}%
\pgfsetbuttcap%
\pgfsetroundjoin%
\definecolor{currentfill}{rgb}{0.517157,0.358333,0.325980}%
\pgfsetfillcolor{currentfill}%
\pgfsetlinewidth{0.752812pt}%
\definecolor{currentstroke}{rgb}{0.240000,0.240000,0.240000}%
\pgfsetstrokecolor{currentstroke}%
\pgfsetdash{}{0pt}%
\pgfpathmoveto{\pgfqpoint{6.347207in}{9.576569in}}%
\pgfpathlineto{\pgfqpoint{7.234895in}{9.576569in}}%
\pgfpathlineto{\pgfqpoint{7.234895in}{9.576569in}}%
\pgfpathlineto{\pgfqpoint{6.347207in}{9.576569in}}%
\pgfpathlineto{\pgfqpoint{6.347207in}{9.576569in}}%
\pgfpathclose%
\pgfusepath{stroke,fill}%
\end{pgfscope}%
\begin{pgfscope}%
\pgfpathrectangle{\pgfqpoint{0.688192in}{9.576569in}}{\pgfqpoint{11.096108in}{4.223431in}}%
\pgfusepath{clip}%
\pgfsetbuttcap%
\pgfsetroundjoin%
\definecolor{currentfill}{rgb}{0.604646,0.477521,0.451635}%
\pgfsetfillcolor{currentfill}%
\pgfsetlinewidth{0.752812pt}%
\definecolor{currentstroke}{rgb}{0.240000,0.240000,0.240000}%
\pgfsetstrokecolor{currentstroke}%
\pgfsetdash{}{0pt}%
\pgfpathmoveto{\pgfqpoint{6.569129in}{9.576569in}}%
\pgfpathlineto{\pgfqpoint{7.012973in}{9.576569in}}%
\pgfpathlineto{\pgfqpoint{7.012973in}{9.576569in}}%
\pgfpathlineto{\pgfqpoint{6.569129in}{9.576569in}}%
\pgfpathlineto{\pgfqpoint{6.569129in}{9.576569in}}%
\pgfpathclose%
\pgfusepath{stroke,fill}%
\end{pgfscope}%
\begin{pgfscope}%
\pgfpathrectangle{\pgfqpoint{0.688192in}{9.576569in}}{\pgfqpoint{11.096108in}{4.223431in}}%
\pgfusepath{clip}%
\pgfsetbuttcap%
\pgfsetroundjoin%
\definecolor{currentfill}{rgb}{0.675311,0.573788,0.553126}%
\pgfsetfillcolor{currentfill}%
\pgfsetlinewidth{0.752812pt}%
\definecolor{currentstroke}{rgb}{0.240000,0.240000,0.240000}%
\pgfsetstrokecolor{currentstroke}%
\pgfsetdash{}{0pt}%
\pgfpathmoveto{\pgfqpoint{6.680090in}{9.576569in}}%
\pgfpathlineto{\pgfqpoint{6.902012in}{9.576569in}}%
\pgfpathlineto{\pgfqpoint{6.902012in}{9.576569in}}%
\pgfpathlineto{\pgfqpoint{6.680090in}{9.576569in}}%
\pgfpathlineto{\pgfqpoint{6.680090in}{9.576569in}}%
\pgfpathclose%
\pgfusepath{stroke,fill}%
\end{pgfscope}%
\begin{pgfscope}%
\pgfpathrectangle{\pgfqpoint{0.688192in}{9.576569in}}{\pgfqpoint{11.096108in}{4.223431in}}%
\pgfusepath{clip}%
\pgfsetbuttcap%
\pgfsetroundjoin%
\pgfsetlinewidth{1.003750pt}%
\definecolor{currentstroke}{rgb}{0.450000,0.450000,0.450000}%
\pgfsetstrokecolor{currentstroke}%
\pgfsetdash{}{0pt}%
\pgfpathmoveto{\pgfqpoint{0.000000in}{-0.034722in}}%
\pgfpathcurveto{\pgfqpoint{0.009208in}{-0.034722in}}{\pgfqpoint{0.018041in}{-0.031064in}}{\pgfqpoint{0.024552in}{-0.024552in}}%
\pgfpathcurveto{\pgfqpoint{0.031064in}{-0.018041in}}{\pgfqpoint{0.034722in}{-0.009208in}}{\pgfqpoint{0.034722in}{0.000000in}}%
\pgfpathcurveto{\pgfqpoint{0.034722in}{0.009208in}}{\pgfqpoint{0.031064in}{0.018041in}}{\pgfqpoint{0.024552in}{0.024552in}}%
\pgfpathcurveto{\pgfqpoint{0.018041in}{0.031064in}}{\pgfqpoint{0.009208in}{0.034722in}}{\pgfqpoint{0.000000in}{0.034722in}}%
\pgfpathcurveto{\pgfqpoint{-0.009208in}{0.034722in}}{\pgfqpoint{-0.018041in}{0.031064in}}{\pgfqpoint{-0.024552in}{0.024552in}}%
\pgfpathcurveto{\pgfqpoint{-0.031064in}{0.018041in}}{\pgfqpoint{-0.034722in}{0.009208in}}{\pgfqpoint{-0.034722in}{0.000000in}}%
\pgfpathcurveto{\pgfqpoint{-0.034722in}{-0.009208in}}{\pgfqpoint{-0.031064in}{-0.018041in}}{\pgfqpoint{-0.024552in}{-0.024552in}}%
\pgfpathcurveto{\pgfqpoint{-0.018041in}{-0.031064in}}{\pgfqpoint{-0.009208in}{-0.034722in}}{\pgfqpoint{0.000000in}{-0.034722in}}%
\pgfusepath{stroke}%
\end{pgfscope}%
\begin{pgfscope}%
\pgfpathrectangle{\pgfqpoint{0.688192in}{9.576569in}}{\pgfqpoint{11.096108in}{4.223431in}}%
\pgfusepath{clip}%
\pgfsetbuttcap%
\pgfsetroundjoin%
\definecolor{currentfill}{rgb}{0.878118,0.675269,0.815953}%
\pgfsetfillcolor{currentfill}%
\pgfsetlinewidth{0.752812pt}%
\definecolor{currentstroke}{rgb}{0.240000,0.240000,0.240000}%
\pgfsetstrokecolor{currentstroke}%
\pgfsetdash{}{0pt}%
\pgfpathmoveto{\pgfqpoint{7.789701in}{9.576569in}}%
\pgfpathlineto{\pgfqpoint{8.011623in}{9.576569in}}%
\pgfpathlineto{\pgfqpoint{8.011623in}{9.576569in}}%
\pgfpathlineto{\pgfqpoint{7.789701in}{9.576569in}}%
\pgfpathlineto{\pgfqpoint{7.789701in}{9.576569in}}%
\pgfpathclose%
\pgfusepath{stroke,fill}%
\end{pgfscope}%
\begin{pgfscope}%
\pgfpathrectangle{\pgfqpoint{0.688192in}{9.576569in}}{\pgfqpoint{11.096108in}{4.223431in}}%
\pgfusepath{clip}%
\pgfsetbuttcap%
\pgfsetroundjoin%
\definecolor{currentfill}{rgb}{0.859860,0.605718,0.782104}%
\pgfsetfillcolor{currentfill}%
\pgfsetlinewidth{0.752812pt}%
\definecolor{currentstroke}{rgb}{0.240000,0.240000,0.240000}%
\pgfsetstrokecolor{currentstroke}%
\pgfsetdash{}{0pt}%
\pgfpathmoveto{\pgfqpoint{7.678740in}{9.576569in}}%
\pgfpathlineto{\pgfqpoint{8.122584in}{9.576569in}}%
\pgfpathlineto{\pgfqpoint{8.122584in}{9.576569in}}%
\pgfpathlineto{\pgfqpoint{7.678740in}{9.576569in}}%
\pgfpathlineto{\pgfqpoint{7.678740in}{9.576569in}}%
\pgfpathclose%
\pgfusepath{stroke,fill}%
\end{pgfscope}%
\begin{pgfscope}%
\pgfpathrectangle{\pgfqpoint{0.688192in}{9.576569in}}{\pgfqpoint{11.096108in}{4.223431in}}%
\pgfusepath{clip}%
\pgfsetbuttcap%
\pgfsetroundjoin%
\definecolor{currentfill}{rgb}{0.837255,0.519608,0.740196}%
\pgfsetfillcolor{currentfill}%
\pgfsetlinewidth{0.752812pt}%
\definecolor{currentstroke}{rgb}{0.240000,0.240000,0.240000}%
\pgfsetstrokecolor{currentstroke}%
\pgfsetdash{}{0pt}%
\pgfpathmoveto{\pgfqpoint{7.456817in}{9.576569in}}%
\pgfpathlineto{\pgfqpoint{8.344506in}{9.576569in}}%
\pgfpathlineto{\pgfqpoint{8.344506in}{9.576569in}}%
\pgfpathlineto{\pgfqpoint{7.456817in}{9.576569in}}%
\pgfpathlineto{\pgfqpoint{7.456817in}{9.576569in}}%
\pgfpathclose%
\pgfusepath{stroke,fill}%
\end{pgfscope}%
\begin{pgfscope}%
\pgfpathrectangle{\pgfqpoint{0.688192in}{9.576569in}}{\pgfqpoint{11.096108in}{4.223431in}}%
\pgfusepath{clip}%
\pgfsetbuttcap%
\pgfsetroundjoin%
\definecolor{currentfill}{rgb}{0.859860,0.605718,0.782104}%
\pgfsetfillcolor{currentfill}%
\pgfsetlinewidth{0.752812pt}%
\definecolor{currentstroke}{rgb}{0.240000,0.240000,0.240000}%
\pgfsetstrokecolor{currentstroke}%
\pgfsetdash{}{0pt}%
\pgfpathmoveto{\pgfqpoint{7.678740in}{9.576569in}}%
\pgfpathlineto{\pgfqpoint{8.122584in}{9.576569in}}%
\pgfpathlineto{\pgfqpoint{8.122584in}{9.576569in}}%
\pgfpathlineto{\pgfqpoint{7.678740in}{9.576569in}}%
\pgfpathlineto{\pgfqpoint{7.678740in}{9.576569in}}%
\pgfpathclose%
\pgfusepath{stroke,fill}%
\end{pgfscope}%
\begin{pgfscope}%
\pgfpathrectangle{\pgfqpoint{0.688192in}{9.576569in}}{\pgfqpoint{11.096108in}{4.223431in}}%
\pgfusepath{clip}%
\pgfsetbuttcap%
\pgfsetroundjoin%
\definecolor{currentfill}{rgb}{0.878118,0.675269,0.815953}%
\pgfsetfillcolor{currentfill}%
\pgfsetlinewidth{0.752812pt}%
\definecolor{currentstroke}{rgb}{0.240000,0.240000,0.240000}%
\pgfsetstrokecolor{currentstroke}%
\pgfsetdash{}{0pt}%
\pgfpathmoveto{\pgfqpoint{7.789701in}{9.576569in}}%
\pgfpathlineto{\pgfqpoint{8.011623in}{9.576569in}}%
\pgfpathlineto{\pgfqpoint{8.011623in}{9.576569in}}%
\pgfpathlineto{\pgfqpoint{7.789701in}{9.576569in}}%
\pgfpathlineto{\pgfqpoint{7.789701in}{9.576569in}}%
\pgfpathclose%
\pgfusepath{stroke,fill}%
\end{pgfscope}%
\begin{pgfscope}%
\pgfpathrectangle{\pgfqpoint{0.688192in}{9.576569in}}{\pgfqpoint{11.096108in}{4.223431in}}%
\pgfusepath{clip}%
\pgfsetbuttcap%
\pgfsetroundjoin%
\pgfsetlinewidth{1.003750pt}%
\definecolor{currentstroke}{rgb}{0.450000,0.450000,0.450000}%
\pgfsetstrokecolor{currentstroke}%
\pgfsetdash{}{0pt}%
\pgfpathmoveto{\pgfqpoint{0.000000in}{-0.034722in}}%
\pgfpathcurveto{\pgfqpoint{0.009208in}{-0.034722in}}{\pgfqpoint{0.018041in}{-0.031064in}}{\pgfqpoint{0.024552in}{-0.024552in}}%
\pgfpathcurveto{\pgfqpoint{0.031064in}{-0.018041in}}{\pgfqpoint{0.034722in}{-0.009208in}}{\pgfqpoint{0.034722in}{0.000000in}}%
\pgfpathcurveto{\pgfqpoint{0.034722in}{0.009208in}}{\pgfqpoint{0.031064in}{0.018041in}}{\pgfqpoint{0.024552in}{0.024552in}}%
\pgfpathcurveto{\pgfqpoint{0.018041in}{0.031064in}}{\pgfqpoint{0.009208in}{0.034722in}}{\pgfqpoint{0.000000in}{0.034722in}}%
\pgfpathcurveto{\pgfqpoint{-0.009208in}{0.034722in}}{\pgfqpoint{-0.018041in}{0.031064in}}{\pgfqpoint{-0.024552in}{0.024552in}}%
\pgfpathcurveto{\pgfqpoint{-0.031064in}{0.018041in}}{\pgfqpoint{-0.034722in}{0.009208in}}{\pgfqpoint{-0.034722in}{0.000000in}}%
\pgfpathcurveto{\pgfqpoint{-0.034722in}{-0.009208in}}{\pgfqpoint{-0.031064in}{-0.018041in}}{\pgfqpoint{-0.024552in}{-0.024552in}}%
\pgfpathcurveto{\pgfqpoint{-0.018041in}{-0.031064in}}{\pgfqpoint{-0.009208in}{-0.034722in}}{\pgfqpoint{0.000000in}{-0.034722in}}%
\pgfusepath{stroke}%
\end{pgfscope}%
\begin{pgfscope}%
\pgfpathrectangle{\pgfqpoint{0.688192in}{9.576569in}}{\pgfqpoint{11.096108in}{4.223431in}}%
\pgfusepath{clip}%
\pgfsetbuttcap%
\pgfsetroundjoin%
\definecolor{currentfill}{rgb}{0.662237,0.662248,0.662203}%
\pgfsetfillcolor{currentfill}%
\pgfsetlinewidth{0.752812pt}%
\definecolor{currentstroke}{rgb}{0.240000,0.240000,0.240000}%
\pgfsetstrokecolor{currentstroke}%
\pgfsetdash{}{0pt}%
\pgfpathmoveto{\pgfqpoint{8.899311in}{10.354612in}}%
\pgfpathlineto{\pgfqpoint{9.121234in}{10.354612in}}%
\pgfpathlineto{\pgfqpoint{9.121234in}{10.364189in}}%
\pgfpathlineto{\pgfqpoint{8.899311in}{10.364189in}}%
\pgfpathlineto{\pgfqpoint{8.899311in}{10.354612in}}%
\pgfpathclose%
\pgfusepath{stroke,fill}%
\end{pgfscope}%
\begin{pgfscope}%
\pgfpathrectangle{\pgfqpoint{0.688192in}{9.576569in}}{\pgfqpoint{11.096108in}{4.223431in}}%
\pgfusepath{clip}%
\pgfsetbuttcap%
\pgfsetroundjoin%
\definecolor{currentfill}{rgb}{0.588872,0.588878,0.588853}%
\pgfsetfillcolor{currentfill}%
\pgfsetlinewidth{0.752812pt}%
\definecolor{currentstroke}{rgb}{0.240000,0.240000,0.240000}%
\pgfsetstrokecolor{currentstroke}%
\pgfsetdash{}{0pt}%
\pgfpathmoveto{\pgfqpoint{8.788350in}{10.364189in}}%
\pgfpathlineto{\pgfqpoint{9.232195in}{10.364189in}}%
\pgfpathlineto{\pgfqpoint{9.232195in}{10.373368in}}%
\pgfpathlineto{\pgfqpoint{8.788350in}{10.373368in}}%
\pgfpathlineto{\pgfqpoint{8.788350in}{10.364189in}}%
\pgfpathclose%
\pgfusepath{stroke,fill}%
\end{pgfscope}%
\begin{pgfscope}%
\pgfpathrectangle{\pgfqpoint{0.688192in}{9.576569in}}{\pgfqpoint{11.096108in}{4.223431in}}%
\pgfusepath{clip}%
\pgfsetbuttcap%
\pgfsetroundjoin%
\definecolor{currentfill}{rgb}{0.498039,0.498039,0.498039}%
\pgfsetfillcolor{currentfill}%
\pgfsetlinewidth{0.752812pt}%
\definecolor{currentstroke}{rgb}{0.240000,0.240000,0.240000}%
\pgfsetstrokecolor{currentstroke}%
\pgfsetdash{}{0pt}%
\pgfpathmoveto{\pgfqpoint{8.566428in}{10.373368in}}%
\pgfpathlineto{\pgfqpoint{9.454117in}{10.373368in}}%
\pgfpathlineto{\pgfqpoint{9.454117in}{10.557432in}}%
\pgfpathlineto{\pgfqpoint{8.566428in}{10.557432in}}%
\pgfpathlineto{\pgfqpoint{8.566428in}{10.373368in}}%
\pgfpathclose%
\pgfusepath{stroke,fill}%
\end{pgfscope}%
\begin{pgfscope}%
\pgfpathrectangle{\pgfqpoint{0.688192in}{9.576569in}}{\pgfqpoint{11.096108in}{4.223431in}}%
\pgfusepath{clip}%
\pgfsetbuttcap%
\pgfsetroundjoin%
\definecolor{currentfill}{rgb}{0.588872,0.588878,0.588853}%
\pgfsetfillcolor{currentfill}%
\pgfsetlinewidth{0.752812pt}%
\definecolor{currentstroke}{rgb}{0.240000,0.240000,0.240000}%
\pgfsetstrokecolor{currentstroke}%
\pgfsetdash{}{0pt}%
\pgfpathmoveto{\pgfqpoint{8.788350in}{10.557432in}}%
\pgfpathlineto{\pgfqpoint{9.232195in}{10.557432in}}%
\pgfpathlineto{\pgfqpoint{9.232195in}{10.597577in}}%
\pgfpathlineto{\pgfqpoint{8.788350in}{10.597577in}}%
\pgfpathlineto{\pgfqpoint{8.788350in}{10.557432in}}%
\pgfpathclose%
\pgfusepath{stroke,fill}%
\end{pgfscope}%
\begin{pgfscope}%
\pgfpathrectangle{\pgfqpoint{0.688192in}{9.576569in}}{\pgfqpoint{11.096108in}{4.223431in}}%
\pgfusepath{clip}%
\pgfsetbuttcap%
\pgfsetroundjoin%
\definecolor{currentfill}{rgb}{0.662237,0.662248,0.662203}%
\pgfsetfillcolor{currentfill}%
\pgfsetlinewidth{0.752812pt}%
\definecolor{currentstroke}{rgb}{0.240000,0.240000,0.240000}%
\pgfsetstrokecolor{currentstroke}%
\pgfsetdash{}{0pt}%
\pgfpathmoveto{\pgfqpoint{8.899311in}{10.597577in}}%
\pgfpathlineto{\pgfqpoint{9.121234in}{10.597577in}}%
\pgfpathlineto{\pgfqpoint{9.121234in}{10.669931in}}%
\pgfpathlineto{\pgfqpoint{8.899311in}{10.669931in}}%
\pgfpathlineto{\pgfqpoint{8.899311in}{10.597577in}}%
\pgfpathclose%
\pgfusepath{stroke,fill}%
\end{pgfscope}%
\begin{pgfscope}%
\pgfpathrectangle{\pgfqpoint{0.688192in}{9.576569in}}{\pgfqpoint{11.096108in}{4.223431in}}%
\pgfusepath{clip}%
\pgfsetbuttcap%
\pgfsetroundjoin%
\pgfsetlinewidth{1.003750pt}%
\definecolor{currentstroke}{rgb}{0.450000,0.450000,0.450000}%
\pgfsetstrokecolor{currentstroke}%
\pgfsetdash{}{0pt}%
\pgfpathmoveto{\pgfqpoint{9.010272in}{10.171484in}}%
\pgfpathcurveto{\pgfqpoint{9.019481in}{10.171484in}}{\pgfqpoint{9.028313in}{10.175142in}}{\pgfqpoint{9.034825in}{10.181654in}}%
\pgfpathcurveto{\pgfqpoint{9.041336in}{10.188165in}}{\pgfqpoint{9.044995in}{10.196998in}}{\pgfqpoint{9.044995in}{10.206206in}}%
\pgfpathcurveto{\pgfqpoint{9.044995in}{10.215415in}}{\pgfqpoint{9.041336in}{10.224247in}}{\pgfqpoint{9.034825in}{10.230758in}}%
\pgfpathcurveto{\pgfqpoint{9.028313in}{10.237270in}}{\pgfqpoint{9.019481in}{10.240928in}}{\pgfqpoint{9.010272in}{10.240928in}}%
\pgfpathcurveto{\pgfqpoint{9.001064in}{10.240928in}}{\pgfqpoint{8.992232in}{10.237270in}}{\pgfqpoint{8.985720in}{10.230758in}}%
\pgfpathcurveto{\pgfqpoint{8.979209in}{10.224247in}}{\pgfqpoint{8.975550in}{10.215415in}}{\pgfqpoint{8.975550in}{10.206206in}}%
\pgfpathcurveto{\pgfqpoint{8.975550in}{10.196998in}}{\pgfqpoint{8.979209in}{10.188165in}}{\pgfqpoint{8.985720in}{10.181654in}}%
\pgfpathcurveto{\pgfqpoint{8.992232in}{10.175142in}}{\pgfqpoint{9.001064in}{10.171484in}}{\pgfqpoint{9.010272in}{10.171484in}}%
\pgfpathlineto{\pgfqpoint{9.010272in}{10.171484in}}%
\pgfpathclose%
\pgfusepath{stroke}%
\end{pgfscope}%
\begin{pgfscope}%
\pgfpathrectangle{\pgfqpoint{0.688192in}{9.576569in}}{\pgfqpoint{11.096108in}{4.223431in}}%
\pgfusepath{clip}%
\pgfsetbuttcap%
\pgfsetroundjoin%
\pgfsetlinewidth{1.003750pt}%
\definecolor{currentstroke}{rgb}{0.450000,0.450000,0.450000}%
\pgfsetstrokecolor{currentstroke}%
\pgfsetdash{}{0pt}%
\pgfpathmoveto{\pgfqpoint{9.010272in}{10.657960in}}%
\pgfpathcurveto{\pgfqpoint{9.019481in}{10.657960in}}{\pgfqpoint{9.028313in}{10.661618in}}{\pgfqpoint{9.034825in}{10.668130in}}%
\pgfpathcurveto{\pgfqpoint{9.041336in}{10.674641in}}{\pgfqpoint{9.044995in}{10.683473in}}{\pgfqpoint{9.044995in}{10.692682in}}%
\pgfpathcurveto{\pgfqpoint{9.044995in}{10.701890in}}{\pgfqpoint{9.041336in}{10.710723in}}{\pgfqpoint{9.034825in}{10.717234in}}%
\pgfpathcurveto{\pgfqpoint{9.028313in}{10.723746in}}{\pgfqpoint{9.019481in}{10.727404in}}{\pgfqpoint{9.010272in}{10.727404in}}%
\pgfpathcurveto{\pgfqpoint{9.001064in}{10.727404in}}{\pgfqpoint{8.992232in}{10.723746in}}{\pgfqpoint{8.985720in}{10.717234in}}%
\pgfpathcurveto{\pgfqpoint{8.979209in}{10.710723in}}{\pgfqpoint{8.975550in}{10.701890in}}{\pgfqpoint{8.975550in}{10.692682in}}%
\pgfpathcurveto{\pgfqpoint{8.975550in}{10.683473in}}{\pgfqpoint{8.979209in}{10.674641in}}{\pgfqpoint{8.985720in}{10.668130in}}%
\pgfpathcurveto{\pgfqpoint{8.992232in}{10.661618in}}{\pgfqpoint{9.001064in}{10.657960in}}{\pgfqpoint{9.010272in}{10.657960in}}%
\pgfpathlineto{\pgfqpoint{9.010272in}{10.657960in}}%
\pgfpathclose%
\pgfusepath{stroke}%
\end{pgfscope}%
\begin{pgfscope}%
\pgfpathrectangle{\pgfqpoint{0.688192in}{9.576569in}}{\pgfqpoint{11.096108in}{4.223431in}}%
\pgfusepath{clip}%
\pgfsetbuttcap%
\pgfsetroundjoin%
\pgfsetlinewidth{1.003750pt}%
\definecolor{currentstroke}{rgb}{0.450000,0.450000,0.450000}%
\pgfsetstrokecolor{currentstroke}%
\pgfsetdash{}{0pt}%
\pgfpathmoveto{\pgfqpoint{9.010272in}{10.776610in}}%
\pgfpathcurveto{\pgfqpoint{9.019481in}{10.776610in}}{\pgfqpoint{9.028313in}{10.780268in}}{\pgfqpoint{9.034825in}{10.786780in}}%
\pgfpathcurveto{\pgfqpoint{9.041336in}{10.793291in}}{\pgfqpoint{9.044995in}{10.802124in}}{\pgfqpoint{9.044995in}{10.811332in}}%
\pgfpathcurveto{\pgfqpoint{9.044995in}{10.820540in}}{\pgfqpoint{9.041336in}{10.829373in}}{\pgfqpoint{9.034825in}{10.835884in}}%
\pgfpathcurveto{\pgfqpoint{9.028313in}{10.842396in}}{\pgfqpoint{9.019481in}{10.846054in}}{\pgfqpoint{9.010272in}{10.846054in}}%
\pgfpathcurveto{\pgfqpoint{9.001064in}{10.846054in}}{\pgfqpoint{8.992232in}{10.842396in}}{\pgfqpoint{8.985720in}{10.835884in}}%
\pgfpathcurveto{\pgfqpoint{8.979209in}{10.829373in}}{\pgfqpoint{8.975550in}{10.820540in}}{\pgfqpoint{8.975550in}{10.811332in}}%
\pgfpathcurveto{\pgfqpoint{8.975550in}{10.802124in}}{\pgfqpoint{8.979209in}{10.793291in}}{\pgfqpoint{8.985720in}{10.786780in}}%
\pgfpathcurveto{\pgfqpoint{8.992232in}{10.780268in}}{\pgfqpoint{9.001064in}{10.776610in}}{\pgfqpoint{9.010272in}{10.776610in}}%
\pgfpathlineto{\pgfqpoint{9.010272in}{10.776610in}}%
\pgfpathclose%
\pgfusepath{stroke}%
\end{pgfscope}%
\begin{pgfscope}%
\pgfpathrectangle{\pgfqpoint{0.688192in}{9.576569in}}{\pgfqpoint{11.096108in}{4.223431in}}%
\pgfusepath{clip}%
\pgfsetbuttcap%
\pgfsetroundjoin%
\pgfsetlinewidth{1.003750pt}%
\definecolor{currentstroke}{rgb}{0.450000,0.450000,0.450000}%
\pgfsetstrokecolor{currentstroke}%
\pgfsetdash{}{0pt}%
\pgfpathmoveto{\pgfqpoint{9.010272in}{10.264262in}}%
\pgfpathcurveto{\pgfqpoint{9.019481in}{10.264262in}}{\pgfqpoint{9.028313in}{10.267920in}}{\pgfqpoint{9.034825in}{10.274432in}}%
\pgfpathcurveto{\pgfqpoint{9.041336in}{10.280943in}}{\pgfqpoint{9.044995in}{10.289776in}}{\pgfqpoint{9.044995in}{10.298984in}}%
\pgfpathcurveto{\pgfqpoint{9.044995in}{10.308192in}}{\pgfqpoint{9.041336in}{10.317025in}}{\pgfqpoint{9.034825in}{10.323536in}}%
\pgfpathcurveto{\pgfqpoint{9.028313in}{10.330048in}}{\pgfqpoint{9.019481in}{10.333706in}}{\pgfqpoint{9.010272in}{10.333706in}}%
\pgfpathcurveto{\pgfqpoint{9.001064in}{10.333706in}}{\pgfqpoint{8.992232in}{10.330048in}}{\pgfqpoint{8.985720in}{10.323536in}}%
\pgfpathcurveto{\pgfqpoint{8.979209in}{10.317025in}}{\pgfqpoint{8.975550in}{10.308192in}}{\pgfqpoint{8.975550in}{10.298984in}}%
\pgfpathcurveto{\pgfqpoint{8.975550in}{10.289776in}}{\pgfqpoint{8.979209in}{10.280943in}}{\pgfqpoint{8.985720in}{10.274432in}}%
\pgfpathcurveto{\pgfqpoint{8.992232in}{10.267920in}}{\pgfqpoint{9.001064in}{10.264262in}}{\pgfqpoint{9.010272in}{10.264262in}}%
\pgfpathlineto{\pgfqpoint{9.010272in}{10.264262in}}%
\pgfpathclose%
\pgfusepath{stroke}%
\end{pgfscope}%
\begin{pgfscope}%
\pgfpathrectangle{\pgfqpoint{0.688192in}{9.576569in}}{\pgfqpoint{11.096108in}{4.223431in}}%
\pgfusepath{clip}%
\pgfsetbuttcap%
\pgfsetroundjoin%
\pgfsetlinewidth{1.003750pt}%
\definecolor{currentstroke}{rgb}{0.450000,0.450000,0.450000}%
\pgfsetstrokecolor{currentstroke}%
\pgfsetdash{}{0pt}%
\pgfpathmoveto{\pgfqpoint{9.010272in}{10.157659in}}%
\pgfpathcurveto{\pgfqpoint{9.019481in}{10.157659in}}{\pgfqpoint{9.028313in}{10.161318in}}{\pgfqpoint{9.034825in}{10.167829in}}%
\pgfpathcurveto{\pgfqpoint{9.041336in}{10.174340in}}{\pgfqpoint{9.044995in}{10.183173in}}{\pgfqpoint{9.044995in}{10.192381in}}%
\pgfpathcurveto{\pgfqpoint{9.044995in}{10.201590in}}{\pgfqpoint{9.041336in}{10.210422in}}{\pgfqpoint{9.034825in}{10.216934in}}%
\pgfpathcurveto{\pgfqpoint{9.028313in}{10.223445in}}{\pgfqpoint{9.019481in}{10.227104in}}{\pgfqpoint{9.010272in}{10.227104in}}%
\pgfpathcurveto{\pgfqpoint{9.001064in}{10.227104in}}{\pgfqpoint{8.992232in}{10.223445in}}{\pgfqpoint{8.985720in}{10.216934in}}%
\pgfpathcurveto{\pgfqpoint{8.979209in}{10.210422in}}{\pgfqpoint{8.975550in}{10.201590in}}{\pgfqpoint{8.975550in}{10.192381in}}%
\pgfpathcurveto{\pgfqpoint{8.975550in}{10.183173in}}{\pgfqpoint{8.979209in}{10.174340in}}{\pgfqpoint{8.985720in}{10.167829in}}%
\pgfpathcurveto{\pgfqpoint{8.992232in}{10.161318in}}{\pgfqpoint{9.001064in}{10.157659in}}{\pgfqpoint{9.010272in}{10.157659in}}%
\pgfpathlineto{\pgfqpoint{9.010272in}{10.157659in}}%
\pgfpathclose%
\pgfusepath{stroke}%
\end{pgfscope}%
\begin{pgfscope}%
\pgfpathrectangle{\pgfqpoint{0.688192in}{9.576569in}}{\pgfqpoint{11.096108in}{4.223431in}}%
\pgfusepath{clip}%
\pgfsetbuttcap%
\pgfsetroundjoin%
\pgfsetlinewidth{1.003750pt}%
\definecolor{currentstroke}{rgb}{0.450000,0.450000,0.450000}%
\pgfsetstrokecolor{currentstroke}%
\pgfsetdash{}{0pt}%
\pgfpathmoveto{\pgfqpoint{9.010272in}{10.719046in}}%
\pgfpathcurveto{\pgfqpoint{9.019481in}{10.719046in}}{\pgfqpoint{9.028313in}{10.722705in}}{\pgfqpoint{9.034825in}{10.729216in}}%
\pgfpathcurveto{\pgfqpoint{9.041336in}{10.735727in}}{\pgfqpoint{9.044995in}{10.744560in}}{\pgfqpoint{9.044995in}{10.753768in}}%
\pgfpathcurveto{\pgfqpoint{9.044995in}{10.762977in}}{\pgfqpoint{9.041336in}{10.771809in}}{\pgfqpoint{9.034825in}{10.778321in}}%
\pgfpathcurveto{\pgfqpoint{9.028313in}{10.784832in}}{\pgfqpoint{9.019481in}{10.788490in}}{\pgfqpoint{9.010272in}{10.788490in}}%
\pgfpathcurveto{\pgfqpoint{9.001064in}{10.788490in}}{\pgfqpoint{8.992232in}{10.784832in}}{\pgfqpoint{8.985720in}{10.778321in}}%
\pgfpathcurveto{\pgfqpoint{8.979209in}{10.771809in}}{\pgfqpoint{8.975550in}{10.762977in}}{\pgfqpoint{8.975550in}{10.753768in}}%
\pgfpathcurveto{\pgfqpoint{8.975550in}{10.744560in}}{\pgfqpoint{8.979209in}{10.735727in}}{\pgfqpoint{8.985720in}{10.729216in}}%
\pgfpathcurveto{\pgfqpoint{8.992232in}{10.722705in}}{\pgfqpoint{9.001064in}{10.719046in}}{\pgfqpoint{9.010272in}{10.719046in}}%
\pgfpathlineto{\pgfqpoint{9.010272in}{10.719046in}}%
\pgfpathclose%
\pgfusepath{stroke}%
\end{pgfscope}%
\begin{pgfscope}%
\pgfpathrectangle{\pgfqpoint{0.688192in}{9.576569in}}{\pgfqpoint{11.096108in}{4.223431in}}%
\pgfusepath{clip}%
\pgfsetbuttcap%
\pgfsetroundjoin%
\pgfsetlinewidth{1.003750pt}%
\definecolor{currentstroke}{rgb}{0.450000,0.450000,0.450000}%
\pgfsetstrokecolor{currentstroke}%
\pgfsetdash{}{0pt}%
\pgfpathmoveto{\pgfqpoint{9.010272in}{10.677683in}}%
\pgfpathcurveto{\pgfqpoint{9.019481in}{10.677683in}}{\pgfqpoint{9.028313in}{10.681342in}}{\pgfqpoint{9.034825in}{10.687853in}}%
\pgfpathcurveto{\pgfqpoint{9.041336in}{10.694364in}}{\pgfqpoint{9.044995in}{10.703197in}}{\pgfqpoint{9.044995in}{10.712405in}}%
\pgfpathcurveto{\pgfqpoint{9.044995in}{10.721614in}}{\pgfqpoint{9.041336in}{10.730446in}}{\pgfqpoint{9.034825in}{10.736958in}}%
\pgfpathcurveto{\pgfqpoint{9.028313in}{10.743469in}}{\pgfqpoint{9.019481in}{10.747128in}}{\pgfqpoint{9.010272in}{10.747128in}}%
\pgfpathcurveto{\pgfqpoint{9.001064in}{10.747128in}}{\pgfqpoint{8.992232in}{10.743469in}}{\pgfqpoint{8.985720in}{10.736958in}}%
\pgfpathcurveto{\pgfqpoint{8.979209in}{10.730446in}}{\pgfqpoint{8.975550in}{10.721614in}}{\pgfqpoint{8.975550in}{10.712405in}}%
\pgfpathcurveto{\pgfqpoint{8.975550in}{10.703197in}}{\pgfqpoint{8.979209in}{10.694364in}}{\pgfqpoint{8.985720in}{10.687853in}}%
\pgfpathcurveto{\pgfqpoint{8.992232in}{10.681342in}}{\pgfqpoint{9.001064in}{10.677683in}}{\pgfqpoint{9.010272in}{10.677683in}}%
\pgfpathlineto{\pgfqpoint{9.010272in}{10.677683in}}%
\pgfpathclose%
\pgfusepath{stroke}%
\end{pgfscope}%
\begin{pgfscope}%
\pgfpathrectangle{\pgfqpoint{0.688192in}{9.576569in}}{\pgfqpoint{11.096108in}{4.223431in}}%
\pgfusepath{clip}%
\pgfsetbuttcap%
\pgfsetroundjoin%
\pgfsetlinewidth{1.003750pt}%
\definecolor{currentstroke}{rgb}{0.450000,0.450000,0.450000}%
\pgfsetstrokecolor{currentstroke}%
\pgfsetdash{}{0pt}%
\pgfpathmoveto{\pgfqpoint{9.010272in}{10.314947in}}%
\pgfpathcurveto{\pgfqpoint{9.019481in}{10.314947in}}{\pgfqpoint{9.028313in}{10.318606in}}{\pgfqpoint{9.034825in}{10.325117in}}%
\pgfpathcurveto{\pgfqpoint{9.041336in}{10.331629in}}{\pgfqpoint{9.044995in}{10.340461in}}{\pgfqpoint{9.044995in}{10.349669in}}%
\pgfpathcurveto{\pgfqpoint{9.044995in}{10.358878in}}{\pgfqpoint{9.041336in}{10.367710in}}{\pgfqpoint{9.034825in}{10.374222in}}%
\pgfpathcurveto{\pgfqpoint{9.028313in}{10.380733in}}{\pgfqpoint{9.019481in}{10.384392in}}{\pgfqpoint{9.010272in}{10.384392in}}%
\pgfpathcurveto{\pgfqpoint{9.001064in}{10.384392in}}{\pgfqpoint{8.992232in}{10.380733in}}{\pgfqpoint{8.985720in}{10.374222in}}%
\pgfpathcurveto{\pgfqpoint{8.979209in}{10.367710in}}{\pgfqpoint{8.975550in}{10.358878in}}{\pgfqpoint{8.975550in}{10.349669in}}%
\pgfpathcurveto{\pgfqpoint{8.975550in}{10.340461in}}{\pgfqpoint{8.979209in}{10.331629in}}{\pgfqpoint{8.985720in}{10.325117in}}%
\pgfpathcurveto{\pgfqpoint{8.992232in}{10.318606in}}{\pgfqpoint{9.001064in}{10.314947in}}{\pgfqpoint{9.010272in}{10.314947in}}%
\pgfpathlineto{\pgfqpoint{9.010272in}{10.314947in}}%
\pgfpathclose%
\pgfusepath{stroke}%
\end{pgfscope}%
\begin{pgfscope}%
\pgfpathrectangle{\pgfqpoint{0.688192in}{9.576569in}}{\pgfqpoint{11.096108in}{4.223431in}}%
\pgfusepath{clip}%
\pgfsetbuttcap%
\pgfsetroundjoin%
\definecolor{currentfill}{rgb}{0.767534,0.769674,0.455219}%
\pgfsetfillcolor{currentfill}%
\pgfsetlinewidth{0.752812pt}%
\definecolor{currentstroke}{rgb}{0.240000,0.240000,0.240000}%
\pgfsetstrokecolor{currentstroke}%
\pgfsetdash{}{0pt}%
\pgfpathmoveto{\pgfqpoint{10.008922in}{9.576569in}}%
\pgfpathlineto{\pgfqpoint{10.230844in}{9.576569in}}%
\pgfpathlineto{\pgfqpoint{10.230844in}{9.576569in}}%
\pgfpathlineto{\pgfqpoint{10.008922in}{9.576569in}}%
\pgfpathlineto{\pgfqpoint{10.008922in}{9.576569in}}%
\pgfpathclose%
\pgfusepath{stroke,fill}%
\end{pgfscope}%
\begin{pgfscope}%
\pgfpathrectangle{\pgfqpoint{0.688192in}{9.576569in}}{\pgfqpoint{11.096108in}{4.223431in}}%
\pgfusepath{clip}%
\pgfsetbuttcap%
\pgfsetroundjoin%
\definecolor{currentfill}{rgb}{0.720495,0.722993,0.345346}%
\pgfsetfillcolor{currentfill}%
\pgfsetlinewidth{0.752812pt}%
\definecolor{currentstroke}{rgb}{0.240000,0.240000,0.240000}%
\pgfsetstrokecolor{currentstroke}%
\pgfsetdash{}{0pt}%
\pgfpathmoveto{\pgfqpoint{9.897961in}{9.576569in}}%
\pgfpathlineto{\pgfqpoint{10.341805in}{9.576569in}}%
\pgfpathlineto{\pgfqpoint{10.341805in}{9.576569in}}%
\pgfpathlineto{\pgfqpoint{9.897961in}{9.576569in}}%
\pgfpathlineto{\pgfqpoint{9.897961in}{9.576569in}}%
\pgfpathclose%
\pgfusepath{stroke,fill}%
\end{pgfscope}%
\begin{pgfscope}%
\pgfpathrectangle{\pgfqpoint{0.688192in}{9.576569in}}{\pgfqpoint{11.096108in}{4.223431in}}%
\pgfusepath{clip}%
\pgfsetbuttcap%
\pgfsetroundjoin%
\definecolor{currentfill}{rgb}{0.662255,0.665196,0.209314}%
\pgfsetfillcolor{currentfill}%
\pgfsetlinewidth{0.752812pt}%
\definecolor{currentstroke}{rgb}{0.240000,0.240000,0.240000}%
\pgfsetstrokecolor{currentstroke}%
\pgfsetdash{}{0pt}%
\pgfpathmoveto{\pgfqpoint{9.676039in}{9.576569in}}%
\pgfpathlineto{\pgfqpoint{10.563728in}{9.576569in}}%
\pgfpathlineto{\pgfqpoint{10.563728in}{11.491650in}}%
\pgfpathlineto{\pgfqpoint{9.676039in}{11.491650in}}%
\pgfpathlineto{\pgfqpoint{9.676039in}{9.576569in}}%
\pgfpathclose%
\pgfusepath{stroke,fill}%
\end{pgfscope}%
\begin{pgfscope}%
\pgfpathrectangle{\pgfqpoint{0.688192in}{9.576569in}}{\pgfqpoint{11.096108in}{4.223431in}}%
\pgfusepath{clip}%
\pgfsetbuttcap%
\pgfsetroundjoin%
\definecolor{currentfill}{rgb}{0.720495,0.722993,0.345346}%
\pgfsetfillcolor{currentfill}%
\pgfsetlinewidth{0.752812pt}%
\definecolor{currentstroke}{rgb}{0.240000,0.240000,0.240000}%
\pgfsetstrokecolor{currentstroke}%
\pgfsetdash{}{0pt}%
\pgfpathmoveto{\pgfqpoint{9.897961in}{11.491650in}}%
\pgfpathlineto{\pgfqpoint{10.341805in}{11.491650in}}%
\pgfpathlineto{\pgfqpoint{10.341805in}{11.943042in}}%
\pgfpathlineto{\pgfqpoint{9.897961in}{11.943042in}}%
\pgfpathlineto{\pgfqpoint{9.897961in}{11.491650in}}%
\pgfpathclose%
\pgfusepath{stroke,fill}%
\end{pgfscope}%
\begin{pgfscope}%
\pgfpathrectangle{\pgfqpoint{0.688192in}{9.576569in}}{\pgfqpoint{11.096108in}{4.223431in}}%
\pgfusepath{clip}%
\pgfsetbuttcap%
\pgfsetroundjoin%
\definecolor{currentfill}{rgb}{0.767534,0.769674,0.455219}%
\pgfsetfillcolor{currentfill}%
\pgfsetlinewidth{0.752812pt}%
\definecolor{currentstroke}{rgb}{0.240000,0.240000,0.240000}%
\pgfsetstrokecolor{currentstroke}%
\pgfsetdash{}{0pt}%
\pgfpathmoveto{\pgfqpoint{10.008922in}{11.943042in}}%
\pgfpathlineto{\pgfqpoint{10.230844in}{11.943042in}}%
\pgfpathlineto{\pgfqpoint{10.230844in}{12.085744in}}%
\pgfpathlineto{\pgfqpoint{10.008922in}{12.085744in}}%
\pgfpathlineto{\pgfqpoint{10.008922in}{11.943042in}}%
\pgfpathclose%
\pgfusepath{stroke,fill}%
\end{pgfscope}%
\begin{pgfscope}%
\pgfpathrectangle{\pgfqpoint{0.688192in}{9.576569in}}{\pgfqpoint{11.096108in}{4.223431in}}%
\pgfusepath{clip}%
\pgfsetbuttcap%
\pgfsetroundjoin%
\pgfsetlinewidth{1.003750pt}%
\definecolor{currentstroke}{rgb}{0.450000,0.450000,0.450000}%
\pgfsetstrokecolor{currentstroke}%
\pgfsetdash{}{0pt}%
\pgfpathmoveto{\pgfqpoint{10.119883in}{12.259327in}}%
\pgfpathcurveto{\pgfqpoint{10.129092in}{12.259327in}}{\pgfqpoint{10.137924in}{12.262986in}}{\pgfqpoint{10.144436in}{12.269497in}}%
\pgfpathcurveto{\pgfqpoint{10.150947in}{12.276008in}}{\pgfqpoint{10.154606in}{12.284841in}}{\pgfqpoint{10.154606in}{12.294049in}}%
\pgfpathcurveto{\pgfqpoint{10.154606in}{12.303258in}}{\pgfqpoint{10.150947in}{12.312090in}}{\pgfqpoint{10.144436in}{12.318602in}}%
\pgfpathcurveto{\pgfqpoint{10.137924in}{12.325113in}}{\pgfqpoint{10.129092in}{12.328772in}}{\pgfqpoint{10.119883in}{12.328772in}}%
\pgfpathcurveto{\pgfqpoint{10.110675in}{12.328772in}}{\pgfqpoint{10.101842in}{12.325113in}}{\pgfqpoint{10.095331in}{12.318602in}}%
\pgfpathcurveto{\pgfqpoint{10.088820in}{12.312090in}}{\pgfqpoint{10.085161in}{12.303258in}}{\pgfqpoint{10.085161in}{12.294049in}}%
\pgfpathcurveto{\pgfqpoint{10.085161in}{12.284841in}}{\pgfqpoint{10.088820in}{12.276008in}}{\pgfqpoint{10.095331in}{12.269497in}}%
\pgfpathcurveto{\pgfqpoint{10.101842in}{12.262986in}}{\pgfqpoint{10.110675in}{12.259327in}}{\pgfqpoint{10.119883in}{12.259327in}}%
\pgfpathlineto{\pgfqpoint{10.119883in}{12.259327in}}%
\pgfpathclose%
\pgfusepath{stroke}%
\end{pgfscope}%
\begin{pgfscope}%
\pgfpathrectangle{\pgfqpoint{0.688192in}{9.576569in}}{\pgfqpoint{11.096108in}{4.223431in}}%
\pgfusepath{clip}%
\pgfsetbuttcap%
\pgfsetroundjoin%
\pgfsetlinewidth{1.003750pt}%
\definecolor{currentstroke}{rgb}{0.450000,0.450000,0.450000}%
\pgfsetstrokecolor{currentstroke}%
\pgfsetdash{}{0pt}%
\pgfpathmoveto{\pgfqpoint{10.119883in}{12.150504in}}%
\pgfpathcurveto{\pgfqpoint{10.129092in}{12.150504in}}{\pgfqpoint{10.137924in}{12.154162in}}{\pgfqpoint{10.144436in}{12.160674in}}%
\pgfpathcurveto{\pgfqpoint{10.150947in}{12.167185in}}{\pgfqpoint{10.154606in}{12.176017in}}{\pgfqpoint{10.154606in}{12.185226in}}%
\pgfpathcurveto{\pgfqpoint{10.154606in}{12.194434in}}{\pgfqpoint{10.150947in}{12.203267in}}{\pgfqpoint{10.144436in}{12.209778in}}%
\pgfpathcurveto{\pgfqpoint{10.137924in}{12.216290in}}{\pgfqpoint{10.129092in}{12.219948in}}{\pgfqpoint{10.119883in}{12.219948in}}%
\pgfpathcurveto{\pgfqpoint{10.110675in}{12.219948in}}{\pgfqpoint{10.101842in}{12.216290in}}{\pgfqpoint{10.095331in}{12.209778in}}%
\pgfpathcurveto{\pgfqpoint{10.088820in}{12.203267in}}{\pgfqpoint{10.085161in}{12.194434in}}{\pgfqpoint{10.085161in}{12.185226in}}%
\pgfpathcurveto{\pgfqpoint{10.085161in}{12.176017in}}{\pgfqpoint{10.088820in}{12.167185in}}{\pgfqpoint{10.095331in}{12.160674in}}%
\pgfpathcurveto{\pgfqpoint{10.101842in}{12.154162in}}{\pgfqpoint{10.110675in}{12.150504in}}{\pgfqpoint{10.119883in}{12.150504in}}%
\pgfpathlineto{\pgfqpoint{10.119883in}{12.150504in}}%
\pgfpathclose%
\pgfusepath{stroke}%
\end{pgfscope}%
\begin{pgfscope}%
\pgfpathrectangle{\pgfqpoint{0.688192in}{9.576569in}}{\pgfqpoint{11.096108in}{4.223431in}}%
\pgfusepath{clip}%
\pgfsetbuttcap%
\pgfsetroundjoin%
\pgfsetlinewidth{1.003750pt}%
\definecolor{currentstroke}{rgb}{0.450000,0.450000,0.450000}%
\pgfsetstrokecolor{currentstroke}%
\pgfsetdash{}{0pt}%
\pgfpathmoveto{\pgfqpoint{10.119883in}{12.149284in}}%
\pgfpathcurveto{\pgfqpoint{10.129092in}{12.149284in}}{\pgfqpoint{10.137924in}{12.152943in}}{\pgfqpoint{10.144436in}{12.159454in}}%
\pgfpathcurveto{\pgfqpoint{10.150947in}{12.165965in}}{\pgfqpoint{10.154606in}{12.174798in}}{\pgfqpoint{10.154606in}{12.184006in}}%
\pgfpathcurveto{\pgfqpoint{10.154606in}{12.193215in}}{\pgfqpoint{10.150947in}{12.202047in}}{\pgfqpoint{10.144436in}{12.208559in}}%
\pgfpathcurveto{\pgfqpoint{10.137924in}{12.215070in}}{\pgfqpoint{10.129092in}{12.218729in}}{\pgfqpoint{10.119883in}{12.218729in}}%
\pgfpathcurveto{\pgfqpoint{10.110675in}{12.218729in}}{\pgfqpoint{10.101842in}{12.215070in}}{\pgfqpoint{10.095331in}{12.208559in}}%
\pgfpathcurveto{\pgfqpoint{10.088820in}{12.202047in}}{\pgfqpoint{10.085161in}{12.193215in}}{\pgfqpoint{10.085161in}{12.184006in}}%
\pgfpathcurveto{\pgfqpoint{10.085161in}{12.174798in}}{\pgfqpoint{10.088820in}{12.165965in}}{\pgfqpoint{10.095331in}{12.159454in}}%
\pgfpathcurveto{\pgfqpoint{10.101842in}{12.152943in}}{\pgfqpoint{10.110675in}{12.149284in}}{\pgfqpoint{10.119883in}{12.149284in}}%
\pgfpathlineto{\pgfqpoint{10.119883in}{12.149284in}}%
\pgfpathclose%
\pgfusepath{stroke}%
\end{pgfscope}%
\begin{pgfscope}%
\pgfpathrectangle{\pgfqpoint{0.688192in}{9.576569in}}{\pgfqpoint{11.096108in}{4.223431in}}%
\pgfusepath{clip}%
\pgfsetbuttcap%
\pgfsetroundjoin%
\pgfsetlinewidth{1.003750pt}%
\definecolor{currentstroke}{rgb}{0.450000,0.450000,0.450000}%
\pgfsetstrokecolor{currentstroke}%
\pgfsetdash{}{0pt}%
\pgfpathmoveto{\pgfqpoint{10.119883in}{12.124027in}}%
\pgfpathcurveto{\pgfqpoint{10.129092in}{12.124027in}}{\pgfqpoint{10.137924in}{12.127685in}}{\pgfqpoint{10.144436in}{12.134197in}}%
\pgfpathcurveto{\pgfqpoint{10.150947in}{12.140708in}}{\pgfqpoint{10.154606in}{12.149541in}}{\pgfqpoint{10.154606in}{12.158749in}}%
\pgfpathcurveto{\pgfqpoint{10.154606in}{12.167957in}}{\pgfqpoint{10.150947in}{12.176790in}}{\pgfqpoint{10.144436in}{12.183301in}}%
\pgfpathcurveto{\pgfqpoint{10.137924in}{12.189813in}}{\pgfqpoint{10.129092in}{12.193471in}}{\pgfqpoint{10.119883in}{12.193471in}}%
\pgfpathcurveto{\pgfqpoint{10.110675in}{12.193471in}}{\pgfqpoint{10.101842in}{12.189813in}}{\pgfqpoint{10.095331in}{12.183301in}}%
\pgfpathcurveto{\pgfqpoint{10.088820in}{12.176790in}}{\pgfqpoint{10.085161in}{12.167957in}}{\pgfqpoint{10.085161in}{12.158749in}}%
\pgfpathcurveto{\pgfqpoint{10.085161in}{12.149541in}}{\pgfqpoint{10.088820in}{12.140708in}}{\pgfqpoint{10.095331in}{12.134197in}}%
\pgfpathcurveto{\pgfqpoint{10.101842in}{12.127685in}}{\pgfqpoint{10.110675in}{12.124027in}}{\pgfqpoint{10.119883in}{12.124027in}}%
\pgfpathlineto{\pgfqpoint{10.119883in}{12.124027in}}%
\pgfpathclose%
\pgfusepath{stroke}%
\end{pgfscope}%
\begin{pgfscope}%
\pgfpathrectangle{\pgfqpoint{0.688192in}{9.576569in}}{\pgfqpoint{11.096108in}{4.223431in}}%
\pgfusepath{clip}%
\pgfsetbuttcap%
\pgfsetroundjoin%
\definecolor{currentfill}{rgb}{0.455462,0.773335,0.806273}%
\pgfsetfillcolor{currentfill}%
\pgfsetlinewidth{0.752812pt}%
\definecolor{currentstroke}{rgb}{0.240000,0.240000,0.240000}%
\pgfsetstrokecolor{currentstroke}%
\pgfsetdash{}{0pt}%
\pgfpathmoveto{\pgfqpoint{11.118533in}{9.576569in}}%
\pgfpathlineto{\pgfqpoint{11.340455in}{9.576569in}}%
\pgfpathlineto{\pgfqpoint{11.340455in}{9.576569in}}%
\pgfpathlineto{\pgfqpoint{11.118533in}{9.576569in}}%
\pgfpathlineto{\pgfqpoint{11.118533in}{9.576569in}}%
\pgfpathclose%
\pgfusepath{stroke,fill}%
\end{pgfscope}%
\begin{pgfscope}%
\pgfpathrectangle{\pgfqpoint{0.688192in}{9.576569in}}{\pgfqpoint{11.096108in}{4.223431in}}%
\pgfusepath{clip}%
\pgfsetbuttcap%
\pgfsetroundjoin%
\definecolor{currentfill}{rgb}{0.332558,0.727865,0.768426}%
\pgfsetfillcolor{currentfill}%
\pgfsetlinewidth{0.752812pt}%
\definecolor{currentstroke}{rgb}{0.240000,0.240000,0.240000}%
\pgfsetstrokecolor{currentstroke}%
\pgfsetdash{}{0pt}%
\pgfpathmoveto{\pgfqpoint{11.007572in}{9.576569in}}%
\pgfpathlineto{\pgfqpoint{11.451416in}{9.576569in}}%
\pgfpathlineto{\pgfqpoint{11.451416in}{9.576569in}}%
\pgfpathlineto{\pgfqpoint{11.007572in}{9.576569in}}%
\pgfpathlineto{\pgfqpoint{11.007572in}{9.576569in}}%
\pgfpathclose%
\pgfusepath{stroke,fill}%
\end{pgfscope}%
\begin{pgfscope}%
\pgfpathrectangle{\pgfqpoint{0.688192in}{9.576569in}}{\pgfqpoint{11.096108in}{4.223431in}}%
\pgfusepath{clip}%
\pgfsetbuttcap%
\pgfsetroundjoin%
\definecolor{currentfill}{rgb}{0.180392,0.671569,0.721569}%
\pgfsetfillcolor{currentfill}%
\pgfsetlinewidth{0.752812pt}%
\definecolor{currentstroke}{rgb}{0.240000,0.240000,0.240000}%
\pgfsetstrokecolor{currentstroke}%
\pgfsetdash{}{0pt}%
\pgfpathmoveto{\pgfqpoint{10.785650in}{9.576569in}}%
\pgfpathlineto{\pgfqpoint{11.673338in}{9.576569in}}%
\pgfpathlineto{\pgfqpoint{11.673338in}{9.576569in}}%
\pgfpathlineto{\pgfqpoint{10.785650in}{9.576569in}}%
\pgfpathlineto{\pgfqpoint{10.785650in}{9.576569in}}%
\pgfpathclose%
\pgfusepath{stroke,fill}%
\end{pgfscope}%
\begin{pgfscope}%
\pgfpathrectangle{\pgfqpoint{0.688192in}{9.576569in}}{\pgfqpoint{11.096108in}{4.223431in}}%
\pgfusepath{clip}%
\pgfsetbuttcap%
\pgfsetroundjoin%
\definecolor{currentfill}{rgb}{0.332558,0.727865,0.768426}%
\pgfsetfillcolor{currentfill}%
\pgfsetlinewidth{0.752812pt}%
\definecolor{currentstroke}{rgb}{0.240000,0.240000,0.240000}%
\pgfsetstrokecolor{currentstroke}%
\pgfsetdash{}{0pt}%
\pgfpathmoveto{\pgfqpoint{11.007572in}{9.576569in}}%
\pgfpathlineto{\pgfqpoint{11.451416in}{9.576569in}}%
\pgfpathlineto{\pgfqpoint{11.451416in}{9.576569in}}%
\pgfpathlineto{\pgfqpoint{11.007572in}{9.576569in}}%
\pgfpathlineto{\pgfqpoint{11.007572in}{9.576569in}}%
\pgfpathclose%
\pgfusepath{stroke,fill}%
\end{pgfscope}%
\begin{pgfscope}%
\pgfpathrectangle{\pgfqpoint{0.688192in}{9.576569in}}{\pgfqpoint{11.096108in}{4.223431in}}%
\pgfusepath{clip}%
\pgfsetbuttcap%
\pgfsetroundjoin%
\definecolor{currentfill}{rgb}{0.455462,0.773335,0.806273}%
\pgfsetfillcolor{currentfill}%
\pgfsetlinewidth{0.752812pt}%
\definecolor{currentstroke}{rgb}{0.240000,0.240000,0.240000}%
\pgfsetstrokecolor{currentstroke}%
\pgfsetdash{}{0pt}%
\pgfpathmoveto{\pgfqpoint{11.118533in}{9.576569in}}%
\pgfpathlineto{\pgfqpoint{11.340455in}{9.576569in}}%
\pgfpathlineto{\pgfqpoint{11.340455in}{9.576569in}}%
\pgfpathlineto{\pgfqpoint{11.118533in}{9.576569in}}%
\pgfpathlineto{\pgfqpoint{11.118533in}{9.576569in}}%
\pgfpathclose%
\pgfusepath{stroke,fill}%
\end{pgfscope}%
\begin{pgfscope}%
\pgfpathrectangle{\pgfqpoint{0.688192in}{9.576569in}}{\pgfqpoint{11.096108in}{4.223431in}}%
\pgfusepath{clip}%
\pgfsetbuttcap%
\pgfsetroundjoin%
\pgfsetlinewidth{1.003750pt}%
\definecolor{currentstroke}{rgb}{0.450000,0.450000,0.450000}%
\pgfsetstrokecolor{currentstroke}%
\pgfsetdash{}{0pt}%
\pgfpathmoveto{\pgfqpoint{11.229494in}{9.694790in}}%
\pgfpathcurveto{\pgfqpoint{11.238703in}{9.694790in}}{\pgfqpoint{11.247535in}{9.698449in}}{\pgfqpoint{11.254046in}{9.704960in}}%
\pgfpathcurveto{\pgfqpoint{11.260558in}{9.711471in}}{\pgfqpoint{11.264216in}{9.720304in}}{\pgfqpoint{11.264216in}{9.729512in}}%
\pgfpathcurveto{\pgfqpoint{11.264216in}{9.738721in}}{\pgfqpoint{11.260558in}{9.747553in}}{\pgfqpoint{11.254046in}{9.754065in}}%
\pgfpathcurveto{\pgfqpoint{11.247535in}{9.760576in}}{\pgfqpoint{11.238703in}{9.764235in}}{\pgfqpoint{11.229494in}{9.764235in}}%
\pgfpathcurveto{\pgfqpoint{11.220286in}{9.764235in}}{\pgfqpoint{11.211453in}{9.760576in}}{\pgfqpoint{11.204942in}{9.754065in}}%
\pgfpathcurveto{\pgfqpoint{11.198430in}{9.747553in}}{\pgfqpoint{11.194772in}{9.738721in}}{\pgfqpoint{11.194772in}{9.729512in}}%
\pgfpathcurveto{\pgfqpoint{11.194772in}{9.720304in}}{\pgfqpoint{11.198430in}{9.711471in}}{\pgfqpoint{11.204942in}{9.704960in}}%
\pgfpathcurveto{\pgfqpoint{11.211453in}{9.698449in}}{\pgfqpoint{11.220286in}{9.694790in}}{\pgfqpoint{11.229494in}{9.694790in}}%
\pgfpathlineto{\pgfqpoint{11.229494in}{9.694790in}}%
\pgfpathclose%
\pgfusepath{stroke}%
\end{pgfscope}%
\begin{pgfscope}%
\pgfpathrectangle{\pgfqpoint{0.688192in}{9.576569in}}{\pgfqpoint{11.096108in}{4.223431in}}%
\pgfusepath{clip}%
\pgfsetbuttcap%
\pgfsetroundjoin%
\pgfsetlinewidth{1.003750pt}%
\definecolor{currentstroke}{rgb}{0.450000,0.450000,0.450000}%
\pgfsetstrokecolor{currentstroke}%
\pgfsetdash{}{0pt}%
\pgfpathmoveto{\pgfqpoint{11.229494in}{9.726788in}}%
\pgfpathcurveto{\pgfqpoint{11.238703in}{9.726788in}}{\pgfqpoint{11.247535in}{9.730446in}}{\pgfqpoint{11.254046in}{9.736958in}}%
\pgfpathcurveto{\pgfqpoint{11.260558in}{9.743469in}}{\pgfqpoint{11.264216in}{9.752301in}}{\pgfqpoint{11.264216in}{9.761510in}}%
\pgfpathcurveto{\pgfqpoint{11.264216in}{9.770718in}}{\pgfqpoint{11.260558in}{9.779551in}}{\pgfqpoint{11.254046in}{9.786062in}}%
\pgfpathcurveto{\pgfqpoint{11.247535in}{9.792574in}}{\pgfqpoint{11.238703in}{9.796232in}}{\pgfqpoint{11.229494in}{9.796232in}}%
\pgfpathcurveto{\pgfqpoint{11.220286in}{9.796232in}}{\pgfqpoint{11.211453in}{9.792574in}}{\pgfqpoint{11.204942in}{9.786062in}}%
\pgfpathcurveto{\pgfqpoint{11.198430in}{9.779551in}}{\pgfqpoint{11.194772in}{9.770718in}}{\pgfqpoint{11.194772in}{9.761510in}}%
\pgfpathcurveto{\pgfqpoint{11.194772in}{9.752301in}}{\pgfqpoint{11.198430in}{9.743469in}}{\pgfqpoint{11.204942in}{9.736958in}}%
\pgfpathcurveto{\pgfqpoint{11.211453in}{9.730446in}}{\pgfqpoint{11.220286in}{9.726788in}}{\pgfqpoint{11.229494in}{9.726788in}}%
\pgfpathlineto{\pgfqpoint{11.229494in}{9.726788in}}%
\pgfpathclose%
\pgfusepath{stroke}%
\end{pgfscope}%
\begin{pgfscope}%
\pgfpathrectangle{\pgfqpoint{0.688192in}{9.576569in}}{\pgfqpoint{11.096108in}{4.223431in}}%
\pgfusepath{clip}%
\pgfsetrectcap%
\pgfsetroundjoin%
\pgfsetlinewidth{0.803000pt}%
\definecolor{currentstroke}{rgb}{0.690196,0.690196,0.690196}%
\pgfsetstrokecolor{currentstroke}%
\pgfsetstrokeopacity{0.200000}%
\pgfsetdash{}{0pt}%
\pgfpathmoveto{\pgfqpoint{1.242997in}{9.576569in}}%
\pgfpathlineto{\pgfqpoint{1.242997in}{13.800000in}}%
\pgfusepath{stroke}%
\end{pgfscope}%
\begin{pgfscope}%
\pgfsetbuttcap%
\pgfsetroundjoin%
\definecolor{currentfill}{rgb}{0.000000,0.000000,0.000000}%
\pgfsetfillcolor{currentfill}%
\pgfsetlinewidth{0.803000pt}%
\definecolor{currentstroke}{rgb}{0.000000,0.000000,0.000000}%
\pgfsetstrokecolor{currentstroke}%
\pgfsetdash{}{0pt}%
\pgfsys@defobject{currentmarker}{\pgfqpoint{0.000000in}{-0.048611in}}{\pgfqpoint{0.000000in}{0.000000in}}{%
\pgfpathmoveto{\pgfqpoint{0.000000in}{0.000000in}}%
\pgfpathlineto{\pgfqpoint{0.000000in}{-0.048611in}}%
\pgfusepath{stroke,fill}%
}%
\begin{pgfscope}%
\pgfsys@transformshift{1.242997in}{9.576569in}%
\pgfsys@useobject{currentmarker}{}%
\end{pgfscope}%
\end{pgfscope}%
\begin{pgfscope}%
\pgfpathrectangle{\pgfqpoint{0.688192in}{9.576569in}}{\pgfqpoint{11.096108in}{4.223431in}}%
\pgfusepath{clip}%
\pgfsetrectcap%
\pgfsetroundjoin%
\pgfsetlinewidth{0.803000pt}%
\definecolor{currentstroke}{rgb}{0.690196,0.690196,0.690196}%
\pgfsetstrokecolor{currentstroke}%
\pgfsetstrokeopacity{0.200000}%
\pgfsetdash{}{0pt}%
\pgfpathmoveto{\pgfqpoint{2.352608in}{9.576569in}}%
\pgfpathlineto{\pgfqpoint{2.352608in}{13.800000in}}%
\pgfusepath{stroke}%
\end{pgfscope}%
\begin{pgfscope}%
\pgfsetbuttcap%
\pgfsetroundjoin%
\definecolor{currentfill}{rgb}{0.000000,0.000000,0.000000}%
\pgfsetfillcolor{currentfill}%
\pgfsetlinewidth{0.803000pt}%
\definecolor{currentstroke}{rgb}{0.000000,0.000000,0.000000}%
\pgfsetstrokecolor{currentstroke}%
\pgfsetdash{}{0pt}%
\pgfsys@defobject{currentmarker}{\pgfqpoint{0.000000in}{-0.048611in}}{\pgfqpoint{0.000000in}{0.000000in}}{%
\pgfpathmoveto{\pgfqpoint{0.000000in}{0.000000in}}%
\pgfpathlineto{\pgfqpoint{0.000000in}{-0.048611in}}%
\pgfusepath{stroke,fill}%
}%
\begin{pgfscope}%
\pgfsys@transformshift{2.352608in}{9.576569in}%
\pgfsys@useobject{currentmarker}{}%
\end{pgfscope}%
\end{pgfscope}%
\begin{pgfscope}%
\pgfpathrectangle{\pgfqpoint{0.688192in}{9.576569in}}{\pgfqpoint{11.096108in}{4.223431in}}%
\pgfusepath{clip}%
\pgfsetrectcap%
\pgfsetroundjoin%
\pgfsetlinewidth{0.803000pt}%
\definecolor{currentstroke}{rgb}{0.690196,0.690196,0.690196}%
\pgfsetstrokecolor{currentstroke}%
\pgfsetstrokeopacity{0.200000}%
\pgfsetdash{}{0pt}%
\pgfpathmoveto{\pgfqpoint{3.462219in}{9.576569in}}%
\pgfpathlineto{\pgfqpoint{3.462219in}{13.800000in}}%
\pgfusepath{stroke}%
\end{pgfscope}%
\begin{pgfscope}%
\pgfsetbuttcap%
\pgfsetroundjoin%
\definecolor{currentfill}{rgb}{0.000000,0.000000,0.000000}%
\pgfsetfillcolor{currentfill}%
\pgfsetlinewidth{0.803000pt}%
\definecolor{currentstroke}{rgb}{0.000000,0.000000,0.000000}%
\pgfsetstrokecolor{currentstroke}%
\pgfsetdash{}{0pt}%
\pgfsys@defobject{currentmarker}{\pgfqpoint{0.000000in}{-0.048611in}}{\pgfqpoint{0.000000in}{0.000000in}}{%
\pgfpathmoveto{\pgfqpoint{0.000000in}{0.000000in}}%
\pgfpathlineto{\pgfqpoint{0.000000in}{-0.048611in}}%
\pgfusepath{stroke,fill}%
}%
\begin{pgfscope}%
\pgfsys@transformshift{3.462219in}{9.576569in}%
\pgfsys@useobject{currentmarker}{}%
\end{pgfscope}%
\end{pgfscope}%
\begin{pgfscope}%
\pgfpathrectangle{\pgfqpoint{0.688192in}{9.576569in}}{\pgfqpoint{11.096108in}{4.223431in}}%
\pgfusepath{clip}%
\pgfsetrectcap%
\pgfsetroundjoin%
\pgfsetlinewidth{0.803000pt}%
\definecolor{currentstroke}{rgb}{0.690196,0.690196,0.690196}%
\pgfsetstrokecolor{currentstroke}%
\pgfsetstrokeopacity{0.200000}%
\pgfsetdash{}{0pt}%
\pgfpathmoveto{\pgfqpoint{4.571829in}{9.576569in}}%
\pgfpathlineto{\pgfqpoint{4.571829in}{13.800000in}}%
\pgfusepath{stroke}%
\end{pgfscope}%
\begin{pgfscope}%
\pgfsetbuttcap%
\pgfsetroundjoin%
\definecolor{currentfill}{rgb}{0.000000,0.000000,0.000000}%
\pgfsetfillcolor{currentfill}%
\pgfsetlinewidth{0.803000pt}%
\definecolor{currentstroke}{rgb}{0.000000,0.000000,0.000000}%
\pgfsetstrokecolor{currentstroke}%
\pgfsetdash{}{0pt}%
\pgfsys@defobject{currentmarker}{\pgfqpoint{0.000000in}{-0.048611in}}{\pgfqpoint{0.000000in}{0.000000in}}{%
\pgfpathmoveto{\pgfqpoint{0.000000in}{0.000000in}}%
\pgfpathlineto{\pgfqpoint{0.000000in}{-0.048611in}}%
\pgfusepath{stroke,fill}%
}%
\begin{pgfscope}%
\pgfsys@transformshift{4.571829in}{9.576569in}%
\pgfsys@useobject{currentmarker}{}%
\end{pgfscope}%
\end{pgfscope}%
\begin{pgfscope}%
\pgfpathrectangle{\pgfqpoint{0.688192in}{9.576569in}}{\pgfqpoint{11.096108in}{4.223431in}}%
\pgfusepath{clip}%
\pgfsetrectcap%
\pgfsetroundjoin%
\pgfsetlinewidth{0.803000pt}%
\definecolor{currentstroke}{rgb}{0.690196,0.690196,0.690196}%
\pgfsetstrokecolor{currentstroke}%
\pgfsetstrokeopacity{0.200000}%
\pgfsetdash{}{0pt}%
\pgfpathmoveto{\pgfqpoint{5.681440in}{9.576569in}}%
\pgfpathlineto{\pgfqpoint{5.681440in}{13.800000in}}%
\pgfusepath{stroke}%
\end{pgfscope}%
\begin{pgfscope}%
\pgfsetbuttcap%
\pgfsetroundjoin%
\definecolor{currentfill}{rgb}{0.000000,0.000000,0.000000}%
\pgfsetfillcolor{currentfill}%
\pgfsetlinewidth{0.803000pt}%
\definecolor{currentstroke}{rgb}{0.000000,0.000000,0.000000}%
\pgfsetstrokecolor{currentstroke}%
\pgfsetdash{}{0pt}%
\pgfsys@defobject{currentmarker}{\pgfqpoint{0.000000in}{-0.048611in}}{\pgfqpoint{0.000000in}{0.000000in}}{%
\pgfpathmoveto{\pgfqpoint{0.000000in}{0.000000in}}%
\pgfpathlineto{\pgfqpoint{0.000000in}{-0.048611in}}%
\pgfusepath{stroke,fill}%
}%
\begin{pgfscope}%
\pgfsys@transformshift{5.681440in}{9.576569in}%
\pgfsys@useobject{currentmarker}{}%
\end{pgfscope}%
\end{pgfscope}%
\begin{pgfscope}%
\pgfpathrectangle{\pgfqpoint{0.688192in}{9.576569in}}{\pgfqpoint{11.096108in}{4.223431in}}%
\pgfusepath{clip}%
\pgfsetrectcap%
\pgfsetroundjoin%
\pgfsetlinewidth{0.803000pt}%
\definecolor{currentstroke}{rgb}{0.690196,0.690196,0.690196}%
\pgfsetstrokecolor{currentstroke}%
\pgfsetstrokeopacity{0.200000}%
\pgfsetdash{}{0pt}%
\pgfpathmoveto{\pgfqpoint{6.791051in}{9.576569in}}%
\pgfpathlineto{\pgfqpoint{6.791051in}{13.800000in}}%
\pgfusepath{stroke}%
\end{pgfscope}%
\begin{pgfscope}%
\pgfsetbuttcap%
\pgfsetroundjoin%
\definecolor{currentfill}{rgb}{0.000000,0.000000,0.000000}%
\pgfsetfillcolor{currentfill}%
\pgfsetlinewidth{0.803000pt}%
\definecolor{currentstroke}{rgb}{0.000000,0.000000,0.000000}%
\pgfsetstrokecolor{currentstroke}%
\pgfsetdash{}{0pt}%
\pgfsys@defobject{currentmarker}{\pgfqpoint{0.000000in}{-0.048611in}}{\pgfqpoint{0.000000in}{0.000000in}}{%
\pgfpathmoveto{\pgfqpoint{0.000000in}{0.000000in}}%
\pgfpathlineto{\pgfqpoint{0.000000in}{-0.048611in}}%
\pgfusepath{stroke,fill}%
}%
\begin{pgfscope}%
\pgfsys@transformshift{6.791051in}{9.576569in}%
\pgfsys@useobject{currentmarker}{}%
\end{pgfscope}%
\end{pgfscope}%
\begin{pgfscope}%
\pgfpathrectangle{\pgfqpoint{0.688192in}{9.576569in}}{\pgfqpoint{11.096108in}{4.223431in}}%
\pgfusepath{clip}%
\pgfsetrectcap%
\pgfsetroundjoin%
\pgfsetlinewidth{0.803000pt}%
\definecolor{currentstroke}{rgb}{0.690196,0.690196,0.690196}%
\pgfsetstrokecolor{currentstroke}%
\pgfsetstrokeopacity{0.200000}%
\pgfsetdash{}{0pt}%
\pgfpathmoveto{\pgfqpoint{7.900662in}{9.576569in}}%
\pgfpathlineto{\pgfqpoint{7.900662in}{13.800000in}}%
\pgfusepath{stroke}%
\end{pgfscope}%
\begin{pgfscope}%
\pgfsetbuttcap%
\pgfsetroundjoin%
\definecolor{currentfill}{rgb}{0.000000,0.000000,0.000000}%
\pgfsetfillcolor{currentfill}%
\pgfsetlinewidth{0.803000pt}%
\definecolor{currentstroke}{rgb}{0.000000,0.000000,0.000000}%
\pgfsetstrokecolor{currentstroke}%
\pgfsetdash{}{0pt}%
\pgfsys@defobject{currentmarker}{\pgfqpoint{0.000000in}{-0.048611in}}{\pgfqpoint{0.000000in}{0.000000in}}{%
\pgfpathmoveto{\pgfqpoint{0.000000in}{0.000000in}}%
\pgfpathlineto{\pgfqpoint{0.000000in}{-0.048611in}}%
\pgfusepath{stroke,fill}%
}%
\begin{pgfscope}%
\pgfsys@transformshift{7.900662in}{9.576569in}%
\pgfsys@useobject{currentmarker}{}%
\end{pgfscope}%
\end{pgfscope}%
\begin{pgfscope}%
\pgfpathrectangle{\pgfqpoint{0.688192in}{9.576569in}}{\pgfqpoint{11.096108in}{4.223431in}}%
\pgfusepath{clip}%
\pgfsetrectcap%
\pgfsetroundjoin%
\pgfsetlinewidth{0.803000pt}%
\definecolor{currentstroke}{rgb}{0.690196,0.690196,0.690196}%
\pgfsetstrokecolor{currentstroke}%
\pgfsetstrokeopacity{0.200000}%
\pgfsetdash{}{0pt}%
\pgfpathmoveto{\pgfqpoint{9.010272in}{9.576569in}}%
\pgfpathlineto{\pgfqpoint{9.010272in}{13.800000in}}%
\pgfusepath{stroke}%
\end{pgfscope}%
\begin{pgfscope}%
\pgfsetbuttcap%
\pgfsetroundjoin%
\definecolor{currentfill}{rgb}{0.000000,0.000000,0.000000}%
\pgfsetfillcolor{currentfill}%
\pgfsetlinewidth{0.803000pt}%
\definecolor{currentstroke}{rgb}{0.000000,0.000000,0.000000}%
\pgfsetstrokecolor{currentstroke}%
\pgfsetdash{}{0pt}%
\pgfsys@defobject{currentmarker}{\pgfqpoint{0.000000in}{-0.048611in}}{\pgfqpoint{0.000000in}{0.000000in}}{%
\pgfpathmoveto{\pgfqpoint{0.000000in}{0.000000in}}%
\pgfpathlineto{\pgfqpoint{0.000000in}{-0.048611in}}%
\pgfusepath{stroke,fill}%
}%
\begin{pgfscope}%
\pgfsys@transformshift{9.010272in}{9.576569in}%
\pgfsys@useobject{currentmarker}{}%
\end{pgfscope}%
\end{pgfscope}%
\begin{pgfscope}%
\pgfpathrectangle{\pgfqpoint{0.688192in}{9.576569in}}{\pgfqpoint{11.096108in}{4.223431in}}%
\pgfusepath{clip}%
\pgfsetrectcap%
\pgfsetroundjoin%
\pgfsetlinewidth{0.803000pt}%
\definecolor{currentstroke}{rgb}{0.690196,0.690196,0.690196}%
\pgfsetstrokecolor{currentstroke}%
\pgfsetstrokeopacity{0.200000}%
\pgfsetdash{}{0pt}%
\pgfpathmoveto{\pgfqpoint{10.119883in}{9.576569in}}%
\pgfpathlineto{\pgfqpoint{10.119883in}{13.800000in}}%
\pgfusepath{stroke}%
\end{pgfscope}%
\begin{pgfscope}%
\pgfsetbuttcap%
\pgfsetroundjoin%
\definecolor{currentfill}{rgb}{0.000000,0.000000,0.000000}%
\pgfsetfillcolor{currentfill}%
\pgfsetlinewidth{0.803000pt}%
\definecolor{currentstroke}{rgb}{0.000000,0.000000,0.000000}%
\pgfsetstrokecolor{currentstroke}%
\pgfsetdash{}{0pt}%
\pgfsys@defobject{currentmarker}{\pgfqpoint{0.000000in}{-0.048611in}}{\pgfqpoint{0.000000in}{0.000000in}}{%
\pgfpathmoveto{\pgfqpoint{0.000000in}{0.000000in}}%
\pgfpathlineto{\pgfqpoint{0.000000in}{-0.048611in}}%
\pgfusepath{stroke,fill}%
}%
\begin{pgfscope}%
\pgfsys@transformshift{10.119883in}{9.576569in}%
\pgfsys@useobject{currentmarker}{}%
\end{pgfscope}%
\end{pgfscope}%
\begin{pgfscope}%
\pgfpathrectangle{\pgfqpoint{0.688192in}{9.576569in}}{\pgfqpoint{11.096108in}{4.223431in}}%
\pgfusepath{clip}%
\pgfsetrectcap%
\pgfsetroundjoin%
\pgfsetlinewidth{0.803000pt}%
\definecolor{currentstroke}{rgb}{0.690196,0.690196,0.690196}%
\pgfsetstrokecolor{currentstroke}%
\pgfsetstrokeopacity{0.200000}%
\pgfsetdash{}{0pt}%
\pgfpathmoveto{\pgfqpoint{11.229494in}{9.576569in}}%
\pgfpathlineto{\pgfqpoint{11.229494in}{13.800000in}}%
\pgfusepath{stroke}%
\end{pgfscope}%
\begin{pgfscope}%
\pgfsetbuttcap%
\pgfsetroundjoin%
\definecolor{currentfill}{rgb}{0.000000,0.000000,0.000000}%
\pgfsetfillcolor{currentfill}%
\pgfsetlinewidth{0.803000pt}%
\definecolor{currentstroke}{rgb}{0.000000,0.000000,0.000000}%
\pgfsetstrokecolor{currentstroke}%
\pgfsetdash{}{0pt}%
\pgfsys@defobject{currentmarker}{\pgfqpoint{0.000000in}{-0.048611in}}{\pgfqpoint{0.000000in}{0.000000in}}{%
\pgfpathmoveto{\pgfqpoint{0.000000in}{0.000000in}}%
\pgfpathlineto{\pgfqpoint{0.000000in}{-0.048611in}}%
\pgfusepath{stroke,fill}%
}%
\begin{pgfscope}%
\pgfsys@transformshift{11.229494in}{9.576569in}%
\pgfsys@useobject{currentmarker}{}%
\end{pgfscope}%
\end{pgfscope}%
\begin{pgfscope}%
\pgfpathrectangle{\pgfqpoint{0.688192in}{9.576569in}}{\pgfqpoint{11.096108in}{4.223431in}}%
\pgfusepath{clip}%
\pgfsetrectcap%
\pgfsetroundjoin%
\pgfsetlinewidth{0.803000pt}%
\definecolor{currentstroke}{rgb}{0.690196,0.690196,0.690196}%
\pgfsetstrokecolor{currentstroke}%
\pgfsetstrokeopacity{0.050000}%
\pgfsetdash{}{0pt}%
\pgfpathmoveto{\pgfqpoint{0.799153in}{9.576569in}}%
\pgfpathlineto{\pgfqpoint{0.799153in}{13.800000in}}%
\pgfusepath{stroke}%
\end{pgfscope}%
\begin{pgfscope}%
\pgfsetbuttcap%
\pgfsetroundjoin%
\definecolor{currentfill}{rgb}{0.000000,0.000000,0.000000}%
\pgfsetfillcolor{currentfill}%
\pgfsetlinewidth{0.602250pt}%
\definecolor{currentstroke}{rgb}{0.000000,0.000000,0.000000}%
\pgfsetstrokecolor{currentstroke}%
\pgfsetdash{}{0pt}%
\pgfsys@defobject{currentmarker}{\pgfqpoint{0.000000in}{-0.027778in}}{\pgfqpoint{0.000000in}{0.000000in}}{%
\pgfpathmoveto{\pgfqpoint{0.000000in}{0.000000in}}%
\pgfpathlineto{\pgfqpoint{0.000000in}{-0.027778in}}%
\pgfusepath{stroke,fill}%
}%
\begin{pgfscope}%
\pgfsys@transformshift{0.799153in}{9.576569in}%
\pgfsys@useobject{currentmarker}{}%
\end{pgfscope}%
\end{pgfscope}%
\begin{pgfscope}%
\pgfpathrectangle{\pgfqpoint{0.688192in}{9.576569in}}{\pgfqpoint{11.096108in}{4.223431in}}%
\pgfusepath{clip}%
\pgfsetrectcap%
\pgfsetroundjoin%
\pgfsetlinewidth{0.803000pt}%
\definecolor{currentstroke}{rgb}{0.690196,0.690196,0.690196}%
\pgfsetstrokecolor{currentstroke}%
\pgfsetstrokeopacity{0.050000}%
\pgfsetdash{}{0pt}%
\pgfpathmoveto{\pgfqpoint{1.021075in}{9.576569in}}%
\pgfpathlineto{\pgfqpoint{1.021075in}{13.800000in}}%
\pgfusepath{stroke}%
\end{pgfscope}%
\begin{pgfscope}%
\pgfsetbuttcap%
\pgfsetroundjoin%
\definecolor{currentfill}{rgb}{0.000000,0.000000,0.000000}%
\pgfsetfillcolor{currentfill}%
\pgfsetlinewidth{0.602250pt}%
\definecolor{currentstroke}{rgb}{0.000000,0.000000,0.000000}%
\pgfsetstrokecolor{currentstroke}%
\pgfsetdash{}{0pt}%
\pgfsys@defobject{currentmarker}{\pgfqpoint{0.000000in}{-0.027778in}}{\pgfqpoint{0.000000in}{0.000000in}}{%
\pgfpathmoveto{\pgfqpoint{0.000000in}{0.000000in}}%
\pgfpathlineto{\pgfqpoint{0.000000in}{-0.027778in}}%
\pgfusepath{stroke,fill}%
}%
\begin{pgfscope}%
\pgfsys@transformshift{1.021075in}{9.576569in}%
\pgfsys@useobject{currentmarker}{}%
\end{pgfscope}%
\end{pgfscope}%
\begin{pgfscope}%
\pgfpathrectangle{\pgfqpoint{0.688192in}{9.576569in}}{\pgfqpoint{11.096108in}{4.223431in}}%
\pgfusepath{clip}%
\pgfsetrectcap%
\pgfsetroundjoin%
\pgfsetlinewidth{0.803000pt}%
\definecolor{currentstroke}{rgb}{0.690196,0.690196,0.690196}%
\pgfsetstrokecolor{currentstroke}%
\pgfsetstrokeopacity{0.050000}%
\pgfsetdash{}{0pt}%
\pgfpathmoveto{\pgfqpoint{1.464919in}{9.576569in}}%
\pgfpathlineto{\pgfqpoint{1.464919in}{13.800000in}}%
\pgfusepath{stroke}%
\end{pgfscope}%
\begin{pgfscope}%
\pgfsetbuttcap%
\pgfsetroundjoin%
\definecolor{currentfill}{rgb}{0.000000,0.000000,0.000000}%
\pgfsetfillcolor{currentfill}%
\pgfsetlinewidth{0.602250pt}%
\definecolor{currentstroke}{rgb}{0.000000,0.000000,0.000000}%
\pgfsetstrokecolor{currentstroke}%
\pgfsetdash{}{0pt}%
\pgfsys@defobject{currentmarker}{\pgfqpoint{0.000000in}{-0.027778in}}{\pgfqpoint{0.000000in}{0.000000in}}{%
\pgfpathmoveto{\pgfqpoint{0.000000in}{0.000000in}}%
\pgfpathlineto{\pgfqpoint{0.000000in}{-0.027778in}}%
\pgfusepath{stroke,fill}%
}%
\begin{pgfscope}%
\pgfsys@transformshift{1.464919in}{9.576569in}%
\pgfsys@useobject{currentmarker}{}%
\end{pgfscope}%
\end{pgfscope}%
\begin{pgfscope}%
\pgfpathrectangle{\pgfqpoint{0.688192in}{9.576569in}}{\pgfqpoint{11.096108in}{4.223431in}}%
\pgfusepath{clip}%
\pgfsetrectcap%
\pgfsetroundjoin%
\pgfsetlinewidth{0.803000pt}%
\definecolor{currentstroke}{rgb}{0.690196,0.690196,0.690196}%
\pgfsetstrokecolor{currentstroke}%
\pgfsetstrokeopacity{0.050000}%
\pgfsetdash{}{0pt}%
\pgfpathmoveto{\pgfqpoint{1.686841in}{9.576569in}}%
\pgfpathlineto{\pgfqpoint{1.686841in}{13.800000in}}%
\pgfusepath{stroke}%
\end{pgfscope}%
\begin{pgfscope}%
\pgfsetbuttcap%
\pgfsetroundjoin%
\definecolor{currentfill}{rgb}{0.000000,0.000000,0.000000}%
\pgfsetfillcolor{currentfill}%
\pgfsetlinewidth{0.602250pt}%
\definecolor{currentstroke}{rgb}{0.000000,0.000000,0.000000}%
\pgfsetstrokecolor{currentstroke}%
\pgfsetdash{}{0pt}%
\pgfsys@defobject{currentmarker}{\pgfqpoint{0.000000in}{-0.027778in}}{\pgfqpoint{0.000000in}{0.000000in}}{%
\pgfpathmoveto{\pgfqpoint{0.000000in}{0.000000in}}%
\pgfpathlineto{\pgfqpoint{0.000000in}{-0.027778in}}%
\pgfusepath{stroke,fill}%
}%
\begin{pgfscope}%
\pgfsys@transformshift{1.686841in}{9.576569in}%
\pgfsys@useobject{currentmarker}{}%
\end{pgfscope}%
\end{pgfscope}%
\begin{pgfscope}%
\pgfpathrectangle{\pgfqpoint{0.688192in}{9.576569in}}{\pgfqpoint{11.096108in}{4.223431in}}%
\pgfusepath{clip}%
\pgfsetrectcap%
\pgfsetroundjoin%
\pgfsetlinewidth{0.803000pt}%
\definecolor{currentstroke}{rgb}{0.690196,0.690196,0.690196}%
\pgfsetstrokecolor{currentstroke}%
\pgfsetstrokeopacity{0.050000}%
\pgfsetdash{}{0pt}%
\pgfpathmoveto{\pgfqpoint{1.908763in}{9.576569in}}%
\pgfpathlineto{\pgfqpoint{1.908763in}{13.800000in}}%
\pgfusepath{stroke}%
\end{pgfscope}%
\begin{pgfscope}%
\pgfsetbuttcap%
\pgfsetroundjoin%
\definecolor{currentfill}{rgb}{0.000000,0.000000,0.000000}%
\pgfsetfillcolor{currentfill}%
\pgfsetlinewidth{0.602250pt}%
\definecolor{currentstroke}{rgb}{0.000000,0.000000,0.000000}%
\pgfsetstrokecolor{currentstroke}%
\pgfsetdash{}{0pt}%
\pgfsys@defobject{currentmarker}{\pgfqpoint{0.000000in}{-0.027778in}}{\pgfqpoint{0.000000in}{0.000000in}}{%
\pgfpathmoveto{\pgfqpoint{0.000000in}{0.000000in}}%
\pgfpathlineto{\pgfqpoint{0.000000in}{-0.027778in}}%
\pgfusepath{stroke,fill}%
}%
\begin{pgfscope}%
\pgfsys@transformshift{1.908763in}{9.576569in}%
\pgfsys@useobject{currentmarker}{}%
\end{pgfscope}%
\end{pgfscope}%
\begin{pgfscope}%
\pgfpathrectangle{\pgfqpoint{0.688192in}{9.576569in}}{\pgfqpoint{11.096108in}{4.223431in}}%
\pgfusepath{clip}%
\pgfsetrectcap%
\pgfsetroundjoin%
\pgfsetlinewidth{0.803000pt}%
\definecolor{currentstroke}{rgb}{0.690196,0.690196,0.690196}%
\pgfsetstrokecolor{currentstroke}%
\pgfsetstrokeopacity{0.050000}%
\pgfsetdash{}{0pt}%
\pgfpathmoveto{\pgfqpoint{2.130686in}{9.576569in}}%
\pgfpathlineto{\pgfqpoint{2.130686in}{13.800000in}}%
\pgfusepath{stroke}%
\end{pgfscope}%
\begin{pgfscope}%
\pgfsetbuttcap%
\pgfsetroundjoin%
\definecolor{currentfill}{rgb}{0.000000,0.000000,0.000000}%
\pgfsetfillcolor{currentfill}%
\pgfsetlinewidth{0.602250pt}%
\definecolor{currentstroke}{rgb}{0.000000,0.000000,0.000000}%
\pgfsetstrokecolor{currentstroke}%
\pgfsetdash{}{0pt}%
\pgfsys@defobject{currentmarker}{\pgfqpoint{0.000000in}{-0.027778in}}{\pgfqpoint{0.000000in}{0.000000in}}{%
\pgfpathmoveto{\pgfqpoint{0.000000in}{0.000000in}}%
\pgfpathlineto{\pgfqpoint{0.000000in}{-0.027778in}}%
\pgfusepath{stroke,fill}%
}%
\begin{pgfscope}%
\pgfsys@transformshift{2.130686in}{9.576569in}%
\pgfsys@useobject{currentmarker}{}%
\end{pgfscope}%
\end{pgfscope}%
\begin{pgfscope}%
\pgfpathrectangle{\pgfqpoint{0.688192in}{9.576569in}}{\pgfqpoint{11.096108in}{4.223431in}}%
\pgfusepath{clip}%
\pgfsetrectcap%
\pgfsetroundjoin%
\pgfsetlinewidth{0.803000pt}%
\definecolor{currentstroke}{rgb}{0.690196,0.690196,0.690196}%
\pgfsetstrokecolor{currentstroke}%
\pgfsetstrokeopacity{0.050000}%
\pgfsetdash{}{0pt}%
\pgfpathmoveto{\pgfqpoint{2.574530in}{9.576569in}}%
\pgfpathlineto{\pgfqpoint{2.574530in}{13.800000in}}%
\pgfusepath{stroke}%
\end{pgfscope}%
\begin{pgfscope}%
\pgfsetbuttcap%
\pgfsetroundjoin%
\definecolor{currentfill}{rgb}{0.000000,0.000000,0.000000}%
\pgfsetfillcolor{currentfill}%
\pgfsetlinewidth{0.602250pt}%
\definecolor{currentstroke}{rgb}{0.000000,0.000000,0.000000}%
\pgfsetstrokecolor{currentstroke}%
\pgfsetdash{}{0pt}%
\pgfsys@defobject{currentmarker}{\pgfqpoint{0.000000in}{-0.027778in}}{\pgfqpoint{0.000000in}{0.000000in}}{%
\pgfpathmoveto{\pgfqpoint{0.000000in}{0.000000in}}%
\pgfpathlineto{\pgfqpoint{0.000000in}{-0.027778in}}%
\pgfusepath{stroke,fill}%
}%
\begin{pgfscope}%
\pgfsys@transformshift{2.574530in}{9.576569in}%
\pgfsys@useobject{currentmarker}{}%
\end{pgfscope}%
\end{pgfscope}%
\begin{pgfscope}%
\pgfpathrectangle{\pgfqpoint{0.688192in}{9.576569in}}{\pgfqpoint{11.096108in}{4.223431in}}%
\pgfusepath{clip}%
\pgfsetrectcap%
\pgfsetroundjoin%
\pgfsetlinewidth{0.803000pt}%
\definecolor{currentstroke}{rgb}{0.690196,0.690196,0.690196}%
\pgfsetstrokecolor{currentstroke}%
\pgfsetstrokeopacity{0.050000}%
\pgfsetdash{}{0pt}%
\pgfpathmoveto{\pgfqpoint{2.796452in}{9.576569in}}%
\pgfpathlineto{\pgfqpoint{2.796452in}{13.800000in}}%
\pgfusepath{stroke}%
\end{pgfscope}%
\begin{pgfscope}%
\pgfsetbuttcap%
\pgfsetroundjoin%
\definecolor{currentfill}{rgb}{0.000000,0.000000,0.000000}%
\pgfsetfillcolor{currentfill}%
\pgfsetlinewidth{0.602250pt}%
\definecolor{currentstroke}{rgb}{0.000000,0.000000,0.000000}%
\pgfsetstrokecolor{currentstroke}%
\pgfsetdash{}{0pt}%
\pgfsys@defobject{currentmarker}{\pgfqpoint{0.000000in}{-0.027778in}}{\pgfqpoint{0.000000in}{0.000000in}}{%
\pgfpathmoveto{\pgfqpoint{0.000000in}{0.000000in}}%
\pgfpathlineto{\pgfqpoint{0.000000in}{-0.027778in}}%
\pgfusepath{stroke,fill}%
}%
\begin{pgfscope}%
\pgfsys@transformshift{2.796452in}{9.576569in}%
\pgfsys@useobject{currentmarker}{}%
\end{pgfscope}%
\end{pgfscope}%
\begin{pgfscope}%
\pgfpathrectangle{\pgfqpoint{0.688192in}{9.576569in}}{\pgfqpoint{11.096108in}{4.223431in}}%
\pgfusepath{clip}%
\pgfsetrectcap%
\pgfsetroundjoin%
\pgfsetlinewidth{0.803000pt}%
\definecolor{currentstroke}{rgb}{0.690196,0.690196,0.690196}%
\pgfsetstrokecolor{currentstroke}%
\pgfsetstrokeopacity{0.050000}%
\pgfsetdash{}{0pt}%
\pgfpathmoveto{\pgfqpoint{3.018374in}{9.576569in}}%
\pgfpathlineto{\pgfqpoint{3.018374in}{13.800000in}}%
\pgfusepath{stroke}%
\end{pgfscope}%
\begin{pgfscope}%
\pgfsetbuttcap%
\pgfsetroundjoin%
\definecolor{currentfill}{rgb}{0.000000,0.000000,0.000000}%
\pgfsetfillcolor{currentfill}%
\pgfsetlinewidth{0.602250pt}%
\definecolor{currentstroke}{rgb}{0.000000,0.000000,0.000000}%
\pgfsetstrokecolor{currentstroke}%
\pgfsetdash{}{0pt}%
\pgfsys@defobject{currentmarker}{\pgfqpoint{0.000000in}{-0.027778in}}{\pgfqpoint{0.000000in}{0.000000in}}{%
\pgfpathmoveto{\pgfqpoint{0.000000in}{0.000000in}}%
\pgfpathlineto{\pgfqpoint{0.000000in}{-0.027778in}}%
\pgfusepath{stroke,fill}%
}%
\begin{pgfscope}%
\pgfsys@transformshift{3.018374in}{9.576569in}%
\pgfsys@useobject{currentmarker}{}%
\end{pgfscope}%
\end{pgfscope}%
\begin{pgfscope}%
\pgfpathrectangle{\pgfqpoint{0.688192in}{9.576569in}}{\pgfqpoint{11.096108in}{4.223431in}}%
\pgfusepath{clip}%
\pgfsetrectcap%
\pgfsetroundjoin%
\pgfsetlinewidth{0.803000pt}%
\definecolor{currentstroke}{rgb}{0.690196,0.690196,0.690196}%
\pgfsetstrokecolor{currentstroke}%
\pgfsetstrokeopacity{0.050000}%
\pgfsetdash{}{0pt}%
\pgfpathmoveto{\pgfqpoint{3.240296in}{9.576569in}}%
\pgfpathlineto{\pgfqpoint{3.240296in}{13.800000in}}%
\pgfusepath{stroke}%
\end{pgfscope}%
\begin{pgfscope}%
\pgfsetbuttcap%
\pgfsetroundjoin%
\definecolor{currentfill}{rgb}{0.000000,0.000000,0.000000}%
\pgfsetfillcolor{currentfill}%
\pgfsetlinewidth{0.602250pt}%
\definecolor{currentstroke}{rgb}{0.000000,0.000000,0.000000}%
\pgfsetstrokecolor{currentstroke}%
\pgfsetdash{}{0pt}%
\pgfsys@defobject{currentmarker}{\pgfqpoint{0.000000in}{-0.027778in}}{\pgfqpoint{0.000000in}{0.000000in}}{%
\pgfpathmoveto{\pgfqpoint{0.000000in}{0.000000in}}%
\pgfpathlineto{\pgfqpoint{0.000000in}{-0.027778in}}%
\pgfusepath{stroke,fill}%
}%
\begin{pgfscope}%
\pgfsys@transformshift{3.240296in}{9.576569in}%
\pgfsys@useobject{currentmarker}{}%
\end{pgfscope}%
\end{pgfscope}%
\begin{pgfscope}%
\pgfpathrectangle{\pgfqpoint{0.688192in}{9.576569in}}{\pgfqpoint{11.096108in}{4.223431in}}%
\pgfusepath{clip}%
\pgfsetrectcap%
\pgfsetroundjoin%
\pgfsetlinewidth{0.803000pt}%
\definecolor{currentstroke}{rgb}{0.690196,0.690196,0.690196}%
\pgfsetstrokecolor{currentstroke}%
\pgfsetstrokeopacity{0.050000}%
\pgfsetdash{}{0pt}%
\pgfpathmoveto{\pgfqpoint{3.684141in}{9.576569in}}%
\pgfpathlineto{\pgfqpoint{3.684141in}{13.800000in}}%
\pgfusepath{stroke}%
\end{pgfscope}%
\begin{pgfscope}%
\pgfsetbuttcap%
\pgfsetroundjoin%
\definecolor{currentfill}{rgb}{0.000000,0.000000,0.000000}%
\pgfsetfillcolor{currentfill}%
\pgfsetlinewidth{0.602250pt}%
\definecolor{currentstroke}{rgb}{0.000000,0.000000,0.000000}%
\pgfsetstrokecolor{currentstroke}%
\pgfsetdash{}{0pt}%
\pgfsys@defobject{currentmarker}{\pgfqpoint{0.000000in}{-0.027778in}}{\pgfqpoint{0.000000in}{0.000000in}}{%
\pgfpathmoveto{\pgfqpoint{0.000000in}{0.000000in}}%
\pgfpathlineto{\pgfqpoint{0.000000in}{-0.027778in}}%
\pgfusepath{stroke,fill}%
}%
\begin{pgfscope}%
\pgfsys@transformshift{3.684141in}{9.576569in}%
\pgfsys@useobject{currentmarker}{}%
\end{pgfscope}%
\end{pgfscope}%
\begin{pgfscope}%
\pgfpathrectangle{\pgfqpoint{0.688192in}{9.576569in}}{\pgfqpoint{11.096108in}{4.223431in}}%
\pgfusepath{clip}%
\pgfsetrectcap%
\pgfsetroundjoin%
\pgfsetlinewidth{0.803000pt}%
\definecolor{currentstroke}{rgb}{0.690196,0.690196,0.690196}%
\pgfsetstrokecolor{currentstroke}%
\pgfsetstrokeopacity{0.050000}%
\pgfsetdash{}{0pt}%
\pgfpathmoveto{\pgfqpoint{3.906063in}{9.576569in}}%
\pgfpathlineto{\pgfqpoint{3.906063in}{13.800000in}}%
\pgfusepath{stroke}%
\end{pgfscope}%
\begin{pgfscope}%
\pgfsetbuttcap%
\pgfsetroundjoin%
\definecolor{currentfill}{rgb}{0.000000,0.000000,0.000000}%
\pgfsetfillcolor{currentfill}%
\pgfsetlinewidth{0.602250pt}%
\definecolor{currentstroke}{rgb}{0.000000,0.000000,0.000000}%
\pgfsetstrokecolor{currentstroke}%
\pgfsetdash{}{0pt}%
\pgfsys@defobject{currentmarker}{\pgfqpoint{0.000000in}{-0.027778in}}{\pgfqpoint{0.000000in}{0.000000in}}{%
\pgfpathmoveto{\pgfqpoint{0.000000in}{0.000000in}}%
\pgfpathlineto{\pgfqpoint{0.000000in}{-0.027778in}}%
\pgfusepath{stroke,fill}%
}%
\begin{pgfscope}%
\pgfsys@transformshift{3.906063in}{9.576569in}%
\pgfsys@useobject{currentmarker}{}%
\end{pgfscope}%
\end{pgfscope}%
\begin{pgfscope}%
\pgfpathrectangle{\pgfqpoint{0.688192in}{9.576569in}}{\pgfqpoint{11.096108in}{4.223431in}}%
\pgfusepath{clip}%
\pgfsetrectcap%
\pgfsetroundjoin%
\pgfsetlinewidth{0.803000pt}%
\definecolor{currentstroke}{rgb}{0.690196,0.690196,0.690196}%
\pgfsetstrokecolor{currentstroke}%
\pgfsetstrokeopacity{0.050000}%
\pgfsetdash{}{0pt}%
\pgfpathmoveto{\pgfqpoint{4.127985in}{9.576569in}}%
\pgfpathlineto{\pgfqpoint{4.127985in}{13.800000in}}%
\pgfusepath{stroke}%
\end{pgfscope}%
\begin{pgfscope}%
\pgfsetbuttcap%
\pgfsetroundjoin%
\definecolor{currentfill}{rgb}{0.000000,0.000000,0.000000}%
\pgfsetfillcolor{currentfill}%
\pgfsetlinewidth{0.602250pt}%
\definecolor{currentstroke}{rgb}{0.000000,0.000000,0.000000}%
\pgfsetstrokecolor{currentstroke}%
\pgfsetdash{}{0pt}%
\pgfsys@defobject{currentmarker}{\pgfqpoint{0.000000in}{-0.027778in}}{\pgfqpoint{0.000000in}{0.000000in}}{%
\pgfpathmoveto{\pgfqpoint{0.000000in}{0.000000in}}%
\pgfpathlineto{\pgfqpoint{0.000000in}{-0.027778in}}%
\pgfusepath{stroke,fill}%
}%
\begin{pgfscope}%
\pgfsys@transformshift{4.127985in}{9.576569in}%
\pgfsys@useobject{currentmarker}{}%
\end{pgfscope}%
\end{pgfscope}%
\begin{pgfscope}%
\pgfpathrectangle{\pgfqpoint{0.688192in}{9.576569in}}{\pgfqpoint{11.096108in}{4.223431in}}%
\pgfusepath{clip}%
\pgfsetrectcap%
\pgfsetroundjoin%
\pgfsetlinewidth{0.803000pt}%
\definecolor{currentstroke}{rgb}{0.690196,0.690196,0.690196}%
\pgfsetstrokecolor{currentstroke}%
\pgfsetstrokeopacity{0.050000}%
\pgfsetdash{}{0pt}%
\pgfpathmoveto{\pgfqpoint{4.349907in}{9.576569in}}%
\pgfpathlineto{\pgfqpoint{4.349907in}{13.800000in}}%
\pgfusepath{stroke}%
\end{pgfscope}%
\begin{pgfscope}%
\pgfsetbuttcap%
\pgfsetroundjoin%
\definecolor{currentfill}{rgb}{0.000000,0.000000,0.000000}%
\pgfsetfillcolor{currentfill}%
\pgfsetlinewidth{0.602250pt}%
\definecolor{currentstroke}{rgb}{0.000000,0.000000,0.000000}%
\pgfsetstrokecolor{currentstroke}%
\pgfsetdash{}{0pt}%
\pgfsys@defobject{currentmarker}{\pgfqpoint{0.000000in}{-0.027778in}}{\pgfqpoint{0.000000in}{0.000000in}}{%
\pgfpathmoveto{\pgfqpoint{0.000000in}{0.000000in}}%
\pgfpathlineto{\pgfqpoint{0.000000in}{-0.027778in}}%
\pgfusepath{stroke,fill}%
}%
\begin{pgfscope}%
\pgfsys@transformshift{4.349907in}{9.576569in}%
\pgfsys@useobject{currentmarker}{}%
\end{pgfscope}%
\end{pgfscope}%
\begin{pgfscope}%
\pgfpathrectangle{\pgfqpoint{0.688192in}{9.576569in}}{\pgfqpoint{11.096108in}{4.223431in}}%
\pgfusepath{clip}%
\pgfsetrectcap%
\pgfsetroundjoin%
\pgfsetlinewidth{0.803000pt}%
\definecolor{currentstroke}{rgb}{0.690196,0.690196,0.690196}%
\pgfsetstrokecolor{currentstroke}%
\pgfsetstrokeopacity{0.050000}%
\pgfsetdash{}{0pt}%
\pgfpathmoveto{\pgfqpoint{4.793751in}{9.576569in}}%
\pgfpathlineto{\pgfqpoint{4.793751in}{13.800000in}}%
\pgfusepath{stroke}%
\end{pgfscope}%
\begin{pgfscope}%
\pgfsetbuttcap%
\pgfsetroundjoin%
\definecolor{currentfill}{rgb}{0.000000,0.000000,0.000000}%
\pgfsetfillcolor{currentfill}%
\pgfsetlinewidth{0.602250pt}%
\definecolor{currentstroke}{rgb}{0.000000,0.000000,0.000000}%
\pgfsetstrokecolor{currentstroke}%
\pgfsetdash{}{0pt}%
\pgfsys@defobject{currentmarker}{\pgfqpoint{0.000000in}{-0.027778in}}{\pgfqpoint{0.000000in}{0.000000in}}{%
\pgfpathmoveto{\pgfqpoint{0.000000in}{0.000000in}}%
\pgfpathlineto{\pgfqpoint{0.000000in}{-0.027778in}}%
\pgfusepath{stroke,fill}%
}%
\begin{pgfscope}%
\pgfsys@transformshift{4.793751in}{9.576569in}%
\pgfsys@useobject{currentmarker}{}%
\end{pgfscope}%
\end{pgfscope}%
\begin{pgfscope}%
\pgfpathrectangle{\pgfqpoint{0.688192in}{9.576569in}}{\pgfqpoint{11.096108in}{4.223431in}}%
\pgfusepath{clip}%
\pgfsetrectcap%
\pgfsetroundjoin%
\pgfsetlinewidth{0.803000pt}%
\definecolor{currentstroke}{rgb}{0.690196,0.690196,0.690196}%
\pgfsetstrokecolor{currentstroke}%
\pgfsetstrokeopacity{0.050000}%
\pgfsetdash{}{0pt}%
\pgfpathmoveto{\pgfqpoint{5.015674in}{9.576569in}}%
\pgfpathlineto{\pgfqpoint{5.015674in}{13.800000in}}%
\pgfusepath{stroke}%
\end{pgfscope}%
\begin{pgfscope}%
\pgfsetbuttcap%
\pgfsetroundjoin%
\definecolor{currentfill}{rgb}{0.000000,0.000000,0.000000}%
\pgfsetfillcolor{currentfill}%
\pgfsetlinewidth{0.602250pt}%
\definecolor{currentstroke}{rgb}{0.000000,0.000000,0.000000}%
\pgfsetstrokecolor{currentstroke}%
\pgfsetdash{}{0pt}%
\pgfsys@defobject{currentmarker}{\pgfqpoint{0.000000in}{-0.027778in}}{\pgfqpoint{0.000000in}{0.000000in}}{%
\pgfpathmoveto{\pgfqpoint{0.000000in}{0.000000in}}%
\pgfpathlineto{\pgfqpoint{0.000000in}{-0.027778in}}%
\pgfusepath{stroke,fill}%
}%
\begin{pgfscope}%
\pgfsys@transformshift{5.015674in}{9.576569in}%
\pgfsys@useobject{currentmarker}{}%
\end{pgfscope}%
\end{pgfscope}%
\begin{pgfscope}%
\pgfpathrectangle{\pgfqpoint{0.688192in}{9.576569in}}{\pgfqpoint{11.096108in}{4.223431in}}%
\pgfusepath{clip}%
\pgfsetrectcap%
\pgfsetroundjoin%
\pgfsetlinewidth{0.803000pt}%
\definecolor{currentstroke}{rgb}{0.690196,0.690196,0.690196}%
\pgfsetstrokecolor{currentstroke}%
\pgfsetstrokeopacity{0.050000}%
\pgfsetdash{}{0pt}%
\pgfpathmoveto{\pgfqpoint{5.237596in}{9.576569in}}%
\pgfpathlineto{\pgfqpoint{5.237596in}{13.800000in}}%
\pgfusepath{stroke}%
\end{pgfscope}%
\begin{pgfscope}%
\pgfsetbuttcap%
\pgfsetroundjoin%
\definecolor{currentfill}{rgb}{0.000000,0.000000,0.000000}%
\pgfsetfillcolor{currentfill}%
\pgfsetlinewidth{0.602250pt}%
\definecolor{currentstroke}{rgb}{0.000000,0.000000,0.000000}%
\pgfsetstrokecolor{currentstroke}%
\pgfsetdash{}{0pt}%
\pgfsys@defobject{currentmarker}{\pgfqpoint{0.000000in}{-0.027778in}}{\pgfqpoint{0.000000in}{0.000000in}}{%
\pgfpathmoveto{\pgfqpoint{0.000000in}{0.000000in}}%
\pgfpathlineto{\pgfqpoint{0.000000in}{-0.027778in}}%
\pgfusepath{stroke,fill}%
}%
\begin{pgfscope}%
\pgfsys@transformshift{5.237596in}{9.576569in}%
\pgfsys@useobject{currentmarker}{}%
\end{pgfscope}%
\end{pgfscope}%
\begin{pgfscope}%
\pgfpathrectangle{\pgfqpoint{0.688192in}{9.576569in}}{\pgfqpoint{11.096108in}{4.223431in}}%
\pgfusepath{clip}%
\pgfsetrectcap%
\pgfsetroundjoin%
\pgfsetlinewidth{0.803000pt}%
\definecolor{currentstroke}{rgb}{0.690196,0.690196,0.690196}%
\pgfsetstrokecolor{currentstroke}%
\pgfsetstrokeopacity{0.050000}%
\pgfsetdash{}{0pt}%
\pgfpathmoveto{\pgfqpoint{5.459518in}{9.576569in}}%
\pgfpathlineto{\pgfqpoint{5.459518in}{13.800000in}}%
\pgfusepath{stroke}%
\end{pgfscope}%
\begin{pgfscope}%
\pgfsetbuttcap%
\pgfsetroundjoin%
\definecolor{currentfill}{rgb}{0.000000,0.000000,0.000000}%
\pgfsetfillcolor{currentfill}%
\pgfsetlinewidth{0.602250pt}%
\definecolor{currentstroke}{rgb}{0.000000,0.000000,0.000000}%
\pgfsetstrokecolor{currentstroke}%
\pgfsetdash{}{0pt}%
\pgfsys@defobject{currentmarker}{\pgfqpoint{0.000000in}{-0.027778in}}{\pgfqpoint{0.000000in}{0.000000in}}{%
\pgfpathmoveto{\pgfqpoint{0.000000in}{0.000000in}}%
\pgfpathlineto{\pgfqpoint{0.000000in}{-0.027778in}}%
\pgfusepath{stroke,fill}%
}%
\begin{pgfscope}%
\pgfsys@transformshift{5.459518in}{9.576569in}%
\pgfsys@useobject{currentmarker}{}%
\end{pgfscope}%
\end{pgfscope}%
\begin{pgfscope}%
\pgfpathrectangle{\pgfqpoint{0.688192in}{9.576569in}}{\pgfqpoint{11.096108in}{4.223431in}}%
\pgfusepath{clip}%
\pgfsetrectcap%
\pgfsetroundjoin%
\pgfsetlinewidth{0.803000pt}%
\definecolor{currentstroke}{rgb}{0.690196,0.690196,0.690196}%
\pgfsetstrokecolor{currentstroke}%
\pgfsetstrokeopacity{0.050000}%
\pgfsetdash{}{0pt}%
\pgfpathmoveto{\pgfqpoint{5.903362in}{9.576569in}}%
\pgfpathlineto{\pgfqpoint{5.903362in}{13.800000in}}%
\pgfusepath{stroke}%
\end{pgfscope}%
\begin{pgfscope}%
\pgfsetbuttcap%
\pgfsetroundjoin%
\definecolor{currentfill}{rgb}{0.000000,0.000000,0.000000}%
\pgfsetfillcolor{currentfill}%
\pgfsetlinewidth{0.602250pt}%
\definecolor{currentstroke}{rgb}{0.000000,0.000000,0.000000}%
\pgfsetstrokecolor{currentstroke}%
\pgfsetdash{}{0pt}%
\pgfsys@defobject{currentmarker}{\pgfqpoint{0.000000in}{-0.027778in}}{\pgfqpoint{0.000000in}{0.000000in}}{%
\pgfpathmoveto{\pgfqpoint{0.000000in}{0.000000in}}%
\pgfpathlineto{\pgfqpoint{0.000000in}{-0.027778in}}%
\pgfusepath{stroke,fill}%
}%
\begin{pgfscope}%
\pgfsys@transformshift{5.903362in}{9.576569in}%
\pgfsys@useobject{currentmarker}{}%
\end{pgfscope}%
\end{pgfscope}%
\begin{pgfscope}%
\pgfpathrectangle{\pgfqpoint{0.688192in}{9.576569in}}{\pgfqpoint{11.096108in}{4.223431in}}%
\pgfusepath{clip}%
\pgfsetrectcap%
\pgfsetroundjoin%
\pgfsetlinewidth{0.803000pt}%
\definecolor{currentstroke}{rgb}{0.690196,0.690196,0.690196}%
\pgfsetstrokecolor{currentstroke}%
\pgfsetstrokeopacity{0.050000}%
\pgfsetdash{}{0pt}%
\pgfpathmoveto{\pgfqpoint{6.125284in}{9.576569in}}%
\pgfpathlineto{\pgfqpoint{6.125284in}{13.800000in}}%
\pgfusepath{stroke}%
\end{pgfscope}%
\begin{pgfscope}%
\pgfsetbuttcap%
\pgfsetroundjoin%
\definecolor{currentfill}{rgb}{0.000000,0.000000,0.000000}%
\pgfsetfillcolor{currentfill}%
\pgfsetlinewidth{0.602250pt}%
\definecolor{currentstroke}{rgb}{0.000000,0.000000,0.000000}%
\pgfsetstrokecolor{currentstroke}%
\pgfsetdash{}{0pt}%
\pgfsys@defobject{currentmarker}{\pgfqpoint{0.000000in}{-0.027778in}}{\pgfqpoint{0.000000in}{0.000000in}}{%
\pgfpathmoveto{\pgfqpoint{0.000000in}{0.000000in}}%
\pgfpathlineto{\pgfqpoint{0.000000in}{-0.027778in}}%
\pgfusepath{stroke,fill}%
}%
\begin{pgfscope}%
\pgfsys@transformshift{6.125284in}{9.576569in}%
\pgfsys@useobject{currentmarker}{}%
\end{pgfscope}%
\end{pgfscope}%
\begin{pgfscope}%
\pgfpathrectangle{\pgfqpoint{0.688192in}{9.576569in}}{\pgfqpoint{11.096108in}{4.223431in}}%
\pgfusepath{clip}%
\pgfsetrectcap%
\pgfsetroundjoin%
\pgfsetlinewidth{0.803000pt}%
\definecolor{currentstroke}{rgb}{0.690196,0.690196,0.690196}%
\pgfsetstrokecolor{currentstroke}%
\pgfsetstrokeopacity{0.050000}%
\pgfsetdash{}{0pt}%
\pgfpathmoveto{\pgfqpoint{6.347207in}{9.576569in}}%
\pgfpathlineto{\pgfqpoint{6.347207in}{13.800000in}}%
\pgfusepath{stroke}%
\end{pgfscope}%
\begin{pgfscope}%
\pgfsetbuttcap%
\pgfsetroundjoin%
\definecolor{currentfill}{rgb}{0.000000,0.000000,0.000000}%
\pgfsetfillcolor{currentfill}%
\pgfsetlinewidth{0.602250pt}%
\definecolor{currentstroke}{rgb}{0.000000,0.000000,0.000000}%
\pgfsetstrokecolor{currentstroke}%
\pgfsetdash{}{0pt}%
\pgfsys@defobject{currentmarker}{\pgfqpoint{0.000000in}{-0.027778in}}{\pgfqpoint{0.000000in}{0.000000in}}{%
\pgfpathmoveto{\pgfqpoint{0.000000in}{0.000000in}}%
\pgfpathlineto{\pgfqpoint{0.000000in}{-0.027778in}}%
\pgfusepath{stroke,fill}%
}%
\begin{pgfscope}%
\pgfsys@transformshift{6.347207in}{9.576569in}%
\pgfsys@useobject{currentmarker}{}%
\end{pgfscope}%
\end{pgfscope}%
\begin{pgfscope}%
\pgfpathrectangle{\pgfqpoint{0.688192in}{9.576569in}}{\pgfqpoint{11.096108in}{4.223431in}}%
\pgfusepath{clip}%
\pgfsetrectcap%
\pgfsetroundjoin%
\pgfsetlinewidth{0.803000pt}%
\definecolor{currentstroke}{rgb}{0.690196,0.690196,0.690196}%
\pgfsetstrokecolor{currentstroke}%
\pgfsetstrokeopacity{0.050000}%
\pgfsetdash{}{0pt}%
\pgfpathmoveto{\pgfqpoint{6.569129in}{9.576569in}}%
\pgfpathlineto{\pgfqpoint{6.569129in}{13.800000in}}%
\pgfusepath{stroke}%
\end{pgfscope}%
\begin{pgfscope}%
\pgfsetbuttcap%
\pgfsetroundjoin%
\definecolor{currentfill}{rgb}{0.000000,0.000000,0.000000}%
\pgfsetfillcolor{currentfill}%
\pgfsetlinewidth{0.602250pt}%
\definecolor{currentstroke}{rgb}{0.000000,0.000000,0.000000}%
\pgfsetstrokecolor{currentstroke}%
\pgfsetdash{}{0pt}%
\pgfsys@defobject{currentmarker}{\pgfqpoint{0.000000in}{-0.027778in}}{\pgfqpoint{0.000000in}{0.000000in}}{%
\pgfpathmoveto{\pgfqpoint{0.000000in}{0.000000in}}%
\pgfpathlineto{\pgfqpoint{0.000000in}{-0.027778in}}%
\pgfusepath{stroke,fill}%
}%
\begin{pgfscope}%
\pgfsys@transformshift{6.569129in}{9.576569in}%
\pgfsys@useobject{currentmarker}{}%
\end{pgfscope}%
\end{pgfscope}%
\begin{pgfscope}%
\pgfpathrectangle{\pgfqpoint{0.688192in}{9.576569in}}{\pgfqpoint{11.096108in}{4.223431in}}%
\pgfusepath{clip}%
\pgfsetrectcap%
\pgfsetroundjoin%
\pgfsetlinewidth{0.803000pt}%
\definecolor{currentstroke}{rgb}{0.690196,0.690196,0.690196}%
\pgfsetstrokecolor{currentstroke}%
\pgfsetstrokeopacity{0.050000}%
\pgfsetdash{}{0pt}%
\pgfpathmoveto{\pgfqpoint{7.012973in}{9.576569in}}%
\pgfpathlineto{\pgfqpoint{7.012973in}{13.800000in}}%
\pgfusepath{stroke}%
\end{pgfscope}%
\begin{pgfscope}%
\pgfsetbuttcap%
\pgfsetroundjoin%
\definecolor{currentfill}{rgb}{0.000000,0.000000,0.000000}%
\pgfsetfillcolor{currentfill}%
\pgfsetlinewidth{0.602250pt}%
\definecolor{currentstroke}{rgb}{0.000000,0.000000,0.000000}%
\pgfsetstrokecolor{currentstroke}%
\pgfsetdash{}{0pt}%
\pgfsys@defobject{currentmarker}{\pgfqpoint{0.000000in}{-0.027778in}}{\pgfqpoint{0.000000in}{0.000000in}}{%
\pgfpathmoveto{\pgfqpoint{0.000000in}{0.000000in}}%
\pgfpathlineto{\pgfqpoint{0.000000in}{-0.027778in}}%
\pgfusepath{stroke,fill}%
}%
\begin{pgfscope}%
\pgfsys@transformshift{7.012973in}{9.576569in}%
\pgfsys@useobject{currentmarker}{}%
\end{pgfscope}%
\end{pgfscope}%
\begin{pgfscope}%
\pgfpathrectangle{\pgfqpoint{0.688192in}{9.576569in}}{\pgfqpoint{11.096108in}{4.223431in}}%
\pgfusepath{clip}%
\pgfsetrectcap%
\pgfsetroundjoin%
\pgfsetlinewidth{0.803000pt}%
\definecolor{currentstroke}{rgb}{0.690196,0.690196,0.690196}%
\pgfsetstrokecolor{currentstroke}%
\pgfsetstrokeopacity{0.050000}%
\pgfsetdash{}{0pt}%
\pgfpathmoveto{\pgfqpoint{7.234895in}{9.576569in}}%
\pgfpathlineto{\pgfqpoint{7.234895in}{13.800000in}}%
\pgfusepath{stroke}%
\end{pgfscope}%
\begin{pgfscope}%
\pgfsetbuttcap%
\pgfsetroundjoin%
\definecolor{currentfill}{rgb}{0.000000,0.000000,0.000000}%
\pgfsetfillcolor{currentfill}%
\pgfsetlinewidth{0.602250pt}%
\definecolor{currentstroke}{rgb}{0.000000,0.000000,0.000000}%
\pgfsetstrokecolor{currentstroke}%
\pgfsetdash{}{0pt}%
\pgfsys@defobject{currentmarker}{\pgfqpoint{0.000000in}{-0.027778in}}{\pgfqpoint{0.000000in}{0.000000in}}{%
\pgfpathmoveto{\pgfqpoint{0.000000in}{0.000000in}}%
\pgfpathlineto{\pgfqpoint{0.000000in}{-0.027778in}}%
\pgfusepath{stroke,fill}%
}%
\begin{pgfscope}%
\pgfsys@transformshift{7.234895in}{9.576569in}%
\pgfsys@useobject{currentmarker}{}%
\end{pgfscope}%
\end{pgfscope}%
\begin{pgfscope}%
\pgfpathrectangle{\pgfqpoint{0.688192in}{9.576569in}}{\pgfqpoint{11.096108in}{4.223431in}}%
\pgfusepath{clip}%
\pgfsetrectcap%
\pgfsetroundjoin%
\pgfsetlinewidth{0.803000pt}%
\definecolor{currentstroke}{rgb}{0.690196,0.690196,0.690196}%
\pgfsetstrokecolor{currentstroke}%
\pgfsetstrokeopacity{0.050000}%
\pgfsetdash{}{0pt}%
\pgfpathmoveto{\pgfqpoint{7.456817in}{9.576569in}}%
\pgfpathlineto{\pgfqpoint{7.456817in}{13.800000in}}%
\pgfusepath{stroke}%
\end{pgfscope}%
\begin{pgfscope}%
\pgfsetbuttcap%
\pgfsetroundjoin%
\definecolor{currentfill}{rgb}{0.000000,0.000000,0.000000}%
\pgfsetfillcolor{currentfill}%
\pgfsetlinewidth{0.602250pt}%
\definecolor{currentstroke}{rgb}{0.000000,0.000000,0.000000}%
\pgfsetstrokecolor{currentstroke}%
\pgfsetdash{}{0pt}%
\pgfsys@defobject{currentmarker}{\pgfqpoint{0.000000in}{-0.027778in}}{\pgfqpoint{0.000000in}{0.000000in}}{%
\pgfpathmoveto{\pgfqpoint{0.000000in}{0.000000in}}%
\pgfpathlineto{\pgfqpoint{0.000000in}{-0.027778in}}%
\pgfusepath{stroke,fill}%
}%
\begin{pgfscope}%
\pgfsys@transformshift{7.456817in}{9.576569in}%
\pgfsys@useobject{currentmarker}{}%
\end{pgfscope}%
\end{pgfscope}%
\begin{pgfscope}%
\pgfpathrectangle{\pgfqpoint{0.688192in}{9.576569in}}{\pgfqpoint{11.096108in}{4.223431in}}%
\pgfusepath{clip}%
\pgfsetrectcap%
\pgfsetroundjoin%
\pgfsetlinewidth{0.803000pt}%
\definecolor{currentstroke}{rgb}{0.690196,0.690196,0.690196}%
\pgfsetstrokecolor{currentstroke}%
\pgfsetstrokeopacity{0.050000}%
\pgfsetdash{}{0pt}%
\pgfpathmoveto{\pgfqpoint{7.678740in}{9.576569in}}%
\pgfpathlineto{\pgfqpoint{7.678740in}{13.800000in}}%
\pgfusepath{stroke}%
\end{pgfscope}%
\begin{pgfscope}%
\pgfsetbuttcap%
\pgfsetroundjoin%
\definecolor{currentfill}{rgb}{0.000000,0.000000,0.000000}%
\pgfsetfillcolor{currentfill}%
\pgfsetlinewidth{0.602250pt}%
\definecolor{currentstroke}{rgb}{0.000000,0.000000,0.000000}%
\pgfsetstrokecolor{currentstroke}%
\pgfsetdash{}{0pt}%
\pgfsys@defobject{currentmarker}{\pgfqpoint{0.000000in}{-0.027778in}}{\pgfqpoint{0.000000in}{0.000000in}}{%
\pgfpathmoveto{\pgfqpoint{0.000000in}{0.000000in}}%
\pgfpathlineto{\pgfqpoint{0.000000in}{-0.027778in}}%
\pgfusepath{stroke,fill}%
}%
\begin{pgfscope}%
\pgfsys@transformshift{7.678740in}{9.576569in}%
\pgfsys@useobject{currentmarker}{}%
\end{pgfscope}%
\end{pgfscope}%
\begin{pgfscope}%
\pgfpathrectangle{\pgfqpoint{0.688192in}{9.576569in}}{\pgfqpoint{11.096108in}{4.223431in}}%
\pgfusepath{clip}%
\pgfsetrectcap%
\pgfsetroundjoin%
\pgfsetlinewidth{0.803000pt}%
\definecolor{currentstroke}{rgb}{0.690196,0.690196,0.690196}%
\pgfsetstrokecolor{currentstroke}%
\pgfsetstrokeopacity{0.050000}%
\pgfsetdash{}{0pt}%
\pgfpathmoveto{\pgfqpoint{8.122584in}{9.576569in}}%
\pgfpathlineto{\pgfqpoint{8.122584in}{13.800000in}}%
\pgfusepath{stroke}%
\end{pgfscope}%
\begin{pgfscope}%
\pgfsetbuttcap%
\pgfsetroundjoin%
\definecolor{currentfill}{rgb}{0.000000,0.000000,0.000000}%
\pgfsetfillcolor{currentfill}%
\pgfsetlinewidth{0.602250pt}%
\definecolor{currentstroke}{rgb}{0.000000,0.000000,0.000000}%
\pgfsetstrokecolor{currentstroke}%
\pgfsetdash{}{0pt}%
\pgfsys@defobject{currentmarker}{\pgfqpoint{0.000000in}{-0.027778in}}{\pgfqpoint{0.000000in}{0.000000in}}{%
\pgfpathmoveto{\pgfqpoint{0.000000in}{0.000000in}}%
\pgfpathlineto{\pgfqpoint{0.000000in}{-0.027778in}}%
\pgfusepath{stroke,fill}%
}%
\begin{pgfscope}%
\pgfsys@transformshift{8.122584in}{9.576569in}%
\pgfsys@useobject{currentmarker}{}%
\end{pgfscope}%
\end{pgfscope}%
\begin{pgfscope}%
\pgfpathrectangle{\pgfqpoint{0.688192in}{9.576569in}}{\pgfqpoint{11.096108in}{4.223431in}}%
\pgfusepath{clip}%
\pgfsetrectcap%
\pgfsetroundjoin%
\pgfsetlinewidth{0.803000pt}%
\definecolor{currentstroke}{rgb}{0.690196,0.690196,0.690196}%
\pgfsetstrokecolor{currentstroke}%
\pgfsetstrokeopacity{0.050000}%
\pgfsetdash{}{0pt}%
\pgfpathmoveto{\pgfqpoint{8.344506in}{9.576569in}}%
\pgfpathlineto{\pgfqpoint{8.344506in}{13.800000in}}%
\pgfusepath{stroke}%
\end{pgfscope}%
\begin{pgfscope}%
\pgfsetbuttcap%
\pgfsetroundjoin%
\definecolor{currentfill}{rgb}{0.000000,0.000000,0.000000}%
\pgfsetfillcolor{currentfill}%
\pgfsetlinewidth{0.602250pt}%
\definecolor{currentstroke}{rgb}{0.000000,0.000000,0.000000}%
\pgfsetstrokecolor{currentstroke}%
\pgfsetdash{}{0pt}%
\pgfsys@defobject{currentmarker}{\pgfqpoint{0.000000in}{-0.027778in}}{\pgfqpoint{0.000000in}{0.000000in}}{%
\pgfpathmoveto{\pgfqpoint{0.000000in}{0.000000in}}%
\pgfpathlineto{\pgfqpoint{0.000000in}{-0.027778in}}%
\pgfusepath{stroke,fill}%
}%
\begin{pgfscope}%
\pgfsys@transformshift{8.344506in}{9.576569in}%
\pgfsys@useobject{currentmarker}{}%
\end{pgfscope}%
\end{pgfscope}%
\begin{pgfscope}%
\pgfpathrectangle{\pgfqpoint{0.688192in}{9.576569in}}{\pgfqpoint{11.096108in}{4.223431in}}%
\pgfusepath{clip}%
\pgfsetrectcap%
\pgfsetroundjoin%
\pgfsetlinewidth{0.803000pt}%
\definecolor{currentstroke}{rgb}{0.690196,0.690196,0.690196}%
\pgfsetstrokecolor{currentstroke}%
\pgfsetstrokeopacity{0.050000}%
\pgfsetdash{}{0pt}%
\pgfpathmoveto{\pgfqpoint{8.566428in}{9.576569in}}%
\pgfpathlineto{\pgfqpoint{8.566428in}{13.800000in}}%
\pgfusepath{stroke}%
\end{pgfscope}%
\begin{pgfscope}%
\pgfsetbuttcap%
\pgfsetroundjoin%
\definecolor{currentfill}{rgb}{0.000000,0.000000,0.000000}%
\pgfsetfillcolor{currentfill}%
\pgfsetlinewidth{0.602250pt}%
\definecolor{currentstroke}{rgb}{0.000000,0.000000,0.000000}%
\pgfsetstrokecolor{currentstroke}%
\pgfsetdash{}{0pt}%
\pgfsys@defobject{currentmarker}{\pgfqpoint{0.000000in}{-0.027778in}}{\pgfqpoint{0.000000in}{0.000000in}}{%
\pgfpathmoveto{\pgfqpoint{0.000000in}{0.000000in}}%
\pgfpathlineto{\pgfqpoint{0.000000in}{-0.027778in}}%
\pgfusepath{stroke,fill}%
}%
\begin{pgfscope}%
\pgfsys@transformshift{8.566428in}{9.576569in}%
\pgfsys@useobject{currentmarker}{}%
\end{pgfscope}%
\end{pgfscope}%
\begin{pgfscope}%
\pgfpathrectangle{\pgfqpoint{0.688192in}{9.576569in}}{\pgfqpoint{11.096108in}{4.223431in}}%
\pgfusepath{clip}%
\pgfsetrectcap%
\pgfsetroundjoin%
\pgfsetlinewidth{0.803000pt}%
\definecolor{currentstroke}{rgb}{0.690196,0.690196,0.690196}%
\pgfsetstrokecolor{currentstroke}%
\pgfsetstrokeopacity{0.050000}%
\pgfsetdash{}{0pt}%
\pgfpathmoveto{\pgfqpoint{8.788350in}{9.576569in}}%
\pgfpathlineto{\pgfqpoint{8.788350in}{13.800000in}}%
\pgfusepath{stroke}%
\end{pgfscope}%
\begin{pgfscope}%
\pgfsetbuttcap%
\pgfsetroundjoin%
\definecolor{currentfill}{rgb}{0.000000,0.000000,0.000000}%
\pgfsetfillcolor{currentfill}%
\pgfsetlinewidth{0.602250pt}%
\definecolor{currentstroke}{rgb}{0.000000,0.000000,0.000000}%
\pgfsetstrokecolor{currentstroke}%
\pgfsetdash{}{0pt}%
\pgfsys@defobject{currentmarker}{\pgfqpoint{0.000000in}{-0.027778in}}{\pgfqpoint{0.000000in}{0.000000in}}{%
\pgfpathmoveto{\pgfqpoint{0.000000in}{0.000000in}}%
\pgfpathlineto{\pgfqpoint{0.000000in}{-0.027778in}}%
\pgfusepath{stroke,fill}%
}%
\begin{pgfscope}%
\pgfsys@transformshift{8.788350in}{9.576569in}%
\pgfsys@useobject{currentmarker}{}%
\end{pgfscope}%
\end{pgfscope}%
\begin{pgfscope}%
\pgfpathrectangle{\pgfqpoint{0.688192in}{9.576569in}}{\pgfqpoint{11.096108in}{4.223431in}}%
\pgfusepath{clip}%
\pgfsetrectcap%
\pgfsetroundjoin%
\pgfsetlinewidth{0.803000pt}%
\definecolor{currentstroke}{rgb}{0.690196,0.690196,0.690196}%
\pgfsetstrokecolor{currentstroke}%
\pgfsetstrokeopacity{0.050000}%
\pgfsetdash{}{0pt}%
\pgfpathmoveto{\pgfqpoint{9.232195in}{9.576569in}}%
\pgfpathlineto{\pgfqpoint{9.232195in}{13.800000in}}%
\pgfusepath{stroke}%
\end{pgfscope}%
\begin{pgfscope}%
\pgfsetbuttcap%
\pgfsetroundjoin%
\definecolor{currentfill}{rgb}{0.000000,0.000000,0.000000}%
\pgfsetfillcolor{currentfill}%
\pgfsetlinewidth{0.602250pt}%
\definecolor{currentstroke}{rgb}{0.000000,0.000000,0.000000}%
\pgfsetstrokecolor{currentstroke}%
\pgfsetdash{}{0pt}%
\pgfsys@defobject{currentmarker}{\pgfqpoint{0.000000in}{-0.027778in}}{\pgfqpoint{0.000000in}{0.000000in}}{%
\pgfpathmoveto{\pgfqpoint{0.000000in}{0.000000in}}%
\pgfpathlineto{\pgfqpoint{0.000000in}{-0.027778in}}%
\pgfusepath{stroke,fill}%
}%
\begin{pgfscope}%
\pgfsys@transformshift{9.232195in}{9.576569in}%
\pgfsys@useobject{currentmarker}{}%
\end{pgfscope}%
\end{pgfscope}%
\begin{pgfscope}%
\pgfpathrectangle{\pgfqpoint{0.688192in}{9.576569in}}{\pgfqpoint{11.096108in}{4.223431in}}%
\pgfusepath{clip}%
\pgfsetrectcap%
\pgfsetroundjoin%
\pgfsetlinewidth{0.803000pt}%
\definecolor{currentstroke}{rgb}{0.690196,0.690196,0.690196}%
\pgfsetstrokecolor{currentstroke}%
\pgfsetstrokeopacity{0.050000}%
\pgfsetdash{}{0pt}%
\pgfpathmoveto{\pgfqpoint{9.454117in}{9.576569in}}%
\pgfpathlineto{\pgfqpoint{9.454117in}{13.800000in}}%
\pgfusepath{stroke}%
\end{pgfscope}%
\begin{pgfscope}%
\pgfsetbuttcap%
\pgfsetroundjoin%
\definecolor{currentfill}{rgb}{0.000000,0.000000,0.000000}%
\pgfsetfillcolor{currentfill}%
\pgfsetlinewidth{0.602250pt}%
\definecolor{currentstroke}{rgb}{0.000000,0.000000,0.000000}%
\pgfsetstrokecolor{currentstroke}%
\pgfsetdash{}{0pt}%
\pgfsys@defobject{currentmarker}{\pgfqpoint{0.000000in}{-0.027778in}}{\pgfqpoint{0.000000in}{0.000000in}}{%
\pgfpathmoveto{\pgfqpoint{0.000000in}{0.000000in}}%
\pgfpathlineto{\pgfqpoint{0.000000in}{-0.027778in}}%
\pgfusepath{stroke,fill}%
}%
\begin{pgfscope}%
\pgfsys@transformshift{9.454117in}{9.576569in}%
\pgfsys@useobject{currentmarker}{}%
\end{pgfscope}%
\end{pgfscope}%
\begin{pgfscope}%
\pgfpathrectangle{\pgfqpoint{0.688192in}{9.576569in}}{\pgfqpoint{11.096108in}{4.223431in}}%
\pgfusepath{clip}%
\pgfsetrectcap%
\pgfsetroundjoin%
\pgfsetlinewidth{0.803000pt}%
\definecolor{currentstroke}{rgb}{0.690196,0.690196,0.690196}%
\pgfsetstrokecolor{currentstroke}%
\pgfsetstrokeopacity{0.050000}%
\pgfsetdash{}{0pt}%
\pgfpathmoveto{\pgfqpoint{9.676039in}{9.576569in}}%
\pgfpathlineto{\pgfqpoint{9.676039in}{13.800000in}}%
\pgfusepath{stroke}%
\end{pgfscope}%
\begin{pgfscope}%
\pgfsetbuttcap%
\pgfsetroundjoin%
\definecolor{currentfill}{rgb}{0.000000,0.000000,0.000000}%
\pgfsetfillcolor{currentfill}%
\pgfsetlinewidth{0.602250pt}%
\definecolor{currentstroke}{rgb}{0.000000,0.000000,0.000000}%
\pgfsetstrokecolor{currentstroke}%
\pgfsetdash{}{0pt}%
\pgfsys@defobject{currentmarker}{\pgfqpoint{0.000000in}{-0.027778in}}{\pgfqpoint{0.000000in}{0.000000in}}{%
\pgfpathmoveto{\pgfqpoint{0.000000in}{0.000000in}}%
\pgfpathlineto{\pgfqpoint{0.000000in}{-0.027778in}}%
\pgfusepath{stroke,fill}%
}%
\begin{pgfscope}%
\pgfsys@transformshift{9.676039in}{9.576569in}%
\pgfsys@useobject{currentmarker}{}%
\end{pgfscope}%
\end{pgfscope}%
\begin{pgfscope}%
\pgfpathrectangle{\pgfqpoint{0.688192in}{9.576569in}}{\pgfqpoint{11.096108in}{4.223431in}}%
\pgfusepath{clip}%
\pgfsetrectcap%
\pgfsetroundjoin%
\pgfsetlinewidth{0.803000pt}%
\definecolor{currentstroke}{rgb}{0.690196,0.690196,0.690196}%
\pgfsetstrokecolor{currentstroke}%
\pgfsetstrokeopacity{0.050000}%
\pgfsetdash{}{0pt}%
\pgfpathmoveto{\pgfqpoint{9.897961in}{9.576569in}}%
\pgfpathlineto{\pgfqpoint{9.897961in}{13.800000in}}%
\pgfusepath{stroke}%
\end{pgfscope}%
\begin{pgfscope}%
\pgfsetbuttcap%
\pgfsetroundjoin%
\definecolor{currentfill}{rgb}{0.000000,0.000000,0.000000}%
\pgfsetfillcolor{currentfill}%
\pgfsetlinewidth{0.602250pt}%
\definecolor{currentstroke}{rgb}{0.000000,0.000000,0.000000}%
\pgfsetstrokecolor{currentstroke}%
\pgfsetdash{}{0pt}%
\pgfsys@defobject{currentmarker}{\pgfqpoint{0.000000in}{-0.027778in}}{\pgfqpoint{0.000000in}{0.000000in}}{%
\pgfpathmoveto{\pgfqpoint{0.000000in}{0.000000in}}%
\pgfpathlineto{\pgfqpoint{0.000000in}{-0.027778in}}%
\pgfusepath{stroke,fill}%
}%
\begin{pgfscope}%
\pgfsys@transformshift{9.897961in}{9.576569in}%
\pgfsys@useobject{currentmarker}{}%
\end{pgfscope}%
\end{pgfscope}%
\begin{pgfscope}%
\pgfpathrectangle{\pgfqpoint{0.688192in}{9.576569in}}{\pgfqpoint{11.096108in}{4.223431in}}%
\pgfusepath{clip}%
\pgfsetrectcap%
\pgfsetroundjoin%
\pgfsetlinewidth{0.803000pt}%
\definecolor{currentstroke}{rgb}{0.690196,0.690196,0.690196}%
\pgfsetstrokecolor{currentstroke}%
\pgfsetstrokeopacity{0.050000}%
\pgfsetdash{}{0pt}%
\pgfpathmoveto{\pgfqpoint{10.341805in}{9.576569in}}%
\pgfpathlineto{\pgfqpoint{10.341805in}{13.800000in}}%
\pgfusepath{stroke}%
\end{pgfscope}%
\begin{pgfscope}%
\pgfsetbuttcap%
\pgfsetroundjoin%
\definecolor{currentfill}{rgb}{0.000000,0.000000,0.000000}%
\pgfsetfillcolor{currentfill}%
\pgfsetlinewidth{0.602250pt}%
\definecolor{currentstroke}{rgb}{0.000000,0.000000,0.000000}%
\pgfsetstrokecolor{currentstroke}%
\pgfsetdash{}{0pt}%
\pgfsys@defobject{currentmarker}{\pgfqpoint{0.000000in}{-0.027778in}}{\pgfqpoint{0.000000in}{0.000000in}}{%
\pgfpathmoveto{\pgfqpoint{0.000000in}{0.000000in}}%
\pgfpathlineto{\pgfqpoint{0.000000in}{-0.027778in}}%
\pgfusepath{stroke,fill}%
}%
\begin{pgfscope}%
\pgfsys@transformshift{10.341805in}{9.576569in}%
\pgfsys@useobject{currentmarker}{}%
\end{pgfscope}%
\end{pgfscope}%
\begin{pgfscope}%
\pgfpathrectangle{\pgfqpoint{0.688192in}{9.576569in}}{\pgfqpoint{11.096108in}{4.223431in}}%
\pgfusepath{clip}%
\pgfsetrectcap%
\pgfsetroundjoin%
\pgfsetlinewidth{0.803000pt}%
\definecolor{currentstroke}{rgb}{0.690196,0.690196,0.690196}%
\pgfsetstrokecolor{currentstroke}%
\pgfsetstrokeopacity{0.050000}%
\pgfsetdash{}{0pt}%
\pgfpathmoveto{\pgfqpoint{10.563728in}{9.576569in}}%
\pgfpathlineto{\pgfqpoint{10.563728in}{13.800000in}}%
\pgfusepath{stroke}%
\end{pgfscope}%
\begin{pgfscope}%
\pgfsetbuttcap%
\pgfsetroundjoin%
\definecolor{currentfill}{rgb}{0.000000,0.000000,0.000000}%
\pgfsetfillcolor{currentfill}%
\pgfsetlinewidth{0.602250pt}%
\definecolor{currentstroke}{rgb}{0.000000,0.000000,0.000000}%
\pgfsetstrokecolor{currentstroke}%
\pgfsetdash{}{0pt}%
\pgfsys@defobject{currentmarker}{\pgfqpoint{0.000000in}{-0.027778in}}{\pgfqpoint{0.000000in}{0.000000in}}{%
\pgfpathmoveto{\pgfqpoint{0.000000in}{0.000000in}}%
\pgfpathlineto{\pgfqpoint{0.000000in}{-0.027778in}}%
\pgfusepath{stroke,fill}%
}%
\begin{pgfscope}%
\pgfsys@transformshift{10.563728in}{9.576569in}%
\pgfsys@useobject{currentmarker}{}%
\end{pgfscope}%
\end{pgfscope}%
\begin{pgfscope}%
\pgfpathrectangle{\pgfqpoint{0.688192in}{9.576569in}}{\pgfqpoint{11.096108in}{4.223431in}}%
\pgfusepath{clip}%
\pgfsetrectcap%
\pgfsetroundjoin%
\pgfsetlinewidth{0.803000pt}%
\definecolor{currentstroke}{rgb}{0.690196,0.690196,0.690196}%
\pgfsetstrokecolor{currentstroke}%
\pgfsetstrokeopacity{0.050000}%
\pgfsetdash{}{0pt}%
\pgfpathmoveto{\pgfqpoint{10.785650in}{9.576569in}}%
\pgfpathlineto{\pgfqpoint{10.785650in}{13.800000in}}%
\pgfusepath{stroke}%
\end{pgfscope}%
\begin{pgfscope}%
\pgfsetbuttcap%
\pgfsetroundjoin%
\definecolor{currentfill}{rgb}{0.000000,0.000000,0.000000}%
\pgfsetfillcolor{currentfill}%
\pgfsetlinewidth{0.602250pt}%
\definecolor{currentstroke}{rgb}{0.000000,0.000000,0.000000}%
\pgfsetstrokecolor{currentstroke}%
\pgfsetdash{}{0pt}%
\pgfsys@defobject{currentmarker}{\pgfqpoint{0.000000in}{-0.027778in}}{\pgfqpoint{0.000000in}{0.000000in}}{%
\pgfpathmoveto{\pgfqpoint{0.000000in}{0.000000in}}%
\pgfpathlineto{\pgfqpoint{0.000000in}{-0.027778in}}%
\pgfusepath{stroke,fill}%
}%
\begin{pgfscope}%
\pgfsys@transformshift{10.785650in}{9.576569in}%
\pgfsys@useobject{currentmarker}{}%
\end{pgfscope}%
\end{pgfscope}%
\begin{pgfscope}%
\pgfpathrectangle{\pgfqpoint{0.688192in}{9.576569in}}{\pgfqpoint{11.096108in}{4.223431in}}%
\pgfusepath{clip}%
\pgfsetrectcap%
\pgfsetroundjoin%
\pgfsetlinewidth{0.803000pt}%
\definecolor{currentstroke}{rgb}{0.690196,0.690196,0.690196}%
\pgfsetstrokecolor{currentstroke}%
\pgfsetstrokeopacity{0.050000}%
\pgfsetdash{}{0pt}%
\pgfpathmoveto{\pgfqpoint{11.007572in}{9.576569in}}%
\pgfpathlineto{\pgfqpoint{11.007572in}{13.800000in}}%
\pgfusepath{stroke}%
\end{pgfscope}%
\begin{pgfscope}%
\pgfsetbuttcap%
\pgfsetroundjoin%
\definecolor{currentfill}{rgb}{0.000000,0.000000,0.000000}%
\pgfsetfillcolor{currentfill}%
\pgfsetlinewidth{0.602250pt}%
\definecolor{currentstroke}{rgb}{0.000000,0.000000,0.000000}%
\pgfsetstrokecolor{currentstroke}%
\pgfsetdash{}{0pt}%
\pgfsys@defobject{currentmarker}{\pgfqpoint{0.000000in}{-0.027778in}}{\pgfqpoint{0.000000in}{0.000000in}}{%
\pgfpathmoveto{\pgfqpoint{0.000000in}{0.000000in}}%
\pgfpathlineto{\pgfqpoint{0.000000in}{-0.027778in}}%
\pgfusepath{stroke,fill}%
}%
\begin{pgfscope}%
\pgfsys@transformshift{11.007572in}{9.576569in}%
\pgfsys@useobject{currentmarker}{}%
\end{pgfscope}%
\end{pgfscope}%
\begin{pgfscope}%
\pgfpathrectangle{\pgfqpoint{0.688192in}{9.576569in}}{\pgfqpoint{11.096108in}{4.223431in}}%
\pgfusepath{clip}%
\pgfsetrectcap%
\pgfsetroundjoin%
\pgfsetlinewidth{0.803000pt}%
\definecolor{currentstroke}{rgb}{0.690196,0.690196,0.690196}%
\pgfsetstrokecolor{currentstroke}%
\pgfsetstrokeopacity{0.050000}%
\pgfsetdash{}{0pt}%
\pgfpathmoveto{\pgfqpoint{11.451416in}{9.576569in}}%
\pgfpathlineto{\pgfqpoint{11.451416in}{13.800000in}}%
\pgfusepath{stroke}%
\end{pgfscope}%
\begin{pgfscope}%
\pgfsetbuttcap%
\pgfsetroundjoin%
\definecolor{currentfill}{rgb}{0.000000,0.000000,0.000000}%
\pgfsetfillcolor{currentfill}%
\pgfsetlinewidth{0.602250pt}%
\definecolor{currentstroke}{rgb}{0.000000,0.000000,0.000000}%
\pgfsetstrokecolor{currentstroke}%
\pgfsetdash{}{0pt}%
\pgfsys@defobject{currentmarker}{\pgfqpoint{0.000000in}{-0.027778in}}{\pgfqpoint{0.000000in}{0.000000in}}{%
\pgfpathmoveto{\pgfqpoint{0.000000in}{0.000000in}}%
\pgfpathlineto{\pgfqpoint{0.000000in}{-0.027778in}}%
\pgfusepath{stroke,fill}%
}%
\begin{pgfscope}%
\pgfsys@transformshift{11.451416in}{9.576569in}%
\pgfsys@useobject{currentmarker}{}%
\end{pgfscope}%
\end{pgfscope}%
\begin{pgfscope}%
\pgfpathrectangle{\pgfqpoint{0.688192in}{9.576569in}}{\pgfqpoint{11.096108in}{4.223431in}}%
\pgfusepath{clip}%
\pgfsetrectcap%
\pgfsetroundjoin%
\pgfsetlinewidth{0.803000pt}%
\definecolor{currentstroke}{rgb}{0.690196,0.690196,0.690196}%
\pgfsetstrokecolor{currentstroke}%
\pgfsetstrokeopacity{0.050000}%
\pgfsetdash{}{0pt}%
\pgfpathmoveto{\pgfqpoint{11.673338in}{9.576569in}}%
\pgfpathlineto{\pgfqpoint{11.673338in}{13.800000in}}%
\pgfusepath{stroke}%
\end{pgfscope}%
\begin{pgfscope}%
\pgfsetbuttcap%
\pgfsetroundjoin%
\definecolor{currentfill}{rgb}{0.000000,0.000000,0.000000}%
\pgfsetfillcolor{currentfill}%
\pgfsetlinewidth{0.602250pt}%
\definecolor{currentstroke}{rgb}{0.000000,0.000000,0.000000}%
\pgfsetstrokecolor{currentstroke}%
\pgfsetdash{}{0pt}%
\pgfsys@defobject{currentmarker}{\pgfqpoint{0.000000in}{-0.027778in}}{\pgfqpoint{0.000000in}{0.000000in}}{%
\pgfpathmoveto{\pgfqpoint{0.000000in}{0.000000in}}%
\pgfpathlineto{\pgfqpoint{0.000000in}{-0.027778in}}%
\pgfusepath{stroke,fill}%
}%
\begin{pgfscope}%
\pgfsys@transformshift{11.673338in}{9.576569in}%
\pgfsys@useobject{currentmarker}{}%
\end{pgfscope}%
\end{pgfscope}%
\begin{pgfscope}%
\pgfpathrectangle{\pgfqpoint{0.688192in}{9.576569in}}{\pgfqpoint{11.096108in}{4.223431in}}%
\pgfusepath{clip}%
\pgfsetrectcap%
\pgfsetroundjoin%
\pgfsetlinewidth{0.803000pt}%
\definecolor{currentstroke}{rgb}{0.690196,0.690196,0.690196}%
\pgfsetstrokecolor{currentstroke}%
\pgfsetstrokeopacity{0.200000}%
\pgfsetdash{}{0pt}%
\pgfpathmoveto{\pgfqpoint{0.688192in}{9.576569in}}%
\pgfpathlineto{\pgfqpoint{11.784299in}{9.576569in}}%
\pgfusepath{stroke}%
\end{pgfscope}%
\begin{pgfscope}%
\pgfsetbuttcap%
\pgfsetroundjoin%
\definecolor{currentfill}{rgb}{0.000000,0.000000,0.000000}%
\pgfsetfillcolor{currentfill}%
\pgfsetlinewidth{0.803000pt}%
\definecolor{currentstroke}{rgb}{0.000000,0.000000,0.000000}%
\pgfsetstrokecolor{currentstroke}%
\pgfsetdash{}{0pt}%
\pgfsys@defobject{currentmarker}{\pgfqpoint{-0.048611in}{0.000000in}}{\pgfqpoint{-0.000000in}{0.000000in}}{%
\pgfpathmoveto{\pgfqpoint{-0.000000in}{0.000000in}}%
\pgfpathlineto{\pgfqpoint{-0.048611in}{0.000000in}}%
\pgfusepath{stroke,fill}%
}%
\begin{pgfscope}%
\pgfsys@transformshift{0.688192in}{9.576569in}%
\pgfsys@useobject{currentmarker}{}%
\end{pgfscope}%
\end{pgfscope}%
\begin{pgfscope}%
\definecolor{textcolor}{rgb}{0.000000,0.000000,0.000000}%
\pgfsetstrokecolor{textcolor}%
\pgfsetfillcolor{textcolor}%
\pgftext[x=0.493054in, y=9.507125in, left, base]{\color{textcolor}{\rmfamily\fontsize{14.000000}{16.800000}\selectfont\catcode`\^=\active\def^{\ifmmode\sp\else\^{}\fi}\catcode`\%=\active\def%{\%}$\mathdefault{0}$}}%
\end{pgfscope}%
\begin{pgfscope}%
\pgfpathrectangle{\pgfqpoint{0.688192in}{9.576569in}}{\pgfqpoint{11.096108in}{4.223431in}}%
\pgfusepath{clip}%
\pgfsetrectcap%
\pgfsetroundjoin%
\pgfsetlinewidth{0.803000pt}%
\definecolor{currentstroke}{rgb}{0.690196,0.690196,0.690196}%
\pgfsetstrokecolor{currentstroke}%
\pgfsetstrokeopacity{0.200000}%
\pgfsetdash{}{0pt}%
\pgfpathmoveto{\pgfqpoint{0.688192in}{10.358686in}}%
\pgfpathlineto{\pgfqpoint{11.784299in}{10.358686in}}%
\pgfusepath{stroke}%
\end{pgfscope}%
\begin{pgfscope}%
\pgfsetbuttcap%
\pgfsetroundjoin%
\definecolor{currentfill}{rgb}{0.000000,0.000000,0.000000}%
\pgfsetfillcolor{currentfill}%
\pgfsetlinewidth{0.803000pt}%
\definecolor{currentstroke}{rgb}{0.000000,0.000000,0.000000}%
\pgfsetstrokecolor{currentstroke}%
\pgfsetdash{}{0pt}%
\pgfsys@defobject{currentmarker}{\pgfqpoint{-0.048611in}{0.000000in}}{\pgfqpoint{-0.000000in}{0.000000in}}{%
\pgfpathmoveto{\pgfqpoint{-0.000000in}{0.000000in}}%
\pgfpathlineto{\pgfqpoint{-0.048611in}{0.000000in}}%
\pgfusepath{stroke,fill}%
}%
\begin{pgfscope}%
\pgfsys@transformshift{0.688192in}{10.358686in}%
\pgfsys@useobject{currentmarker}{}%
\end{pgfscope}%
\end{pgfscope}%
\begin{pgfscope}%
\definecolor{textcolor}{rgb}{0.000000,0.000000,0.000000}%
\pgfsetstrokecolor{textcolor}%
\pgfsetfillcolor{textcolor}%
\pgftext[x=0.493054in, y=10.289242in, left, base]{\color{textcolor}{\rmfamily\fontsize{14.000000}{16.800000}\selectfont\catcode`\^=\active\def^{\ifmmode\sp\else\^{}\fi}\catcode`\%=\active\def%{\%}$\mathdefault{5}$}}%
\end{pgfscope}%
\begin{pgfscope}%
\pgfpathrectangle{\pgfqpoint{0.688192in}{9.576569in}}{\pgfqpoint{11.096108in}{4.223431in}}%
\pgfusepath{clip}%
\pgfsetrectcap%
\pgfsetroundjoin%
\pgfsetlinewidth{0.803000pt}%
\definecolor{currentstroke}{rgb}{0.690196,0.690196,0.690196}%
\pgfsetstrokecolor{currentstroke}%
\pgfsetstrokeopacity{0.200000}%
\pgfsetdash{}{0pt}%
\pgfpathmoveto{\pgfqpoint{0.688192in}{11.140803in}}%
\pgfpathlineto{\pgfqpoint{11.784299in}{11.140803in}}%
\pgfusepath{stroke}%
\end{pgfscope}%
\begin{pgfscope}%
\pgfsetbuttcap%
\pgfsetroundjoin%
\definecolor{currentfill}{rgb}{0.000000,0.000000,0.000000}%
\pgfsetfillcolor{currentfill}%
\pgfsetlinewidth{0.803000pt}%
\definecolor{currentstroke}{rgb}{0.000000,0.000000,0.000000}%
\pgfsetstrokecolor{currentstroke}%
\pgfsetdash{}{0pt}%
\pgfsys@defobject{currentmarker}{\pgfqpoint{-0.048611in}{0.000000in}}{\pgfqpoint{-0.000000in}{0.000000in}}{%
\pgfpathmoveto{\pgfqpoint{-0.000000in}{0.000000in}}%
\pgfpathlineto{\pgfqpoint{-0.048611in}{0.000000in}}%
\pgfusepath{stroke,fill}%
}%
\begin{pgfscope}%
\pgfsys@transformshift{0.688192in}{11.140803in}%
\pgfsys@useobject{currentmarker}{}%
\end{pgfscope}%
\end{pgfscope}%
\begin{pgfscope}%
\definecolor{textcolor}{rgb}{0.000000,0.000000,0.000000}%
\pgfsetstrokecolor{textcolor}%
\pgfsetfillcolor{textcolor}%
\pgftext[x=0.395138in, y=11.071359in, left, base]{\color{textcolor}{\rmfamily\fontsize{14.000000}{16.800000}\selectfont\catcode`\^=\active\def^{\ifmmode\sp\else\^{}\fi}\catcode`\%=\active\def%{\%}$\mathdefault{10}$}}%
\end{pgfscope}%
\begin{pgfscope}%
\pgfpathrectangle{\pgfqpoint{0.688192in}{9.576569in}}{\pgfqpoint{11.096108in}{4.223431in}}%
\pgfusepath{clip}%
\pgfsetrectcap%
\pgfsetroundjoin%
\pgfsetlinewidth{0.803000pt}%
\definecolor{currentstroke}{rgb}{0.690196,0.690196,0.690196}%
\pgfsetstrokecolor{currentstroke}%
\pgfsetstrokeopacity{0.200000}%
\pgfsetdash{}{0pt}%
\pgfpathmoveto{\pgfqpoint{0.688192in}{11.922920in}}%
\pgfpathlineto{\pgfqpoint{11.784299in}{11.922920in}}%
\pgfusepath{stroke}%
\end{pgfscope}%
\begin{pgfscope}%
\pgfsetbuttcap%
\pgfsetroundjoin%
\definecolor{currentfill}{rgb}{0.000000,0.000000,0.000000}%
\pgfsetfillcolor{currentfill}%
\pgfsetlinewidth{0.803000pt}%
\definecolor{currentstroke}{rgb}{0.000000,0.000000,0.000000}%
\pgfsetstrokecolor{currentstroke}%
\pgfsetdash{}{0pt}%
\pgfsys@defobject{currentmarker}{\pgfqpoint{-0.048611in}{0.000000in}}{\pgfqpoint{-0.000000in}{0.000000in}}{%
\pgfpathmoveto{\pgfqpoint{-0.000000in}{0.000000in}}%
\pgfpathlineto{\pgfqpoint{-0.048611in}{0.000000in}}%
\pgfusepath{stroke,fill}%
}%
\begin{pgfscope}%
\pgfsys@transformshift{0.688192in}{11.922920in}%
\pgfsys@useobject{currentmarker}{}%
\end{pgfscope}%
\end{pgfscope}%
\begin{pgfscope}%
\definecolor{textcolor}{rgb}{0.000000,0.000000,0.000000}%
\pgfsetstrokecolor{textcolor}%
\pgfsetfillcolor{textcolor}%
\pgftext[x=0.395138in, y=11.853475in, left, base]{\color{textcolor}{\rmfamily\fontsize{14.000000}{16.800000}\selectfont\catcode`\^=\active\def^{\ifmmode\sp\else\^{}\fi}\catcode`\%=\active\def%{\%}$\mathdefault{15}$}}%
\end{pgfscope}%
\begin{pgfscope}%
\pgfpathrectangle{\pgfqpoint{0.688192in}{9.576569in}}{\pgfqpoint{11.096108in}{4.223431in}}%
\pgfusepath{clip}%
\pgfsetrectcap%
\pgfsetroundjoin%
\pgfsetlinewidth{0.803000pt}%
\definecolor{currentstroke}{rgb}{0.690196,0.690196,0.690196}%
\pgfsetstrokecolor{currentstroke}%
\pgfsetstrokeopacity{0.200000}%
\pgfsetdash{}{0pt}%
\pgfpathmoveto{\pgfqpoint{0.688192in}{12.705036in}}%
\pgfpathlineto{\pgfqpoint{11.784299in}{12.705036in}}%
\pgfusepath{stroke}%
\end{pgfscope}%
\begin{pgfscope}%
\pgfsetbuttcap%
\pgfsetroundjoin%
\definecolor{currentfill}{rgb}{0.000000,0.000000,0.000000}%
\pgfsetfillcolor{currentfill}%
\pgfsetlinewidth{0.803000pt}%
\definecolor{currentstroke}{rgb}{0.000000,0.000000,0.000000}%
\pgfsetstrokecolor{currentstroke}%
\pgfsetdash{}{0pt}%
\pgfsys@defobject{currentmarker}{\pgfqpoint{-0.048611in}{0.000000in}}{\pgfqpoint{-0.000000in}{0.000000in}}{%
\pgfpathmoveto{\pgfqpoint{-0.000000in}{0.000000in}}%
\pgfpathlineto{\pgfqpoint{-0.048611in}{0.000000in}}%
\pgfusepath{stroke,fill}%
}%
\begin{pgfscope}%
\pgfsys@transformshift{0.688192in}{12.705036in}%
\pgfsys@useobject{currentmarker}{}%
\end{pgfscope}%
\end{pgfscope}%
\begin{pgfscope}%
\definecolor{textcolor}{rgb}{0.000000,0.000000,0.000000}%
\pgfsetstrokecolor{textcolor}%
\pgfsetfillcolor{textcolor}%
\pgftext[x=0.395138in, y=12.635592in, left, base]{\color{textcolor}{\rmfamily\fontsize{14.000000}{16.800000}\selectfont\catcode`\^=\active\def^{\ifmmode\sp\else\^{}\fi}\catcode`\%=\active\def%{\%}$\mathdefault{20}$}}%
\end{pgfscope}%
\begin{pgfscope}%
\pgfpathrectangle{\pgfqpoint{0.688192in}{9.576569in}}{\pgfqpoint{11.096108in}{4.223431in}}%
\pgfusepath{clip}%
\pgfsetrectcap%
\pgfsetroundjoin%
\pgfsetlinewidth{0.803000pt}%
\definecolor{currentstroke}{rgb}{0.690196,0.690196,0.690196}%
\pgfsetstrokecolor{currentstroke}%
\pgfsetstrokeopacity{0.200000}%
\pgfsetdash{}{0pt}%
\pgfpathmoveto{\pgfqpoint{0.688192in}{13.487153in}}%
\pgfpathlineto{\pgfqpoint{11.784299in}{13.487153in}}%
\pgfusepath{stroke}%
\end{pgfscope}%
\begin{pgfscope}%
\pgfsetbuttcap%
\pgfsetroundjoin%
\definecolor{currentfill}{rgb}{0.000000,0.000000,0.000000}%
\pgfsetfillcolor{currentfill}%
\pgfsetlinewidth{0.803000pt}%
\definecolor{currentstroke}{rgb}{0.000000,0.000000,0.000000}%
\pgfsetstrokecolor{currentstroke}%
\pgfsetdash{}{0pt}%
\pgfsys@defobject{currentmarker}{\pgfqpoint{-0.048611in}{0.000000in}}{\pgfqpoint{-0.000000in}{0.000000in}}{%
\pgfpathmoveto{\pgfqpoint{-0.000000in}{0.000000in}}%
\pgfpathlineto{\pgfqpoint{-0.048611in}{0.000000in}}%
\pgfusepath{stroke,fill}%
}%
\begin{pgfscope}%
\pgfsys@transformshift{0.688192in}{13.487153in}%
\pgfsys@useobject{currentmarker}{}%
\end{pgfscope}%
\end{pgfscope}%
\begin{pgfscope}%
\definecolor{textcolor}{rgb}{0.000000,0.000000,0.000000}%
\pgfsetstrokecolor{textcolor}%
\pgfsetfillcolor{textcolor}%
\pgftext[x=0.395138in, y=13.417709in, left, base]{\color{textcolor}{\rmfamily\fontsize{14.000000}{16.800000}\selectfont\catcode`\^=\active\def^{\ifmmode\sp\else\^{}\fi}\catcode`\%=\active\def%{\%}$\mathdefault{25}$}}%
\end{pgfscope}%
\begin{pgfscope}%
\pgfpathrectangle{\pgfqpoint{0.688192in}{9.576569in}}{\pgfqpoint{11.096108in}{4.223431in}}%
\pgfusepath{clip}%
\pgfsetrectcap%
\pgfsetroundjoin%
\pgfsetlinewidth{0.803000pt}%
\definecolor{currentstroke}{rgb}{0.690196,0.690196,0.690196}%
\pgfsetstrokecolor{currentstroke}%
\pgfsetstrokeopacity{0.050000}%
\pgfsetdash{}{0pt}%
\pgfpathmoveto{\pgfqpoint{0.688192in}{9.732993in}}%
\pgfpathlineto{\pgfqpoint{11.784299in}{9.732993in}}%
\pgfusepath{stroke}%
\end{pgfscope}%
\begin{pgfscope}%
\pgfsetbuttcap%
\pgfsetroundjoin%
\definecolor{currentfill}{rgb}{0.000000,0.000000,0.000000}%
\pgfsetfillcolor{currentfill}%
\pgfsetlinewidth{0.602250pt}%
\definecolor{currentstroke}{rgb}{0.000000,0.000000,0.000000}%
\pgfsetstrokecolor{currentstroke}%
\pgfsetdash{}{0pt}%
\pgfsys@defobject{currentmarker}{\pgfqpoint{-0.027778in}{0.000000in}}{\pgfqpoint{-0.000000in}{0.000000in}}{%
\pgfpathmoveto{\pgfqpoint{-0.000000in}{0.000000in}}%
\pgfpathlineto{\pgfqpoint{-0.027778in}{0.000000in}}%
\pgfusepath{stroke,fill}%
}%
\begin{pgfscope}%
\pgfsys@transformshift{0.688192in}{9.732993in}%
\pgfsys@useobject{currentmarker}{}%
\end{pgfscope}%
\end{pgfscope}%
\begin{pgfscope}%
\pgfpathrectangle{\pgfqpoint{0.688192in}{9.576569in}}{\pgfqpoint{11.096108in}{4.223431in}}%
\pgfusepath{clip}%
\pgfsetrectcap%
\pgfsetroundjoin%
\pgfsetlinewidth{0.803000pt}%
\definecolor{currentstroke}{rgb}{0.690196,0.690196,0.690196}%
\pgfsetstrokecolor{currentstroke}%
\pgfsetstrokeopacity{0.050000}%
\pgfsetdash{}{0pt}%
\pgfpathmoveto{\pgfqpoint{0.688192in}{9.889416in}}%
\pgfpathlineto{\pgfqpoint{11.784299in}{9.889416in}}%
\pgfusepath{stroke}%
\end{pgfscope}%
\begin{pgfscope}%
\pgfsetbuttcap%
\pgfsetroundjoin%
\definecolor{currentfill}{rgb}{0.000000,0.000000,0.000000}%
\pgfsetfillcolor{currentfill}%
\pgfsetlinewidth{0.602250pt}%
\definecolor{currentstroke}{rgb}{0.000000,0.000000,0.000000}%
\pgfsetstrokecolor{currentstroke}%
\pgfsetdash{}{0pt}%
\pgfsys@defobject{currentmarker}{\pgfqpoint{-0.027778in}{0.000000in}}{\pgfqpoint{-0.000000in}{0.000000in}}{%
\pgfpathmoveto{\pgfqpoint{-0.000000in}{0.000000in}}%
\pgfpathlineto{\pgfqpoint{-0.027778in}{0.000000in}}%
\pgfusepath{stroke,fill}%
}%
\begin{pgfscope}%
\pgfsys@transformshift{0.688192in}{9.889416in}%
\pgfsys@useobject{currentmarker}{}%
\end{pgfscope}%
\end{pgfscope}%
\begin{pgfscope}%
\pgfpathrectangle{\pgfqpoint{0.688192in}{9.576569in}}{\pgfqpoint{11.096108in}{4.223431in}}%
\pgfusepath{clip}%
\pgfsetrectcap%
\pgfsetroundjoin%
\pgfsetlinewidth{0.803000pt}%
\definecolor{currentstroke}{rgb}{0.690196,0.690196,0.690196}%
\pgfsetstrokecolor{currentstroke}%
\pgfsetstrokeopacity{0.050000}%
\pgfsetdash{}{0pt}%
\pgfpathmoveto{\pgfqpoint{0.688192in}{10.045839in}}%
\pgfpathlineto{\pgfqpoint{11.784299in}{10.045839in}}%
\pgfusepath{stroke}%
\end{pgfscope}%
\begin{pgfscope}%
\pgfsetbuttcap%
\pgfsetroundjoin%
\definecolor{currentfill}{rgb}{0.000000,0.000000,0.000000}%
\pgfsetfillcolor{currentfill}%
\pgfsetlinewidth{0.602250pt}%
\definecolor{currentstroke}{rgb}{0.000000,0.000000,0.000000}%
\pgfsetstrokecolor{currentstroke}%
\pgfsetdash{}{0pt}%
\pgfsys@defobject{currentmarker}{\pgfqpoint{-0.027778in}{0.000000in}}{\pgfqpoint{-0.000000in}{0.000000in}}{%
\pgfpathmoveto{\pgfqpoint{-0.000000in}{0.000000in}}%
\pgfpathlineto{\pgfqpoint{-0.027778in}{0.000000in}}%
\pgfusepath{stroke,fill}%
}%
\begin{pgfscope}%
\pgfsys@transformshift{0.688192in}{10.045839in}%
\pgfsys@useobject{currentmarker}{}%
\end{pgfscope}%
\end{pgfscope}%
\begin{pgfscope}%
\pgfpathrectangle{\pgfqpoint{0.688192in}{9.576569in}}{\pgfqpoint{11.096108in}{4.223431in}}%
\pgfusepath{clip}%
\pgfsetrectcap%
\pgfsetroundjoin%
\pgfsetlinewidth{0.803000pt}%
\definecolor{currentstroke}{rgb}{0.690196,0.690196,0.690196}%
\pgfsetstrokecolor{currentstroke}%
\pgfsetstrokeopacity{0.050000}%
\pgfsetdash{}{0pt}%
\pgfpathmoveto{\pgfqpoint{0.688192in}{10.202263in}}%
\pgfpathlineto{\pgfqpoint{11.784299in}{10.202263in}}%
\pgfusepath{stroke}%
\end{pgfscope}%
\begin{pgfscope}%
\pgfsetbuttcap%
\pgfsetroundjoin%
\definecolor{currentfill}{rgb}{0.000000,0.000000,0.000000}%
\pgfsetfillcolor{currentfill}%
\pgfsetlinewidth{0.602250pt}%
\definecolor{currentstroke}{rgb}{0.000000,0.000000,0.000000}%
\pgfsetstrokecolor{currentstroke}%
\pgfsetdash{}{0pt}%
\pgfsys@defobject{currentmarker}{\pgfqpoint{-0.027778in}{0.000000in}}{\pgfqpoint{-0.000000in}{0.000000in}}{%
\pgfpathmoveto{\pgfqpoint{-0.000000in}{0.000000in}}%
\pgfpathlineto{\pgfqpoint{-0.027778in}{0.000000in}}%
\pgfusepath{stroke,fill}%
}%
\begin{pgfscope}%
\pgfsys@transformshift{0.688192in}{10.202263in}%
\pgfsys@useobject{currentmarker}{}%
\end{pgfscope}%
\end{pgfscope}%
\begin{pgfscope}%
\pgfpathrectangle{\pgfqpoint{0.688192in}{9.576569in}}{\pgfqpoint{11.096108in}{4.223431in}}%
\pgfusepath{clip}%
\pgfsetrectcap%
\pgfsetroundjoin%
\pgfsetlinewidth{0.803000pt}%
\definecolor{currentstroke}{rgb}{0.690196,0.690196,0.690196}%
\pgfsetstrokecolor{currentstroke}%
\pgfsetstrokeopacity{0.050000}%
\pgfsetdash{}{0pt}%
\pgfpathmoveto{\pgfqpoint{0.688192in}{10.515109in}}%
\pgfpathlineto{\pgfqpoint{11.784299in}{10.515109in}}%
\pgfusepath{stroke}%
\end{pgfscope}%
\begin{pgfscope}%
\pgfsetbuttcap%
\pgfsetroundjoin%
\definecolor{currentfill}{rgb}{0.000000,0.000000,0.000000}%
\pgfsetfillcolor{currentfill}%
\pgfsetlinewidth{0.602250pt}%
\definecolor{currentstroke}{rgb}{0.000000,0.000000,0.000000}%
\pgfsetstrokecolor{currentstroke}%
\pgfsetdash{}{0pt}%
\pgfsys@defobject{currentmarker}{\pgfqpoint{-0.027778in}{0.000000in}}{\pgfqpoint{-0.000000in}{0.000000in}}{%
\pgfpathmoveto{\pgfqpoint{-0.000000in}{0.000000in}}%
\pgfpathlineto{\pgfqpoint{-0.027778in}{0.000000in}}%
\pgfusepath{stroke,fill}%
}%
\begin{pgfscope}%
\pgfsys@transformshift{0.688192in}{10.515109in}%
\pgfsys@useobject{currentmarker}{}%
\end{pgfscope}%
\end{pgfscope}%
\begin{pgfscope}%
\pgfpathrectangle{\pgfqpoint{0.688192in}{9.576569in}}{\pgfqpoint{11.096108in}{4.223431in}}%
\pgfusepath{clip}%
\pgfsetrectcap%
\pgfsetroundjoin%
\pgfsetlinewidth{0.803000pt}%
\definecolor{currentstroke}{rgb}{0.690196,0.690196,0.690196}%
\pgfsetstrokecolor{currentstroke}%
\pgfsetstrokeopacity{0.050000}%
\pgfsetdash{}{0pt}%
\pgfpathmoveto{\pgfqpoint{0.688192in}{10.671533in}}%
\pgfpathlineto{\pgfqpoint{11.784299in}{10.671533in}}%
\pgfusepath{stroke}%
\end{pgfscope}%
\begin{pgfscope}%
\pgfsetbuttcap%
\pgfsetroundjoin%
\definecolor{currentfill}{rgb}{0.000000,0.000000,0.000000}%
\pgfsetfillcolor{currentfill}%
\pgfsetlinewidth{0.602250pt}%
\definecolor{currentstroke}{rgb}{0.000000,0.000000,0.000000}%
\pgfsetstrokecolor{currentstroke}%
\pgfsetdash{}{0pt}%
\pgfsys@defobject{currentmarker}{\pgfqpoint{-0.027778in}{0.000000in}}{\pgfqpoint{-0.000000in}{0.000000in}}{%
\pgfpathmoveto{\pgfqpoint{-0.000000in}{0.000000in}}%
\pgfpathlineto{\pgfqpoint{-0.027778in}{0.000000in}}%
\pgfusepath{stroke,fill}%
}%
\begin{pgfscope}%
\pgfsys@transformshift{0.688192in}{10.671533in}%
\pgfsys@useobject{currentmarker}{}%
\end{pgfscope}%
\end{pgfscope}%
\begin{pgfscope}%
\pgfpathrectangle{\pgfqpoint{0.688192in}{9.576569in}}{\pgfqpoint{11.096108in}{4.223431in}}%
\pgfusepath{clip}%
\pgfsetrectcap%
\pgfsetroundjoin%
\pgfsetlinewidth{0.803000pt}%
\definecolor{currentstroke}{rgb}{0.690196,0.690196,0.690196}%
\pgfsetstrokecolor{currentstroke}%
\pgfsetstrokeopacity{0.050000}%
\pgfsetdash{}{0pt}%
\pgfpathmoveto{\pgfqpoint{0.688192in}{10.827956in}}%
\pgfpathlineto{\pgfqpoint{11.784299in}{10.827956in}}%
\pgfusepath{stroke}%
\end{pgfscope}%
\begin{pgfscope}%
\pgfsetbuttcap%
\pgfsetroundjoin%
\definecolor{currentfill}{rgb}{0.000000,0.000000,0.000000}%
\pgfsetfillcolor{currentfill}%
\pgfsetlinewidth{0.602250pt}%
\definecolor{currentstroke}{rgb}{0.000000,0.000000,0.000000}%
\pgfsetstrokecolor{currentstroke}%
\pgfsetdash{}{0pt}%
\pgfsys@defobject{currentmarker}{\pgfqpoint{-0.027778in}{0.000000in}}{\pgfqpoint{-0.000000in}{0.000000in}}{%
\pgfpathmoveto{\pgfqpoint{-0.000000in}{0.000000in}}%
\pgfpathlineto{\pgfqpoint{-0.027778in}{0.000000in}}%
\pgfusepath{stroke,fill}%
}%
\begin{pgfscope}%
\pgfsys@transformshift{0.688192in}{10.827956in}%
\pgfsys@useobject{currentmarker}{}%
\end{pgfscope}%
\end{pgfscope}%
\begin{pgfscope}%
\pgfpathrectangle{\pgfqpoint{0.688192in}{9.576569in}}{\pgfqpoint{11.096108in}{4.223431in}}%
\pgfusepath{clip}%
\pgfsetrectcap%
\pgfsetroundjoin%
\pgfsetlinewidth{0.803000pt}%
\definecolor{currentstroke}{rgb}{0.690196,0.690196,0.690196}%
\pgfsetstrokecolor{currentstroke}%
\pgfsetstrokeopacity{0.050000}%
\pgfsetdash{}{0pt}%
\pgfpathmoveto{\pgfqpoint{0.688192in}{10.984380in}}%
\pgfpathlineto{\pgfqpoint{11.784299in}{10.984380in}}%
\pgfusepath{stroke}%
\end{pgfscope}%
\begin{pgfscope}%
\pgfsetbuttcap%
\pgfsetroundjoin%
\definecolor{currentfill}{rgb}{0.000000,0.000000,0.000000}%
\pgfsetfillcolor{currentfill}%
\pgfsetlinewidth{0.602250pt}%
\definecolor{currentstroke}{rgb}{0.000000,0.000000,0.000000}%
\pgfsetstrokecolor{currentstroke}%
\pgfsetdash{}{0pt}%
\pgfsys@defobject{currentmarker}{\pgfqpoint{-0.027778in}{0.000000in}}{\pgfqpoint{-0.000000in}{0.000000in}}{%
\pgfpathmoveto{\pgfqpoint{-0.000000in}{0.000000in}}%
\pgfpathlineto{\pgfqpoint{-0.027778in}{0.000000in}}%
\pgfusepath{stroke,fill}%
}%
\begin{pgfscope}%
\pgfsys@transformshift{0.688192in}{10.984380in}%
\pgfsys@useobject{currentmarker}{}%
\end{pgfscope}%
\end{pgfscope}%
\begin{pgfscope}%
\pgfpathrectangle{\pgfqpoint{0.688192in}{9.576569in}}{\pgfqpoint{11.096108in}{4.223431in}}%
\pgfusepath{clip}%
\pgfsetrectcap%
\pgfsetroundjoin%
\pgfsetlinewidth{0.803000pt}%
\definecolor{currentstroke}{rgb}{0.690196,0.690196,0.690196}%
\pgfsetstrokecolor{currentstroke}%
\pgfsetstrokeopacity{0.050000}%
\pgfsetdash{}{0pt}%
\pgfpathmoveto{\pgfqpoint{0.688192in}{11.297226in}}%
\pgfpathlineto{\pgfqpoint{11.784299in}{11.297226in}}%
\pgfusepath{stroke}%
\end{pgfscope}%
\begin{pgfscope}%
\pgfsetbuttcap%
\pgfsetroundjoin%
\definecolor{currentfill}{rgb}{0.000000,0.000000,0.000000}%
\pgfsetfillcolor{currentfill}%
\pgfsetlinewidth{0.602250pt}%
\definecolor{currentstroke}{rgb}{0.000000,0.000000,0.000000}%
\pgfsetstrokecolor{currentstroke}%
\pgfsetdash{}{0pt}%
\pgfsys@defobject{currentmarker}{\pgfqpoint{-0.027778in}{0.000000in}}{\pgfqpoint{-0.000000in}{0.000000in}}{%
\pgfpathmoveto{\pgfqpoint{-0.000000in}{0.000000in}}%
\pgfpathlineto{\pgfqpoint{-0.027778in}{0.000000in}}%
\pgfusepath{stroke,fill}%
}%
\begin{pgfscope}%
\pgfsys@transformshift{0.688192in}{11.297226in}%
\pgfsys@useobject{currentmarker}{}%
\end{pgfscope}%
\end{pgfscope}%
\begin{pgfscope}%
\pgfpathrectangle{\pgfqpoint{0.688192in}{9.576569in}}{\pgfqpoint{11.096108in}{4.223431in}}%
\pgfusepath{clip}%
\pgfsetrectcap%
\pgfsetroundjoin%
\pgfsetlinewidth{0.803000pt}%
\definecolor{currentstroke}{rgb}{0.690196,0.690196,0.690196}%
\pgfsetstrokecolor{currentstroke}%
\pgfsetstrokeopacity{0.050000}%
\pgfsetdash{}{0pt}%
\pgfpathmoveto{\pgfqpoint{0.688192in}{11.453650in}}%
\pgfpathlineto{\pgfqpoint{11.784299in}{11.453650in}}%
\pgfusepath{stroke}%
\end{pgfscope}%
\begin{pgfscope}%
\pgfsetbuttcap%
\pgfsetroundjoin%
\definecolor{currentfill}{rgb}{0.000000,0.000000,0.000000}%
\pgfsetfillcolor{currentfill}%
\pgfsetlinewidth{0.602250pt}%
\definecolor{currentstroke}{rgb}{0.000000,0.000000,0.000000}%
\pgfsetstrokecolor{currentstroke}%
\pgfsetdash{}{0pt}%
\pgfsys@defobject{currentmarker}{\pgfqpoint{-0.027778in}{0.000000in}}{\pgfqpoint{-0.000000in}{0.000000in}}{%
\pgfpathmoveto{\pgfqpoint{-0.000000in}{0.000000in}}%
\pgfpathlineto{\pgfqpoint{-0.027778in}{0.000000in}}%
\pgfusepath{stroke,fill}%
}%
\begin{pgfscope}%
\pgfsys@transformshift{0.688192in}{11.453650in}%
\pgfsys@useobject{currentmarker}{}%
\end{pgfscope}%
\end{pgfscope}%
\begin{pgfscope}%
\pgfpathrectangle{\pgfqpoint{0.688192in}{9.576569in}}{\pgfqpoint{11.096108in}{4.223431in}}%
\pgfusepath{clip}%
\pgfsetrectcap%
\pgfsetroundjoin%
\pgfsetlinewidth{0.803000pt}%
\definecolor{currentstroke}{rgb}{0.690196,0.690196,0.690196}%
\pgfsetstrokecolor{currentstroke}%
\pgfsetstrokeopacity{0.050000}%
\pgfsetdash{}{0pt}%
\pgfpathmoveto{\pgfqpoint{0.688192in}{11.610073in}}%
\pgfpathlineto{\pgfqpoint{11.784299in}{11.610073in}}%
\pgfusepath{stroke}%
\end{pgfscope}%
\begin{pgfscope}%
\pgfsetbuttcap%
\pgfsetroundjoin%
\definecolor{currentfill}{rgb}{0.000000,0.000000,0.000000}%
\pgfsetfillcolor{currentfill}%
\pgfsetlinewidth{0.602250pt}%
\definecolor{currentstroke}{rgb}{0.000000,0.000000,0.000000}%
\pgfsetstrokecolor{currentstroke}%
\pgfsetdash{}{0pt}%
\pgfsys@defobject{currentmarker}{\pgfqpoint{-0.027778in}{0.000000in}}{\pgfqpoint{-0.000000in}{0.000000in}}{%
\pgfpathmoveto{\pgfqpoint{-0.000000in}{0.000000in}}%
\pgfpathlineto{\pgfqpoint{-0.027778in}{0.000000in}}%
\pgfusepath{stroke,fill}%
}%
\begin{pgfscope}%
\pgfsys@transformshift{0.688192in}{11.610073in}%
\pgfsys@useobject{currentmarker}{}%
\end{pgfscope}%
\end{pgfscope}%
\begin{pgfscope}%
\pgfpathrectangle{\pgfqpoint{0.688192in}{9.576569in}}{\pgfqpoint{11.096108in}{4.223431in}}%
\pgfusepath{clip}%
\pgfsetrectcap%
\pgfsetroundjoin%
\pgfsetlinewidth{0.803000pt}%
\definecolor{currentstroke}{rgb}{0.690196,0.690196,0.690196}%
\pgfsetstrokecolor{currentstroke}%
\pgfsetstrokeopacity{0.050000}%
\pgfsetdash{}{0pt}%
\pgfpathmoveto{\pgfqpoint{0.688192in}{11.766496in}}%
\pgfpathlineto{\pgfqpoint{11.784299in}{11.766496in}}%
\pgfusepath{stroke}%
\end{pgfscope}%
\begin{pgfscope}%
\pgfsetbuttcap%
\pgfsetroundjoin%
\definecolor{currentfill}{rgb}{0.000000,0.000000,0.000000}%
\pgfsetfillcolor{currentfill}%
\pgfsetlinewidth{0.602250pt}%
\definecolor{currentstroke}{rgb}{0.000000,0.000000,0.000000}%
\pgfsetstrokecolor{currentstroke}%
\pgfsetdash{}{0pt}%
\pgfsys@defobject{currentmarker}{\pgfqpoint{-0.027778in}{0.000000in}}{\pgfqpoint{-0.000000in}{0.000000in}}{%
\pgfpathmoveto{\pgfqpoint{-0.000000in}{0.000000in}}%
\pgfpathlineto{\pgfqpoint{-0.027778in}{0.000000in}}%
\pgfusepath{stroke,fill}%
}%
\begin{pgfscope}%
\pgfsys@transformshift{0.688192in}{11.766496in}%
\pgfsys@useobject{currentmarker}{}%
\end{pgfscope}%
\end{pgfscope}%
\begin{pgfscope}%
\pgfpathrectangle{\pgfqpoint{0.688192in}{9.576569in}}{\pgfqpoint{11.096108in}{4.223431in}}%
\pgfusepath{clip}%
\pgfsetrectcap%
\pgfsetroundjoin%
\pgfsetlinewidth{0.803000pt}%
\definecolor{currentstroke}{rgb}{0.690196,0.690196,0.690196}%
\pgfsetstrokecolor{currentstroke}%
\pgfsetstrokeopacity{0.050000}%
\pgfsetdash{}{0pt}%
\pgfpathmoveto{\pgfqpoint{0.688192in}{12.079343in}}%
\pgfpathlineto{\pgfqpoint{11.784299in}{12.079343in}}%
\pgfusepath{stroke}%
\end{pgfscope}%
\begin{pgfscope}%
\pgfsetbuttcap%
\pgfsetroundjoin%
\definecolor{currentfill}{rgb}{0.000000,0.000000,0.000000}%
\pgfsetfillcolor{currentfill}%
\pgfsetlinewidth{0.602250pt}%
\definecolor{currentstroke}{rgb}{0.000000,0.000000,0.000000}%
\pgfsetstrokecolor{currentstroke}%
\pgfsetdash{}{0pt}%
\pgfsys@defobject{currentmarker}{\pgfqpoint{-0.027778in}{0.000000in}}{\pgfqpoint{-0.000000in}{0.000000in}}{%
\pgfpathmoveto{\pgfqpoint{-0.000000in}{0.000000in}}%
\pgfpathlineto{\pgfqpoint{-0.027778in}{0.000000in}}%
\pgfusepath{stroke,fill}%
}%
\begin{pgfscope}%
\pgfsys@transformshift{0.688192in}{12.079343in}%
\pgfsys@useobject{currentmarker}{}%
\end{pgfscope}%
\end{pgfscope}%
\begin{pgfscope}%
\pgfpathrectangle{\pgfqpoint{0.688192in}{9.576569in}}{\pgfqpoint{11.096108in}{4.223431in}}%
\pgfusepath{clip}%
\pgfsetrectcap%
\pgfsetroundjoin%
\pgfsetlinewidth{0.803000pt}%
\definecolor{currentstroke}{rgb}{0.690196,0.690196,0.690196}%
\pgfsetstrokecolor{currentstroke}%
\pgfsetstrokeopacity{0.050000}%
\pgfsetdash{}{0pt}%
\pgfpathmoveto{\pgfqpoint{0.688192in}{12.235766in}}%
\pgfpathlineto{\pgfqpoint{11.784299in}{12.235766in}}%
\pgfusepath{stroke}%
\end{pgfscope}%
\begin{pgfscope}%
\pgfsetbuttcap%
\pgfsetroundjoin%
\definecolor{currentfill}{rgb}{0.000000,0.000000,0.000000}%
\pgfsetfillcolor{currentfill}%
\pgfsetlinewidth{0.602250pt}%
\definecolor{currentstroke}{rgb}{0.000000,0.000000,0.000000}%
\pgfsetstrokecolor{currentstroke}%
\pgfsetdash{}{0pt}%
\pgfsys@defobject{currentmarker}{\pgfqpoint{-0.027778in}{0.000000in}}{\pgfqpoint{-0.000000in}{0.000000in}}{%
\pgfpathmoveto{\pgfqpoint{-0.000000in}{0.000000in}}%
\pgfpathlineto{\pgfqpoint{-0.027778in}{0.000000in}}%
\pgfusepath{stroke,fill}%
}%
\begin{pgfscope}%
\pgfsys@transformshift{0.688192in}{12.235766in}%
\pgfsys@useobject{currentmarker}{}%
\end{pgfscope}%
\end{pgfscope}%
\begin{pgfscope}%
\pgfpathrectangle{\pgfqpoint{0.688192in}{9.576569in}}{\pgfqpoint{11.096108in}{4.223431in}}%
\pgfusepath{clip}%
\pgfsetrectcap%
\pgfsetroundjoin%
\pgfsetlinewidth{0.803000pt}%
\definecolor{currentstroke}{rgb}{0.690196,0.690196,0.690196}%
\pgfsetstrokecolor{currentstroke}%
\pgfsetstrokeopacity{0.050000}%
\pgfsetdash{}{0pt}%
\pgfpathmoveto{\pgfqpoint{0.688192in}{12.392190in}}%
\pgfpathlineto{\pgfqpoint{11.784299in}{12.392190in}}%
\pgfusepath{stroke}%
\end{pgfscope}%
\begin{pgfscope}%
\pgfsetbuttcap%
\pgfsetroundjoin%
\definecolor{currentfill}{rgb}{0.000000,0.000000,0.000000}%
\pgfsetfillcolor{currentfill}%
\pgfsetlinewidth{0.602250pt}%
\definecolor{currentstroke}{rgb}{0.000000,0.000000,0.000000}%
\pgfsetstrokecolor{currentstroke}%
\pgfsetdash{}{0pt}%
\pgfsys@defobject{currentmarker}{\pgfqpoint{-0.027778in}{0.000000in}}{\pgfqpoint{-0.000000in}{0.000000in}}{%
\pgfpathmoveto{\pgfqpoint{-0.000000in}{0.000000in}}%
\pgfpathlineto{\pgfqpoint{-0.027778in}{0.000000in}}%
\pgfusepath{stroke,fill}%
}%
\begin{pgfscope}%
\pgfsys@transformshift{0.688192in}{12.392190in}%
\pgfsys@useobject{currentmarker}{}%
\end{pgfscope}%
\end{pgfscope}%
\begin{pgfscope}%
\pgfpathrectangle{\pgfqpoint{0.688192in}{9.576569in}}{\pgfqpoint{11.096108in}{4.223431in}}%
\pgfusepath{clip}%
\pgfsetrectcap%
\pgfsetroundjoin%
\pgfsetlinewidth{0.803000pt}%
\definecolor{currentstroke}{rgb}{0.690196,0.690196,0.690196}%
\pgfsetstrokecolor{currentstroke}%
\pgfsetstrokeopacity{0.050000}%
\pgfsetdash{}{0pt}%
\pgfpathmoveto{\pgfqpoint{0.688192in}{12.548613in}}%
\pgfpathlineto{\pgfqpoint{11.784299in}{12.548613in}}%
\pgfusepath{stroke}%
\end{pgfscope}%
\begin{pgfscope}%
\pgfsetbuttcap%
\pgfsetroundjoin%
\definecolor{currentfill}{rgb}{0.000000,0.000000,0.000000}%
\pgfsetfillcolor{currentfill}%
\pgfsetlinewidth{0.602250pt}%
\definecolor{currentstroke}{rgb}{0.000000,0.000000,0.000000}%
\pgfsetstrokecolor{currentstroke}%
\pgfsetdash{}{0pt}%
\pgfsys@defobject{currentmarker}{\pgfqpoint{-0.027778in}{0.000000in}}{\pgfqpoint{-0.000000in}{0.000000in}}{%
\pgfpathmoveto{\pgfqpoint{-0.000000in}{0.000000in}}%
\pgfpathlineto{\pgfqpoint{-0.027778in}{0.000000in}}%
\pgfusepath{stroke,fill}%
}%
\begin{pgfscope}%
\pgfsys@transformshift{0.688192in}{12.548613in}%
\pgfsys@useobject{currentmarker}{}%
\end{pgfscope}%
\end{pgfscope}%
\begin{pgfscope}%
\pgfpathrectangle{\pgfqpoint{0.688192in}{9.576569in}}{\pgfqpoint{11.096108in}{4.223431in}}%
\pgfusepath{clip}%
\pgfsetrectcap%
\pgfsetroundjoin%
\pgfsetlinewidth{0.803000pt}%
\definecolor{currentstroke}{rgb}{0.690196,0.690196,0.690196}%
\pgfsetstrokecolor{currentstroke}%
\pgfsetstrokeopacity{0.050000}%
\pgfsetdash{}{0pt}%
\pgfpathmoveto{\pgfqpoint{0.688192in}{12.861460in}}%
\pgfpathlineto{\pgfqpoint{11.784299in}{12.861460in}}%
\pgfusepath{stroke}%
\end{pgfscope}%
\begin{pgfscope}%
\pgfsetbuttcap%
\pgfsetroundjoin%
\definecolor{currentfill}{rgb}{0.000000,0.000000,0.000000}%
\pgfsetfillcolor{currentfill}%
\pgfsetlinewidth{0.602250pt}%
\definecolor{currentstroke}{rgb}{0.000000,0.000000,0.000000}%
\pgfsetstrokecolor{currentstroke}%
\pgfsetdash{}{0pt}%
\pgfsys@defobject{currentmarker}{\pgfqpoint{-0.027778in}{0.000000in}}{\pgfqpoint{-0.000000in}{0.000000in}}{%
\pgfpathmoveto{\pgfqpoint{-0.000000in}{0.000000in}}%
\pgfpathlineto{\pgfqpoint{-0.027778in}{0.000000in}}%
\pgfusepath{stroke,fill}%
}%
\begin{pgfscope}%
\pgfsys@transformshift{0.688192in}{12.861460in}%
\pgfsys@useobject{currentmarker}{}%
\end{pgfscope}%
\end{pgfscope}%
\begin{pgfscope}%
\pgfpathrectangle{\pgfqpoint{0.688192in}{9.576569in}}{\pgfqpoint{11.096108in}{4.223431in}}%
\pgfusepath{clip}%
\pgfsetrectcap%
\pgfsetroundjoin%
\pgfsetlinewidth{0.803000pt}%
\definecolor{currentstroke}{rgb}{0.690196,0.690196,0.690196}%
\pgfsetstrokecolor{currentstroke}%
\pgfsetstrokeopacity{0.050000}%
\pgfsetdash{}{0pt}%
\pgfpathmoveto{\pgfqpoint{0.688192in}{13.017883in}}%
\pgfpathlineto{\pgfqpoint{11.784299in}{13.017883in}}%
\pgfusepath{stroke}%
\end{pgfscope}%
\begin{pgfscope}%
\pgfsetbuttcap%
\pgfsetroundjoin%
\definecolor{currentfill}{rgb}{0.000000,0.000000,0.000000}%
\pgfsetfillcolor{currentfill}%
\pgfsetlinewidth{0.602250pt}%
\definecolor{currentstroke}{rgb}{0.000000,0.000000,0.000000}%
\pgfsetstrokecolor{currentstroke}%
\pgfsetdash{}{0pt}%
\pgfsys@defobject{currentmarker}{\pgfqpoint{-0.027778in}{0.000000in}}{\pgfqpoint{-0.000000in}{0.000000in}}{%
\pgfpathmoveto{\pgfqpoint{-0.000000in}{0.000000in}}%
\pgfpathlineto{\pgfqpoint{-0.027778in}{0.000000in}}%
\pgfusepath{stroke,fill}%
}%
\begin{pgfscope}%
\pgfsys@transformshift{0.688192in}{13.017883in}%
\pgfsys@useobject{currentmarker}{}%
\end{pgfscope}%
\end{pgfscope}%
\begin{pgfscope}%
\pgfpathrectangle{\pgfqpoint{0.688192in}{9.576569in}}{\pgfqpoint{11.096108in}{4.223431in}}%
\pgfusepath{clip}%
\pgfsetrectcap%
\pgfsetroundjoin%
\pgfsetlinewidth{0.803000pt}%
\definecolor{currentstroke}{rgb}{0.690196,0.690196,0.690196}%
\pgfsetstrokecolor{currentstroke}%
\pgfsetstrokeopacity{0.050000}%
\pgfsetdash{}{0pt}%
\pgfpathmoveto{\pgfqpoint{0.688192in}{13.174307in}}%
\pgfpathlineto{\pgfqpoint{11.784299in}{13.174307in}}%
\pgfusepath{stroke}%
\end{pgfscope}%
\begin{pgfscope}%
\pgfsetbuttcap%
\pgfsetroundjoin%
\definecolor{currentfill}{rgb}{0.000000,0.000000,0.000000}%
\pgfsetfillcolor{currentfill}%
\pgfsetlinewidth{0.602250pt}%
\definecolor{currentstroke}{rgb}{0.000000,0.000000,0.000000}%
\pgfsetstrokecolor{currentstroke}%
\pgfsetdash{}{0pt}%
\pgfsys@defobject{currentmarker}{\pgfqpoint{-0.027778in}{0.000000in}}{\pgfqpoint{-0.000000in}{0.000000in}}{%
\pgfpathmoveto{\pgfqpoint{-0.000000in}{0.000000in}}%
\pgfpathlineto{\pgfqpoint{-0.027778in}{0.000000in}}%
\pgfusepath{stroke,fill}%
}%
\begin{pgfscope}%
\pgfsys@transformshift{0.688192in}{13.174307in}%
\pgfsys@useobject{currentmarker}{}%
\end{pgfscope}%
\end{pgfscope}%
\begin{pgfscope}%
\pgfpathrectangle{\pgfqpoint{0.688192in}{9.576569in}}{\pgfqpoint{11.096108in}{4.223431in}}%
\pgfusepath{clip}%
\pgfsetrectcap%
\pgfsetroundjoin%
\pgfsetlinewidth{0.803000pt}%
\definecolor{currentstroke}{rgb}{0.690196,0.690196,0.690196}%
\pgfsetstrokecolor{currentstroke}%
\pgfsetstrokeopacity{0.050000}%
\pgfsetdash{}{0pt}%
\pgfpathmoveto{\pgfqpoint{0.688192in}{13.330730in}}%
\pgfpathlineto{\pgfqpoint{11.784299in}{13.330730in}}%
\pgfusepath{stroke}%
\end{pgfscope}%
\begin{pgfscope}%
\pgfsetbuttcap%
\pgfsetroundjoin%
\definecolor{currentfill}{rgb}{0.000000,0.000000,0.000000}%
\pgfsetfillcolor{currentfill}%
\pgfsetlinewidth{0.602250pt}%
\definecolor{currentstroke}{rgb}{0.000000,0.000000,0.000000}%
\pgfsetstrokecolor{currentstroke}%
\pgfsetdash{}{0pt}%
\pgfsys@defobject{currentmarker}{\pgfqpoint{-0.027778in}{0.000000in}}{\pgfqpoint{-0.000000in}{0.000000in}}{%
\pgfpathmoveto{\pgfqpoint{-0.000000in}{0.000000in}}%
\pgfpathlineto{\pgfqpoint{-0.027778in}{0.000000in}}%
\pgfusepath{stroke,fill}%
}%
\begin{pgfscope}%
\pgfsys@transformshift{0.688192in}{13.330730in}%
\pgfsys@useobject{currentmarker}{}%
\end{pgfscope}%
\end{pgfscope}%
\begin{pgfscope}%
\pgfpathrectangle{\pgfqpoint{0.688192in}{9.576569in}}{\pgfqpoint{11.096108in}{4.223431in}}%
\pgfusepath{clip}%
\pgfsetrectcap%
\pgfsetroundjoin%
\pgfsetlinewidth{0.803000pt}%
\definecolor{currentstroke}{rgb}{0.690196,0.690196,0.690196}%
\pgfsetstrokecolor{currentstroke}%
\pgfsetstrokeopacity{0.050000}%
\pgfsetdash{}{0pt}%
\pgfpathmoveto{\pgfqpoint{0.688192in}{13.643577in}}%
\pgfpathlineto{\pgfqpoint{11.784299in}{13.643577in}}%
\pgfusepath{stroke}%
\end{pgfscope}%
\begin{pgfscope}%
\pgfsetbuttcap%
\pgfsetroundjoin%
\definecolor{currentfill}{rgb}{0.000000,0.000000,0.000000}%
\pgfsetfillcolor{currentfill}%
\pgfsetlinewidth{0.602250pt}%
\definecolor{currentstroke}{rgb}{0.000000,0.000000,0.000000}%
\pgfsetstrokecolor{currentstroke}%
\pgfsetdash{}{0pt}%
\pgfsys@defobject{currentmarker}{\pgfqpoint{-0.027778in}{0.000000in}}{\pgfqpoint{-0.000000in}{0.000000in}}{%
\pgfpathmoveto{\pgfqpoint{-0.000000in}{0.000000in}}%
\pgfpathlineto{\pgfqpoint{-0.027778in}{0.000000in}}%
\pgfusepath{stroke,fill}%
}%
\begin{pgfscope}%
\pgfsys@transformshift{0.688192in}{13.643577in}%
\pgfsys@useobject{currentmarker}{}%
\end{pgfscope}%
\end{pgfscope}%
\begin{pgfscope}%
\pgfpathrectangle{\pgfqpoint{0.688192in}{9.576569in}}{\pgfqpoint{11.096108in}{4.223431in}}%
\pgfusepath{clip}%
\pgfsetrectcap%
\pgfsetroundjoin%
\pgfsetlinewidth{0.803000pt}%
\definecolor{currentstroke}{rgb}{0.690196,0.690196,0.690196}%
\pgfsetstrokecolor{currentstroke}%
\pgfsetstrokeopacity{0.050000}%
\pgfsetdash{}{0pt}%
\pgfpathmoveto{\pgfqpoint{0.688192in}{13.800000in}}%
\pgfpathlineto{\pgfqpoint{11.784299in}{13.800000in}}%
\pgfusepath{stroke}%
\end{pgfscope}%
\begin{pgfscope}%
\pgfsetbuttcap%
\pgfsetroundjoin%
\definecolor{currentfill}{rgb}{0.000000,0.000000,0.000000}%
\pgfsetfillcolor{currentfill}%
\pgfsetlinewidth{0.602250pt}%
\definecolor{currentstroke}{rgb}{0.000000,0.000000,0.000000}%
\pgfsetstrokecolor{currentstroke}%
\pgfsetdash{}{0pt}%
\pgfsys@defobject{currentmarker}{\pgfqpoint{-0.027778in}{0.000000in}}{\pgfqpoint{-0.000000in}{0.000000in}}{%
\pgfpathmoveto{\pgfqpoint{-0.000000in}{0.000000in}}%
\pgfpathlineto{\pgfqpoint{-0.027778in}{0.000000in}}%
\pgfusepath{stroke,fill}%
}%
\begin{pgfscope}%
\pgfsys@transformshift{0.688192in}{13.800000in}%
\pgfsys@useobject{currentmarker}{}%
\end{pgfscope}%
\end{pgfscope}%
\begin{pgfscope}%
\definecolor{textcolor}{rgb}{0.000000,0.000000,0.000000}%
\pgfsetstrokecolor{textcolor}%
\pgfsetfillcolor{textcolor}%
\pgftext[x=0.339583in,y=11.688285in,,bottom,rotate=90.000000]{\color{textcolor}{\rmfamily\fontsize{18.000000}{21.600000}\selectfont\catcode`\^=\active\def^{\ifmmode\sp\else\^{}\fi}\catcode`\%=\active\def%{\%}Osier Capacity (GW)}}%
\end{pgfscope}%
\begin{pgfscope}%
\pgfpathrectangle{\pgfqpoint{0.688192in}{9.576569in}}{\pgfqpoint{11.096108in}{4.223431in}}%
\pgfusepath{clip}%
\pgfsetbuttcap%
\pgfsetroundjoin%
\pgfsetlinewidth{0.941016pt}%
\definecolor{currentstroke}{rgb}{0.240000,0.240000,0.240000}%
\pgfsetstrokecolor{currentstroke}%
\pgfsetdash{}{0pt}%
\pgfpathmoveto{\pgfqpoint{0.799153in}{12.791344in}}%
\pgfpathlineto{\pgfqpoint{1.686841in}{12.791344in}}%
\pgfusepath{stroke}%
\end{pgfscope}%
\begin{pgfscope}%
\pgfpathrectangle{\pgfqpoint{0.688192in}{9.576569in}}{\pgfqpoint{11.096108in}{4.223431in}}%
\pgfusepath{clip}%
\pgfsetbuttcap%
\pgfsetroundjoin%
\pgfsetlinewidth{0.941016pt}%
\definecolor{currentstroke}{rgb}{0.240000,0.240000,0.240000}%
\pgfsetstrokecolor{currentstroke}%
\pgfsetdash{}{0pt}%
\pgfpathmoveto{\pgfqpoint{1.908763in}{9.576569in}}%
\pgfpathlineto{\pgfqpoint{2.796452in}{9.576569in}}%
\pgfusepath{stroke}%
\end{pgfscope}%
\begin{pgfscope}%
\pgfpathrectangle{\pgfqpoint{0.688192in}{9.576569in}}{\pgfqpoint{11.096108in}{4.223431in}}%
\pgfusepath{clip}%
\pgfsetbuttcap%
\pgfsetroundjoin%
\pgfsetlinewidth{0.941016pt}%
\definecolor{currentstroke}{rgb}{0.240000,0.240000,0.240000}%
\pgfsetstrokecolor{currentstroke}%
\pgfsetdash{}{0pt}%
\pgfpathmoveto{\pgfqpoint{3.018374in}{10.329942in}}%
\pgfpathlineto{\pgfqpoint{3.906063in}{10.329942in}}%
\pgfusepath{stroke}%
\end{pgfscope}%
\begin{pgfscope}%
\pgfpathrectangle{\pgfqpoint{0.688192in}{9.576569in}}{\pgfqpoint{11.096108in}{4.223431in}}%
\pgfusepath{clip}%
\pgfsetbuttcap%
\pgfsetroundjoin%
\pgfsetlinewidth{0.941016pt}%
\definecolor{currentstroke}{rgb}{0.240000,0.240000,0.240000}%
\pgfsetstrokecolor{currentstroke}%
\pgfsetdash{}{0pt}%
\pgfpathmoveto{\pgfqpoint{4.127985in}{9.576569in}}%
\pgfpathlineto{\pgfqpoint{5.015674in}{9.576569in}}%
\pgfusepath{stroke}%
\end{pgfscope}%
\begin{pgfscope}%
\pgfpathrectangle{\pgfqpoint{0.688192in}{9.576569in}}{\pgfqpoint{11.096108in}{4.223431in}}%
\pgfusepath{clip}%
\pgfsetbuttcap%
\pgfsetroundjoin%
\pgfsetlinewidth{0.941016pt}%
\definecolor{currentstroke}{rgb}{0.240000,0.240000,0.240000}%
\pgfsetstrokecolor{currentstroke}%
\pgfsetdash{}{0pt}%
\pgfpathmoveto{\pgfqpoint{5.237596in}{9.576569in}}%
\pgfpathlineto{\pgfqpoint{6.125284in}{9.576569in}}%
\pgfusepath{stroke}%
\end{pgfscope}%
\begin{pgfscope}%
\pgfpathrectangle{\pgfqpoint{0.688192in}{9.576569in}}{\pgfqpoint{11.096108in}{4.223431in}}%
\pgfusepath{clip}%
\pgfsetbuttcap%
\pgfsetroundjoin%
\pgfsetlinewidth{0.941016pt}%
\definecolor{currentstroke}{rgb}{0.240000,0.240000,0.240000}%
\pgfsetstrokecolor{currentstroke}%
\pgfsetdash{}{0pt}%
\pgfpathmoveto{\pgfqpoint{6.347207in}{9.576569in}}%
\pgfpathlineto{\pgfqpoint{7.234895in}{9.576569in}}%
\pgfusepath{stroke}%
\end{pgfscope}%
\begin{pgfscope}%
\pgfpathrectangle{\pgfqpoint{0.688192in}{9.576569in}}{\pgfqpoint{11.096108in}{4.223431in}}%
\pgfusepath{clip}%
\pgfsetbuttcap%
\pgfsetroundjoin%
\pgfsetlinewidth{0.941016pt}%
\definecolor{currentstroke}{rgb}{0.240000,0.240000,0.240000}%
\pgfsetstrokecolor{currentstroke}%
\pgfsetdash{}{0pt}%
\pgfpathmoveto{\pgfqpoint{7.456817in}{9.576569in}}%
\pgfpathlineto{\pgfqpoint{8.344506in}{9.576569in}}%
\pgfusepath{stroke}%
\end{pgfscope}%
\begin{pgfscope}%
\pgfpathrectangle{\pgfqpoint{0.688192in}{9.576569in}}{\pgfqpoint{11.096108in}{4.223431in}}%
\pgfusepath{clip}%
\pgfsetbuttcap%
\pgfsetroundjoin%
\pgfsetlinewidth{0.941016pt}%
\definecolor{currentstroke}{rgb}{0.240000,0.240000,0.240000}%
\pgfsetstrokecolor{currentstroke}%
\pgfsetdash{}{0pt}%
\pgfpathmoveto{\pgfqpoint{8.566428in}{10.438118in}}%
\pgfpathlineto{\pgfqpoint{9.454117in}{10.438118in}}%
\pgfusepath{stroke}%
\end{pgfscope}%
\begin{pgfscope}%
\pgfpathrectangle{\pgfqpoint{0.688192in}{9.576569in}}{\pgfqpoint{11.096108in}{4.223431in}}%
\pgfusepath{clip}%
\pgfsetbuttcap%
\pgfsetroundjoin%
\pgfsetlinewidth{0.941016pt}%
\definecolor{currentstroke}{rgb}{0.240000,0.240000,0.240000}%
\pgfsetstrokecolor{currentstroke}%
\pgfsetdash{}{0pt}%
\pgfpathmoveto{\pgfqpoint{9.676039in}{10.226529in}}%
\pgfpathlineto{\pgfqpoint{10.563728in}{10.226529in}}%
\pgfusepath{stroke}%
\end{pgfscope}%
\begin{pgfscope}%
\pgfpathrectangle{\pgfqpoint{0.688192in}{9.576569in}}{\pgfqpoint{11.096108in}{4.223431in}}%
\pgfusepath{clip}%
\pgfsetbuttcap%
\pgfsetroundjoin%
\pgfsetlinewidth{0.941016pt}%
\definecolor{currentstroke}{rgb}{0.240000,0.240000,0.240000}%
\pgfsetstrokecolor{currentstroke}%
\pgfsetdash{}{0pt}%
\pgfpathmoveto{\pgfqpoint{10.785650in}{9.576569in}}%
\pgfpathlineto{\pgfqpoint{11.673338in}{9.576569in}}%
\pgfusepath{stroke}%
\end{pgfscope}%
\begin{pgfscope}%
\pgfsetrectcap%
\pgfsetmiterjoin%
\pgfsetlinewidth{0.803000pt}%
\definecolor{currentstroke}{rgb}{0.000000,0.000000,0.000000}%
\pgfsetstrokecolor{currentstroke}%
\pgfsetdash{}{0pt}%
\pgfpathmoveto{\pgfqpoint{0.688192in}{9.576569in}}%
\pgfpathlineto{\pgfqpoint{0.688192in}{13.800000in}}%
\pgfusepath{stroke}%
\end{pgfscope}%
\begin{pgfscope}%
\pgfsetrectcap%
\pgfsetmiterjoin%
\pgfsetlinewidth{0.803000pt}%
\definecolor{currentstroke}{rgb}{0.000000,0.000000,0.000000}%
\pgfsetstrokecolor{currentstroke}%
\pgfsetdash{}{0pt}%
\pgfpathmoveto{\pgfqpoint{11.784299in}{9.576569in}}%
\pgfpathlineto{\pgfqpoint{11.784299in}{13.800000in}}%
\pgfusepath{stroke}%
\end{pgfscope}%
\begin{pgfscope}%
\pgfsetrectcap%
\pgfsetmiterjoin%
\pgfsetlinewidth{0.803000pt}%
\definecolor{currentstroke}{rgb}{0.000000,0.000000,0.000000}%
\pgfsetstrokecolor{currentstroke}%
\pgfsetdash{}{0pt}%
\pgfpathmoveto{\pgfqpoint{0.688192in}{9.576569in}}%
\pgfpathlineto{\pgfqpoint{11.784299in}{9.576569in}}%
\pgfusepath{stroke}%
\end{pgfscope}%
\begin{pgfscope}%
\pgfsetrectcap%
\pgfsetmiterjoin%
\pgfsetlinewidth{0.803000pt}%
\definecolor{currentstroke}{rgb}{0.000000,0.000000,0.000000}%
\pgfsetstrokecolor{currentstroke}%
\pgfsetdash{}{0pt}%
\pgfpathmoveto{\pgfqpoint{0.688192in}{13.800000in}}%
\pgfpathlineto{\pgfqpoint{11.784299in}{13.800000in}}%
\pgfusepath{stroke}%
\end{pgfscope}%
\begin{pgfscope}%
\pgfsetbuttcap%
\pgfsetmiterjoin%
\definecolor{currentfill}{rgb}{1.000000,1.000000,1.000000}%
\pgfsetfillcolor{currentfill}%
\pgfsetlinewidth{1.003750pt}%
\definecolor{currentstroke}{rgb}{0.000000,0.000000,0.000000}%
\pgfsetstrokecolor{currentstroke}%
\pgfsetdash{}{0pt}%
\pgfpathmoveto{\pgfqpoint{0.842629in}{13.372943in}}%
\pgfpathlineto{\pgfqpoint{1.129478in}{13.372943in}}%
\pgfpathlineto{\pgfqpoint{1.129478in}{13.685721in}}%
\pgfpathlineto{\pgfqpoint{0.842629in}{13.685721in}}%
\pgfpathlineto{\pgfqpoint{0.842629in}{13.372943in}}%
\pgfpathclose%
\pgfusepath{stroke,fill}%
\end{pgfscope}%
\begin{pgfscope}%
\definecolor{textcolor}{rgb}{0.000000,0.000000,0.000000}%
\pgfsetstrokecolor{textcolor}%
\pgfsetfillcolor{textcolor}%
\pgftext[x=0.899018in,y=13.479332in,left,base]{\color{textcolor}{\rmfamily\fontsize{14.000000}{16.800000}\selectfont\catcode`\^=\active\def^{\ifmmode\sp\else\^{}\fi}\catcode`\%=\active\def%{\%}a)}}%
\end{pgfscope}%
\begin{pgfscope}%
\pgfsetbuttcap%
\pgfsetmiterjoin%
\definecolor{currentfill}{rgb}{1.000000,1.000000,1.000000}%
\pgfsetfillcolor{currentfill}%
\pgfsetlinewidth{0.000000pt}%
\definecolor{currentstroke}{rgb}{0.000000,0.000000,0.000000}%
\pgfsetstrokecolor{currentstroke}%
\pgfsetstrokeopacity{0.000000}%
\pgfsetdash{}{0pt}%
\pgfpathmoveto{\pgfqpoint{0.688192in}{5.094806in}}%
\pgfpathlineto{\pgfqpoint{11.784299in}{5.094806in}}%
\pgfpathlineto{\pgfqpoint{11.784299in}{9.318236in}}%
\pgfpathlineto{\pgfqpoint{0.688192in}{9.318236in}}%
\pgfpathlineto{\pgfqpoint{0.688192in}{5.094806in}}%
\pgfpathclose%
\pgfusepath{fill}%
\end{pgfscope}%
\begin{pgfscope}%
\pgfpathrectangle{\pgfqpoint{0.688192in}{5.094806in}}{\pgfqpoint{11.096108in}{4.223431in}}%
\pgfusepath{clip}%
\pgfsetbuttcap%
\pgfsetroundjoin%
\definecolor{currentfill}{rgb}{0.469005,0.634306,0.749958}%
\pgfsetfillcolor{currentfill}%
\pgfsetlinewidth{0.752812pt}%
\definecolor{currentstroke}{rgb}{0.240000,0.240000,0.240000}%
\pgfsetstrokecolor{currentstroke}%
\pgfsetdash{}{0pt}%
\pgfpathmoveto{\pgfqpoint{1.132036in}{7.521066in}}%
\pgfpathlineto{\pgfqpoint{1.353958in}{7.521066in}}%
\pgfpathlineto{\pgfqpoint{1.353958in}{7.616041in}}%
\pgfpathlineto{\pgfqpoint{1.132036in}{7.616041in}}%
\pgfpathlineto{\pgfqpoint{1.132036in}{7.521066in}}%
\pgfpathclose%
\pgfusepath{stroke,fill}%
\end{pgfscope}%
\begin{pgfscope}%
\pgfpathrectangle{\pgfqpoint{0.688192in}{5.094806in}}{\pgfqpoint{11.096108in}{4.223431in}}%
\pgfusepath{clip}%
\pgfsetbuttcap%
\pgfsetroundjoin%
\definecolor{currentfill}{rgb}{0.346402,0.553490,0.697630}%
\pgfsetfillcolor{currentfill}%
\pgfsetlinewidth{0.752812pt}%
\definecolor{currentstroke}{rgb}{0.240000,0.240000,0.240000}%
\pgfsetstrokecolor{currentstroke}%
\pgfsetdash{}{0pt}%
\pgfpathmoveto{\pgfqpoint{1.021075in}{7.616041in}}%
\pgfpathlineto{\pgfqpoint{1.464919in}{7.616041in}}%
\pgfpathlineto{\pgfqpoint{1.464919in}{7.769678in}}%
\pgfpathlineto{\pgfqpoint{1.021075in}{7.769678in}}%
\pgfpathlineto{\pgfqpoint{1.021075in}{7.616041in}}%
\pgfpathclose%
\pgfusepath{stroke,fill}%
\end{pgfscope}%
\begin{pgfscope}%
\pgfpathrectangle{\pgfqpoint{0.688192in}{5.094806in}}{\pgfqpoint{11.096108in}{4.223431in}}%
\pgfusepath{clip}%
\pgfsetbuttcap%
\pgfsetroundjoin%
\definecolor{currentfill}{rgb}{0.194608,0.453431,0.632843}%
\pgfsetfillcolor{currentfill}%
\pgfsetlinewidth{0.752812pt}%
\definecolor{currentstroke}{rgb}{0.240000,0.240000,0.240000}%
\pgfsetstrokecolor{currentstroke}%
\pgfsetdash{}{0pt}%
\pgfpathmoveto{\pgfqpoint{0.799153in}{7.769678in}}%
\pgfpathlineto{\pgfqpoint{1.686841in}{7.769678in}}%
\pgfpathlineto{\pgfqpoint{1.686841in}{8.546202in}}%
\pgfpathlineto{\pgfqpoint{0.799153in}{8.546202in}}%
\pgfpathlineto{\pgfqpoint{0.799153in}{7.769678in}}%
\pgfpathclose%
\pgfusepath{stroke,fill}%
\end{pgfscope}%
\begin{pgfscope}%
\pgfpathrectangle{\pgfqpoint{0.688192in}{5.094806in}}{\pgfqpoint{11.096108in}{4.223431in}}%
\pgfusepath{clip}%
\pgfsetbuttcap%
\pgfsetroundjoin%
\definecolor{currentfill}{rgb}{0.346402,0.553490,0.697630}%
\pgfsetfillcolor{currentfill}%
\pgfsetlinewidth{0.752812pt}%
\definecolor{currentstroke}{rgb}{0.240000,0.240000,0.240000}%
\pgfsetstrokecolor{currentstroke}%
\pgfsetdash{}{0pt}%
\pgfpathmoveto{\pgfqpoint{1.021075in}{8.546202in}}%
\pgfpathlineto{\pgfqpoint{1.464919in}{8.546202in}}%
\pgfpathlineto{\pgfqpoint{1.464919in}{8.658963in}}%
\pgfpathlineto{\pgfqpoint{1.021075in}{8.658963in}}%
\pgfpathlineto{\pgfqpoint{1.021075in}{8.546202in}}%
\pgfpathclose%
\pgfusepath{stroke,fill}%
\end{pgfscope}%
\begin{pgfscope}%
\pgfpathrectangle{\pgfqpoint{0.688192in}{5.094806in}}{\pgfqpoint{11.096108in}{4.223431in}}%
\pgfusepath{clip}%
\pgfsetbuttcap%
\pgfsetroundjoin%
\definecolor{currentfill}{rgb}{0.469005,0.634306,0.749958}%
\pgfsetfillcolor{currentfill}%
\pgfsetlinewidth{0.752812pt}%
\definecolor{currentstroke}{rgb}{0.240000,0.240000,0.240000}%
\pgfsetstrokecolor{currentstroke}%
\pgfsetdash{}{0pt}%
\pgfpathmoveto{\pgfqpoint{1.132036in}{8.658963in}}%
\pgfpathlineto{\pgfqpoint{1.353958in}{8.658963in}}%
\pgfpathlineto{\pgfqpoint{1.353958in}{8.977254in}}%
\pgfpathlineto{\pgfqpoint{1.132036in}{8.977254in}}%
\pgfpathlineto{\pgfqpoint{1.132036in}{8.658963in}}%
\pgfpathclose%
\pgfusepath{stroke,fill}%
\end{pgfscope}%
\begin{pgfscope}%
\pgfpathrectangle{\pgfqpoint{0.688192in}{5.094806in}}{\pgfqpoint{11.096108in}{4.223431in}}%
\pgfusepath{clip}%
\pgfsetbuttcap%
\pgfsetroundjoin%
\pgfsetlinewidth{1.003750pt}%
\definecolor{currentstroke}{rgb}{0.450000,0.450000,0.450000}%
\pgfsetstrokecolor{currentstroke}%
\pgfsetdash{}{0pt}%
\pgfpathmoveto{\pgfqpoint{1.242997in}{7.324157in}}%
\pgfpathcurveto{\pgfqpoint{1.252205in}{7.324157in}}{\pgfqpoint{1.261038in}{7.327816in}}{\pgfqpoint{1.267549in}{7.334327in}}%
\pgfpathcurveto{\pgfqpoint{1.274061in}{7.340839in}}{\pgfqpoint{1.277719in}{7.349671in}}{\pgfqpoint{1.277719in}{7.358880in}}%
\pgfpathcurveto{\pgfqpoint{1.277719in}{7.368088in}}{\pgfqpoint{1.274061in}{7.376921in}}{\pgfqpoint{1.267549in}{7.383432in}}%
\pgfpathcurveto{\pgfqpoint{1.261038in}{7.389943in}}{\pgfqpoint{1.252205in}{7.393602in}}{\pgfqpoint{1.242997in}{7.393602in}}%
\pgfpathcurveto{\pgfqpoint{1.233788in}{7.393602in}}{\pgfqpoint{1.224956in}{7.389943in}}{\pgfqpoint{1.218445in}{7.383432in}}%
\pgfpathcurveto{\pgfqpoint{1.211933in}{7.376921in}}{\pgfqpoint{1.208275in}{7.368088in}}{\pgfqpoint{1.208275in}{7.358880in}}%
\pgfpathcurveto{\pgfqpoint{1.208275in}{7.349671in}}{\pgfqpoint{1.211933in}{7.340839in}}{\pgfqpoint{1.218445in}{7.334327in}}%
\pgfpathcurveto{\pgfqpoint{1.224956in}{7.327816in}}{\pgfqpoint{1.233788in}{7.324157in}}{\pgfqpoint{1.242997in}{7.324157in}}%
\pgfpathlineto{\pgfqpoint{1.242997in}{7.324157in}}%
\pgfpathclose%
\pgfusepath{stroke}%
\end{pgfscope}%
\begin{pgfscope}%
\pgfpathrectangle{\pgfqpoint{0.688192in}{5.094806in}}{\pgfqpoint{11.096108in}{4.223431in}}%
\pgfusepath{clip}%
\pgfsetbuttcap%
\pgfsetroundjoin%
\pgfsetlinewidth{1.003750pt}%
\definecolor{currentstroke}{rgb}{0.450000,0.450000,0.450000}%
\pgfsetstrokecolor{currentstroke}%
\pgfsetdash{}{0pt}%
\pgfpathmoveto{\pgfqpoint{1.242997in}{9.036596in}}%
\pgfpathcurveto{\pgfqpoint{1.252205in}{9.036596in}}{\pgfqpoint{1.261038in}{9.040255in}}{\pgfqpoint{1.267549in}{9.046766in}}%
\pgfpathcurveto{\pgfqpoint{1.274061in}{9.053277in}}{\pgfqpoint{1.277719in}{9.062110in}}{\pgfqpoint{1.277719in}{9.071318in}}%
\pgfpathcurveto{\pgfqpoint{1.277719in}{9.080527in}}{\pgfqpoint{1.274061in}{9.089359in}}{\pgfqpoint{1.267549in}{9.095871in}}%
\pgfpathcurveto{\pgfqpoint{1.261038in}{9.102382in}}{\pgfqpoint{1.252205in}{9.106041in}}{\pgfqpoint{1.242997in}{9.106041in}}%
\pgfpathcurveto{\pgfqpoint{1.233788in}{9.106041in}}{\pgfqpoint{1.224956in}{9.102382in}}{\pgfqpoint{1.218445in}{9.095871in}}%
\pgfpathcurveto{\pgfqpoint{1.211933in}{9.089359in}}{\pgfqpoint{1.208275in}{9.080527in}}{\pgfqpoint{1.208275in}{9.071318in}}%
\pgfpathcurveto{\pgfqpoint{1.208275in}{9.062110in}}{\pgfqpoint{1.211933in}{9.053277in}}{\pgfqpoint{1.218445in}{9.046766in}}%
\pgfpathcurveto{\pgfqpoint{1.224956in}{9.040255in}}{\pgfqpoint{1.233788in}{9.036596in}}{\pgfqpoint{1.242997in}{9.036596in}}%
\pgfpathlineto{\pgfqpoint{1.242997in}{9.036596in}}%
\pgfpathclose%
\pgfusepath{stroke}%
\end{pgfscope}%
\begin{pgfscope}%
\pgfpathrectangle{\pgfqpoint{0.688192in}{5.094806in}}{\pgfqpoint{11.096108in}{4.223431in}}%
\pgfusepath{clip}%
\pgfsetbuttcap%
\pgfsetroundjoin%
\pgfsetlinewidth{1.003750pt}%
\definecolor{currentstroke}{rgb}{0.450000,0.450000,0.450000}%
\pgfsetstrokecolor{currentstroke}%
\pgfsetdash{}{0pt}%
\pgfpathmoveto{\pgfqpoint{1.242997in}{7.321853in}}%
\pgfpathcurveto{\pgfqpoint{1.252205in}{7.321853in}}{\pgfqpoint{1.261038in}{7.325511in}}{\pgfqpoint{1.267549in}{7.332023in}}%
\pgfpathcurveto{\pgfqpoint{1.274061in}{7.338534in}}{\pgfqpoint{1.277719in}{7.347367in}}{\pgfqpoint{1.277719in}{7.356575in}}%
\pgfpathcurveto{\pgfqpoint{1.277719in}{7.365783in}}{\pgfqpoint{1.274061in}{7.374616in}}{\pgfqpoint{1.267549in}{7.381127in}}%
\pgfpathcurveto{\pgfqpoint{1.261038in}{7.387639in}}{\pgfqpoint{1.252205in}{7.391297in}}{\pgfqpoint{1.242997in}{7.391297in}}%
\pgfpathcurveto{\pgfqpoint{1.233788in}{7.391297in}}{\pgfqpoint{1.224956in}{7.387639in}}{\pgfqpoint{1.218445in}{7.381127in}}%
\pgfpathcurveto{\pgfqpoint{1.211933in}{7.374616in}}{\pgfqpoint{1.208275in}{7.365783in}}{\pgfqpoint{1.208275in}{7.356575in}}%
\pgfpathcurveto{\pgfqpoint{1.208275in}{7.347367in}}{\pgfqpoint{1.211933in}{7.338534in}}{\pgfqpoint{1.218445in}{7.332023in}}%
\pgfpathcurveto{\pgfqpoint{1.224956in}{7.325511in}}{\pgfqpoint{1.233788in}{7.321853in}}{\pgfqpoint{1.242997in}{7.321853in}}%
\pgfpathlineto{\pgfqpoint{1.242997in}{7.321853in}}%
\pgfpathclose%
\pgfusepath{stroke}%
\end{pgfscope}%
\begin{pgfscope}%
\pgfpathrectangle{\pgfqpoint{0.688192in}{5.094806in}}{\pgfqpoint{11.096108in}{4.223431in}}%
\pgfusepath{clip}%
\pgfsetbuttcap%
\pgfsetroundjoin%
\pgfsetlinewidth{1.003750pt}%
\definecolor{currentstroke}{rgb}{0.450000,0.450000,0.450000}%
\pgfsetstrokecolor{currentstroke}%
\pgfsetdash{}{0pt}%
\pgfpathmoveto{\pgfqpoint{1.242997in}{8.995332in}}%
\pgfpathcurveto{\pgfqpoint{1.252205in}{8.995332in}}{\pgfqpoint{1.261038in}{8.998991in}}{\pgfqpoint{1.267549in}{9.005502in}}%
\pgfpathcurveto{\pgfqpoint{1.274061in}{9.012014in}}{\pgfqpoint{1.277719in}{9.020846in}}{\pgfqpoint{1.277719in}{9.030055in}}%
\pgfpathcurveto{\pgfqpoint{1.277719in}{9.039263in}}{\pgfqpoint{1.274061in}{9.048095in}}{\pgfqpoint{1.267549in}{9.054607in}}%
\pgfpathcurveto{\pgfqpoint{1.261038in}{9.061118in}}{\pgfqpoint{1.252205in}{9.064777in}}{\pgfqpoint{1.242997in}{9.064777in}}%
\pgfpathcurveto{\pgfqpoint{1.233788in}{9.064777in}}{\pgfqpoint{1.224956in}{9.061118in}}{\pgfqpoint{1.218445in}{9.054607in}}%
\pgfpathcurveto{\pgfqpoint{1.211933in}{9.048095in}}{\pgfqpoint{1.208275in}{9.039263in}}{\pgfqpoint{1.208275in}{9.030055in}}%
\pgfpathcurveto{\pgfqpoint{1.208275in}{9.020846in}}{\pgfqpoint{1.211933in}{9.012014in}}{\pgfqpoint{1.218445in}{9.005502in}}%
\pgfpathcurveto{\pgfqpoint{1.224956in}{8.998991in}}{\pgfqpoint{1.233788in}{8.995332in}}{\pgfqpoint{1.242997in}{8.995332in}}%
\pgfpathlineto{\pgfqpoint{1.242997in}{8.995332in}}%
\pgfpathclose%
\pgfusepath{stroke}%
\end{pgfscope}%
\begin{pgfscope}%
\pgfpathrectangle{\pgfqpoint{0.688192in}{5.094806in}}{\pgfqpoint{11.096108in}{4.223431in}}%
\pgfusepath{clip}%
\pgfsetbuttcap%
\pgfsetroundjoin%
\definecolor{currentfill}{rgb}{0.907545,0.666205,0.454979}%
\pgfsetfillcolor{currentfill}%
\pgfsetlinewidth{0.752812pt}%
\definecolor{currentstroke}{rgb}{0.240000,0.240000,0.240000}%
\pgfsetstrokecolor{currentstroke}%
\pgfsetdash{}{0pt}%
\pgfpathmoveto{\pgfqpoint{2.241647in}{5.094806in}}%
\pgfpathlineto{\pgfqpoint{2.463569in}{5.094806in}}%
\pgfpathlineto{\pgfqpoint{2.463569in}{5.094806in}}%
\pgfpathlineto{\pgfqpoint{2.241647in}{5.094806in}}%
\pgfpathlineto{\pgfqpoint{2.241647in}{5.094806in}}%
\pgfpathclose%
\pgfusepath{stroke,fill}%
\end{pgfscope}%
\begin{pgfscope}%
\pgfpathrectangle{\pgfqpoint{0.688192in}{5.094806in}}{\pgfqpoint{11.096108in}{4.223431in}}%
\pgfusepath{clip}%
\pgfsetbuttcap%
\pgfsetroundjoin%
\definecolor{currentfill}{rgb}{0.896070,0.594353,0.329006}%
\pgfsetfillcolor{currentfill}%
\pgfsetlinewidth{0.752812pt}%
\definecolor{currentstroke}{rgb}{0.240000,0.240000,0.240000}%
\pgfsetstrokecolor{currentstroke}%
\pgfsetdash{}{0pt}%
\pgfpathmoveto{\pgfqpoint{2.130686in}{5.094806in}}%
\pgfpathlineto{\pgfqpoint{2.574530in}{5.094806in}}%
\pgfpathlineto{\pgfqpoint{2.574530in}{5.094806in}}%
\pgfpathlineto{\pgfqpoint{2.130686in}{5.094806in}}%
\pgfpathlineto{\pgfqpoint{2.130686in}{5.094806in}}%
\pgfpathclose%
\pgfusepath{stroke,fill}%
\end{pgfscope}%
\begin{pgfscope}%
\pgfpathrectangle{\pgfqpoint{0.688192in}{5.094806in}}{\pgfqpoint{11.096108in}{4.223431in}}%
\pgfusepath{clip}%
\pgfsetbuttcap%
\pgfsetroundjoin%
\definecolor{currentfill}{rgb}{0.881863,0.505392,0.173039}%
\pgfsetfillcolor{currentfill}%
\pgfsetlinewidth{0.752812pt}%
\definecolor{currentstroke}{rgb}{0.240000,0.240000,0.240000}%
\pgfsetstrokecolor{currentstroke}%
\pgfsetdash{}{0pt}%
\pgfpathmoveto{\pgfqpoint{1.908763in}{5.094806in}}%
\pgfpathlineto{\pgfqpoint{2.796452in}{5.094806in}}%
\pgfpathlineto{\pgfqpoint{2.796452in}{5.312081in}}%
\pgfpathlineto{\pgfqpoint{1.908763in}{5.312081in}}%
\pgfpathlineto{\pgfqpoint{1.908763in}{5.094806in}}%
\pgfpathclose%
\pgfusepath{stroke,fill}%
\end{pgfscope}%
\begin{pgfscope}%
\pgfpathrectangle{\pgfqpoint{0.688192in}{5.094806in}}{\pgfqpoint{11.096108in}{4.223431in}}%
\pgfusepath{clip}%
\pgfsetbuttcap%
\pgfsetroundjoin%
\definecolor{currentfill}{rgb}{0.896070,0.594353,0.329006}%
\pgfsetfillcolor{currentfill}%
\pgfsetlinewidth{0.752812pt}%
\definecolor{currentstroke}{rgb}{0.240000,0.240000,0.240000}%
\pgfsetstrokecolor{currentstroke}%
\pgfsetdash{}{0pt}%
\pgfpathmoveto{\pgfqpoint{2.130686in}{5.312081in}}%
\pgfpathlineto{\pgfqpoint{2.574530in}{5.312081in}}%
\pgfpathlineto{\pgfqpoint{2.574530in}{5.883325in}}%
\pgfpathlineto{\pgfqpoint{2.130686in}{5.883325in}}%
\pgfpathlineto{\pgfqpoint{2.130686in}{5.312081in}}%
\pgfpathclose%
\pgfusepath{stroke,fill}%
\end{pgfscope}%
\begin{pgfscope}%
\pgfpathrectangle{\pgfqpoint{0.688192in}{5.094806in}}{\pgfqpoint{11.096108in}{4.223431in}}%
\pgfusepath{clip}%
\pgfsetbuttcap%
\pgfsetroundjoin%
\definecolor{currentfill}{rgb}{0.907545,0.666205,0.454979}%
\pgfsetfillcolor{currentfill}%
\pgfsetlinewidth{0.752812pt}%
\definecolor{currentstroke}{rgb}{0.240000,0.240000,0.240000}%
\pgfsetstrokecolor{currentstroke}%
\pgfsetdash{}{0pt}%
\pgfpathmoveto{\pgfqpoint{2.241647in}{5.883325in}}%
\pgfpathlineto{\pgfqpoint{2.463569in}{5.883325in}}%
\pgfpathlineto{\pgfqpoint{2.463569in}{5.995289in}}%
\pgfpathlineto{\pgfqpoint{2.241647in}{5.995289in}}%
\pgfpathlineto{\pgfqpoint{2.241647in}{5.883325in}}%
\pgfpathclose%
\pgfusepath{stroke,fill}%
\end{pgfscope}%
\begin{pgfscope}%
\pgfpathrectangle{\pgfqpoint{0.688192in}{5.094806in}}{\pgfqpoint{11.096108in}{4.223431in}}%
\pgfusepath{clip}%
\pgfsetbuttcap%
\pgfsetroundjoin%
\pgfsetlinewidth{1.003750pt}%
\definecolor{currentstroke}{rgb}{0.450000,0.450000,0.450000}%
\pgfsetstrokecolor{currentstroke}%
\pgfsetdash{}{0pt}%
\pgfpathmoveto{\pgfqpoint{2.352608in}{5.985380in}}%
\pgfpathcurveto{\pgfqpoint{2.361816in}{5.985380in}}{\pgfqpoint{2.370649in}{5.989039in}}{\pgfqpoint{2.377160in}{5.995550in}}%
\pgfpathcurveto{\pgfqpoint{2.383671in}{6.002062in}}{\pgfqpoint{2.387330in}{6.010894in}}{\pgfqpoint{2.387330in}{6.020103in}}%
\pgfpathcurveto{\pgfqpoint{2.387330in}{6.029311in}}{\pgfqpoint{2.383671in}{6.038144in}}{\pgfqpoint{2.377160in}{6.044655in}}%
\pgfpathcurveto{\pgfqpoint{2.370649in}{6.051166in}}{\pgfqpoint{2.361816in}{6.054825in}}{\pgfqpoint{2.352608in}{6.054825in}}%
\pgfpathcurveto{\pgfqpoint{2.343399in}{6.054825in}}{\pgfqpoint{2.334567in}{6.051166in}}{\pgfqpoint{2.328055in}{6.044655in}}%
\pgfpathcurveto{\pgfqpoint{2.321544in}{6.038144in}}{\pgfqpoint{2.317885in}{6.029311in}}{\pgfqpoint{2.317885in}{6.020103in}}%
\pgfpathcurveto{\pgfqpoint{2.317885in}{6.010894in}}{\pgfqpoint{2.321544in}{6.002062in}}{\pgfqpoint{2.328055in}{5.995550in}}%
\pgfpathcurveto{\pgfqpoint{2.334567in}{5.989039in}}{\pgfqpoint{2.343399in}{5.985380in}}{\pgfqpoint{2.352608in}{5.985380in}}%
\pgfpathlineto{\pgfqpoint{2.352608in}{5.985380in}}%
\pgfpathclose%
\pgfusepath{stroke}%
\end{pgfscope}%
\begin{pgfscope}%
\pgfpathrectangle{\pgfqpoint{0.688192in}{5.094806in}}{\pgfqpoint{11.096108in}{4.223431in}}%
\pgfusepath{clip}%
\pgfsetbuttcap%
\pgfsetroundjoin%
\pgfsetlinewidth{1.003750pt}%
\definecolor{currentstroke}{rgb}{0.450000,0.450000,0.450000}%
\pgfsetstrokecolor{currentstroke}%
\pgfsetdash{}{0pt}%
\pgfpathmoveto{\pgfqpoint{2.352608in}{5.972414in}}%
\pgfpathcurveto{\pgfqpoint{2.361816in}{5.972414in}}{\pgfqpoint{2.370649in}{5.976073in}}{\pgfqpoint{2.377160in}{5.982584in}}%
\pgfpathcurveto{\pgfqpoint{2.383671in}{5.989095in}}{\pgfqpoint{2.387330in}{5.997928in}}{\pgfqpoint{2.387330in}{6.007136in}}%
\pgfpathcurveto{\pgfqpoint{2.387330in}{6.016345in}}{\pgfqpoint{2.383671in}{6.025177in}}{\pgfqpoint{2.377160in}{6.031689in}}%
\pgfpathcurveto{\pgfqpoint{2.370649in}{6.038200in}}{\pgfqpoint{2.361816in}{6.041859in}}{\pgfqpoint{2.352608in}{6.041859in}}%
\pgfpathcurveto{\pgfqpoint{2.343399in}{6.041859in}}{\pgfqpoint{2.334567in}{6.038200in}}{\pgfqpoint{2.328055in}{6.031689in}}%
\pgfpathcurveto{\pgfqpoint{2.321544in}{6.025177in}}{\pgfqpoint{2.317885in}{6.016345in}}{\pgfqpoint{2.317885in}{6.007136in}}%
\pgfpathcurveto{\pgfqpoint{2.317885in}{5.997928in}}{\pgfqpoint{2.321544in}{5.989095in}}{\pgfqpoint{2.328055in}{5.982584in}}%
\pgfpathcurveto{\pgfqpoint{2.334567in}{5.976073in}}{\pgfqpoint{2.343399in}{5.972414in}}{\pgfqpoint{2.352608in}{5.972414in}}%
\pgfpathlineto{\pgfqpoint{2.352608in}{5.972414in}}%
\pgfpathclose%
\pgfusepath{stroke}%
\end{pgfscope}%
\begin{pgfscope}%
\pgfpathrectangle{\pgfqpoint{0.688192in}{5.094806in}}{\pgfqpoint{11.096108in}{4.223431in}}%
\pgfusepath{clip}%
\pgfsetbuttcap%
\pgfsetroundjoin%
\definecolor{currentfill}{rgb}{0.485914,0.710682,0.485881}%
\pgfsetfillcolor{currentfill}%
\pgfsetlinewidth{0.752812pt}%
\definecolor{currentstroke}{rgb}{0.240000,0.240000,0.240000}%
\pgfsetstrokecolor{currentstroke}%
\pgfsetdash{}{0pt}%
\pgfpathmoveto{\pgfqpoint{3.351257in}{5.094806in}}%
\pgfpathlineto{\pgfqpoint{3.573180in}{5.094806in}}%
\pgfpathlineto{\pgfqpoint{3.573180in}{5.094806in}}%
\pgfpathlineto{\pgfqpoint{3.351257in}{5.094806in}}%
\pgfpathlineto{\pgfqpoint{3.351257in}{5.094806in}}%
\pgfpathclose%
\pgfusepath{stroke,fill}%
\end{pgfscope}%
\begin{pgfscope}%
\pgfpathrectangle{\pgfqpoint{0.688192in}{5.094806in}}{\pgfqpoint{11.096108in}{4.223431in}}%
\pgfusepath{clip}%
\pgfsetbuttcap%
\pgfsetroundjoin%
\definecolor{currentfill}{rgb}{0.371306,0.648087,0.371288}%
\pgfsetfillcolor{currentfill}%
\pgfsetlinewidth{0.752812pt}%
\definecolor{currentstroke}{rgb}{0.240000,0.240000,0.240000}%
\pgfsetstrokecolor{currentstroke}%
\pgfsetdash{}{0pt}%
\pgfpathmoveto{\pgfqpoint{3.240296in}{5.094806in}}%
\pgfpathlineto{\pgfqpoint{3.684141in}{5.094806in}}%
\pgfpathlineto{\pgfqpoint{3.684141in}{5.094806in}}%
\pgfpathlineto{\pgfqpoint{3.240296in}{5.094806in}}%
\pgfpathlineto{\pgfqpoint{3.240296in}{5.094806in}}%
\pgfpathclose%
\pgfusepath{stroke,fill}%
\end{pgfscope}%
\begin{pgfscope}%
\pgfpathrectangle{\pgfqpoint{0.688192in}{5.094806in}}{\pgfqpoint{11.096108in}{4.223431in}}%
\pgfusepath{clip}%
\pgfsetbuttcap%
\pgfsetroundjoin%
\definecolor{currentfill}{rgb}{0.229412,0.570588,0.229412}%
\pgfsetfillcolor{currentfill}%
\pgfsetlinewidth{0.752812pt}%
\definecolor{currentstroke}{rgb}{0.240000,0.240000,0.240000}%
\pgfsetstrokecolor{currentstroke}%
\pgfsetdash{}{0pt}%
\pgfpathmoveto{\pgfqpoint{3.018374in}{5.094806in}}%
\pgfpathlineto{\pgfqpoint{3.906063in}{5.094806in}}%
\pgfpathlineto{\pgfqpoint{3.906063in}{5.849456in}}%
\pgfpathlineto{\pgfqpoint{3.018374in}{5.849456in}}%
\pgfpathlineto{\pgfqpoint{3.018374in}{5.094806in}}%
\pgfpathclose%
\pgfusepath{stroke,fill}%
\end{pgfscope}%
\begin{pgfscope}%
\pgfpathrectangle{\pgfqpoint{0.688192in}{5.094806in}}{\pgfqpoint{11.096108in}{4.223431in}}%
\pgfusepath{clip}%
\pgfsetbuttcap%
\pgfsetroundjoin%
\definecolor{currentfill}{rgb}{0.371306,0.648087,0.371288}%
\pgfsetfillcolor{currentfill}%
\pgfsetlinewidth{0.752812pt}%
\definecolor{currentstroke}{rgb}{0.240000,0.240000,0.240000}%
\pgfsetstrokecolor{currentstroke}%
\pgfsetdash{}{0pt}%
\pgfpathmoveto{\pgfqpoint{3.240296in}{5.849456in}}%
\pgfpathlineto{\pgfqpoint{3.684141in}{5.849456in}}%
\pgfpathlineto{\pgfqpoint{3.684141in}{5.913591in}}%
\pgfpathlineto{\pgfqpoint{3.240296in}{5.913591in}}%
\pgfpathlineto{\pgfqpoint{3.240296in}{5.849456in}}%
\pgfpathclose%
\pgfusepath{stroke,fill}%
\end{pgfscope}%
\begin{pgfscope}%
\pgfpathrectangle{\pgfqpoint{0.688192in}{5.094806in}}{\pgfqpoint{11.096108in}{4.223431in}}%
\pgfusepath{clip}%
\pgfsetbuttcap%
\pgfsetroundjoin%
\definecolor{currentfill}{rgb}{0.485914,0.710682,0.485881}%
\pgfsetfillcolor{currentfill}%
\pgfsetlinewidth{0.752812pt}%
\definecolor{currentstroke}{rgb}{0.240000,0.240000,0.240000}%
\pgfsetstrokecolor{currentstroke}%
\pgfsetdash{}{0pt}%
\pgfpathmoveto{\pgfqpoint{3.351257in}{5.913591in}}%
\pgfpathlineto{\pgfqpoint{3.573180in}{5.913591in}}%
\pgfpathlineto{\pgfqpoint{3.573180in}{5.970376in}}%
\pgfpathlineto{\pgfqpoint{3.351257in}{5.970376in}}%
\pgfpathlineto{\pgfqpoint{3.351257in}{5.913591in}}%
\pgfpathclose%
\pgfusepath{stroke,fill}%
\end{pgfscope}%
\begin{pgfscope}%
\pgfpathrectangle{\pgfqpoint{0.688192in}{5.094806in}}{\pgfqpoint{11.096108in}{4.223431in}}%
\pgfusepath{clip}%
\pgfsetbuttcap%
\pgfsetroundjoin%
\pgfsetlinewidth{1.003750pt}%
\definecolor{currentstroke}{rgb}{0.450000,0.450000,0.450000}%
\pgfsetstrokecolor{currentstroke}%
\pgfsetdash{}{0pt}%
\pgfpathmoveto{\pgfqpoint{3.462219in}{6.128172in}}%
\pgfpathcurveto{\pgfqpoint{3.471427in}{6.128172in}}{\pgfqpoint{3.480259in}{6.131830in}}{\pgfqpoint{3.486771in}{6.138342in}}%
\pgfpathcurveto{\pgfqpoint{3.493282in}{6.144853in}}{\pgfqpoint{3.496941in}{6.153686in}}{\pgfqpoint{3.496941in}{6.162894in}}%
\pgfpathcurveto{\pgfqpoint{3.496941in}{6.172102in}}{\pgfqpoint{3.493282in}{6.180935in}}{\pgfqpoint{3.486771in}{6.187446in}}%
\pgfpathcurveto{\pgfqpoint{3.480259in}{6.193958in}}{\pgfqpoint{3.471427in}{6.197616in}}{\pgfqpoint{3.462219in}{6.197616in}}%
\pgfpathcurveto{\pgfqpoint{3.453010in}{6.197616in}}{\pgfqpoint{3.444178in}{6.193958in}}{\pgfqpoint{3.437666in}{6.187446in}}%
\pgfpathcurveto{\pgfqpoint{3.431155in}{6.180935in}}{\pgfqpoint{3.427496in}{6.172102in}}{\pgfqpoint{3.427496in}{6.162894in}}%
\pgfpathcurveto{\pgfqpoint{3.427496in}{6.153686in}}{\pgfqpoint{3.431155in}{6.144853in}}{\pgfqpoint{3.437666in}{6.138342in}}%
\pgfpathcurveto{\pgfqpoint{3.444178in}{6.131830in}}{\pgfqpoint{3.453010in}{6.128172in}}{\pgfqpoint{3.462219in}{6.128172in}}%
\pgfpathlineto{\pgfqpoint{3.462219in}{6.128172in}}%
\pgfpathclose%
\pgfusepath{stroke}%
\end{pgfscope}%
\begin{pgfscope}%
\pgfpathrectangle{\pgfqpoint{0.688192in}{5.094806in}}{\pgfqpoint{11.096108in}{4.223431in}}%
\pgfusepath{clip}%
\pgfsetbuttcap%
\pgfsetroundjoin%
\pgfsetlinewidth{1.003750pt}%
\definecolor{currentstroke}{rgb}{0.450000,0.450000,0.450000}%
\pgfsetstrokecolor{currentstroke}%
\pgfsetdash{}{0pt}%
\pgfpathmoveto{\pgfqpoint{3.462219in}{6.144725in}}%
\pgfpathcurveto{\pgfqpoint{3.471427in}{6.144725in}}{\pgfqpoint{3.480259in}{6.148384in}}{\pgfqpoint{3.486771in}{6.154895in}}%
\pgfpathcurveto{\pgfqpoint{3.493282in}{6.161407in}}{\pgfqpoint{3.496941in}{6.170239in}}{\pgfqpoint{3.496941in}{6.179448in}}%
\pgfpathcurveto{\pgfqpoint{3.496941in}{6.188656in}}{\pgfqpoint{3.493282in}{6.197489in}}{\pgfqpoint{3.486771in}{6.204000in}}%
\pgfpathcurveto{\pgfqpoint{3.480259in}{6.210511in}}{\pgfqpoint{3.471427in}{6.214170in}}{\pgfqpoint{3.462219in}{6.214170in}}%
\pgfpathcurveto{\pgfqpoint{3.453010in}{6.214170in}}{\pgfqpoint{3.444178in}{6.210511in}}{\pgfqpoint{3.437666in}{6.204000in}}%
\pgfpathcurveto{\pgfqpoint{3.431155in}{6.197489in}}{\pgfqpoint{3.427496in}{6.188656in}}{\pgfqpoint{3.427496in}{6.179448in}}%
\pgfpathcurveto{\pgfqpoint{3.427496in}{6.170239in}}{\pgfqpoint{3.431155in}{6.161407in}}{\pgfqpoint{3.437666in}{6.154895in}}%
\pgfpathcurveto{\pgfqpoint{3.444178in}{6.148384in}}{\pgfqpoint{3.453010in}{6.144725in}}{\pgfqpoint{3.462219in}{6.144725in}}%
\pgfpathlineto{\pgfqpoint{3.462219in}{6.144725in}}%
\pgfpathclose%
\pgfusepath{stroke}%
\end{pgfscope}%
\begin{pgfscope}%
\pgfpathrectangle{\pgfqpoint{0.688192in}{5.094806in}}{\pgfqpoint{11.096108in}{4.223431in}}%
\pgfusepath{clip}%
\pgfsetbuttcap%
\pgfsetroundjoin%
\definecolor{currentfill}{rgb}{0.825642,0.497939,0.499757}%
\pgfsetfillcolor{currentfill}%
\pgfsetlinewidth{0.752812pt}%
\definecolor{currentstroke}{rgb}{0.240000,0.240000,0.240000}%
\pgfsetstrokecolor{currentstroke}%
\pgfsetdash{}{0pt}%
\pgfpathmoveto{\pgfqpoint{4.460868in}{5.094806in}}%
\pgfpathlineto{\pgfqpoint{4.682790in}{5.094806in}}%
\pgfpathlineto{\pgfqpoint{4.682790in}{5.094806in}}%
\pgfpathlineto{\pgfqpoint{4.460868in}{5.094806in}}%
\pgfpathlineto{\pgfqpoint{4.460868in}{5.094806in}}%
\pgfpathclose%
\pgfusepath{stroke,fill}%
\end{pgfscope}%
\begin{pgfscope}%
\pgfpathrectangle{\pgfqpoint{0.688192in}{5.094806in}}{\pgfqpoint{11.096108in}{4.223431in}}%
\pgfusepath{clip}%
\pgfsetbuttcap%
\pgfsetroundjoin%
\definecolor{currentfill}{rgb}{0.793378,0.382120,0.384440}%
\pgfsetfillcolor{currentfill}%
\pgfsetlinewidth{0.752812pt}%
\definecolor{currentstroke}{rgb}{0.240000,0.240000,0.240000}%
\pgfsetstrokecolor{currentstroke}%
\pgfsetdash{}{0pt}%
\pgfpathmoveto{\pgfqpoint{4.349907in}{5.094806in}}%
\pgfpathlineto{\pgfqpoint{4.793751in}{5.094806in}}%
\pgfpathlineto{\pgfqpoint{4.793751in}{5.094806in}}%
\pgfpathlineto{\pgfqpoint{4.349907in}{5.094806in}}%
\pgfpathlineto{\pgfqpoint{4.349907in}{5.094806in}}%
\pgfpathclose%
\pgfusepath{stroke,fill}%
\end{pgfscope}%
\begin{pgfscope}%
\pgfpathrectangle{\pgfqpoint{0.688192in}{5.094806in}}{\pgfqpoint{11.096108in}{4.223431in}}%
\pgfusepath{clip}%
\pgfsetbuttcap%
\pgfsetroundjoin%
\definecolor{currentfill}{rgb}{0.753431,0.238725,0.241667}%
\pgfsetfillcolor{currentfill}%
\pgfsetlinewidth{0.752812pt}%
\definecolor{currentstroke}{rgb}{0.240000,0.240000,0.240000}%
\pgfsetstrokecolor{currentstroke}%
\pgfsetdash{}{0pt}%
\pgfpathmoveto{\pgfqpoint{4.127985in}{5.094806in}}%
\pgfpathlineto{\pgfqpoint{5.015674in}{5.094806in}}%
\pgfpathlineto{\pgfqpoint{5.015674in}{5.771970in}}%
\pgfpathlineto{\pgfqpoint{4.127985in}{5.771970in}}%
\pgfpathlineto{\pgfqpoint{4.127985in}{5.094806in}}%
\pgfpathclose%
\pgfusepath{stroke,fill}%
\end{pgfscope}%
\begin{pgfscope}%
\pgfpathrectangle{\pgfqpoint{0.688192in}{5.094806in}}{\pgfqpoint{11.096108in}{4.223431in}}%
\pgfusepath{clip}%
\pgfsetbuttcap%
\pgfsetroundjoin%
\definecolor{currentfill}{rgb}{0.793378,0.382120,0.384440}%
\pgfsetfillcolor{currentfill}%
\pgfsetlinewidth{0.752812pt}%
\definecolor{currentstroke}{rgb}{0.240000,0.240000,0.240000}%
\pgfsetstrokecolor{currentstroke}%
\pgfsetdash{}{0pt}%
\pgfpathmoveto{\pgfqpoint{4.349907in}{5.771970in}}%
\pgfpathlineto{\pgfqpoint{4.793751in}{5.771970in}}%
\pgfpathlineto{\pgfqpoint{4.793751in}{5.948816in}}%
\pgfpathlineto{\pgfqpoint{4.349907in}{5.948816in}}%
\pgfpathlineto{\pgfqpoint{4.349907in}{5.771970in}}%
\pgfpathclose%
\pgfusepath{stroke,fill}%
\end{pgfscope}%
\begin{pgfscope}%
\pgfpathrectangle{\pgfqpoint{0.688192in}{5.094806in}}{\pgfqpoint{11.096108in}{4.223431in}}%
\pgfusepath{clip}%
\pgfsetbuttcap%
\pgfsetroundjoin%
\definecolor{currentfill}{rgb}{0.825642,0.497939,0.499757}%
\pgfsetfillcolor{currentfill}%
\pgfsetlinewidth{0.752812pt}%
\definecolor{currentstroke}{rgb}{0.240000,0.240000,0.240000}%
\pgfsetstrokecolor{currentstroke}%
\pgfsetdash{}{0pt}%
\pgfpathmoveto{\pgfqpoint{4.460868in}{5.948816in}}%
\pgfpathlineto{\pgfqpoint{4.682790in}{5.948816in}}%
\pgfpathlineto{\pgfqpoint{4.682790in}{6.097166in}}%
\pgfpathlineto{\pgfqpoint{4.460868in}{6.097166in}}%
\pgfpathlineto{\pgfqpoint{4.460868in}{5.948816in}}%
\pgfpathclose%
\pgfusepath{stroke,fill}%
\end{pgfscope}%
\begin{pgfscope}%
\pgfpathrectangle{\pgfqpoint{0.688192in}{5.094806in}}{\pgfqpoint{11.096108in}{4.223431in}}%
\pgfusepath{clip}%
\pgfsetbuttcap%
\pgfsetroundjoin%
\pgfsetlinewidth{1.003750pt}%
\definecolor{currentstroke}{rgb}{0.450000,0.450000,0.450000}%
\pgfsetstrokecolor{currentstroke}%
\pgfsetdash{}{0pt}%
\pgfpathmoveto{\pgfqpoint{4.571829in}{6.065856in}}%
\pgfpathcurveto{\pgfqpoint{4.581038in}{6.065856in}}{\pgfqpoint{4.589870in}{6.069515in}}{\pgfqpoint{4.596382in}{6.076026in}}%
\pgfpathcurveto{\pgfqpoint{4.602893in}{6.082538in}}{\pgfqpoint{4.606552in}{6.091370in}}{\pgfqpoint{4.606552in}{6.100579in}}%
\pgfpathcurveto{\pgfqpoint{4.606552in}{6.109787in}}{\pgfqpoint{4.602893in}{6.118620in}}{\pgfqpoint{4.596382in}{6.125131in}}%
\pgfpathcurveto{\pgfqpoint{4.589870in}{6.131642in}}{\pgfqpoint{4.581038in}{6.135301in}}{\pgfqpoint{4.571829in}{6.135301in}}%
\pgfpathcurveto{\pgfqpoint{4.562621in}{6.135301in}}{\pgfqpoint{4.553788in}{6.131642in}}{\pgfqpoint{4.547277in}{6.125131in}}%
\pgfpathcurveto{\pgfqpoint{4.540766in}{6.118620in}}{\pgfqpoint{4.537107in}{6.109787in}}{\pgfqpoint{4.537107in}{6.100579in}}%
\pgfpathcurveto{\pgfqpoint{4.537107in}{6.091370in}}{\pgfqpoint{4.540766in}{6.082538in}}{\pgfqpoint{4.547277in}{6.076026in}}%
\pgfpathcurveto{\pgfqpoint{4.553788in}{6.069515in}}{\pgfqpoint{4.562621in}{6.065856in}}{\pgfqpoint{4.571829in}{6.065856in}}%
\pgfpathlineto{\pgfqpoint{4.571829in}{6.065856in}}%
\pgfpathclose%
\pgfusepath{stroke}%
\end{pgfscope}%
\begin{pgfscope}%
\pgfpathrectangle{\pgfqpoint{0.688192in}{5.094806in}}{\pgfqpoint{11.096108in}{4.223431in}}%
\pgfusepath{clip}%
\pgfsetbuttcap%
\pgfsetroundjoin%
\pgfsetlinewidth{1.003750pt}%
\definecolor{currentstroke}{rgb}{0.450000,0.450000,0.450000}%
\pgfsetstrokecolor{currentstroke}%
\pgfsetdash{}{0pt}%
\pgfpathmoveto{\pgfqpoint{4.571829in}{6.075598in}}%
\pgfpathcurveto{\pgfqpoint{4.581038in}{6.075598in}}{\pgfqpoint{4.589870in}{6.079257in}}{\pgfqpoint{4.596382in}{6.085768in}}%
\pgfpathcurveto{\pgfqpoint{4.602893in}{6.092280in}}{\pgfqpoint{4.606552in}{6.101112in}}{\pgfqpoint{4.606552in}{6.110321in}}%
\pgfpathcurveto{\pgfqpoint{4.606552in}{6.119529in}}{\pgfqpoint{4.602893in}{6.128362in}}{\pgfqpoint{4.596382in}{6.134873in}}%
\pgfpathcurveto{\pgfqpoint{4.589870in}{6.141384in}}{\pgfqpoint{4.581038in}{6.145043in}}{\pgfqpoint{4.571829in}{6.145043in}}%
\pgfpathcurveto{\pgfqpoint{4.562621in}{6.145043in}}{\pgfqpoint{4.553788in}{6.141384in}}{\pgfqpoint{4.547277in}{6.134873in}}%
\pgfpathcurveto{\pgfqpoint{4.540766in}{6.128362in}}{\pgfqpoint{4.537107in}{6.119529in}}{\pgfqpoint{4.537107in}{6.110321in}}%
\pgfpathcurveto{\pgfqpoint{4.537107in}{6.101112in}}{\pgfqpoint{4.540766in}{6.092280in}}{\pgfqpoint{4.547277in}{6.085768in}}%
\pgfpathcurveto{\pgfqpoint{4.553788in}{6.079257in}}{\pgfqpoint{4.562621in}{6.075598in}}{\pgfqpoint{4.571829in}{6.075598in}}%
\pgfpathlineto{\pgfqpoint{4.571829in}{6.075598in}}%
\pgfpathclose%
\pgfusepath{stroke}%
\end{pgfscope}%
\begin{pgfscope}%
\pgfpathrectangle{\pgfqpoint{0.688192in}{5.094806in}}{\pgfqpoint{11.096108in}{4.223431in}}%
\pgfusepath{clip}%
\pgfsetbuttcap%
\pgfsetroundjoin%
\definecolor{currentfill}{rgb}{0.713429,0.629184,0.790364}%
\pgfsetfillcolor{currentfill}%
\pgfsetlinewidth{0.752812pt}%
\definecolor{currentstroke}{rgb}{0.240000,0.240000,0.240000}%
\pgfsetstrokecolor{currentstroke}%
\pgfsetdash{}{0pt}%
\pgfpathmoveto{\pgfqpoint{5.570479in}{5.094806in}}%
\pgfpathlineto{\pgfqpoint{5.792401in}{5.094806in}}%
\pgfpathlineto{\pgfqpoint{5.792401in}{5.094806in}}%
\pgfpathlineto{\pgfqpoint{5.570479in}{5.094806in}}%
\pgfpathlineto{\pgfqpoint{5.570479in}{5.094806in}}%
\pgfpathclose%
\pgfusepath{stroke,fill}%
\end{pgfscope}%
\begin{pgfscope}%
\pgfpathrectangle{\pgfqpoint{0.688192in}{5.094806in}}{\pgfqpoint{11.096108in}{4.223431in}}%
\pgfusepath{clip}%
\pgfsetbuttcap%
\pgfsetroundjoin%
\definecolor{currentfill}{rgb}{0.653111,0.547371,0.749551}%
\pgfsetfillcolor{currentfill}%
\pgfsetlinewidth{0.752812pt}%
\definecolor{currentstroke}{rgb}{0.240000,0.240000,0.240000}%
\pgfsetstrokecolor{currentstroke}%
\pgfsetdash{}{0pt}%
\pgfpathmoveto{\pgfqpoint{5.459518in}{5.094806in}}%
\pgfpathlineto{\pgfqpoint{5.903362in}{5.094806in}}%
\pgfpathlineto{\pgfqpoint{5.903362in}{5.094806in}}%
\pgfpathlineto{\pgfqpoint{5.459518in}{5.094806in}}%
\pgfpathlineto{\pgfqpoint{5.459518in}{5.094806in}}%
\pgfpathclose%
\pgfusepath{stroke,fill}%
\end{pgfscope}%
\begin{pgfscope}%
\pgfpathrectangle{\pgfqpoint{0.688192in}{5.094806in}}{\pgfqpoint{11.096108in}{4.223431in}}%
\pgfusepath{clip}%
\pgfsetbuttcap%
\pgfsetroundjoin%
\definecolor{currentfill}{rgb}{0.578431,0.446078,0.699020}%
\pgfsetfillcolor{currentfill}%
\pgfsetlinewidth{0.752812pt}%
\definecolor{currentstroke}{rgb}{0.240000,0.240000,0.240000}%
\pgfsetstrokecolor{currentstroke}%
\pgfsetdash{}{0pt}%
\pgfpathmoveto{\pgfqpoint{5.237596in}{5.094806in}}%
\pgfpathlineto{\pgfqpoint{6.125284in}{5.094806in}}%
\pgfpathlineto{\pgfqpoint{6.125284in}{5.094806in}}%
\pgfpathlineto{\pgfqpoint{5.237596in}{5.094806in}}%
\pgfpathlineto{\pgfqpoint{5.237596in}{5.094806in}}%
\pgfpathclose%
\pgfusepath{stroke,fill}%
\end{pgfscope}%
\begin{pgfscope}%
\pgfpathrectangle{\pgfqpoint{0.688192in}{5.094806in}}{\pgfqpoint{11.096108in}{4.223431in}}%
\pgfusepath{clip}%
\pgfsetbuttcap%
\pgfsetroundjoin%
\definecolor{currentfill}{rgb}{0.653111,0.547371,0.749551}%
\pgfsetfillcolor{currentfill}%
\pgfsetlinewidth{0.752812pt}%
\definecolor{currentstroke}{rgb}{0.240000,0.240000,0.240000}%
\pgfsetstrokecolor{currentstroke}%
\pgfsetdash{}{0pt}%
\pgfpathmoveto{\pgfqpoint{5.459518in}{5.094806in}}%
\pgfpathlineto{\pgfqpoint{5.903362in}{5.094806in}}%
\pgfpathlineto{\pgfqpoint{5.903362in}{5.094806in}}%
\pgfpathlineto{\pgfqpoint{5.459518in}{5.094806in}}%
\pgfpathlineto{\pgfqpoint{5.459518in}{5.094806in}}%
\pgfpathclose%
\pgfusepath{stroke,fill}%
\end{pgfscope}%
\begin{pgfscope}%
\pgfpathrectangle{\pgfqpoint{0.688192in}{5.094806in}}{\pgfqpoint{11.096108in}{4.223431in}}%
\pgfusepath{clip}%
\pgfsetbuttcap%
\pgfsetroundjoin%
\definecolor{currentfill}{rgb}{0.713429,0.629184,0.790364}%
\pgfsetfillcolor{currentfill}%
\pgfsetlinewidth{0.752812pt}%
\definecolor{currentstroke}{rgb}{0.240000,0.240000,0.240000}%
\pgfsetstrokecolor{currentstroke}%
\pgfsetdash{}{0pt}%
\pgfpathmoveto{\pgfqpoint{5.570479in}{5.094806in}}%
\pgfpathlineto{\pgfqpoint{5.792401in}{5.094806in}}%
\pgfpathlineto{\pgfqpoint{5.792401in}{5.094806in}}%
\pgfpathlineto{\pgfqpoint{5.570479in}{5.094806in}}%
\pgfpathlineto{\pgfqpoint{5.570479in}{5.094806in}}%
\pgfpathclose%
\pgfusepath{stroke,fill}%
\end{pgfscope}%
\begin{pgfscope}%
\pgfpathrectangle{\pgfqpoint{0.688192in}{5.094806in}}{\pgfqpoint{11.096108in}{4.223431in}}%
\pgfusepath{clip}%
\pgfsetbuttcap%
\pgfsetroundjoin%
\pgfsetlinewidth{1.003750pt}%
\definecolor{currentstroke}{rgb}{0.450000,0.450000,0.450000}%
\pgfsetstrokecolor{currentstroke}%
\pgfsetdash{}{0pt}%
\pgfpathmoveto{\pgfqpoint{5.681440in}{5.128773in}}%
\pgfpathcurveto{\pgfqpoint{5.690649in}{5.128773in}}{\pgfqpoint{5.699481in}{5.132431in}}{\pgfqpoint{5.705992in}{5.138943in}}%
\pgfpathcurveto{\pgfqpoint{5.712504in}{5.145454in}}{\pgfqpoint{5.716162in}{5.154287in}}{\pgfqpoint{5.716162in}{5.163495in}}%
\pgfpathcurveto{\pgfqpoint{5.716162in}{5.172703in}}{\pgfqpoint{5.712504in}{5.181536in}}{\pgfqpoint{5.705992in}{5.188047in}}%
\pgfpathcurveto{\pgfqpoint{5.699481in}{5.194559in}}{\pgfqpoint{5.690649in}{5.198217in}}{\pgfqpoint{5.681440in}{5.198217in}}%
\pgfpathcurveto{\pgfqpoint{5.672232in}{5.198217in}}{\pgfqpoint{5.663399in}{5.194559in}}{\pgfqpoint{5.656888in}{5.188047in}}%
\pgfpathcurveto{\pgfqpoint{5.650376in}{5.181536in}}{\pgfqpoint{5.646718in}{5.172703in}}{\pgfqpoint{5.646718in}{5.163495in}}%
\pgfpathcurveto{\pgfqpoint{5.646718in}{5.154287in}}{\pgfqpoint{5.650376in}{5.145454in}}{\pgfqpoint{5.656888in}{5.138943in}}%
\pgfpathcurveto{\pgfqpoint{5.663399in}{5.132431in}}{\pgfqpoint{5.672232in}{5.128773in}}{\pgfqpoint{5.681440in}{5.128773in}}%
\pgfpathlineto{\pgfqpoint{5.681440in}{5.128773in}}%
\pgfpathclose%
\pgfusepath{stroke}%
\end{pgfscope}%
\begin{pgfscope}%
\pgfpathrectangle{\pgfqpoint{0.688192in}{5.094806in}}{\pgfqpoint{11.096108in}{4.223431in}}%
\pgfusepath{clip}%
\pgfsetbuttcap%
\pgfsetroundjoin%
\definecolor{currentfill}{rgb}{0.675311,0.573788,0.553126}%
\pgfsetfillcolor{currentfill}%
\pgfsetlinewidth{0.752812pt}%
\definecolor{currentstroke}{rgb}{0.240000,0.240000,0.240000}%
\pgfsetstrokecolor{currentstroke}%
\pgfsetdash{}{0pt}%
\pgfpathmoveto{\pgfqpoint{6.680090in}{5.094806in}}%
\pgfpathlineto{\pgfqpoint{6.902012in}{5.094806in}}%
\pgfpathlineto{\pgfqpoint{6.902012in}{5.094806in}}%
\pgfpathlineto{\pgfqpoint{6.680090in}{5.094806in}}%
\pgfpathlineto{\pgfqpoint{6.680090in}{5.094806in}}%
\pgfpathclose%
\pgfusepath{stroke,fill}%
\end{pgfscope}%
\begin{pgfscope}%
\pgfpathrectangle{\pgfqpoint{0.688192in}{5.094806in}}{\pgfqpoint{11.096108in}{4.223431in}}%
\pgfusepath{clip}%
\pgfsetbuttcap%
\pgfsetroundjoin%
\definecolor{currentfill}{rgb}{0.604646,0.477521,0.451635}%
\pgfsetfillcolor{currentfill}%
\pgfsetlinewidth{0.752812pt}%
\definecolor{currentstroke}{rgb}{0.240000,0.240000,0.240000}%
\pgfsetstrokecolor{currentstroke}%
\pgfsetdash{}{0pt}%
\pgfpathmoveto{\pgfqpoint{6.569129in}{5.094806in}}%
\pgfpathlineto{\pgfqpoint{7.012973in}{5.094806in}}%
\pgfpathlineto{\pgfqpoint{7.012973in}{5.094806in}}%
\pgfpathlineto{\pgfqpoint{6.569129in}{5.094806in}}%
\pgfpathlineto{\pgfqpoint{6.569129in}{5.094806in}}%
\pgfpathclose%
\pgfusepath{stroke,fill}%
\end{pgfscope}%
\begin{pgfscope}%
\pgfpathrectangle{\pgfqpoint{0.688192in}{5.094806in}}{\pgfqpoint{11.096108in}{4.223431in}}%
\pgfusepath{clip}%
\pgfsetbuttcap%
\pgfsetroundjoin%
\definecolor{currentfill}{rgb}{0.517157,0.358333,0.325980}%
\pgfsetfillcolor{currentfill}%
\pgfsetlinewidth{0.752812pt}%
\definecolor{currentstroke}{rgb}{0.240000,0.240000,0.240000}%
\pgfsetstrokecolor{currentstroke}%
\pgfsetdash{}{0pt}%
\pgfpathmoveto{\pgfqpoint{6.347207in}{5.094806in}}%
\pgfpathlineto{\pgfqpoint{7.234895in}{5.094806in}}%
\pgfpathlineto{\pgfqpoint{7.234895in}{5.094806in}}%
\pgfpathlineto{\pgfqpoint{6.347207in}{5.094806in}}%
\pgfpathlineto{\pgfqpoint{6.347207in}{5.094806in}}%
\pgfpathclose%
\pgfusepath{stroke,fill}%
\end{pgfscope}%
\begin{pgfscope}%
\pgfpathrectangle{\pgfqpoint{0.688192in}{5.094806in}}{\pgfqpoint{11.096108in}{4.223431in}}%
\pgfusepath{clip}%
\pgfsetbuttcap%
\pgfsetroundjoin%
\definecolor{currentfill}{rgb}{0.604646,0.477521,0.451635}%
\pgfsetfillcolor{currentfill}%
\pgfsetlinewidth{0.752812pt}%
\definecolor{currentstroke}{rgb}{0.240000,0.240000,0.240000}%
\pgfsetstrokecolor{currentstroke}%
\pgfsetdash{}{0pt}%
\pgfpathmoveto{\pgfqpoint{6.569129in}{5.094806in}}%
\pgfpathlineto{\pgfqpoint{7.012973in}{5.094806in}}%
\pgfpathlineto{\pgfqpoint{7.012973in}{5.094806in}}%
\pgfpathlineto{\pgfqpoint{6.569129in}{5.094806in}}%
\pgfpathlineto{\pgfqpoint{6.569129in}{5.094806in}}%
\pgfpathclose%
\pgfusepath{stroke,fill}%
\end{pgfscope}%
\begin{pgfscope}%
\pgfpathrectangle{\pgfqpoint{0.688192in}{5.094806in}}{\pgfqpoint{11.096108in}{4.223431in}}%
\pgfusepath{clip}%
\pgfsetbuttcap%
\pgfsetroundjoin%
\definecolor{currentfill}{rgb}{0.675311,0.573788,0.553126}%
\pgfsetfillcolor{currentfill}%
\pgfsetlinewidth{0.752812pt}%
\definecolor{currentstroke}{rgb}{0.240000,0.240000,0.240000}%
\pgfsetstrokecolor{currentstroke}%
\pgfsetdash{}{0pt}%
\pgfpathmoveto{\pgfqpoint{6.680090in}{5.094806in}}%
\pgfpathlineto{\pgfqpoint{6.902012in}{5.094806in}}%
\pgfpathlineto{\pgfqpoint{6.902012in}{5.094806in}}%
\pgfpathlineto{\pgfqpoint{6.680090in}{5.094806in}}%
\pgfpathlineto{\pgfqpoint{6.680090in}{5.094806in}}%
\pgfpathclose%
\pgfusepath{stroke,fill}%
\end{pgfscope}%
\begin{pgfscope}%
\pgfpathrectangle{\pgfqpoint{0.688192in}{5.094806in}}{\pgfqpoint{11.096108in}{4.223431in}}%
\pgfusepath{clip}%
\pgfsetbuttcap%
\pgfsetroundjoin%
\pgfsetlinewidth{1.003750pt}%
\definecolor{currentstroke}{rgb}{0.450000,0.450000,0.450000}%
\pgfsetstrokecolor{currentstroke}%
\pgfsetdash{}{0pt}%
\pgfpathmoveto{\pgfqpoint{0.000000in}{-0.034722in}}%
\pgfpathcurveto{\pgfqpoint{0.009208in}{-0.034722in}}{\pgfqpoint{0.018041in}{-0.031064in}}{\pgfqpoint{0.024552in}{-0.024552in}}%
\pgfpathcurveto{\pgfqpoint{0.031064in}{-0.018041in}}{\pgfqpoint{0.034722in}{-0.009208in}}{\pgfqpoint{0.034722in}{0.000000in}}%
\pgfpathcurveto{\pgfqpoint{0.034722in}{0.009208in}}{\pgfqpoint{0.031064in}{0.018041in}}{\pgfqpoint{0.024552in}{0.024552in}}%
\pgfpathcurveto{\pgfqpoint{0.018041in}{0.031064in}}{\pgfqpoint{0.009208in}{0.034722in}}{\pgfqpoint{0.000000in}{0.034722in}}%
\pgfpathcurveto{\pgfqpoint{-0.009208in}{0.034722in}}{\pgfqpoint{-0.018041in}{0.031064in}}{\pgfqpoint{-0.024552in}{0.024552in}}%
\pgfpathcurveto{\pgfqpoint{-0.031064in}{0.018041in}}{\pgfqpoint{-0.034722in}{0.009208in}}{\pgfqpoint{-0.034722in}{0.000000in}}%
\pgfpathcurveto{\pgfqpoint{-0.034722in}{-0.009208in}}{\pgfqpoint{-0.031064in}{-0.018041in}}{\pgfqpoint{-0.024552in}{-0.024552in}}%
\pgfpathcurveto{\pgfqpoint{-0.018041in}{-0.031064in}}{\pgfqpoint{-0.009208in}{-0.034722in}}{\pgfqpoint{0.000000in}{-0.034722in}}%
\pgfusepath{stroke}%
\end{pgfscope}%
\begin{pgfscope}%
\pgfpathrectangle{\pgfqpoint{0.688192in}{5.094806in}}{\pgfqpoint{11.096108in}{4.223431in}}%
\pgfusepath{clip}%
\pgfsetbuttcap%
\pgfsetroundjoin%
\definecolor{currentfill}{rgb}{0.878118,0.675269,0.815953}%
\pgfsetfillcolor{currentfill}%
\pgfsetlinewidth{0.752812pt}%
\definecolor{currentstroke}{rgb}{0.240000,0.240000,0.240000}%
\pgfsetstrokecolor{currentstroke}%
\pgfsetdash{}{0pt}%
\pgfpathmoveto{\pgfqpoint{7.789701in}{5.094806in}}%
\pgfpathlineto{\pgfqpoint{8.011623in}{5.094806in}}%
\pgfpathlineto{\pgfqpoint{8.011623in}{5.094806in}}%
\pgfpathlineto{\pgfqpoint{7.789701in}{5.094806in}}%
\pgfpathlineto{\pgfqpoint{7.789701in}{5.094806in}}%
\pgfpathclose%
\pgfusepath{stroke,fill}%
\end{pgfscope}%
\begin{pgfscope}%
\pgfpathrectangle{\pgfqpoint{0.688192in}{5.094806in}}{\pgfqpoint{11.096108in}{4.223431in}}%
\pgfusepath{clip}%
\pgfsetbuttcap%
\pgfsetroundjoin%
\definecolor{currentfill}{rgb}{0.859860,0.605718,0.782104}%
\pgfsetfillcolor{currentfill}%
\pgfsetlinewidth{0.752812pt}%
\definecolor{currentstroke}{rgb}{0.240000,0.240000,0.240000}%
\pgfsetstrokecolor{currentstroke}%
\pgfsetdash{}{0pt}%
\pgfpathmoveto{\pgfqpoint{7.678740in}{5.094806in}}%
\pgfpathlineto{\pgfqpoint{8.122584in}{5.094806in}}%
\pgfpathlineto{\pgfqpoint{8.122584in}{5.094806in}}%
\pgfpathlineto{\pgfqpoint{7.678740in}{5.094806in}}%
\pgfpathlineto{\pgfqpoint{7.678740in}{5.094806in}}%
\pgfpathclose%
\pgfusepath{stroke,fill}%
\end{pgfscope}%
\begin{pgfscope}%
\pgfpathrectangle{\pgfqpoint{0.688192in}{5.094806in}}{\pgfqpoint{11.096108in}{4.223431in}}%
\pgfusepath{clip}%
\pgfsetbuttcap%
\pgfsetroundjoin%
\definecolor{currentfill}{rgb}{0.837255,0.519608,0.740196}%
\pgfsetfillcolor{currentfill}%
\pgfsetlinewidth{0.752812pt}%
\definecolor{currentstroke}{rgb}{0.240000,0.240000,0.240000}%
\pgfsetstrokecolor{currentstroke}%
\pgfsetdash{}{0pt}%
\pgfpathmoveto{\pgfqpoint{7.456817in}{5.094806in}}%
\pgfpathlineto{\pgfqpoint{8.344506in}{5.094806in}}%
\pgfpathlineto{\pgfqpoint{8.344506in}{5.094806in}}%
\pgfpathlineto{\pgfqpoint{7.456817in}{5.094806in}}%
\pgfpathlineto{\pgfqpoint{7.456817in}{5.094806in}}%
\pgfpathclose%
\pgfusepath{stroke,fill}%
\end{pgfscope}%
\begin{pgfscope}%
\pgfpathrectangle{\pgfqpoint{0.688192in}{5.094806in}}{\pgfqpoint{11.096108in}{4.223431in}}%
\pgfusepath{clip}%
\pgfsetbuttcap%
\pgfsetroundjoin%
\definecolor{currentfill}{rgb}{0.859860,0.605718,0.782104}%
\pgfsetfillcolor{currentfill}%
\pgfsetlinewidth{0.752812pt}%
\definecolor{currentstroke}{rgb}{0.240000,0.240000,0.240000}%
\pgfsetstrokecolor{currentstroke}%
\pgfsetdash{}{0pt}%
\pgfpathmoveto{\pgfqpoint{7.678740in}{5.094806in}}%
\pgfpathlineto{\pgfqpoint{8.122584in}{5.094806in}}%
\pgfpathlineto{\pgfqpoint{8.122584in}{5.094806in}}%
\pgfpathlineto{\pgfqpoint{7.678740in}{5.094806in}}%
\pgfpathlineto{\pgfqpoint{7.678740in}{5.094806in}}%
\pgfpathclose%
\pgfusepath{stroke,fill}%
\end{pgfscope}%
\begin{pgfscope}%
\pgfpathrectangle{\pgfqpoint{0.688192in}{5.094806in}}{\pgfqpoint{11.096108in}{4.223431in}}%
\pgfusepath{clip}%
\pgfsetbuttcap%
\pgfsetroundjoin%
\definecolor{currentfill}{rgb}{0.878118,0.675269,0.815953}%
\pgfsetfillcolor{currentfill}%
\pgfsetlinewidth{0.752812pt}%
\definecolor{currentstroke}{rgb}{0.240000,0.240000,0.240000}%
\pgfsetstrokecolor{currentstroke}%
\pgfsetdash{}{0pt}%
\pgfpathmoveto{\pgfqpoint{7.789701in}{5.094806in}}%
\pgfpathlineto{\pgfqpoint{8.011623in}{5.094806in}}%
\pgfpathlineto{\pgfqpoint{8.011623in}{5.094806in}}%
\pgfpathlineto{\pgfqpoint{7.789701in}{5.094806in}}%
\pgfpathlineto{\pgfqpoint{7.789701in}{5.094806in}}%
\pgfpathclose%
\pgfusepath{stroke,fill}%
\end{pgfscope}%
\begin{pgfscope}%
\pgfpathrectangle{\pgfqpoint{0.688192in}{5.094806in}}{\pgfqpoint{11.096108in}{4.223431in}}%
\pgfusepath{clip}%
\pgfsetbuttcap%
\pgfsetroundjoin%
\pgfsetlinewidth{1.003750pt}%
\definecolor{currentstroke}{rgb}{0.450000,0.450000,0.450000}%
\pgfsetstrokecolor{currentstroke}%
\pgfsetdash{}{0pt}%
\pgfpathmoveto{\pgfqpoint{0.000000in}{-0.034722in}}%
\pgfpathcurveto{\pgfqpoint{0.009208in}{-0.034722in}}{\pgfqpoint{0.018041in}{-0.031064in}}{\pgfqpoint{0.024552in}{-0.024552in}}%
\pgfpathcurveto{\pgfqpoint{0.031064in}{-0.018041in}}{\pgfqpoint{0.034722in}{-0.009208in}}{\pgfqpoint{0.034722in}{0.000000in}}%
\pgfpathcurveto{\pgfqpoint{0.034722in}{0.009208in}}{\pgfqpoint{0.031064in}{0.018041in}}{\pgfqpoint{0.024552in}{0.024552in}}%
\pgfpathcurveto{\pgfqpoint{0.018041in}{0.031064in}}{\pgfqpoint{0.009208in}{0.034722in}}{\pgfqpoint{0.000000in}{0.034722in}}%
\pgfpathcurveto{\pgfqpoint{-0.009208in}{0.034722in}}{\pgfqpoint{-0.018041in}{0.031064in}}{\pgfqpoint{-0.024552in}{0.024552in}}%
\pgfpathcurveto{\pgfqpoint{-0.031064in}{0.018041in}}{\pgfqpoint{-0.034722in}{0.009208in}}{\pgfqpoint{-0.034722in}{0.000000in}}%
\pgfpathcurveto{\pgfqpoint{-0.034722in}{-0.009208in}}{\pgfqpoint{-0.031064in}{-0.018041in}}{\pgfqpoint{-0.024552in}{-0.024552in}}%
\pgfpathcurveto{\pgfqpoint{-0.018041in}{-0.031064in}}{\pgfqpoint{-0.009208in}{-0.034722in}}{\pgfqpoint{0.000000in}{-0.034722in}}%
\pgfusepath{stroke}%
\end{pgfscope}%
\begin{pgfscope}%
\pgfpathrectangle{\pgfqpoint{0.688192in}{5.094806in}}{\pgfqpoint{11.096108in}{4.223431in}}%
\pgfusepath{clip}%
\pgfsetbuttcap%
\pgfsetroundjoin%
\definecolor{currentfill}{rgb}{0.662237,0.662248,0.662203}%
\pgfsetfillcolor{currentfill}%
\pgfsetlinewidth{0.752812pt}%
\definecolor{currentstroke}{rgb}{0.240000,0.240000,0.240000}%
\pgfsetstrokecolor{currentstroke}%
\pgfsetdash{}{0pt}%
\pgfpathmoveto{\pgfqpoint{8.899311in}{5.293080in}}%
\pgfpathlineto{\pgfqpoint{9.121234in}{5.293080in}}%
\pgfpathlineto{\pgfqpoint{9.121234in}{5.550320in}}%
\pgfpathlineto{\pgfqpoint{8.899311in}{5.550320in}}%
\pgfpathlineto{\pgfqpoint{8.899311in}{5.293080in}}%
\pgfpathclose%
\pgfusepath{stroke,fill}%
\end{pgfscope}%
\begin{pgfscope}%
\pgfpathrectangle{\pgfqpoint{0.688192in}{5.094806in}}{\pgfqpoint{11.096108in}{4.223431in}}%
\pgfusepath{clip}%
\pgfsetbuttcap%
\pgfsetroundjoin%
\definecolor{currentfill}{rgb}{0.588872,0.588878,0.588853}%
\pgfsetfillcolor{currentfill}%
\pgfsetlinewidth{0.752812pt}%
\definecolor{currentstroke}{rgb}{0.240000,0.240000,0.240000}%
\pgfsetstrokecolor{currentstroke}%
\pgfsetdash{}{0pt}%
\pgfpathmoveto{\pgfqpoint{8.788350in}{5.550320in}}%
\pgfpathlineto{\pgfqpoint{9.232195in}{5.550320in}}%
\pgfpathlineto{\pgfqpoint{9.232195in}{5.817912in}}%
\pgfpathlineto{\pgfqpoint{8.788350in}{5.817912in}}%
\pgfpathlineto{\pgfqpoint{8.788350in}{5.550320in}}%
\pgfpathclose%
\pgfusepath{stroke,fill}%
\end{pgfscope}%
\begin{pgfscope}%
\pgfpathrectangle{\pgfqpoint{0.688192in}{5.094806in}}{\pgfqpoint{11.096108in}{4.223431in}}%
\pgfusepath{clip}%
\pgfsetbuttcap%
\pgfsetroundjoin%
\definecolor{currentfill}{rgb}{0.498039,0.498039,0.498039}%
\pgfsetfillcolor{currentfill}%
\pgfsetlinewidth{0.752812pt}%
\definecolor{currentstroke}{rgb}{0.240000,0.240000,0.240000}%
\pgfsetstrokecolor{currentstroke}%
\pgfsetdash{}{0pt}%
\pgfpathmoveto{\pgfqpoint{8.566428in}{5.817912in}}%
\pgfpathlineto{\pgfqpoint{9.454117in}{5.817912in}}%
\pgfpathlineto{\pgfqpoint{9.454117in}{6.307703in}}%
\pgfpathlineto{\pgfqpoint{8.566428in}{6.307703in}}%
\pgfpathlineto{\pgfqpoint{8.566428in}{5.817912in}}%
\pgfpathclose%
\pgfusepath{stroke,fill}%
\end{pgfscope}%
\begin{pgfscope}%
\pgfpathrectangle{\pgfqpoint{0.688192in}{5.094806in}}{\pgfqpoint{11.096108in}{4.223431in}}%
\pgfusepath{clip}%
\pgfsetbuttcap%
\pgfsetroundjoin%
\definecolor{currentfill}{rgb}{0.588872,0.588878,0.588853}%
\pgfsetfillcolor{currentfill}%
\pgfsetlinewidth{0.752812pt}%
\definecolor{currentstroke}{rgb}{0.240000,0.240000,0.240000}%
\pgfsetstrokecolor{currentstroke}%
\pgfsetdash{}{0pt}%
\pgfpathmoveto{\pgfqpoint{8.788350in}{6.307703in}}%
\pgfpathlineto{\pgfqpoint{9.232195in}{6.307703in}}%
\pgfpathlineto{\pgfqpoint{9.232195in}{6.898184in}}%
\pgfpathlineto{\pgfqpoint{8.788350in}{6.898184in}}%
\pgfpathlineto{\pgfqpoint{8.788350in}{6.307703in}}%
\pgfpathclose%
\pgfusepath{stroke,fill}%
\end{pgfscope}%
\begin{pgfscope}%
\pgfpathrectangle{\pgfqpoint{0.688192in}{5.094806in}}{\pgfqpoint{11.096108in}{4.223431in}}%
\pgfusepath{clip}%
\pgfsetbuttcap%
\pgfsetroundjoin%
\definecolor{currentfill}{rgb}{0.662237,0.662248,0.662203}%
\pgfsetfillcolor{currentfill}%
\pgfsetlinewidth{0.752812pt}%
\definecolor{currentstroke}{rgb}{0.240000,0.240000,0.240000}%
\pgfsetstrokecolor{currentstroke}%
\pgfsetdash{}{0pt}%
\pgfpathmoveto{\pgfqpoint{8.899311in}{6.898184in}}%
\pgfpathlineto{\pgfqpoint{9.121234in}{6.898184in}}%
\pgfpathlineto{\pgfqpoint{9.121234in}{6.994344in}}%
\pgfpathlineto{\pgfqpoint{8.899311in}{6.994344in}}%
\pgfpathlineto{\pgfqpoint{8.899311in}{6.898184in}}%
\pgfpathclose%
\pgfusepath{stroke,fill}%
\end{pgfscope}%
\begin{pgfscope}%
\pgfpathrectangle{\pgfqpoint{0.688192in}{5.094806in}}{\pgfqpoint{11.096108in}{4.223431in}}%
\pgfusepath{clip}%
\pgfsetbuttcap%
\pgfsetroundjoin%
\pgfsetlinewidth{1.003750pt}%
\definecolor{currentstroke}{rgb}{0.450000,0.450000,0.450000}%
\pgfsetstrokecolor{currentstroke}%
\pgfsetdash{}{0pt}%
\pgfpathmoveto{\pgfqpoint{9.010272in}{7.053670in}}%
\pgfpathcurveto{\pgfqpoint{9.019481in}{7.053670in}}{\pgfqpoint{9.028313in}{7.057328in}}{\pgfqpoint{9.034825in}{7.063839in}}%
\pgfpathcurveto{\pgfqpoint{9.041336in}{7.070351in}}{\pgfqpoint{9.044995in}{7.079183in}}{\pgfqpoint{9.044995in}{7.088392in}}%
\pgfpathcurveto{\pgfqpoint{9.044995in}{7.097600in}}{\pgfqpoint{9.041336in}{7.106433in}}{\pgfqpoint{9.034825in}{7.112944in}}%
\pgfpathcurveto{\pgfqpoint{9.028313in}{7.119455in}}{\pgfqpoint{9.019481in}{7.123114in}}{\pgfqpoint{9.010272in}{7.123114in}}%
\pgfpathcurveto{\pgfqpoint{9.001064in}{7.123114in}}{\pgfqpoint{8.992232in}{7.119455in}}{\pgfqpoint{8.985720in}{7.112944in}}%
\pgfpathcurveto{\pgfqpoint{8.979209in}{7.106433in}}{\pgfqpoint{8.975550in}{7.097600in}}{\pgfqpoint{8.975550in}{7.088392in}}%
\pgfpathcurveto{\pgfqpoint{8.975550in}{7.079183in}}{\pgfqpoint{8.979209in}{7.070351in}}{\pgfqpoint{8.985720in}{7.063839in}}%
\pgfpathcurveto{\pgfqpoint{8.992232in}{7.057328in}}{\pgfqpoint{9.001064in}{7.053670in}}{\pgfqpoint{9.010272in}{7.053670in}}%
\pgfpathlineto{\pgfqpoint{9.010272in}{7.053670in}}%
\pgfpathclose%
\pgfusepath{stroke}%
\end{pgfscope}%
\begin{pgfscope}%
\pgfpathrectangle{\pgfqpoint{0.688192in}{5.094806in}}{\pgfqpoint{11.096108in}{4.223431in}}%
\pgfusepath{clip}%
\pgfsetbuttcap%
\pgfsetroundjoin%
\pgfsetlinewidth{1.003750pt}%
\definecolor{currentstroke}{rgb}{0.450000,0.450000,0.450000}%
\pgfsetstrokecolor{currentstroke}%
\pgfsetdash{}{0pt}%
\pgfpathmoveto{\pgfqpoint{9.010272in}{5.233795in}}%
\pgfpathcurveto{\pgfqpoint{9.019481in}{5.233795in}}{\pgfqpoint{9.028313in}{5.237453in}}{\pgfqpoint{9.034825in}{5.243964in}}%
\pgfpathcurveto{\pgfqpoint{9.041336in}{5.250476in}}{\pgfqpoint{9.044995in}{5.259308in}}{\pgfqpoint{9.044995in}{5.268517in}}%
\pgfpathcurveto{\pgfqpoint{9.044995in}{5.277725in}}{\pgfqpoint{9.041336in}{5.286558in}}{\pgfqpoint{9.034825in}{5.293069in}}%
\pgfpathcurveto{\pgfqpoint{9.028313in}{5.299580in}}{\pgfqpoint{9.019481in}{5.303239in}}{\pgfqpoint{9.010272in}{5.303239in}}%
\pgfpathcurveto{\pgfqpoint{9.001064in}{5.303239in}}{\pgfqpoint{8.992232in}{5.299580in}}{\pgfqpoint{8.985720in}{5.293069in}}%
\pgfpathcurveto{\pgfqpoint{8.979209in}{5.286558in}}{\pgfqpoint{8.975550in}{5.277725in}}{\pgfqpoint{8.975550in}{5.268517in}}%
\pgfpathcurveto{\pgfqpoint{8.975550in}{5.259308in}}{\pgfqpoint{8.979209in}{5.250476in}}{\pgfqpoint{8.985720in}{5.243964in}}%
\pgfpathcurveto{\pgfqpoint{8.992232in}{5.237453in}}{\pgfqpoint{9.001064in}{5.233795in}}{\pgfqpoint{9.010272in}{5.233795in}}%
\pgfpathlineto{\pgfqpoint{9.010272in}{5.233795in}}%
\pgfpathclose%
\pgfusepath{stroke}%
\end{pgfscope}%
\begin{pgfscope}%
\pgfpathrectangle{\pgfqpoint{0.688192in}{5.094806in}}{\pgfqpoint{11.096108in}{4.223431in}}%
\pgfusepath{clip}%
\pgfsetbuttcap%
\pgfsetroundjoin%
\pgfsetlinewidth{1.003750pt}%
\definecolor{currentstroke}{rgb}{0.450000,0.450000,0.450000}%
\pgfsetstrokecolor{currentstroke}%
\pgfsetdash{}{0pt}%
\pgfpathmoveto{\pgfqpoint{9.010272in}{7.013536in}}%
\pgfpathcurveto{\pgfqpoint{9.019481in}{7.013536in}}{\pgfqpoint{9.028313in}{7.017194in}}{\pgfqpoint{9.034825in}{7.023706in}}%
\pgfpathcurveto{\pgfqpoint{9.041336in}{7.030217in}}{\pgfqpoint{9.044995in}{7.039049in}}{\pgfqpoint{9.044995in}{7.048258in}}%
\pgfpathcurveto{\pgfqpoint{9.044995in}{7.057466in}}{\pgfqpoint{9.041336in}{7.066299in}}{\pgfqpoint{9.034825in}{7.072810in}}%
\pgfpathcurveto{\pgfqpoint{9.028313in}{7.079322in}}{\pgfqpoint{9.019481in}{7.082980in}}{\pgfqpoint{9.010272in}{7.082980in}}%
\pgfpathcurveto{\pgfqpoint{9.001064in}{7.082980in}}{\pgfqpoint{8.992232in}{7.079322in}}{\pgfqpoint{8.985720in}{7.072810in}}%
\pgfpathcurveto{\pgfqpoint{8.979209in}{7.066299in}}{\pgfqpoint{8.975550in}{7.057466in}}{\pgfqpoint{8.975550in}{7.048258in}}%
\pgfpathcurveto{\pgfqpoint{8.975550in}{7.039049in}}{\pgfqpoint{8.979209in}{7.030217in}}{\pgfqpoint{8.985720in}{7.023706in}}%
\pgfpathcurveto{\pgfqpoint{8.992232in}{7.017194in}}{\pgfqpoint{9.001064in}{7.013536in}}{\pgfqpoint{9.010272in}{7.013536in}}%
\pgfpathlineto{\pgfqpoint{9.010272in}{7.013536in}}%
\pgfpathclose%
\pgfusepath{stroke}%
\end{pgfscope}%
\begin{pgfscope}%
\pgfpathrectangle{\pgfqpoint{0.688192in}{5.094806in}}{\pgfqpoint{11.096108in}{4.223431in}}%
\pgfusepath{clip}%
\pgfsetbuttcap%
\pgfsetroundjoin%
\pgfsetlinewidth{1.003750pt}%
\definecolor{currentstroke}{rgb}{0.450000,0.450000,0.450000}%
\pgfsetstrokecolor{currentstroke}%
\pgfsetdash{}{0pt}%
\pgfpathmoveto{\pgfqpoint{9.010272in}{5.241834in}}%
\pgfpathcurveto{\pgfqpoint{9.019481in}{5.241834in}}{\pgfqpoint{9.028313in}{5.245493in}}{\pgfqpoint{9.034825in}{5.252004in}}%
\pgfpathcurveto{\pgfqpoint{9.041336in}{5.258515in}}{\pgfqpoint{9.044995in}{5.267348in}}{\pgfqpoint{9.044995in}{5.276556in}}%
\pgfpathcurveto{\pgfqpoint{9.044995in}{5.285765in}}{\pgfqpoint{9.041336in}{5.294597in}}{\pgfqpoint{9.034825in}{5.301109in}}%
\pgfpathcurveto{\pgfqpoint{9.028313in}{5.307620in}}{\pgfqpoint{9.019481in}{5.311278in}}{\pgfqpoint{9.010272in}{5.311278in}}%
\pgfpathcurveto{\pgfqpoint{9.001064in}{5.311278in}}{\pgfqpoint{8.992232in}{5.307620in}}{\pgfqpoint{8.985720in}{5.301109in}}%
\pgfpathcurveto{\pgfqpoint{8.979209in}{5.294597in}}{\pgfqpoint{8.975550in}{5.285765in}}{\pgfqpoint{8.975550in}{5.276556in}}%
\pgfpathcurveto{\pgfqpoint{8.975550in}{5.267348in}}{\pgfqpoint{8.979209in}{5.258515in}}{\pgfqpoint{8.985720in}{5.252004in}}%
\pgfpathcurveto{\pgfqpoint{8.992232in}{5.245493in}}{\pgfqpoint{9.001064in}{5.241834in}}{\pgfqpoint{9.010272in}{5.241834in}}%
\pgfpathlineto{\pgfqpoint{9.010272in}{5.241834in}}%
\pgfpathclose%
\pgfusepath{stroke}%
\end{pgfscope}%
\begin{pgfscope}%
\pgfpathrectangle{\pgfqpoint{0.688192in}{5.094806in}}{\pgfqpoint{11.096108in}{4.223431in}}%
\pgfusepath{clip}%
\pgfsetbuttcap%
\pgfsetroundjoin%
\definecolor{currentfill}{rgb}{0.767534,0.769674,0.455219}%
\pgfsetfillcolor{currentfill}%
\pgfsetlinewidth{0.752812pt}%
\definecolor{currentstroke}{rgb}{0.240000,0.240000,0.240000}%
\pgfsetstrokecolor{currentstroke}%
\pgfsetdash{}{0pt}%
\pgfpathmoveto{\pgfqpoint{10.008922in}{5.094806in}}%
\pgfpathlineto{\pgfqpoint{10.230844in}{5.094806in}}%
\pgfpathlineto{\pgfqpoint{10.230844in}{5.094806in}}%
\pgfpathlineto{\pgfqpoint{10.008922in}{5.094806in}}%
\pgfpathlineto{\pgfqpoint{10.008922in}{5.094806in}}%
\pgfpathclose%
\pgfusepath{stroke,fill}%
\end{pgfscope}%
\begin{pgfscope}%
\pgfpathrectangle{\pgfqpoint{0.688192in}{5.094806in}}{\pgfqpoint{11.096108in}{4.223431in}}%
\pgfusepath{clip}%
\pgfsetbuttcap%
\pgfsetroundjoin%
\definecolor{currentfill}{rgb}{0.720495,0.722993,0.345346}%
\pgfsetfillcolor{currentfill}%
\pgfsetlinewidth{0.752812pt}%
\definecolor{currentstroke}{rgb}{0.240000,0.240000,0.240000}%
\pgfsetstrokecolor{currentstroke}%
\pgfsetdash{}{0pt}%
\pgfpathmoveto{\pgfqpoint{9.897961in}{5.094806in}}%
\pgfpathlineto{\pgfqpoint{10.341805in}{5.094806in}}%
\pgfpathlineto{\pgfqpoint{10.341805in}{5.094806in}}%
\pgfpathlineto{\pgfqpoint{9.897961in}{5.094806in}}%
\pgfpathlineto{\pgfqpoint{9.897961in}{5.094806in}}%
\pgfpathclose%
\pgfusepath{stroke,fill}%
\end{pgfscope}%
\begin{pgfscope}%
\pgfpathrectangle{\pgfqpoint{0.688192in}{5.094806in}}{\pgfqpoint{11.096108in}{4.223431in}}%
\pgfusepath{clip}%
\pgfsetbuttcap%
\pgfsetroundjoin%
\definecolor{currentfill}{rgb}{0.662255,0.665196,0.209314}%
\pgfsetfillcolor{currentfill}%
\pgfsetlinewidth{0.752812pt}%
\definecolor{currentstroke}{rgb}{0.240000,0.240000,0.240000}%
\pgfsetstrokecolor{currentstroke}%
\pgfsetdash{}{0pt}%
\pgfpathmoveto{\pgfqpoint{9.676039in}{5.094806in}}%
\pgfpathlineto{\pgfqpoint{10.563728in}{5.094806in}}%
\pgfpathlineto{\pgfqpoint{10.563728in}{7.310446in}}%
\pgfpathlineto{\pgfqpoint{9.676039in}{7.310446in}}%
\pgfpathlineto{\pgfqpoint{9.676039in}{5.094806in}}%
\pgfpathclose%
\pgfusepath{stroke,fill}%
\end{pgfscope}%
\begin{pgfscope}%
\pgfpathrectangle{\pgfqpoint{0.688192in}{5.094806in}}{\pgfqpoint{11.096108in}{4.223431in}}%
\pgfusepath{clip}%
\pgfsetbuttcap%
\pgfsetroundjoin%
\definecolor{currentfill}{rgb}{0.720495,0.722993,0.345346}%
\pgfsetfillcolor{currentfill}%
\pgfsetlinewidth{0.752812pt}%
\definecolor{currentstroke}{rgb}{0.240000,0.240000,0.240000}%
\pgfsetstrokecolor{currentstroke}%
\pgfsetdash{}{0pt}%
\pgfpathmoveto{\pgfqpoint{9.897961in}{7.310446in}}%
\pgfpathlineto{\pgfqpoint{10.341805in}{7.310446in}}%
\pgfpathlineto{\pgfqpoint{10.341805in}{7.559760in}}%
\pgfpathlineto{\pgfqpoint{9.897961in}{7.559760in}}%
\pgfpathlineto{\pgfqpoint{9.897961in}{7.310446in}}%
\pgfpathclose%
\pgfusepath{stroke,fill}%
\end{pgfscope}%
\begin{pgfscope}%
\pgfpathrectangle{\pgfqpoint{0.688192in}{5.094806in}}{\pgfqpoint{11.096108in}{4.223431in}}%
\pgfusepath{clip}%
\pgfsetbuttcap%
\pgfsetroundjoin%
\definecolor{currentfill}{rgb}{0.767534,0.769674,0.455219}%
\pgfsetfillcolor{currentfill}%
\pgfsetlinewidth{0.752812pt}%
\definecolor{currentstroke}{rgb}{0.240000,0.240000,0.240000}%
\pgfsetstrokecolor{currentstroke}%
\pgfsetdash{}{0pt}%
\pgfpathmoveto{\pgfqpoint{10.008922in}{7.559760in}}%
\pgfpathlineto{\pgfqpoint{10.230844in}{7.559760in}}%
\pgfpathlineto{\pgfqpoint{10.230844in}{7.657138in}}%
\pgfpathlineto{\pgfqpoint{10.008922in}{7.657138in}}%
\pgfpathlineto{\pgfqpoint{10.008922in}{7.559760in}}%
\pgfpathclose%
\pgfusepath{stroke,fill}%
\end{pgfscope}%
\begin{pgfscope}%
\pgfpathrectangle{\pgfqpoint{0.688192in}{5.094806in}}{\pgfqpoint{11.096108in}{4.223431in}}%
\pgfusepath{clip}%
\pgfsetbuttcap%
\pgfsetroundjoin%
\pgfsetlinewidth{1.003750pt}%
\definecolor{currentstroke}{rgb}{0.450000,0.450000,0.450000}%
\pgfsetstrokecolor{currentstroke}%
\pgfsetdash{}{0pt}%
\pgfpathmoveto{\pgfqpoint{10.119883in}{7.777563in}}%
\pgfpathcurveto{\pgfqpoint{10.129092in}{7.777563in}}{\pgfqpoint{10.137924in}{7.781222in}}{\pgfqpoint{10.144436in}{7.787733in}}%
\pgfpathcurveto{\pgfqpoint{10.150947in}{7.794245in}}{\pgfqpoint{10.154606in}{7.803077in}}{\pgfqpoint{10.154606in}{7.812286in}}%
\pgfpathcurveto{\pgfqpoint{10.154606in}{7.821494in}}{\pgfqpoint{10.150947in}{7.830327in}}{\pgfqpoint{10.144436in}{7.836838in}}%
\pgfpathcurveto{\pgfqpoint{10.137924in}{7.843349in}}{\pgfqpoint{10.129092in}{7.847008in}}{\pgfqpoint{10.119883in}{7.847008in}}%
\pgfpathcurveto{\pgfqpoint{10.110675in}{7.847008in}}{\pgfqpoint{10.101842in}{7.843349in}}{\pgfqpoint{10.095331in}{7.836838in}}%
\pgfpathcurveto{\pgfqpoint{10.088820in}{7.830327in}}{\pgfqpoint{10.085161in}{7.821494in}}{\pgfqpoint{10.085161in}{7.812286in}}%
\pgfpathcurveto{\pgfqpoint{10.085161in}{7.803077in}}{\pgfqpoint{10.088820in}{7.794245in}}{\pgfqpoint{10.095331in}{7.787733in}}%
\pgfpathcurveto{\pgfqpoint{10.101842in}{7.781222in}}{\pgfqpoint{10.110675in}{7.777563in}}{\pgfqpoint{10.119883in}{7.777563in}}%
\pgfpathlineto{\pgfqpoint{10.119883in}{7.777563in}}%
\pgfpathclose%
\pgfusepath{stroke}%
\end{pgfscope}%
\begin{pgfscope}%
\pgfpathrectangle{\pgfqpoint{0.688192in}{5.094806in}}{\pgfqpoint{11.096108in}{4.223431in}}%
\pgfusepath{clip}%
\pgfsetbuttcap%
\pgfsetroundjoin%
\pgfsetlinewidth{1.003750pt}%
\definecolor{currentstroke}{rgb}{0.450000,0.450000,0.450000}%
\pgfsetstrokecolor{currentstroke}%
\pgfsetdash{}{0pt}%
\pgfpathmoveto{\pgfqpoint{10.119883in}{7.777563in}}%
\pgfpathcurveto{\pgfqpoint{10.129092in}{7.777563in}}{\pgfqpoint{10.137924in}{7.781222in}}{\pgfqpoint{10.144436in}{7.787733in}}%
\pgfpathcurveto{\pgfqpoint{10.150947in}{7.794245in}}{\pgfqpoint{10.154606in}{7.803077in}}{\pgfqpoint{10.154606in}{7.812286in}}%
\pgfpathcurveto{\pgfqpoint{10.154606in}{7.821494in}}{\pgfqpoint{10.150947in}{7.830327in}}{\pgfqpoint{10.144436in}{7.836838in}}%
\pgfpathcurveto{\pgfqpoint{10.137924in}{7.843349in}}{\pgfqpoint{10.129092in}{7.847008in}}{\pgfqpoint{10.119883in}{7.847008in}}%
\pgfpathcurveto{\pgfqpoint{10.110675in}{7.847008in}}{\pgfqpoint{10.101842in}{7.843349in}}{\pgfqpoint{10.095331in}{7.836838in}}%
\pgfpathcurveto{\pgfqpoint{10.088820in}{7.830327in}}{\pgfqpoint{10.085161in}{7.821494in}}{\pgfqpoint{10.085161in}{7.812286in}}%
\pgfpathcurveto{\pgfqpoint{10.085161in}{7.803077in}}{\pgfqpoint{10.088820in}{7.794245in}}{\pgfqpoint{10.095331in}{7.787733in}}%
\pgfpathcurveto{\pgfqpoint{10.101842in}{7.781222in}}{\pgfqpoint{10.110675in}{7.777563in}}{\pgfqpoint{10.119883in}{7.777563in}}%
\pgfpathlineto{\pgfqpoint{10.119883in}{7.777563in}}%
\pgfpathclose%
\pgfusepath{stroke}%
\end{pgfscope}%
\begin{pgfscope}%
\pgfpathrectangle{\pgfqpoint{0.688192in}{5.094806in}}{\pgfqpoint{11.096108in}{4.223431in}}%
\pgfusepath{clip}%
\pgfsetbuttcap%
\pgfsetroundjoin%
\definecolor{currentfill}{rgb}{0.455462,0.773335,0.806273}%
\pgfsetfillcolor{currentfill}%
\pgfsetlinewidth{0.752812pt}%
\definecolor{currentstroke}{rgb}{0.240000,0.240000,0.240000}%
\pgfsetstrokecolor{currentstroke}%
\pgfsetdash{}{0pt}%
\pgfpathmoveto{\pgfqpoint{11.118533in}{5.094806in}}%
\pgfpathlineto{\pgfqpoint{11.340455in}{5.094806in}}%
\pgfpathlineto{\pgfqpoint{11.340455in}{5.094806in}}%
\pgfpathlineto{\pgfqpoint{11.118533in}{5.094806in}}%
\pgfpathlineto{\pgfqpoint{11.118533in}{5.094806in}}%
\pgfpathclose%
\pgfusepath{stroke,fill}%
\end{pgfscope}%
\begin{pgfscope}%
\pgfpathrectangle{\pgfqpoint{0.688192in}{5.094806in}}{\pgfqpoint{11.096108in}{4.223431in}}%
\pgfusepath{clip}%
\pgfsetbuttcap%
\pgfsetroundjoin%
\definecolor{currentfill}{rgb}{0.332558,0.727865,0.768426}%
\pgfsetfillcolor{currentfill}%
\pgfsetlinewidth{0.752812pt}%
\definecolor{currentstroke}{rgb}{0.240000,0.240000,0.240000}%
\pgfsetstrokecolor{currentstroke}%
\pgfsetdash{}{0pt}%
\pgfpathmoveto{\pgfqpoint{11.007572in}{5.094806in}}%
\pgfpathlineto{\pgfqpoint{11.451416in}{5.094806in}}%
\pgfpathlineto{\pgfqpoint{11.451416in}{5.094806in}}%
\pgfpathlineto{\pgfqpoint{11.007572in}{5.094806in}}%
\pgfpathlineto{\pgfqpoint{11.007572in}{5.094806in}}%
\pgfpathclose%
\pgfusepath{stroke,fill}%
\end{pgfscope}%
\begin{pgfscope}%
\pgfpathrectangle{\pgfqpoint{0.688192in}{5.094806in}}{\pgfqpoint{11.096108in}{4.223431in}}%
\pgfusepath{clip}%
\pgfsetbuttcap%
\pgfsetroundjoin%
\definecolor{currentfill}{rgb}{0.180392,0.671569,0.721569}%
\pgfsetfillcolor{currentfill}%
\pgfsetlinewidth{0.752812pt}%
\definecolor{currentstroke}{rgb}{0.240000,0.240000,0.240000}%
\pgfsetstrokecolor{currentstroke}%
\pgfsetdash{}{0pt}%
\pgfpathmoveto{\pgfqpoint{10.785650in}{5.094806in}}%
\pgfpathlineto{\pgfqpoint{11.673338in}{5.094806in}}%
\pgfpathlineto{\pgfqpoint{11.673338in}{5.186402in}}%
\pgfpathlineto{\pgfqpoint{10.785650in}{5.186402in}}%
\pgfpathlineto{\pgfqpoint{10.785650in}{5.094806in}}%
\pgfpathclose%
\pgfusepath{stroke,fill}%
\end{pgfscope}%
\begin{pgfscope}%
\pgfpathrectangle{\pgfqpoint{0.688192in}{5.094806in}}{\pgfqpoint{11.096108in}{4.223431in}}%
\pgfusepath{clip}%
\pgfsetbuttcap%
\pgfsetroundjoin%
\definecolor{currentfill}{rgb}{0.332558,0.727865,0.768426}%
\pgfsetfillcolor{currentfill}%
\pgfsetlinewidth{0.752812pt}%
\definecolor{currentstroke}{rgb}{0.240000,0.240000,0.240000}%
\pgfsetstrokecolor{currentstroke}%
\pgfsetdash{}{0pt}%
\pgfpathmoveto{\pgfqpoint{11.007572in}{5.186402in}}%
\pgfpathlineto{\pgfqpoint{11.451416in}{5.186402in}}%
\pgfpathlineto{\pgfqpoint{11.451416in}{5.967125in}}%
\pgfpathlineto{\pgfqpoint{11.007572in}{5.967125in}}%
\pgfpathlineto{\pgfqpoint{11.007572in}{5.186402in}}%
\pgfpathclose%
\pgfusepath{stroke,fill}%
\end{pgfscope}%
\begin{pgfscope}%
\pgfpathrectangle{\pgfqpoint{0.688192in}{5.094806in}}{\pgfqpoint{11.096108in}{4.223431in}}%
\pgfusepath{clip}%
\pgfsetbuttcap%
\pgfsetroundjoin%
\definecolor{currentfill}{rgb}{0.455462,0.773335,0.806273}%
\pgfsetfillcolor{currentfill}%
\pgfsetlinewidth{0.752812pt}%
\definecolor{currentstroke}{rgb}{0.240000,0.240000,0.240000}%
\pgfsetstrokecolor{currentstroke}%
\pgfsetdash{}{0pt}%
\pgfpathmoveto{\pgfqpoint{11.118533in}{5.967125in}}%
\pgfpathlineto{\pgfqpoint{11.340455in}{5.967125in}}%
\pgfpathlineto{\pgfqpoint{11.340455in}{6.008726in}}%
\pgfpathlineto{\pgfqpoint{11.118533in}{6.008726in}}%
\pgfpathlineto{\pgfqpoint{11.118533in}{5.967125in}}%
\pgfpathclose%
\pgfusepath{stroke,fill}%
\end{pgfscope}%
\begin{pgfscope}%
\pgfpathrectangle{\pgfqpoint{0.688192in}{5.094806in}}{\pgfqpoint{11.096108in}{4.223431in}}%
\pgfusepath{clip}%
\pgfsetbuttcap%
\pgfsetroundjoin%
\pgfsetlinewidth{1.003750pt}%
\definecolor{currentstroke}{rgb}{0.450000,0.450000,0.450000}%
\pgfsetstrokecolor{currentstroke}%
\pgfsetdash{}{0pt}%
\pgfpathmoveto{\pgfqpoint{11.229494in}{6.250421in}}%
\pgfpathcurveto{\pgfqpoint{11.238703in}{6.250421in}}{\pgfqpoint{11.247535in}{6.254079in}}{\pgfqpoint{11.254046in}{6.260591in}}%
\pgfpathcurveto{\pgfqpoint{11.260558in}{6.267102in}}{\pgfqpoint{11.264216in}{6.275934in}}{\pgfqpoint{11.264216in}{6.285143in}}%
\pgfpathcurveto{\pgfqpoint{11.264216in}{6.294351in}}{\pgfqpoint{11.260558in}{6.303184in}}{\pgfqpoint{11.254046in}{6.309695in}}%
\pgfpathcurveto{\pgfqpoint{11.247535in}{6.316207in}}{\pgfqpoint{11.238703in}{6.319865in}}{\pgfqpoint{11.229494in}{6.319865in}}%
\pgfpathcurveto{\pgfqpoint{11.220286in}{6.319865in}}{\pgfqpoint{11.211453in}{6.316207in}}{\pgfqpoint{11.204942in}{6.309695in}}%
\pgfpathcurveto{\pgfqpoint{11.198430in}{6.303184in}}{\pgfqpoint{11.194772in}{6.294351in}}{\pgfqpoint{11.194772in}{6.285143in}}%
\pgfpathcurveto{\pgfqpoint{11.194772in}{6.275934in}}{\pgfqpoint{11.198430in}{6.267102in}}{\pgfqpoint{11.204942in}{6.260591in}}%
\pgfpathcurveto{\pgfqpoint{11.211453in}{6.254079in}}{\pgfqpoint{11.220286in}{6.250421in}}{\pgfqpoint{11.229494in}{6.250421in}}%
\pgfpathlineto{\pgfqpoint{11.229494in}{6.250421in}}%
\pgfpathclose%
\pgfusepath{stroke}%
\end{pgfscope}%
\begin{pgfscope}%
\pgfpathrectangle{\pgfqpoint{0.688192in}{5.094806in}}{\pgfqpoint{11.096108in}{4.223431in}}%
\pgfusepath{clip}%
\pgfsetbuttcap%
\pgfsetroundjoin%
\pgfsetlinewidth{1.003750pt}%
\definecolor{currentstroke}{rgb}{0.450000,0.450000,0.450000}%
\pgfsetstrokecolor{currentstroke}%
\pgfsetdash{}{0pt}%
\pgfpathmoveto{\pgfqpoint{11.229494in}{6.086464in}}%
\pgfpathcurveto{\pgfqpoint{11.238703in}{6.086464in}}{\pgfqpoint{11.247535in}{6.090123in}}{\pgfqpoint{11.254046in}{6.096634in}}%
\pgfpathcurveto{\pgfqpoint{11.260558in}{6.103145in}}{\pgfqpoint{11.264216in}{6.111978in}}{\pgfqpoint{11.264216in}{6.121186in}}%
\pgfpathcurveto{\pgfqpoint{11.264216in}{6.130395in}}{\pgfqpoint{11.260558in}{6.139227in}}{\pgfqpoint{11.254046in}{6.145739in}}%
\pgfpathcurveto{\pgfqpoint{11.247535in}{6.152250in}}{\pgfqpoint{11.238703in}{6.155909in}}{\pgfqpoint{11.229494in}{6.155909in}}%
\pgfpathcurveto{\pgfqpoint{11.220286in}{6.155909in}}{\pgfqpoint{11.211453in}{6.152250in}}{\pgfqpoint{11.204942in}{6.145739in}}%
\pgfpathcurveto{\pgfqpoint{11.198430in}{6.139227in}}{\pgfqpoint{11.194772in}{6.130395in}}{\pgfqpoint{11.194772in}{6.121186in}}%
\pgfpathcurveto{\pgfqpoint{11.194772in}{6.111978in}}{\pgfqpoint{11.198430in}{6.103145in}}{\pgfqpoint{11.204942in}{6.096634in}}%
\pgfpathcurveto{\pgfqpoint{11.211453in}{6.090123in}}{\pgfqpoint{11.220286in}{6.086464in}}{\pgfqpoint{11.229494in}{6.086464in}}%
\pgfpathlineto{\pgfqpoint{11.229494in}{6.086464in}}%
\pgfpathclose%
\pgfusepath{stroke}%
\end{pgfscope}%
\begin{pgfscope}%
\pgfpathrectangle{\pgfqpoint{0.688192in}{5.094806in}}{\pgfqpoint{11.096108in}{4.223431in}}%
\pgfusepath{clip}%
\pgfsetrectcap%
\pgfsetroundjoin%
\pgfsetlinewidth{0.803000pt}%
\definecolor{currentstroke}{rgb}{0.690196,0.690196,0.690196}%
\pgfsetstrokecolor{currentstroke}%
\pgfsetstrokeopacity{0.200000}%
\pgfsetdash{}{0pt}%
\pgfpathmoveto{\pgfqpoint{1.242997in}{5.094806in}}%
\pgfpathlineto{\pgfqpoint{1.242997in}{9.318236in}}%
\pgfusepath{stroke}%
\end{pgfscope}%
\begin{pgfscope}%
\pgfsetbuttcap%
\pgfsetroundjoin%
\definecolor{currentfill}{rgb}{0.000000,0.000000,0.000000}%
\pgfsetfillcolor{currentfill}%
\pgfsetlinewidth{0.803000pt}%
\definecolor{currentstroke}{rgb}{0.000000,0.000000,0.000000}%
\pgfsetstrokecolor{currentstroke}%
\pgfsetdash{}{0pt}%
\pgfsys@defobject{currentmarker}{\pgfqpoint{0.000000in}{-0.048611in}}{\pgfqpoint{0.000000in}{0.000000in}}{%
\pgfpathmoveto{\pgfqpoint{0.000000in}{0.000000in}}%
\pgfpathlineto{\pgfqpoint{0.000000in}{-0.048611in}}%
\pgfusepath{stroke,fill}%
}%
\begin{pgfscope}%
\pgfsys@transformshift{1.242997in}{5.094806in}%
\pgfsys@useobject{currentmarker}{}%
\end{pgfscope}%
\end{pgfscope}%
\begin{pgfscope}%
\pgfpathrectangle{\pgfqpoint{0.688192in}{5.094806in}}{\pgfqpoint{11.096108in}{4.223431in}}%
\pgfusepath{clip}%
\pgfsetrectcap%
\pgfsetroundjoin%
\pgfsetlinewidth{0.803000pt}%
\definecolor{currentstroke}{rgb}{0.690196,0.690196,0.690196}%
\pgfsetstrokecolor{currentstroke}%
\pgfsetstrokeopacity{0.200000}%
\pgfsetdash{}{0pt}%
\pgfpathmoveto{\pgfqpoint{2.352608in}{5.094806in}}%
\pgfpathlineto{\pgfqpoint{2.352608in}{9.318236in}}%
\pgfusepath{stroke}%
\end{pgfscope}%
\begin{pgfscope}%
\pgfsetbuttcap%
\pgfsetroundjoin%
\definecolor{currentfill}{rgb}{0.000000,0.000000,0.000000}%
\pgfsetfillcolor{currentfill}%
\pgfsetlinewidth{0.803000pt}%
\definecolor{currentstroke}{rgb}{0.000000,0.000000,0.000000}%
\pgfsetstrokecolor{currentstroke}%
\pgfsetdash{}{0pt}%
\pgfsys@defobject{currentmarker}{\pgfqpoint{0.000000in}{-0.048611in}}{\pgfqpoint{0.000000in}{0.000000in}}{%
\pgfpathmoveto{\pgfqpoint{0.000000in}{0.000000in}}%
\pgfpathlineto{\pgfqpoint{0.000000in}{-0.048611in}}%
\pgfusepath{stroke,fill}%
}%
\begin{pgfscope}%
\pgfsys@transformshift{2.352608in}{5.094806in}%
\pgfsys@useobject{currentmarker}{}%
\end{pgfscope}%
\end{pgfscope}%
\begin{pgfscope}%
\pgfpathrectangle{\pgfqpoint{0.688192in}{5.094806in}}{\pgfqpoint{11.096108in}{4.223431in}}%
\pgfusepath{clip}%
\pgfsetrectcap%
\pgfsetroundjoin%
\pgfsetlinewidth{0.803000pt}%
\definecolor{currentstroke}{rgb}{0.690196,0.690196,0.690196}%
\pgfsetstrokecolor{currentstroke}%
\pgfsetstrokeopacity{0.200000}%
\pgfsetdash{}{0pt}%
\pgfpathmoveto{\pgfqpoint{3.462219in}{5.094806in}}%
\pgfpathlineto{\pgfqpoint{3.462219in}{9.318236in}}%
\pgfusepath{stroke}%
\end{pgfscope}%
\begin{pgfscope}%
\pgfsetbuttcap%
\pgfsetroundjoin%
\definecolor{currentfill}{rgb}{0.000000,0.000000,0.000000}%
\pgfsetfillcolor{currentfill}%
\pgfsetlinewidth{0.803000pt}%
\definecolor{currentstroke}{rgb}{0.000000,0.000000,0.000000}%
\pgfsetstrokecolor{currentstroke}%
\pgfsetdash{}{0pt}%
\pgfsys@defobject{currentmarker}{\pgfqpoint{0.000000in}{-0.048611in}}{\pgfqpoint{0.000000in}{0.000000in}}{%
\pgfpathmoveto{\pgfqpoint{0.000000in}{0.000000in}}%
\pgfpathlineto{\pgfqpoint{0.000000in}{-0.048611in}}%
\pgfusepath{stroke,fill}%
}%
\begin{pgfscope}%
\pgfsys@transformshift{3.462219in}{5.094806in}%
\pgfsys@useobject{currentmarker}{}%
\end{pgfscope}%
\end{pgfscope}%
\begin{pgfscope}%
\pgfpathrectangle{\pgfqpoint{0.688192in}{5.094806in}}{\pgfqpoint{11.096108in}{4.223431in}}%
\pgfusepath{clip}%
\pgfsetrectcap%
\pgfsetroundjoin%
\pgfsetlinewidth{0.803000pt}%
\definecolor{currentstroke}{rgb}{0.690196,0.690196,0.690196}%
\pgfsetstrokecolor{currentstroke}%
\pgfsetstrokeopacity{0.200000}%
\pgfsetdash{}{0pt}%
\pgfpathmoveto{\pgfqpoint{4.571829in}{5.094806in}}%
\pgfpathlineto{\pgfqpoint{4.571829in}{9.318236in}}%
\pgfusepath{stroke}%
\end{pgfscope}%
\begin{pgfscope}%
\pgfsetbuttcap%
\pgfsetroundjoin%
\definecolor{currentfill}{rgb}{0.000000,0.000000,0.000000}%
\pgfsetfillcolor{currentfill}%
\pgfsetlinewidth{0.803000pt}%
\definecolor{currentstroke}{rgb}{0.000000,0.000000,0.000000}%
\pgfsetstrokecolor{currentstroke}%
\pgfsetdash{}{0pt}%
\pgfsys@defobject{currentmarker}{\pgfqpoint{0.000000in}{-0.048611in}}{\pgfqpoint{0.000000in}{0.000000in}}{%
\pgfpathmoveto{\pgfqpoint{0.000000in}{0.000000in}}%
\pgfpathlineto{\pgfqpoint{0.000000in}{-0.048611in}}%
\pgfusepath{stroke,fill}%
}%
\begin{pgfscope}%
\pgfsys@transformshift{4.571829in}{5.094806in}%
\pgfsys@useobject{currentmarker}{}%
\end{pgfscope}%
\end{pgfscope}%
\begin{pgfscope}%
\pgfpathrectangle{\pgfqpoint{0.688192in}{5.094806in}}{\pgfqpoint{11.096108in}{4.223431in}}%
\pgfusepath{clip}%
\pgfsetrectcap%
\pgfsetroundjoin%
\pgfsetlinewidth{0.803000pt}%
\definecolor{currentstroke}{rgb}{0.690196,0.690196,0.690196}%
\pgfsetstrokecolor{currentstroke}%
\pgfsetstrokeopacity{0.200000}%
\pgfsetdash{}{0pt}%
\pgfpathmoveto{\pgfqpoint{5.681440in}{5.094806in}}%
\pgfpathlineto{\pgfqpoint{5.681440in}{9.318236in}}%
\pgfusepath{stroke}%
\end{pgfscope}%
\begin{pgfscope}%
\pgfsetbuttcap%
\pgfsetroundjoin%
\definecolor{currentfill}{rgb}{0.000000,0.000000,0.000000}%
\pgfsetfillcolor{currentfill}%
\pgfsetlinewidth{0.803000pt}%
\definecolor{currentstroke}{rgb}{0.000000,0.000000,0.000000}%
\pgfsetstrokecolor{currentstroke}%
\pgfsetdash{}{0pt}%
\pgfsys@defobject{currentmarker}{\pgfqpoint{0.000000in}{-0.048611in}}{\pgfqpoint{0.000000in}{0.000000in}}{%
\pgfpathmoveto{\pgfqpoint{0.000000in}{0.000000in}}%
\pgfpathlineto{\pgfqpoint{0.000000in}{-0.048611in}}%
\pgfusepath{stroke,fill}%
}%
\begin{pgfscope}%
\pgfsys@transformshift{5.681440in}{5.094806in}%
\pgfsys@useobject{currentmarker}{}%
\end{pgfscope}%
\end{pgfscope}%
\begin{pgfscope}%
\pgfpathrectangle{\pgfqpoint{0.688192in}{5.094806in}}{\pgfqpoint{11.096108in}{4.223431in}}%
\pgfusepath{clip}%
\pgfsetrectcap%
\pgfsetroundjoin%
\pgfsetlinewidth{0.803000pt}%
\definecolor{currentstroke}{rgb}{0.690196,0.690196,0.690196}%
\pgfsetstrokecolor{currentstroke}%
\pgfsetstrokeopacity{0.200000}%
\pgfsetdash{}{0pt}%
\pgfpathmoveto{\pgfqpoint{6.791051in}{5.094806in}}%
\pgfpathlineto{\pgfqpoint{6.791051in}{9.318236in}}%
\pgfusepath{stroke}%
\end{pgfscope}%
\begin{pgfscope}%
\pgfsetbuttcap%
\pgfsetroundjoin%
\definecolor{currentfill}{rgb}{0.000000,0.000000,0.000000}%
\pgfsetfillcolor{currentfill}%
\pgfsetlinewidth{0.803000pt}%
\definecolor{currentstroke}{rgb}{0.000000,0.000000,0.000000}%
\pgfsetstrokecolor{currentstroke}%
\pgfsetdash{}{0pt}%
\pgfsys@defobject{currentmarker}{\pgfqpoint{0.000000in}{-0.048611in}}{\pgfqpoint{0.000000in}{0.000000in}}{%
\pgfpathmoveto{\pgfqpoint{0.000000in}{0.000000in}}%
\pgfpathlineto{\pgfqpoint{0.000000in}{-0.048611in}}%
\pgfusepath{stroke,fill}%
}%
\begin{pgfscope}%
\pgfsys@transformshift{6.791051in}{5.094806in}%
\pgfsys@useobject{currentmarker}{}%
\end{pgfscope}%
\end{pgfscope}%
\begin{pgfscope}%
\pgfpathrectangle{\pgfqpoint{0.688192in}{5.094806in}}{\pgfqpoint{11.096108in}{4.223431in}}%
\pgfusepath{clip}%
\pgfsetrectcap%
\pgfsetroundjoin%
\pgfsetlinewidth{0.803000pt}%
\definecolor{currentstroke}{rgb}{0.690196,0.690196,0.690196}%
\pgfsetstrokecolor{currentstroke}%
\pgfsetstrokeopacity{0.200000}%
\pgfsetdash{}{0pt}%
\pgfpathmoveto{\pgfqpoint{7.900662in}{5.094806in}}%
\pgfpathlineto{\pgfqpoint{7.900662in}{9.318236in}}%
\pgfusepath{stroke}%
\end{pgfscope}%
\begin{pgfscope}%
\pgfsetbuttcap%
\pgfsetroundjoin%
\definecolor{currentfill}{rgb}{0.000000,0.000000,0.000000}%
\pgfsetfillcolor{currentfill}%
\pgfsetlinewidth{0.803000pt}%
\definecolor{currentstroke}{rgb}{0.000000,0.000000,0.000000}%
\pgfsetstrokecolor{currentstroke}%
\pgfsetdash{}{0pt}%
\pgfsys@defobject{currentmarker}{\pgfqpoint{0.000000in}{-0.048611in}}{\pgfqpoint{0.000000in}{0.000000in}}{%
\pgfpathmoveto{\pgfqpoint{0.000000in}{0.000000in}}%
\pgfpathlineto{\pgfqpoint{0.000000in}{-0.048611in}}%
\pgfusepath{stroke,fill}%
}%
\begin{pgfscope}%
\pgfsys@transformshift{7.900662in}{5.094806in}%
\pgfsys@useobject{currentmarker}{}%
\end{pgfscope}%
\end{pgfscope}%
\begin{pgfscope}%
\pgfpathrectangle{\pgfqpoint{0.688192in}{5.094806in}}{\pgfqpoint{11.096108in}{4.223431in}}%
\pgfusepath{clip}%
\pgfsetrectcap%
\pgfsetroundjoin%
\pgfsetlinewidth{0.803000pt}%
\definecolor{currentstroke}{rgb}{0.690196,0.690196,0.690196}%
\pgfsetstrokecolor{currentstroke}%
\pgfsetstrokeopacity{0.200000}%
\pgfsetdash{}{0pt}%
\pgfpathmoveto{\pgfqpoint{9.010272in}{5.094806in}}%
\pgfpathlineto{\pgfqpoint{9.010272in}{9.318236in}}%
\pgfusepath{stroke}%
\end{pgfscope}%
\begin{pgfscope}%
\pgfsetbuttcap%
\pgfsetroundjoin%
\definecolor{currentfill}{rgb}{0.000000,0.000000,0.000000}%
\pgfsetfillcolor{currentfill}%
\pgfsetlinewidth{0.803000pt}%
\definecolor{currentstroke}{rgb}{0.000000,0.000000,0.000000}%
\pgfsetstrokecolor{currentstroke}%
\pgfsetdash{}{0pt}%
\pgfsys@defobject{currentmarker}{\pgfqpoint{0.000000in}{-0.048611in}}{\pgfqpoint{0.000000in}{0.000000in}}{%
\pgfpathmoveto{\pgfqpoint{0.000000in}{0.000000in}}%
\pgfpathlineto{\pgfqpoint{0.000000in}{-0.048611in}}%
\pgfusepath{stroke,fill}%
}%
\begin{pgfscope}%
\pgfsys@transformshift{9.010272in}{5.094806in}%
\pgfsys@useobject{currentmarker}{}%
\end{pgfscope}%
\end{pgfscope}%
\begin{pgfscope}%
\pgfpathrectangle{\pgfqpoint{0.688192in}{5.094806in}}{\pgfqpoint{11.096108in}{4.223431in}}%
\pgfusepath{clip}%
\pgfsetrectcap%
\pgfsetroundjoin%
\pgfsetlinewidth{0.803000pt}%
\definecolor{currentstroke}{rgb}{0.690196,0.690196,0.690196}%
\pgfsetstrokecolor{currentstroke}%
\pgfsetstrokeopacity{0.200000}%
\pgfsetdash{}{0pt}%
\pgfpathmoveto{\pgfqpoint{10.119883in}{5.094806in}}%
\pgfpathlineto{\pgfqpoint{10.119883in}{9.318236in}}%
\pgfusepath{stroke}%
\end{pgfscope}%
\begin{pgfscope}%
\pgfsetbuttcap%
\pgfsetroundjoin%
\definecolor{currentfill}{rgb}{0.000000,0.000000,0.000000}%
\pgfsetfillcolor{currentfill}%
\pgfsetlinewidth{0.803000pt}%
\definecolor{currentstroke}{rgb}{0.000000,0.000000,0.000000}%
\pgfsetstrokecolor{currentstroke}%
\pgfsetdash{}{0pt}%
\pgfsys@defobject{currentmarker}{\pgfqpoint{0.000000in}{-0.048611in}}{\pgfqpoint{0.000000in}{0.000000in}}{%
\pgfpathmoveto{\pgfqpoint{0.000000in}{0.000000in}}%
\pgfpathlineto{\pgfqpoint{0.000000in}{-0.048611in}}%
\pgfusepath{stroke,fill}%
}%
\begin{pgfscope}%
\pgfsys@transformshift{10.119883in}{5.094806in}%
\pgfsys@useobject{currentmarker}{}%
\end{pgfscope}%
\end{pgfscope}%
\begin{pgfscope}%
\pgfpathrectangle{\pgfqpoint{0.688192in}{5.094806in}}{\pgfqpoint{11.096108in}{4.223431in}}%
\pgfusepath{clip}%
\pgfsetrectcap%
\pgfsetroundjoin%
\pgfsetlinewidth{0.803000pt}%
\definecolor{currentstroke}{rgb}{0.690196,0.690196,0.690196}%
\pgfsetstrokecolor{currentstroke}%
\pgfsetstrokeopacity{0.200000}%
\pgfsetdash{}{0pt}%
\pgfpathmoveto{\pgfqpoint{11.229494in}{5.094806in}}%
\pgfpathlineto{\pgfqpoint{11.229494in}{9.318236in}}%
\pgfusepath{stroke}%
\end{pgfscope}%
\begin{pgfscope}%
\pgfsetbuttcap%
\pgfsetroundjoin%
\definecolor{currentfill}{rgb}{0.000000,0.000000,0.000000}%
\pgfsetfillcolor{currentfill}%
\pgfsetlinewidth{0.803000pt}%
\definecolor{currentstroke}{rgb}{0.000000,0.000000,0.000000}%
\pgfsetstrokecolor{currentstroke}%
\pgfsetdash{}{0pt}%
\pgfsys@defobject{currentmarker}{\pgfqpoint{0.000000in}{-0.048611in}}{\pgfqpoint{0.000000in}{0.000000in}}{%
\pgfpathmoveto{\pgfqpoint{0.000000in}{0.000000in}}%
\pgfpathlineto{\pgfqpoint{0.000000in}{-0.048611in}}%
\pgfusepath{stroke,fill}%
}%
\begin{pgfscope}%
\pgfsys@transformshift{11.229494in}{5.094806in}%
\pgfsys@useobject{currentmarker}{}%
\end{pgfscope}%
\end{pgfscope}%
\begin{pgfscope}%
\pgfpathrectangle{\pgfqpoint{0.688192in}{5.094806in}}{\pgfqpoint{11.096108in}{4.223431in}}%
\pgfusepath{clip}%
\pgfsetrectcap%
\pgfsetroundjoin%
\pgfsetlinewidth{0.803000pt}%
\definecolor{currentstroke}{rgb}{0.690196,0.690196,0.690196}%
\pgfsetstrokecolor{currentstroke}%
\pgfsetstrokeopacity{0.050000}%
\pgfsetdash{}{0pt}%
\pgfpathmoveto{\pgfqpoint{0.799153in}{5.094806in}}%
\pgfpathlineto{\pgfqpoint{0.799153in}{9.318236in}}%
\pgfusepath{stroke}%
\end{pgfscope}%
\begin{pgfscope}%
\pgfsetbuttcap%
\pgfsetroundjoin%
\definecolor{currentfill}{rgb}{0.000000,0.000000,0.000000}%
\pgfsetfillcolor{currentfill}%
\pgfsetlinewidth{0.602250pt}%
\definecolor{currentstroke}{rgb}{0.000000,0.000000,0.000000}%
\pgfsetstrokecolor{currentstroke}%
\pgfsetdash{}{0pt}%
\pgfsys@defobject{currentmarker}{\pgfqpoint{0.000000in}{-0.027778in}}{\pgfqpoint{0.000000in}{0.000000in}}{%
\pgfpathmoveto{\pgfqpoint{0.000000in}{0.000000in}}%
\pgfpathlineto{\pgfqpoint{0.000000in}{-0.027778in}}%
\pgfusepath{stroke,fill}%
}%
\begin{pgfscope}%
\pgfsys@transformshift{0.799153in}{5.094806in}%
\pgfsys@useobject{currentmarker}{}%
\end{pgfscope}%
\end{pgfscope}%
\begin{pgfscope}%
\pgfpathrectangle{\pgfqpoint{0.688192in}{5.094806in}}{\pgfqpoint{11.096108in}{4.223431in}}%
\pgfusepath{clip}%
\pgfsetrectcap%
\pgfsetroundjoin%
\pgfsetlinewidth{0.803000pt}%
\definecolor{currentstroke}{rgb}{0.690196,0.690196,0.690196}%
\pgfsetstrokecolor{currentstroke}%
\pgfsetstrokeopacity{0.050000}%
\pgfsetdash{}{0pt}%
\pgfpathmoveto{\pgfqpoint{1.021075in}{5.094806in}}%
\pgfpathlineto{\pgfqpoint{1.021075in}{9.318236in}}%
\pgfusepath{stroke}%
\end{pgfscope}%
\begin{pgfscope}%
\pgfsetbuttcap%
\pgfsetroundjoin%
\definecolor{currentfill}{rgb}{0.000000,0.000000,0.000000}%
\pgfsetfillcolor{currentfill}%
\pgfsetlinewidth{0.602250pt}%
\definecolor{currentstroke}{rgb}{0.000000,0.000000,0.000000}%
\pgfsetstrokecolor{currentstroke}%
\pgfsetdash{}{0pt}%
\pgfsys@defobject{currentmarker}{\pgfqpoint{0.000000in}{-0.027778in}}{\pgfqpoint{0.000000in}{0.000000in}}{%
\pgfpathmoveto{\pgfqpoint{0.000000in}{0.000000in}}%
\pgfpathlineto{\pgfqpoint{0.000000in}{-0.027778in}}%
\pgfusepath{stroke,fill}%
}%
\begin{pgfscope}%
\pgfsys@transformshift{1.021075in}{5.094806in}%
\pgfsys@useobject{currentmarker}{}%
\end{pgfscope}%
\end{pgfscope}%
\begin{pgfscope}%
\pgfpathrectangle{\pgfqpoint{0.688192in}{5.094806in}}{\pgfqpoint{11.096108in}{4.223431in}}%
\pgfusepath{clip}%
\pgfsetrectcap%
\pgfsetroundjoin%
\pgfsetlinewidth{0.803000pt}%
\definecolor{currentstroke}{rgb}{0.690196,0.690196,0.690196}%
\pgfsetstrokecolor{currentstroke}%
\pgfsetstrokeopacity{0.050000}%
\pgfsetdash{}{0pt}%
\pgfpathmoveto{\pgfqpoint{1.464919in}{5.094806in}}%
\pgfpathlineto{\pgfqpoint{1.464919in}{9.318236in}}%
\pgfusepath{stroke}%
\end{pgfscope}%
\begin{pgfscope}%
\pgfsetbuttcap%
\pgfsetroundjoin%
\definecolor{currentfill}{rgb}{0.000000,0.000000,0.000000}%
\pgfsetfillcolor{currentfill}%
\pgfsetlinewidth{0.602250pt}%
\definecolor{currentstroke}{rgb}{0.000000,0.000000,0.000000}%
\pgfsetstrokecolor{currentstroke}%
\pgfsetdash{}{0pt}%
\pgfsys@defobject{currentmarker}{\pgfqpoint{0.000000in}{-0.027778in}}{\pgfqpoint{0.000000in}{0.000000in}}{%
\pgfpathmoveto{\pgfqpoint{0.000000in}{0.000000in}}%
\pgfpathlineto{\pgfqpoint{0.000000in}{-0.027778in}}%
\pgfusepath{stroke,fill}%
}%
\begin{pgfscope}%
\pgfsys@transformshift{1.464919in}{5.094806in}%
\pgfsys@useobject{currentmarker}{}%
\end{pgfscope}%
\end{pgfscope}%
\begin{pgfscope}%
\pgfpathrectangle{\pgfqpoint{0.688192in}{5.094806in}}{\pgfqpoint{11.096108in}{4.223431in}}%
\pgfusepath{clip}%
\pgfsetrectcap%
\pgfsetroundjoin%
\pgfsetlinewidth{0.803000pt}%
\definecolor{currentstroke}{rgb}{0.690196,0.690196,0.690196}%
\pgfsetstrokecolor{currentstroke}%
\pgfsetstrokeopacity{0.050000}%
\pgfsetdash{}{0pt}%
\pgfpathmoveto{\pgfqpoint{1.686841in}{5.094806in}}%
\pgfpathlineto{\pgfqpoint{1.686841in}{9.318236in}}%
\pgfusepath{stroke}%
\end{pgfscope}%
\begin{pgfscope}%
\pgfsetbuttcap%
\pgfsetroundjoin%
\definecolor{currentfill}{rgb}{0.000000,0.000000,0.000000}%
\pgfsetfillcolor{currentfill}%
\pgfsetlinewidth{0.602250pt}%
\definecolor{currentstroke}{rgb}{0.000000,0.000000,0.000000}%
\pgfsetstrokecolor{currentstroke}%
\pgfsetdash{}{0pt}%
\pgfsys@defobject{currentmarker}{\pgfqpoint{0.000000in}{-0.027778in}}{\pgfqpoint{0.000000in}{0.000000in}}{%
\pgfpathmoveto{\pgfqpoint{0.000000in}{0.000000in}}%
\pgfpathlineto{\pgfqpoint{0.000000in}{-0.027778in}}%
\pgfusepath{stroke,fill}%
}%
\begin{pgfscope}%
\pgfsys@transformshift{1.686841in}{5.094806in}%
\pgfsys@useobject{currentmarker}{}%
\end{pgfscope}%
\end{pgfscope}%
\begin{pgfscope}%
\pgfpathrectangle{\pgfqpoint{0.688192in}{5.094806in}}{\pgfqpoint{11.096108in}{4.223431in}}%
\pgfusepath{clip}%
\pgfsetrectcap%
\pgfsetroundjoin%
\pgfsetlinewidth{0.803000pt}%
\definecolor{currentstroke}{rgb}{0.690196,0.690196,0.690196}%
\pgfsetstrokecolor{currentstroke}%
\pgfsetstrokeopacity{0.050000}%
\pgfsetdash{}{0pt}%
\pgfpathmoveto{\pgfqpoint{1.908763in}{5.094806in}}%
\pgfpathlineto{\pgfqpoint{1.908763in}{9.318236in}}%
\pgfusepath{stroke}%
\end{pgfscope}%
\begin{pgfscope}%
\pgfsetbuttcap%
\pgfsetroundjoin%
\definecolor{currentfill}{rgb}{0.000000,0.000000,0.000000}%
\pgfsetfillcolor{currentfill}%
\pgfsetlinewidth{0.602250pt}%
\definecolor{currentstroke}{rgb}{0.000000,0.000000,0.000000}%
\pgfsetstrokecolor{currentstroke}%
\pgfsetdash{}{0pt}%
\pgfsys@defobject{currentmarker}{\pgfqpoint{0.000000in}{-0.027778in}}{\pgfqpoint{0.000000in}{0.000000in}}{%
\pgfpathmoveto{\pgfqpoint{0.000000in}{0.000000in}}%
\pgfpathlineto{\pgfqpoint{0.000000in}{-0.027778in}}%
\pgfusepath{stroke,fill}%
}%
\begin{pgfscope}%
\pgfsys@transformshift{1.908763in}{5.094806in}%
\pgfsys@useobject{currentmarker}{}%
\end{pgfscope}%
\end{pgfscope}%
\begin{pgfscope}%
\pgfpathrectangle{\pgfqpoint{0.688192in}{5.094806in}}{\pgfqpoint{11.096108in}{4.223431in}}%
\pgfusepath{clip}%
\pgfsetrectcap%
\pgfsetroundjoin%
\pgfsetlinewidth{0.803000pt}%
\definecolor{currentstroke}{rgb}{0.690196,0.690196,0.690196}%
\pgfsetstrokecolor{currentstroke}%
\pgfsetstrokeopacity{0.050000}%
\pgfsetdash{}{0pt}%
\pgfpathmoveto{\pgfqpoint{2.130686in}{5.094806in}}%
\pgfpathlineto{\pgfqpoint{2.130686in}{9.318236in}}%
\pgfusepath{stroke}%
\end{pgfscope}%
\begin{pgfscope}%
\pgfsetbuttcap%
\pgfsetroundjoin%
\definecolor{currentfill}{rgb}{0.000000,0.000000,0.000000}%
\pgfsetfillcolor{currentfill}%
\pgfsetlinewidth{0.602250pt}%
\definecolor{currentstroke}{rgb}{0.000000,0.000000,0.000000}%
\pgfsetstrokecolor{currentstroke}%
\pgfsetdash{}{0pt}%
\pgfsys@defobject{currentmarker}{\pgfqpoint{0.000000in}{-0.027778in}}{\pgfqpoint{0.000000in}{0.000000in}}{%
\pgfpathmoveto{\pgfqpoint{0.000000in}{0.000000in}}%
\pgfpathlineto{\pgfqpoint{0.000000in}{-0.027778in}}%
\pgfusepath{stroke,fill}%
}%
\begin{pgfscope}%
\pgfsys@transformshift{2.130686in}{5.094806in}%
\pgfsys@useobject{currentmarker}{}%
\end{pgfscope}%
\end{pgfscope}%
\begin{pgfscope}%
\pgfpathrectangle{\pgfqpoint{0.688192in}{5.094806in}}{\pgfqpoint{11.096108in}{4.223431in}}%
\pgfusepath{clip}%
\pgfsetrectcap%
\pgfsetroundjoin%
\pgfsetlinewidth{0.803000pt}%
\definecolor{currentstroke}{rgb}{0.690196,0.690196,0.690196}%
\pgfsetstrokecolor{currentstroke}%
\pgfsetstrokeopacity{0.050000}%
\pgfsetdash{}{0pt}%
\pgfpathmoveto{\pgfqpoint{2.574530in}{5.094806in}}%
\pgfpathlineto{\pgfqpoint{2.574530in}{9.318236in}}%
\pgfusepath{stroke}%
\end{pgfscope}%
\begin{pgfscope}%
\pgfsetbuttcap%
\pgfsetroundjoin%
\definecolor{currentfill}{rgb}{0.000000,0.000000,0.000000}%
\pgfsetfillcolor{currentfill}%
\pgfsetlinewidth{0.602250pt}%
\definecolor{currentstroke}{rgb}{0.000000,0.000000,0.000000}%
\pgfsetstrokecolor{currentstroke}%
\pgfsetdash{}{0pt}%
\pgfsys@defobject{currentmarker}{\pgfqpoint{0.000000in}{-0.027778in}}{\pgfqpoint{0.000000in}{0.000000in}}{%
\pgfpathmoveto{\pgfqpoint{0.000000in}{0.000000in}}%
\pgfpathlineto{\pgfqpoint{0.000000in}{-0.027778in}}%
\pgfusepath{stroke,fill}%
}%
\begin{pgfscope}%
\pgfsys@transformshift{2.574530in}{5.094806in}%
\pgfsys@useobject{currentmarker}{}%
\end{pgfscope}%
\end{pgfscope}%
\begin{pgfscope}%
\pgfpathrectangle{\pgfqpoint{0.688192in}{5.094806in}}{\pgfqpoint{11.096108in}{4.223431in}}%
\pgfusepath{clip}%
\pgfsetrectcap%
\pgfsetroundjoin%
\pgfsetlinewidth{0.803000pt}%
\definecolor{currentstroke}{rgb}{0.690196,0.690196,0.690196}%
\pgfsetstrokecolor{currentstroke}%
\pgfsetstrokeopacity{0.050000}%
\pgfsetdash{}{0pt}%
\pgfpathmoveto{\pgfqpoint{2.796452in}{5.094806in}}%
\pgfpathlineto{\pgfqpoint{2.796452in}{9.318236in}}%
\pgfusepath{stroke}%
\end{pgfscope}%
\begin{pgfscope}%
\pgfsetbuttcap%
\pgfsetroundjoin%
\definecolor{currentfill}{rgb}{0.000000,0.000000,0.000000}%
\pgfsetfillcolor{currentfill}%
\pgfsetlinewidth{0.602250pt}%
\definecolor{currentstroke}{rgb}{0.000000,0.000000,0.000000}%
\pgfsetstrokecolor{currentstroke}%
\pgfsetdash{}{0pt}%
\pgfsys@defobject{currentmarker}{\pgfqpoint{0.000000in}{-0.027778in}}{\pgfqpoint{0.000000in}{0.000000in}}{%
\pgfpathmoveto{\pgfqpoint{0.000000in}{0.000000in}}%
\pgfpathlineto{\pgfqpoint{0.000000in}{-0.027778in}}%
\pgfusepath{stroke,fill}%
}%
\begin{pgfscope}%
\pgfsys@transformshift{2.796452in}{5.094806in}%
\pgfsys@useobject{currentmarker}{}%
\end{pgfscope}%
\end{pgfscope}%
\begin{pgfscope}%
\pgfpathrectangle{\pgfqpoint{0.688192in}{5.094806in}}{\pgfqpoint{11.096108in}{4.223431in}}%
\pgfusepath{clip}%
\pgfsetrectcap%
\pgfsetroundjoin%
\pgfsetlinewidth{0.803000pt}%
\definecolor{currentstroke}{rgb}{0.690196,0.690196,0.690196}%
\pgfsetstrokecolor{currentstroke}%
\pgfsetstrokeopacity{0.050000}%
\pgfsetdash{}{0pt}%
\pgfpathmoveto{\pgfqpoint{3.018374in}{5.094806in}}%
\pgfpathlineto{\pgfqpoint{3.018374in}{9.318236in}}%
\pgfusepath{stroke}%
\end{pgfscope}%
\begin{pgfscope}%
\pgfsetbuttcap%
\pgfsetroundjoin%
\definecolor{currentfill}{rgb}{0.000000,0.000000,0.000000}%
\pgfsetfillcolor{currentfill}%
\pgfsetlinewidth{0.602250pt}%
\definecolor{currentstroke}{rgb}{0.000000,0.000000,0.000000}%
\pgfsetstrokecolor{currentstroke}%
\pgfsetdash{}{0pt}%
\pgfsys@defobject{currentmarker}{\pgfqpoint{0.000000in}{-0.027778in}}{\pgfqpoint{0.000000in}{0.000000in}}{%
\pgfpathmoveto{\pgfqpoint{0.000000in}{0.000000in}}%
\pgfpathlineto{\pgfqpoint{0.000000in}{-0.027778in}}%
\pgfusepath{stroke,fill}%
}%
\begin{pgfscope}%
\pgfsys@transformshift{3.018374in}{5.094806in}%
\pgfsys@useobject{currentmarker}{}%
\end{pgfscope}%
\end{pgfscope}%
\begin{pgfscope}%
\pgfpathrectangle{\pgfqpoint{0.688192in}{5.094806in}}{\pgfqpoint{11.096108in}{4.223431in}}%
\pgfusepath{clip}%
\pgfsetrectcap%
\pgfsetroundjoin%
\pgfsetlinewidth{0.803000pt}%
\definecolor{currentstroke}{rgb}{0.690196,0.690196,0.690196}%
\pgfsetstrokecolor{currentstroke}%
\pgfsetstrokeopacity{0.050000}%
\pgfsetdash{}{0pt}%
\pgfpathmoveto{\pgfqpoint{3.240296in}{5.094806in}}%
\pgfpathlineto{\pgfqpoint{3.240296in}{9.318236in}}%
\pgfusepath{stroke}%
\end{pgfscope}%
\begin{pgfscope}%
\pgfsetbuttcap%
\pgfsetroundjoin%
\definecolor{currentfill}{rgb}{0.000000,0.000000,0.000000}%
\pgfsetfillcolor{currentfill}%
\pgfsetlinewidth{0.602250pt}%
\definecolor{currentstroke}{rgb}{0.000000,0.000000,0.000000}%
\pgfsetstrokecolor{currentstroke}%
\pgfsetdash{}{0pt}%
\pgfsys@defobject{currentmarker}{\pgfqpoint{0.000000in}{-0.027778in}}{\pgfqpoint{0.000000in}{0.000000in}}{%
\pgfpathmoveto{\pgfqpoint{0.000000in}{0.000000in}}%
\pgfpathlineto{\pgfqpoint{0.000000in}{-0.027778in}}%
\pgfusepath{stroke,fill}%
}%
\begin{pgfscope}%
\pgfsys@transformshift{3.240296in}{5.094806in}%
\pgfsys@useobject{currentmarker}{}%
\end{pgfscope}%
\end{pgfscope}%
\begin{pgfscope}%
\pgfpathrectangle{\pgfqpoint{0.688192in}{5.094806in}}{\pgfqpoint{11.096108in}{4.223431in}}%
\pgfusepath{clip}%
\pgfsetrectcap%
\pgfsetroundjoin%
\pgfsetlinewidth{0.803000pt}%
\definecolor{currentstroke}{rgb}{0.690196,0.690196,0.690196}%
\pgfsetstrokecolor{currentstroke}%
\pgfsetstrokeopacity{0.050000}%
\pgfsetdash{}{0pt}%
\pgfpathmoveto{\pgfqpoint{3.684141in}{5.094806in}}%
\pgfpathlineto{\pgfqpoint{3.684141in}{9.318236in}}%
\pgfusepath{stroke}%
\end{pgfscope}%
\begin{pgfscope}%
\pgfsetbuttcap%
\pgfsetroundjoin%
\definecolor{currentfill}{rgb}{0.000000,0.000000,0.000000}%
\pgfsetfillcolor{currentfill}%
\pgfsetlinewidth{0.602250pt}%
\definecolor{currentstroke}{rgb}{0.000000,0.000000,0.000000}%
\pgfsetstrokecolor{currentstroke}%
\pgfsetdash{}{0pt}%
\pgfsys@defobject{currentmarker}{\pgfqpoint{0.000000in}{-0.027778in}}{\pgfqpoint{0.000000in}{0.000000in}}{%
\pgfpathmoveto{\pgfqpoint{0.000000in}{0.000000in}}%
\pgfpathlineto{\pgfqpoint{0.000000in}{-0.027778in}}%
\pgfusepath{stroke,fill}%
}%
\begin{pgfscope}%
\pgfsys@transformshift{3.684141in}{5.094806in}%
\pgfsys@useobject{currentmarker}{}%
\end{pgfscope}%
\end{pgfscope}%
\begin{pgfscope}%
\pgfpathrectangle{\pgfqpoint{0.688192in}{5.094806in}}{\pgfqpoint{11.096108in}{4.223431in}}%
\pgfusepath{clip}%
\pgfsetrectcap%
\pgfsetroundjoin%
\pgfsetlinewidth{0.803000pt}%
\definecolor{currentstroke}{rgb}{0.690196,0.690196,0.690196}%
\pgfsetstrokecolor{currentstroke}%
\pgfsetstrokeopacity{0.050000}%
\pgfsetdash{}{0pt}%
\pgfpathmoveto{\pgfqpoint{3.906063in}{5.094806in}}%
\pgfpathlineto{\pgfqpoint{3.906063in}{9.318236in}}%
\pgfusepath{stroke}%
\end{pgfscope}%
\begin{pgfscope}%
\pgfsetbuttcap%
\pgfsetroundjoin%
\definecolor{currentfill}{rgb}{0.000000,0.000000,0.000000}%
\pgfsetfillcolor{currentfill}%
\pgfsetlinewidth{0.602250pt}%
\definecolor{currentstroke}{rgb}{0.000000,0.000000,0.000000}%
\pgfsetstrokecolor{currentstroke}%
\pgfsetdash{}{0pt}%
\pgfsys@defobject{currentmarker}{\pgfqpoint{0.000000in}{-0.027778in}}{\pgfqpoint{0.000000in}{0.000000in}}{%
\pgfpathmoveto{\pgfqpoint{0.000000in}{0.000000in}}%
\pgfpathlineto{\pgfqpoint{0.000000in}{-0.027778in}}%
\pgfusepath{stroke,fill}%
}%
\begin{pgfscope}%
\pgfsys@transformshift{3.906063in}{5.094806in}%
\pgfsys@useobject{currentmarker}{}%
\end{pgfscope}%
\end{pgfscope}%
\begin{pgfscope}%
\pgfpathrectangle{\pgfqpoint{0.688192in}{5.094806in}}{\pgfqpoint{11.096108in}{4.223431in}}%
\pgfusepath{clip}%
\pgfsetrectcap%
\pgfsetroundjoin%
\pgfsetlinewidth{0.803000pt}%
\definecolor{currentstroke}{rgb}{0.690196,0.690196,0.690196}%
\pgfsetstrokecolor{currentstroke}%
\pgfsetstrokeopacity{0.050000}%
\pgfsetdash{}{0pt}%
\pgfpathmoveto{\pgfqpoint{4.127985in}{5.094806in}}%
\pgfpathlineto{\pgfqpoint{4.127985in}{9.318236in}}%
\pgfusepath{stroke}%
\end{pgfscope}%
\begin{pgfscope}%
\pgfsetbuttcap%
\pgfsetroundjoin%
\definecolor{currentfill}{rgb}{0.000000,0.000000,0.000000}%
\pgfsetfillcolor{currentfill}%
\pgfsetlinewidth{0.602250pt}%
\definecolor{currentstroke}{rgb}{0.000000,0.000000,0.000000}%
\pgfsetstrokecolor{currentstroke}%
\pgfsetdash{}{0pt}%
\pgfsys@defobject{currentmarker}{\pgfqpoint{0.000000in}{-0.027778in}}{\pgfqpoint{0.000000in}{0.000000in}}{%
\pgfpathmoveto{\pgfqpoint{0.000000in}{0.000000in}}%
\pgfpathlineto{\pgfqpoint{0.000000in}{-0.027778in}}%
\pgfusepath{stroke,fill}%
}%
\begin{pgfscope}%
\pgfsys@transformshift{4.127985in}{5.094806in}%
\pgfsys@useobject{currentmarker}{}%
\end{pgfscope}%
\end{pgfscope}%
\begin{pgfscope}%
\pgfpathrectangle{\pgfqpoint{0.688192in}{5.094806in}}{\pgfqpoint{11.096108in}{4.223431in}}%
\pgfusepath{clip}%
\pgfsetrectcap%
\pgfsetroundjoin%
\pgfsetlinewidth{0.803000pt}%
\definecolor{currentstroke}{rgb}{0.690196,0.690196,0.690196}%
\pgfsetstrokecolor{currentstroke}%
\pgfsetstrokeopacity{0.050000}%
\pgfsetdash{}{0pt}%
\pgfpathmoveto{\pgfqpoint{4.349907in}{5.094806in}}%
\pgfpathlineto{\pgfqpoint{4.349907in}{9.318236in}}%
\pgfusepath{stroke}%
\end{pgfscope}%
\begin{pgfscope}%
\pgfsetbuttcap%
\pgfsetroundjoin%
\definecolor{currentfill}{rgb}{0.000000,0.000000,0.000000}%
\pgfsetfillcolor{currentfill}%
\pgfsetlinewidth{0.602250pt}%
\definecolor{currentstroke}{rgb}{0.000000,0.000000,0.000000}%
\pgfsetstrokecolor{currentstroke}%
\pgfsetdash{}{0pt}%
\pgfsys@defobject{currentmarker}{\pgfqpoint{0.000000in}{-0.027778in}}{\pgfqpoint{0.000000in}{0.000000in}}{%
\pgfpathmoveto{\pgfqpoint{0.000000in}{0.000000in}}%
\pgfpathlineto{\pgfqpoint{0.000000in}{-0.027778in}}%
\pgfusepath{stroke,fill}%
}%
\begin{pgfscope}%
\pgfsys@transformshift{4.349907in}{5.094806in}%
\pgfsys@useobject{currentmarker}{}%
\end{pgfscope}%
\end{pgfscope}%
\begin{pgfscope}%
\pgfpathrectangle{\pgfqpoint{0.688192in}{5.094806in}}{\pgfqpoint{11.096108in}{4.223431in}}%
\pgfusepath{clip}%
\pgfsetrectcap%
\pgfsetroundjoin%
\pgfsetlinewidth{0.803000pt}%
\definecolor{currentstroke}{rgb}{0.690196,0.690196,0.690196}%
\pgfsetstrokecolor{currentstroke}%
\pgfsetstrokeopacity{0.050000}%
\pgfsetdash{}{0pt}%
\pgfpathmoveto{\pgfqpoint{4.793751in}{5.094806in}}%
\pgfpathlineto{\pgfqpoint{4.793751in}{9.318236in}}%
\pgfusepath{stroke}%
\end{pgfscope}%
\begin{pgfscope}%
\pgfsetbuttcap%
\pgfsetroundjoin%
\definecolor{currentfill}{rgb}{0.000000,0.000000,0.000000}%
\pgfsetfillcolor{currentfill}%
\pgfsetlinewidth{0.602250pt}%
\definecolor{currentstroke}{rgb}{0.000000,0.000000,0.000000}%
\pgfsetstrokecolor{currentstroke}%
\pgfsetdash{}{0pt}%
\pgfsys@defobject{currentmarker}{\pgfqpoint{0.000000in}{-0.027778in}}{\pgfqpoint{0.000000in}{0.000000in}}{%
\pgfpathmoveto{\pgfqpoint{0.000000in}{0.000000in}}%
\pgfpathlineto{\pgfqpoint{0.000000in}{-0.027778in}}%
\pgfusepath{stroke,fill}%
}%
\begin{pgfscope}%
\pgfsys@transformshift{4.793751in}{5.094806in}%
\pgfsys@useobject{currentmarker}{}%
\end{pgfscope}%
\end{pgfscope}%
\begin{pgfscope}%
\pgfpathrectangle{\pgfqpoint{0.688192in}{5.094806in}}{\pgfqpoint{11.096108in}{4.223431in}}%
\pgfusepath{clip}%
\pgfsetrectcap%
\pgfsetroundjoin%
\pgfsetlinewidth{0.803000pt}%
\definecolor{currentstroke}{rgb}{0.690196,0.690196,0.690196}%
\pgfsetstrokecolor{currentstroke}%
\pgfsetstrokeopacity{0.050000}%
\pgfsetdash{}{0pt}%
\pgfpathmoveto{\pgfqpoint{5.015674in}{5.094806in}}%
\pgfpathlineto{\pgfqpoint{5.015674in}{9.318236in}}%
\pgfusepath{stroke}%
\end{pgfscope}%
\begin{pgfscope}%
\pgfsetbuttcap%
\pgfsetroundjoin%
\definecolor{currentfill}{rgb}{0.000000,0.000000,0.000000}%
\pgfsetfillcolor{currentfill}%
\pgfsetlinewidth{0.602250pt}%
\definecolor{currentstroke}{rgb}{0.000000,0.000000,0.000000}%
\pgfsetstrokecolor{currentstroke}%
\pgfsetdash{}{0pt}%
\pgfsys@defobject{currentmarker}{\pgfqpoint{0.000000in}{-0.027778in}}{\pgfqpoint{0.000000in}{0.000000in}}{%
\pgfpathmoveto{\pgfqpoint{0.000000in}{0.000000in}}%
\pgfpathlineto{\pgfqpoint{0.000000in}{-0.027778in}}%
\pgfusepath{stroke,fill}%
}%
\begin{pgfscope}%
\pgfsys@transformshift{5.015674in}{5.094806in}%
\pgfsys@useobject{currentmarker}{}%
\end{pgfscope}%
\end{pgfscope}%
\begin{pgfscope}%
\pgfpathrectangle{\pgfqpoint{0.688192in}{5.094806in}}{\pgfqpoint{11.096108in}{4.223431in}}%
\pgfusepath{clip}%
\pgfsetrectcap%
\pgfsetroundjoin%
\pgfsetlinewidth{0.803000pt}%
\definecolor{currentstroke}{rgb}{0.690196,0.690196,0.690196}%
\pgfsetstrokecolor{currentstroke}%
\pgfsetstrokeopacity{0.050000}%
\pgfsetdash{}{0pt}%
\pgfpathmoveto{\pgfqpoint{5.237596in}{5.094806in}}%
\pgfpathlineto{\pgfqpoint{5.237596in}{9.318236in}}%
\pgfusepath{stroke}%
\end{pgfscope}%
\begin{pgfscope}%
\pgfsetbuttcap%
\pgfsetroundjoin%
\definecolor{currentfill}{rgb}{0.000000,0.000000,0.000000}%
\pgfsetfillcolor{currentfill}%
\pgfsetlinewidth{0.602250pt}%
\definecolor{currentstroke}{rgb}{0.000000,0.000000,0.000000}%
\pgfsetstrokecolor{currentstroke}%
\pgfsetdash{}{0pt}%
\pgfsys@defobject{currentmarker}{\pgfqpoint{0.000000in}{-0.027778in}}{\pgfqpoint{0.000000in}{0.000000in}}{%
\pgfpathmoveto{\pgfqpoint{0.000000in}{0.000000in}}%
\pgfpathlineto{\pgfqpoint{0.000000in}{-0.027778in}}%
\pgfusepath{stroke,fill}%
}%
\begin{pgfscope}%
\pgfsys@transformshift{5.237596in}{5.094806in}%
\pgfsys@useobject{currentmarker}{}%
\end{pgfscope}%
\end{pgfscope}%
\begin{pgfscope}%
\pgfpathrectangle{\pgfqpoint{0.688192in}{5.094806in}}{\pgfqpoint{11.096108in}{4.223431in}}%
\pgfusepath{clip}%
\pgfsetrectcap%
\pgfsetroundjoin%
\pgfsetlinewidth{0.803000pt}%
\definecolor{currentstroke}{rgb}{0.690196,0.690196,0.690196}%
\pgfsetstrokecolor{currentstroke}%
\pgfsetstrokeopacity{0.050000}%
\pgfsetdash{}{0pt}%
\pgfpathmoveto{\pgfqpoint{5.459518in}{5.094806in}}%
\pgfpathlineto{\pgfqpoint{5.459518in}{9.318236in}}%
\pgfusepath{stroke}%
\end{pgfscope}%
\begin{pgfscope}%
\pgfsetbuttcap%
\pgfsetroundjoin%
\definecolor{currentfill}{rgb}{0.000000,0.000000,0.000000}%
\pgfsetfillcolor{currentfill}%
\pgfsetlinewidth{0.602250pt}%
\definecolor{currentstroke}{rgb}{0.000000,0.000000,0.000000}%
\pgfsetstrokecolor{currentstroke}%
\pgfsetdash{}{0pt}%
\pgfsys@defobject{currentmarker}{\pgfqpoint{0.000000in}{-0.027778in}}{\pgfqpoint{0.000000in}{0.000000in}}{%
\pgfpathmoveto{\pgfqpoint{0.000000in}{0.000000in}}%
\pgfpathlineto{\pgfqpoint{0.000000in}{-0.027778in}}%
\pgfusepath{stroke,fill}%
}%
\begin{pgfscope}%
\pgfsys@transformshift{5.459518in}{5.094806in}%
\pgfsys@useobject{currentmarker}{}%
\end{pgfscope}%
\end{pgfscope}%
\begin{pgfscope}%
\pgfpathrectangle{\pgfqpoint{0.688192in}{5.094806in}}{\pgfqpoint{11.096108in}{4.223431in}}%
\pgfusepath{clip}%
\pgfsetrectcap%
\pgfsetroundjoin%
\pgfsetlinewidth{0.803000pt}%
\definecolor{currentstroke}{rgb}{0.690196,0.690196,0.690196}%
\pgfsetstrokecolor{currentstroke}%
\pgfsetstrokeopacity{0.050000}%
\pgfsetdash{}{0pt}%
\pgfpathmoveto{\pgfqpoint{5.903362in}{5.094806in}}%
\pgfpathlineto{\pgfqpoint{5.903362in}{9.318236in}}%
\pgfusepath{stroke}%
\end{pgfscope}%
\begin{pgfscope}%
\pgfsetbuttcap%
\pgfsetroundjoin%
\definecolor{currentfill}{rgb}{0.000000,0.000000,0.000000}%
\pgfsetfillcolor{currentfill}%
\pgfsetlinewidth{0.602250pt}%
\definecolor{currentstroke}{rgb}{0.000000,0.000000,0.000000}%
\pgfsetstrokecolor{currentstroke}%
\pgfsetdash{}{0pt}%
\pgfsys@defobject{currentmarker}{\pgfqpoint{0.000000in}{-0.027778in}}{\pgfqpoint{0.000000in}{0.000000in}}{%
\pgfpathmoveto{\pgfqpoint{0.000000in}{0.000000in}}%
\pgfpathlineto{\pgfqpoint{0.000000in}{-0.027778in}}%
\pgfusepath{stroke,fill}%
}%
\begin{pgfscope}%
\pgfsys@transformshift{5.903362in}{5.094806in}%
\pgfsys@useobject{currentmarker}{}%
\end{pgfscope}%
\end{pgfscope}%
\begin{pgfscope}%
\pgfpathrectangle{\pgfqpoint{0.688192in}{5.094806in}}{\pgfqpoint{11.096108in}{4.223431in}}%
\pgfusepath{clip}%
\pgfsetrectcap%
\pgfsetroundjoin%
\pgfsetlinewidth{0.803000pt}%
\definecolor{currentstroke}{rgb}{0.690196,0.690196,0.690196}%
\pgfsetstrokecolor{currentstroke}%
\pgfsetstrokeopacity{0.050000}%
\pgfsetdash{}{0pt}%
\pgfpathmoveto{\pgfqpoint{6.125284in}{5.094806in}}%
\pgfpathlineto{\pgfqpoint{6.125284in}{9.318236in}}%
\pgfusepath{stroke}%
\end{pgfscope}%
\begin{pgfscope}%
\pgfsetbuttcap%
\pgfsetroundjoin%
\definecolor{currentfill}{rgb}{0.000000,0.000000,0.000000}%
\pgfsetfillcolor{currentfill}%
\pgfsetlinewidth{0.602250pt}%
\definecolor{currentstroke}{rgb}{0.000000,0.000000,0.000000}%
\pgfsetstrokecolor{currentstroke}%
\pgfsetdash{}{0pt}%
\pgfsys@defobject{currentmarker}{\pgfqpoint{0.000000in}{-0.027778in}}{\pgfqpoint{0.000000in}{0.000000in}}{%
\pgfpathmoveto{\pgfqpoint{0.000000in}{0.000000in}}%
\pgfpathlineto{\pgfqpoint{0.000000in}{-0.027778in}}%
\pgfusepath{stroke,fill}%
}%
\begin{pgfscope}%
\pgfsys@transformshift{6.125284in}{5.094806in}%
\pgfsys@useobject{currentmarker}{}%
\end{pgfscope}%
\end{pgfscope}%
\begin{pgfscope}%
\pgfpathrectangle{\pgfqpoint{0.688192in}{5.094806in}}{\pgfqpoint{11.096108in}{4.223431in}}%
\pgfusepath{clip}%
\pgfsetrectcap%
\pgfsetroundjoin%
\pgfsetlinewidth{0.803000pt}%
\definecolor{currentstroke}{rgb}{0.690196,0.690196,0.690196}%
\pgfsetstrokecolor{currentstroke}%
\pgfsetstrokeopacity{0.050000}%
\pgfsetdash{}{0pt}%
\pgfpathmoveto{\pgfqpoint{6.347207in}{5.094806in}}%
\pgfpathlineto{\pgfqpoint{6.347207in}{9.318236in}}%
\pgfusepath{stroke}%
\end{pgfscope}%
\begin{pgfscope}%
\pgfsetbuttcap%
\pgfsetroundjoin%
\definecolor{currentfill}{rgb}{0.000000,0.000000,0.000000}%
\pgfsetfillcolor{currentfill}%
\pgfsetlinewidth{0.602250pt}%
\definecolor{currentstroke}{rgb}{0.000000,0.000000,0.000000}%
\pgfsetstrokecolor{currentstroke}%
\pgfsetdash{}{0pt}%
\pgfsys@defobject{currentmarker}{\pgfqpoint{0.000000in}{-0.027778in}}{\pgfqpoint{0.000000in}{0.000000in}}{%
\pgfpathmoveto{\pgfqpoint{0.000000in}{0.000000in}}%
\pgfpathlineto{\pgfqpoint{0.000000in}{-0.027778in}}%
\pgfusepath{stroke,fill}%
}%
\begin{pgfscope}%
\pgfsys@transformshift{6.347207in}{5.094806in}%
\pgfsys@useobject{currentmarker}{}%
\end{pgfscope}%
\end{pgfscope}%
\begin{pgfscope}%
\pgfpathrectangle{\pgfqpoint{0.688192in}{5.094806in}}{\pgfqpoint{11.096108in}{4.223431in}}%
\pgfusepath{clip}%
\pgfsetrectcap%
\pgfsetroundjoin%
\pgfsetlinewidth{0.803000pt}%
\definecolor{currentstroke}{rgb}{0.690196,0.690196,0.690196}%
\pgfsetstrokecolor{currentstroke}%
\pgfsetstrokeopacity{0.050000}%
\pgfsetdash{}{0pt}%
\pgfpathmoveto{\pgfqpoint{6.569129in}{5.094806in}}%
\pgfpathlineto{\pgfqpoint{6.569129in}{9.318236in}}%
\pgfusepath{stroke}%
\end{pgfscope}%
\begin{pgfscope}%
\pgfsetbuttcap%
\pgfsetroundjoin%
\definecolor{currentfill}{rgb}{0.000000,0.000000,0.000000}%
\pgfsetfillcolor{currentfill}%
\pgfsetlinewidth{0.602250pt}%
\definecolor{currentstroke}{rgb}{0.000000,0.000000,0.000000}%
\pgfsetstrokecolor{currentstroke}%
\pgfsetdash{}{0pt}%
\pgfsys@defobject{currentmarker}{\pgfqpoint{0.000000in}{-0.027778in}}{\pgfqpoint{0.000000in}{0.000000in}}{%
\pgfpathmoveto{\pgfqpoint{0.000000in}{0.000000in}}%
\pgfpathlineto{\pgfqpoint{0.000000in}{-0.027778in}}%
\pgfusepath{stroke,fill}%
}%
\begin{pgfscope}%
\pgfsys@transformshift{6.569129in}{5.094806in}%
\pgfsys@useobject{currentmarker}{}%
\end{pgfscope}%
\end{pgfscope}%
\begin{pgfscope}%
\pgfpathrectangle{\pgfqpoint{0.688192in}{5.094806in}}{\pgfqpoint{11.096108in}{4.223431in}}%
\pgfusepath{clip}%
\pgfsetrectcap%
\pgfsetroundjoin%
\pgfsetlinewidth{0.803000pt}%
\definecolor{currentstroke}{rgb}{0.690196,0.690196,0.690196}%
\pgfsetstrokecolor{currentstroke}%
\pgfsetstrokeopacity{0.050000}%
\pgfsetdash{}{0pt}%
\pgfpathmoveto{\pgfqpoint{7.012973in}{5.094806in}}%
\pgfpathlineto{\pgfqpoint{7.012973in}{9.318236in}}%
\pgfusepath{stroke}%
\end{pgfscope}%
\begin{pgfscope}%
\pgfsetbuttcap%
\pgfsetroundjoin%
\definecolor{currentfill}{rgb}{0.000000,0.000000,0.000000}%
\pgfsetfillcolor{currentfill}%
\pgfsetlinewidth{0.602250pt}%
\definecolor{currentstroke}{rgb}{0.000000,0.000000,0.000000}%
\pgfsetstrokecolor{currentstroke}%
\pgfsetdash{}{0pt}%
\pgfsys@defobject{currentmarker}{\pgfqpoint{0.000000in}{-0.027778in}}{\pgfqpoint{0.000000in}{0.000000in}}{%
\pgfpathmoveto{\pgfqpoint{0.000000in}{0.000000in}}%
\pgfpathlineto{\pgfqpoint{0.000000in}{-0.027778in}}%
\pgfusepath{stroke,fill}%
}%
\begin{pgfscope}%
\pgfsys@transformshift{7.012973in}{5.094806in}%
\pgfsys@useobject{currentmarker}{}%
\end{pgfscope}%
\end{pgfscope}%
\begin{pgfscope}%
\pgfpathrectangle{\pgfqpoint{0.688192in}{5.094806in}}{\pgfqpoint{11.096108in}{4.223431in}}%
\pgfusepath{clip}%
\pgfsetrectcap%
\pgfsetroundjoin%
\pgfsetlinewidth{0.803000pt}%
\definecolor{currentstroke}{rgb}{0.690196,0.690196,0.690196}%
\pgfsetstrokecolor{currentstroke}%
\pgfsetstrokeopacity{0.050000}%
\pgfsetdash{}{0pt}%
\pgfpathmoveto{\pgfqpoint{7.234895in}{5.094806in}}%
\pgfpathlineto{\pgfqpoint{7.234895in}{9.318236in}}%
\pgfusepath{stroke}%
\end{pgfscope}%
\begin{pgfscope}%
\pgfsetbuttcap%
\pgfsetroundjoin%
\definecolor{currentfill}{rgb}{0.000000,0.000000,0.000000}%
\pgfsetfillcolor{currentfill}%
\pgfsetlinewidth{0.602250pt}%
\definecolor{currentstroke}{rgb}{0.000000,0.000000,0.000000}%
\pgfsetstrokecolor{currentstroke}%
\pgfsetdash{}{0pt}%
\pgfsys@defobject{currentmarker}{\pgfqpoint{0.000000in}{-0.027778in}}{\pgfqpoint{0.000000in}{0.000000in}}{%
\pgfpathmoveto{\pgfqpoint{0.000000in}{0.000000in}}%
\pgfpathlineto{\pgfqpoint{0.000000in}{-0.027778in}}%
\pgfusepath{stroke,fill}%
}%
\begin{pgfscope}%
\pgfsys@transformshift{7.234895in}{5.094806in}%
\pgfsys@useobject{currentmarker}{}%
\end{pgfscope}%
\end{pgfscope}%
\begin{pgfscope}%
\pgfpathrectangle{\pgfqpoint{0.688192in}{5.094806in}}{\pgfqpoint{11.096108in}{4.223431in}}%
\pgfusepath{clip}%
\pgfsetrectcap%
\pgfsetroundjoin%
\pgfsetlinewidth{0.803000pt}%
\definecolor{currentstroke}{rgb}{0.690196,0.690196,0.690196}%
\pgfsetstrokecolor{currentstroke}%
\pgfsetstrokeopacity{0.050000}%
\pgfsetdash{}{0pt}%
\pgfpathmoveto{\pgfqpoint{7.456817in}{5.094806in}}%
\pgfpathlineto{\pgfqpoint{7.456817in}{9.318236in}}%
\pgfusepath{stroke}%
\end{pgfscope}%
\begin{pgfscope}%
\pgfsetbuttcap%
\pgfsetroundjoin%
\definecolor{currentfill}{rgb}{0.000000,0.000000,0.000000}%
\pgfsetfillcolor{currentfill}%
\pgfsetlinewidth{0.602250pt}%
\definecolor{currentstroke}{rgb}{0.000000,0.000000,0.000000}%
\pgfsetstrokecolor{currentstroke}%
\pgfsetdash{}{0pt}%
\pgfsys@defobject{currentmarker}{\pgfqpoint{0.000000in}{-0.027778in}}{\pgfqpoint{0.000000in}{0.000000in}}{%
\pgfpathmoveto{\pgfqpoint{0.000000in}{0.000000in}}%
\pgfpathlineto{\pgfqpoint{0.000000in}{-0.027778in}}%
\pgfusepath{stroke,fill}%
}%
\begin{pgfscope}%
\pgfsys@transformshift{7.456817in}{5.094806in}%
\pgfsys@useobject{currentmarker}{}%
\end{pgfscope}%
\end{pgfscope}%
\begin{pgfscope}%
\pgfpathrectangle{\pgfqpoint{0.688192in}{5.094806in}}{\pgfqpoint{11.096108in}{4.223431in}}%
\pgfusepath{clip}%
\pgfsetrectcap%
\pgfsetroundjoin%
\pgfsetlinewidth{0.803000pt}%
\definecolor{currentstroke}{rgb}{0.690196,0.690196,0.690196}%
\pgfsetstrokecolor{currentstroke}%
\pgfsetstrokeopacity{0.050000}%
\pgfsetdash{}{0pt}%
\pgfpathmoveto{\pgfqpoint{7.678740in}{5.094806in}}%
\pgfpathlineto{\pgfqpoint{7.678740in}{9.318236in}}%
\pgfusepath{stroke}%
\end{pgfscope}%
\begin{pgfscope}%
\pgfsetbuttcap%
\pgfsetroundjoin%
\definecolor{currentfill}{rgb}{0.000000,0.000000,0.000000}%
\pgfsetfillcolor{currentfill}%
\pgfsetlinewidth{0.602250pt}%
\definecolor{currentstroke}{rgb}{0.000000,0.000000,0.000000}%
\pgfsetstrokecolor{currentstroke}%
\pgfsetdash{}{0pt}%
\pgfsys@defobject{currentmarker}{\pgfqpoint{0.000000in}{-0.027778in}}{\pgfqpoint{0.000000in}{0.000000in}}{%
\pgfpathmoveto{\pgfqpoint{0.000000in}{0.000000in}}%
\pgfpathlineto{\pgfqpoint{0.000000in}{-0.027778in}}%
\pgfusepath{stroke,fill}%
}%
\begin{pgfscope}%
\pgfsys@transformshift{7.678740in}{5.094806in}%
\pgfsys@useobject{currentmarker}{}%
\end{pgfscope}%
\end{pgfscope}%
\begin{pgfscope}%
\pgfpathrectangle{\pgfqpoint{0.688192in}{5.094806in}}{\pgfqpoint{11.096108in}{4.223431in}}%
\pgfusepath{clip}%
\pgfsetrectcap%
\pgfsetroundjoin%
\pgfsetlinewidth{0.803000pt}%
\definecolor{currentstroke}{rgb}{0.690196,0.690196,0.690196}%
\pgfsetstrokecolor{currentstroke}%
\pgfsetstrokeopacity{0.050000}%
\pgfsetdash{}{0pt}%
\pgfpathmoveto{\pgfqpoint{8.122584in}{5.094806in}}%
\pgfpathlineto{\pgfqpoint{8.122584in}{9.318236in}}%
\pgfusepath{stroke}%
\end{pgfscope}%
\begin{pgfscope}%
\pgfsetbuttcap%
\pgfsetroundjoin%
\definecolor{currentfill}{rgb}{0.000000,0.000000,0.000000}%
\pgfsetfillcolor{currentfill}%
\pgfsetlinewidth{0.602250pt}%
\definecolor{currentstroke}{rgb}{0.000000,0.000000,0.000000}%
\pgfsetstrokecolor{currentstroke}%
\pgfsetdash{}{0pt}%
\pgfsys@defobject{currentmarker}{\pgfqpoint{0.000000in}{-0.027778in}}{\pgfqpoint{0.000000in}{0.000000in}}{%
\pgfpathmoveto{\pgfqpoint{0.000000in}{0.000000in}}%
\pgfpathlineto{\pgfqpoint{0.000000in}{-0.027778in}}%
\pgfusepath{stroke,fill}%
}%
\begin{pgfscope}%
\pgfsys@transformshift{8.122584in}{5.094806in}%
\pgfsys@useobject{currentmarker}{}%
\end{pgfscope}%
\end{pgfscope}%
\begin{pgfscope}%
\pgfpathrectangle{\pgfqpoint{0.688192in}{5.094806in}}{\pgfqpoint{11.096108in}{4.223431in}}%
\pgfusepath{clip}%
\pgfsetrectcap%
\pgfsetroundjoin%
\pgfsetlinewidth{0.803000pt}%
\definecolor{currentstroke}{rgb}{0.690196,0.690196,0.690196}%
\pgfsetstrokecolor{currentstroke}%
\pgfsetstrokeopacity{0.050000}%
\pgfsetdash{}{0pt}%
\pgfpathmoveto{\pgfqpoint{8.344506in}{5.094806in}}%
\pgfpathlineto{\pgfqpoint{8.344506in}{9.318236in}}%
\pgfusepath{stroke}%
\end{pgfscope}%
\begin{pgfscope}%
\pgfsetbuttcap%
\pgfsetroundjoin%
\definecolor{currentfill}{rgb}{0.000000,0.000000,0.000000}%
\pgfsetfillcolor{currentfill}%
\pgfsetlinewidth{0.602250pt}%
\definecolor{currentstroke}{rgb}{0.000000,0.000000,0.000000}%
\pgfsetstrokecolor{currentstroke}%
\pgfsetdash{}{0pt}%
\pgfsys@defobject{currentmarker}{\pgfqpoint{0.000000in}{-0.027778in}}{\pgfqpoint{0.000000in}{0.000000in}}{%
\pgfpathmoveto{\pgfqpoint{0.000000in}{0.000000in}}%
\pgfpathlineto{\pgfqpoint{0.000000in}{-0.027778in}}%
\pgfusepath{stroke,fill}%
}%
\begin{pgfscope}%
\pgfsys@transformshift{8.344506in}{5.094806in}%
\pgfsys@useobject{currentmarker}{}%
\end{pgfscope}%
\end{pgfscope}%
\begin{pgfscope}%
\pgfpathrectangle{\pgfqpoint{0.688192in}{5.094806in}}{\pgfqpoint{11.096108in}{4.223431in}}%
\pgfusepath{clip}%
\pgfsetrectcap%
\pgfsetroundjoin%
\pgfsetlinewidth{0.803000pt}%
\definecolor{currentstroke}{rgb}{0.690196,0.690196,0.690196}%
\pgfsetstrokecolor{currentstroke}%
\pgfsetstrokeopacity{0.050000}%
\pgfsetdash{}{0pt}%
\pgfpathmoveto{\pgfqpoint{8.566428in}{5.094806in}}%
\pgfpathlineto{\pgfqpoint{8.566428in}{9.318236in}}%
\pgfusepath{stroke}%
\end{pgfscope}%
\begin{pgfscope}%
\pgfsetbuttcap%
\pgfsetroundjoin%
\definecolor{currentfill}{rgb}{0.000000,0.000000,0.000000}%
\pgfsetfillcolor{currentfill}%
\pgfsetlinewidth{0.602250pt}%
\definecolor{currentstroke}{rgb}{0.000000,0.000000,0.000000}%
\pgfsetstrokecolor{currentstroke}%
\pgfsetdash{}{0pt}%
\pgfsys@defobject{currentmarker}{\pgfqpoint{0.000000in}{-0.027778in}}{\pgfqpoint{0.000000in}{0.000000in}}{%
\pgfpathmoveto{\pgfqpoint{0.000000in}{0.000000in}}%
\pgfpathlineto{\pgfqpoint{0.000000in}{-0.027778in}}%
\pgfusepath{stroke,fill}%
}%
\begin{pgfscope}%
\pgfsys@transformshift{8.566428in}{5.094806in}%
\pgfsys@useobject{currentmarker}{}%
\end{pgfscope}%
\end{pgfscope}%
\begin{pgfscope}%
\pgfpathrectangle{\pgfqpoint{0.688192in}{5.094806in}}{\pgfqpoint{11.096108in}{4.223431in}}%
\pgfusepath{clip}%
\pgfsetrectcap%
\pgfsetroundjoin%
\pgfsetlinewidth{0.803000pt}%
\definecolor{currentstroke}{rgb}{0.690196,0.690196,0.690196}%
\pgfsetstrokecolor{currentstroke}%
\pgfsetstrokeopacity{0.050000}%
\pgfsetdash{}{0pt}%
\pgfpathmoveto{\pgfqpoint{8.788350in}{5.094806in}}%
\pgfpathlineto{\pgfqpoint{8.788350in}{9.318236in}}%
\pgfusepath{stroke}%
\end{pgfscope}%
\begin{pgfscope}%
\pgfsetbuttcap%
\pgfsetroundjoin%
\definecolor{currentfill}{rgb}{0.000000,0.000000,0.000000}%
\pgfsetfillcolor{currentfill}%
\pgfsetlinewidth{0.602250pt}%
\definecolor{currentstroke}{rgb}{0.000000,0.000000,0.000000}%
\pgfsetstrokecolor{currentstroke}%
\pgfsetdash{}{0pt}%
\pgfsys@defobject{currentmarker}{\pgfqpoint{0.000000in}{-0.027778in}}{\pgfqpoint{0.000000in}{0.000000in}}{%
\pgfpathmoveto{\pgfqpoint{0.000000in}{0.000000in}}%
\pgfpathlineto{\pgfqpoint{0.000000in}{-0.027778in}}%
\pgfusepath{stroke,fill}%
}%
\begin{pgfscope}%
\pgfsys@transformshift{8.788350in}{5.094806in}%
\pgfsys@useobject{currentmarker}{}%
\end{pgfscope}%
\end{pgfscope}%
\begin{pgfscope}%
\pgfpathrectangle{\pgfqpoint{0.688192in}{5.094806in}}{\pgfqpoint{11.096108in}{4.223431in}}%
\pgfusepath{clip}%
\pgfsetrectcap%
\pgfsetroundjoin%
\pgfsetlinewidth{0.803000pt}%
\definecolor{currentstroke}{rgb}{0.690196,0.690196,0.690196}%
\pgfsetstrokecolor{currentstroke}%
\pgfsetstrokeopacity{0.050000}%
\pgfsetdash{}{0pt}%
\pgfpathmoveto{\pgfqpoint{9.232195in}{5.094806in}}%
\pgfpathlineto{\pgfqpoint{9.232195in}{9.318236in}}%
\pgfusepath{stroke}%
\end{pgfscope}%
\begin{pgfscope}%
\pgfsetbuttcap%
\pgfsetroundjoin%
\definecolor{currentfill}{rgb}{0.000000,0.000000,0.000000}%
\pgfsetfillcolor{currentfill}%
\pgfsetlinewidth{0.602250pt}%
\definecolor{currentstroke}{rgb}{0.000000,0.000000,0.000000}%
\pgfsetstrokecolor{currentstroke}%
\pgfsetdash{}{0pt}%
\pgfsys@defobject{currentmarker}{\pgfqpoint{0.000000in}{-0.027778in}}{\pgfqpoint{0.000000in}{0.000000in}}{%
\pgfpathmoveto{\pgfqpoint{0.000000in}{0.000000in}}%
\pgfpathlineto{\pgfqpoint{0.000000in}{-0.027778in}}%
\pgfusepath{stroke,fill}%
}%
\begin{pgfscope}%
\pgfsys@transformshift{9.232195in}{5.094806in}%
\pgfsys@useobject{currentmarker}{}%
\end{pgfscope}%
\end{pgfscope}%
\begin{pgfscope}%
\pgfpathrectangle{\pgfqpoint{0.688192in}{5.094806in}}{\pgfqpoint{11.096108in}{4.223431in}}%
\pgfusepath{clip}%
\pgfsetrectcap%
\pgfsetroundjoin%
\pgfsetlinewidth{0.803000pt}%
\definecolor{currentstroke}{rgb}{0.690196,0.690196,0.690196}%
\pgfsetstrokecolor{currentstroke}%
\pgfsetstrokeopacity{0.050000}%
\pgfsetdash{}{0pt}%
\pgfpathmoveto{\pgfqpoint{9.454117in}{5.094806in}}%
\pgfpathlineto{\pgfqpoint{9.454117in}{9.318236in}}%
\pgfusepath{stroke}%
\end{pgfscope}%
\begin{pgfscope}%
\pgfsetbuttcap%
\pgfsetroundjoin%
\definecolor{currentfill}{rgb}{0.000000,0.000000,0.000000}%
\pgfsetfillcolor{currentfill}%
\pgfsetlinewidth{0.602250pt}%
\definecolor{currentstroke}{rgb}{0.000000,0.000000,0.000000}%
\pgfsetstrokecolor{currentstroke}%
\pgfsetdash{}{0pt}%
\pgfsys@defobject{currentmarker}{\pgfqpoint{0.000000in}{-0.027778in}}{\pgfqpoint{0.000000in}{0.000000in}}{%
\pgfpathmoveto{\pgfqpoint{0.000000in}{0.000000in}}%
\pgfpathlineto{\pgfqpoint{0.000000in}{-0.027778in}}%
\pgfusepath{stroke,fill}%
}%
\begin{pgfscope}%
\pgfsys@transformshift{9.454117in}{5.094806in}%
\pgfsys@useobject{currentmarker}{}%
\end{pgfscope}%
\end{pgfscope}%
\begin{pgfscope}%
\pgfpathrectangle{\pgfqpoint{0.688192in}{5.094806in}}{\pgfqpoint{11.096108in}{4.223431in}}%
\pgfusepath{clip}%
\pgfsetrectcap%
\pgfsetroundjoin%
\pgfsetlinewidth{0.803000pt}%
\definecolor{currentstroke}{rgb}{0.690196,0.690196,0.690196}%
\pgfsetstrokecolor{currentstroke}%
\pgfsetstrokeopacity{0.050000}%
\pgfsetdash{}{0pt}%
\pgfpathmoveto{\pgfqpoint{9.676039in}{5.094806in}}%
\pgfpathlineto{\pgfqpoint{9.676039in}{9.318236in}}%
\pgfusepath{stroke}%
\end{pgfscope}%
\begin{pgfscope}%
\pgfsetbuttcap%
\pgfsetroundjoin%
\definecolor{currentfill}{rgb}{0.000000,0.000000,0.000000}%
\pgfsetfillcolor{currentfill}%
\pgfsetlinewidth{0.602250pt}%
\definecolor{currentstroke}{rgb}{0.000000,0.000000,0.000000}%
\pgfsetstrokecolor{currentstroke}%
\pgfsetdash{}{0pt}%
\pgfsys@defobject{currentmarker}{\pgfqpoint{0.000000in}{-0.027778in}}{\pgfqpoint{0.000000in}{0.000000in}}{%
\pgfpathmoveto{\pgfqpoint{0.000000in}{0.000000in}}%
\pgfpathlineto{\pgfqpoint{0.000000in}{-0.027778in}}%
\pgfusepath{stroke,fill}%
}%
\begin{pgfscope}%
\pgfsys@transformshift{9.676039in}{5.094806in}%
\pgfsys@useobject{currentmarker}{}%
\end{pgfscope}%
\end{pgfscope}%
\begin{pgfscope}%
\pgfpathrectangle{\pgfqpoint{0.688192in}{5.094806in}}{\pgfqpoint{11.096108in}{4.223431in}}%
\pgfusepath{clip}%
\pgfsetrectcap%
\pgfsetroundjoin%
\pgfsetlinewidth{0.803000pt}%
\definecolor{currentstroke}{rgb}{0.690196,0.690196,0.690196}%
\pgfsetstrokecolor{currentstroke}%
\pgfsetstrokeopacity{0.050000}%
\pgfsetdash{}{0pt}%
\pgfpathmoveto{\pgfqpoint{9.897961in}{5.094806in}}%
\pgfpathlineto{\pgfqpoint{9.897961in}{9.318236in}}%
\pgfusepath{stroke}%
\end{pgfscope}%
\begin{pgfscope}%
\pgfsetbuttcap%
\pgfsetroundjoin%
\definecolor{currentfill}{rgb}{0.000000,0.000000,0.000000}%
\pgfsetfillcolor{currentfill}%
\pgfsetlinewidth{0.602250pt}%
\definecolor{currentstroke}{rgb}{0.000000,0.000000,0.000000}%
\pgfsetstrokecolor{currentstroke}%
\pgfsetdash{}{0pt}%
\pgfsys@defobject{currentmarker}{\pgfqpoint{0.000000in}{-0.027778in}}{\pgfqpoint{0.000000in}{0.000000in}}{%
\pgfpathmoveto{\pgfqpoint{0.000000in}{0.000000in}}%
\pgfpathlineto{\pgfqpoint{0.000000in}{-0.027778in}}%
\pgfusepath{stroke,fill}%
}%
\begin{pgfscope}%
\pgfsys@transformshift{9.897961in}{5.094806in}%
\pgfsys@useobject{currentmarker}{}%
\end{pgfscope}%
\end{pgfscope}%
\begin{pgfscope}%
\pgfpathrectangle{\pgfqpoint{0.688192in}{5.094806in}}{\pgfqpoint{11.096108in}{4.223431in}}%
\pgfusepath{clip}%
\pgfsetrectcap%
\pgfsetroundjoin%
\pgfsetlinewidth{0.803000pt}%
\definecolor{currentstroke}{rgb}{0.690196,0.690196,0.690196}%
\pgfsetstrokecolor{currentstroke}%
\pgfsetstrokeopacity{0.050000}%
\pgfsetdash{}{0pt}%
\pgfpathmoveto{\pgfqpoint{10.341805in}{5.094806in}}%
\pgfpathlineto{\pgfqpoint{10.341805in}{9.318236in}}%
\pgfusepath{stroke}%
\end{pgfscope}%
\begin{pgfscope}%
\pgfsetbuttcap%
\pgfsetroundjoin%
\definecolor{currentfill}{rgb}{0.000000,0.000000,0.000000}%
\pgfsetfillcolor{currentfill}%
\pgfsetlinewidth{0.602250pt}%
\definecolor{currentstroke}{rgb}{0.000000,0.000000,0.000000}%
\pgfsetstrokecolor{currentstroke}%
\pgfsetdash{}{0pt}%
\pgfsys@defobject{currentmarker}{\pgfqpoint{0.000000in}{-0.027778in}}{\pgfqpoint{0.000000in}{0.000000in}}{%
\pgfpathmoveto{\pgfqpoint{0.000000in}{0.000000in}}%
\pgfpathlineto{\pgfqpoint{0.000000in}{-0.027778in}}%
\pgfusepath{stroke,fill}%
}%
\begin{pgfscope}%
\pgfsys@transformshift{10.341805in}{5.094806in}%
\pgfsys@useobject{currentmarker}{}%
\end{pgfscope}%
\end{pgfscope}%
\begin{pgfscope}%
\pgfpathrectangle{\pgfqpoint{0.688192in}{5.094806in}}{\pgfqpoint{11.096108in}{4.223431in}}%
\pgfusepath{clip}%
\pgfsetrectcap%
\pgfsetroundjoin%
\pgfsetlinewidth{0.803000pt}%
\definecolor{currentstroke}{rgb}{0.690196,0.690196,0.690196}%
\pgfsetstrokecolor{currentstroke}%
\pgfsetstrokeopacity{0.050000}%
\pgfsetdash{}{0pt}%
\pgfpathmoveto{\pgfqpoint{10.563728in}{5.094806in}}%
\pgfpathlineto{\pgfqpoint{10.563728in}{9.318236in}}%
\pgfusepath{stroke}%
\end{pgfscope}%
\begin{pgfscope}%
\pgfsetbuttcap%
\pgfsetroundjoin%
\definecolor{currentfill}{rgb}{0.000000,0.000000,0.000000}%
\pgfsetfillcolor{currentfill}%
\pgfsetlinewidth{0.602250pt}%
\definecolor{currentstroke}{rgb}{0.000000,0.000000,0.000000}%
\pgfsetstrokecolor{currentstroke}%
\pgfsetdash{}{0pt}%
\pgfsys@defobject{currentmarker}{\pgfqpoint{0.000000in}{-0.027778in}}{\pgfqpoint{0.000000in}{0.000000in}}{%
\pgfpathmoveto{\pgfqpoint{0.000000in}{0.000000in}}%
\pgfpathlineto{\pgfqpoint{0.000000in}{-0.027778in}}%
\pgfusepath{stroke,fill}%
}%
\begin{pgfscope}%
\pgfsys@transformshift{10.563728in}{5.094806in}%
\pgfsys@useobject{currentmarker}{}%
\end{pgfscope}%
\end{pgfscope}%
\begin{pgfscope}%
\pgfpathrectangle{\pgfqpoint{0.688192in}{5.094806in}}{\pgfqpoint{11.096108in}{4.223431in}}%
\pgfusepath{clip}%
\pgfsetrectcap%
\pgfsetroundjoin%
\pgfsetlinewidth{0.803000pt}%
\definecolor{currentstroke}{rgb}{0.690196,0.690196,0.690196}%
\pgfsetstrokecolor{currentstroke}%
\pgfsetstrokeopacity{0.050000}%
\pgfsetdash{}{0pt}%
\pgfpathmoveto{\pgfqpoint{10.785650in}{5.094806in}}%
\pgfpathlineto{\pgfqpoint{10.785650in}{9.318236in}}%
\pgfusepath{stroke}%
\end{pgfscope}%
\begin{pgfscope}%
\pgfsetbuttcap%
\pgfsetroundjoin%
\definecolor{currentfill}{rgb}{0.000000,0.000000,0.000000}%
\pgfsetfillcolor{currentfill}%
\pgfsetlinewidth{0.602250pt}%
\definecolor{currentstroke}{rgb}{0.000000,0.000000,0.000000}%
\pgfsetstrokecolor{currentstroke}%
\pgfsetdash{}{0pt}%
\pgfsys@defobject{currentmarker}{\pgfqpoint{0.000000in}{-0.027778in}}{\pgfqpoint{0.000000in}{0.000000in}}{%
\pgfpathmoveto{\pgfqpoint{0.000000in}{0.000000in}}%
\pgfpathlineto{\pgfqpoint{0.000000in}{-0.027778in}}%
\pgfusepath{stroke,fill}%
}%
\begin{pgfscope}%
\pgfsys@transformshift{10.785650in}{5.094806in}%
\pgfsys@useobject{currentmarker}{}%
\end{pgfscope}%
\end{pgfscope}%
\begin{pgfscope}%
\pgfpathrectangle{\pgfqpoint{0.688192in}{5.094806in}}{\pgfqpoint{11.096108in}{4.223431in}}%
\pgfusepath{clip}%
\pgfsetrectcap%
\pgfsetroundjoin%
\pgfsetlinewidth{0.803000pt}%
\definecolor{currentstroke}{rgb}{0.690196,0.690196,0.690196}%
\pgfsetstrokecolor{currentstroke}%
\pgfsetstrokeopacity{0.050000}%
\pgfsetdash{}{0pt}%
\pgfpathmoveto{\pgfqpoint{11.007572in}{5.094806in}}%
\pgfpathlineto{\pgfqpoint{11.007572in}{9.318236in}}%
\pgfusepath{stroke}%
\end{pgfscope}%
\begin{pgfscope}%
\pgfsetbuttcap%
\pgfsetroundjoin%
\definecolor{currentfill}{rgb}{0.000000,0.000000,0.000000}%
\pgfsetfillcolor{currentfill}%
\pgfsetlinewidth{0.602250pt}%
\definecolor{currentstroke}{rgb}{0.000000,0.000000,0.000000}%
\pgfsetstrokecolor{currentstroke}%
\pgfsetdash{}{0pt}%
\pgfsys@defobject{currentmarker}{\pgfqpoint{0.000000in}{-0.027778in}}{\pgfqpoint{0.000000in}{0.000000in}}{%
\pgfpathmoveto{\pgfqpoint{0.000000in}{0.000000in}}%
\pgfpathlineto{\pgfqpoint{0.000000in}{-0.027778in}}%
\pgfusepath{stroke,fill}%
}%
\begin{pgfscope}%
\pgfsys@transformshift{11.007572in}{5.094806in}%
\pgfsys@useobject{currentmarker}{}%
\end{pgfscope}%
\end{pgfscope}%
\begin{pgfscope}%
\pgfpathrectangle{\pgfqpoint{0.688192in}{5.094806in}}{\pgfqpoint{11.096108in}{4.223431in}}%
\pgfusepath{clip}%
\pgfsetrectcap%
\pgfsetroundjoin%
\pgfsetlinewidth{0.803000pt}%
\definecolor{currentstroke}{rgb}{0.690196,0.690196,0.690196}%
\pgfsetstrokecolor{currentstroke}%
\pgfsetstrokeopacity{0.050000}%
\pgfsetdash{}{0pt}%
\pgfpathmoveto{\pgfqpoint{11.451416in}{5.094806in}}%
\pgfpathlineto{\pgfqpoint{11.451416in}{9.318236in}}%
\pgfusepath{stroke}%
\end{pgfscope}%
\begin{pgfscope}%
\pgfsetbuttcap%
\pgfsetroundjoin%
\definecolor{currentfill}{rgb}{0.000000,0.000000,0.000000}%
\pgfsetfillcolor{currentfill}%
\pgfsetlinewidth{0.602250pt}%
\definecolor{currentstroke}{rgb}{0.000000,0.000000,0.000000}%
\pgfsetstrokecolor{currentstroke}%
\pgfsetdash{}{0pt}%
\pgfsys@defobject{currentmarker}{\pgfqpoint{0.000000in}{-0.027778in}}{\pgfqpoint{0.000000in}{0.000000in}}{%
\pgfpathmoveto{\pgfqpoint{0.000000in}{0.000000in}}%
\pgfpathlineto{\pgfqpoint{0.000000in}{-0.027778in}}%
\pgfusepath{stroke,fill}%
}%
\begin{pgfscope}%
\pgfsys@transformshift{11.451416in}{5.094806in}%
\pgfsys@useobject{currentmarker}{}%
\end{pgfscope}%
\end{pgfscope}%
\begin{pgfscope}%
\pgfpathrectangle{\pgfqpoint{0.688192in}{5.094806in}}{\pgfqpoint{11.096108in}{4.223431in}}%
\pgfusepath{clip}%
\pgfsetrectcap%
\pgfsetroundjoin%
\pgfsetlinewidth{0.803000pt}%
\definecolor{currentstroke}{rgb}{0.690196,0.690196,0.690196}%
\pgfsetstrokecolor{currentstroke}%
\pgfsetstrokeopacity{0.050000}%
\pgfsetdash{}{0pt}%
\pgfpathmoveto{\pgfqpoint{11.673338in}{5.094806in}}%
\pgfpathlineto{\pgfqpoint{11.673338in}{9.318236in}}%
\pgfusepath{stroke}%
\end{pgfscope}%
\begin{pgfscope}%
\pgfsetbuttcap%
\pgfsetroundjoin%
\definecolor{currentfill}{rgb}{0.000000,0.000000,0.000000}%
\pgfsetfillcolor{currentfill}%
\pgfsetlinewidth{0.602250pt}%
\definecolor{currentstroke}{rgb}{0.000000,0.000000,0.000000}%
\pgfsetstrokecolor{currentstroke}%
\pgfsetdash{}{0pt}%
\pgfsys@defobject{currentmarker}{\pgfqpoint{0.000000in}{-0.027778in}}{\pgfqpoint{0.000000in}{0.000000in}}{%
\pgfpathmoveto{\pgfqpoint{0.000000in}{0.000000in}}%
\pgfpathlineto{\pgfqpoint{0.000000in}{-0.027778in}}%
\pgfusepath{stroke,fill}%
}%
\begin{pgfscope}%
\pgfsys@transformshift{11.673338in}{5.094806in}%
\pgfsys@useobject{currentmarker}{}%
\end{pgfscope}%
\end{pgfscope}%
\begin{pgfscope}%
\pgfpathrectangle{\pgfqpoint{0.688192in}{5.094806in}}{\pgfqpoint{11.096108in}{4.223431in}}%
\pgfusepath{clip}%
\pgfsetrectcap%
\pgfsetroundjoin%
\pgfsetlinewidth{0.803000pt}%
\definecolor{currentstroke}{rgb}{0.690196,0.690196,0.690196}%
\pgfsetstrokecolor{currentstroke}%
\pgfsetstrokeopacity{0.200000}%
\pgfsetdash{}{0pt}%
\pgfpathmoveto{\pgfqpoint{0.688192in}{5.094806in}}%
\pgfpathlineto{\pgfqpoint{11.784299in}{5.094806in}}%
\pgfusepath{stroke}%
\end{pgfscope}%
\begin{pgfscope}%
\pgfsetbuttcap%
\pgfsetroundjoin%
\definecolor{currentfill}{rgb}{0.000000,0.000000,0.000000}%
\pgfsetfillcolor{currentfill}%
\pgfsetlinewidth{0.803000pt}%
\definecolor{currentstroke}{rgb}{0.000000,0.000000,0.000000}%
\pgfsetstrokecolor{currentstroke}%
\pgfsetdash{}{0pt}%
\pgfsys@defobject{currentmarker}{\pgfqpoint{-0.048611in}{0.000000in}}{\pgfqpoint{-0.000000in}{0.000000in}}{%
\pgfpathmoveto{\pgfqpoint{-0.000000in}{0.000000in}}%
\pgfpathlineto{\pgfqpoint{-0.048611in}{0.000000in}}%
\pgfusepath{stroke,fill}%
}%
\begin{pgfscope}%
\pgfsys@transformshift{0.688192in}{5.094806in}%
\pgfsys@useobject{currentmarker}{}%
\end{pgfscope}%
\end{pgfscope}%
\begin{pgfscope}%
\definecolor{textcolor}{rgb}{0.000000,0.000000,0.000000}%
\pgfsetstrokecolor{textcolor}%
\pgfsetfillcolor{textcolor}%
\pgftext[x=0.493054in, y=5.025361in, left, base]{\color{textcolor}{\rmfamily\fontsize{14.000000}{16.800000}\selectfont\catcode`\^=\active\def^{\ifmmode\sp\else\^{}\fi}\catcode`\%=\active\def%{\%}$\mathdefault{0}$}}%
\end{pgfscope}%
\begin{pgfscope}%
\pgfpathrectangle{\pgfqpoint{0.688192in}{5.094806in}}{\pgfqpoint{11.096108in}{4.223431in}}%
\pgfusepath{clip}%
\pgfsetrectcap%
\pgfsetroundjoin%
\pgfsetlinewidth{0.803000pt}%
\definecolor{currentstroke}{rgb}{0.690196,0.690196,0.690196}%
\pgfsetstrokecolor{currentstroke}%
\pgfsetstrokeopacity{0.200000}%
\pgfsetdash{}{0pt}%
\pgfpathmoveto{\pgfqpoint{0.688192in}{5.876922in}}%
\pgfpathlineto{\pgfqpoint{11.784299in}{5.876922in}}%
\pgfusepath{stroke}%
\end{pgfscope}%
\begin{pgfscope}%
\pgfsetbuttcap%
\pgfsetroundjoin%
\definecolor{currentfill}{rgb}{0.000000,0.000000,0.000000}%
\pgfsetfillcolor{currentfill}%
\pgfsetlinewidth{0.803000pt}%
\definecolor{currentstroke}{rgb}{0.000000,0.000000,0.000000}%
\pgfsetstrokecolor{currentstroke}%
\pgfsetdash{}{0pt}%
\pgfsys@defobject{currentmarker}{\pgfqpoint{-0.048611in}{0.000000in}}{\pgfqpoint{-0.000000in}{0.000000in}}{%
\pgfpathmoveto{\pgfqpoint{-0.000000in}{0.000000in}}%
\pgfpathlineto{\pgfqpoint{-0.048611in}{0.000000in}}%
\pgfusepath{stroke,fill}%
}%
\begin{pgfscope}%
\pgfsys@transformshift{0.688192in}{5.876922in}%
\pgfsys@useobject{currentmarker}{}%
\end{pgfscope}%
\end{pgfscope}%
\begin{pgfscope}%
\definecolor{textcolor}{rgb}{0.000000,0.000000,0.000000}%
\pgfsetstrokecolor{textcolor}%
\pgfsetfillcolor{textcolor}%
\pgftext[x=0.493054in, y=5.807478in, left, base]{\color{textcolor}{\rmfamily\fontsize{14.000000}{16.800000}\selectfont\catcode`\^=\active\def^{\ifmmode\sp\else\^{}\fi}\catcode`\%=\active\def%{\%}$\mathdefault{5}$}}%
\end{pgfscope}%
\begin{pgfscope}%
\pgfpathrectangle{\pgfqpoint{0.688192in}{5.094806in}}{\pgfqpoint{11.096108in}{4.223431in}}%
\pgfusepath{clip}%
\pgfsetrectcap%
\pgfsetroundjoin%
\pgfsetlinewidth{0.803000pt}%
\definecolor{currentstroke}{rgb}{0.690196,0.690196,0.690196}%
\pgfsetstrokecolor{currentstroke}%
\pgfsetstrokeopacity{0.200000}%
\pgfsetdash{}{0pt}%
\pgfpathmoveto{\pgfqpoint{0.688192in}{6.659039in}}%
\pgfpathlineto{\pgfqpoint{11.784299in}{6.659039in}}%
\pgfusepath{stroke}%
\end{pgfscope}%
\begin{pgfscope}%
\pgfsetbuttcap%
\pgfsetroundjoin%
\definecolor{currentfill}{rgb}{0.000000,0.000000,0.000000}%
\pgfsetfillcolor{currentfill}%
\pgfsetlinewidth{0.803000pt}%
\definecolor{currentstroke}{rgb}{0.000000,0.000000,0.000000}%
\pgfsetstrokecolor{currentstroke}%
\pgfsetdash{}{0pt}%
\pgfsys@defobject{currentmarker}{\pgfqpoint{-0.048611in}{0.000000in}}{\pgfqpoint{-0.000000in}{0.000000in}}{%
\pgfpathmoveto{\pgfqpoint{-0.000000in}{0.000000in}}%
\pgfpathlineto{\pgfqpoint{-0.048611in}{0.000000in}}%
\pgfusepath{stroke,fill}%
}%
\begin{pgfscope}%
\pgfsys@transformshift{0.688192in}{6.659039in}%
\pgfsys@useobject{currentmarker}{}%
\end{pgfscope}%
\end{pgfscope}%
\begin{pgfscope}%
\definecolor{textcolor}{rgb}{0.000000,0.000000,0.000000}%
\pgfsetstrokecolor{textcolor}%
\pgfsetfillcolor{textcolor}%
\pgftext[x=0.395138in, y=6.589595in, left, base]{\color{textcolor}{\rmfamily\fontsize{14.000000}{16.800000}\selectfont\catcode`\^=\active\def^{\ifmmode\sp\else\^{}\fi}\catcode`\%=\active\def%{\%}$\mathdefault{10}$}}%
\end{pgfscope}%
\begin{pgfscope}%
\pgfpathrectangle{\pgfqpoint{0.688192in}{5.094806in}}{\pgfqpoint{11.096108in}{4.223431in}}%
\pgfusepath{clip}%
\pgfsetrectcap%
\pgfsetroundjoin%
\pgfsetlinewidth{0.803000pt}%
\definecolor{currentstroke}{rgb}{0.690196,0.690196,0.690196}%
\pgfsetstrokecolor{currentstroke}%
\pgfsetstrokeopacity{0.200000}%
\pgfsetdash{}{0pt}%
\pgfpathmoveto{\pgfqpoint{0.688192in}{7.441156in}}%
\pgfpathlineto{\pgfqpoint{11.784299in}{7.441156in}}%
\pgfusepath{stroke}%
\end{pgfscope}%
\begin{pgfscope}%
\pgfsetbuttcap%
\pgfsetroundjoin%
\definecolor{currentfill}{rgb}{0.000000,0.000000,0.000000}%
\pgfsetfillcolor{currentfill}%
\pgfsetlinewidth{0.803000pt}%
\definecolor{currentstroke}{rgb}{0.000000,0.000000,0.000000}%
\pgfsetstrokecolor{currentstroke}%
\pgfsetdash{}{0pt}%
\pgfsys@defobject{currentmarker}{\pgfqpoint{-0.048611in}{0.000000in}}{\pgfqpoint{-0.000000in}{0.000000in}}{%
\pgfpathmoveto{\pgfqpoint{-0.000000in}{0.000000in}}%
\pgfpathlineto{\pgfqpoint{-0.048611in}{0.000000in}}%
\pgfusepath{stroke,fill}%
}%
\begin{pgfscope}%
\pgfsys@transformshift{0.688192in}{7.441156in}%
\pgfsys@useobject{currentmarker}{}%
\end{pgfscope}%
\end{pgfscope}%
\begin{pgfscope}%
\definecolor{textcolor}{rgb}{0.000000,0.000000,0.000000}%
\pgfsetstrokecolor{textcolor}%
\pgfsetfillcolor{textcolor}%
\pgftext[x=0.395138in, y=7.371712in, left, base]{\color{textcolor}{\rmfamily\fontsize{14.000000}{16.800000}\selectfont\catcode`\^=\active\def^{\ifmmode\sp\else\^{}\fi}\catcode`\%=\active\def%{\%}$\mathdefault{15}$}}%
\end{pgfscope}%
\begin{pgfscope}%
\pgfpathrectangle{\pgfqpoint{0.688192in}{5.094806in}}{\pgfqpoint{11.096108in}{4.223431in}}%
\pgfusepath{clip}%
\pgfsetrectcap%
\pgfsetroundjoin%
\pgfsetlinewidth{0.803000pt}%
\definecolor{currentstroke}{rgb}{0.690196,0.690196,0.690196}%
\pgfsetstrokecolor{currentstroke}%
\pgfsetstrokeopacity{0.200000}%
\pgfsetdash{}{0pt}%
\pgfpathmoveto{\pgfqpoint{0.688192in}{8.223273in}}%
\pgfpathlineto{\pgfqpoint{11.784299in}{8.223273in}}%
\pgfusepath{stroke}%
\end{pgfscope}%
\begin{pgfscope}%
\pgfsetbuttcap%
\pgfsetroundjoin%
\definecolor{currentfill}{rgb}{0.000000,0.000000,0.000000}%
\pgfsetfillcolor{currentfill}%
\pgfsetlinewidth{0.803000pt}%
\definecolor{currentstroke}{rgb}{0.000000,0.000000,0.000000}%
\pgfsetstrokecolor{currentstroke}%
\pgfsetdash{}{0pt}%
\pgfsys@defobject{currentmarker}{\pgfqpoint{-0.048611in}{0.000000in}}{\pgfqpoint{-0.000000in}{0.000000in}}{%
\pgfpathmoveto{\pgfqpoint{-0.000000in}{0.000000in}}%
\pgfpathlineto{\pgfqpoint{-0.048611in}{0.000000in}}%
\pgfusepath{stroke,fill}%
}%
\begin{pgfscope}%
\pgfsys@transformshift{0.688192in}{8.223273in}%
\pgfsys@useobject{currentmarker}{}%
\end{pgfscope}%
\end{pgfscope}%
\begin{pgfscope}%
\definecolor{textcolor}{rgb}{0.000000,0.000000,0.000000}%
\pgfsetstrokecolor{textcolor}%
\pgfsetfillcolor{textcolor}%
\pgftext[x=0.395138in, y=8.153828in, left, base]{\color{textcolor}{\rmfamily\fontsize{14.000000}{16.800000}\selectfont\catcode`\^=\active\def^{\ifmmode\sp\else\^{}\fi}\catcode`\%=\active\def%{\%}$\mathdefault{20}$}}%
\end{pgfscope}%
\begin{pgfscope}%
\pgfpathrectangle{\pgfqpoint{0.688192in}{5.094806in}}{\pgfqpoint{11.096108in}{4.223431in}}%
\pgfusepath{clip}%
\pgfsetrectcap%
\pgfsetroundjoin%
\pgfsetlinewidth{0.803000pt}%
\definecolor{currentstroke}{rgb}{0.690196,0.690196,0.690196}%
\pgfsetstrokecolor{currentstroke}%
\pgfsetstrokeopacity{0.200000}%
\pgfsetdash{}{0pt}%
\pgfpathmoveto{\pgfqpoint{0.688192in}{9.005390in}}%
\pgfpathlineto{\pgfqpoint{11.784299in}{9.005390in}}%
\pgfusepath{stroke}%
\end{pgfscope}%
\begin{pgfscope}%
\pgfsetbuttcap%
\pgfsetroundjoin%
\definecolor{currentfill}{rgb}{0.000000,0.000000,0.000000}%
\pgfsetfillcolor{currentfill}%
\pgfsetlinewidth{0.803000pt}%
\definecolor{currentstroke}{rgb}{0.000000,0.000000,0.000000}%
\pgfsetstrokecolor{currentstroke}%
\pgfsetdash{}{0pt}%
\pgfsys@defobject{currentmarker}{\pgfqpoint{-0.048611in}{0.000000in}}{\pgfqpoint{-0.000000in}{0.000000in}}{%
\pgfpathmoveto{\pgfqpoint{-0.000000in}{0.000000in}}%
\pgfpathlineto{\pgfqpoint{-0.048611in}{0.000000in}}%
\pgfusepath{stroke,fill}%
}%
\begin{pgfscope}%
\pgfsys@transformshift{0.688192in}{9.005390in}%
\pgfsys@useobject{currentmarker}{}%
\end{pgfscope}%
\end{pgfscope}%
\begin{pgfscope}%
\definecolor{textcolor}{rgb}{0.000000,0.000000,0.000000}%
\pgfsetstrokecolor{textcolor}%
\pgfsetfillcolor{textcolor}%
\pgftext[x=0.395138in, y=8.935945in, left, base]{\color{textcolor}{\rmfamily\fontsize{14.000000}{16.800000}\selectfont\catcode`\^=\active\def^{\ifmmode\sp\else\^{}\fi}\catcode`\%=\active\def%{\%}$\mathdefault{25}$}}%
\end{pgfscope}%
\begin{pgfscope}%
\pgfpathrectangle{\pgfqpoint{0.688192in}{5.094806in}}{\pgfqpoint{11.096108in}{4.223431in}}%
\pgfusepath{clip}%
\pgfsetrectcap%
\pgfsetroundjoin%
\pgfsetlinewidth{0.803000pt}%
\definecolor{currentstroke}{rgb}{0.690196,0.690196,0.690196}%
\pgfsetstrokecolor{currentstroke}%
\pgfsetstrokeopacity{0.050000}%
\pgfsetdash{}{0pt}%
\pgfpathmoveto{\pgfqpoint{0.688192in}{5.251229in}}%
\pgfpathlineto{\pgfqpoint{11.784299in}{5.251229in}}%
\pgfusepath{stroke}%
\end{pgfscope}%
\begin{pgfscope}%
\pgfsetbuttcap%
\pgfsetroundjoin%
\definecolor{currentfill}{rgb}{0.000000,0.000000,0.000000}%
\pgfsetfillcolor{currentfill}%
\pgfsetlinewidth{0.602250pt}%
\definecolor{currentstroke}{rgb}{0.000000,0.000000,0.000000}%
\pgfsetstrokecolor{currentstroke}%
\pgfsetdash{}{0pt}%
\pgfsys@defobject{currentmarker}{\pgfqpoint{-0.027778in}{0.000000in}}{\pgfqpoint{-0.000000in}{0.000000in}}{%
\pgfpathmoveto{\pgfqpoint{-0.000000in}{0.000000in}}%
\pgfpathlineto{\pgfqpoint{-0.027778in}{0.000000in}}%
\pgfusepath{stroke,fill}%
}%
\begin{pgfscope}%
\pgfsys@transformshift{0.688192in}{5.251229in}%
\pgfsys@useobject{currentmarker}{}%
\end{pgfscope}%
\end{pgfscope}%
\begin{pgfscope}%
\pgfpathrectangle{\pgfqpoint{0.688192in}{5.094806in}}{\pgfqpoint{11.096108in}{4.223431in}}%
\pgfusepath{clip}%
\pgfsetrectcap%
\pgfsetroundjoin%
\pgfsetlinewidth{0.803000pt}%
\definecolor{currentstroke}{rgb}{0.690196,0.690196,0.690196}%
\pgfsetstrokecolor{currentstroke}%
\pgfsetstrokeopacity{0.050000}%
\pgfsetdash{}{0pt}%
\pgfpathmoveto{\pgfqpoint{0.688192in}{5.407652in}}%
\pgfpathlineto{\pgfqpoint{11.784299in}{5.407652in}}%
\pgfusepath{stroke}%
\end{pgfscope}%
\begin{pgfscope}%
\pgfsetbuttcap%
\pgfsetroundjoin%
\definecolor{currentfill}{rgb}{0.000000,0.000000,0.000000}%
\pgfsetfillcolor{currentfill}%
\pgfsetlinewidth{0.602250pt}%
\definecolor{currentstroke}{rgb}{0.000000,0.000000,0.000000}%
\pgfsetstrokecolor{currentstroke}%
\pgfsetdash{}{0pt}%
\pgfsys@defobject{currentmarker}{\pgfqpoint{-0.027778in}{0.000000in}}{\pgfqpoint{-0.000000in}{0.000000in}}{%
\pgfpathmoveto{\pgfqpoint{-0.000000in}{0.000000in}}%
\pgfpathlineto{\pgfqpoint{-0.027778in}{0.000000in}}%
\pgfusepath{stroke,fill}%
}%
\begin{pgfscope}%
\pgfsys@transformshift{0.688192in}{5.407652in}%
\pgfsys@useobject{currentmarker}{}%
\end{pgfscope}%
\end{pgfscope}%
\begin{pgfscope}%
\pgfpathrectangle{\pgfqpoint{0.688192in}{5.094806in}}{\pgfqpoint{11.096108in}{4.223431in}}%
\pgfusepath{clip}%
\pgfsetrectcap%
\pgfsetroundjoin%
\pgfsetlinewidth{0.803000pt}%
\definecolor{currentstroke}{rgb}{0.690196,0.690196,0.690196}%
\pgfsetstrokecolor{currentstroke}%
\pgfsetstrokeopacity{0.050000}%
\pgfsetdash{}{0pt}%
\pgfpathmoveto{\pgfqpoint{0.688192in}{5.564076in}}%
\pgfpathlineto{\pgfqpoint{11.784299in}{5.564076in}}%
\pgfusepath{stroke}%
\end{pgfscope}%
\begin{pgfscope}%
\pgfsetbuttcap%
\pgfsetroundjoin%
\definecolor{currentfill}{rgb}{0.000000,0.000000,0.000000}%
\pgfsetfillcolor{currentfill}%
\pgfsetlinewidth{0.602250pt}%
\definecolor{currentstroke}{rgb}{0.000000,0.000000,0.000000}%
\pgfsetstrokecolor{currentstroke}%
\pgfsetdash{}{0pt}%
\pgfsys@defobject{currentmarker}{\pgfqpoint{-0.027778in}{0.000000in}}{\pgfqpoint{-0.000000in}{0.000000in}}{%
\pgfpathmoveto{\pgfqpoint{-0.000000in}{0.000000in}}%
\pgfpathlineto{\pgfqpoint{-0.027778in}{0.000000in}}%
\pgfusepath{stroke,fill}%
}%
\begin{pgfscope}%
\pgfsys@transformshift{0.688192in}{5.564076in}%
\pgfsys@useobject{currentmarker}{}%
\end{pgfscope}%
\end{pgfscope}%
\begin{pgfscope}%
\pgfpathrectangle{\pgfqpoint{0.688192in}{5.094806in}}{\pgfqpoint{11.096108in}{4.223431in}}%
\pgfusepath{clip}%
\pgfsetrectcap%
\pgfsetroundjoin%
\pgfsetlinewidth{0.803000pt}%
\definecolor{currentstroke}{rgb}{0.690196,0.690196,0.690196}%
\pgfsetstrokecolor{currentstroke}%
\pgfsetstrokeopacity{0.050000}%
\pgfsetdash{}{0pt}%
\pgfpathmoveto{\pgfqpoint{0.688192in}{5.720499in}}%
\pgfpathlineto{\pgfqpoint{11.784299in}{5.720499in}}%
\pgfusepath{stroke}%
\end{pgfscope}%
\begin{pgfscope}%
\pgfsetbuttcap%
\pgfsetroundjoin%
\definecolor{currentfill}{rgb}{0.000000,0.000000,0.000000}%
\pgfsetfillcolor{currentfill}%
\pgfsetlinewidth{0.602250pt}%
\definecolor{currentstroke}{rgb}{0.000000,0.000000,0.000000}%
\pgfsetstrokecolor{currentstroke}%
\pgfsetdash{}{0pt}%
\pgfsys@defobject{currentmarker}{\pgfqpoint{-0.027778in}{0.000000in}}{\pgfqpoint{-0.000000in}{0.000000in}}{%
\pgfpathmoveto{\pgfqpoint{-0.000000in}{0.000000in}}%
\pgfpathlineto{\pgfqpoint{-0.027778in}{0.000000in}}%
\pgfusepath{stroke,fill}%
}%
\begin{pgfscope}%
\pgfsys@transformshift{0.688192in}{5.720499in}%
\pgfsys@useobject{currentmarker}{}%
\end{pgfscope}%
\end{pgfscope}%
\begin{pgfscope}%
\pgfpathrectangle{\pgfqpoint{0.688192in}{5.094806in}}{\pgfqpoint{11.096108in}{4.223431in}}%
\pgfusepath{clip}%
\pgfsetrectcap%
\pgfsetroundjoin%
\pgfsetlinewidth{0.803000pt}%
\definecolor{currentstroke}{rgb}{0.690196,0.690196,0.690196}%
\pgfsetstrokecolor{currentstroke}%
\pgfsetstrokeopacity{0.050000}%
\pgfsetdash{}{0pt}%
\pgfpathmoveto{\pgfqpoint{0.688192in}{6.033346in}}%
\pgfpathlineto{\pgfqpoint{11.784299in}{6.033346in}}%
\pgfusepath{stroke}%
\end{pgfscope}%
\begin{pgfscope}%
\pgfsetbuttcap%
\pgfsetroundjoin%
\definecolor{currentfill}{rgb}{0.000000,0.000000,0.000000}%
\pgfsetfillcolor{currentfill}%
\pgfsetlinewidth{0.602250pt}%
\definecolor{currentstroke}{rgb}{0.000000,0.000000,0.000000}%
\pgfsetstrokecolor{currentstroke}%
\pgfsetdash{}{0pt}%
\pgfsys@defobject{currentmarker}{\pgfqpoint{-0.027778in}{0.000000in}}{\pgfqpoint{-0.000000in}{0.000000in}}{%
\pgfpathmoveto{\pgfqpoint{-0.000000in}{0.000000in}}%
\pgfpathlineto{\pgfqpoint{-0.027778in}{0.000000in}}%
\pgfusepath{stroke,fill}%
}%
\begin{pgfscope}%
\pgfsys@transformshift{0.688192in}{6.033346in}%
\pgfsys@useobject{currentmarker}{}%
\end{pgfscope}%
\end{pgfscope}%
\begin{pgfscope}%
\pgfpathrectangle{\pgfqpoint{0.688192in}{5.094806in}}{\pgfqpoint{11.096108in}{4.223431in}}%
\pgfusepath{clip}%
\pgfsetrectcap%
\pgfsetroundjoin%
\pgfsetlinewidth{0.803000pt}%
\definecolor{currentstroke}{rgb}{0.690196,0.690196,0.690196}%
\pgfsetstrokecolor{currentstroke}%
\pgfsetstrokeopacity{0.050000}%
\pgfsetdash{}{0pt}%
\pgfpathmoveto{\pgfqpoint{0.688192in}{6.189769in}}%
\pgfpathlineto{\pgfqpoint{11.784299in}{6.189769in}}%
\pgfusepath{stroke}%
\end{pgfscope}%
\begin{pgfscope}%
\pgfsetbuttcap%
\pgfsetroundjoin%
\definecolor{currentfill}{rgb}{0.000000,0.000000,0.000000}%
\pgfsetfillcolor{currentfill}%
\pgfsetlinewidth{0.602250pt}%
\definecolor{currentstroke}{rgb}{0.000000,0.000000,0.000000}%
\pgfsetstrokecolor{currentstroke}%
\pgfsetdash{}{0pt}%
\pgfsys@defobject{currentmarker}{\pgfqpoint{-0.027778in}{0.000000in}}{\pgfqpoint{-0.000000in}{0.000000in}}{%
\pgfpathmoveto{\pgfqpoint{-0.000000in}{0.000000in}}%
\pgfpathlineto{\pgfqpoint{-0.027778in}{0.000000in}}%
\pgfusepath{stroke,fill}%
}%
\begin{pgfscope}%
\pgfsys@transformshift{0.688192in}{6.189769in}%
\pgfsys@useobject{currentmarker}{}%
\end{pgfscope}%
\end{pgfscope}%
\begin{pgfscope}%
\pgfpathrectangle{\pgfqpoint{0.688192in}{5.094806in}}{\pgfqpoint{11.096108in}{4.223431in}}%
\pgfusepath{clip}%
\pgfsetrectcap%
\pgfsetroundjoin%
\pgfsetlinewidth{0.803000pt}%
\definecolor{currentstroke}{rgb}{0.690196,0.690196,0.690196}%
\pgfsetstrokecolor{currentstroke}%
\pgfsetstrokeopacity{0.050000}%
\pgfsetdash{}{0pt}%
\pgfpathmoveto{\pgfqpoint{0.688192in}{6.346192in}}%
\pgfpathlineto{\pgfqpoint{11.784299in}{6.346192in}}%
\pgfusepath{stroke}%
\end{pgfscope}%
\begin{pgfscope}%
\pgfsetbuttcap%
\pgfsetroundjoin%
\definecolor{currentfill}{rgb}{0.000000,0.000000,0.000000}%
\pgfsetfillcolor{currentfill}%
\pgfsetlinewidth{0.602250pt}%
\definecolor{currentstroke}{rgb}{0.000000,0.000000,0.000000}%
\pgfsetstrokecolor{currentstroke}%
\pgfsetdash{}{0pt}%
\pgfsys@defobject{currentmarker}{\pgfqpoint{-0.027778in}{0.000000in}}{\pgfqpoint{-0.000000in}{0.000000in}}{%
\pgfpathmoveto{\pgfqpoint{-0.000000in}{0.000000in}}%
\pgfpathlineto{\pgfqpoint{-0.027778in}{0.000000in}}%
\pgfusepath{stroke,fill}%
}%
\begin{pgfscope}%
\pgfsys@transformshift{0.688192in}{6.346192in}%
\pgfsys@useobject{currentmarker}{}%
\end{pgfscope}%
\end{pgfscope}%
\begin{pgfscope}%
\pgfpathrectangle{\pgfqpoint{0.688192in}{5.094806in}}{\pgfqpoint{11.096108in}{4.223431in}}%
\pgfusepath{clip}%
\pgfsetrectcap%
\pgfsetroundjoin%
\pgfsetlinewidth{0.803000pt}%
\definecolor{currentstroke}{rgb}{0.690196,0.690196,0.690196}%
\pgfsetstrokecolor{currentstroke}%
\pgfsetstrokeopacity{0.050000}%
\pgfsetdash{}{0pt}%
\pgfpathmoveto{\pgfqpoint{0.688192in}{6.502616in}}%
\pgfpathlineto{\pgfqpoint{11.784299in}{6.502616in}}%
\pgfusepath{stroke}%
\end{pgfscope}%
\begin{pgfscope}%
\pgfsetbuttcap%
\pgfsetroundjoin%
\definecolor{currentfill}{rgb}{0.000000,0.000000,0.000000}%
\pgfsetfillcolor{currentfill}%
\pgfsetlinewidth{0.602250pt}%
\definecolor{currentstroke}{rgb}{0.000000,0.000000,0.000000}%
\pgfsetstrokecolor{currentstroke}%
\pgfsetdash{}{0pt}%
\pgfsys@defobject{currentmarker}{\pgfqpoint{-0.027778in}{0.000000in}}{\pgfqpoint{-0.000000in}{0.000000in}}{%
\pgfpathmoveto{\pgfqpoint{-0.000000in}{0.000000in}}%
\pgfpathlineto{\pgfqpoint{-0.027778in}{0.000000in}}%
\pgfusepath{stroke,fill}%
}%
\begin{pgfscope}%
\pgfsys@transformshift{0.688192in}{6.502616in}%
\pgfsys@useobject{currentmarker}{}%
\end{pgfscope}%
\end{pgfscope}%
\begin{pgfscope}%
\pgfpathrectangle{\pgfqpoint{0.688192in}{5.094806in}}{\pgfqpoint{11.096108in}{4.223431in}}%
\pgfusepath{clip}%
\pgfsetrectcap%
\pgfsetroundjoin%
\pgfsetlinewidth{0.803000pt}%
\definecolor{currentstroke}{rgb}{0.690196,0.690196,0.690196}%
\pgfsetstrokecolor{currentstroke}%
\pgfsetstrokeopacity{0.050000}%
\pgfsetdash{}{0pt}%
\pgfpathmoveto{\pgfqpoint{0.688192in}{6.815463in}}%
\pgfpathlineto{\pgfqpoint{11.784299in}{6.815463in}}%
\pgfusepath{stroke}%
\end{pgfscope}%
\begin{pgfscope}%
\pgfsetbuttcap%
\pgfsetroundjoin%
\definecolor{currentfill}{rgb}{0.000000,0.000000,0.000000}%
\pgfsetfillcolor{currentfill}%
\pgfsetlinewidth{0.602250pt}%
\definecolor{currentstroke}{rgb}{0.000000,0.000000,0.000000}%
\pgfsetstrokecolor{currentstroke}%
\pgfsetdash{}{0pt}%
\pgfsys@defobject{currentmarker}{\pgfqpoint{-0.027778in}{0.000000in}}{\pgfqpoint{-0.000000in}{0.000000in}}{%
\pgfpathmoveto{\pgfqpoint{-0.000000in}{0.000000in}}%
\pgfpathlineto{\pgfqpoint{-0.027778in}{0.000000in}}%
\pgfusepath{stroke,fill}%
}%
\begin{pgfscope}%
\pgfsys@transformshift{0.688192in}{6.815463in}%
\pgfsys@useobject{currentmarker}{}%
\end{pgfscope}%
\end{pgfscope}%
\begin{pgfscope}%
\pgfpathrectangle{\pgfqpoint{0.688192in}{5.094806in}}{\pgfqpoint{11.096108in}{4.223431in}}%
\pgfusepath{clip}%
\pgfsetrectcap%
\pgfsetroundjoin%
\pgfsetlinewidth{0.803000pt}%
\definecolor{currentstroke}{rgb}{0.690196,0.690196,0.690196}%
\pgfsetstrokecolor{currentstroke}%
\pgfsetstrokeopacity{0.050000}%
\pgfsetdash{}{0pt}%
\pgfpathmoveto{\pgfqpoint{0.688192in}{6.971886in}}%
\pgfpathlineto{\pgfqpoint{11.784299in}{6.971886in}}%
\pgfusepath{stroke}%
\end{pgfscope}%
\begin{pgfscope}%
\pgfsetbuttcap%
\pgfsetroundjoin%
\definecolor{currentfill}{rgb}{0.000000,0.000000,0.000000}%
\pgfsetfillcolor{currentfill}%
\pgfsetlinewidth{0.602250pt}%
\definecolor{currentstroke}{rgb}{0.000000,0.000000,0.000000}%
\pgfsetstrokecolor{currentstroke}%
\pgfsetdash{}{0pt}%
\pgfsys@defobject{currentmarker}{\pgfqpoint{-0.027778in}{0.000000in}}{\pgfqpoint{-0.000000in}{0.000000in}}{%
\pgfpathmoveto{\pgfqpoint{-0.000000in}{0.000000in}}%
\pgfpathlineto{\pgfqpoint{-0.027778in}{0.000000in}}%
\pgfusepath{stroke,fill}%
}%
\begin{pgfscope}%
\pgfsys@transformshift{0.688192in}{6.971886in}%
\pgfsys@useobject{currentmarker}{}%
\end{pgfscope}%
\end{pgfscope}%
\begin{pgfscope}%
\pgfpathrectangle{\pgfqpoint{0.688192in}{5.094806in}}{\pgfqpoint{11.096108in}{4.223431in}}%
\pgfusepath{clip}%
\pgfsetrectcap%
\pgfsetroundjoin%
\pgfsetlinewidth{0.803000pt}%
\definecolor{currentstroke}{rgb}{0.690196,0.690196,0.690196}%
\pgfsetstrokecolor{currentstroke}%
\pgfsetstrokeopacity{0.050000}%
\pgfsetdash{}{0pt}%
\pgfpathmoveto{\pgfqpoint{0.688192in}{7.128309in}}%
\pgfpathlineto{\pgfqpoint{11.784299in}{7.128309in}}%
\pgfusepath{stroke}%
\end{pgfscope}%
\begin{pgfscope}%
\pgfsetbuttcap%
\pgfsetroundjoin%
\definecolor{currentfill}{rgb}{0.000000,0.000000,0.000000}%
\pgfsetfillcolor{currentfill}%
\pgfsetlinewidth{0.602250pt}%
\definecolor{currentstroke}{rgb}{0.000000,0.000000,0.000000}%
\pgfsetstrokecolor{currentstroke}%
\pgfsetdash{}{0pt}%
\pgfsys@defobject{currentmarker}{\pgfqpoint{-0.027778in}{0.000000in}}{\pgfqpoint{-0.000000in}{0.000000in}}{%
\pgfpathmoveto{\pgfqpoint{-0.000000in}{0.000000in}}%
\pgfpathlineto{\pgfqpoint{-0.027778in}{0.000000in}}%
\pgfusepath{stroke,fill}%
}%
\begin{pgfscope}%
\pgfsys@transformshift{0.688192in}{7.128309in}%
\pgfsys@useobject{currentmarker}{}%
\end{pgfscope}%
\end{pgfscope}%
\begin{pgfscope}%
\pgfpathrectangle{\pgfqpoint{0.688192in}{5.094806in}}{\pgfqpoint{11.096108in}{4.223431in}}%
\pgfusepath{clip}%
\pgfsetrectcap%
\pgfsetroundjoin%
\pgfsetlinewidth{0.803000pt}%
\definecolor{currentstroke}{rgb}{0.690196,0.690196,0.690196}%
\pgfsetstrokecolor{currentstroke}%
\pgfsetstrokeopacity{0.050000}%
\pgfsetdash{}{0pt}%
\pgfpathmoveto{\pgfqpoint{0.688192in}{7.284733in}}%
\pgfpathlineto{\pgfqpoint{11.784299in}{7.284733in}}%
\pgfusepath{stroke}%
\end{pgfscope}%
\begin{pgfscope}%
\pgfsetbuttcap%
\pgfsetroundjoin%
\definecolor{currentfill}{rgb}{0.000000,0.000000,0.000000}%
\pgfsetfillcolor{currentfill}%
\pgfsetlinewidth{0.602250pt}%
\definecolor{currentstroke}{rgb}{0.000000,0.000000,0.000000}%
\pgfsetstrokecolor{currentstroke}%
\pgfsetdash{}{0pt}%
\pgfsys@defobject{currentmarker}{\pgfqpoint{-0.027778in}{0.000000in}}{\pgfqpoint{-0.000000in}{0.000000in}}{%
\pgfpathmoveto{\pgfqpoint{-0.000000in}{0.000000in}}%
\pgfpathlineto{\pgfqpoint{-0.027778in}{0.000000in}}%
\pgfusepath{stroke,fill}%
}%
\begin{pgfscope}%
\pgfsys@transformshift{0.688192in}{7.284733in}%
\pgfsys@useobject{currentmarker}{}%
\end{pgfscope}%
\end{pgfscope}%
\begin{pgfscope}%
\pgfpathrectangle{\pgfqpoint{0.688192in}{5.094806in}}{\pgfqpoint{11.096108in}{4.223431in}}%
\pgfusepath{clip}%
\pgfsetrectcap%
\pgfsetroundjoin%
\pgfsetlinewidth{0.803000pt}%
\definecolor{currentstroke}{rgb}{0.690196,0.690196,0.690196}%
\pgfsetstrokecolor{currentstroke}%
\pgfsetstrokeopacity{0.050000}%
\pgfsetdash{}{0pt}%
\pgfpathmoveto{\pgfqpoint{0.688192in}{7.597579in}}%
\pgfpathlineto{\pgfqpoint{11.784299in}{7.597579in}}%
\pgfusepath{stroke}%
\end{pgfscope}%
\begin{pgfscope}%
\pgfsetbuttcap%
\pgfsetroundjoin%
\definecolor{currentfill}{rgb}{0.000000,0.000000,0.000000}%
\pgfsetfillcolor{currentfill}%
\pgfsetlinewidth{0.602250pt}%
\definecolor{currentstroke}{rgb}{0.000000,0.000000,0.000000}%
\pgfsetstrokecolor{currentstroke}%
\pgfsetdash{}{0pt}%
\pgfsys@defobject{currentmarker}{\pgfqpoint{-0.027778in}{0.000000in}}{\pgfqpoint{-0.000000in}{0.000000in}}{%
\pgfpathmoveto{\pgfqpoint{-0.000000in}{0.000000in}}%
\pgfpathlineto{\pgfqpoint{-0.027778in}{0.000000in}}%
\pgfusepath{stroke,fill}%
}%
\begin{pgfscope}%
\pgfsys@transformshift{0.688192in}{7.597579in}%
\pgfsys@useobject{currentmarker}{}%
\end{pgfscope}%
\end{pgfscope}%
\begin{pgfscope}%
\pgfpathrectangle{\pgfqpoint{0.688192in}{5.094806in}}{\pgfqpoint{11.096108in}{4.223431in}}%
\pgfusepath{clip}%
\pgfsetrectcap%
\pgfsetroundjoin%
\pgfsetlinewidth{0.803000pt}%
\definecolor{currentstroke}{rgb}{0.690196,0.690196,0.690196}%
\pgfsetstrokecolor{currentstroke}%
\pgfsetstrokeopacity{0.050000}%
\pgfsetdash{}{0pt}%
\pgfpathmoveto{\pgfqpoint{0.688192in}{7.754003in}}%
\pgfpathlineto{\pgfqpoint{11.784299in}{7.754003in}}%
\pgfusepath{stroke}%
\end{pgfscope}%
\begin{pgfscope}%
\pgfsetbuttcap%
\pgfsetroundjoin%
\definecolor{currentfill}{rgb}{0.000000,0.000000,0.000000}%
\pgfsetfillcolor{currentfill}%
\pgfsetlinewidth{0.602250pt}%
\definecolor{currentstroke}{rgb}{0.000000,0.000000,0.000000}%
\pgfsetstrokecolor{currentstroke}%
\pgfsetdash{}{0pt}%
\pgfsys@defobject{currentmarker}{\pgfqpoint{-0.027778in}{0.000000in}}{\pgfqpoint{-0.000000in}{0.000000in}}{%
\pgfpathmoveto{\pgfqpoint{-0.000000in}{0.000000in}}%
\pgfpathlineto{\pgfqpoint{-0.027778in}{0.000000in}}%
\pgfusepath{stroke,fill}%
}%
\begin{pgfscope}%
\pgfsys@transformshift{0.688192in}{7.754003in}%
\pgfsys@useobject{currentmarker}{}%
\end{pgfscope}%
\end{pgfscope}%
\begin{pgfscope}%
\pgfpathrectangle{\pgfqpoint{0.688192in}{5.094806in}}{\pgfqpoint{11.096108in}{4.223431in}}%
\pgfusepath{clip}%
\pgfsetrectcap%
\pgfsetroundjoin%
\pgfsetlinewidth{0.803000pt}%
\definecolor{currentstroke}{rgb}{0.690196,0.690196,0.690196}%
\pgfsetstrokecolor{currentstroke}%
\pgfsetstrokeopacity{0.050000}%
\pgfsetdash{}{0pt}%
\pgfpathmoveto{\pgfqpoint{0.688192in}{7.910426in}}%
\pgfpathlineto{\pgfqpoint{11.784299in}{7.910426in}}%
\pgfusepath{stroke}%
\end{pgfscope}%
\begin{pgfscope}%
\pgfsetbuttcap%
\pgfsetroundjoin%
\definecolor{currentfill}{rgb}{0.000000,0.000000,0.000000}%
\pgfsetfillcolor{currentfill}%
\pgfsetlinewidth{0.602250pt}%
\definecolor{currentstroke}{rgb}{0.000000,0.000000,0.000000}%
\pgfsetstrokecolor{currentstroke}%
\pgfsetdash{}{0pt}%
\pgfsys@defobject{currentmarker}{\pgfqpoint{-0.027778in}{0.000000in}}{\pgfqpoint{-0.000000in}{0.000000in}}{%
\pgfpathmoveto{\pgfqpoint{-0.000000in}{0.000000in}}%
\pgfpathlineto{\pgfqpoint{-0.027778in}{0.000000in}}%
\pgfusepath{stroke,fill}%
}%
\begin{pgfscope}%
\pgfsys@transformshift{0.688192in}{7.910426in}%
\pgfsys@useobject{currentmarker}{}%
\end{pgfscope}%
\end{pgfscope}%
\begin{pgfscope}%
\pgfpathrectangle{\pgfqpoint{0.688192in}{5.094806in}}{\pgfqpoint{11.096108in}{4.223431in}}%
\pgfusepath{clip}%
\pgfsetrectcap%
\pgfsetroundjoin%
\pgfsetlinewidth{0.803000pt}%
\definecolor{currentstroke}{rgb}{0.690196,0.690196,0.690196}%
\pgfsetstrokecolor{currentstroke}%
\pgfsetstrokeopacity{0.050000}%
\pgfsetdash{}{0pt}%
\pgfpathmoveto{\pgfqpoint{0.688192in}{8.066849in}}%
\pgfpathlineto{\pgfqpoint{11.784299in}{8.066849in}}%
\pgfusepath{stroke}%
\end{pgfscope}%
\begin{pgfscope}%
\pgfsetbuttcap%
\pgfsetroundjoin%
\definecolor{currentfill}{rgb}{0.000000,0.000000,0.000000}%
\pgfsetfillcolor{currentfill}%
\pgfsetlinewidth{0.602250pt}%
\definecolor{currentstroke}{rgb}{0.000000,0.000000,0.000000}%
\pgfsetstrokecolor{currentstroke}%
\pgfsetdash{}{0pt}%
\pgfsys@defobject{currentmarker}{\pgfqpoint{-0.027778in}{0.000000in}}{\pgfqpoint{-0.000000in}{0.000000in}}{%
\pgfpathmoveto{\pgfqpoint{-0.000000in}{0.000000in}}%
\pgfpathlineto{\pgfqpoint{-0.027778in}{0.000000in}}%
\pgfusepath{stroke,fill}%
}%
\begin{pgfscope}%
\pgfsys@transformshift{0.688192in}{8.066849in}%
\pgfsys@useobject{currentmarker}{}%
\end{pgfscope}%
\end{pgfscope}%
\begin{pgfscope}%
\pgfpathrectangle{\pgfqpoint{0.688192in}{5.094806in}}{\pgfqpoint{11.096108in}{4.223431in}}%
\pgfusepath{clip}%
\pgfsetrectcap%
\pgfsetroundjoin%
\pgfsetlinewidth{0.803000pt}%
\definecolor{currentstroke}{rgb}{0.690196,0.690196,0.690196}%
\pgfsetstrokecolor{currentstroke}%
\pgfsetstrokeopacity{0.050000}%
\pgfsetdash{}{0pt}%
\pgfpathmoveto{\pgfqpoint{0.688192in}{8.379696in}}%
\pgfpathlineto{\pgfqpoint{11.784299in}{8.379696in}}%
\pgfusepath{stroke}%
\end{pgfscope}%
\begin{pgfscope}%
\pgfsetbuttcap%
\pgfsetroundjoin%
\definecolor{currentfill}{rgb}{0.000000,0.000000,0.000000}%
\pgfsetfillcolor{currentfill}%
\pgfsetlinewidth{0.602250pt}%
\definecolor{currentstroke}{rgb}{0.000000,0.000000,0.000000}%
\pgfsetstrokecolor{currentstroke}%
\pgfsetdash{}{0pt}%
\pgfsys@defobject{currentmarker}{\pgfqpoint{-0.027778in}{0.000000in}}{\pgfqpoint{-0.000000in}{0.000000in}}{%
\pgfpathmoveto{\pgfqpoint{-0.000000in}{0.000000in}}%
\pgfpathlineto{\pgfqpoint{-0.027778in}{0.000000in}}%
\pgfusepath{stroke,fill}%
}%
\begin{pgfscope}%
\pgfsys@transformshift{0.688192in}{8.379696in}%
\pgfsys@useobject{currentmarker}{}%
\end{pgfscope}%
\end{pgfscope}%
\begin{pgfscope}%
\pgfpathrectangle{\pgfqpoint{0.688192in}{5.094806in}}{\pgfqpoint{11.096108in}{4.223431in}}%
\pgfusepath{clip}%
\pgfsetrectcap%
\pgfsetroundjoin%
\pgfsetlinewidth{0.803000pt}%
\definecolor{currentstroke}{rgb}{0.690196,0.690196,0.690196}%
\pgfsetstrokecolor{currentstroke}%
\pgfsetstrokeopacity{0.050000}%
\pgfsetdash{}{0pt}%
\pgfpathmoveto{\pgfqpoint{0.688192in}{8.536119in}}%
\pgfpathlineto{\pgfqpoint{11.784299in}{8.536119in}}%
\pgfusepath{stroke}%
\end{pgfscope}%
\begin{pgfscope}%
\pgfsetbuttcap%
\pgfsetroundjoin%
\definecolor{currentfill}{rgb}{0.000000,0.000000,0.000000}%
\pgfsetfillcolor{currentfill}%
\pgfsetlinewidth{0.602250pt}%
\definecolor{currentstroke}{rgb}{0.000000,0.000000,0.000000}%
\pgfsetstrokecolor{currentstroke}%
\pgfsetdash{}{0pt}%
\pgfsys@defobject{currentmarker}{\pgfqpoint{-0.027778in}{0.000000in}}{\pgfqpoint{-0.000000in}{0.000000in}}{%
\pgfpathmoveto{\pgfqpoint{-0.000000in}{0.000000in}}%
\pgfpathlineto{\pgfqpoint{-0.027778in}{0.000000in}}%
\pgfusepath{stroke,fill}%
}%
\begin{pgfscope}%
\pgfsys@transformshift{0.688192in}{8.536119in}%
\pgfsys@useobject{currentmarker}{}%
\end{pgfscope}%
\end{pgfscope}%
\begin{pgfscope}%
\pgfpathrectangle{\pgfqpoint{0.688192in}{5.094806in}}{\pgfqpoint{11.096108in}{4.223431in}}%
\pgfusepath{clip}%
\pgfsetrectcap%
\pgfsetroundjoin%
\pgfsetlinewidth{0.803000pt}%
\definecolor{currentstroke}{rgb}{0.690196,0.690196,0.690196}%
\pgfsetstrokecolor{currentstroke}%
\pgfsetstrokeopacity{0.050000}%
\pgfsetdash{}{0pt}%
\pgfpathmoveto{\pgfqpoint{0.688192in}{8.692543in}}%
\pgfpathlineto{\pgfqpoint{11.784299in}{8.692543in}}%
\pgfusepath{stroke}%
\end{pgfscope}%
\begin{pgfscope}%
\pgfsetbuttcap%
\pgfsetroundjoin%
\definecolor{currentfill}{rgb}{0.000000,0.000000,0.000000}%
\pgfsetfillcolor{currentfill}%
\pgfsetlinewidth{0.602250pt}%
\definecolor{currentstroke}{rgb}{0.000000,0.000000,0.000000}%
\pgfsetstrokecolor{currentstroke}%
\pgfsetdash{}{0pt}%
\pgfsys@defobject{currentmarker}{\pgfqpoint{-0.027778in}{0.000000in}}{\pgfqpoint{-0.000000in}{0.000000in}}{%
\pgfpathmoveto{\pgfqpoint{-0.000000in}{0.000000in}}%
\pgfpathlineto{\pgfqpoint{-0.027778in}{0.000000in}}%
\pgfusepath{stroke,fill}%
}%
\begin{pgfscope}%
\pgfsys@transformshift{0.688192in}{8.692543in}%
\pgfsys@useobject{currentmarker}{}%
\end{pgfscope}%
\end{pgfscope}%
\begin{pgfscope}%
\pgfpathrectangle{\pgfqpoint{0.688192in}{5.094806in}}{\pgfqpoint{11.096108in}{4.223431in}}%
\pgfusepath{clip}%
\pgfsetrectcap%
\pgfsetroundjoin%
\pgfsetlinewidth{0.803000pt}%
\definecolor{currentstroke}{rgb}{0.690196,0.690196,0.690196}%
\pgfsetstrokecolor{currentstroke}%
\pgfsetstrokeopacity{0.050000}%
\pgfsetdash{}{0pt}%
\pgfpathmoveto{\pgfqpoint{0.688192in}{8.848966in}}%
\pgfpathlineto{\pgfqpoint{11.784299in}{8.848966in}}%
\pgfusepath{stroke}%
\end{pgfscope}%
\begin{pgfscope}%
\pgfsetbuttcap%
\pgfsetroundjoin%
\definecolor{currentfill}{rgb}{0.000000,0.000000,0.000000}%
\pgfsetfillcolor{currentfill}%
\pgfsetlinewidth{0.602250pt}%
\definecolor{currentstroke}{rgb}{0.000000,0.000000,0.000000}%
\pgfsetstrokecolor{currentstroke}%
\pgfsetdash{}{0pt}%
\pgfsys@defobject{currentmarker}{\pgfqpoint{-0.027778in}{0.000000in}}{\pgfqpoint{-0.000000in}{0.000000in}}{%
\pgfpathmoveto{\pgfqpoint{-0.000000in}{0.000000in}}%
\pgfpathlineto{\pgfqpoint{-0.027778in}{0.000000in}}%
\pgfusepath{stroke,fill}%
}%
\begin{pgfscope}%
\pgfsys@transformshift{0.688192in}{8.848966in}%
\pgfsys@useobject{currentmarker}{}%
\end{pgfscope}%
\end{pgfscope}%
\begin{pgfscope}%
\pgfpathrectangle{\pgfqpoint{0.688192in}{5.094806in}}{\pgfqpoint{11.096108in}{4.223431in}}%
\pgfusepath{clip}%
\pgfsetrectcap%
\pgfsetroundjoin%
\pgfsetlinewidth{0.803000pt}%
\definecolor{currentstroke}{rgb}{0.690196,0.690196,0.690196}%
\pgfsetstrokecolor{currentstroke}%
\pgfsetstrokeopacity{0.050000}%
\pgfsetdash{}{0pt}%
\pgfpathmoveto{\pgfqpoint{0.688192in}{9.161813in}}%
\pgfpathlineto{\pgfqpoint{11.784299in}{9.161813in}}%
\pgfusepath{stroke}%
\end{pgfscope}%
\begin{pgfscope}%
\pgfsetbuttcap%
\pgfsetroundjoin%
\definecolor{currentfill}{rgb}{0.000000,0.000000,0.000000}%
\pgfsetfillcolor{currentfill}%
\pgfsetlinewidth{0.602250pt}%
\definecolor{currentstroke}{rgb}{0.000000,0.000000,0.000000}%
\pgfsetstrokecolor{currentstroke}%
\pgfsetdash{}{0pt}%
\pgfsys@defobject{currentmarker}{\pgfqpoint{-0.027778in}{0.000000in}}{\pgfqpoint{-0.000000in}{0.000000in}}{%
\pgfpathmoveto{\pgfqpoint{-0.000000in}{0.000000in}}%
\pgfpathlineto{\pgfqpoint{-0.027778in}{0.000000in}}%
\pgfusepath{stroke,fill}%
}%
\begin{pgfscope}%
\pgfsys@transformshift{0.688192in}{9.161813in}%
\pgfsys@useobject{currentmarker}{}%
\end{pgfscope}%
\end{pgfscope}%
\begin{pgfscope}%
\pgfpathrectangle{\pgfqpoint{0.688192in}{5.094806in}}{\pgfqpoint{11.096108in}{4.223431in}}%
\pgfusepath{clip}%
\pgfsetrectcap%
\pgfsetroundjoin%
\pgfsetlinewidth{0.803000pt}%
\definecolor{currentstroke}{rgb}{0.690196,0.690196,0.690196}%
\pgfsetstrokecolor{currentstroke}%
\pgfsetstrokeopacity{0.050000}%
\pgfsetdash{}{0pt}%
\pgfpathmoveto{\pgfqpoint{0.688192in}{9.318236in}}%
\pgfpathlineto{\pgfqpoint{11.784299in}{9.318236in}}%
\pgfusepath{stroke}%
\end{pgfscope}%
\begin{pgfscope}%
\pgfsetbuttcap%
\pgfsetroundjoin%
\definecolor{currentfill}{rgb}{0.000000,0.000000,0.000000}%
\pgfsetfillcolor{currentfill}%
\pgfsetlinewidth{0.602250pt}%
\definecolor{currentstroke}{rgb}{0.000000,0.000000,0.000000}%
\pgfsetstrokecolor{currentstroke}%
\pgfsetdash{}{0pt}%
\pgfsys@defobject{currentmarker}{\pgfqpoint{-0.027778in}{0.000000in}}{\pgfqpoint{-0.000000in}{0.000000in}}{%
\pgfpathmoveto{\pgfqpoint{-0.000000in}{0.000000in}}%
\pgfpathlineto{\pgfqpoint{-0.027778in}{0.000000in}}%
\pgfusepath{stroke,fill}%
}%
\begin{pgfscope}%
\pgfsys@transformshift{0.688192in}{9.318236in}%
\pgfsys@useobject{currentmarker}{}%
\end{pgfscope}%
\end{pgfscope}%
\begin{pgfscope}%
\definecolor{textcolor}{rgb}{0.000000,0.000000,0.000000}%
\pgfsetstrokecolor{textcolor}%
\pgfsetfillcolor{textcolor}%
\pgftext[x=0.339583in,y=7.206521in,,bottom,rotate=90.000000]{\color{textcolor}{\rmfamily\fontsize{18.000000}{21.600000}\selectfont\catcode`\^=\active\def^{\ifmmode\sp\else\^{}\fi}\catcode`\%=\active\def%{\%}Osier MGA Capacity (GW)}}%
\end{pgfscope}%
\begin{pgfscope}%
\pgfpathrectangle{\pgfqpoint{0.688192in}{5.094806in}}{\pgfqpoint{11.096108in}{4.223431in}}%
\pgfusepath{clip}%
\pgfsetbuttcap%
\pgfsetroundjoin%
\pgfsetlinewidth{0.941016pt}%
\definecolor{currentstroke}{rgb}{0.240000,0.240000,0.240000}%
\pgfsetstrokecolor{currentstroke}%
\pgfsetdash{}{0pt}%
\pgfpathmoveto{\pgfqpoint{0.799153in}{7.965062in}}%
\pgfpathlineto{\pgfqpoint{1.686841in}{7.965062in}}%
\pgfusepath{stroke}%
\end{pgfscope}%
\begin{pgfscope}%
\pgfpathrectangle{\pgfqpoint{0.688192in}{5.094806in}}{\pgfqpoint{11.096108in}{4.223431in}}%
\pgfusepath{clip}%
\pgfsetbuttcap%
\pgfsetroundjoin%
\pgfsetlinewidth{0.941016pt}%
\definecolor{currentstroke}{rgb}{0.240000,0.240000,0.240000}%
\pgfsetstrokecolor{currentstroke}%
\pgfsetdash{}{0pt}%
\pgfpathmoveto{\pgfqpoint{1.908763in}{5.094806in}}%
\pgfpathlineto{\pgfqpoint{2.796452in}{5.094806in}}%
\pgfusepath{stroke}%
\end{pgfscope}%
\begin{pgfscope}%
\pgfpathrectangle{\pgfqpoint{0.688192in}{5.094806in}}{\pgfqpoint{11.096108in}{4.223431in}}%
\pgfusepath{clip}%
\pgfsetbuttcap%
\pgfsetroundjoin%
\pgfsetlinewidth{0.941016pt}%
\definecolor{currentstroke}{rgb}{0.240000,0.240000,0.240000}%
\pgfsetstrokecolor{currentstroke}%
\pgfsetdash{}{0pt}%
\pgfpathmoveto{\pgfqpoint{3.018374in}{5.173926in}}%
\pgfpathlineto{\pgfqpoint{3.906063in}{5.173926in}}%
\pgfusepath{stroke}%
\end{pgfscope}%
\begin{pgfscope}%
\pgfpathrectangle{\pgfqpoint{0.688192in}{5.094806in}}{\pgfqpoint{11.096108in}{4.223431in}}%
\pgfusepath{clip}%
\pgfsetbuttcap%
\pgfsetroundjoin%
\pgfsetlinewidth{0.941016pt}%
\definecolor{currentstroke}{rgb}{0.240000,0.240000,0.240000}%
\pgfsetstrokecolor{currentstroke}%
\pgfsetdash{}{0pt}%
\pgfpathmoveto{\pgfqpoint{4.127985in}{5.094806in}}%
\pgfpathlineto{\pgfqpoint{5.015674in}{5.094806in}}%
\pgfusepath{stroke}%
\end{pgfscope}%
\begin{pgfscope}%
\pgfpathrectangle{\pgfqpoint{0.688192in}{5.094806in}}{\pgfqpoint{11.096108in}{4.223431in}}%
\pgfusepath{clip}%
\pgfsetbuttcap%
\pgfsetroundjoin%
\pgfsetlinewidth{0.941016pt}%
\definecolor{currentstroke}{rgb}{0.240000,0.240000,0.240000}%
\pgfsetstrokecolor{currentstroke}%
\pgfsetdash{}{0pt}%
\pgfpathmoveto{\pgfqpoint{5.237596in}{5.094806in}}%
\pgfpathlineto{\pgfqpoint{6.125284in}{5.094806in}}%
\pgfusepath{stroke}%
\end{pgfscope}%
\begin{pgfscope}%
\pgfpathrectangle{\pgfqpoint{0.688192in}{5.094806in}}{\pgfqpoint{11.096108in}{4.223431in}}%
\pgfusepath{clip}%
\pgfsetbuttcap%
\pgfsetroundjoin%
\pgfsetlinewidth{0.941016pt}%
\definecolor{currentstroke}{rgb}{0.240000,0.240000,0.240000}%
\pgfsetstrokecolor{currentstroke}%
\pgfsetdash{}{0pt}%
\pgfpathmoveto{\pgfqpoint{6.347207in}{5.094806in}}%
\pgfpathlineto{\pgfqpoint{7.234895in}{5.094806in}}%
\pgfusepath{stroke}%
\end{pgfscope}%
\begin{pgfscope}%
\pgfpathrectangle{\pgfqpoint{0.688192in}{5.094806in}}{\pgfqpoint{11.096108in}{4.223431in}}%
\pgfusepath{clip}%
\pgfsetbuttcap%
\pgfsetroundjoin%
\pgfsetlinewidth{0.941016pt}%
\definecolor{currentstroke}{rgb}{0.240000,0.240000,0.240000}%
\pgfsetstrokecolor{currentstroke}%
\pgfsetdash{}{0pt}%
\pgfpathmoveto{\pgfqpoint{7.456817in}{5.094806in}}%
\pgfpathlineto{\pgfqpoint{8.344506in}{5.094806in}}%
\pgfusepath{stroke}%
\end{pgfscope}%
\begin{pgfscope}%
\pgfpathrectangle{\pgfqpoint{0.688192in}{5.094806in}}{\pgfqpoint{11.096108in}{4.223431in}}%
\pgfusepath{clip}%
\pgfsetbuttcap%
\pgfsetroundjoin%
\pgfsetlinewidth{0.941016pt}%
\definecolor{currentstroke}{rgb}{0.240000,0.240000,0.240000}%
\pgfsetstrokecolor{currentstroke}%
\pgfsetdash{}{0pt}%
\pgfpathmoveto{\pgfqpoint{8.566428in}{5.999654in}}%
\pgfpathlineto{\pgfqpoint{9.454117in}{5.999654in}}%
\pgfusepath{stroke}%
\end{pgfscope}%
\begin{pgfscope}%
\pgfpathrectangle{\pgfqpoint{0.688192in}{5.094806in}}{\pgfqpoint{11.096108in}{4.223431in}}%
\pgfusepath{clip}%
\pgfsetbuttcap%
\pgfsetroundjoin%
\pgfsetlinewidth{0.941016pt}%
\definecolor{currentstroke}{rgb}{0.240000,0.240000,0.240000}%
\pgfsetstrokecolor{currentstroke}%
\pgfsetdash{}{0pt}%
\pgfpathmoveto{\pgfqpoint{9.676039in}{5.094806in}}%
\pgfpathlineto{\pgfqpoint{10.563728in}{5.094806in}}%
\pgfusepath{stroke}%
\end{pgfscope}%
\begin{pgfscope}%
\pgfpathrectangle{\pgfqpoint{0.688192in}{5.094806in}}{\pgfqpoint{11.096108in}{4.223431in}}%
\pgfusepath{clip}%
\pgfsetbuttcap%
\pgfsetroundjoin%
\pgfsetlinewidth{0.941016pt}%
\definecolor{currentstroke}{rgb}{0.240000,0.240000,0.240000}%
\pgfsetstrokecolor{currentstroke}%
\pgfsetdash{}{0pt}%
\pgfpathmoveto{\pgfqpoint{10.785650in}{5.094806in}}%
\pgfpathlineto{\pgfqpoint{11.673338in}{5.094806in}}%
\pgfusepath{stroke}%
\end{pgfscope}%
\begin{pgfscope}%
\pgfsetrectcap%
\pgfsetmiterjoin%
\pgfsetlinewidth{0.803000pt}%
\definecolor{currentstroke}{rgb}{0.000000,0.000000,0.000000}%
\pgfsetstrokecolor{currentstroke}%
\pgfsetdash{}{0pt}%
\pgfpathmoveto{\pgfqpoint{0.688192in}{5.094806in}}%
\pgfpathlineto{\pgfqpoint{0.688192in}{9.318236in}}%
\pgfusepath{stroke}%
\end{pgfscope}%
\begin{pgfscope}%
\pgfsetrectcap%
\pgfsetmiterjoin%
\pgfsetlinewidth{0.803000pt}%
\definecolor{currentstroke}{rgb}{0.000000,0.000000,0.000000}%
\pgfsetstrokecolor{currentstroke}%
\pgfsetdash{}{0pt}%
\pgfpathmoveto{\pgfqpoint{11.784299in}{5.094806in}}%
\pgfpathlineto{\pgfqpoint{11.784299in}{9.318236in}}%
\pgfusepath{stroke}%
\end{pgfscope}%
\begin{pgfscope}%
\pgfsetrectcap%
\pgfsetmiterjoin%
\pgfsetlinewidth{0.803000pt}%
\definecolor{currentstroke}{rgb}{0.000000,0.000000,0.000000}%
\pgfsetstrokecolor{currentstroke}%
\pgfsetdash{}{0pt}%
\pgfpathmoveto{\pgfqpoint{0.688192in}{5.094806in}}%
\pgfpathlineto{\pgfqpoint{11.784299in}{5.094806in}}%
\pgfusepath{stroke}%
\end{pgfscope}%
\begin{pgfscope}%
\pgfsetrectcap%
\pgfsetmiterjoin%
\pgfsetlinewidth{0.803000pt}%
\definecolor{currentstroke}{rgb}{0.000000,0.000000,0.000000}%
\pgfsetstrokecolor{currentstroke}%
\pgfsetdash{}{0pt}%
\pgfpathmoveto{\pgfqpoint{0.688192in}{9.318236in}}%
\pgfpathlineto{\pgfqpoint{11.784299in}{9.318236in}}%
\pgfusepath{stroke}%
\end{pgfscope}%
\begin{pgfscope}%
\pgfsetbuttcap%
\pgfsetmiterjoin%
\definecolor{currentfill}{rgb}{1.000000,1.000000,1.000000}%
\pgfsetfillcolor{currentfill}%
\pgfsetlinewidth{1.003750pt}%
\definecolor{currentstroke}{rgb}{0.000000,0.000000,0.000000}%
\pgfsetstrokecolor{currentstroke}%
\pgfsetdash{}{0pt}%
\pgfpathmoveto{\pgfqpoint{0.842629in}{8.891180in}}%
\pgfpathlineto{\pgfqpoint{1.140358in}{8.891180in}}%
\pgfpathlineto{\pgfqpoint{1.140358in}{9.203957in}}%
\pgfpathlineto{\pgfqpoint{0.842629in}{9.203957in}}%
\pgfpathlineto{\pgfqpoint{0.842629in}{8.891180in}}%
\pgfpathclose%
\pgfusepath{stroke,fill}%
\end{pgfscope}%
\begin{pgfscope}%
\definecolor{textcolor}{rgb}{0.000000,0.000000,0.000000}%
\pgfsetstrokecolor{textcolor}%
\pgfsetfillcolor{textcolor}%
\pgftext[x=0.899018in,y=8.997568in,left,base]{\color{textcolor}{\rmfamily\fontsize{14.000000}{16.800000}\selectfont\catcode`\^=\active\def^{\ifmmode\sp\else\^{}\fi}\catcode`\%=\active\def%{\%}b)}}%
\end{pgfscope}%
\begin{pgfscope}%
\pgfsetbuttcap%
\pgfsetmiterjoin%
\definecolor{currentfill}{rgb}{1.000000,1.000000,1.000000}%
\pgfsetfillcolor{currentfill}%
\pgfsetlinewidth{0.000000pt}%
\definecolor{currentstroke}{rgb}{0.000000,0.000000,0.000000}%
\pgfsetstrokecolor{currentstroke}%
\pgfsetstrokeopacity{0.000000}%
\pgfsetdash{}{0pt}%
\pgfpathmoveto{\pgfqpoint{0.688192in}{0.613042in}}%
\pgfpathlineto{\pgfqpoint{11.784299in}{0.613042in}}%
\pgfpathlineto{\pgfqpoint{11.784299in}{4.836473in}}%
\pgfpathlineto{\pgfqpoint{0.688192in}{4.836473in}}%
\pgfpathlineto{\pgfqpoint{0.688192in}{0.613042in}}%
\pgfpathclose%
\pgfusepath{fill}%
\end{pgfscope}%
\begin{pgfscope}%
\pgfpathrectangle{\pgfqpoint{0.688192in}{0.613042in}}{\pgfqpoint{11.096108in}{4.223431in}}%
\pgfusepath{clip}%
\pgfsetbuttcap%
\pgfsetroundjoin%
\definecolor{currentfill}{rgb}{0.469005,0.634306,0.749958}%
\pgfsetfillcolor{currentfill}%
\pgfsetlinewidth{0.752812pt}%
\definecolor{currentstroke}{rgb}{0.240000,0.240000,0.240000}%
\pgfsetstrokecolor{currentstroke}%
\pgfsetdash{}{0pt}%
\pgfpathmoveto{\pgfqpoint{1.132036in}{2.763503in}}%
\pgfpathlineto{\pgfqpoint{1.353958in}{2.763503in}}%
\pgfpathlineto{\pgfqpoint{1.353958in}{2.942225in}}%
\pgfpathlineto{\pgfqpoint{1.132036in}{2.942225in}}%
\pgfpathlineto{\pgfqpoint{1.132036in}{2.763503in}}%
\pgfpathclose%
\pgfusepath{stroke,fill}%
\end{pgfscope}%
\begin{pgfscope}%
\pgfpathrectangle{\pgfqpoint{0.688192in}{0.613042in}}{\pgfqpoint{11.096108in}{4.223431in}}%
\pgfusepath{clip}%
\pgfsetbuttcap%
\pgfsetroundjoin%
\definecolor{currentfill}{rgb}{0.346402,0.553490,0.697630}%
\pgfsetfillcolor{currentfill}%
\pgfsetlinewidth{0.752812pt}%
\definecolor{currentstroke}{rgb}{0.240000,0.240000,0.240000}%
\pgfsetstrokecolor{currentstroke}%
\pgfsetdash{}{0pt}%
\pgfpathmoveto{\pgfqpoint{1.021075in}{2.942225in}}%
\pgfpathlineto{\pgfqpoint{1.464919in}{2.942225in}}%
\pgfpathlineto{\pgfqpoint{1.464919in}{2.988431in}}%
\pgfpathlineto{\pgfqpoint{1.021075in}{2.988431in}}%
\pgfpathlineto{\pgfqpoint{1.021075in}{2.942225in}}%
\pgfpathclose%
\pgfusepath{stroke,fill}%
\end{pgfscope}%
\begin{pgfscope}%
\pgfpathrectangle{\pgfqpoint{0.688192in}{0.613042in}}{\pgfqpoint{11.096108in}{4.223431in}}%
\pgfusepath{clip}%
\pgfsetbuttcap%
\pgfsetroundjoin%
\definecolor{currentfill}{rgb}{0.194608,0.453431,0.632843}%
\pgfsetfillcolor{currentfill}%
\pgfsetlinewidth{0.752812pt}%
\definecolor{currentstroke}{rgb}{0.240000,0.240000,0.240000}%
\pgfsetstrokecolor{currentstroke}%
\pgfsetdash{}{0pt}%
\pgfpathmoveto{\pgfqpoint{0.799153in}{2.988431in}}%
\pgfpathlineto{\pgfqpoint{1.686841in}{2.988431in}}%
\pgfpathlineto{\pgfqpoint{1.686841in}{3.154404in}}%
\pgfpathlineto{\pgfqpoint{0.799153in}{3.154404in}}%
\pgfpathlineto{\pgfqpoint{0.799153in}{2.988431in}}%
\pgfpathclose%
\pgfusepath{stroke,fill}%
\end{pgfscope}%
\begin{pgfscope}%
\pgfpathrectangle{\pgfqpoint{0.688192in}{0.613042in}}{\pgfqpoint{11.096108in}{4.223431in}}%
\pgfusepath{clip}%
\pgfsetbuttcap%
\pgfsetroundjoin%
\definecolor{currentfill}{rgb}{0.346402,0.553490,0.697630}%
\pgfsetfillcolor{currentfill}%
\pgfsetlinewidth{0.752812pt}%
\definecolor{currentstroke}{rgb}{0.240000,0.240000,0.240000}%
\pgfsetstrokecolor{currentstroke}%
\pgfsetdash{}{0pt}%
\pgfpathmoveto{\pgfqpoint{1.021075in}{3.154404in}}%
\pgfpathlineto{\pgfqpoint{1.464919in}{3.154404in}}%
\pgfpathlineto{\pgfqpoint{1.464919in}{3.211257in}}%
\pgfpathlineto{\pgfqpoint{1.021075in}{3.211257in}}%
\pgfpathlineto{\pgfqpoint{1.021075in}{3.154404in}}%
\pgfpathclose%
\pgfusepath{stroke,fill}%
\end{pgfscope}%
\begin{pgfscope}%
\pgfpathrectangle{\pgfqpoint{0.688192in}{0.613042in}}{\pgfqpoint{11.096108in}{4.223431in}}%
\pgfusepath{clip}%
\pgfsetbuttcap%
\pgfsetroundjoin%
\definecolor{currentfill}{rgb}{0.469005,0.634306,0.749958}%
\pgfsetfillcolor{currentfill}%
\pgfsetlinewidth{0.752812pt}%
\definecolor{currentstroke}{rgb}{0.240000,0.240000,0.240000}%
\pgfsetstrokecolor{currentstroke}%
\pgfsetdash{}{0pt}%
\pgfpathmoveto{\pgfqpoint{1.132036in}{3.211257in}}%
\pgfpathlineto{\pgfqpoint{1.353958in}{3.211257in}}%
\pgfpathlineto{\pgfqpoint{1.353958in}{3.227140in}}%
\pgfpathlineto{\pgfqpoint{1.132036in}{3.227140in}}%
\pgfpathlineto{\pgfqpoint{1.132036in}{3.211257in}}%
\pgfpathclose%
\pgfusepath{stroke,fill}%
\end{pgfscope}%
\begin{pgfscope}%
\pgfpathrectangle{\pgfqpoint{0.688192in}{0.613042in}}{\pgfqpoint{11.096108in}{4.223431in}}%
\pgfusepath{clip}%
\pgfsetbuttcap%
\pgfsetroundjoin%
\pgfsetlinewidth{1.003750pt}%
\definecolor{currentstroke}{rgb}{0.450000,0.450000,0.450000}%
\pgfsetstrokecolor{currentstroke}%
\pgfsetdash{}{0pt}%
\pgfpathmoveto{\pgfqpoint{1.242997in}{3.489265in}}%
\pgfpathcurveto{\pgfqpoint{1.252205in}{3.489265in}}{\pgfqpoint{1.261038in}{3.492924in}}{\pgfqpoint{1.267549in}{3.499435in}}%
\pgfpathcurveto{\pgfqpoint{1.274061in}{3.505946in}}{\pgfqpoint{1.277719in}{3.514779in}}{\pgfqpoint{1.277719in}{3.523987in}}%
\pgfpathcurveto{\pgfqpoint{1.277719in}{3.533196in}}{\pgfqpoint{1.274061in}{3.542028in}}{\pgfqpoint{1.267549in}{3.548540in}}%
\pgfpathcurveto{\pgfqpoint{1.261038in}{3.555051in}}{\pgfqpoint{1.252205in}{3.558710in}}{\pgfqpoint{1.242997in}{3.558710in}}%
\pgfpathcurveto{\pgfqpoint{1.233788in}{3.558710in}}{\pgfqpoint{1.224956in}{3.555051in}}{\pgfqpoint{1.218445in}{3.548540in}}%
\pgfpathcurveto{\pgfqpoint{1.211933in}{3.542028in}}{\pgfqpoint{1.208275in}{3.533196in}}{\pgfqpoint{1.208275in}{3.523987in}}%
\pgfpathcurveto{\pgfqpoint{1.208275in}{3.514779in}}{\pgfqpoint{1.211933in}{3.505946in}}{\pgfqpoint{1.218445in}{3.499435in}}%
\pgfpathcurveto{\pgfqpoint{1.224956in}{3.492924in}}{\pgfqpoint{1.233788in}{3.489265in}}{\pgfqpoint{1.242997in}{3.489265in}}%
\pgfpathlineto{\pgfqpoint{1.242997in}{3.489265in}}%
\pgfpathclose%
\pgfusepath{stroke}%
\end{pgfscope}%
\begin{pgfscope}%
\pgfpathrectangle{\pgfqpoint{0.688192in}{0.613042in}}{\pgfqpoint{11.096108in}{4.223431in}}%
\pgfusepath{clip}%
\pgfsetbuttcap%
\pgfsetroundjoin%
\pgfsetlinewidth{1.003750pt}%
\definecolor{currentstroke}{rgb}{0.450000,0.450000,0.450000}%
\pgfsetstrokecolor{currentstroke}%
\pgfsetdash{}{0pt}%
\pgfpathmoveto{\pgfqpoint{1.242997in}{2.220196in}}%
\pgfpathcurveto{\pgfqpoint{1.252205in}{2.220196in}}{\pgfqpoint{1.261038in}{2.223854in}}{\pgfqpoint{1.267549in}{2.230366in}}%
\pgfpathcurveto{\pgfqpoint{1.274061in}{2.236877in}}{\pgfqpoint{1.277719in}{2.245710in}}{\pgfqpoint{1.277719in}{2.254918in}}%
\pgfpathcurveto{\pgfqpoint{1.277719in}{2.264127in}}{\pgfqpoint{1.274061in}{2.272959in}}{\pgfqpoint{1.267549in}{2.279470in}}%
\pgfpathcurveto{\pgfqpoint{1.261038in}{2.285982in}}{\pgfqpoint{1.252205in}{2.289640in}}{\pgfqpoint{1.242997in}{2.289640in}}%
\pgfpathcurveto{\pgfqpoint{1.233788in}{2.289640in}}{\pgfqpoint{1.224956in}{2.285982in}}{\pgfqpoint{1.218445in}{2.279470in}}%
\pgfpathcurveto{\pgfqpoint{1.211933in}{2.272959in}}{\pgfqpoint{1.208275in}{2.264127in}}{\pgfqpoint{1.208275in}{2.254918in}}%
\pgfpathcurveto{\pgfqpoint{1.208275in}{2.245710in}}{\pgfqpoint{1.211933in}{2.236877in}}{\pgfqpoint{1.218445in}{2.230366in}}%
\pgfpathcurveto{\pgfqpoint{1.224956in}{2.223854in}}{\pgfqpoint{1.233788in}{2.220196in}}{\pgfqpoint{1.242997in}{2.220196in}}%
\pgfpathlineto{\pgfqpoint{1.242997in}{2.220196in}}%
\pgfpathclose%
\pgfusepath{stroke}%
\end{pgfscope}%
\begin{pgfscope}%
\pgfpathrectangle{\pgfqpoint{0.688192in}{0.613042in}}{\pgfqpoint{11.096108in}{4.223431in}}%
\pgfusepath{clip}%
\pgfsetbuttcap%
\pgfsetroundjoin%
\pgfsetlinewidth{1.003750pt}%
\definecolor{currentstroke}{rgb}{0.450000,0.450000,0.450000}%
\pgfsetstrokecolor{currentstroke}%
\pgfsetdash{}{0pt}%
\pgfpathmoveto{\pgfqpoint{1.242997in}{2.220196in}}%
\pgfpathcurveto{\pgfqpoint{1.252205in}{2.220196in}}{\pgfqpoint{1.261038in}{2.223854in}}{\pgfqpoint{1.267549in}{2.230366in}}%
\pgfpathcurveto{\pgfqpoint{1.274061in}{2.236877in}}{\pgfqpoint{1.277719in}{2.245710in}}{\pgfqpoint{1.277719in}{2.254918in}}%
\pgfpathcurveto{\pgfqpoint{1.277719in}{2.264127in}}{\pgfqpoint{1.274061in}{2.272959in}}{\pgfqpoint{1.267549in}{2.279470in}}%
\pgfpathcurveto{\pgfqpoint{1.261038in}{2.285982in}}{\pgfqpoint{1.252205in}{2.289640in}}{\pgfqpoint{1.242997in}{2.289640in}}%
\pgfpathcurveto{\pgfqpoint{1.233788in}{2.289640in}}{\pgfqpoint{1.224956in}{2.285982in}}{\pgfqpoint{1.218445in}{2.279470in}}%
\pgfpathcurveto{\pgfqpoint{1.211933in}{2.272959in}}{\pgfqpoint{1.208275in}{2.264127in}}{\pgfqpoint{1.208275in}{2.254918in}}%
\pgfpathcurveto{\pgfqpoint{1.208275in}{2.245710in}}{\pgfqpoint{1.211933in}{2.236877in}}{\pgfqpoint{1.218445in}{2.230366in}}%
\pgfpathcurveto{\pgfqpoint{1.224956in}{2.223854in}}{\pgfqpoint{1.233788in}{2.220196in}}{\pgfqpoint{1.242997in}{2.220196in}}%
\pgfpathlineto{\pgfqpoint{1.242997in}{2.220196in}}%
\pgfpathclose%
\pgfusepath{stroke}%
\end{pgfscope}%
\begin{pgfscope}%
\pgfpathrectangle{\pgfqpoint{0.688192in}{0.613042in}}{\pgfqpoint{11.096108in}{4.223431in}}%
\pgfusepath{clip}%
\pgfsetbuttcap%
\pgfsetroundjoin%
\pgfsetlinewidth{1.003750pt}%
\definecolor{currentstroke}{rgb}{0.450000,0.450000,0.450000}%
\pgfsetstrokecolor{currentstroke}%
\pgfsetdash{}{0pt}%
\pgfpathmoveto{\pgfqpoint{1.242997in}{3.194516in}}%
\pgfpathcurveto{\pgfqpoint{1.252205in}{3.194516in}}{\pgfqpoint{1.261038in}{3.198174in}}{\pgfqpoint{1.267549in}{3.204686in}}%
\pgfpathcurveto{\pgfqpoint{1.274061in}{3.211197in}}{\pgfqpoint{1.277719in}{3.220029in}}{\pgfqpoint{1.277719in}{3.229238in}}%
\pgfpathcurveto{\pgfqpoint{1.277719in}{3.238446in}}{\pgfqpoint{1.274061in}{3.247279in}}{\pgfqpoint{1.267549in}{3.253790in}}%
\pgfpathcurveto{\pgfqpoint{1.261038in}{3.260302in}}{\pgfqpoint{1.252205in}{3.263960in}}{\pgfqpoint{1.242997in}{3.263960in}}%
\pgfpathcurveto{\pgfqpoint{1.233788in}{3.263960in}}{\pgfqpoint{1.224956in}{3.260302in}}{\pgfqpoint{1.218445in}{3.253790in}}%
\pgfpathcurveto{\pgfqpoint{1.211933in}{3.247279in}}{\pgfqpoint{1.208275in}{3.238446in}}{\pgfqpoint{1.208275in}{3.229238in}}%
\pgfpathcurveto{\pgfqpoint{1.208275in}{3.220029in}}{\pgfqpoint{1.211933in}{3.211197in}}{\pgfqpoint{1.218445in}{3.204686in}}%
\pgfpathcurveto{\pgfqpoint{1.224956in}{3.198174in}}{\pgfqpoint{1.233788in}{3.194516in}}{\pgfqpoint{1.242997in}{3.194516in}}%
\pgfpathlineto{\pgfqpoint{1.242997in}{3.194516in}}%
\pgfpathclose%
\pgfusepath{stroke}%
\end{pgfscope}%
\begin{pgfscope}%
\pgfpathrectangle{\pgfqpoint{0.688192in}{0.613042in}}{\pgfqpoint{11.096108in}{4.223431in}}%
\pgfusepath{clip}%
\pgfsetbuttcap%
\pgfsetroundjoin%
\definecolor{currentfill}{rgb}{0.907545,0.666205,0.454979}%
\pgfsetfillcolor{currentfill}%
\pgfsetlinewidth{0.752812pt}%
\definecolor{currentstroke}{rgb}{0.240000,0.240000,0.240000}%
\pgfsetstrokecolor{currentstroke}%
\pgfsetdash{}{0pt}%
\pgfpathmoveto{\pgfqpoint{2.241647in}{0.613042in}}%
\pgfpathlineto{\pgfqpoint{2.463569in}{0.613042in}}%
\pgfpathlineto{\pgfqpoint{2.463569in}{0.613042in}}%
\pgfpathlineto{\pgfqpoint{2.241647in}{0.613042in}}%
\pgfpathlineto{\pgfqpoint{2.241647in}{0.613042in}}%
\pgfpathclose%
\pgfusepath{stroke,fill}%
\end{pgfscope}%
\begin{pgfscope}%
\pgfpathrectangle{\pgfqpoint{0.688192in}{0.613042in}}{\pgfqpoint{11.096108in}{4.223431in}}%
\pgfusepath{clip}%
\pgfsetbuttcap%
\pgfsetroundjoin%
\definecolor{currentfill}{rgb}{0.896070,0.594353,0.329006}%
\pgfsetfillcolor{currentfill}%
\pgfsetlinewidth{0.752812pt}%
\definecolor{currentstroke}{rgb}{0.240000,0.240000,0.240000}%
\pgfsetstrokecolor{currentstroke}%
\pgfsetdash{}{0pt}%
\pgfpathmoveto{\pgfqpoint{2.130686in}{0.613042in}}%
\pgfpathlineto{\pgfqpoint{2.574530in}{0.613042in}}%
\pgfpathlineto{\pgfqpoint{2.574530in}{0.613042in}}%
\pgfpathlineto{\pgfqpoint{2.130686in}{0.613042in}}%
\pgfpathlineto{\pgfqpoint{2.130686in}{0.613042in}}%
\pgfpathclose%
\pgfusepath{stroke,fill}%
\end{pgfscope}%
\begin{pgfscope}%
\pgfpathrectangle{\pgfqpoint{0.688192in}{0.613042in}}{\pgfqpoint{11.096108in}{4.223431in}}%
\pgfusepath{clip}%
\pgfsetbuttcap%
\pgfsetroundjoin%
\definecolor{currentfill}{rgb}{0.881863,0.505392,0.173039}%
\pgfsetfillcolor{currentfill}%
\pgfsetlinewidth{0.752812pt}%
\definecolor{currentstroke}{rgb}{0.240000,0.240000,0.240000}%
\pgfsetstrokecolor{currentstroke}%
\pgfsetdash{}{0pt}%
\pgfpathmoveto{\pgfqpoint{1.908763in}{0.613042in}}%
\pgfpathlineto{\pgfqpoint{2.796452in}{0.613042in}}%
\pgfpathlineto{\pgfqpoint{2.796452in}{0.729395in}}%
\pgfpathlineto{\pgfqpoint{1.908763in}{0.729395in}}%
\pgfpathlineto{\pgfqpoint{1.908763in}{0.613042in}}%
\pgfpathclose%
\pgfusepath{stroke,fill}%
\end{pgfscope}%
\begin{pgfscope}%
\pgfpathrectangle{\pgfqpoint{0.688192in}{0.613042in}}{\pgfqpoint{11.096108in}{4.223431in}}%
\pgfusepath{clip}%
\pgfsetbuttcap%
\pgfsetroundjoin%
\definecolor{currentfill}{rgb}{0.896070,0.594353,0.329006}%
\pgfsetfillcolor{currentfill}%
\pgfsetlinewidth{0.752812pt}%
\definecolor{currentstroke}{rgb}{0.240000,0.240000,0.240000}%
\pgfsetstrokecolor{currentstroke}%
\pgfsetdash{}{0pt}%
\pgfpathmoveto{\pgfqpoint{2.130686in}{0.729395in}}%
\pgfpathlineto{\pgfqpoint{2.574530in}{0.729395in}}%
\pgfpathlineto{\pgfqpoint{2.574530in}{0.819148in}}%
\pgfpathlineto{\pgfqpoint{2.130686in}{0.819148in}}%
\pgfpathlineto{\pgfqpoint{2.130686in}{0.729395in}}%
\pgfpathclose%
\pgfusepath{stroke,fill}%
\end{pgfscope}%
\begin{pgfscope}%
\pgfpathrectangle{\pgfqpoint{0.688192in}{0.613042in}}{\pgfqpoint{11.096108in}{4.223431in}}%
\pgfusepath{clip}%
\pgfsetbuttcap%
\pgfsetroundjoin%
\definecolor{currentfill}{rgb}{0.907545,0.666205,0.454979}%
\pgfsetfillcolor{currentfill}%
\pgfsetlinewidth{0.752812pt}%
\definecolor{currentstroke}{rgb}{0.240000,0.240000,0.240000}%
\pgfsetstrokecolor{currentstroke}%
\pgfsetdash{}{0pt}%
\pgfpathmoveto{\pgfqpoint{2.241647in}{0.819148in}}%
\pgfpathlineto{\pgfqpoint{2.463569in}{0.819148in}}%
\pgfpathlineto{\pgfqpoint{2.463569in}{0.860460in}}%
\pgfpathlineto{\pgfqpoint{2.241647in}{0.860460in}}%
\pgfpathlineto{\pgfqpoint{2.241647in}{0.819148in}}%
\pgfpathclose%
\pgfusepath{stroke,fill}%
\end{pgfscope}%
\begin{pgfscope}%
\pgfpathrectangle{\pgfqpoint{0.688192in}{0.613042in}}{\pgfqpoint{11.096108in}{4.223431in}}%
\pgfusepath{clip}%
\pgfsetbuttcap%
\pgfsetroundjoin%
\pgfsetlinewidth{1.003750pt}%
\definecolor{currentstroke}{rgb}{0.450000,0.450000,0.450000}%
\pgfsetstrokecolor{currentstroke}%
\pgfsetdash{}{0pt}%
\pgfpathmoveto{\pgfqpoint{2.352608in}{0.839363in}}%
\pgfpathcurveto{\pgfqpoint{2.361816in}{0.839363in}}{\pgfqpoint{2.370649in}{0.843022in}}{\pgfqpoint{2.377160in}{0.849533in}}%
\pgfpathcurveto{\pgfqpoint{2.383671in}{0.856045in}}{\pgfqpoint{2.387330in}{0.864877in}}{\pgfqpoint{2.387330in}{0.874086in}}%
\pgfpathcurveto{\pgfqpoint{2.387330in}{0.883294in}}{\pgfqpoint{2.383671in}{0.892126in}}{\pgfqpoint{2.377160in}{0.898638in}}%
\pgfpathcurveto{\pgfqpoint{2.370649in}{0.905149in}}{\pgfqpoint{2.361816in}{0.908808in}}{\pgfqpoint{2.352608in}{0.908808in}}%
\pgfpathcurveto{\pgfqpoint{2.343399in}{0.908808in}}{\pgfqpoint{2.334567in}{0.905149in}}{\pgfqpoint{2.328055in}{0.898638in}}%
\pgfpathcurveto{\pgfqpoint{2.321544in}{0.892126in}}{\pgfqpoint{2.317885in}{0.883294in}}{\pgfqpoint{2.317885in}{0.874086in}}%
\pgfpathcurveto{\pgfqpoint{2.317885in}{0.864877in}}{\pgfqpoint{2.321544in}{0.856045in}}{\pgfqpoint{2.328055in}{0.849533in}}%
\pgfpathcurveto{\pgfqpoint{2.334567in}{0.843022in}}{\pgfqpoint{2.343399in}{0.839363in}}{\pgfqpoint{2.352608in}{0.839363in}}%
\pgfpathlineto{\pgfqpoint{2.352608in}{0.839363in}}%
\pgfpathclose%
\pgfusepath{stroke}%
\end{pgfscope}%
\begin{pgfscope}%
\pgfpathrectangle{\pgfqpoint{0.688192in}{0.613042in}}{\pgfqpoint{11.096108in}{4.223431in}}%
\pgfusepath{clip}%
\pgfsetbuttcap%
\pgfsetroundjoin%
\pgfsetlinewidth{1.003750pt}%
\definecolor{currentstroke}{rgb}{0.450000,0.450000,0.450000}%
\pgfsetstrokecolor{currentstroke}%
\pgfsetdash{}{0pt}%
\pgfpathmoveto{\pgfqpoint{2.352608in}{0.831317in}}%
\pgfpathcurveto{\pgfqpoint{2.361816in}{0.831317in}}{\pgfqpoint{2.370649in}{0.834976in}}{\pgfqpoint{2.377160in}{0.841487in}}%
\pgfpathcurveto{\pgfqpoint{2.383671in}{0.847998in}}{\pgfqpoint{2.387330in}{0.856831in}}{\pgfqpoint{2.387330in}{0.866039in}}%
\pgfpathcurveto{\pgfqpoint{2.387330in}{0.875248in}}{\pgfqpoint{2.383671in}{0.884080in}}{\pgfqpoint{2.377160in}{0.890591in}}%
\pgfpathcurveto{\pgfqpoint{2.370649in}{0.897103in}}{\pgfqpoint{2.361816in}{0.900761in}}{\pgfqpoint{2.352608in}{0.900761in}}%
\pgfpathcurveto{\pgfqpoint{2.343399in}{0.900761in}}{\pgfqpoint{2.334567in}{0.897103in}}{\pgfqpoint{2.328055in}{0.890591in}}%
\pgfpathcurveto{\pgfqpoint{2.321544in}{0.884080in}}{\pgfqpoint{2.317885in}{0.875248in}}{\pgfqpoint{2.317885in}{0.866039in}}%
\pgfpathcurveto{\pgfqpoint{2.317885in}{0.856831in}}{\pgfqpoint{2.321544in}{0.847998in}}{\pgfqpoint{2.328055in}{0.841487in}}%
\pgfpathcurveto{\pgfqpoint{2.334567in}{0.834976in}}{\pgfqpoint{2.343399in}{0.831317in}}{\pgfqpoint{2.352608in}{0.831317in}}%
\pgfpathlineto{\pgfqpoint{2.352608in}{0.831317in}}%
\pgfpathclose%
\pgfusepath{stroke}%
\end{pgfscope}%
\begin{pgfscope}%
\pgfpathrectangle{\pgfqpoint{0.688192in}{0.613042in}}{\pgfqpoint{11.096108in}{4.223431in}}%
\pgfusepath{clip}%
\pgfsetbuttcap%
\pgfsetroundjoin%
\definecolor{currentfill}{rgb}{0.485914,0.710682,0.485881}%
\pgfsetfillcolor{currentfill}%
\pgfsetlinewidth{0.752812pt}%
\definecolor{currentstroke}{rgb}{0.240000,0.240000,0.240000}%
\pgfsetstrokecolor{currentstroke}%
\pgfsetdash{}{0pt}%
\pgfpathmoveto{\pgfqpoint{3.351257in}{1.566645in}}%
\pgfpathlineto{\pgfqpoint{3.573180in}{1.566645in}}%
\pgfpathlineto{\pgfqpoint{3.573180in}{1.616905in}}%
\pgfpathlineto{\pgfqpoint{3.351257in}{1.616905in}}%
\pgfpathlineto{\pgfqpoint{3.351257in}{1.566645in}}%
\pgfpathclose%
\pgfusepath{stroke,fill}%
\end{pgfscope}%
\begin{pgfscope}%
\pgfpathrectangle{\pgfqpoint{0.688192in}{0.613042in}}{\pgfqpoint{11.096108in}{4.223431in}}%
\pgfusepath{clip}%
\pgfsetbuttcap%
\pgfsetroundjoin%
\definecolor{currentfill}{rgb}{0.371306,0.648087,0.371288}%
\pgfsetfillcolor{currentfill}%
\pgfsetlinewidth{0.752812pt}%
\definecolor{currentstroke}{rgb}{0.240000,0.240000,0.240000}%
\pgfsetstrokecolor{currentstroke}%
\pgfsetdash{}{0pt}%
\pgfpathmoveto{\pgfqpoint{3.240296in}{1.616905in}}%
\pgfpathlineto{\pgfqpoint{3.684141in}{1.616905in}}%
\pgfpathlineto{\pgfqpoint{3.684141in}{1.722586in}}%
\pgfpathlineto{\pgfqpoint{3.240296in}{1.722586in}}%
\pgfpathlineto{\pgfqpoint{3.240296in}{1.616905in}}%
\pgfpathclose%
\pgfusepath{stroke,fill}%
\end{pgfscope}%
\begin{pgfscope}%
\pgfpathrectangle{\pgfqpoint{0.688192in}{0.613042in}}{\pgfqpoint{11.096108in}{4.223431in}}%
\pgfusepath{clip}%
\pgfsetbuttcap%
\pgfsetroundjoin%
\definecolor{currentfill}{rgb}{0.229412,0.570588,0.229412}%
\pgfsetfillcolor{currentfill}%
\pgfsetlinewidth{0.752812pt}%
\definecolor{currentstroke}{rgb}{0.240000,0.240000,0.240000}%
\pgfsetstrokecolor{currentstroke}%
\pgfsetdash{}{0pt}%
\pgfpathmoveto{\pgfqpoint{3.018374in}{1.722586in}}%
\pgfpathlineto{\pgfqpoint{3.906063in}{1.722586in}}%
\pgfpathlineto{\pgfqpoint{3.906063in}{1.868199in}}%
\pgfpathlineto{\pgfqpoint{3.018374in}{1.868199in}}%
\pgfpathlineto{\pgfqpoint{3.018374in}{1.722586in}}%
\pgfpathclose%
\pgfusepath{stroke,fill}%
\end{pgfscope}%
\begin{pgfscope}%
\pgfpathrectangle{\pgfqpoint{0.688192in}{0.613042in}}{\pgfqpoint{11.096108in}{4.223431in}}%
\pgfusepath{clip}%
\pgfsetbuttcap%
\pgfsetroundjoin%
\definecolor{currentfill}{rgb}{0.371306,0.648087,0.371288}%
\pgfsetfillcolor{currentfill}%
\pgfsetlinewidth{0.752812pt}%
\definecolor{currentstroke}{rgb}{0.240000,0.240000,0.240000}%
\pgfsetstrokecolor{currentstroke}%
\pgfsetdash{}{0pt}%
\pgfpathmoveto{\pgfqpoint{3.240296in}{1.868199in}}%
\pgfpathlineto{\pgfqpoint{3.684141in}{1.868199in}}%
\pgfpathlineto{\pgfqpoint{3.684141in}{1.870534in}}%
\pgfpathlineto{\pgfqpoint{3.240296in}{1.870534in}}%
\pgfpathlineto{\pgfqpoint{3.240296in}{1.868199in}}%
\pgfpathclose%
\pgfusepath{stroke,fill}%
\end{pgfscope}%
\begin{pgfscope}%
\pgfpathrectangle{\pgfqpoint{0.688192in}{0.613042in}}{\pgfqpoint{11.096108in}{4.223431in}}%
\pgfusepath{clip}%
\pgfsetbuttcap%
\pgfsetroundjoin%
\definecolor{currentfill}{rgb}{0.485914,0.710682,0.485881}%
\pgfsetfillcolor{currentfill}%
\pgfsetlinewidth{0.752812pt}%
\definecolor{currentstroke}{rgb}{0.240000,0.240000,0.240000}%
\pgfsetstrokecolor{currentstroke}%
\pgfsetdash{}{0pt}%
\pgfpathmoveto{\pgfqpoint{3.351257in}{1.870534in}}%
\pgfpathlineto{\pgfqpoint{3.573180in}{1.870534in}}%
\pgfpathlineto{\pgfqpoint{3.573180in}{1.921256in}}%
\pgfpathlineto{\pgfqpoint{3.351257in}{1.921256in}}%
\pgfpathlineto{\pgfqpoint{3.351257in}{1.870534in}}%
\pgfpathclose%
\pgfusepath{stroke,fill}%
\end{pgfscope}%
\begin{pgfscope}%
\pgfpathrectangle{\pgfqpoint{0.688192in}{0.613042in}}{\pgfqpoint{11.096108in}{4.223431in}}%
\pgfusepath{clip}%
\pgfsetbuttcap%
\pgfsetroundjoin%
\pgfsetlinewidth{1.003750pt}%
\definecolor{currentstroke}{rgb}{0.450000,0.450000,0.450000}%
\pgfsetstrokecolor{currentstroke}%
\pgfsetdash{}{0pt}%
\pgfpathmoveto{\pgfqpoint{3.462219in}{1.888386in}}%
\pgfpathcurveto{\pgfqpoint{3.471427in}{1.888386in}}{\pgfqpoint{3.480259in}{1.892045in}}{\pgfqpoint{3.486771in}{1.898556in}}%
\pgfpathcurveto{\pgfqpoint{3.493282in}{1.905067in}}{\pgfqpoint{3.496941in}{1.913900in}}{\pgfqpoint{3.496941in}{1.923108in}}%
\pgfpathcurveto{\pgfqpoint{3.496941in}{1.932317in}}{\pgfqpoint{3.493282in}{1.941149in}}{\pgfqpoint{3.486771in}{1.947661in}}%
\pgfpathcurveto{\pgfqpoint{3.480259in}{1.954172in}}{\pgfqpoint{3.471427in}{1.957831in}}{\pgfqpoint{3.462219in}{1.957831in}}%
\pgfpathcurveto{\pgfqpoint{3.453010in}{1.957831in}}{\pgfqpoint{3.444178in}{1.954172in}}{\pgfqpoint{3.437666in}{1.947661in}}%
\pgfpathcurveto{\pgfqpoint{3.431155in}{1.941149in}}{\pgfqpoint{3.427496in}{1.932317in}}{\pgfqpoint{3.427496in}{1.923108in}}%
\pgfpathcurveto{\pgfqpoint{3.427496in}{1.913900in}}{\pgfqpoint{3.431155in}{1.905067in}}{\pgfqpoint{3.437666in}{1.898556in}}%
\pgfpathcurveto{\pgfqpoint{3.444178in}{1.892045in}}{\pgfqpoint{3.453010in}{1.888386in}}{\pgfqpoint{3.462219in}{1.888386in}}%
\pgfpathlineto{\pgfqpoint{3.462219in}{1.888386in}}%
\pgfpathclose%
\pgfusepath{stroke}%
\end{pgfscope}%
\begin{pgfscope}%
\pgfpathrectangle{\pgfqpoint{0.688192in}{0.613042in}}{\pgfqpoint{11.096108in}{4.223431in}}%
\pgfusepath{clip}%
\pgfsetbuttcap%
\pgfsetroundjoin%
\pgfsetlinewidth{1.003750pt}%
\definecolor{currentstroke}{rgb}{0.450000,0.450000,0.450000}%
\pgfsetstrokecolor{currentstroke}%
\pgfsetdash{}{0pt}%
\pgfpathmoveto{\pgfqpoint{3.462219in}{1.494702in}}%
\pgfpathcurveto{\pgfqpoint{3.471427in}{1.494702in}}{\pgfqpoint{3.480259in}{1.498360in}}{\pgfqpoint{3.486771in}{1.504871in}}%
\pgfpathcurveto{\pgfqpoint{3.493282in}{1.511383in}}{\pgfqpoint{3.496941in}{1.520215in}}{\pgfqpoint{3.496941in}{1.529424in}}%
\pgfpathcurveto{\pgfqpoint{3.496941in}{1.538632in}}{\pgfqpoint{3.493282in}{1.547465in}}{\pgfqpoint{3.486771in}{1.553976in}}%
\pgfpathcurveto{\pgfqpoint{3.480259in}{1.560487in}}{\pgfqpoint{3.471427in}{1.564146in}}{\pgfqpoint{3.462219in}{1.564146in}}%
\pgfpathcurveto{\pgfqpoint{3.453010in}{1.564146in}}{\pgfqpoint{3.444178in}{1.560487in}}{\pgfqpoint{3.437666in}{1.553976in}}%
\pgfpathcurveto{\pgfqpoint{3.431155in}{1.547465in}}{\pgfqpoint{3.427496in}{1.538632in}}{\pgfqpoint{3.427496in}{1.529424in}}%
\pgfpathcurveto{\pgfqpoint{3.427496in}{1.520215in}}{\pgfqpoint{3.431155in}{1.511383in}}{\pgfqpoint{3.437666in}{1.504871in}}%
\pgfpathcurveto{\pgfqpoint{3.444178in}{1.498360in}}{\pgfqpoint{3.453010in}{1.494702in}}{\pgfqpoint{3.462219in}{1.494702in}}%
\pgfpathlineto{\pgfqpoint{3.462219in}{1.494702in}}%
\pgfpathclose%
\pgfusepath{stroke}%
\end{pgfscope}%
\begin{pgfscope}%
\pgfpathrectangle{\pgfqpoint{0.688192in}{0.613042in}}{\pgfqpoint{11.096108in}{4.223431in}}%
\pgfusepath{clip}%
\pgfsetbuttcap%
\pgfsetroundjoin%
\pgfsetlinewidth{1.003750pt}%
\definecolor{currentstroke}{rgb}{0.450000,0.450000,0.450000}%
\pgfsetstrokecolor{currentstroke}%
\pgfsetdash{}{0pt}%
\pgfpathmoveto{\pgfqpoint{3.462219in}{1.494702in}}%
\pgfpathcurveto{\pgfqpoint{3.471427in}{1.494702in}}{\pgfqpoint{3.480259in}{1.498360in}}{\pgfqpoint{3.486771in}{1.504871in}}%
\pgfpathcurveto{\pgfqpoint{3.493282in}{1.511383in}}{\pgfqpoint{3.496941in}{1.520215in}}{\pgfqpoint{3.496941in}{1.529424in}}%
\pgfpathcurveto{\pgfqpoint{3.496941in}{1.538632in}}{\pgfqpoint{3.493282in}{1.547465in}}{\pgfqpoint{3.486771in}{1.553976in}}%
\pgfpathcurveto{\pgfqpoint{3.480259in}{1.560487in}}{\pgfqpoint{3.471427in}{1.564146in}}{\pgfqpoint{3.462219in}{1.564146in}}%
\pgfpathcurveto{\pgfqpoint{3.453010in}{1.564146in}}{\pgfqpoint{3.444178in}{1.560487in}}{\pgfqpoint{3.437666in}{1.553976in}}%
\pgfpathcurveto{\pgfqpoint{3.431155in}{1.547465in}}{\pgfqpoint{3.427496in}{1.538632in}}{\pgfqpoint{3.427496in}{1.529424in}}%
\pgfpathcurveto{\pgfqpoint{3.427496in}{1.520215in}}{\pgfqpoint{3.431155in}{1.511383in}}{\pgfqpoint{3.437666in}{1.504871in}}%
\pgfpathcurveto{\pgfqpoint{3.444178in}{1.498360in}}{\pgfqpoint{3.453010in}{1.494702in}}{\pgfqpoint{3.462219in}{1.494702in}}%
\pgfpathlineto{\pgfqpoint{3.462219in}{1.494702in}}%
\pgfpathclose%
\pgfusepath{stroke}%
\end{pgfscope}%
\begin{pgfscope}%
\pgfpathrectangle{\pgfqpoint{0.688192in}{0.613042in}}{\pgfqpoint{11.096108in}{4.223431in}}%
\pgfusepath{clip}%
\pgfsetbuttcap%
\pgfsetroundjoin%
\pgfsetlinewidth{1.003750pt}%
\definecolor{currentstroke}{rgb}{0.450000,0.450000,0.450000}%
\pgfsetstrokecolor{currentstroke}%
\pgfsetdash{}{0pt}%
\pgfpathmoveto{\pgfqpoint{3.462219in}{1.891625in}}%
\pgfpathcurveto{\pgfqpoint{3.471427in}{1.891625in}}{\pgfqpoint{3.480259in}{1.895284in}}{\pgfqpoint{3.486771in}{1.901795in}}%
\pgfpathcurveto{\pgfqpoint{3.493282in}{1.908306in}}{\pgfqpoint{3.496941in}{1.917139in}}{\pgfqpoint{3.496941in}{1.926347in}}%
\pgfpathcurveto{\pgfqpoint{3.496941in}{1.935556in}}{\pgfqpoint{3.493282in}{1.944388in}}{\pgfqpoint{3.486771in}{1.950900in}}%
\pgfpathcurveto{\pgfqpoint{3.480259in}{1.957411in}}{\pgfqpoint{3.471427in}{1.961070in}}{\pgfqpoint{3.462219in}{1.961070in}}%
\pgfpathcurveto{\pgfqpoint{3.453010in}{1.961070in}}{\pgfqpoint{3.444178in}{1.957411in}}{\pgfqpoint{3.437666in}{1.950900in}}%
\pgfpathcurveto{\pgfqpoint{3.431155in}{1.944388in}}{\pgfqpoint{3.427496in}{1.935556in}}{\pgfqpoint{3.427496in}{1.926347in}}%
\pgfpathcurveto{\pgfqpoint{3.427496in}{1.917139in}}{\pgfqpoint{3.431155in}{1.908306in}}{\pgfqpoint{3.437666in}{1.901795in}}%
\pgfpathcurveto{\pgfqpoint{3.444178in}{1.895284in}}{\pgfqpoint{3.453010in}{1.891625in}}{\pgfqpoint{3.462219in}{1.891625in}}%
\pgfpathlineto{\pgfqpoint{3.462219in}{1.891625in}}%
\pgfpathclose%
\pgfusepath{stroke}%
\end{pgfscope}%
\begin{pgfscope}%
\pgfpathrectangle{\pgfqpoint{0.688192in}{0.613042in}}{\pgfqpoint{11.096108in}{4.223431in}}%
\pgfusepath{clip}%
\pgfsetbuttcap%
\pgfsetroundjoin%
\definecolor{currentfill}{rgb}{0.825642,0.497939,0.499757}%
\pgfsetfillcolor{currentfill}%
\pgfsetlinewidth{0.752812pt}%
\definecolor{currentstroke}{rgb}{0.240000,0.240000,0.240000}%
\pgfsetstrokecolor{currentstroke}%
\pgfsetdash{}{0pt}%
\pgfpathmoveto{\pgfqpoint{4.460868in}{0.613042in}}%
\pgfpathlineto{\pgfqpoint{4.682790in}{0.613042in}}%
\pgfpathlineto{\pgfqpoint{4.682790in}{0.613042in}}%
\pgfpathlineto{\pgfqpoint{4.460868in}{0.613042in}}%
\pgfpathlineto{\pgfqpoint{4.460868in}{0.613042in}}%
\pgfpathclose%
\pgfusepath{stroke,fill}%
\end{pgfscope}%
\begin{pgfscope}%
\pgfpathrectangle{\pgfqpoint{0.688192in}{0.613042in}}{\pgfqpoint{11.096108in}{4.223431in}}%
\pgfusepath{clip}%
\pgfsetbuttcap%
\pgfsetroundjoin%
\definecolor{currentfill}{rgb}{0.793378,0.382120,0.384440}%
\pgfsetfillcolor{currentfill}%
\pgfsetlinewidth{0.752812pt}%
\definecolor{currentstroke}{rgb}{0.240000,0.240000,0.240000}%
\pgfsetstrokecolor{currentstroke}%
\pgfsetdash{}{0pt}%
\pgfpathmoveto{\pgfqpoint{4.349907in}{0.613042in}}%
\pgfpathlineto{\pgfqpoint{4.793751in}{0.613042in}}%
\pgfpathlineto{\pgfqpoint{4.793751in}{0.635568in}}%
\pgfpathlineto{\pgfqpoint{4.349907in}{0.635568in}}%
\pgfpathlineto{\pgfqpoint{4.349907in}{0.613042in}}%
\pgfpathclose%
\pgfusepath{stroke,fill}%
\end{pgfscope}%
\begin{pgfscope}%
\pgfpathrectangle{\pgfqpoint{0.688192in}{0.613042in}}{\pgfqpoint{11.096108in}{4.223431in}}%
\pgfusepath{clip}%
\pgfsetbuttcap%
\pgfsetroundjoin%
\definecolor{currentfill}{rgb}{0.753431,0.238725,0.241667}%
\pgfsetfillcolor{currentfill}%
\pgfsetlinewidth{0.752812pt}%
\definecolor{currentstroke}{rgb}{0.240000,0.240000,0.240000}%
\pgfsetstrokecolor{currentstroke}%
\pgfsetdash{}{0pt}%
\pgfpathmoveto{\pgfqpoint{4.127985in}{0.635568in}}%
\pgfpathlineto{\pgfqpoint{5.015674in}{0.635568in}}%
\pgfpathlineto{\pgfqpoint{5.015674in}{0.937558in}}%
\pgfpathlineto{\pgfqpoint{4.127985in}{0.937558in}}%
\pgfpathlineto{\pgfqpoint{4.127985in}{0.635568in}}%
\pgfpathclose%
\pgfusepath{stroke,fill}%
\end{pgfscope}%
\begin{pgfscope}%
\pgfpathrectangle{\pgfqpoint{0.688192in}{0.613042in}}{\pgfqpoint{11.096108in}{4.223431in}}%
\pgfusepath{clip}%
\pgfsetbuttcap%
\pgfsetroundjoin%
\definecolor{currentfill}{rgb}{0.793378,0.382120,0.384440}%
\pgfsetfillcolor{currentfill}%
\pgfsetlinewidth{0.752812pt}%
\definecolor{currentstroke}{rgb}{0.240000,0.240000,0.240000}%
\pgfsetstrokecolor{currentstroke}%
\pgfsetdash{}{0pt}%
\pgfpathmoveto{\pgfqpoint{4.349907in}{0.937558in}}%
\pgfpathlineto{\pgfqpoint{4.793751in}{0.937558in}}%
\pgfpathlineto{\pgfqpoint{4.793751in}{1.103845in}}%
\pgfpathlineto{\pgfqpoint{4.349907in}{1.103845in}}%
\pgfpathlineto{\pgfqpoint{4.349907in}{0.937558in}}%
\pgfpathclose%
\pgfusepath{stroke,fill}%
\end{pgfscope}%
\begin{pgfscope}%
\pgfpathrectangle{\pgfqpoint{0.688192in}{0.613042in}}{\pgfqpoint{11.096108in}{4.223431in}}%
\pgfusepath{clip}%
\pgfsetbuttcap%
\pgfsetroundjoin%
\definecolor{currentfill}{rgb}{0.825642,0.497939,0.499757}%
\pgfsetfillcolor{currentfill}%
\pgfsetlinewidth{0.752812pt}%
\definecolor{currentstroke}{rgb}{0.240000,0.240000,0.240000}%
\pgfsetstrokecolor{currentstroke}%
\pgfsetdash{}{0pt}%
\pgfpathmoveto{\pgfqpoint{4.460868in}{1.103845in}}%
\pgfpathlineto{\pgfqpoint{4.682790in}{1.103845in}}%
\pgfpathlineto{\pgfqpoint{4.682790in}{1.170317in}}%
\pgfpathlineto{\pgfqpoint{4.460868in}{1.170317in}}%
\pgfpathlineto{\pgfqpoint{4.460868in}{1.103845in}}%
\pgfpathclose%
\pgfusepath{stroke,fill}%
\end{pgfscope}%
\begin{pgfscope}%
\pgfpathrectangle{\pgfqpoint{0.688192in}{0.613042in}}{\pgfqpoint{11.096108in}{4.223431in}}%
\pgfusepath{clip}%
\pgfsetbuttcap%
\pgfsetroundjoin%
\pgfsetlinewidth{1.003750pt}%
\definecolor{currentstroke}{rgb}{0.450000,0.450000,0.450000}%
\pgfsetstrokecolor{currentstroke}%
\pgfsetdash{}{0pt}%
\pgfpathmoveto{\pgfqpoint{4.571829in}{1.183355in}}%
\pgfpathcurveto{\pgfqpoint{4.581038in}{1.183355in}}{\pgfqpoint{4.589870in}{1.187013in}}{\pgfqpoint{4.596382in}{1.193525in}}%
\pgfpathcurveto{\pgfqpoint{4.602893in}{1.200036in}}{\pgfqpoint{4.606552in}{1.208869in}}{\pgfqpoint{4.606552in}{1.218077in}}%
\pgfpathcurveto{\pgfqpoint{4.606552in}{1.227286in}}{\pgfqpoint{4.602893in}{1.236118in}}{\pgfqpoint{4.596382in}{1.242629in}}%
\pgfpathcurveto{\pgfqpoint{4.589870in}{1.249141in}}{\pgfqpoint{4.581038in}{1.252799in}}{\pgfqpoint{4.571829in}{1.252799in}}%
\pgfpathcurveto{\pgfqpoint{4.562621in}{1.252799in}}{\pgfqpoint{4.553788in}{1.249141in}}{\pgfqpoint{4.547277in}{1.242629in}}%
\pgfpathcurveto{\pgfqpoint{4.540766in}{1.236118in}}{\pgfqpoint{4.537107in}{1.227286in}}{\pgfqpoint{4.537107in}{1.218077in}}%
\pgfpathcurveto{\pgfqpoint{4.537107in}{1.208869in}}{\pgfqpoint{4.540766in}{1.200036in}}{\pgfqpoint{4.547277in}{1.193525in}}%
\pgfpathcurveto{\pgfqpoint{4.553788in}{1.187013in}}{\pgfqpoint{4.562621in}{1.183355in}}{\pgfqpoint{4.571829in}{1.183355in}}%
\pgfpathlineto{\pgfqpoint{4.571829in}{1.183355in}}%
\pgfpathclose%
\pgfusepath{stroke}%
\end{pgfscope}%
\begin{pgfscope}%
\pgfpathrectangle{\pgfqpoint{0.688192in}{0.613042in}}{\pgfqpoint{11.096108in}{4.223431in}}%
\pgfusepath{clip}%
\pgfsetbuttcap%
\pgfsetroundjoin%
\pgfsetlinewidth{1.003750pt}%
\definecolor{currentstroke}{rgb}{0.450000,0.450000,0.450000}%
\pgfsetstrokecolor{currentstroke}%
\pgfsetdash{}{0pt}%
\pgfpathmoveto{\pgfqpoint{4.571829in}{0.578320in}}%
\pgfpathcurveto{\pgfqpoint{4.581038in}{0.578320in}}{\pgfqpoint{4.589870in}{0.581978in}}{\pgfqpoint{4.596382in}{0.588490in}}%
\pgfpathcurveto{\pgfqpoint{4.602893in}{0.595001in}}{\pgfqpoint{4.606552in}{0.603833in}}{\pgfqpoint{4.606552in}{0.613042in}}%
\pgfpathcurveto{\pgfqpoint{4.606552in}{0.622250in}}{\pgfqpoint{4.602893in}{0.631083in}}{\pgfqpoint{4.596382in}{0.637594in}}%
\pgfpathcurveto{\pgfqpoint{4.589870in}{0.644106in}}{\pgfqpoint{4.581038in}{0.647764in}}{\pgfqpoint{4.571829in}{0.647764in}}%
\pgfpathcurveto{\pgfqpoint{4.562621in}{0.647764in}}{\pgfqpoint{4.553788in}{0.644106in}}{\pgfqpoint{4.547277in}{0.637594in}}%
\pgfpathcurveto{\pgfqpoint{4.540766in}{0.631083in}}{\pgfqpoint{4.537107in}{0.622250in}}{\pgfqpoint{4.537107in}{0.613042in}}%
\pgfpathcurveto{\pgfqpoint{4.537107in}{0.603833in}}{\pgfqpoint{4.540766in}{0.595001in}}{\pgfqpoint{4.547277in}{0.588490in}}%
\pgfpathcurveto{\pgfqpoint{4.553788in}{0.581978in}}{\pgfqpoint{4.562621in}{0.578320in}}{\pgfqpoint{4.571829in}{0.578320in}}%
\pgfusepath{stroke}%
\end{pgfscope}%
\begin{pgfscope}%
\pgfpathrectangle{\pgfqpoint{0.688192in}{0.613042in}}{\pgfqpoint{11.096108in}{4.223431in}}%
\pgfusepath{clip}%
\pgfsetbuttcap%
\pgfsetroundjoin%
\pgfsetlinewidth{1.003750pt}%
\definecolor{currentstroke}{rgb}{0.450000,0.450000,0.450000}%
\pgfsetstrokecolor{currentstroke}%
\pgfsetdash{}{0pt}%
\pgfpathmoveto{\pgfqpoint{4.571829in}{1.184124in}}%
\pgfpathcurveto{\pgfqpoint{4.581038in}{1.184124in}}{\pgfqpoint{4.589870in}{1.187782in}}{\pgfqpoint{4.596382in}{1.194294in}}%
\pgfpathcurveto{\pgfqpoint{4.602893in}{1.200805in}}{\pgfqpoint{4.606552in}{1.209638in}}{\pgfqpoint{4.606552in}{1.218846in}}%
\pgfpathcurveto{\pgfqpoint{4.606552in}{1.228054in}}{\pgfqpoint{4.602893in}{1.236887in}}{\pgfqpoint{4.596382in}{1.243398in}}%
\pgfpathcurveto{\pgfqpoint{4.589870in}{1.249910in}}{\pgfqpoint{4.581038in}{1.253568in}}{\pgfqpoint{4.571829in}{1.253568in}}%
\pgfpathcurveto{\pgfqpoint{4.562621in}{1.253568in}}{\pgfqpoint{4.553788in}{1.249910in}}{\pgfqpoint{4.547277in}{1.243398in}}%
\pgfpathcurveto{\pgfqpoint{4.540766in}{1.236887in}}{\pgfqpoint{4.537107in}{1.228054in}}{\pgfqpoint{4.537107in}{1.218846in}}%
\pgfpathcurveto{\pgfqpoint{4.537107in}{1.209638in}}{\pgfqpoint{4.540766in}{1.200805in}}{\pgfqpoint{4.547277in}{1.194294in}}%
\pgfpathcurveto{\pgfqpoint{4.553788in}{1.187782in}}{\pgfqpoint{4.562621in}{1.184124in}}{\pgfqpoint{4.571829in}{1.184124in}}%
\pgfpathlineto{\pgfqpoint{4.571829in}{1.184124in}}%
\pgfpathclose%
\pgfusepath{stroke}%
\end{pgfscope}%
\begin{pgfscope}%
\pgfpathrectangle{\pgfqpoint{0.688192in}{0.613042in}}{\pgfqpoint{11.096108in}{4.223431in}}%
\pgfusepath{clip}%
\pgfsetbuttcap%
\pgfsetroundjoin%
\definecolor{currentfill}{rgb}{0.713429,0.629184,0.790364}%
\pgfsetfillcolor{currentfill}%
\pgfsetlinewidth{0.752812pt}%
\definecolor{currentstroke}{rgb}{0.240000,0.240000,0.240000}%
\pgfsetstrokecolor{currentstroke}%
\pgfsetdash{}{0pt}%
\pgfpathmoveto{\pgfqpoint{5.570479in}{0.615323in}}%
\pgfpathlineto{\pgfqpoint{5.792401in}{0.615323in}}%
\pgfpathlineto{\pgfqpoint{5.792401in}{0.666373in}}%
\pgfpathlineto{\pgfqpoint{5.570479in}{0.666373in}}%
\pgfpathlineto{\pgfqpoint{5.570479in}{0.615323in}}%
\pgfpathclose%
\pgfusepath{stroke,fill}%
\end{pgfscope}%
\begin{pgfscope}%
\pgfpathrectangle{\pgfqpoint{0.688192in}{0.613042in}}{\pgfqpoint{11.096108in}{4.223431in}}%
\pgfusepath{clip}%
\pgfsetbuttcap%
\pgfsetroundjoin%
\definecolor{currentfill}{rgb}{0.653111,0.547371,0.749551}%
\pgfsetfillcolor{currentfill}%
\pgfsetlinewidth{0.752812pt}%
\definecolor{currentstroke}{rgb}{0.240000,0.240000,0.240000}%
\pgfsetstrokecolor{currentstroke}%
\pgfsetdash{}{0pt}%
\pgfpathmoveto{\pgfqpoint{5.459518in}{0.666373in}}%
\pgfpathlineto{\pgfqpoint{5.903362in}{0.666373in}}%
\pgfpathlineto{\pgfqpoint{5.903362in}{0.707375in}}%
\pgfpathlineto{\pgfqpoint{5.459518in}{0.707375in}}%
\pgfpathlineto{\pgfqpoint{5.459518in}{0.666373in}}%
\pgfpathclose%
\pgfusepath{stroke,fill}%
\end{pgfscope}%
\begin{pgfscope}%
\pgfpathrectangle{\pgfqpoint{0.688192in}{0.613042in}}{\pgfqpoint{11.096108in}{4.223431in}}%
\pgfusepath{clip}%
\pgfsetbuttcap%
\pgfsetroundjoin%
\definecolor{currentfill}{rgb}{0.578431,0.446078,0.699020}%
\pgfsetfillcolor{currentfill}%
\pgfsetlinewidth{0.752812pt}%
\definecolor{currentstroke}{rgb}{0.240000,0.240000,0.240000}%
\pgfsetstrokecolor{currentstroke}%
\pgfsetdash{}{0pt}%
\pgfpathmoveto{\pgfqpoint{5.237596in}{0.707375in}}%
\pgfpathlineto{\pgfqpoint{6.125284in}{0.707375in}}%
\pgfpathlineto{\pgfqpoint{6.125284in}{0.935698in}}%
\pgfpathlineto{\pgfqpoint{5.237596in}{0.935698in}}%
\pgfpathlineto{\pgfqpoint{5.237596in}{0.707375in}}%
\pgfpathclose%
\pgfusepath{stroke,fill}%
\end{pgfscope}%
\begin{pgfscope}%
\pgfpathrectangle{\pgfqpoint{0.688192in}{0.613042in}}{\pgfqpoint{11.096108in}{4.223431in}}%
\pgfusepath{clip}%
\pgfsetbuttcap%
\pgfsetroundjoin%
\definecolor{currentfill}{rgb}{0.653111,0.547371,0.749551}%
\pgfsetfillcolor{currentfill}%
\pgfsetlinewidth{0.752812pt}%
\definecolor{currentstroke}{rgb}{0.240000,0.240000,0.240000}%
\pgfsetstrokecolor{currentstroke}%
\pgfsetdash{}{0pt}%
\pgfpathmoveto{\pgfqpoint{5.459518in}{0.935698in}}%
\pgfpathlineto{\pgfqpoint{5.903362in}{0.935698in}}%
\pgfpathlineto{\pgfqpoint{5.903362in}{1.095664in}}%
\pgfpathlineto{\pgfqpoint{5.459518in}{1.095664in}}%
\pgfpathlineto{\pgfqpoint{5.459518in}{0.935698in}}%
\pgfpathclose%
\pgfusepath{stroke,fill}%
\end{pgfscope}%
\begin{pgfscope}%
\pgfpathrectangle{\pgfqpoint{0.688192in}{0.613042in}}{\pgfqpoint{11.096108in}{4.223431in}}%
\pgfusepath{clip}%
\pgfsetbuttcap%
\pgfsetroundjoin%
\definecolor{currentfill}{rgb}{0.713429,0.629184,0.790364}%
\pgfsetfillcolor{currentfill}%
\pgfsetlinewidth{0.752812pt}%
\definecolor{currentstroke}{rgb}{0.240000,0.240000,0.240000}%
\pgfsetstrokecolor{currentstroke}%
\pgfsetdash{}{0pt}%
\pgfpathmoveto{\pgfqpoint{5.570479in}{1.095664in}}%
\pgfpathlineto{\pgfqpoint{5.792401in}{1.095664in}}%
\pgfpathlineto{\pgfqpoint{5.792401in}{1.437740in}}%
\pgfpathlineto{\pgfqpoint{5.570479in}{1.437740in}}%
\pgfpathlineto{\pgfqpoint{5.570479in}{1.095664in}}%
\pgfpathclose%
\pgfusepath{stroke,fill}%
\end{pgfscope}%
\begin{pgfscope}%
\pgfpathrectangle{\pgfqpoint{0.688192in}{0.613042in}}{\pgfqpoint{11.096108in}{4.223431in}}%
\pgfusepath{clip}%
\pgfsetbuttcap%
\pgfsetroundjoin%
\pgfsetlinewidth{1.003750pt}%
\definecolor{currentstroke}{rgb}{0.450000,0.450000,0.450000}%
\pgfsetstrokecolor{currentstroke}%
\pgfsetdash{}{0pt}%
\pgfpathmoveto{\pgfqpoint{5.681440in}{0.578320in}}%
\pgfpathcurveto{\pgfqpoint{5.690649in}{0.578320in}}{\pgfqpoint{5.699481in}{0.581978in}}{\pgfqpoint{5.705992in}{0.588490in}}%
\pgfpathcurveto{\pgfqpoint{5.712504in}{0.595001in}}{\pgfqpoint{5.716162in}{0.603833in}}{\pgfqpoint{5.716162in}{0.613042in}}%
\pgfpathcurveto{\pgfqpoint{5.716162in}{0.622250in}}{\pgfqpoint{5.712504in}{0.631083in}}{\pgfqpoint{5.705992in}{0.637594in}}%
\pgfpathcurveto{\pgfqpoint{5.699481in}{0.644106in}}{\pgfqpoint{5.690649in}{0.647764in}}{\pgfqpoint{5.681440in}{0.647764in}}%
\pgfpathcurveto{\pgfqpoint{5.672232in}{0.647764in}}{\pgfqpoint{5.663399in}{0.644106in}}{\pgfqpoint{5.656888in}{0.637594in}}%
\pgfpathcurveto{\pgfqpoint{5.650376in}{0.631083in}}{\pgfqpoint{5.646718in}{0.622250in}}{\pgfqpoint{5.646718in}{0.613042in}}%
\pgfpathcurveto{\pgfqpoint{5.646718in}{0.603833in}}{\pgfqpoint{5.650376in}{0.595001in}}{\pgfqpoint{5.656888in}{0.588490in}}%
\pgfpathcurveto{\pgfqpoint{5.663399in}{0.581978in}}{\pgfqpoint{5.672232in}{0.578320in}}{\pgfqpoint{5.681440in}{0.578320in}}%
\pgfusepath{stroke}%
\end{pgfscope}%
\begin{pgfscope}%
\pgfpathrectangle{\pgfqpoint{0.688192in}{0.613042in}}{\pgfqpoint{11.096108in}{4.223431in}}%
\pgfusepath{clip}%
\pgfsetbuttcap%
\pgfsetroundjoin%
\pgfsetlinewidth{1.003750pt}%
\definecolor{currentstroke}{rgb}{0.450000,0.450000,0.450000}%
\pgfsetstrokecolor{currentstroke}%
\pgfsetdash{}{0pt}%
\pgfpathmoveto{\pgfqpoint{5.681440in}{2.234609in}}%
\pgfpathcurveto{\pgfqpoint{5.690649in}{2.234609in}}{\pgfqpoint{5.699481in}{2.238267in}}{\pgfqpoint{5.705992in}{2.244779in}}%
\pgfpathcurveto{\pgfqpoint{5.712504in}{2.251290in}}{\pgfqpoint{5.716162in}{2.260122in}}{\pgfqpoint{5.716162in}{2.269331in}}%
\pgfpathcurveto{\pgfqpoint{5.716162in}{2.278539in}}{\pgfqpoint{5.712504in}{2.287372in}}{\pgfqpoint{5.705992in}{2.293883in}}%
\pgfpathcurveto{\pgfqpoint{5.699481in}{2.300395in}}{\pgfqpoint{5.690649in}{2.304053in}}{\pgfqpoint{5.681440in}{2.304053in}}%
\pgfpathcurveto{\pgfqpoint{5.672232in}{2.304053in}}{\pgfqpoint{5.663399in}{2.300395in}}{\pgfqpoint{5.656888in}{2.293883in}}%
\pgfpathcurveto{\pgfqpoint{5.650376in}{2.287372in}}{\pgfqpoint{5.646718in}{2.278539in}}{\pgfqpoint{5.646718in}{2.269331in}}%
\pgfpathcurveto{\pgfqpoint{5.646718in}{2.260122in}}{\pgfqpoint{5.650376in}{2.251290in}}{\pgfqpoint{5.656888in}{2.244779in}}%
\pgfpathcurveto{\pgfqpoint{5.663399in}{2.238267in}}{\pgfqpoint{5.672232in}{2.234609in}}{\pgfqpoint{5.681440in}{2.234609in}}%
\pgfpathlineto{\pgfqpoint{5.681440in}{2.234609in}}%
\pgfpathclose%
\pgfusepath{stroke}%
\end{pgfscope}%
\begin{pgfscope}%
\pgfpathrectangle{\pgfqpoint{0.688192in}{0.613042in}}{\pgfqpoint{11.096108in}{4.223431in}}%
\pgfusepath{clip}%
\pgfsetbuttcap%
\pgfsetroundjoin%
\pgfsetlinewidth{1.003750pt}%
\definecolor{currentstroke}{rgb}{0.450000,0.450000,0.450000}%
\pgfsetstrokecolor{currentstroke}%
\pgfsetdash{}{0pt}%
\pgfpathmoveto{\pgfqpoint{5.681440in}{2.234609in}}%
\pgfpathcurveto{\pgfqpoint{5.690649in}{2.234609in}}{\pgfqpoint{5.699481in}{2.238267in}}{\pgfqpoint{5.705992in}{2.244779in}}%
\pgfpathcurveto{\pgfqpoint{5.712504in}{2.251290in}}{\pgfqpoint{5.716162in}{2.260122in}}{\pgfqpoint{5.716162in}{2.269331in}}%
\pgfpathcurveto{\pgfqpoint{5.716162in}{2.278539in}}{\pgfqpoint{5.712504in}{2.287372in}}{\pgfqpoint{5.705992in}{2.293883in}}%
\pgfpathcurveto{\pgfqpoint{5.699481in}{2.300395in}}{\pgfqpoint{5.690649in}{2.304053in}}{\pgfqpoint{5.681440in}{2.304053in}}%
\pgfpathcurveto{\pgfqpoint{5.672232in}{2.304053in}}{\pgfqpoint{5.663399in}{2.300395in}}{\pgfqpoint{5.656888in}{2.293883in}}%
\pgfpathcurveto{\pgfqpoint{5.650376in}{2.287372in}}{\pgfqpoint{5.646718in}{2.278539in}}{\pgfqpoint{5.646718in}{2.269331in}}%
\pgfpathcurveto{\pgfqpoint{5.646718in}{2.260122in}}{\pgfqpoint{5.650376in}{2.251290in}}{\pgfqpoint{5.656888in}{2.244779in}}%
\pgfpathcurveto{\pgfqpoint{5.663399in}{2.238267in}}{\pgfqpoint{5.672232in}{2.234609in}}{\pgfqpoint{5.681440in}{2.234609in}}%
\pgfpathlineto{\pgfqpoint{5.681440in}{2.234609in}}%
\pgfpathclose%
\pgfusepath{stroke}%
\end{pgfscope}%
\begin{pgfscope}%
\pgfpathrectangle{\pgfqpoint{0.688192in}{0.613042in}}{\pgfqpoint{11.096108in}{4.223431in}}%
\pgfusepath{clip}%
\pgfsetbuttcap%
\pgfsetroundjoin%
\pgfsetlinewidth{1.003750pt}%
\definecolor{currentstroke}{rgb}{0.450000,0.450000,0.450000}%
\pgfsetstrokecolor{currentstroke}%
\pgfsetdash{}{0pt}%
\pgfpathmoveto{\pgfqpoint{5.681440in}{0.578320in}}%
\pgfpathcurveto{\pgfqpoint{5.690649in}{0.578320in}}{\pgfqpoint{5.699481in}{0.581978in}}{\pgfqpoint{5.705992in}{0.588490in}}%
\pgfpathcurveto{\pgfqpoint{5.712504in}{0.595001in}}{\pgfqpoint{5.716162in}{0.603833in}}{\pgfqpoint{5.716162in}{0.613042in}}%
\pgfpathcurveto{\pgfqpoint{5.716162in}{0.622250in}}{\pgfqpoint{5.712504in}{0.631083in}}{\pgfqpoint{5.705992in}{0.637594in}}%
\pgfpathcurveto{\pgfqpoint{5.699481in}{0.644106in}}{\pgfqpoint{5.690649in}{0.647764in}}{\pgfqpoint{5.681440in}{0.647764in}}%
\pgfpathcurveto{\pgfqpoint{5.672232in}{0.647764in}}{\pgfqpoint{5.663399in}{0.644106in}}{\pgfqpoint{5.656888in}{0.637594in}}%
\pgfpathcurveto{\pgfqpoint{5.650376in}{0.631083in}}{\pgfqpoint{5.646718in}{0.622250in}}{\pgfqpoint{5.646718in}{0.613042in}}%
\pgfpathcurveto{\pgfqpoint{5.646718in}{0.603833in}}{\pgfqpoint{5.650376in}{0.595001in}}{\pgfqpoint{5.656888in}{0.588490in}}%
\pgfpathcurveto{\pgfqpoint{5.663399in}{0.581978in}}{\pgfqpoint{5.672232in}{0.578320in}}{\pgfqpoint{5.681440in}{0.578320in}}%
\pgfusepath{stroke}%
\end{pgfscope}%
\begin{pgfscope}%
\pgfpathrectangle{\pgfqpoint{0.688192in}{0.613042in}}{\pgfqpoint{11.096108in}{4.223431in}}%
\pgfusepath{clip}%
\pgfsetbuttcap%
\pgfsetroundjoin%
\definecolor{currentfill}{rgb}{0.675311,0.573788,0.553126}%
\pgfsetfillcolor{currentfill}%
\pgfsetlinewidth{0.752812pt}%
\definecolor{currentstroke}{rgb}{0.240000,0.240000,0.240000}%
\pgfsetstrokecolor{currentstroke}%
\pgfsetdash{}{0pt}%
\pgfpathmoveto{\pgfqpoint{6.680090in}{0.613042in}}%
\pgfpathlineto{\pgfqpoint{6.902012in}{0.613042in}}%
\pgfpathlineto{\pgfqpoint{6.902012in}{0.613042in}}%
\pgfpathlineto{\pgfqpoint{6.680090in}{0.613042in}}%
\pgfpathlineto{\pgfqpoint{6.680090in}{0.613042in}}%
\pgfpathclose%
\pgfusepath{stroke,fill}%
\end{pgfscope}%
\begin{pgfscope}%
\pgfpathrectangle{\pgfqpoint{0.688192in}{0.613042in}}{\pgfqpoint{11.096108in}{4.223431in}}%
\pgfusepath{clip}%
\pgfsetbuttcap%
\pgfsetroundjoin%
\definecolor{currentfill}{rgb}{0.604646,0.477521,0.451635}%
\pgfsetfillcolor{currentfill}%
\pgfsetlinewidth{0.752812pt}%
\definecolor{currentstroke}{rgb}{0.240000,0.240000,0.240000}%
\pgfsetstrokecolor{currentstroke}%
\pgfsetdash{}{0pt}%
\pgfpathmoveto{\pgfqpoint{6.569129in}{0.613042in}}%
\pgfpathlineto{\pgfqpoint{7.012973in}{0.613042in}}%
\pgfpathlineto{\pgfqpoint{7.012973in}{0.613042in}}%
\pgfpathlineto{\pgfqpoint{6.569129in}{0.613042in}}%
\pgfpathlineto{\pgfqpoint{6.569129in}{0.613042in}}%
\pgfpathclose%
\pgfusepath{stroke,fill}%
\end{pgfscope}%
\begin{pgfscope}%
\pgfpathrectangle{\pgfqpoint{0.688192in}{0.613042in}}{\pgfqpoint{11.096108in}{4.223431in}}%
\pgfusepath{clip}%
\pgfsetbuttcap%
\pgfsetroundjoin%
\definecolor{currentfill}{rgb}{0.517157,0.358333,0.325980}%
\pgfsetfillcolor{currentfill}%
\pgfsetlinewidth{0.752812pt}%
\definecolor{currentstroke}{rgb}{0.240000,0.240000,0.240000}%
\pgfsetstrokecolor{currentstroke}%
\pgfsetdash{}{0pt}%
\pgfpathmoveto{\pgfqpoint{6.347207in}{0.613042in}}%
\pgfpathlineto{\pgfqpoint{7.234895in}{0.613042in}}%
\pgfpathlineto{\pgfqpoint{7.234895in}{0.613042in}}%
\pgfpathlineto{\pgfqpoint{6.347207in}{0.613042in}}%
\pgfpathlineto{\pgfqpoint{6.347207in}{0.613042in}}%
\pgfpathclose%
\pgfusepath{stroke,fill}%
\end{pgfscope}%
\begin{pgfscope}%
\pgfpathrectangle{\pgfqpoint{0.688192in}{0.613042in}}{\pgfqpoint{11.096108in}{4.223431in}}%
\pgfusepath{clip}%
\pgfsetbuttcap%
\pgfsetroundjoin%
\definecolor{currentfill}{rgb}{0.604646,0.477521,0.451635}%
\pgfsetfillcolor{currentfill}%
\pgfsetlinewidth{0.752812pt}%
\definecolor{currentstroke}{rgb}{0.240000,0.240000,0.240000}%
\pgfsetstrokecolor{currentstroke}%
\pgfsetdash{}{0pt}%
\pgfpathmoveto{\pgfqpoint{6.569129in}{0.613042in}}%
\pgfpathlineto{\pgfqpoint{7.012973in}{0.613042in}}%
\pgfpathlineto{\pgfqpoint{7.012973in}{0.630691in}}%
\pgfpathlineto{\pgfqpoint{6.569129in}{0.630691in}}%
\pgfpathlineto{\pgfqpoint{6.569129in}{0.613042in}}%
\pgfpathclose%
\pgfusepath{stroke,fill}%
\end{pgfscope}%
\begin{pgfscope}%
\pgfpathrectangle{\pgfqpoint{0.688192in}{0.613042in}}{\pgfqpoint{11.096108in}{4.223431in}}%
\pgfusepath{clip}%
\pgfsetbuttcap%
\pgfsetroundjoin%
\definecolor{currentfill}{rgb}{0.675311,0.573788,0.553126}%
\pgfsetfillcolor{currentfill}%
\pgfsetlinewidth{0.752812pt}%
\definecolor{currentstroke}{rgb}{0.240000,0.240000,0.240000}%
\pgfsetstrokecolor{currentstroke}%
\pgfsetdash{}{0pt}%
\pgfpathmoveto{\pgfqpoint{6.680090in}{0.630691in}}%
\pgfpathlineto{\pgfqpoint{6.902012in}{0.630691in}}%
\pgfpathlineto{\pgfqpoint{6.902012in}{0.733652in}}%
\pgfpathlineto{\pgfqpoint{6.680090in}{0.733652in}}%
\pgfpathlineto{\pgfqpoint{6.680090in}{0.630691in}}%
\pgfpathclose%
\pgfusepath{stroke,fill}%
\end{pgfscope}%
\begin{pgfscope}%
\pgfpathrectangle{\pgfqpoint{0.688192in}{0.613042in}}{\pgfqpoint{11.096108in}{4.223431in}}%
\pgfusepath{clip}%
\pgfsetbuttcap%
\pgfsetroundjoin%
\pgfsetlinewidth{1.003750pt}%
\definecolor{currentstroke}{rgb}{0.450000,0.450000,0.450000}%
\pgfsetstrokecolor{currentstroke}%
\pgfsetdash{}{0pt}%
\pgfpathmoveto{\pgfqpoint{6.791051in}{0.708585in}}%
\pgfpathcurveto{\pgfqpoint{6.800259in}{0.708585in}}{\pgfqpoint{6.809092in}{0.712243in}}{\pgfqpoint{6.815603in}{0.718755in}}%
\pgfpathcurveto{\pgfqpoint{6.822115in}{0.725266in}}{\pgfqpoint{6.825773in}{0.734099in}}{\pgfqpoint{6.825773in}{0.743307in}}%
\pgfpathcurveto{\pgfqpoint{6.825773in}{0.752516in}}{\pgfqpoint{6.822115in}{0.761348in}}{\pgfqpoint{6.815603in}{0.767859in}}%
\pgfpathcurveto{\pgfqpoint{6.809092in}{0.774371in}}{\pgfqpoint{6.800259in}{0.778029in}}{\pgfqpoint{6.791051in}{0.778029in}}%
\pgfpathcurveto{\pgfqpoint{6.781842in}{0.778029in}}{\pgfqpoint{6.773010in}{0.774371in}}{\pgfqpoint{6.766499in}{0.767859in}}%
\pgfpathcurveto{\pgfqpoint{6.759987in}{0.761348in}}{\pgfqpoint{6.756329in}{0.752516in}}{\pgfqpoint{6.756329in}{0.743307in}}%
\pgfpathcurveto{\pgfqpoint{6.756329in}{0.734099in}}{\pgfqpoint{6.759987in}{0.725266in}}{\pgfqpoint{6.766499in}{0.718755in}}%
\pgfpathcurveto{\pgfqpoint{6.773010in}{0.712243in}}{\pgfqpoint{6.781842in}{0.708585in}}{\pgfqpoint{6.791051in}{0.708585in}}%
\pgfpathlineto{\pgfqpoint{6.791051in}{0.708585in}}%
\pgfpathclose%
\pgfusepath{stroke}%
\end{pgfscope}%
\begin{pgfscope}%
\pgfpathrectangle{\pgfqpoint{0.688192in}{0.613042in}}{\pgfqpoint{11.096108in}{4.223431in}}%
\pgfusepath{clip}%
\pgfsetbuttcap%
\pgfsetroundjoin%
\pgfsetlinewidth{1.003750pt}%
\definecolor{currentstroke}{rgb}{0.450000,0.450000,0.450000}%
\pgfsetstrokecolor{currentstroke}%
\pgfsetdash{}{0pt}%
\pgfpathmoveto{\pgfqpoint{6.791051in}{0.708585in}}%
\pgfpathcurveto{\pgfqpoint{6.800259in}{0.708585in}}{\pgfqpoint{6.809092in}{0.712243in}}{\pgfqpoint{6.815603in}{0.718755in}}%
\pgfpathcurveto{\pgfqpoint{6.822115in}{0.725266in}}{\pgfqpoint{6.825773in}{0.734099in}}{\pgfqpoint{6.825773in}{0.743307in}}%
\pgfpathcurveto{\pgfqpoint{6.825773in}{0.752516in}}{\pgfqpoint{6.822115in}{0.761348in}}{\pgfqpoint{6.815603in}{0.767859in}}%
\pgfpathcurveto{\pgfqpoint{6.809092in}{0.774371in}}{\pgfqpoint{6.800259in}{0.778029in}}{\pgfqpoint{6.791051in}{0.778029in}}%
\pgfpathcurveto{\pgfqpoint{6.781842in}{0.778029in}}{\pgfqpoint{6.773010in}{0.774371in}}{\pgfqpoint{6.766499in}{0.767859in}}%
\pgfpathcurveto{\pgfqpoint{6.759987in}{0.761348in}}{\pgfqpoint{6.756329in}{0.752516in}}{\pgfqpoint{6.756329in}{0.743307in}}%
\pgfpathcurveto{\pgfqpoint{6.756329in}{0.734099in}}{\pgfqpoint{6.759987in}{0.725266in}}{\pgfqpoint{6.766499in}{0.718755in}}%
\pgfpathcurveto{\pgfqpoint{6.773010in}{0.712243in}}{\pgfqpoint{6.781842in}{0.708585in}}{\pgfqpoint{6.791051in}{0.708585in}}%
\pgfpathlineto{\pgfqpoint{6.791051in}{0.708585in}}%
\pgfpathclose%
\pgfusepath{stroke}%
\end{pgfscope}%
\begin{pgfscope}%
\pgfpathrectangle{\pgfqpoint{0.688192in}{0.613042in}}{\pgfqpoint{11.096108in}{4.223431in}}%
\pgfusepath{clip}%
\pgfsetbuttcap%
\pgfsetroundjoin%
\definecolor{currentfill}{rgb}{0.878118,0.675269,0.815953}%
\pgfsetfillcolor{currentfill}%
\pgfsetlinewidth{0.752812pt}%
\definecolor{currentstroke}{rgb}{0.240000,0.240000,0.240000}%
\pgfsetstrokecolor{currentstroke}%
\pgfsetdash{}{0pt}%
\pgfpathmoveto{\pgfqpoint{7.789701in}{0.613042in}}%
\pgfpathlineto{\pgfqpoint{8.011623in}{0.613042in}}%
\pgfpathlineto{\pgfqpoint{8.011623in}{0.613042in}}%
\pgfpathlineto{\pgfqpoint{7.789701in}{0.613042in}}%
\pgfpathlineto{\pgfqpoint{7.789701in}{0.613042in}}%
\pgfpathclose%
\pgfusepath{stroke,fill}%
\end{pgfscope}%
\begin{pgfscope}%
\pgfpathrectangle{\pgfqpoint{0.688192in}{0.613042in}}{\pgfqpoint{11.096108in}{4.223431in}}%
\pgfusepath{clip}%
\pgfsetbuttcap%
\pgfsetroundjoin%
\definecolor{currentfill}{rgb}{0.859860,0.605718,0.782104}%
\pgfsetfillcolor{currentfill}%
\pgfsetlinewidth{0.752812pt}%
\definecolor{currentstroke}{rgb}{0.240000,0.240000,0.240000}%
\pgfsetstrokecolor{currentstroke}%
\pgfsetdash{}{0pt}%
\pgfpathmoveto{\pgfqpoint{7.678740in}{0.613042in}}%
\pgfpathlineto{\pgfqpoint{8.122584in}{0.613042in}}%
\pgfpathlineto{\pgfqpoint{8.122584in}{0.613042in}}%
\pgfpathlineto{\pgfqpoint{7.678740in}{0.613042in}}%
\pgfpathlineto{\pgfqpoint{7.678740in}{0.613042in}}%
\pgfpathclose%
\pgfusepath{stroke,fill}%
\end{pgfscope}%
\begin{pgfscope}%
\pgfpathrectangle{\pgfqpoint{0.688192in}{0.613042in}}{\pgfqpoint{11.096108in}{4.223431in}}%
\pgfusepath{clip}%
\pgfsetbuttcap%
\pgfsetroundjoin%
\definecolor{currentfill}{rgb}{0.837255,0.519608,0.740196}%
\pgfsetfillcolor{currentfill}%
\pgfsetlinewidth{0.752812pt}%
\definecolor{currentstroke}{rgb}{0.240000,0.240000,0.240000}%
\pgfsetstrokecolor{currentstroke}%
\pgfsetdash{}{0pt}%
\pgfpathmoveto{\pgfqpoint{7.456817in}{0.613042in}}%
\pgfpathlineto{\pgfqpoint{8.344506in}{0.613042in}}%
\pgfpathlineto{\pgfqpoint{8.344506in}{0.613042in}}%
\pgfpathlineto{\pgfqpoint{7.456817in}{0.613042in}}%
\pgfpathlineto{\pgfqpoint{7.456817in}{0.613042in}}%
\pgfpathclose%
\pgfusepath{stroke,fill}%
\end{pgfscope}%
\begin{pgfscope}%
\pgfpathrectangle{\pgfqpoint{0.688192in}{0.613042in}}{\pgfqpoint{11.096108in}{4.223431in}}%
\pgfusepath{clip}%
\pgfsetbuttcap%
\pgfsetroundjoin%
\definecolor{currentfill}{rgb}{0.859860,0.605718,0.782104}%
\pgfsetfillcolor{currentfill}%
\pgfsetlinewidth{0.752812pt}%
\definecolor{currentstroke}{rgb}{0.240000,0.240000,0.240000}%
\pgfsetstrokecolor{currentstroke}%
\pgfsetdash{}{0pt}%
\pgfpathmoveto{\pgfqpoint{7.678740in}{0.613042in}}%
\pgfpathlineto{\pgfqpoint{8.122584in}{0.613042in}}%
\pgfpathlineto{\pgfqpoint{8.122584in}{0.613042in}}%
\pgfpathlineto{\pgfqpoint{7.678740in}{0.613042in}}%
\pgfpathlineto{\pgfqpoint{7.678740in}{0.613042in}}%
\pgfpathclose%
\pgfusepath{stroke,fill}%
\end{pgfscope}%
\begin{pgfscope}%
\pgfpathrectangle{\pgfqpoint{0.688192in}{0.613042in}}{\pgfqpoint{11.096108in}{4.223431in}}%
\pgfusepath{clip}%
\pgfsetbuttcap%
\pgfsetroundjoin%
\definecolor{currentfill}{rgb}{0.878118,0.675269,0.815953}%
\pgfsetfillcolor{currentfill}%
\pgfsetlinewidth{0.752812pt}%
\definecolor{currentstroke}{rgb}{0.240000,0.240000,0.240000}%
\pgfsetstrokecolor{currentstroke}%
\pgfsetdash{}{0pt}%
\pgfpathmoveto{\pgfqpoint{7.789701in}{0.613042in}}%
\pgfpathlineto{\pgfqpoint{8.011623in}{0.613042in}}%
\pgfpathlineto{\pgfqpoint{8.011623in}{0.627085in}}%
\pgfpathlineto{\pgfqpoint{7.789701in}{0.627085in}}%
\pgfpathlineto{\pgfqpoint{7.789701in}{0.613042in}}%
\pgfpathclose%
\pgfusepath{stroke,fill}%
\end{pgfscope}%
\begin{pgfscope}%
\pgfpathrectangle{\pgfqpoint{0.688192in}{0.613042in}}{\pgfqpoint{11.096108in}{4.223431in}}%
\pgfusepath{clip}%
\pgfsetbuttcap%
\pgfsetroundjoin%
\pgfsetlinewidth{1.003750pt}%
\definecolor{currentstroke}{rgb}{0.450000,0.450000,0.450000}%
\pgfsetstrokecolor{currentstroke}%
\pgfsetdash{}{0pt}%
\pgfpathmoveto{\pgfqpoint{7.900662in}{0.709506in}}%
\pgfpathcurveto{\pgfqpoint{7.909870in}{0.709506in}}{\pgfqpoint{7.918703in}{0.713164in}}{\pgfqpoint{7.925214in}{0.719676in}}%
\pgfpathcurveto{\pgfqpoint{7.931725in}{0.726187in}}{\pgfqpoint{7.935384in}{0.735020in}}{\pgfqpoint{7.935384in}{0.744228in}}%
\pgfpathcurveto{\pgfqpoint{7.935384in}{0.753436in}}{\pgfqpoint{7.931725in}{0.762269in}}{\pgfqpoint{7.925214in}{0.768780in}}%
\pgfpathcurveto{\pgfqpoint{7.918703in}{0.775292in}}{\pgfqpoint{7.909870in}{0.778950in}}{\pgfqpoint{7.900662in}{0.778950in}}%
\pgfpathcurveto{\pgfqpoint{7.891453in}{0.778950in}}{\pgfqpoint{7.882621in}{0.775292in}}{\pgfqpoint{7.876109in}{0.768780in}}%
\pgfpathcurveto{\pgfqpoint{7.869598in}{0.762269in}}{\pgfqpoint{7.865939in}{0.753436in}}{\pgfqpoint{7.865939in}{0.744228in}}%
\pgfpathcurveto{\pgfqpoint{7.865939in}{0.735020in}}{\pgfqpoint{7.869598in}{0.726187in}}{\pgfqpoint{7.876109in}{0.719676in}}%
\pgfpathcurveto{\pgfqpoint{7.882621in}{0.713164in}}{\pgfqpoint{7.891453in}{0.709506in}}{\pgfqpoint{7.900662in}{0.709506in}}%
\pgfpathlineto{\pgfqpoint{7.900662in}{0.709506in}}%
\pgfpathclose%
\pgfusepath{stroke}%
\end{pgfscope}%
\begin{pgfscope}%
\pgfpathrectangle{\pgfqpoint{0.688192in}{0.613042in}}{\pgfqpoint{11.096108in}{4.223431in}}%
\pgfusepath{clip}%
\pgfsetbuttcap%
\pgfsetroundjoin%
\pgfsetlinewidth{1.003750pt}%
\definecolor{currentstroke}{rgb}{0.450000,0.450000,0.450000}%
\pgfsetstrokecolor{currentstroke}%
\pgfsetdash{}{0pt}%
\pgfpathmoveto{\pgfqpoint{7.900662in}{0.690664in}}%
\pgfpathcurveto{\pgfqpoint{7.909870in}{0.690664in}}{\pgfqpoint{7.918703in}{0.694323in}}{\pgfqpoint{7.925214in}{0.700834in}}%
\pgfpathcurveto{\pgfqpoint{7.931725in}{0.707346in}}{\pgfqpoint{7.935384in}{0.716178in}}{\pgfqpoint{7.935384in}{0.725387in}}%
\pgfpathcurveto{\pgfqpoint{7.935384in}{0.734595in}}{\pgfqpoint{7.931725in}{0.743428in}}{\pgfqpoint{7.925214in}{0.749939in}}%
\pgfpathcurveto{\pgfqpoint{7.918703in}{0.756450in}}{\pgfqpoint{7.909870in}{0.760109in}}{\pgfqpoint{7.900662in}{0.760109in}}%
\pgfpathcurveto{\pgfqpoint{7.891453in}{0.760109in}}{\pgfqpoint{7.882621in}{0.756450in}}{\pgfqpoint{7.876109in}{0.749939in}}%
\pgfpathcurveto{\pgfqpoint{7.869598in}{0.743428in}}{\pgfqpoint{7.865939in}{0.734595in}}{\pgfqpoint{7.865939in}{0.725387in}}%
\pgfpathcurveto{\pgfqpoint{7.865939in}{0.716178in}}{\pgfqpoint{7.869598in}{0.707346in}}{\pgfqpoint{7.876109in}{0.700834in}}%
\pgfpathcurveto{\pgfqpoint{7.882621in}{0.694323in}}{\pgfqpoint{7.891453in}{0.690664in}}{\pgfqpoint{7.900662in}{0.690664in}}%
\pgfpathlineto{\pgfqpoint{7.900662in}{0.690664in}}%
\pgfpathclose%
\pgfusepath{stroke}%
\end{pgfscope}%
\begin{pgfscope}%
\pgfpathrectangle{\pgfqpoint{0.688192in}{0.613042in}}{\pgfqpoint{11.096108in}{4.223431in}}%
\pgfusepath{clip}%
\pgfsetbuttcap%
\pgfsetroundjoin%
\definecolor{currentfill}{rgb}{0.662237,0.662248,0.662203}%
\pgfsetfillcolor{currentfill}%
\pgfsetlinewidth{0.752812pt}%
\definecolor{currentstroke}{rgb}{0.240000,0.240000,0.240000}%
\pgfsetstrokecolor{currentstroke}%
\pgfsetdash{}{0pt}%
\pgfpathmoveto{\pgfqpoint{8.899311in}{0.696837in}}%
\pgfpathlineto{\pgfqpoint{9.121234in}{0.696837in}}%
\pgfpathlineto{\pgfqpoint{9.121234in}{0.696837in}}%
\pgfpathlineto{\pgfqpoint{8.899311in}{0.696837in}}%
\pgfpathlineto{\pgfqpoint{8.899311in}{0.696837in}}%
\pgfpathclose%
\pgfusepath{stroke,fill}%
\end{pgfscope}%
\begin{pgfscope}%
\pgfpathrectangle{\pgfqpoint{0.688192in}{0.613042in}}{\pgfqpoint{11.096108in}{4.223431in}}%
\pgfusepath{clip}%
\pgfsetbuttcap%
\pgfsetroundjoin%
\definecolor{currentfill}{rgb}{0.588872,0.588878,0.588853}%
\pgfsetfillcolor{currentfill}%
\pgfsetlinewidth{0.752812pt}%
\definecolor{currentstroke}{rgb}{0.240000,0.240000,0.240000}%
\pgfsetstrokecolor{currentstroke}%
\pgfsetdash{}{0pt}%
\pgfpathmoveto{\pgfqpoint{8.788350in}{0.696837in}}%
\pgfpathlineto{\pgfqpoint{9.232195in}{0.696837in}}%
\pgfpathlineto{\pgfqpoint{9.232195in}{0.696837in}}%
\pgfpathlineto{\pgfqpoint{8.788350in}{0.696837in}}%
\pgfpathlineto{\pgfqpoint{8.788350in}{0.696837in}}%
\pgfpathclose%
\pgfusepath{stroke,fill}%
\end{pgfscope}%
\begin{pgfscope}%
\pgfpathrectangle{\pgfqpoint{0.688192in}{0.613042in}}{\pgfqpoint{11.096108in}{4.223431in}}%
\pgfusepath{clip}%
\pgfsetbuttcap%
\pgfsetroundjoin%
\definecolor{currentfill}{rgb}{0.498039,0.498039,0.498039}%
\pgfsetfillcolor{currentfill}%
\pgfsetlinewidth{0.752812pt}%
\definecolor{currentstroke}{rgb}{0.240000,0.240000,0.240000}%
\pgfsetstrokecolor{currentstroke}%
\pgfsetdash{}{0pt}%
\pgfpathmoveto{\pgfqpoint{8.566428in}{0.696837in}}%
\pgfpathlineto{\pgfqpoint{9.454117in}{0.696837in}}%
\pgfpathlineto{\pgfqpoint{9.454117in}{0.771938in}}%
\pgfpathlineto{\pgfqpoint{8.566428in}{0.771938in}}%
\pgfpathlineto{\pgfqpoint{8.566428in}{0.696837in}}%
\pgfpathclose%
\pgfusepath{stroke,fill}%
\end{pgfscope}%
\begin{pgfscope}%
\pgfpathrectangle{\pgfqpoint{0.688192in}{0.613042in}}{\pgfqpoint{11.096108in}{4.223431in}}%
\pgfusepath{clip}%
\pgfsetbuttcap%
\pgfsetroundjoin%
\definecolor{currentfill}{rgb}{0.588872,0.588878,0.588853}%
\pgfsetfillcolor{currentfill}%
\pgfsetlinewidth{0.752812pt}%
\definecolor{currentstroke}{rgb}{0.240000,0.240000,0.240000}%
\pgfsetstrokecolor{currentstroke}%
\pgfsetdash{}{0pt}%
\pgfpathmoveto{\pgfqpoint{8.788350in}{0.771938in}}%
\pgfpathlineto{\pgfqpoint{9.232195in}{0.771938in}}%
\pgfpathlineto{\pgfqpoint{9.232195in}{0.772933in}}%
\pgfpathlineto{\pgfqpoint{8.788350in}{0.772933in}}%
\pgfpathlineto{\pgfqpoint{8.788350in}{0.771938in}}%
\pgfpathclose%
\pgfusepath{stroke,fill}%
\end{pgfscope}%
\begin{pgfscope}%
\pgfpathrectangle{\pgfqpoint{0.688192in}{0.613042in}}{\pgfqpoint{11.096108in}{4.223431in}}%
\pgfusepath{clip}%
\pgfsetbuttcap%
\pgfsetroundjoin%
\definecolor{currentfill}{rgb}{0.662237,0.662248,0.662203}%
\pgfsetfillcolor{currentfill}%
\pgfsetlinewidth{0.752812pt}%
\definecolor{currentstroke}{rgb}{0.240000,0.240000,0.240000}%
\pgfsetstrokecolor{currentstroke}%
\pgfsetdash{}{0pt}%
\pgfpathmoveto{\pgfqpoint{8.899311in}{0.772933in}}%
\pgfpathlineto{\pgfqpoint{9.121234in}{0.772933in}}%
\pgfpathlineto{\pgfqpoint{9.121234in}{0.796333in}}%
\pgfpathlineto{\pgfqpoint{8.899311in}{0.796333in}}%
\pgfpathlineto{\pgfqpoint{8.899311in}{0.772933in}}%
\pgfpathclose%
\pgfusepath{stroke,fill}%
\end{pgfscope}%
\begin{pgfscope}%
\pgfpathrectangle{\pgfqpoint{0.688192in}{0.613042in}}{\pgfqpoint{11.096108in}{4.223431in}}%
\pgfusepath{clip}%
\pgfsetbuttcap%
\pgfsetroundjoin%
\pgfsetlinewidth{1.003750pt}%
\definecolor{currentstroke}{rgb}{0.450000,0.450000,0.450000}%
\pgfsetstrokecolor{currentstroke}%
\pgfsetdash{}{0pt}%
\pgfpathmoveto{\pgfqpoint{9.010272in}{0.790037in}}%
\pgfpathcurveto{\pgfqpoint{9.019481in}{0.790037in}}{\pgfqpoint{9.028313in}{0.793696in}}{\pgfqpoint{9.034825in}{0.800207in}}%
\pgfpathcurveto{\pgfqpoint{9.041336in}{0.806719in}}{\pgfqpoint{9.044995in}{0.815551in}}{\pgfqpoint{9.044995in}{0.824759in}}%
\pgfpathcurveto{\pgfqpoint{9.044995in}{0.833968in}}{\pgfqpoint{9.041336in}{0.842800in}}{\pgfqpoint{9.034825in}{0.849312in}}%
\pgfpathcurveto{\pgfqpoint{9.028313in}{0.855823in}}{\pgfqpoint{9.019481in}{0.859482in}}{\pgfqpoint{9.010272in}{0.859482in}}%
\pgfpathcurveto{\pgfqpoint{9.001064in}{0.859482in}}{\pgfqpoint{8.992232in}{0.855823in}}{\pgfqpoint{8.985720in}{0.849312in}}%
\pgfpathcurveto{\pgfqpoint{8.979209in}{0.842800in}}{\pgfqpoint{8.975550in}{0.833968in}}{\pgfqpoint{8.975550in}{0.824759in}}%
\pgfpathcurveto{\pgfqpoint{8.975550in}{0.815551in}}{\pgfqpoint{8.979209in}{0.806719in}}{\pgfqpoint{8.985720in}{0.800207in}}%
\pgfpathcurveto{\pgfqpoint{8.992232in}{0.793696in}}{\pgfqpoint{9.001064in}{0.790037in}}{\pgfqpoint{9.010272in}{0.790037in}}%
\pgfpathlineto{\pgfqpoint{9.010272in}{0.790037in}}%
\pgfpathclose%
\pgfusepath{stroke}%
\end{pgfscope}%
\begin{pgfscope}%
\pgfpathrectangle{\pgfqpoint{0.688192in}{0.613042in}}{\pgfqpoint{11.096108in}{4.223431in}}%
\pgfusepath{clip}%
\pgfsetbuttcap%
\pgfsetroundjoin%
\pgfsetlinewidth{1.003750pt}%
\definecolor{currentstroke}{rgb}{0.450000,0.450000,0.450000}%
\pgfsetstrokecolor{currentstroke}%
\pgfsetdash{}{0pt}%
\pgfpathmoveto{\pgfqpoint{9.010272in}{0.763347in}}%
\pgfpathcurveto{\pgfqpoint{9.019481in}{0.763347in}}{\pgfqpoint{9.028313in}{0.767005in}}{\pgfqpoint{9.034825in}{0.773516in}}%
\pgfpathcurveto{\pgfqpoint{9.041336in}{0.780028in}}{\pgfqpoint{9.044995in}{0.788860in}}{\pgfqpoint{9.044995in}{0.798069in}}%
\pgfpathcurveto{\pgfqpoint{9.044995in}{0.807277in}}{\pgfqpoint{9.041336in}{0.816110in}}{\pgfqpoint{9.034825in}{0.822621in}}%
\pgfpathcurveto{\pgfqpoint{9.028313in}{0.829132in}}{\pgfqpoint{9.019481in}{0.832791in}}{\pgfqpoint{9.010272in}{0.832791in}}%
\pgfpathcurveto{\pgfqpoint{9.001064in}{0.832791in}}{\pgfqpoint{8.992232in}{0.829132in}}{\pgfqpoint{8.985720in}{0.822621in}}%
\pgfpathcurveto{\pgfqpoint{8.979209in}{0.816110in}}{\pgfqpoint{8.975550in}{0.807277in}}{\pgfqpoint{8.975550in}{0.798069in}}%
\pgfpathcurveto{\pgfqpoint{8.975550in}{0.788860in}}{\pgfqpoint{8.979209in}{0.780028in}}{\pgfqpoint{8.985720in}{0.773516in}}%
\pgfpathcurveto{\pgfqpoint{8.992232in}{0.767005in}}{\pgfqpoint{9.001064in}{0.763347in}}{\pgfqpoint{9.010272in}{0.763347in}}%
\pgfpathlineto{\pgfqpoint{9.010272in}{0.763347in}}%
\pgfpathclose%
\pgfusepath{stroke}%
\end{pgfscope}%
\begin{pgfscope}%
\pgfpathrectangle{\pgfqpoint{0.688192in}{0.613042in}}{\pgfqpoint{11.096108in}{4.223431in}}%
\pgfusepath{clip}%
\pgfsetbuttcap%
\pgfsetroundjoin%
\definecolor{currentfill}{rgb}{0.767534,0.769674,0.455219}%
\pgfsetfillcolor{currentfill}%
\pgfsetlinewidth{0.752812pt}%
\definecolor{currentstroke}{rgb}{0.240000,0.240000,0.240000}%
\pgfsetstrokecolor{currentstroke}%
\pgfsetdash{}{0pt}%
\pgfpathmoveto{\pgfqpoint{10.008922in}{1.069690in}}%
\pgfpathlineto{\pgfqpoint{10.230844in}{1.069690in}}%
\pgfpathlineto{\pgfqpoint{10.230844in}{1.098636in}}%
\pgfpathlineto{\pgfqpoint{10.008922in}{1.098636in}}%
\pgfpathlineto{\pgfqpoint{10.008922in}{1.069690in}}%
\pgfpathclose%
\pgfusepath{stroke,fill}%
\end{pgfscope}%
\begin{pgfscope}%
\pgfpathrectangle{\pgfqpoint{0.688192in}{0.613042in}}{\pgfqpoint{11.096108in}{4.223431in}}%
\pgfusepath{clip}%
\pgfsetbuttcap%
\pgfsetroundjoin%
\definecolor{currentfill}{rgb}{0.720495,0.722993,0.345346}%
\pgfsetfillcolor{currentfill}%
\pgfsetlinewidth{0.752812pt}%
\definecolor{currentstroke}{rgb}{0.240000,0.240000,0.240000}%
\pgfsetstrokecolor{currentstroke}%
\pgfsetdash{}{0pt}%
\pgfpathmoveto{\pgfqpoint{9.897961in}{1.098636in}}%
\pgfpathlineto{\pgfqpoint{10.341805in}{1.098636in}}%
\pgfpathlineto{\pgfqpoint{10.341805in}{1.111069in}}%
\pgfpathlineto{\pgfqpoint{9.897961in}{1.111069in}}%
\pgfpathlineto{\pgfqpoint{9.897961in}{1.098636in}}%
\pgfpathclose%
\pgfusepath{stroke,fill}%
\end{pgfscope}%
\begin{pgfscope}%
\pgfpathrectangle{\pgfqpoint{0.688192in}{0.613042in}}{\pgfqpoint{11.096108in}{4.223431in}}%
\pgfusepath{clip}%
\pgfsetbuttcap%
\pgfsetroundjoin%
\definecolor{currentfill}{rgb}{0.662255,0.665196,0.209314}%
\pgfsetfillcolor{currentfill}%
\pgfsetlinewidth{0.752812pt}%
\definecolor{currentstroke}{rgb}{0.240000,0.240000,0.240000}%
\pgfsetstrokecolor{currentstroke}%
\pgfsetdash{}{0pt}%
\pgfpathmoveto{\pgfqpoint{9.676039in}{1.111069in}}%
\pgfpathlineto{\pgfqpoint{10.563728in}{1.111069in}}%
\pgfpathlineto{\pgfqpoint{10.563728in}{1.155254in}}%
\pgfpathlineto{\pgfqpoint{9.676039in}{1.155254in}}%
\pgfpathlineto{\pgfqpoint{9.676039in}{1.111069in}}%
\pgfpathclose%
\pgfusepath{stroke,fill}%
\end{pgfscope}%
\begin{pgfscope}%
\pgfpathrectangle{\pgfqpoint{0.688192in}{0.613042in}}{\pgfqpoint{11.096108in}{4.223431in}}%
\pgfusepath{clip}%
\pgfsetbuttcap%
\pgfsetroundjoin%
\definecolor{currentfill}{rgb}{0.720495,0.722993,0.345346}%
\pgfsetfillcolor{currentfill}%
\pgfsetlinewidth{0.752812pt}%
\definecolor{currentstroke}{rgb}{0.240000,0.240000,0.240000}%
\pgfsetstrokecolor{currentstroke}%
\pgfsetdash{}{0pt}%
\pgfpathmoveto{\pgfqpoint{9.897961in}{1.155254in}}%
\pgfpathlineto{\pgfqpoint{10.341805in}{1.155254in}}%
\pgfpathlineto{\pgfqpoint{10.341805in}{1.158886in}}%
\pgfpathlineto{\pgfqpoint{9.897961in}{1.158886in}}%
\pgfpathlineto{\pgfqpoint{9.897961in}{1.155254in}}%
\pgfpathclose%
\pgfusepath{stroke,fill}%
\end{pgfscope}%
\begin{pgfscope}%
\pgfpathrectangle{\pgfqpoint{0.688192in}{0.613042in}}{\pgfqpoint{11.096108in}{4.223431in}}%
\pgfusepath{clip}%
\pgfsetbuttcap%
\pgfsetroundjoin%
\definecolor{currentfill}{rgb}{0.767534,0.769674,0.455219}%
\pgfsetfillcolor{currentfill}%
\pgfsetlinewidth{0.752812pt}%
\definecolor{currentstroke}{rgb}{0.240000,0.240000,0.240000}%
\pgfsetstrokecolor{currentstroke}%
\pgfsetdash{}{0pt}%
\pgfpathmoveto{\pgfqpoint{10.008922in}{1.158886in}}%
\pgfpathlineto{\pgfqpoint{10.230844in}{1.158886in}}%
\pgfpathlineto{\pgfqpoint{10.230844in}{1.158886in}}%
\pgfpathlineto{\pgfqpoint{10.008922in}{1.158886in}}%
\pgfpathlineto{\pgfqpoint{10.008922in}{1.158886in}}%
\pgfpathclose%
\pgfusepath{stroke,fill}%
\end{pgfscope}%
\begin{pgfscope}%
\pgfpathrectangle{\pgfqpoint{0.688192in}{0.613042in}}{\pgfqpoint{11.096108in}{4.223431in}}%
\pgfusepath{clip}%
\pgfsetbuttcap%
\pgfsetroundjoin%
\pgfsetlinewidth{1.003750pt}%
\definecolor{currentstroke}{rgb}{0.450000,0.450000,0.450000}%
\pgfsetstrokecolor{currentstroke}%
\pgfsetdash{}{0pt}%
\pgfpathmoveto{\pgfqpoint{10.119883in}{1.017916in}}%
\pgfpathcurveto{\pgfqpoint{10.129092in}{1.017916in}}{\pgfqpoint{10.137924in}{1.021575in}}{\pgfqpoint{10.144436in}{1.028086in}}%
\pgfpathcurveto{\pgfqpoint{10.150947in}{1.034598in}}{\pgfqpoint{10.154606in}{1.043430in}}{\pgfqpoint{10.154606in}{1.052639in}}%
\pgfpathcurveto{\pgfqpoint{10.154606in}{1.061847in}}{\pgfqpoint{10.150947in}{1.070680in}}{\pgfqpoint{10.144436in}{1.077191in}}%
\pgfpathcurveto{\pgfqpoint{10.137924in}{1.083702in}}{\pgfqpoint{10.129092in}{1.087361in}}{\pgfqpoint{10.119883in}{1.087361in}}%
\pgfpathcurveto{\pgfqpoint{10.110675in}{1.087361in}}{\pgfqpoint{10.101842in}{1.083702in}}{\pgfqpoint{10.095331in}{1.077191in}}%
\pgfpathcurveto{\pgfqpoint{10.088820in}{1.070680in}}{\pgfqpoint{10.085161in}{1.061847in}}{\pgfqpoint{10.085161in}{1.052639in}}%
\pgfpathcurveto{\pgfqpoint{10.085161in}{1.043430in}}{\pgfqpoint{10.088820in}{1.034598in}}{\pgfqpoint{10.095331in}{1.028086in}}%
\pgfpathcurveto{\pgfqpoint{10.101842in}{1.021575in}}{\pgfqpoint{10.110675in}{1.017916in}}{\pgfqpoint{10.119883in}{1.017916in}}%
\pgfpathlineto{\pgfqpoint{10.119883in}{1.017916in}}%
\pgfpathclose%
\pgfusepath{stroke}%
\end{pgfscope}%
\begin{pgfscope}%
\pgfpathrectangle{\pgfqpoint{0.688192in}{0.613042in}}{\pgfqpoint{11.096108in}{4.223431in}}%
\pgfusepath{clip}%
\pgfsetbuttcap%
\pgfsetroundjoin%
\pgfsetlinewidth{1.003750pt}%
\definecolor{currentstroke}{rgb}{0.450000,0.450000,0.450000}%
\pgfsetstrokecolor{currentstroke}%
\pgfsetdash{}{0pt}%
\pgfpathmoveto{\pgfqpoint{10.119883in}{0.986990in}}%
\pgfpathcurveto{\pgfqpoint{10.129092in}{0.986990in}}{\pgfqpoint{10.137924in}{0.990648in}}{\pgfqpoint{10.144436in}{0.997159in}}%
\pgfpathcurveto{\pgfqpoint{10.150947in}{1.003671in}}{\pgfqpoint{10.154606in}{1.012503in}}{\pgfqpoint{10.154606in}{1.021712in}}%
\pgfpathcurveto{\pgfqpoint{10.154606in}{1.030920in}}{\pgfqpoint{10.150947in}{1.039753in}}{\pgfqpoint{10.144436in}{1.046264in}}%
\pgfpathcurveto{\pgfqpoint{10.137924in}{1.052775in}}{\pgfqpoint{10.129092in}{1.056434in}}{\pgfqpoint{10.119883in}{1.056434in}}%
\pgfpathcurveto{\pgfqpoint{10.110675in}{1.056434in}}{\pgfqpoint{10.101842in}{1.052775in}}{\pgfqpoint{10.095331in}{1.046264in}}%
\pgfpathcurveto{\pgfqpoint{10.088820in}{1.039753in}}{\pgfqpoint{10.085161in}{1.030920in}}{\pgfqpoint{10.085161in}{1.021712in}}%
\pgfpathcurveto{\pgfqpoint{10.085161in}{1.012503in}}{\pgfqpoint{10.088820in}{1.003671in}}{\pgfqpoint{10.095331in}{0.997159in}}%
\pgfpathcurveto{\pgfqpoint{10.101842in}{0.990648in}}{\pgfqpoint{10.110675in}{0.986990in}}{\pgfqpoint{10.119883in}{0.986990in}}%
\pgfpathlineto{\pgfqpoint{10.119883in}{0.986990in}}%
\pgfpathclose%
\pgfusepath{stroke}%
\end{pgfscope}%
\begin{pgfscope}%
\pgfpathrectangle{\pgfqpoint{0.688192in}{0.613042in}}{\pgfqpoint{11.096108in}{4.223431in}}%
\pgfusepath{clip}%
\pgfsetbuttcap%
\pgfsetroundjoin%
\definecolor{currentfill}{rgb}{0.455462,0.773335,0.806273}%
\pgfsetfillcolor{currentfill}%
\pgfsetlinewidth{0.752812pt}%
\definecolor{currentstroke}{rgb}{0.240000,0.240000,0.240000}%
\pgfsetstrokecolor{currentstroke}%
\pgfsetdash{}{0pt}%
\pgfpathmoveto{\pgfqpoint{11.118533in}{0.613042in}}%
\pgfpathlineto{\pgfqpoint{11.340455in}{0.613042in}}%
\pgfpathlineto{\pgfqpoint{11.340455in}{0.613042in}}%
\pgfpathlineto{\pgfqpoint{11.118533in}{0.613042in}}%
\pgfpathlineto{\pgfqpoint{11.118533in}{0.613042in}}%
\pgfpathclose%
\pgfusepath{stroke,fill}%
\end{pgfscope}%
\begin{pgfscope}%
\pgfpathrectangle{\pgfqpoint{0.688192in}{0.613042in}}{\pgfqpoint{11.096108in}{4.223431in}}%
\pgfusepath{clip}%
\pgfsetbuttcap%
\pgfsetroundjoin%
\definecolor{currentfill}{rgb}{0.332558,0.727865,0.768426}%
\pgfsetfillcolor{currentfill}%
\pgfsetlinewidth{0.752812pt}%
\definecolor{currentstroke}{rgb}{0.240000,0.240000,0.240000}%
\pgfsetstrokecolor{currentstroke}%
\pgfsetdash{}{0pt}%
\pgfpathmoveto{\pgfqpoint{11.007572in}{0.613042in}}%
\pgfpathlineto{\pgfqpoint{11.451416in}{0.613042in}}%
\pgfpathlineto{\pgfqpoint{11.451416in}{0.613042in}}%
\pgfpathlineto{\pgfqpoint{11.007572in}{0.613042in}}%
\pgfpathlineto{\pgfqpoint{11.007572in}{0.613042in}}%
\pgfpathclose%
\pgfusepath{stroke,fill}%
\end{pgfscope}%
\begin{pgfscope}%
\pgfpathrectangle{\pgfqpoint{0.688192in}{0.613042in}}{\pgfqpoint{11.096108in}{4.223431in}}%
\pgfusepath{clip}%
\pgfsetbuttcap%
\pgfsetroundjoin%
\definecolor{currentfill}{rgb}{0.180392,0.671569,0.721569}%
\pgfsetfillcolor{currentfill}%
\pgfsetlinewidth{0.752812pt}%
\definecolor{currentstroke}{rgb}{0.240000,0.240000,0.240000}%
\pgfsetstrokecolor{currentstroke}%
\pgfsetdash{}{0pt}%
\pgfpathmoveto{\pgfqpoint{10.785650in}{0.613042in}}%
\pgfpathlineto{\pgfqpoint{11.673338in}{0.613042in}}%
\pgfpathlineto{\pgfqpoint{11.673338in}{1.013140in}}%
\pgfpathlineto{\pgfqpoint{10.785650in}{1.013140in}}%
\pgfpathlineto{\pgfqpoint{10.785650in}{0.613042in}}%
\pgfpathclose%
\pgfusepath{stroke,fill}%
\end{pgfscope}%
\begin{pgfscope}%
\pgfpathrectangle{\pgfqpoint{0.688192in}{0.613042in}}{\pgfqpoint{11.096108in}{4.223431in}}%
\pgfusepath{clip}%
\pgfsetbuttcap%
\pgfsetroundjoin%
\definecolor{currentfill}{rgb}{0.332558,0.727865,0.768426}%
\pgfsetfillcolor{currentfill}%
\pgfsetlinewidth{0.752812pt}%
\definecolor{currentstroke}{rgb}{0.240000,0.240000,0.240000}%
\pgfsetstrokecolor{currentstroke}%
\pgfsetdash{}{0pt}%
\pgfpathmoveto{\pgfqpoint{11.007572in}{1.013140in}}%
\pgfpathlineto{\pgfqpoint{11.451416in}{1.013140in}}%
\pgfpathlineto{\pgfqpoint{11.451416in}{1.022187in}}%
\pgfpathlineto{\pgfqpoint{11.007572in}{1.022187in}}%
\pgfpathlineto{\pgfqpoint{11.007572in}{1.013140in}}%
\pgfpathclose%
\pgfusepath{stroke,fill}%
\end{pgfscope}%
\begin{pgfscope}%
\pgfpathrectangle{\pgfqpoint{0.688192in}{0.613042in}}{\pgfqpoint{11.096108in}{4.223431in}}%
\pgfusepath{clip}%
\pgfsetbuttcap%
\pgfsetroundjoin%
\definecolor{currentfill}{rgb}{0.455462,0.773335,0.806273}%
\pgfsetfillcolor{currentfill}%
\pgfsetlinewidth{0.752812pt}%
\definecolor{currentstroke}{rgb}{0.240000,0.240000,0.240000}%
\pgfsetstrokecolor{currentstroke}%
\pgfsetdash{}{0pt}%
\pgfpathmoveto{\pgfqpoint{11.118533in}{1.022187in}}%
\pgfpathlineto{\pgfqpoint{11.340455in}{1.022187in}}%
\pgfpathlineto{\pgfqpoint{11.340455in}{1.296933in}}%
\pgfpathlineto{\pgfqpoint{11.118533in}{1.296933in}}%
\pgfpathlineto{\pgfqpoint{11.118533in}{1.022187in}}%
\pgfpathclose%
\pgfusepath{stroke,fill}%
\end{pgfscope}%
\begin{pgfscope}%
\pgfpathrectangle{\pgfqpoint{0.688192in}{0.613042in}}{\pgfqpoint{11.096108in}{4.223431in}}%
\pgfusepath{clip}%
\pgfsetbuttcap%
\pgfsetroundjoin%
\pgfsetlinewidth{1.003750pt}%
\definecolor{currentstroke}{rgb}{0.450000,0.450000,0.450000}%
\pgfsetstrokecolor{currentstroke}%
\pgfsetdash{}{0pt}%
\pgfpathmoveto{\pgfqpoint{11.229494in}{1.677962in}}%
\pgfpathcurveto{\pgfqpoint{11.238703in}{1.677962in}}{\pgfqpoint{11.247535in}{1.681620in}}{\pgfqpoint{11.254046in}{1.688132in}}%
\pgfpathcurveto{\pgfqpoint{11.260558in}{1.694643in}}{\pgfqpoint{11.264216in}{1.703475in}}{\pgfqpoint{11.264216in}{1.712684in}}%
\pgfpathcurveto{\pgfqpoint{11.264216in}{1.721892in}}{\pgfqpoint{11.260558in}{1.730725in}}{\pgfqpoint{11.254046in}{1.737236in}}%
\pgfpathcurveto{\pgfqpoint{11.247535in}{1.743748in}}{\pgfqpoint{11.238703in}{1.747406in}}{\pgfqpoint{11.229494in}{1.747406in}}%
\pgfpathcurveto{\pgfqpoint{11.220286in}{1.747406in}}{\pgfqpoint{11.211453in}{1.743748in}}{\pgfqpoint{11.204942in}{1.737236in}}%
\pgfpathcurveto{\pgfqpoint{11.198430in}{1.730725in}}{\pgfqpoint{11.194772in}{1.721892in}}{\pgfqpoint{11.194772in}{1.712684in}}%
\pgfpathcurveto{\pgfqpoint{11.194772in}{1.703475in}}{\pgfqpoint{11.198430in}{1.694643in}}{\pgfqpoint{11.204942in}{1.688132in}}%
\pgfpathcurveto{\pgfqpoint{11.211453in}{1.681620in}}{\pgfqpoint{11.220286in}{1.677962in}}{\pgfqpoint{11.229494in}{1.677962in}}%
\pgfpathlineto{\pgfqpoint{11.229494in}{1.677962in}}%
\pgfpathclose%
\pgfusepath{stroke}%
\end{pgfscope}%
\begin{pgfscope}%
\pgfpathrectangle{\pgfqpoint{0.688192in}{0.613042in}}{\pgfqpoint{11.096108in}{4.223431in}}%
\pgfusepath{clip}%
\pgfsetbuttcap%
\pgfsetroundjoin%
\pgfsetlinewidth{1.003750pt}%
\definecolor{currentstroke}{rgb}{0.450000,0.450000,0.450000}%
\pgfsetstrokecolor{currentstroke}%
\pgfsetdash{}{0pt}%
\pgfpathmoveto{\pgfqpoint{11.229494in}{1.283290in}}%
\pgfpathcurveto{\pgfqpoint{11.238703in}{1.283290in}}{\pgfqpoint{11.247535in}{1.286949in}}{\pgfqpoint{11.254046in}{1.293460in}}%
\pgfpathcurveto{\pgfqpoint{11.260558in}{1.299972in}}{\pgfqpoint{11.264216in}{1.308804in}}{\pgfqpoint{11.264216in}{1.318013in}}%
\pgfpathcurveto{\pgfqpoint{11.264216in}{1.327221in}}{\pgfqpoint{11.260558in}{1.336054in}}{\pgfqpoint{11.254046in}{1.342565in}}%
\pgfpathcurveto{\pgfqpoint{11.247535in}{1.349076in}}{\pgfqpoint{11.238703in}{1.352735in}}{\pgfqpoint{11.229494in}{1.352735in}}%
\pgfpathcurveto{\pgfqpoint{11.220286in}{1.352735in}}{\pgfqpoint{11.211453in}{1.349076in}}{\pgfqpoint{11.204942in}{1.342565in}}%
\pgfpathcurveto{\pgfqpoint{11.198430in}{1.336054in}}{\pgfqpoint{11.194772in}{1.327221in}}{\pgfqpoint{11.194772in}{1.318013in}}%
\pgfpathcurveto{\pgfqpoint{11.194772in}{1.308804in}}{\pgfqpoint{11.198430in}{1.299972in}}{\pgfqpoint{11.204942in}{1.293460in}}%
\pgfpathcurveto{\pgfqpoint{11.211453in}{1.286949in}}{\pgfqpoint{11.220286in}{1.283290in}}{\pgfqpoint{11.229494in}{1.283290in}}%
\pgfpathlineto{\pgfqpoint{11.229494in}{1.283290in}}%
\pgfpathclose%
\pgfusepath{stroke}%
\end{pgfscope}%
\begin{pgfscope}%
\pgfpathrectangle{\pgfqpoint{0.688192in}{0.613042in}}{\pgfqpoint{11.096108in}{4.223431in}}%
\pgfusepath{clip}%
\pgfsetrectcap%
\pgfsetroundjoin%
\pgfsetlinewidth{0.803000pt}%
\definecolor{currentstroke}{rgb}{0.690196,0.690196,0.690196}%
\pgfsetstrokecolor{currentstroke}%
\pgfsetstrokeopacity{0.200000}%
\pgfsetdash{}{0pt}%
\pgfpathmoveto{\pgfqpoint{1.242997in}{0.613042in}}%
\pgfpathlineto{\pgfqpoint{1.242997in}{4.836473in}}%
\pgfusepath{stroke}%
\end{pgfscope}%
\begin{pgfscope}%
\pgfsetbuttcap%
\pgfsetroundjoin%
\definecolor{currentfill}{rgb}{0.000000,0.000000,0.000000}%
\pgfsetfillcolor{currentfill}%
\pgfsetlinewidth{0.803000pt}%
\definecolor{currentstroke}{rgb}{0.000000,0.000000,0.000000}%
\pgfsetstrokecolor{currentstroke}%
\pgfsetdash{}{0pt}%
\pgfsys@defobject{currentmarker}{\pgfqpoint{0.000000in}{-0.048611in}}{\pgfqpoint{0.000000in}{0.000000in}}{%
\pgfpathmoveto{\pgfqpoint{0.000000in}{0.000000in}}%
\pgfpathlineto{\pgfqpoint{0.000000in}{-0.048611in}}%
\pgfusepath{stroke,fill}%
}%
\begin{pgfscope}%
\pgfsys@transformshift{1.242997in}{0.613042in}%
\pgfsys@useobject{currentmarker}{}%
\end{pgfscope}%
\end{pgfscope}%
\begin{pgfscope}%
\definecolor{textcolor}{rgb}{0.000000,0.000000,0.000000}%
\pgfsetstrokecolor{textcolor}%
\pgfsetfillcolor{textcolor}%
\pgftext[x=0.984290in, y=0.284116in, left, base,rotate=12.500000]{\color{textcolor}{\rmfamily\fontsize{12.000000}{14.400000}\selectfont\catcode`\^=\active\def^{\ifmmode\sp\else\^{}\fi}\catcode`\%=\active\def%{\%}Nuclear}}%
\end{pgfscope}%
\begin{pgfscope}%
\pgfpathrectangle{\pgfqpoint{0.688192in}{0.613042in}}{\pgfqpoint{11.096108in}{4.223431in}}%
\pgfusepath{clip}%
\pgfsetrectcap%
\pgfsetroundjoin%
\pgfsetlinewidth{0.803000pt}%
\definecolor{currentstroke}{rgb}{0.690196,0.690196,0.690196}%
\pgfsetstrokecolor{currentstroke}%
\pgfsetstrokeopacity{0.200000}%
\pgfsetdash{}{0pt}%
\pgfpathmoveto{\pgfqpoint{2.352608in}{0.613042in}}%
\pgfpathlineto{\pgfqpoint{2.352608in}{4.836473in}}%
\pgfusepath{stroke}%
\end{pgfscope}%
\begin{pgfscope}%
\pgfsetbuttcap%
\pgfsetroundjoin%
\definecolor{currentfill}{rgb}{0.000000,0.000000,0.000000}%
\pgfsetfillcolor{currentfill}%
\pgfsetlinewidth{0.803000pt}%
\definecolor{currentstroke}{rgb}{0.000000,0.000000,0.000000}%
\pgfsetstrokecolor{currentstroke}%
\pgfsetdash{}{0pt}%
\pgfsys@defobject{currentmarker}{\pgfqpoint{0.000000in}{-0.048611in}}{\pgfqpoint{0.000000in}{0.000000in}}{%
\pgfpathmoveto{\pgfqpoint{0.000000in}{0.000000in}}%
\pgfpathlineto{\pgfqpoint{0.000000in}{-0.048611in}}%
\pgfusepath{stroke,fill}%
}%
\begin{pgfscope}%
\pgfsys@transformshift{2.352608in}{0.613042in}%
\pgfsys@useobject{currentmarker}{}%
\end{pgfscope}%
\end{pgfscope}%
\begin{pgfscope}%
\definecolor{textcolor}{rgb}{0.000000,0.000000,0.000000}%
\pgfsetstrokecolor{textcolor}%
\pgfsetfillcolor{textcolor}%
\pgftext[x=1.919203in, y=0.206657in, left, base,rotate=12.500000]{\color{textcolor}{\rmfamily\fontsize{12.000000}{14.400000}\selectfont\catcode`\^=\active\def^{\ifmmode\sp\else\^{}\fi}\catcode`\%=\active\def%{\%}Nuclear\_Adv}}%
\end{pgfscope}%
\begin{pgfscope}%
\pgfpathrectangle{\pgfqpoint{0.688192in}{0.613042in}}{\pgfqpoint{11.096108in}{4.223431in}}%
\pgfusepath{clip}%
\pgfsetrectcap%
\pgfsetroundjoin%
\pgfsetlinewidth{0.803000pt}%
\definecolor{currentstroke}{rgb}{0.690196,0.690196,0.690196}%
\pgfsetstrokecolor{currentstroke}%
\pgfsetstrokeopacity{0.200000}%
\pgfsetdash{}{0pt}%
\pgfpathmoveto{\pgfqpoint{3.462219in}{0.613042in}}%
\pgfpathlineto{\pgfqpoint{3.462219in}{4.836473in}}%
\pgfusepath{stroke}%
\end{pgfscope}%
\begin{pgfscope}%
\pgfsetbuttcap%
\pgfsetroundjoin%
\definecolor{currentfill}{rgb}{0.000000,0.000000,0.000000}%
\pgfsetfillcolor{currentfill}%
\pgfsetlinewidth{0.803000pt}%
\definecolor{currentstroke}{rgb}{0.000000,0.000000,0.000000}%
\pgfsetstrokecolor{currentstroke}%
\pgfsetdash{}{0pt}%
\pgfsys@defobject{currentmarker}{\pgfqpoint{0.000000in}{-0.048611in}}{\pgfqpoint{0.000000in}{0.000000in}}{%
\pgfpathmoveto{\pgfqpoint{0.000000in}{0.000000in}}%
\pgfpathlineto{\pgfqpoint{0.000000in}{-0.048611in}}%
\pgfusepath{stroke,fill}%
}%
\begin{pgfscope}%
\pgfsys@transformshift{3.462219in}{0.613042in}%
\pgfsys@useobject{currentmarker}{}%
\end{pgfscope}%
\end{pgfscope}%
\begin{pgfscope}%
\definecolor{textcolor}{rgb}{0.000000,0.000000,0.000000}%
\pgfsetstrokecolor{textcolor}%
\pgfsetfillcolor{textcolor}%
\pgftext[x=2.859621in, y=0.131639in, left, base,rotate=12.500000]{\color{textcolor}{\rmfamily\fontsize{12.000000}{14.400000}\selectfont\catcode`\^=\active\def^{\ifmmode\sp\else\^{}\fi}\catcode`\%=\active\def%{\%}NaturalGas\_Conv}}%
\end{pgfscope}%
\begin{pgfscope}%
\pgfpathrectangle{\pgfqpoint{0.688192in}{0.613042in}}{\pgfqpoint{11.096108in}{4.223431in}}%
\pgfusepath{clip}%
\pgfsetrectcap%
\pgfsetroundjoin%
\pgfsetlinewidth{0.803000pt}%
\definecolor{currentstroke}{rgb}{0.690196,0.690196,0.690196}%
\pgfsetstrokecolor{currentstroke}%
\pgfsetstrokeopacity{0.200000}%
\pgfsetdash{}{0pt}%
\pgfpathmoveto{\pgfqpoint{4.571829in}{0.613042in}}%
\pgfpathlineto{\pgfqpoint{4.571829in}{4.836473in}}%
\pgfusepath{stroke}%
\end{pgfscope}%
\begin{pgfscope}%
\pgfsetbuttcap%
\pgfsetroundjoin%
\definecolor{currentfill}{rgb}{0.000000,0.000000,0.000000}%
\pgfsetfillcolor{currentfill}%
\pgfsetlinewidth{0.803000pt}%
\definecolor{currentstroke}{rgb}{0.000000,0.000000,0.000000}%
\pgfsetstrokecolor{currentstroke}%
\pgfsetdash{}{0pt}%
\pgfsys@defobject{currentmarker}{\pgfqpoint{0.000000in}{-0.048611in}}{\pgfqpoint{0.000000in}{0.000000in}}{%
\pgfpathmoveto{\pgfqpoint{0.000000in}{0.000000in}}%
\pgfpathlineto{\pgfqpoint{0.000000in}{-0.048611in}}%
\pgfusepath{stroke,fill}%
}%
\begin{pgfscope}%
\pgfsys@transformshift{4.571829in}{0.613042in}%
\pgfsys@useobject{currentmarker}{}%
\end{pgfscope}%
\end{pgfscope}%
\begin{pgfscope}%
\definecolor{textcolor}{rgb}{0.000000,0.000000,0.000000}%
\pgfsetstrokecolor{textcolor}%
\pgfsetfillcolor{textcolor}%
\pgftext[x=4.004666in, y=0.147350in, left, base,rotate=12.500000]{\color{textcolor}{\rmfamily\fontsize{12.000000}{14.400000}\selectfont\catcode`\^=\active\def^{\ifmmode\sp\else\^{}\fi}\catcode`\%=\active\def%{\%}NaturalGas\_Adv}}%
\end{pgfscope}%
\begin{pgfscope}%
\pgfpathrectangle{\pgfqpoint{0.688192in}{0.613042in}}{\pgfqpoint{11.096108in}{4.223431in}}%
\pgfusepath{clip}%
\pgfsetrectcap%
\pgfsetroundjoin%
\pgfsetlinewidth{0.803000pt}%
\definecolor{currentstroke}{rgb}{0.690196,0.690196,0.690196}%
\pgfsetstrokecolor{currentstroke}%
\pgfsetstrokeopacity{0.200000}%
\pgfsetdash{}{0pt}%
\pgfpathmoveto{\pgfqpoint{5.681440in}{0.613042in}}%
\pgfpathlineto{\pgfqpoint{5.681440in}{4.836473in}}%
\pgfusepath{stroke}%
\end{pgfscope}%
\begin{pgfscope}%
\pgfsetbuttcap%
\pgfsetroundjoin%
\definecolor{currentfill}{rgb}{0.000000,0.000000,0.000000}%
\pgfsetfillcolor{currentfill}%
\pgfsetlinewidth{0.803000pt}%
\definecolor{currentstroke}{rgb}{0.000000,0.000000,0.000000}%
\pgfsetstrokecolor{currentstroke}%
\pgfsetdash{}{0pt}%
\pgfsys@defobject{currentmarker}{\pgfqpoint{0.000000in}{-0.048611in}}{\pgfqpoint{0.000000in}{0.000000in}}{%
\pgfpathmoveto{\pgfqpoint{0.000000in}{0.000000in}}%
\pgfpathlineto{\pgfqpoint{0.000000in}{-0.048611in}}%
\pgfusepath{stroke,fill}%
}%
\begin{pgfscope}%
\pgfsys@transformshift{5.681440in}{0.613042in}%
\pgfsys@useobject{currentmarker}{}%
\end{pgfscope}%
\end{pgfscope}%
\begin{pgfscope}%
\definecolor{textcolor}{rgb}{0.000000,0.000000,0.000000}%
\pgfsetstrokecolor{textcolor}%
\pgfsetfillcolor{textcolor}%
\pgftext[x=5.321003in, y=0.239010in, left, base,rotate=12.500000]{\color{textcolor}{\rmfamily\fontsize{12.000000}{14.400000}\selectfont\catcode`\^=\active\def^{\ifmmode\sp\else\^{}\fi}\catcode`\%=\active\def%{\%}Coal\_Conv}}%
\end{pgfscope}%
\begin{pgfscope}%
\pgfpathrectangle{\pgfqpoint{0.688192in}{0.613042in}}{\pgfqpoint{11.096108in}{4.223431in}}%
\pgfusepath{clip}%
\pgfsetrectcap%
\pgfsetroundjoin%
\pgfsetlinewidth{0.803000pt}%
\definecolor{currentstroke}{rgb}{0.690196,0.690196,0.690196}%
\pgfsetstrokecolor{currentstroke}%
\pgfsetstrokeopacity{0.200000}%
\pgfsetdash{}{0pt}%
\pgfpathmoveto{\pgfqpoint{6.791051in}{0.613042in}}%
\pgfpathlineto{\pgfqpoint{6.791051in}{4.836473in}}%
\pgfusepath{stroke}%
\end{pgfscope}%
\begin{pgfscope}%
\pgfsetbuttcap%
\pgfsetroundjoin%
\definecolor{currentfill}{rgb}{0.000000,0.000000,0.000000}%
\pgfsetfillcolor{currentfill}%
\pgfsetlinewidth{0.803000pt}%
\definecolor{currentstroke}{rgb}{0.000000,0.000000,0.000000}%
\pgfsetstrokecolor{currentstroke}%
\pgfsetdash{}{0pt}%
\pgfsys@defobject{currentmarker}{\pgfqpoint{0.000000in}{-0.048611in}}{\pgfqpoint{0.000000in}{0.000000in}}{%
\pgfpathmoveto{\pgfqpoint{0.000000in}{0.000000in}}%
\pgfpathlineto{\pgfqpoint{0.000000in}{-0.048611in}}%
\pgfusepath{stroke,fill}%
}%
\begin{pgfscope}%
\pgfsys@transformshift{6.791051in}{0.613042in}%
\pgfsys@useobject{currentmarker}{}%
\end{pgfscope}%
\end{pgfscope}%
\begin{pgfscope}%
\definecolor{textcolor}{rgb}{0.000000,0.000000,0.000000}%
\pgfsetstrokecolor{textcolor}%
\pgfsetfillcolor{textcolor}%
\pgftext[x=6.466047in, y=0.254721in, left, base,rotate=12.500000]{\color{textcolor}{\rmfamily\fontsize{12.000000}{14.400000}\selectfont\catcode`\^=\active\def^{\ifmmode\sp\else\^{}\fi}\catcode`\%=\active\def%{\%}Coal\_Adv}}%
\end{pgfscope}%
\begin{pgfscope}%
\pgfpathrectangle{\pgfqpoint{0.688192in}{0.613042in}}{\pgfqpoint{11.096108in}{4.223431in}}%
\pgfusepath{clip}%
\pgfsetrectcap%
\pgfsetroundjoin%
\pgfsetlinewidth{0.803000pt}%
\definecolor{currentstroke}{rgb}{0.690196,0.690196,0.690196}%
\pgfsetstrokecolor{currentstroke}%
\pgfsetstrokeopacity{0.200000}%
\pgfsetdash{}{0pt}%
\pgfpathmoveto{\pgfqpoint{7.900662in}{0.613042in}}%
\pgfpathlineto{\pgfqpoint{7.900662in}{4.836473in}}%
\pgfusepath{stroke}%
\end{pgfscope}%
\begin{pgfscope}%
\pgfsetbuttcap%
\pgfsetroundjoin%
\definecolor{currentfill}{rgb}{0.000000,0.000000,0.000000}%
\pgfsetfillcolor{currentfill}%
\pgfsetlinewidth{0.803000pt}%
\definecolor{currentstroke}{rgb}{0.000000,0.000000,0.000000}%
\pgfsetstrokecolor{currentstroke}%
\pgfsetdash{}{0pt}%
\pgfsys@defobject{currentmarker}{\pgfqpoint{0.000000in}{-0.048611in}}{\pgfqpoint{0.000000in}{0.000000in}}{%
\pgfpathmoveto{\pgfqpoint{0.000000in}{0.000000in}}%
\pgfpathlineto{\pgfqpoint{0.000000in}{-0.048611in}}%
\pgfusepath{stroke,fill}%
}%
\begin{pgfscope}%
\pgfsys@transformshift{7.900662in}{0.613042in}%
\pgfsys@useobject{currentmarker}{}%
\end{pgfscope}%
\end{pgfscope}%
\begin{pgfscope}%
\definecolor{textcolor}{rgb}{0.000000,0.000000,0.000000}%
\pgfsetstrokecolor{textcolor}%
\pgfsetfillcolor{textcolor}%
\pgftext[x=7.622246in, y=0.275378in, left, base,rotate=12.500000]{\color{textcolor}{\rmfamily\fontsize{12.000000}{14.400000}\selectfont\catcode`\^=\active\def^{\ifmmode\sp\else\^{}\fi}\catcode`\%=\active\def%{\%}Biomass}}%
\end{pgfscope}%
\begin{pgfscope}%
\pgfpathrectangle{\pgfqpoint{0.688192in}{0.613042in}}{\pgfqpoint{11.096108in}{4.223431in}}%
\pgfusepath{clip}%
\pgfsetrectcap%
\pgfsetroundjoin%
\pgfsetlinewidth{0.803000pt}%
\definecolor{currentstroke}{rgb}{0.690196,0.690196,0.690196}%
\pgfsetstrokecolor{currentstroke}%
\pgfsetstrokeopacity{0.200000}%
\pgfsetdash{}{0pt}%
\pgfpathmoveto{\pgfqpoint{9.010272in}{0.613042in}}%
\pgfpathlineto{\pgfqpoint{9.010272in}{4.836473in}}%
\pgfusepath{stroke}%
\end{pgfscope}%
\begin{pgfscope}%
\pgfsetbuttcap%
\pgfsetroundjoin%
\definecolor{currentfill}{rgb}{0.000000,0.000000,0.000000}%
\pgfsetfillcolor{currentfill}%
\pgfsetlinewidth{0.803000pt}%
\definecolor{currentstroke}{rgb}{0.000000,0.000000,0.000000}%
\pgfsetstrokecolor{currentstroke}%
\pgfsetdash{}{0pt}%
\pgfsys@defobject{currentmarker}{\pgfqpoint{0.000000in}{-0.048611in}}{\pgfqpoint{0.000000in}{0.000000in}}{%
\pgfpathmoveto{\pgfqpoint{0.000000in}{0.000000in}}%
\pgfpathlineto{\pgfqpoint{0.000000in}{-0.048611in}}%
\pgfusepath{stroke,fill}%
}%
\begin{pgfscope}%
\pgfsys@transformshift{9.010272in}{0.613042in}%
\pgfsys@useobject{currentmarker}{}%
\end{pgfscope}%
\end{pgfscope}%
\begin{pgfscope}%
\definecolor{textcolor}{rgb}{0.000000,0.000000,0.000000}%
\pgfsetstrokecolor{textcolor}%
\pgfsetfillcolor{textcolor}%
\pgftext[x=8.752658in, y=0.284600in, left, base,rotate=12.500000]{\color{textcolor}{\rmfamily\fontsize{12.000000}{14.400000}\selectfont\catcode`\^=\active\def^{\ifmmode\sp\else\^{}\fi}\catcode`\%=\active\def%{\%}Battery}}%
\end{pgfscope}%
\begin{pgfscope}%
\pgfpathrectangle{\pgfqpoint{0.688192in}{0.613042in}}{\pgfqpoint{11.096108in}{4.223431in}}%
\pgfusepath{clip}%
\pgfsetrectcap%
\pgfsetroundjoin%
\pgfsetlinewidth{0.803000pt}%
\definecolor{currentstroke}{rgb}{0.690196,0.690196,0.690196}%
\pgfsetstrokecolor{currentstroke}%
\pgfsetstrokeopacity{0.200000}%
\pgfsetdash{}{0pt}%
\pgfpathmoveto{\pgfqpoint{10.119883in}{0.613042in}}%
\pgfpathlineto{\pgfqpoint{10.119883in}{4.836473in}}%
\pgfusepath{stroke}%
\end{pgfscope}%
\begin{pgfscope}%
\pgfsetbuttcap%
\pgfsetroundjoin%
\definecolor{currentfill}{rgb}{0.000000,0.000000,0.000000}%
\pgfsetfillcolor{currentfill}%
\pgfsetlinewidth{0.803000pt}%
\definecolor{currentstroke}{rgb}{0.000000,0.000000,0.000000}%
\pgfsetstrokecolor{currentstroke}%
\pgfsetdash{}{0pt}%
\pgfsys@defobject{currentmarker}{\pgfqpoint{0.000000in}{-0.048611in}}{\pgfqpoint{0.000000in}{0.000000in}}{%
\pgfpathmoveto{\pgfqpoint{0.000000in}{0.000000in}}%
\pgfpathlineto{\pgfqpoint{0.000000in}{-0.048611in}}%
\pgfusepath{stroke,fill}%
}%
\begin{pgfscope}%
\pgfsys@transformshift{10.119883in}{0.613042in}%
\pgfsys@useobject{currentmarker}{}%
\end{pgfscope}%
\end{pgfscope}%
\begin{pgfscope}%
\definecolor{textcolor}{rgb}{0.000000,0.000000,0.000000}%
\pgfsetstrokecolor{textcolor}%
\pgfsetfillcolor{textcolor}%
\pgftext[x=9.758265in, y=0.238486in, left, base,rotate=12.500000]{\color{textcolor}{\rmfamily\fontsize{12.000000}{14.400000}\selectfont\catcode`\^=\active\def^{\ifmmode\sp\else\^{}\fi}\catcode`\%=\active\def%{\%}SolarPanel}}%
\end{pgfscope}%
\begin{pgfscope}%
\pgfpathrectangle{\pgfqpoint{0.688192in}{0.613042in}}{\pgfqpoint{11.096108in}{4.223431in}}%
\pgfusepath{clip}%
\pgfsetrectcap%
\pgfsetroundjoin%
\pgfsetlinewidth{0.803000pt}%
\definecolor{currentstroke}{rgb}{0.690196,0.690196,0.690196}%
\pgfsetstrokecolor{currentstroke}%
\pgfsetstrokeopacity{0.200000}%
\pgfsetdash{}{0pt}%
\pgfpathmoveto{\pgfqpoint{11.229494in}{0.613042in}}%
\pgfpathlineto{\pgfqpoint{11.229494in}{4.836473in}}%
\pgfusepath{stroke}%
\end{pgfscope}%
\begin{pgfscope}%
\pgfsetbuttcap%
\pgfsetroundjoin%
\definecolor{currentfill}{rgb}{0.000000,0.000000,0.000000}%
\pgfsetfillcolor{currentfill}%
\pgfsetlinewidth{0.803000pt}%
\definecolor{currentstroke}{rgb}{0.000000,0.000000,0.000000}%
\pgfsetstrokecolor{currentstroke}%
\pgfsetdash{}{0pt}%
\pgfsys@defobject{currentmarker}{\pgfqpoint{0.000000in}{-0.048611in}}{\pgfqpoint{0.000000in}{0.000000in}}{%
\pgfpathmoveto{\pgfqpoint{0.000000in}{0.000000in}}%
\pgfpathlineto{\pgfqpoint{0.000000in}{-0.048611in}}%
\pgfusepath{stroke,fill}%
}%
\begin{pgfscope}%
\pgfsys@transformshift{11.229494in}{0.613042in}%
\pgfsys@useobject{currentmarker}{}%
\end{pgfscope}%
\end{pgfscope}%
\begin{pgfscope}%
\definecolor{textcolor}{rgb}{0.000000,0.000000,0.000000}%
\pgfsetstrokecolor{textcolor}%
\pgfsetfillcolor{textcolor}%
\pgftext[x=10.773844in, y=0.196794in, left, base,rotate=12.500000]{\color{textcolor}{\rmfamily\fontsize{12.000000}{14.400000}\selectfont\catcode`\^=\active\def^{\ifmmode\sp\else\^{}\fi}\catcode`\%=\active\def%{\%}WindTurbine}}%
\end{pgfscope}%
\begin{pgfscope}%
\pgfpathrectangle{\pgfqpoint{0.688192in}{0.613042in}}{\pgfqpoint{11.096108in}{4.223431in}}%
\pgfusepath{clip}%
\pgfsetrectcap%
\pgfsetroundjoin%
\pgfsetlinewidth{0.803000pt}%
\definecolor{currentstroke}{rgb}{0.690196,0.690196,0.690196}%
\pgfsetstrokecolor{currentstroke}%
\pgfsetstrokeopacity{0.050000}%
\pgfsetdash{}{0pt}%
\pgfpathmoveto{\pgfqpoint{0.799153in}{0.613042in}}%
\pgfpathlineto{\pgfqpoint{0.799153in}{4.836473in}}%
\pgfusepath{stroke}%
\end{pgfscope}%
\begin{pgfscope}%
\pgfsetbuttcap%
\pgfsetroundjoin%
\definecolor{currentfill}{rgb}{0.000000,0.000000,0.000000}%
\pgfsetfillcolor{currentfill}%
\pgfsetlinewidth{0.602250pt}%
\definecolor{currentstroke}{rgb}{0.000000,0.000000,0.000000}%
\pgfsetstrokecolor{currentstroke}%
\pgfsetdash{}{0pt}%
\pgfsys@defobject{currentmarker}{\pgfqpoint{0.000000in}{-0.027778in}}{\pgfqpoint{0.000000in}{0.000000in}}{%
\pgfpathmoveto{\pgfqpoint{0.000000in}{0.000000in}}%
\pgfpathlineto{\pgfqpoint{0.000000in}{-0.027778in}}%
\pgfusepath{stroke,fill}%
}%
\begin{pgfscope}%
\pgfsys@transformshift{0.799153in}{0.613042in}%
\pgfsys@useobject{currentmarker}{}%
\end{pgfscope}%
\end{pgfscope}%
\begin{pgfscope}%
\pgfpathrectangle{\pgfqpoint{0.688192in}{0.613042in}}{\pgfqpoint{11.096108in}{4.223431in}}%
\pgfusepath{clip}%
\pgfsetrectcap%
\pgfsetroundjoin%
\pgfsetlinewidth{0.803000pt}%
\definecolor{currentstroke}{rgb}{0.690196,0.690196,0.690196}%
\pgfsetstrokecolor{currentstroke}%
\pgfsetstrokeopacity{0.050000}%
\pgfsetdash{}{0pt}%
\pgfpathmoveto{\pgfqpoint{1.021075in}{0.613042in}}%
\pgfpathlineto{\pgfqpoint{1.021075in}{4.836473in}}%
\pgfusepath{stroke}%
\end{pgfscope}%
\begin{pgfscope}%
\pgfsetbuttcap%
\pgfsetroundjoin%
\definecolor{currentfill}{rgb}{0.000000,0.000000,0.000000}%
\pgfsetfillcolor{currentfill}%
\pgfsetlinewidth{0.602250pt}%
\definecolor{currentstroke}{rgb}{0.000000,0.000000,0.000000}%
\pgfsetstrokecolor{currentstroke}%
\pgfsetdash{}{0pt}%
\pgfsys@defobject{currentmarker}{\pgfqpoint{0.000000in}{-0.027778in}}{\pgfqpoint{0.000000in}{0.000000in}}{%
\pgfpathmoveto{\pgfqpoint{0.000000in}{0.000000in}}%
\pgfpathlineto{\pgfqpoint{0.000000in}{-0.027778in}}%
\pgfusepath{stroke,fill}%
}%
\begin{pgfscope}%
\pgfsys@transformshift{1.021075in}{0.613042in}%
\pgfsys@useobject{currentmarker}{}%
\end{pgfscope}%
\end{pgfscope}%
\begin{pgfscope}%
\pgfpathrectangle{\pgfqpoint{0.688192in}{0.613042in}}{\pgfqpoint{11.096108in}{4.223431in}}%
\pgfusepath{clip}%
\pgfsetrectcap%
\pgfsetroundjoin%
\pgfsetlinewidth{0.803000pt}%
\definecolor{currentstroke}{rgb}{0.690196,0.690196,0.690196}%
\pgfsetstrokecolor{currentstroke}%
\pgfsetstrokeopacity{0.050000}%
\pgfsetdash{}{0pt}%
\pgfpathmoveto{\pgfqpoint{1.464919in}{0.613042in}}%
\pgfpathlineto{\pgfqpoint{1.464919in}{4.836473in}}%
\pgfusepath{stroke}%
\end{pgfscope}%
\begin{pgfscope}%
\pgfsetbuttcap%
\pgfsetroundjoin%
\definecolor{currentfill}{rgb}{0.000000,0.000000,0.000000}%
\pgfsetfillcolor{currentfill}%
\pgfsetlinewidth{0.602250pt}%
\definecolor{currentstroke}{rgb}{0.000000,0.000000,0.000000}%
\pgfsetstrokecolor{currentstroke}%
\pgfsetdash{}{0pt}%
\pgfsys@defobject{currentmarker}{\pgfqpoint{0.000000in}{-0.027778in}}{\pgfqpoint{0.000000in}{0.000000in}}{%
\pgfpathmoveto{\pgfqpoint{0.000000in}{0.000000in}}%
\pgfpathlineto{\pgfqpoint{0.000000in}{-0.027778in}}%
\pgfusepath{stroke,fill}%
}%
\begin{pgfscope}%
\pgfsys@transformshift{1.464919in}{0.613042in}%
\pgfsys@useobject{currentmarker}{}%
\end{pgfscope}%
\end{pgfscope}%
\begin{pgfscope}%
\pgfpathrectangle{\pgfqpoint{0.688192in}{0.613042in}}{\pgfqpoint{11.096108in}{4.223431in}}%
\pgfusepath{clip}%
\pgfsetrectcap%
\pgfsetroundjoin%
\pgfsetlinewidth{0.803000pt}%
\definecolor{currentstroke}{rgb}{0.690196,0.690196,0.690196}%
\pgfsetstrokecolor{currentstroke}%
\pgfsetstrokeopacity{0.050000}%
\pgfsetdash{}{0pt}%
\pgfpathmoveto{\pgfqpoint{1.686841in}{0.613042in}}%
\pgfpathlineto{\pgfqpoint{1.686841in}{4.836473in}}%
\pgfusepath{stroke}%
\end{pgfscope}%
\begin{pgfscope}%
\pgfsetbuttcap%
\pgfsetroundjoin%
\definecolor{currentfill}{rgb}{0.000000,0.000000,0.000000}%
\pgfsetfillcolor{currentfill}%
\pgfsetlinewidth{0.602250pt}%
\definecolor{currentstroke}{rgb}{0.000000,0.000000,0.000000}%
\pgfsetstrokecolor{currentstroke}%
\pgfsetdash{}{0pt}%
\pgfsys@defobject{currentmarker}{\pgfqpoint{0.000000in}{-0.027778in}}{\pgfqpoint{0.000000in}{0.000000in}}{%
\pgfpathmoveto{\pgfqpoint{0.000000in}{0.000000in}}%
\pgfpathlineto{\pgfqpoint{0.000000in}{-0.027778in}}%
\pgfusepath{stroke,fill}%
}%
\begin{pgfscope}%
\pgfsys@transformshift{1.686841in}{0.613042in}%
\pgfsys@useobject{currentmarker}{}%
\end{pgfscope}%
\end{pgfscope}%
\begin{pgfscope}%
\pgfpathrectangle{\pgfqpoint{0.688192in}{0.613042in}}{\pgfqpoint{11.096108in}{4.223431in}}%
\pgfusepath{clip}%
\pgfsetrectcap%
\pgfsetroundjoin%
\pgfsetlinewidth{0.803000pt}%
\definecolor{currentstroke}{rgb}{0.690196,0.690196,0.690196}%
\pgfsetstrokecolor{currentstroke}%
\pgfsetstrokeopacity{0.050000}%
\pgfsetdash{}{0pt}%
\pgfpathmoveto{\pgfqpoint{1.908763in}{0.613042in}}%
\pgfpathlineto{\pgfqpoint{1.908763in}{4.836473in}}%
\pgfusepath{stroke}%
\end{pgfscope}%
\begin{pgfscope}%
\pgfsetbuttcap%
\pgfsetroundjoin%
\definecolor{currentfill}{rgb}{0.000000,0.000000,0.000000}%
\pgfsetfillcolor{currentfill}%
\pgfsetlinewidth{0.602250pt}%
\definecolor{currentstroke}{rgb}{0.000000,0.000000,0.000000}%
\pgfsetstrokecolor{currentstroke}%
\pgfsetdash{}{0pt}%
\pgfsys@defobject{currentmarker}{\pgfqpoint{0.000000in}{-0.027778in}}{\pgfqpoint{0.000000in}{0.000000in}}{%
\pgfpathmoveto{\pgfqpoint{0.000000in}{0.000000in}}%
\pgfpathlineto{\pgfqpoint{0.000000in}{-0.027778in}}%
\pgfusepath{stroke,fill}%
}%
\begin{pgfscope}%
\pgfsys@transformshift{1.908763in}{0.613042in}%
\pgfsys@useobject{currentmarker}{}%
\end{pgfscope}%
\end{pgfscope}%
\begin{pgfscope}%
\pgfpathrectangle{\pgfqpoint{0.688192in}{0.613042in}}{\pgfqpoint{11.096108in}{4.223431in}}%
\pgfusepath{clip}%
\pgfsetrectcap%
\pgfsetroundjoin%
\pgfsetlinewidth{0.803000pt}%
\definecolor{currentstroke}{rgb}{0.690196,0.690196,0.690196}%
\pgfsetstrokecolor{currentstroke}%
\pgfsetstrokeopacity{0.050000}%
\pgfsetdash{}{0pt}%
\pgfpathmoveto{\pgfqpoint{2.130686in}{0.613042in}}%
\pgfpathlineto{\pgfqpoint{2.130686in}{4.836473in}}%
\pgfusepath{stroke}%
\end{pgfscope}%
\begin{pgfscope}%
\pgfsetbuttcap%
\pgfsetroundjoin%
\definecolor{currentfill}{rgb}{0.000000,0.000000,0.000000}%
\pgfsetfillcolor{currentfill}%
\pgfsetlinewidth{0.602250pt}%
\definecolor{currentstroke}{rgb}{0.000000,0.000000,0.000000}%
\pgfsetstrokecolor{currentstroke}%
\pgfsetdash{}{0pt}%
\pgfsys@defobject{currentmarker}{\pgfqpoint{0.000000in}{-0.027778in}}{\pgfqpoint{0.000000in}{0.000000in}}{%
\pgfpathmoveto{\pgfqpoint{0.000000in}{0.000000in}}%
\pgfpathlineto{\pgfqpoint{0.000000in}{-0.027778in}}%
\pgfusepath{stroke,fill}%
}%
\begin{pgfscope}%
\pgfsys@transformshift{2.130686in}{0.613042in}%
\pgfsys@useobject{currentmarker}{}%
\end{pgfscope}%
\end{pgfscope}%
\begin{pgfscope}%
\pgfpathrectangle{\pgfqpoint{0.688192in}{0.613042in}}{\pgfqpoint{11.096108in}{4.223431in}}%
\pgfusepath{clip}%
\pgfsetrectcap%
\pgfsetroundjoin%
\pgfsetlinewidth{0.803000pt}%
\definecolor{currentstroke}{rgb}{0.690196,0.690196,0.690196}%
\pgfsetstrokecolor{currentstroke}%
\pgfsetstrokeopacity{0.050000}%
\pgfsetdash{}{0pt}%
\pgfpathmoveto{\pgfqpoint{2.574530in}{0.613042in}}%
\pgfpathlineto{\pgfqpoint{2.574530in}{4.836473in}}%
\pgfusepath{stroke}%
\end{pgfscope}%
\begin{pgfscope}%
\pgfsetbuttcap%
\pgfsetroundjoin%
\definecolor{currentfill}{rgb}{0.000000,0.000000,0.000000}%
\pgfsetfillcolor{currentfill}%
\pgfsetlinewidth{0.602250pt}%
\definecolor{currentstroke}{rgb}{0.000000,0.000000,0.000000}%
\pgfsetstrokecolor{currentstroke}%
\pgfsetdash{}{0pt}%
\pgfsys@defobject{currentmarker}{\pgfqpoint{0.000000in}{-0.027778in}}{\pgfqpoint{0.000000in}{0.000000in}}{%
\pgfpathmoveto{\pgfqpoint{0.000000in}{0.000000in}}%
\pgfpathlineto{\pgfqpoint{0.000000in}{-0.027778in}}%
\pgfusepath{stroke,fill}%
}%
\begin{pgfscope}%
\pgfsys@transformshift{2.574530in}{0.613042in}%
\pgfsys@useobject{currentmarker}{}%
\end{pgfscope}%
\end{pgfscope}%
\begin{pgfscope}%
\pgfpathrectangle{\pgfqpoint{0.688192in}{0.613042in}}{\pgfqpoint{11.096108in}{4.223431in}}%
\pgfusepath{clip}%
\pgfsetrectcap%
\pgfsetroundjoin%
\pgfsetlinewidth{0.803000pt}%
\definecolor{currentstroke}{rgb}{0.690196,0.690196,0.690196}%
\pgfsetstrokecolor{currentstroke}%
\pgfsetstrokeopacity{0.050000}%
\pgfsetdash{}{0pt}%
\pgfpathmoveto{\pgfqpoint{2.796452in}{0.613042in}}%
\pgfpathlineto{\pgfqpoint{2.796452in}{4.836473in}}%
\pgfusepath{stroke}%
\end{pgfscope}%
\begin{pgfscope}%
\pgfsetbuttcap%
\pgfsetroundjoin%
\definecolor{currentfill}{rgb}{0.000000,0.000000,0.000000}%
\pgfsetfillcolor{currentfill}%
\pgfsetlinewidth{0.602250pt}%
\definecolor{currentstroke}{rgb}{0.000000,0.000000,0.000000}%
\pgfsetstrokecolor{currentstroke}%
\pgfsetdash{}{0pt}%
\pgfsys@defobject{currentmarker}{\pgfqpoint{0.000000in}{-0.027778in}}{\pgfqpoint{0.000000in}{0.000000in}}{%
\pgfpathmoveto{\pgfqpoint{0.000000in}{0.000000in}}%
\pgfpathlineto{\pgfqpoint{0.000000in}{-0.027778in}}%
\pgfusepath{stroke,fill}%
}%
\begin{pgfscope}%
\pgfsys@transformshift{2.796452in}{0.613042in}%
\pgfsys@useobject{currentmarker}{}%
\end{pgfscope}%
\end{pgfscope}%
\begin{pgfscope}%
\pgfpathrectangle{\pgfqpoint{0.688192in}{0.613042in}}{\pgfqpoint{11.096108in}{4.223431in}}%
\pgfusepath{clip}%
\pgfsetrectcap%
\pgfsetroundjoin%
\pgfsetlinewidth{0.803000pt}%
\definecolor{currentstroke}{rgb}{0.690196,0.690196,0.690196}%
\pgfsetstrokecolor{currentstroke}%
\pgfsetstrokeopacity{0.050000}%
\pgfsetdash{}{0pt}%
\pgfpathmoveto{\pgfqpoint{3.018374in}{0.613042in}}%
\pgfpathlineto{\pgfqpoint{3.018374in}{4.836473in}}%
\pgfusepath{stroke}%
\end{pgfscope}%
\begin{pgfscope}%
\pgfsetbuttcap%
\pgfsetroundjoin%
\definecolor{currentfill}{rgb}{0.000000,0.000000,0.000000}%
\pgfsetfillcolor{currentfill}%
\pgfsetlinewidth{0.602250pt}%
\definecolor{currentstroke}{rgb}{0.000000,0.000000,0.000000}%
\pgfsetstrokecolor{currentstroke}%
\pgfsetdash{}{0pt}%
\pgfsys@defobject{currentmarker}{\pgfqpoint{0.000000in}{-0.027778in}}{\pgfqpoint{0.000000in}{0.000000in}}{%
\pgfpathmoveto{\pgfqpoint{0.000000in}{0.000000in}}%
\pgfpathlineto{\pgfqpoint{0.000000in}{-0.027778in}}%
\pgfusepath{stroke,fill}%
}%
\begin{pgfscope}%
\pgfsys@transformshift{3.018374in}{0.613042in}%
\pgfsys@useobject{currentmarker}{}%
\end{pgfscope}%
\end{pgfscope}%
\begin{pgfscope}%
\pgfpathrectangle{\pgfqpoint{0.688192in}{0.613042in}}{\pgfqpoint{11.096108in}{4.223431in}}%
\pgfusepath{clip}%
\pgfsetrectcap%
\pgfsetroundjoin%
\pgfsetlinewidth{0.803000pt}%
\definecolor{currentstroke}{rgb}{0.690196,0.690196,0.690196}%
\pgfsetstrokecolor{currentstroke}%
\pgfsetstrokeopacity{0.050000}%
\pgfsetdash{}{0pt}%
\pgfpathmoveto{\pgfqpoint{3.240296in}{0.613042in}}%
\pgfpathlineto{\pgfqpoint{3.240296in}{4.836473in}}%
\pgfusepath{stroke}%
\end{pgfscope}%
\begin{pgfscope}%
\pgfsetbuttcap%
\pgfsetroundjoin%
\definecolor{currentfill}{rgb}{0.000000,0.000000,0.000000}%
\pgfsetfillcolor{currentfill}%
\pgfsetlinewidth{0.602250pt}%
\definecolor{currentstroke}{rgb}{0.000000,0.000000,0.000000}%
\pgfsetstrokecolor{currentstroke}%
\pgfsetdash{}{0pt}%
\pgfsys@defobject{currentmarker}{\pgfqpoint{0.000000in}{-0.027778in}}{\pgfqpoint{0.000000in}{0.000000in}}{%
\pgfpathmoveto{\pgfqpoint{0.000000in}{0.000000in}}%
\pgfpathlineto{\pgfqpoint{0.000000in}{-0.027778in}}%
\pgfusepath{stroke,fill}%
}%
\begin{pgfscope}%
\pgfsys@transformshift{3.240296in}{0.613042in}%
\pgfsys@useobject{currentmarker}{}%
\end{pgfscope}%
\end{pgfscope}%
\begin{pgfscope}%
\pgfpathrectangle{\pgfqpoint{0.688192in}{0.613042in}}{\pgfqpoint{11.096108in}{4.223431in}}%
\pgfusepath{clip}%
\pgfsetrectcap%
\pgfsetroundjoin%
\pgfsetlinewidth{0.803000pt}%
\definecolor{currentstroke}{rgb}{0.690196,0.690196,0.690196}%
\pgfsetstrokecolor{currentstroke}%
\pgfsetstrokeopacity{0.050000}%
\pgfsetdash{}{0pt}%
\pgfpathmoveto{\pgfqpoint{3.684141in}{0.613042in}}%
\pgfpathlineto{\pgfqpoint{3.684141in}{4.836473in}}%
\pgfusepath{stroke}%
\end{pgfscope}%
\begin{pgfscope}%
\pgfsetbuttcap%
\pgfsetroundjoin%
\definecolor{currentfill}{rgb}{0.000000,0.000000,0.000000}%
\pgfsetfillcolor{currentfill}%
\pgfsetlinewidth{0.602250pt}%
\definecolor{currentstroke}{rgb}{0.000000,0.000000,0.000000}%
\pgfsetstrokecolor{currentstroke}%
\pgfsetdash{}{0pt}%
\pgfsys@defobject{currentmarker}{\pgfqpoint{0.000000in}{-0.027778in}}{\pgfqpoint{0.000000in}{0.000000in}}{%
\pgfpathmoveto{\pgfqpoint{0.000000in}{0.000000in}}%
\pgfpathlineto{\pgfqpoint{0.000000in}{-0.027778in}}%
\pgfusepath{stroke,fill}%
}%
\begin{pgfscope}%
\pgfsys@transformshift{3.684141in}{0.613042in}%
\pgfsys@useobject{currentmarker}{}%
\end{pgfscope}%
\end{pgfscope}%
\begin{pgfscope}%
\pgfpathrectangle{\pgfqpoint{0.688192in}{0.613042in}}{\pgfqpoint{11.096108in}{4.223431in}}%
\pgfusepath{clip}%
\pgfsetrectcap%
\pgfsetroundjoin%
\pgfsetlinewidth{0.803000pt}%
\definecolor{currentstroke}{rgb}{0.690196,0.690196,0.690196}%
\pgfsetstrokecolor{currentstroke}%
\pgfsetstrokeopacity{0.050000}%
\pgfsetdash{}{0pt}%
\pgfpathmoveto{\pgfqpoint{3.906063in}{0.613042in}}%
\pgfpathlineto{\pgfqpoint{3.906063in}{4.836473in}}%
\pgfusepath{stroke}%
\end{pgfscope}%
\begin{pgfscope}%
\pgfsetbuttcap%
\pgfsetroundjoin%
\definecolor{currentfill}{rgb}{0.000000,0.000000,0.000000}%
\pgfsetfillcolor{currentfill}%
\pgfsetlinewidth{0.602250pt}%
\definecolor{currentstroke}{rgb}{0.000000,0.000000,0.000000}%
\pgfsetstrokecolor{currentstroke}%
\pgfsetdash{}{0pt}%
\pgfsys@defobject{currentmarker}{\pgfqpoint{0.000000in}{-0.027778in}}{\pgfqpoint{0.000000in}{0.000000in}}{%
\pgfpathmoveto{\pgfqpoint{0.000000in}{0.000000in}}%
\pgfpathlineto{\pgfqpoint{0.000000in}{-0.027778in}}%
\pgfusepath{stroke,fill}%
}%
\begin{pgfscope}%
\pgfsys@transformshift{3.906063in}{0.613042in}%
\pgfsys@useobject{currentmarker}{}%
\end{pgfscope}%
\end{pgfscope}%
\begin{pgfscope}%
\pgfpathrectangle{\pgfqpoint{0.688192in}{0.613042in}}{\pgfqpoint{11.096108in}{4.223431in}}%
\pgfusepath{clip}%
\pgfsetrectcap%
\pgfsetroundjoin%
\pgfsetlinewidth{0.803000pt}%
\definecolor{currentstroke}{rgb}{0.690196,0.690196,0.690196}%
\pgfsetstrokecolor{currentstroke}%
\pgfsetstrokeopacity{0.050000}%
\pgfsetdash{}{0pt}%
\pgfpathmoveto{\pgfqpoint{4.127985in}{0.613042in}}%
\pgfpathlineto{\pgfqpoint{4.127985in}{4.836473in}}%
\pgfusepath{stroke}%
\end{pgfscope}%
\begin{pgfscope}%
\pgfsetbuttcap%
\pgfsetroundjoin%
\definecolor{currentfill}{rgb}{0.000000,0.000000,0.000000}%
\pgfsetfillcolor{currentfill}%
\pgfsetlinewidth{0.602250pt}%
\definecolor{currentstroke}{rgb}{0.000000,0.000000,0.000000}%
\pgfsetstrokecolor{currentstroke}%
\pgfsetdash{}{0pt}%
\pgfsys@defobject{currentmarker}{\pgfqpoint{0.000000in}{-0.027778in}}{\pgfqpoint{0.000000in}{0.000000in}}{%
\pgfpathmoveto{\pgfqpoint{0.000000in}{0.000000in}}%
\pgfpathlineto{\pgfqpoint{0.000000in}{-0.027778in}}%
\pgfusepath{stroke,fill}%
}%
\begin{pgfscope}%
\pgfsys@transformshift{4.127985in}{0.613042in}%
\pgfsys@useobject{currentmarker}{}%
\end{pgfscope}%
\end{pgfscope}%
\begin{pgfscope}%
\pgfpathrectangle{\pgfqpoint{0.688192in}{0.613042in}}{\pgfqpoint{11.096108in}{4.223431in}}%
\pgfusepath{clip}%
\pgfsetrectcap%
\pgfsetroundjoin%
\pgfsetlinewidth{0.803000pt}%
\definecolor{currentstroke}{rgb}{0.690196,0.690196,0.690196}%
\pgfsetstrokecolor{currentstroke}%
\pgfsetstrokeopacity{0.050000}%
\pgfsetdash{}{0pt}%
\pgfpathmoveto{\pgfqpoint{4.349907in}{0.613042in}}%
\pgfpathlineto{\pgfqpoint{4.349907in}{4.836473in}}%
\pgfusepath{stroke}%
\end{pgfscope}%
\begin{pgfscope}%
\pgfsetbuttcap%
\pgfsetroundjoin%
\definecolor{currentfill}{rgb}{0.000000,0.000000,0.000000}%
\pgfsetfillcolor{currentfill}%
\pgfsetlinewidth{0.602250pt}%
\definecolor{currentstroke}{rgb}{0.000000,0.000000,0.000000}%
\pgfsetstrokecolor{currentstroke}%
\pgfsetdash{}{0pt}%
\pgfsys@defobject{currentmarker}{\pgfqpoint{0.000000in}{-0.027778in}}{\pgfqpoint{0.000000in}{0.000000in}}{%
\pgfpathmoveto{\pgfqpoint{0.000000in}{0.000000in}}%
\pgfpathlineto{\pgfqpoint{0.000000in}{-0.027778in}}%
\pgfusepath{stroke,fill}%
}%
\begin{pgfscope}%
\pgfsys@transformshift{4.349907in}{0.613042in}%
\pgfsys@useobject{currentmarker}{}%
\end{pgfscope}%
\end{pgfscope}%
\begin{pgfscope}%
\pgfpathrectangle{\pgfqpoint{0.688192in}{0.613042in}}{\pgfqpoint{11.096108in}{4.223431in}}%
\pgfusepath{clip}%
\pgfsetrectcap%
\pgfsetroundjoin%
\pgfsetlinewidth{0.803000pt}%
\definecolor{currentstroke}{rgb}{0.690196,0.690196,0.690196}%
\pgfsetstrokecolor{currentstroke}%
\pgfsetstrokeopacity{0.050000}%
\pgfsetdash{}{0pt}%
\pgfpathmoveto{\pgfqpoint{4.793751in}{0.613042in}}%
\pgfpathlineto{\pgfqpoint{4.793751in}{4.836473in}}%
\pgfusepath{stroke}%
\end{pgfscope}%
\begin{pgfscope}%
\pgfsetbuttcap%
\pgfsetroundjoin%
\definecolor{currentfill}{rgb}{0.000000,0.000000,0.000000}%
\pgfsetfillcolor{currentfill}%
\pgfsetlinewidth{0.602250pt}%
\definecolor{currentstroke}{rgb}{0.000000,0.000000,0.000000}%
\pgfsetstrokecolor{currentstroke}%
\pgfsetdash{}{0pt}%
\pgfsys@defobject{currentmarker}{\pgfqpoint{0.000000in}{-0.027778in}}{\pgfqpoint{0.000000in}{0.000000in}}{%
\pgfpathmoveto{\pgfqpoint{0.000000in}{0.000000in}}%
\pgfpathlineto{\pgfqpoint{0.000000in}{-0.027778in}}%
\pgfusepath{stroke,fill}%
}%
\begin{pgfscope}%
\pgfsys@transformshift{4.793751in}{0.613042in}%
\pgfsys@useobject{currentmarker}{}%
\end{pgfscope}%
\end{pgfscope}%
\begin{pgfscope}%
\pgfpathrectangle{\pgfqpoint{0.688192in}{0.613042in}}{\pgfqpoint{11.096108in}{4.223431in}}%
\pgfusepath{clip}%
\pgfsetrectcap%
\pgfsetroundjoin%
\pgfsetlinewidth{0.803000pt}%
\definecolor{currentstroke}{rgb}{0.690196,0.690196,0.690196}%
\pgfsetstrokecolor{currentstroke}%
\pgfsetstrokeopacity{0.050000}%
\pgfsetdash{}{0pt}%
\pgfpathmoveto{\pgfqpoint{5.015674in}{0.613042in}}%
\pgfpathlineto{\pgfqpoint{5.015674in}{4.836473in}}%
\pgfusepath{stroke}%
\end{pgfscope}%
\begin{pgfscope}%
\pgfsetbuttcap%
\pgfsetroundjoin%
\definecolor{currentfill}{rgb}{0.000000,0.000000,0.000000}%
\pgfsetfillcolor{currentfill}%
\pgfsetlinewidth{0.602250pt}%
\definecolor{currentstroke}{rgb}{0.000000,0.000000,0.000000}%
\pgfsetstrokecolor{currentstroke}%
\pgfsetdash{}{0pt}%
\pgfsys@defobject{currentmarker}{\pgfqpoint{0.000000in}{-0.027778in}}{\pgfqpoint{0.000000in}{0.000000in}}{%
\pgfpathmoveto{\pgfqpoint{0.000000in}{0.000000in}}%
\pgfpathlineto{\pgfqpoint{0.000000in}{-0.027778in}}%
\pgfusepath{stroke,fill}%
}%
\begin{pgfscope}%
\pgfsys@transformshift{5.015674in}{0.613042in}%
\pgfsys@useobject{currentmarker}{}%
\end{pgfscope}%
\end{pgfscope}%
\begin{pgfscope}%
\pgfpathrectangle{\pgfqpoint{0.688192in}{0.613042in}}{\pgfqpoint{11.096108in}{4.223431in}}%
\pgfusepath{clip}%
\pgfsetrectcap%
\pgfsetroundjoin%
\pgfsetlinewidth{0.803000pt}%
\definecolor{currentstroke}{rgb}{0.690196,0.690196,0.690196}%
\pgfsetstrokecolor{currentstroke}%
\pgfsetstrokeopacity{0.050000}%
\pgfsetdash{}{0pt}%
\pgfpathmoveto{\pgfqpoint{5.237596in}{0.613042in}}%
\pgfpathlineto{\pgfqpoint{5.237596in}{4.836473in}}%
\pgfusepath{stroke}%
\end{pgfscope}%
\begin{pgfscope}%
\pgfsetbuttcap%
\pgfsetroundjoin%
\definecolor{currentfill}{rgb}{0.000000,0.000000,0.000000}%
\pgfsetfillcolor{currentfill}%
\pgfsetlinewidth{0.602250pt}%
\definecolor{currentstroke}{rgb}{0.000000,0.000000,0.000000}%
\pgfsetstrokecolor{currentstroke}%
\pgfsetdash{}{0pt}%
\pgfsys@defobject{currentmarker}{\pgfqpoint{0.000000in}{-0.027778in}}{\pgfqpoint{0.000000in}{0.000000in}}{%
\pgfpathmoveto{\pgfqpoint{0.000000in}{0.000000in}}%
\pgfpathlineto{\pgfqpoint{0.000000in}{-0.027778in}}%
\pgfusepath{stroke,fill}%
}%
\begin{pgfscope}%
\pgfsys@transformshift{5.237596in}{0.613042in}%
\pgfsys@useobject{currentmarker}{}%
\end{pgfscope}%
\end{pgfscope}%
\begin{pgfscope}%
\pgfpathrectangle{\pgfqpoint{0.688192in}{0.613042in}}{\pgfqpoint{11.096108in}{4.223431in}}%
\pgfusepath{clip}%
\pgfsetrectcap%
\pgfsetroundjoin%
\pgfsetlinewidth{0.803000pt}%
\definecolor{currentstroke}{rgb}{0.690196,0.690196,0.690196}%
\pgfsetstrokecolor{currentstroke}%
\pgfsetstrokeopacity{0.050000}%
\pgfsetdash{}{0pt}%
\pgfpathmoveto{\pgfqpoint{5.459518in}{0.613042in}}%
\pgfpathlineto{\pgfqpoint{5.459518in}{4.836473in}}%
\pgfusepath{stroke}%
\end{pgfscope}%
\begin{pgfscope}%
\pgfsetbuttcap%
\pgfsetroundjoin%
\definecolor{currentfill}{rgb}{0.000000,0.000000,0.000000}%
\pgfsetfillcolor{currentfill}%
\pgfsetlinewidth{0.602250pt}%
\definecolor{currentstroke}{rgb}{0.000000,0.000000,0.000000}%
\pgfsetstrokecolor{currentstroke}%
\pgfsetdash{}{0pt}%
\pgfsys@defobject{currentmarker}{\pgfqpoint{0.000000in}{-0.027778in}}{\pgfqpoint{0.000000in}{0.000000in}}{%
\pgfpathmoveto{\pgfqpoint{0.000000in}{0.000000in}}%
\pgfpathlineto{\pgfqpoint{0.000000in}{-0.027778in}}%
\pgfusepath{stroke,fill}%
}%
\begin{pgfscope}%
\pgfsys@transformshift{5.459518in}{0.613042in}%
\pgfsys@useobject{currentmarker}{}%
\end{pgfscope}%
\end{pgfscope}%
\begin{pgfscope}%
\pgfpathrectangle{\pgfqpoint{0.688192in}{0.613042in}}{\pgfqpoint{11.096108in}{4.223431in}}%
\pgfusepath{clip}%
\pgfsetrectcap%
\pgfsetroundjoin%
\pgfsetlinewidth{0.803000pt}%
\definecolor{currentstroke}{rgb}{0.690196,0.690196,0.690196}%
\pgfsetstrokecolor{currentstroke}%
\pgfsetstrokeopacity{0.050000}%
\pgfsetdash{}{0pt}%
\pgfpathmoveto{\pgfqpoint{5.903362in}{0.613042in}}%
\pgfpathlineto{\pgfqpoint{5.903362in}{4.836473in}}%
\pgfusepath{stroke}%
\end{pgfscope}%
\begin{pgfscope}%
\pgfsetbuttcap%
\pgfsetroundjoin%
\definecolor{currentfill}{rgb}{0.000000,0.000000,0.000000}%
\pgfsetfillcolor{currentfill}%
\pgfsetlinewidth{0.602250pt}%
\definecolor{currentstroke}{rgb}{0.000000,0.000000,0.000000}%
\pgfsetstrokecolor{currentstroke}%
\pgfsetdash{}{0pt}%
\pgfsys@defobject{currentmarker}{\pgfqpoint{0.000000in}{-0.027778in}}{\pgfqpoint{0.000000in}{0.000000in}}{%
\pgfpathmoveto{\pgfqpoint{0.000000in}{0.000000in}}%
\pgfpathlineto{\pgfqpoint{0.000000in}{-0.027778in}}%
\pgfusepath{stroke,fill}%
}%
\begin{pgfscope}%
\pgfsys@transformshift{5.903362in}{0.613042in}%
\pgfsys@useobject{currentmarker}{}%
\end{pgfscope}%
\end{pgfscope}%
\begin{pgfscope}%
\pgfpathrectangle{\pgfqpoint{0.688192in}{0.613042in}}{\pgfqpoint{11.096108in}{4.223431in}}%
\pgfusepath{clip}%
\pgfsetrectcap%
\pgfsetroundjoin%
\pgfsetlinewidth{0.803000pt}%
\definecolor{currentstroke}{rgb}{0.690196,0.690196,0.690196}%
\pgfsetstrokecolor{currentstroke}%
\pgfsetstrokeopacity{0.050000}%
\pgfsetdash{}{0pt}%
\pgfpathmoveto{\pgfqpoint{6.125284in}{0.613042in}}%
\pgfpathlineto{\pgfqpoint{6.125284in}{4.836473in}}%
\pgfusepath{stroke}%
\end{pgfscope}%
\begin{pgfscope}%
\pgfsetbuttcap%
\pgfsetroundjoin%
\definecolor{currentfill}{rgb}{0.000000,0.000000,0.000000}%
\pgfsetfillcolor{currentfill}%
\pgfsetlinewidth{0.602250pt}%
\definecolor{currentstroke}{rgb}{0.000000,0.000000,0.000000}%
\pgfsetstrokecolor{currentstroke}%
\pgfsetdash{}{0pt}%
\pgfsys@defobject{currentmarker}{\pgfqpoint{0.000000in}{-0.027778in}}{\pgfqpoint{0.000000in}{0.000000in}}{%
\pgfpathmoveto{\pgfqpoint{0.000000in}{0.000000in}}%
\pgfpathlineto{\pgfqpoint{0.000000in}{-0.027778in}}%
\pgfusepath{stroke,fill}%
}%
\begin{pgfscope}%
\pgfsys@transformshift{6.125284in}{0.613042in}%
\pgfsys@useobject{currentmarker}{}%
\end{pgfscope}%
\end{pgfscope}%
\begin{pgfscope}%
\pgfpathrectangle{\pgfqpoint{0.688192in}{0.613042in}}{\pgfqpoint{11.096108in}{4.223431in}}%
\pgfusepath{clip}%
\pgfsetrectcap%
\pgfsetroundjoin%
\pgfsetlinewidth{0.803000pt}%
\definecolor{currentstroke}{rgb}{0.690196,0.690196,0.690196}%
\pgfsetstrokecolor{currentstroke}%
\pgfsetstrokeopacity{0.050000}%
\pgfsetdash{}{0pt}%
\pgfpathmoveto{\pgfqpoint{6.347207in}{0.613042in}}%
\pgfpathlineto{\pgfqpoint{6.347207in}{4.836473in}}%
\pgfusepath{stroke}%
\end{pgfscope}%
\begin{pgfscope}%
\pgfsetbuttcap%
\pgfsetroundjoin%
\definecolor{currentfill}{rgb}{0.000000,0.000000,0.000000}%
\pgfsetfillcolor{currentfill}%
\pgfsetlinewidth{0.602250pt}%
\definecolor{currentstroke}{rgb}{0.000000,0.000000,0.000000}%
\pgfsetstrokecolor{currentstroke}%
\pgfsetdash{}{0pt}%
\pgfsys@defobject{currentmarker}{\pgfqpoint{0.000000in}{-0.027778in}}{\pgfqpoint{0.000000in}{0.000000in}}{%
\pgfpathmoveto{\pgfqpoint{0.000000in}{0.000000in}}%
\pgfpathlineto{\pgfqpoint{0.000000in}{-0.027778in}}%
\pgfusepath{stroke,fill}%
}%
\begin{pgfscope}%
\pgfsys@transformshift{6.347207in}{0.613042in}%
\pgfsys@useobject{currentmarker}{}%
\end{pgfscope}%
\end{pgfscope}%
\begin{pgfscope}%
\pgfpathrectangle{\pgfqpoint{0.688192in}{0.613042in}}{\pgfqpoint{11.096108in}{4.223431in}}%
\pgfusepath{clip}%
\pgfsetrectcap%
\pgfsetroundjoin%
\pgfsetlinewidth{0.803000pt}%
\definecolor{currentstroke}{rgb}{0.690196,0.690196,0.690196}%
\pgfsetstrokecolor{currentstroke}%
\pgfsetstrokeopacity{0.050000}%
\pgfsetdash{}{0pt}%
\pgfpathmoveto{\pgfqpoint{6.569129in}{0.613042in}}%
\pgfpathlineto{\pgfqpoint{6.569129in}{4.836473in}}%
\pgfusepath{stroke}%
\end{pgfscope}%
\begin{pgfscope}%
\pgfsetbuttcap%
\pgfsetroundjoin%
\definecolor{currentfill}{rgb}{0.000000,0.000000,0.000000}%
\pgfsetfillcolor{currentfill}%
\pgfsetlinewidth{0.602250pt}%
\definecolor{currentstroke}{rgb}{0.000000,0.000000,0.000000}%
\pgfsetstrokecolor{currentstroke}%
\pgfsetdash{}{0pt}%
\pgfsys@defobject{currentmarker}{\pgfqpoint{0.000000in}{-0.027778in}}{\pgfqpoint{0.000000in}{0.000000in}}{%
\pgfpathmoveto{\pgfqpoint{0.000000in}{0.000000in}}%
\pgfpathlineto{\pgfqpoint{0.000000in}{-0.027778in}}%
\pgfusepath{stroke,fill}%
}%
\begin{pgfscope}%
\pgfsys@transformshift{6.569129in}{0.613042in}%
\pgfsys@useobject{currentmarker}{}%
\end{pgfscope}%
\end{pgfscope}%
\begin{pgfscope}%
\pgfpathrectangle{\pgfqpoint{0.688192in}{0.613042in}}{\pgfqpoint{11.096108in}{4.223431in}}%
\pgfusepath{clip}%
\pgfsetrectcap%
\pgfsetroundjoin%
\pgfsetlinewidth{0.803000pt}%
\definecolor{currentstroke}{rgb}{0.690196,0.690196,0.690196}%
\pgfsetstrokecolor{currentstroke}%
\pgfsetstrokeopacity{0.050000}%
\pgfsetdash{}{0pt}%
\pgfpathmoveto{\pgfqpoint{7.012973in}{0.613042in}}%
\pgfpathlineto{\pgfqpoint{7.012973in}{4.836473in}}%
\pgfusepath{stroke}%
\end{pgfscope}%
\begin{pgfscope}%
\pgfsetbuttcap%
\pgfsetroundjoin%
\definecolor{currentfill}{rgb}{0.000000,0.000000,0.000000}%
\pgfsetfillcolor{currentfill}%
\pgfsetlinewidth{0.602250pt}%
\definecolor{currentstroke}{rgb}{0.000000,0.000000,0.000000}%
\pgfsetstrokecolor{currentstroke}%
\pgfsetdash{}{0pt}%
\pgfsys@defobject{currentmarker}{\pgfqpoint{0.000000in}{-0.027778in}}{\pgfqpoint{0.000000in}{0.000000in}}{%
\pgfpathmoveto{\pgfqpoint{0.000000in}{0.000000in}}%
\pgfpathlineto{\pgfqpoint{0.000000in}{-0.027778in}}%
\pgfusepath{stroke,fill}%
}%
\begin{pgfscope}%
\pgfsys@transformshift{7.012973in}{0.613042in}%
\pgfsys@useobject{currentmarker}{}%
\end{pgfscope}%
\end{pgfscope}%
\begin{pgfscope}%
\pgfpathrectangle{\pgfqpoint{0.688192in}{0.613042in}}{\pgfqpoint{11.096108in}{4.223431in}}%
\pgfusepath{clip}%
\pgfsetrectcap%
\pgfsetroundjoin%
\pgfsetlinewidth{0.803000pt}%
\definecolor{currentstroke}{rgb}{0.690196,0.690196,0.690196}%
\pgfsetstrokecolor{currentstroke}%
\pgfsetstrokeopacity{0.050000}%
\pgfsetdash{}{0pt}%
\pgfpathmoveto{\pgfqpoint{7.234895in}{0.613042in}}%
\pgfpathlineto{\pgfqpoint{7.234895in}{4.836473in}}%
\pgfusepath{stroke}%
\end{pgfscope}%
\begin{pgfscope}%
\pgfsetbuttcap%
\pgfsetroundjoin%
\definecolor{currentfill}{rgb}{0.000000,0.000000,0.000000}%
\pgfsetfillcolor{currentfill}%
\pgfsetlinewidth{0.602250pt}%
\definecolor{currentstroke}{rgb}{0.000000,0.000000,0.000000}%
\pgfsetstrokecolor{currentstroke}%
\pgfsetdash{}{0pt}%
\pgfsys@defobject{currentmarker}{\pgfqpoint{0.000000in}{-0.027778in}}{\pgfqpoint{0.000000in}{0.000000in}}{%
\pgfpathmoveto{\pgfqpoint{0.000000in}{0.000000in}}%
\pgfpathlineto{\pgfqpoint{0.000000in}{-0.027778in}}%
\pgfusepath{stroke,fill}%
}%
\begin{pgfscope}%
\pgfsys@transformshift{7.234895in}{0.613042in}%
\pgfsys@useobject{currentmarker}{}%
\end{pgfscope}%
\end{pgfscope}%
\begin{pgfscope}%
\pgfpathrectangle{\pgfqpoint{0.688192in}{0.613042in}}{\pgfqpoint{11.096108in}{4.223431in}}%
\pgfusepath{clip}%
\pgfsetrectcap%
\pgfsetroundjoin%
\pgfsetlinewidth{0.803000pt}%
\definecolor{currentstroke}{rgb}{0.690196,0.690196,0.690196}%
\pgfsetstrokecolor{currentstroke}%
\pgfsetstrokeopacity{0.050000}%
\pgfsetdash{}{0pt}%
\pgfpathmoveto{\pgfqpoint{7.456817in}{0.613042in}}%
\pgfpathlineto{\pgfqpoint{7.456817in}{4.836473in}}%
\pgfusepath{stroke}%
\end{pgfscope}%
\begin{pgfscope}%
\pgfsetbuttcap%
\pgfsetroundjoin%
\definecolor{currentfill}{rgb}{0.000000,0.000000,0.000000}%
\pgfsetfillcolor{currentfill}%
\pgfsetlinewidth{0.602250pt}%
\definecolor{currentstroke}{rgb}{0.000000,0.000000,0.000000}%
\pgfsetstrokecolor{currentstroke}%
\pgfsetdash{}{0pt}%
\pgfsys@defobject{currentmarker}{\pgfqpoint{0.000000in}{-0.027778in}}{\pgfqpoint{0.000000in}{0.000000in}}{%
\pgfpathmoveto{\pgfqpoint{0.000000in}{0.000000in}}%
\pgfpathlineto{\pgfqpoint{0.000000in}{-0.027778in}}%
\pgfusepath{stroke,fill}%
}%
\begin{pgfscope}%
\pgfsys@transformshift{7.456817in}{0.613042in}%
\pgfsys@useobject{currentmarker}{}%
\end{pgfscope}%
\end{pgfscope}%
\begin{pgfscope}%
\pgfpathrectangle{\pgfqpoint{0.688192in}{0.613042in}}{\pgfqpoint{11.096108in}{4.223431in}}%
\pgfusepath{clip}%
\pgfsetrectcap%
\pgfsetroundjoin%
\pgfsetlinewidth{0.803000pt}%
\definecolor{currentstroke}{rgb}{0.690196,0.690196,0.690196}%
\pgfsetstrokecolor{currentstroke}%
\pgfsetstrokeopacity{0.050000}%
\pgfsetdash{}{0pt}%
\pgfpathmoveto{\pgfqpoint{7.678740in}{0.613042in}}%
\pgfpathlineto{\pgfqpoint{7.678740in}{4.836473in}}%
\pgfusepath{stroke}%
\end{pgfscope}%
\begin{pgfscope}%
\pgfsetbuttcap%
\pgfsetroundjoin%
\definecolor{currentfill}{rgb}{0.000000,0.000000,0.000000}%
\pgfsetfillcolor{currentfill}%
\pgfsetlinewidth{0.602250pt}%
\definecolor{currentstroke}{rgb}{0.000000,0.000000,0.000000}%
\pgfsetstrokecolor{currentstroke}%
\pgfsetdash{}{0pt}%
\pgfsys@defobject{currentmarker}{\pgfqpoint{0.000000in}{-0.027778in}}{\pgfqpoint{0.000000in}{0.000000in}}{%
\pgfpathmoveto{\pgfqpoint{0.000000in}{0.000000in}}%
\pgfpathlineto{\pgfqpoint{0.000000in}{-0.027778in}}%
\pgfusepath{stroke,fill}%
}%
\begin{pgfscope}%
\pgfsys@transformshift{7.678740in}{0.613042in}%
\pgfsys@useobject{currentmarker}{}%
\end{pgfscope}%
\end{pgfscope}%
\begin{pgfscope}%
\pgfpathrectangle{\pgfqpoint{0.688192in}{0.613042in}}{\pgfqpoint{11.096108in}{4.223431in}}%
\pgfusepath{clip}%
\pgfsetrectcap%
\pgfsetroundjoin%
\pgfsetlinewidth{0.803000pt}%
\definecolor{currentstroke}{rgb}{0.690196,0.690196,0.690196}%
\pgfsetstrokecolor{currentstroke}%
\pgfsetstrokeopacity{0.050000}%
\pgfsetdash{}{0pt}%
\pgfpathmoveto{\pgfqpoint{8.122584in}{0.613042in}}%
\pgfpathlineto{\pgfqpoint{8.122584in}{4.836473in}}%
\pgfusepath{stroke}%
\end{pgfscope}%
\begin{pgfscope}%
\pgfsetbuttcap%
\pgfsetroundjoin%
\definecolor{currentfill}{rgb}{0.000000,0.000000,0.000000}%
\pgfsetfillcolor{currentfill}%
\pgfsetlinewidth{0.602250pt}%
\definecolor{currentstroke}{rgb}{0.000000,0.000000,0.000000}%
\pgfsetstrokecolor{currentstroke}%
\pgfsetdash{}{0pt}%
\pgfsys@defobject{currentmarker}{\pgfqpoint{0.000000in}{-0.027778in}}{\pgfqpoint{0.000000in}{0.000000in}}{%
\pgfpathmoveto{\pgfqpoint{0.000000in}{0.000000in}}%
\pgfpathlineto{\pgfqpoint{0.000000in}{-0.027778in}}%
\pgfusepath{stroke,fill}%
}%
\begin{pgfscope}%
\pgfsys@transformshift{8.122584in}{0.613042in}%
\pgfsys@useobject{currentmarker}{}%
\end{pgfscope}%
\end{pgfscope}%
\begin{pgfscope}%
\pgfpathrectangle{\pgfqpoint{0.688192in}{0.613042in}}{\pgfqpoint{11.096108in}{4.223431in}}%
\pgfusepath{clip}%
\pgfsetrectcap%
\pgfsetroundjoin%
\pgfsetlinewidth{0.803000pt}%
\definecolor{currentstroke}{rgb}{0.690196,0.690196,0.690196}%
\pgfsetstrokecolor{currentstroke}%
\pgfsetstrokeopacity{0.050000}%
\pgfsetdash{}{0pt}%
\pgfpathmoveto{\pgfqpoint{8.344506in}{0.613042in}}%
\pgfpathlineto{\pgfqpoint{8.344506in}{4.836473in}}%
\pgfusepath{stroke}%
\end{pgfscope}%
\begin{pgfscope}%
\pgfsetbuttcap%
\pgfsetroundjoin%
\definecolor{currentfill}{rgb}{0.000000,0.000000,0.000000}%
\pgfsetfillcolor{currentfill}%
\pgfsetlinewidth{0.602250pt}%
\definecolor{currentstroke}{rgb}{0.000000,0.000000,0.000000}%
\pgfsetstrokecolor{currentstroke}%
\pgfsetdash{}{0pt}%
\pgfsys@defobject{currentmarker}{\pgfqpoint{0.000000in}{-0.027778in}}{\pgfqpoint{0.000000in}{0.000000in}}{%
\pgfpathmoveto{\pgfqpoint{0.000000in}{0.000000in}}%
\pgfpathlineto{\pgfqpoint{0.000000in}{-0.027778in}}%
\pgfusepath{stroke,fill}%
}%
\begin{pgfscope}%
\pgfsys@transformshift{8.344506in}{0.613042in}%
\pgfsys@useobject{currentmarker}{}%
\end{pgfscope}%
\end{pgfscope}%
\begin{pgfscope}%
\pgfpathrectangle{\pgfqpoint{0.688192in}{0.613042in}}{\pgfqpoint{11.096108in}{4.223431in}}%
\pgfusepath{clip}%
\pgfsetrectcap%
\pgfsetroundjoin%
\pgfsetlinewidth{0.803000pt}%
\definecolor{currentstroke}{rgb}{0.690196,0.690196,0.690196}%
\pgfsetstrokecolor{currentstroke}%
\pgfsetstrokeopacity{0.050000}%
\pgfsetdash{}{0pt}%
\pgfpathmoveto{\pgfqpoint{8.566428in}{0.613042in}}%
\pgfpathlineto{\pgfqpoint{8.566428in}{4.836473in}}%
\pgfusepath{stroke}%
\end{pgfscope}%
\begin{pgfscope}%
\pgfsetbuttcap%
\pgfsetroundjoin%
\definecolor{currentfill}{rgb}{0.000000,0.000000,0.000000}%
\pgfsetfillcolor{currentfill}%
\pgfsetlinewidth{0.602250pt}%
\definecolor{currentstroke}{rgb}{0.000000,0.000000,0.000000}%
\pgfsetstrokecolor{currentstroke}%
\pgfsetdash{}{0pt}%
\pgfsys@defobject{currentmarker}{\pgfqpoint{0.000000in}{-0.027778in}}{\pgfqpoint{0.000000in}{0.000000in}}{%
\pgfpathmoveto{\pgfqpoint{0.000000in}{0.000000in}}%
\pgfpathlineto{\pgfqpoint{0.000000in}{-0.027778in}}%
\pgfusepath{stroke,fill}%
}%
\begin{pgfscope}%
\pgfsys@transformshift{8.566428in}{0.613042in}%
\pgfsys@useobject{currentmarker}{}%
\end{pgfscope}%
\end{pgfscope}%
\begin{pgfscope}%
\pgfpathrectangle{\pgfqpoint{0.688192in}{0.613042in}}{\pgfqpoint{11.096108in}{4.223431in}}%
\pgfusepath{clip}%
\pgfsetrectcap%
\pgfsetroundjoin%
\pgfsetlinewidth{0.803000pt}%
\definecolor{currentstroke}{rgb}{0.690196,0.690196,0.690196}%
\pgfsetstrokecolor{currentstroke}%
\pgfsetstrokeopacity{0.050000}%
\pgfsetdash{}{0pt}%
\pgfpathmoveto{\pgfqpoint{8.788350in}{0.613042in}}%
\pgfpathlineto{\pgfqpoint{8.788350in}{4.836473in}}%
\pgfusepath{stroke}%
\end{pgfscope}%
\begin{pgfscope}%
\pgfsetbuttcap%
\pgfsetroundjoin%
\definecolor{currentfill}{rgb}{0.000000,0.000000,0.000000}%
\pgfsetfillcolor{currentfill}%
\pgfsetlinewidth{0.602250pt}%
\definecolor{currentstroke}{rgb}{0.000000,0.000000,0.000000}%
\pgfsetstrokecolor{currentstroke}%
\pgfsetdash{}{0pt}%
\pgfsys@defobject{currentmarker}{\pgfqpoint{0.000000in}{-0.027778in}}{\pgfqpoint{0.000000in}{0.000000in}}{%
\pgfpathmoveto{\pgfqpoint{0.000000in}{0.000000in}}%
\pgfpathlineto{\pgfqpoint{0.000000in}{-0.027778in}}%
\pgfusepath{stroke,fill}%
}%
\begin{pgfscope}%
\pgfsys@transformshift{8.788350in}{0.613042in}%
\pgfsys@useobject{currentmarker}{}%
\end{pgfscope}%
\end{pgfscope}%
\begin{pgfscope}%
\pgfpathrectangle{\pgfqpoint{0.688192in}{0.613042in}}{\pgfqpoint{11.096108in}{4.223431in}}%
\pgfusepath{clip}%
\pgfsetrectcap%
\pgfsetroundjoin%
\pgfsetlinewidth{0.803000pt}%
\definecolor{currentstroke}{rgb}{0.690196,0.690196,0.690196}%
\pgfsetstrokecolor{currentstroke}%
\pgfsetstrokeopacity{0.050000}%
\pgfsetdash{}{0pt}%
\pgfpathmoveto{\pgfqpoint{9.232195in}{0.613042in}}%
\pgfpathlineto{\pgfqpoint{9.232195in}{4.836473in}}%
\pgfusepath{stroke}%
\end{pgfscope}%
\begin{pgfscope}%
\pgfsetbuttcap%
\pgfsetroundjoin%
\definecolor{currentfill}{rgb}{0.000000,0.000000,0.000000}%
\pgfsetfillcolor{currentfill}%
\pgfsetlinewidth{0.602250pt}%
\definecolor{currentstroke}{rgb}{0.000000,0.000000,0.000000}%
\pgfsetstrokecolor{currentstroke}%
\pgfsetdash{}{0pt}%
\pgfsys@defobject{currentmarker}{\pgfqpoint{0.000000in}{-0.027778in}}{\pgfqpoint{0.000000in}{0.000000in}}{%
\pgfpathmoveto{\pgfqpoint{0.000000in}{0.000000in}}%
\pgfpathlineto{\pgfqpoint{0.000000in}{-0.027778in}}%
\pgfusepath{stroke,fill}%
}%
\begin{pgfscope}%
\pgfsys@transformshift{9.232195in}{0.613042in}%
\pgfsys@useobject{currentmarker}{}%
\end{pgfscope}%
\end{pgfscope}%
\begin{pgfscope}%
\pgfpathrectangle{\pgfqpoint{0.688192in}{0.613042in}}{\pgfqpoint{11.096108in}{4.223431in}}%
\pgfusepath{clip}%
\pgfsetrectcap%
\pgfsetroundjoin%
\pgfsetlinewidth{0.803000pt}%
\definecolor{currentstroke}{rgb}{0.690196,0.690196,0.690196}%
\pgfsetstrokecolor{currentstroke}%
\pgfsetstrokeopacity{0.050000}%
\pgfsetdash{}{0pt}%
\pgfpathmoveto{\pgfqpoint{9.454117in}{0.613042in}}%
\pgfpathlineto{\pgfqpoint{9.454117in}{4.836473in}}%
\pgfusepath{stroke}%
\end{pgfscope}%
\begin{pgfscope}%
\pgfsetbuttcap%
\pgfsetroundjoin%
\definecolor{currentfill}{rgb}{0.000000,0.000000,0.000000}%
\pgfsetfillcolor{currentfill}%
\pgfsetlinewidth{0.602250pt}%
\definecolor{currentstroke}{rgb}{0.000000,0.000000,0.000000}%
\pgfsetstrokecolor{currentstroke}%
\pgfsetdash{}{0pt}%
\pgfsys@defobject{currentmarker}{\pgfqpoint{0.000000in}{-0.027778in}}{\pgfqpoint{0.000000in}{0.000000in}}{%
\pgfpathmoveto{\pgfqpoint{0.000000in}{0.000000in}}%
\pgfpathlineto{\pgfqpoint{0.000000in}{-0.027778in}}%
\pgfusepath{stroke,fill}%
}%
\begin{pgfscope}%
\pgfsys@transformshift{9.454117in}{0.613042in}%
\pgfsys@useobject{currentmarker}{}%
\end{pgfscope}%
\end{pgfscope}%
\begin{pgfscope}%
\pgfpathrectangle{\pgfqpoint{0.688192in}{0.613042in}}{\pgfqpoint{11.096108in}{4.223431in}}%
\pgfusepath{clip}%
\pgfsetrectcap%
\pgfsetroundjoin%
\pgfsetlinewidth{0.803000pt}%
\definecolor{currentstroke}{rgb}{0.690196,0.690196,0.690196}%
\pgfsetstrokecolor{currentstroke}%
\pgfsetstrokeopacity{0.050000}%
\pgfsetdash{}{0pt}%
\pgfpathmoveto{\pgfqpoint{9.676039in}{0.613042in}}%
\pgfpathlineto{\pgfqpoint{9.676039in}{4.836473in}}%
\pgfusepath{stroke}%
\end{pgfscope}%
\begin{pgfscope}%
\pgfsetbuttcap%
\pgfsetroundjoin%
\definecolor{currentfill}{rgb}{0.000000,0.000000,0.000000}%
\pgfsetfillcolor{currentfill}%
\pgfsetlinewidth{0.602250pt}%
\definecolor{currentstroke}{rgb}{0.000000,0.000000,0.000000}%
\pgfsetstrokecolor{currentstroke}%
\pgfsetdash{}{0pt}%
\pgfsys@defobject{currentmarker}{\pgfqpoint{0.000000in}{-0.027778in}}{\pgfqpoint{0.000000in}{0.000000in}}{%
\pgfpathmoveto{\pgfqpoint{0.000000in}{0.000000in}}%
\pgfpathlineto{\pgfqpoint{0.000000in}{-0.027778in}}%
\pgfusepath{stroke,fill}%
}%
\begin{pgfscope}%
\pgfsys@transformshift{9.676039in}{0.613042in}%
\pgfsys@useobject{currentmarker}{}%
\end{pgfscope}%
\end{pgfscope}%
\begin{pgfscope}%
\pgfpathrectangle{\pgfqpoint{0.688192in}{0.613042in}}{\pgfqpoint{11.096108in}{4.223431in}}%
\pgfusepath{clip}%
\pgfsetrectcap%
\pgfsetroundjoin%
\pgfsetlinewidth{0.803000pt}%
\definecolor{currentstroke}{rgb}{0.690196,0.690196,0.690196}%
\pgfsetstrokecolor{currentstroke}%
\pgfsetstrokeopacity{0.050000}%
\pgfsetdash{}{0pt}%
\pgfpathmoveto{\pgfqpoint{9.897961in}{0.613042in}}%
\pgfpathlineto{\pgfqpoint{9.897961in}{4.836473in}}%
\pgfusepath{stroke}%
\end{pgfscope}%
\begin{pgfscope}%
\pgfsetbuttcap%
\pgfsetroundjoin%
\definecolor{currentfill}{rgb}{0.000000,0.000000,0.000000}%
\pgfsetfillcolor{currentfill}%
\pgfsetlinewidth{0.602250pt}%
\definecolor{currentstroke}{rgb}{0.000000,0.000000,0.000000}%
\pgfsetstrokecolor{currentstroke}%
\pgfsetdash{}{0pt}%
\pgfsys@defobject{currentmarker}{\pgfqpoint{0.000000in}{-0.027778in}}{\pgfqpoint{0.000000in}{0.000000in}}{%
\pgfpathmoveto{\pgfqpoint{0.000000in}{0.000000in}}%
\pgfpathlineto{\pgfqpoint{0.000000in}{-0.027778in}}%
\pgfusepath{stroke,fill}%
}%
\begin{pgfscope}%
\pgfsys@transformshift{9.897961in}{0.613042in}%
\pgfsys@useobject{currentmarker}{}%
\end{pgfscope}%
\end{pgfscope}%
\begin{pgfscope}%
\pgfpathrectangle{\pgfqpoint{0.688192in}{0.613042in}}{\pgfqpoint{11.096108in}{4.223431in}}%
\pgfusepath{clip}%
\pgfsetrectcap%
\pgfsetroundjoin%
\pgfsetlinewidth{0.803000pt}%
\definecolor{currentstroke}{rgb}{0.690196,0.690196,0.690196}%
\pgfsetstrokecolor{currentstroke}%
\pgfsetstrokeopacity{0.050000}%
\pgfsetdash{}{0pt}%
\pgfpathmoveto{\pgfqpoint{10.341805in}{0.613042in}}%
\pgfpathlineto{\pgfqpoint{10.341805in}{4.836473in}}%
\pgfusepath{stroke}%
\end{pgfscope}%
\begin{pgfscope}%
\pgfsetbuttcap%
\pgfsetroundjoin%
\definecolor{currentfill}{rgb}{0.000000,0.000000,0.000000}%
\pgfsetfillcolor{currentfill}%
\pgfsetlinewidth{0.602250pt}%
\definecolor{currentstroke}{rgb}{0.000000,0.000000,0.000000}%
\pgfsetstrokecolor{currentstroke}%
\pgfsetdash{}{0pt}%
\pgfsys@defobject{currentmarker}{\pgfqpoint{0.000000in}{-0.027778in}}{\pgfqpoint{0.000000in}{0.000000in}}{%
\pgfpathmoveto{\pgfqpoint{0.000000in}{0.000000in}}%
\pgfpathlineto{\pgfqpoint{0.000000in}{-0.027778in}}%
\pgfusepath{stroke,fill}%
}%
\begin{pgfscope}%
\pgfsys@transformshift{10.341805in}{0.613042in}%
\pgfsys@useobject{currentmarker}{}%
\end{pgfscope}%
\end{pgfscope}%
\begin{pgfscope}%
\pgfpathrectangle{\pgfqpoint{0.688192in}{0.613042in}}{\pgfqpoint{11.096108in}{4.223431in}}%
\pgfusepath{clip}%
\pgfsetrectcap%
\pgfsetroundjoin%
\pgfsetlinewidth{0.803000pt}%
\definecolor{currentstroke}{rgb}{0.690196,0.690196,0.690196}%
\pgfsetstrokecolor{currentstroke}%
\pgfsetstrokeopacity{0.050000}%
\pgfsetdash{}{0pt}%
\pgfpathmoveto{\pgfqpoint{10.563728in}{0.613042in}}%
\pgfpathlineto{\pgfqpoint{10.563728in}{4.836473in}}%
\pgfusepath{stroke}%
\end{pgfscope}%
\begin{pgfscope}%
\pgfsetbuttcap%
\pgfsetroundjoin%
\definecolor{currentfill}{rgb}{0.000000,0.000000,0.000000}%
\pgfsetfillcolor{currentfill}%
\pgfsetlinewidth{0.602250pt}%
\definecolor{currentstroke}{rgb}{0.000000,0.000000,0.000000}%
\pgfsetstrokecolor{currentstroke}%
\pgfsetdash{}{0pt}%
\pgfsys@defobject{currentmarker}{\pgfqpoint{0.000000in}{-0.027778in}}{\pgfqpoint{0.000000in}{0.000000in}}{%
\pgfpathmoveto{\pgfqpoint{0.000000in}{0.000000in}}%
\pgfpathlineto{\pgfqpoint{0.000000in}{-0.027778in}}%
\pgfusepath{stroke,fill}%
}%
\begin{pgfscope}%
\pgfsys@transformshift{10.563728in}{0.613042in}%
\pgfsys@useobject{currentmarker}{}%
\end{pgfscope}%
\end{pgfscope}%
\begin{pgfscope}%
\pgfpathrectangle{\pgfqpoint{0.688192in}{0.613042in}}{\pgfqpoint{11.096108in}{4.223431in}}%
\pgfusepath{clip}%
\pgfsetrectcap%
\pgfsetroundjoin%
\pgfsetlinewidth{0.803000pt}%
\definecolor{currentstroke}{rgb}{0.690196,0.690196,0.690196}%
\pgfsetstrokecolor{currentstroke}%
\pgfsetstrokeopacity{0.050000}%
\pgfsetdash{}{0pt}%
\pgfpathmoveto{\pgfqpoint{10.785650in}{0.613042in}}%
\pgfpathlineto{\pgfqpoint{10.785650in}{4.836473in}}%
\pgfusepath{stroke}%
\end{pgfscope}%
\begin{pgfscope}%
\pgfsetbuttcap%
\pgfsetroundjoin%
\definecolor{currentfill}{rgb}{0.000000,0.000000,0.000000}%
\pgfsetfillcolor{currentfill}%
\pgfsetlinewidth{0.602250pt}%
\definecolor{currentstroke}{rgb}{0.000000,0.000000,0.000000}%
\pgfsetstrokecolor{currentstroke}%
\pgfsetdash{}{0pt}%
\pgfsys@defobject{currentmarker}{\pgfqpoint{0.000000in}{-0.027778in}}{\pgfqpoint{0.000000in}{0.000000in}}{%
\pgfpathmoveto{\pgfqpoint{0.000000in}{0.000000in}}%
\pgfpathlineto{\pgfqpoint{0.000000in}{-0.027778in}}%
\pgfusepath{stroke,fill}%
}%
\begin{pgfscope}%
\pgfsys@transformshift{10.785650in}{0.613042in}%
\pgfsys@useobject{currentmarker}{}%
\end{pgfscope}%
\end{pgfscope}%
\begin{pgfscope}%
\pgfpathrectangle{\pgfqpoint{0.688192in}{0.613042in}}{\pgfqpoint{11.096108in}{4.223431in}}%
\pgfusepath{clip}%
\pgfsetrectcap%
\pgfsetroundjoin%
\pgfsetlinewidth{0.803000pt}%
\definecolor{currentstroke}{rgb}{0.690196,0.690196,0.690196}%
\pgfsetstrokecolor{currentstroke}%
\pgfsetstrokeopacity{0.050000}%
\pgfsetdash{}{0pt}%
\pgfpathmoveto{\pgfqpoint{11.007572in}{0.613042in}}%
\pgfpathlineto{\pgfqpoint{11.007572in}{4.836473in}}%
\pgfusepath{stroke}%
\end{pgfscope}%
\begin{pgfscope}%
\pgfsetbuttcap%
\pgfsetroundjoin%
\definecolor{currentfill}{rgb}{0.000000,0.000000,0.000000}%
\pgfsetfillcolor{currentfill}%
\pgfsetlinewidth{0.602250pt}%
\definecolor{currentstroke}{rgb}{0.000000,0.000000,0.000000}%
\pgfsetstrokecolor{currentstroke}%
\pgfsetdash{}{0pt}%
\pgfsys@defobject{currentmarker}{\pgfqpoint{0.000000in}{-0.027778in}}{\pgfqpoint{0.000000in}{0.000000in}}{%
\pgfpathmoveto{\pgfqpoint{0.000000in}{0.000000in}}%
\pgfpathlineto{\pgfqpoint{0.000000in}{-0.027778in}}%
\pgfusepath{stroke,fill}%
}%
\begin{pgfscope}%
\pgfsys@transformshift{11.007572in}{0.613042in}%
\pgfsys@useobject{currentmarker}{}%
\end{pgfscope}%
\end{pgfscope}%
\begin{pgfscope}%
\pgfpathrectangle{\pgfqpoint{0.688192in}{0.613042in}}{\pgfqpoint{11.096108in}{4.223431in}}%
\pgfusepath{clip}%
\pgfsetrectcap%
\pgfsetroundjoin%
\pgfsetlinewidth{0.803000pt}%
\definecolor{currentstroke}{rgb}{0.690196,0.690196,0.690196}%
\pgfsetstrokecolor{currentstroke}%
\pgfsetstrokeopacity{0.050000}%
\pgfsetdash{}{0pt}%
\pgfpathmoveto{\pgfqpoint{11.451416in}{0.613042in}}%
\pgfpathlineto{\pgfqpoint{11.451416in}{4.836473in}}%
\pgfusepath{stroke}%
\end{pgfscope}%
\begin{pgfscope}%
\pgfsetbuttcap%
\pgfsetroundjoin%
\definecolor{currentfill}{rgb}{0.000000,0.000000,0.000000}%
\pgfsetfillcolor{currentfill}%
\pgfsetlinewidth{0.602250pt}%
\definecolor{currentstroke}{rgb}{0.000000,0.000000,0.000000}%
\pgfsetstrokecolor{currentstroke}%
\pgfsetdash{}{0pt}%
\pgfsys@defobject{currentmarker}{\pgfqpoint{0.000000in}{-0.027778in}}{\pgfqpoint{0.000000in}{0.000000in}}{%
\pgfpathmoveto{\pgfqpoint{0.000000in}{0.000000in}}%
\pgfpathlineto{\pgfqpoint{0.000000in}{-0.027778in}}%
\pgfusepath{stroke,fill}%
}%
\begin{pgfscope}%
\pgfsys@transformshift{11.451416in}{0.613042in}%
\pgfsys@useobject{currentmarker}{}%
\end{pgfscope}%
\end{pgfscope}%
\begin{pgfscope}%
\pgfpathrectangle{\pgfqpoint{0.688192in}{0.613042in}}{\pgfqpoint{11.096108in}{4.223431in}}%
\pgfusepath{clip}%
\pgfsetrectcap%
\pgfsetroundjoin%
\pgfsetlinewidth{0.803000pt}%
\definecolor{currentstroke}{rgb}{0.690196,0.690196,0.690196}%
\pgfsetstrokecolor{currentstroke}%
\pgfsetstrokeopacity{0.050000}%
\pgfsetdash{}{0pt}%
\pgfpathmoveto{\pgfqpoint{11.673338in}{0.613042in}}%
\pgfpathlineto{\pgfqpoint{11.673338in}{4.836473in}}%
\pgfusepath{stroke}%
\end{pgfscope}%
\begin{pgfscope}%
\pgfsetbuttcap%
\pgfsetroundjoin%
\definecolor{currentfill}{rgb}{0.000000,0.000000,0.000000}%
\pgfsetfillcolor{currentfill}%
\pgfsetlinewidth{0.602250pt}%
\definecolor{currentstroke}{rgb}{0.000000,0.000000,0.000000}%
\pgfsetstrokecolor{currentstroke}%
\pgfsetdash{}{0pt}%
\pgfsys@defobject{currentmarker}{\pgfqpoint{0.000000in}{-0.027778in}}{\pgfqpoint{0.000000in}{0.000000in}}{%
\pgfpathmoveto{\pgfqpoint{0.000000in}{0.000000in}}%
\pgfpathlineto{\pgfqpoint{0.000000in}{-0.027778in}}%
\pgfusepath{stroke,fill}%
}%
\begin{pgfscope}%
\pgfsys@transformshift{11.673338in}{0.613042in}%
\pgfsys@useobject{currentmarker}{}%
\end{pgfscope}%
\end{pgfscope}%
\begin{pgfscope}%
\pgfpathrectangle{\pgfqpoint{0.688192in}{0.613042in}}{\pgfqpoint{11.096108in}{4.223431in}}%
\pgfusepath{clip}%
\pgfsetrectcap%
\pgfsetroundjoin%
\pgfsetlinewidth{0.803000pt}%
\definecolor{currentstroke}{rgb}{0.690196,0.690196,0.690196}%
\pgfsetstrokecolor{currentstroke}%
\pgfsetstrokeopacity{0.200000}%
\pgfsetdash{}{0pt}%
\pgfpathmoveto{\pgfqpoint{0.688192in}{0.613042in}}%
\pgfpathlineto{\pgfqpoint{11.784299in}{0.613042in}}%
\pgfusepath{stroke}%
\end{pgfscope}%
\begin{pgfscope}%
\pgfsetbuttcap%
\pgfsetroundjoin%
\definecolor{currentfill}{rgb}{0.000000,0.000000,0.000000}%
\pgfsetfillcolor{currentfill}%
\pgfsetlinewidth{0.803000pt}%
\definecolor{currentstroke}{rgb}{0.000000,0.000000,0.000000}%
\pgfsetstrokecolor{currentstroke}%
\pgfsetdash{}{0pt}%
\pgfsys@defobject{currentmarker}{\pgfqpoint{-0.048611in}{0.000000in}}{\pgfqpoint{-0.000000in}{0.000000in}}{%
\pgfpathmoveto{\pgfqpoint{-0.000000in}{0.000000in}}%
\pgfpathlineto{\pgfqpoint{-0.048611in}{0.000000in}}%
\pgfusepath{stroke,fill}%
}%
\begin{pgfscope}%
\pgfsys@transformshift{0.688192in}{0.613042in}%
\pgfsys@useobject{currentmarker}{}%
\end{pgfscope}%
\end{pgfscope}%
\begin{pgfscope}%
\definecolor{textcolor}{rgb}{0.000000,0.000000,0.000000}%
\pgfsetstrokecolor{textcolor}%
\pgfsetfillcolor{textcolor}%
\pgftext[x=0.493054in, y=0.543598in, left, base]{\color{textcolor}{\rmfamily\fontsize{14.000000}{16.800000}\selectfont\catcode`\^=\active\def^{\ifmmode\sp\else\^{}\fi}\catcode`\%=\active\def%{\%}$\mathdefault{0}$}}%
\end{pgfscope}%
\begin{pgfscope}%
\pgfpathrectangle{\pgfqpoint{0.688192in}{0.613042in}}{\pgfqpoint{11.096108in}{4.223431in}}%
\pgfusepath{clip}%
\pgfsetrectcap%
\pgfsetroundjoin%
\pgfsetlinewidth{0.803000pt}%
\definecolor{currentstroke}{rgb}{0.690196,0.690196,0.690196}%
\pgfsetstrokecolor{currentstroke}%
\pgfsetstrokeopacity{0.200000}%
\pgfsetdash{}{0pt}%
\pgfpathmoveto{\pgfqpoint{0.688192in}{1.395159in}}%
\pgfpathlineto{\pgfqpoint{11.784299in}{1.395159in}}%
\pgfusepath{stroke}%
\end{pgfscope}%
\begin{pgfscope}%
\pgfsetbuttcap%
\pgfsetroundjoin%
\definecolor{currentfill}{rgb}{0.000000,0.000000,0.000000}%
\pgfsetfillcolor{currentfill}%
\pgfsetlinewidth{0.803000pt}%
\definecolor{currentstroke}{rgb}{0.000000,0.000000,0.000000}%
\pgfsetstrokecolor{currentstroke}%
\pgfsetdash{}{0pt}%
\pgfsys@defobject{currentmarker}{\pgfqpoint{-0.048611in}{0.000000in}}{\pgfqpoint{-0.000000in}{0.000000in}}{%
\pgfpathmoveto{\pgfqpoint{-0.000000in}{0.000000in}}%
\pgfpathlineto{\pgfqpoint{-0.048611in}{0.000000in}}%
\pgfusepath{stroke,fill}%
}%
\begin{pgfscope}%
\pgfsys@transformshift{0.688192in}{1.395159in}%
\pgfsys@useobject{currentmarker}{}%
\end{pgfscope}%
\end{pgfscope}%
\begin{pgfscope}%
\definecolor{textcolor}{rgb}{0.000000,0.000000,0.000000}%
\pgfsetstrokecolor{textcolor}%
\pgfsetfillcolor{textcolor}%
\pgftext[x=0.493054in, y=1.325714in, left, base]{\color{textcolor}{\rmfamily\fontsize{14.000000}{16.800000}\selectfont\catcode`\^=\active\def^{\ifmmode\sp\else\^{}\fi}\catcode`\%=\active\def%{\%}$\mathdefault{5}$}}%
\end{pgfscope}%
\begin{pgfscope}%
\pgfpathrectangle{\pgfqpoint{0.688192in}{0.613042in}}{\pgfqpoint{11.096108in}{4.223431in}}%
\pgfusepath{clip}%
\pgfsetrectcap%
\pgfsetroundjoin%
\pgfsetlinewidth{0.803000pt}%
\definecolor{currentstroke}{rgb}{0.690196,0.690196,0.690196}%
\pgfsetstrokecolor{currentstroke}%
\pgfsetstrokeopacity{0.200000}%
\pgfsetdash{}{0pt}%
\pgfpathmoveto{\pgfqpoint{0.688192in}{2.177275in}}%
\pgfpathlineto{\pgfqpoint{11.784299in}{2.177275in}}%
\pgfusepath{stroke}%
\end{pgfscope}%
\begin{pgfscope}%
\pgfsetbuttcap%
\pgfsetroundjoin%
\definecolor{currentfill}{rgb}{0.000000,0.000000,0.000000}%
\pgfsetfillcolor{currentfill}%
\pgfsetlinewidth{0.803000pt}%
\definecolor{currentstroke}{rgb}{0.000000,0.000000,0.000000}%
\pgfsetstrokecolor{currentstroke}%
\pgfsetdash{}{0pt}%
\pgfsys@defobject{currentmarker}{\pgfqpoint{-0.048611in}{0.000000in}}{\pgfqpoint{-0.000000in}{0.000000in}}{%
\pgfpathmoveto{\pgfqpoint{-0.000000in}{0.000000in}}%
\pgfpathlineto{\pgfqpoint{-0.048611in}{0.000000in}}%
\pgfusepath{stroke,fill}%
}%
\begin{pgfscope}%
\pgfsys@transformshift{0.688192in}{2.177275in}%
\pgfsys@useobject{currentmarker}{}%
\end{pgfscope}%
\end{pgfscope}%
\begin{pgfscope}%
\definecolor{textcolor}{rgb}{0.000000,0.000000,0.000000}%
\pgfsetstrokecolor{textcolor}%
\pgfsetfillcolor{textcolor}%
\pgftext[x=0.395138in, y=2.107831in, left, base]{\color{textcolor}{\rmfamily\fontsize{14.000000}{16.800000}\selectfont\catcode`\^=\active\def^{\ifmmode\sp\else\^{}\fi}\catcode`\%=\active\def%{\%}$\mathdefault{10}$}}%
\end{pgfscope}%
\begin{pgfscope}%
\pgfpathrectangle{\pgfqpoint{0.688192in}{0.613042in}}{\pgfqpoint{11.096108in}{4.223431in}}%
\pgfusepath{clip}%
\pgfsetrectcap%
\pgfsetroundjoin%
\pgfsetlinewidth{0.803000pt}%
\definecolor{currentstroke}{rgb}{0.690196,0.690196,0.690196}%
\pgfsetstrokecolor{currentstroke}%
\pgfsetstrokeopacity{0.200000}%
\pgfsetdash{}{0pt}%
\pgfpathmoveto{\pgfqpoint{0.688192in}{2.959392in}}%
\pgfpathlineto{\pgfqpoint{11.784299in}{2.959392in}}%
\pgfusepath{stroke}%
\end{pgfscope}%
\begin{pgfscope}%
\pgfsetbuttcap%
\pgfsetroundjoin%
\definecolor{currentfill}{rgb}{0.000000,0.000000,0.000000}%
\pgfsetfillcolor{currentfill}%
\pgfsetlinewidth{0.803000pt}%
\definecolor{currentstroke}{rgb}{0.000000,0.000000,0.000000}%
\pgfsetstrokecolor{currentstroke}%
\pgfsetdash{}{0pt}%
\pgfsys@defobject{currentmarker}{\pgfqpoint{-0.048611in}{0.000000in}}{\pgfqpoint{-0.000000in}{0.000000in}}{%
\pgfpathmoveto{\pgfqpoint{-0.000000in}{0.000000in}}%
\pgfpathlineto{\pgfqpoint{-0.048611in}{0.000000in}}%
\pgfusepath{stroke,fill}%
}%
\begin{pgfscope}%
\pgfsys@transformshift{0.688192in}{2.959392in}%
\pgfsys@useobject{currentmarker}{}%
\end{pgfscope}%
\end{pgfscope}%
\begin{pgfscope}%
\definecolor{textcolor}{rgb}{0.000000,0.000000,0.000000}%
\pgfsetstrokecolor{textcolor}%
\pgfsetfillcolor{textcolor}%
\pgftext[x=0.395138in, y=2.889948in, left, base]{\color{textcolor}{\rmfamily\fontsize{14.000000}{16.800000}\selectfont\catcode`\^=\active\def^{\ifmmode\sp\else\^{}\fi}\catcode`\%=\active\def%{\%}$\mathdefault{15}$}}%
\end{pgfscope}%
\begin{pgfscope}%
\pgfpathrectangle{\pgfqpoint{0.688192in}{0.613042in}}{\pgfqpoint{11.096108in}{4.223431in}}%
\pgfusepath{clip}%
\pgfsetrectcap%
\pgfsetroundjoin%
\pgfsetlinewidth{0.803000pt}%
\definecolor{currentstroke}{rgb}{0.690196,0.690196,0.690196}%
\pgfsetstrokecolor{currentstroke}%
\pgfsetstrokeopacity{0.200000}%
\pgfsetdash{}{0pt}%
\pgfpathmoveto{\pgfqpoint{0.688192in}{3.741509in}}%
\pgfpathlineto{\pgfqpoint{11.784299in}{3.741509in}}%
\pgfusepath{stroke}%
\end{pgfscope}%
\begin{pgfscope}%
\pgfsetbuttcap%
\pgfsetroundjoin%
\definecolor{currentfill}{rgb}{0.000000,0.000000,0.000000}%
\pgfsetfillcolor{currentfill}%
\pgfsetlinewidth{0.803000pt}%
\definecolor{currentstroke}{rgb}{0.000000,0.000000,0.000000}%
\pgfsetstrokecolor{currentstroke}%
\pgfsetdash{}{0pt}%
\pgfsys@defobject{currentmarker}{\pgfqpoint{-0.048611in}{0.000000in}}{\pgfqpoint{-0.000000in}{0.000000in}}{%
\pgfpathmoveto{\pgfqpoint{-0.000000in}{0.000000in}}%
\pgfpathlineto{\pgfqpoint{-0.048611in}{0.000000in}}%
\pgfusepath{stroke,fill}%
}%
\begin{pgfscope}%
\pgfsys@transformshift{0.688192in}{3.741509in}%
\pgfsys@useobject{currentmarker}{}%
\end{pgfscope}%
\end{pgfscope}%
\begin{pgfscope}%
\definecolor{textcolor}{rgb}{0.000000,0.000000,0.000000}%
\pgfsetstrokecolor{textcolor}%
\pgfsetfillcolor{textcolor}%
\pgftext[x=0.395138in, y=3.672065in, left, base]{\color{textcolor}{\rmfamily\fontsize{14.000000}{16.800000}\selectfont\catcode`\^=\active\def^{\ifmmode\sp\else\^{}\fi}\catcode`\%=\active\def%{\%}$\mathdefault{20}$}}%
\end{pgfscope}%
\begin{pgfscope}%
\pgfpathrectangle{\pgfqpoint{0.688192in}{0.613042in}}{\pgfqpoint{11.096108in}{4.223431in}}%
\pgfusepath{clip}%
\pgfsetrectcap%
\pgfsetroundjoin%
\pgfsetlinewidth{0.803000pt}%
\definecolor{currentstroke}{rgb}{0.690196,0.690196,0.690196}%
\pgfsetstrokecolor{currentstroke}%
\pgfsetstrokeopacity{0.200000}%
\pgfsetdash{}{0pt}%
\pgfpathmoveto{\pgfqpoint{0.688192in}{4.523626in}}%
\pgfpathlineto{\pgfqpoint{11.784299in}{4.523626in}}%
\pgfusepath{stroke}%
\end{pgfscope}%
\begin{pgfscope}%
\pgfsetbuttcap%
\pgfsetroundjoin%
\definecolor{currentfill}{rgb}{0.000000,0.000000,0.000000}%
\pgfsetfillcolor{currentfill}%
\pgfsetlinewidth{0.803000pt}%
\definecolor{currentstroke}{rgb}{0.000000,0.000000,0.000000}%
\pgfsetstrokecolor{currentstroke}%
\pgfsetdash{}{0pt}%
\pgfsys@defobject{currentmarker}{\pgfqpoint{-0.048611in}{0.000000in}}{\pgfqpoint{-0.000000in}{0.000000in}}{%
\pgfpathmoveto{\pgfqpoint{-0.000000in}{0.000000in}}%
\pgfpathlineto{\pgfqpoint{-0.048611in}{0.000000in}}%
\pgfusepath{stroke,fill}%
}%
\begin{pgfscope}%
\pgfsys@transformshift{0.688192in}{4.523626in}%
\pgfsys@useobject{currentmarker}{}%
\end{pgfscope}%
\end{pgfscope}%
\begin{pgfscope}%
\definecolor{textcolor}{rgb}{0.000000,0.000000,0.000000}%
\pgfsetstrokecolor{textcolor}%
\pgfsetfillcolor{textcolor}%
\pgftext[x=0.395138in, y=4.454182in, left, base]{\color{textcolor}{\rmfamily\fontsize{14.000000}{16.800000}\selectfont\catcode`\^=\active\def^{\ifmmode\sp\else\^{}\fi}\catcode`\%=\active\def%{\%}$\mathdefault{25}$}}%
\end{pgfscope}%
\begin{pgfscope}%
\pgfpathrectangle{\pgfqpoint{0.688192in}{0.613042in}}{\pgfqpoint{11.096108in}{4.223431in}}%
\pgfusepath{clip}%
\pgfsetrectcap%
\pgfsetroundjoin%
\pgfsetlinewidth{0.803000pt}%
\definecolor{currentstroke}{rgb}{0.690196,0.690196,0.690196}%
\pgfsetstrokecolor{currentstroke}%
\pgfsetstrokeopacity{0.050000}%
\pgfsetdash{}{0pt}%
\pgfpathmoveto{\pgfqpoint{0.688192in}{0.769465in}}%
\pgfpathlineto{\pgfqpoint{11.784299in}{0.769465in}}%
\pgfusepath{stroke}%
\end{pgfscope}%
\begin{pgfscope}%
\pgfsetbuttcap%
\pgfsetroundjoin%
\definecolor{currentfill}{rgb}{0.000000,0.000000,0.000000}%
\pgfsetfillcolor{currentfill}%
\pgfsetlinewidth{0.602250pt}%
\definecolor{currentstroke}{rgb}{0.000000,0.000000,0.000000}%
\pgfsetstrokecolor{currentstroke}%
\pgfsetdash{}{0pt}%
\pgfsys@defobject{currentmarker}{\pgfqpoint{-0.027778in}{0.000000in}}{\pgfqpoint{-0.000000in}{0.000000in}}{%
\pgfpathmoveto{\pgfqpoint{-0.000000in}{0.000000in}}%
\pgfpathlineto{\pgfqpoint{-0.027778in}{0.000000in}}%
\pgfusepath{stroke,fill}%
}%
\begin{pgfscope}%
\pgfsys@transformshift{0.688192in}{0.769465in}%
\pgfsys@useobject{currentmarker}{}%
\end{pgfscope}%
\end{pgfscope}%
\begin{pgfscope}%
\pgfpathrectangle{\pgfqpoint{0.688192in}{0.613042in}}{\pgfqpoint{11.096108in}{4.223431in}}%
\pgfusepath{clip}%
\pgfsetrectcap%
\pgfsetroundjoin%
\pgfsetlinewidth{0.803000pt}%
\definecolor{currentstroke}{rgb}{0.690196,0.690196,0.690196}%
\pgfsetstrokecolor{currentstroke}%
\pgfsetstrokeopacity{0.050000}%
\pgfsetdash{}{0pt}%
\pgfpathmoveto{\pgfqpoint{0.688192in}{0.925889in}}%
\pgfpathlineto{\pgfqpoint{11.784299in}{0.925889in}}%
\pgfusepath{stroke}%
\end{pgfscope}%
\begin{pgfscope}%
\pgfsetbuttcap%
\pgfsetroundjoin%
\definecolor{currentfill}{rgb}{0.000000,0.000000,0.000000}%
\pgfsetfillcolor{currentfill}%
\pgfsetlinewidth{0.602250pt}%
\definecolor{currentstroke}{rgb}{0.000000,0.000000,0.000000}%
\pgfsetstrokecolor{currentstroke}%
\pgfsetdash{}{0pt}%
\pgfsys@defobject{currentmarker}{\pgfqpoint{-0.027778in}{0.000000in}}{\pgfqpoint{-0.000000in}{0.000000in}}{%
\pgfpathmoveto{\pgfqpoint{-0.000000in}{0.000000in}}%
\pgfpathlineto{\pgfqpoint{-0.027778in}{0.000000in}}%
\pgfusepath{stroke,fill}%
}%
\begin{pgfscope}%
\pgfsys@transformshift{0.688192in}{0.925889in}%
\pgfsys@useobject{currentmarker}{}%
\end{pgfscope}%
\end{pgfscope}%
\begin{pgfscope}%
\pgfpathrectangle{\pgfqpoint{0.688192in}{0.613042in}}{\pgfqpoint{11.096108in}{4.223431in}}%
\pgfusepath{clip}%
\pgfsetrectcap%
\pgfsetroundjoin%
\pgfsetlinewidth{0.803000pt}%
\definecolor{currentstroke}{rgb}{0.690196,0.690196,0.690196}%
\pgfsetstrokecolor{currentstroke}%
\pgfsetstrokeopacity{0.050000}%
\pgfsetdash{}{0pt}%
\pgfpathmoveto{\pgfqpoint{0.688192in}{1.082312in}}%
\pgfpathlineto{\pgfqpoint{11.784299in}{1.082312in}}%
\pgfusepath{stroke}%
\end{pgfscope}%
\begin{pgfscope}%
\pgfsetbuttcap%
\pgfsetroundjoin%
\definecolor{currentfill}{rgb}{0.000000,0.000000,0.000000}%
\pgfsetfillcolor{currentfill}%
\pgfsetlinewidth{0.602250pt}%
\definecolor{currentstroke}{rgb}{0.000000,0.000000,0.000000}%
\pgfsetstrokecolor{currentstroke}%
\pgfsetdash{}{0pt}%
\pgfsys@defobject{currentmarker}{\pgfqpoint{-0.027778in}{0.000000in}}{\pgfqpoint{-0.000000in}{0.000000in}}{%
\pgfpathmoveto{\pgfqpoint{-0.000000in}{0.000000in}}%
\pgfpathlineto{\pgfqpoint{-0.027778in}{0.000000in}}%
\pgfusepath{stroke,fill}%
}%
\begin{pgfscope}%
\pgfsys@transformshift{0.688192in}{1.082312in}%
\pgfsys@useobject{currentmarker}{}%
\end{pgfscope}%
\end{pgfscope}%
\begin{pgfscope}%
\pgfpathrectangle{\pgfqpoint{0.688192in}{0.613042in}}{\pgfqpoint{11.096108in}{4.223431in}}%
\pgfusepath{clip}%
\pgfsetrectcap%
\pgfsetroundjoin%
\pgfsetlinewidth{0.803000pt}%
\definecolor{currentstroke}{rgb}{0.690196,0.690196,0.690196}%
\pgfsetstrokecolor{currentstroke}%
\pgfsetstrokeopacity{0.050000}%
\pgfsetdash{}{0pt}%
\pgfpathmoveto{\pgfqpoint{0.688192in}{1.238735in}}%
\pgfpathlineto{\pgfqpoint{11.784299in}{1.238735in}}%
\pgfusepath{stroke}%
\end{pgfscope}%
\begin{pgfscope}%
\pgfsetbuttcap%
\pgfsetroundjoin%
\definecolor{currentfill}{rgb}{0.000000,0.000000,0.000000}%
\pgfsetfillcolor{currentfill}%
\pgfsetlinewidth{0.602250pt}%
\definecolor{currentstroke}{rgb}{0.000000,0.000000,0.000000}%
\pgfsetstrokecolor{currentstroke}%
\pgfsetdash{}{0pt}%
\pgfsys@defobject{currentmarker}{\pgfqpoint{-0.027778in}{0.000000in}}{\pgfqpoint{-0.000000in}{0.000000in}}{%
\pgfpathmoveto{\pgfqpoint{-0.000000in}{0.000000in}}%
\pgfpathlineto{\pgfqpoint{-0.027778in}{0.000000in}}%
\pgfusepath{stroke,fill}%
}%
\begin{pgfscope}%
\pgfsys@transformshift{0.688192in}{1.238735in}%
\pgfsys@useobject{currentmarker}{}%
\end{pgfscope}%
\end{pgfscope}%
\begin{pgfscope}%
\pgfpathrectangle{\pgfqpoint{0.688192in}{0.613042in}}{\pgfqpoint{11.096108in}{4.223431in}}%
\pgfusepath{clip}%
\pgfsetrectcap%
\pgfsetroundjoin%
\pgfsetlinewidth{0.803000pt}%
\definecolor{currentstroke}{rgb}{0.690196,0.690196,0.690196}%
\pgfsetstrokecolor{currentstroke}%
\pgfsetstrokeopacity{0.050000}%
\pgfsetdash{}{0pt}%
\pgfpathmoveto{\pgfqpoint{0.688192in}{1.551582in}}%
\pgfpathlineto{\pgfqpoint{11.784299in}{1.551582in}}%
\pgfusepath{stroke}%
\end{pgfscope}%
\begin{pgfscope}%
\pgfsetbuttcap%
\pgfsetroundjoin%
\definecolor{currentfill}{rgb}{0.000000,0.000000,0.000000}%
\pgfsetfillcolor{currentfill}%
\pgfsetlinewidth{0.602250pt}%
\definecolor{currentstroke}{rgb}{0.000000,0.000000,0.000000}%
\pgfsetstrokecolor{currentstroke}%
\pgfsetdash{}{0pt}%
\pgfsys@defobject{currentmarker}{\pgfqpoint{-0.027778in}{0.000000in}}{\pgfqpoint{-0.000000in}{0.000000in}}{%
\pgfpathmoveto{\pgfqpoint{-0.000000in}{0.000000in}}%
\pgfpathlineto{\pgfqpoint{-0.027778in}{0.000000in}}%
\pgfusepath{stroke,fill}%
}%
\begin{pgfscope}%
\pgfsys@transformshift{0.688192in}{1.551582in}%
\pgfsys@useobject{currentmarker}{}%
\end{pgfscope}%
\end{pgfscope}%
\begin{pgfscope}%
\pgfpathrectangle{\pgfqpoint{0.688192in}{0.613042in}}{\pgfqpoint{11.096108in}{4.223431in}}%
\pgfusepath{clip}%
\pgfsetrectcap%
\pgfsetroundjoin%
\pgfsetlinewidth{0.803000pt}%
\definecolor{currentstroke}{rgb}{0.690196,0.690196,0.690196}%
\pgfsetstrokecolor{currentstroke}%
\pgfsetstrokeopacity{0.050000}%
\pgfsetdash{}{0pt}%
\pgfpathmoveto{\pgfqpoint{0.688192in}{1.708005in}}%
\pgfpathlineto{\pgfqpoint{11.784299in}{1.708005in}}%
\pgfusepath{stroke}%
\end{pgfscope}%
\begin{pgfscope}%
\pgfsetbuttcap%
\pgfsetroundjoin%
\definecolor{currentfill}{rgb}{0.000000,0.000000,0.000000}%
\pgfsetfillcolor{currentfill}%
\pgfsetlinewidth{0.602250pt}%
\definecolor{currentstroke}{rgb}{0.000000,0.000000,0.000000}%
\pgfsetstrokecolor{currentstroke}%
\pgfsetdash{}{0pt}%
\pgfsys@defobject{currentmarker}{\pgfqpoint{-0.027778in}{0.000000in}}{\pgfqpoint{-0.000000in}{0.000000in}}{%
\pgfpathmoveto{\pgfqpoint{-0.000000in}{0.000000in}}%
\pgfpathlineto{\pgfqpoint{-0.027778in}{0.000000in}}%
\pgfusepath{stroke,fill}%
}%
\begin{pgfscope}%
\pgfsys@transformshift{0.688192in}{1.708005in}%
\pgfsys@useobject{currentmarker}{}%
\end{pgfscope}%
\end{pgfscope}%
\begin{pgfscope}%
\pgfpathrectangle{\pgfqpoint{0.688192in}{0.613042in}}{\pgfqpoint{11.096108in}{4.223431in}}%
\pgfusepath{clip}%
\pgfsetrectcap%
\pgfsetroundjoin%
\pgfsetlinewidth{0.803000pt}%
\definecolor{currentstroke}{rgb}{0.690196,0.690196,0.690196}%
\pgfsetstrokecolor{currentstroke}%
\pgfsetstrokeopacity{0.050000}%
\pgfsetdash{}{0pt}%
\pgfpathmoveto{\pgfqpoint{0.688192in}{1.864429in}}%
\pgfpathlineto{\pgfqpoint{11.784299in}{1.864429in}}%
\pgfusepath{stroke}%
\end{pgfscope}%
\begin{pgfscope}%
\pgfsetbuttcap%
\pgfsetroundjoin%
\definecolor{currentfill}{rgb}{0.000000,0.000000,0.000000}%
\pgfsetfillcolor{currentfill}%
\pgfsetlinewidth{0.602250pt}%
\definecolor{currentstroke}{rgb}{0.000000,0.000000,0.000000}%
\pgfsetstrokecolor{currentstroke}%
\pgfsetdash{}{0pt}%
\pgfsys@defobject{currentmarker}{\pgfqpoint{-0.027778in}{0.000000in}}{\pgfqpoint{-0.000000in}{0.000000in}}{%
\pgfpathmoveto{\pgfqpoint{-0.000000in}{0.000000in}}%
\pgfpathlineto{\pgfqpoint{-0.027778in}{0.000000in}}%
\pgfusepath{stroke,fill}%
}%
\begin{pgfscope}%
\pgfsys@transformshift{0.688192in}{1.864429in}%
\pgfsys@useobject{currentmarker}{}%
\end{pgfscope}%
\end{pgfscope}%
\begin{pgfscope}%
\pgfpathrectangle{\pgfqpoint{0.688192in}{0.613042in}}{\pgfqpoint{11.096108in}{4.223431in}}%
\pgfusepath{clip}%
\pgfsetrectcap%
\pgfsetroundjoin%
\pgfsetlinewidth{0.803000pt}%
\definecolor{currentstroke}{rgb}{0.690196,0.690196,0.690196}%
\pgfsetstrokecolor{currentstroke}%
\pgfsetstrokeopacity{0.050000}%
\pgfsetdash{}{0pt}%
\pgfpathmoveto{\pgfqpoint{0.688192in}{2.020852in}}%
\pgfpathlineto{\pgfqpoint{11.784299in}{2.020852in}}%
\pgfusepath{stroke}%
\end{pgfscope}%
\begin{pgfscope}%
\pgfsetbuttcap%
\pgfsetroundjoin%
\definecolor{currentfill}{rgb}{0.000000,0.000000,0.000000}%
\pgfsetfillcolor{currentfill}%
\pgfsetlinewidth{0.602250pt}%
\definecolor{currentstroke}{rgb}{0.000000,0.000000,0.000000}%
\pgfsetstrokecolor{currentstroke}%
\pgfsetdash{}{0pt}%
\pgfsys@defobject{currentmarker}{\pgfqpoint{-0.027778in}{0.000000in}}{\pgfqpoint{-0.000000in}{0.000000in}}{%
\pgfpathmoveto{\pgfqpoint{-0.000000in}{0.000000in}}%
\pgfpathlineto{\pgfqpoint{-0.027778in}{0.000000in}}%
\pgfusepath{stroke,fill}%
}%
\begin{pgfscope}%
\pgfsys@transformshift{0.688192in}{2.020852in}%
\pgfsys@useobject{currentmarker}{}%
\end{pgfscope}%
\end{pgfscope}%
\begin{pgfscope}%
\pgfpathrectangle{\pgfqpoint{0.688192in}{0.613042in}}{\pgfqpoint{11.096108in}{4.223431in}}%
\pgfusepath{clip}%
\pgfsetrectcap%
\pgfsetroundjoin%
\pgfsetlinewidth{0.803000pt}%
\definecolor{currentstroke}{rgb}{0.690196,0.690196,0.690196}%
\pgfsetstrokecolor{currentstroke}%
\pgfsetstrokeopacity{0.050000}%
\pgfsetdash{}{0pt}%
\pgfpathmoveto{\pgfqpoint{0.688192in}{2.333699in}}%
\pgfpathlineto{\pgfqpoint{11.784299in}{2.333699in}}%
\pgfusepath{stroke}%
\end{pgfscope}%
\begin{pgfscope}%
\pgfsetbuttcap%
\pgfsetroundjoin%
\definecolor{currentfill}{rgb}{0.000000,0.000000,0.000000}%
\pgfsetfillcolor{currentfill}%
\pgfsetlinewidth{0.602250pt}%
\definecolor{currentstroke}{rgb}{0.000000,0.000000,0.000000}%
\pgfsetstrokecolor{currentstroke}%
\pgfsetdash{}{0pt}%
\pgfsys@defobject{currentmarker}{\pgfqpoint{-0.027778in}{0.000000in}}{\pgfqpoint{-0.000000in}{0.000000in}}{%
\pgfpathmoveto{\pgfqpoint{-0.000000in}{0.000000in}}%
\pgfpathlineto{\pgfqpoint{-0.027778in}{0.000000in}}%
\pgfusepath{stroke,fill}%
}%
\begin{pgfscope}%
\pgfsys@transformshift{0.688192in}{2.333699in}%
\pgfsys@useobject{currentmarker}{}%
\end{pgfscope}%
\end{pgfscope}%
\begin{pgfscope}%
\pgfpathrectangle{\pgfqpoint{0.688192in}{0.613042in}}{\pgfqpoint{11.096108in}{4.223431in}}%
\pgfusepath{clip}%
\pgfsetrectcap%
\pgfsetroundjoin%
\pgfsetlinewidth{0.803000pt}%
\definecolor{currentstroke}{rgb}{0.690196,0.690196,0.690196}%
\pgfsetstrokecolor{currentstroke}%
\pgfsetstrokeopacity{0.050000}%
\pgfsetdash{}{0pt}%
\pgfpathmoveto{\pgfqpoint{0.688192in}{2.490122in}}%
\pgfpathlineto{\pgfqpoint{11.784299in}{2.490122in}}%
\pgfusepath{stroke}%
\end{pgfscope}%
\begin{pgfscope}%
\pgfsetbuttcap%
\pgfsetroundjoin%
\definecolor{currentfill}{rgb}{0.000000,0.000000,0.000000}%
\pgfsetfillcolor{currentfill}%
\pgfsetlinewidth{0.602250pt}%
\definecolor{currentstroke}{rgb}{0.000000,0.000000,0.000000}%
\pgfsetstrokecolor{currentstroke}%
\pgfsetdash{}{0pt}%
\pgfsys@defobject{currentmarker}{\pgfqpoint{-0.027778in}{0.000000in}}{\pgfqpoint{-0.000000in}{0.000000in}}{%
\pgfpathmoveto{\pgfqpoint{-0.000000in}{0.000000in}}%
\pgfpathlineto{\pgfqpoint{-0.027778in}{0.000000in}}%
\pgfusepath{stroke,fill}%
}%
\begin{pgfscope}%
\pgfsys@transformshift{0.688192in}{2.490122in}%
\pgfsys@useobject{currentmarker}{}%
\end{pgfscope}%
\end{pgfscope}%
\begin{pgfscope}%
\pgfpathrectangle{\pgfqpoint{0.688192in}{0.613042in}}{\pgfqpoint{11.096108in}{4.223431in}}%
\pgfusepath{clip}%
\pgfsetrectcap%
\pgfsetroundjoin%
\pgfsetlinewidth{0.803000pt}%
\definecolor{currentstroke}{rgb}{0.690196,0.690196,0.690196}%
\pgfsetstrokecolor{currentstroke}%
\pgfsetstrokeopacity{0.050000}%
\pgfsetdash{}{0pt}%
\pgfpathmoveto{\pgfqpoint{0.688192in}{2.646546in}}%
\pgfpathlineto{\pgfqpoint{11.784299in}{2.646546in}}%
\pgfusepath{stroke}%
\end{pgfscope}%
\begin{pgfscope}%
\pgfsetbuttcap%
\pgfsetroundjoin%
\definecolor{currentfill}{rgb}{0.000000,0.000000,0.000000}%
\pgfsetfillcolor{currentfill}%
\pgfsetlinewidth{0.602250pt}%
\definecolor{currentstroke}{rgb}{0.000000,0.000000,0.000000}%
\pgfsetstrokecolor{currentstroke}%
\pgfsetdash{}{0pt}%
\pgfsys@defobject{currentmarker}{\pgfqpoint{-0.027778in}{0.000000in}}{\pgfqpoint{-0.000000in}{0.000000in}}{%
\pgfpathmoveto{\pgfqpoint{-0.000000in}{0.000000in}}%
\pgfpathlineto{\pgfqpoint{-0.027778in}{0.000000in}}%
\pgfusepath{stroke,fill}%
}%
\begin{pgfscope}%
\pgfsys@transformshift{0.688192in}{2.646546in}%
\pgfsys@useobject{currentmarker}{}%
\end{pgfscope}%
\end{pgfscope}%
\begin{pgfscope}%
\pgfpathrectangle{\pgfqpoint{0.688192in}{0.613042in}}{\pgfqpoint{11.096108in}{4.223431in}}%
\pgfusepath{clip}%
\pgfsetrectcap%
\pgfsetroundjoin%
\pgfsetlinewidth{0.803000pt}%
\definecolor{currentstroke}{rgb}{0.690196,0.690196,0.690196}%
\pgfsetstrokecolor{currentstroke}%
\pgfsetstrokeopacity{0.050000}%
\pgfsetdash{}{0pt}%
\pgfpathmoveto{\pgfqpoint{0.688192in}{2.802969in}}%
\pgfpathlineto{\pgfqpoint{11.784299in}{2.802969in}}%
\pgfusepath{stroke}%
\end{pgfscope}%
\begin{pgfscope}%
\pgfsetbuttcap%
\pgfsetroundjoin%
\definecolor{currentfill}{rgb}{0.000000,0.000000,0.000000}%
\pgfsetfillcolor{currentfill}%
\pgfsetlinewidth{0.602250pt}%
\definecolor{currentstroke}{rgb}{0.000000,0.000000,0.000000}%
\pgfsetstrokecolor{currentstroke}%
\pgfsetdash{}{0pt}%
\pgfsys@defobject{currentmarker}{\pgfqpoint{-0.027778in}{0.000000in}}{\pgfqpoint{-0.000000in}{0.000000in}}{%
\pgfpathmoveto{\pgfqpoint{-0.000000in}{0.000000in}}%
\pgfpathlineto{\pgfqpoint{-0.027778in}{0.000000in}}%
\pgfusepath{stroke,fill}%
}%
\begin{pgfscope}%
\pgfsys@transformshift{0.688192in}{2.802969in}%
\pgfsys@useobject{currentmarker}{}%
\end{pgfscope}%
\end{pgfscope}%
\begin{pgfscope}%
\pgfpathrectangle{\pgfqpoint{0.688192in}{0.613042in}}{\pgfqpoint{11.096108in}{4.223431in}}%
\pgfusepath{clip}%
\pgfsetrectcap%
\pgfsetroundjoin%
\pgfsetlinewidth{0.803000pt}%
\definecolor{currentstroke}{rgb}{0.690196,0.690196,0.690196}%
\pgfsetstrokecolor{currentstroke}%
\pgfsetstrokeopacity{0.050000}%
\pgfsetdash{}{0pt}%
\pgfpathmoveto{\pgfqpoint{0.688192in}{3.115816in}}%
\pgfpathlineto{\pgfqpoint{11.784299in}{3.115816in}}%
\pgfusepath{stroke}%
\end{pgfscope}%
\begin{pgfscope}%
\pgfsetbuttcap%
\pgfsetroundjoin%
\definecolor{currentfill}{rgb}{0.000000,0.000000,0.000000}%
\pgfsetfillcolor{currentfill}%
\pgfsetlinewidth{0.602250pt}%
\definecolor{currentstroke}{rgb}{0.000000,0.000000,0.000000}%
\pgfsetstrokecolor{currentstroke}%
\pgfsetdash{}{0pt}%
\pgfsys@defobject{currentmarker}{\pgfqpoint{-0.027778in}{0.000000in}}{\pgfqpoint{-0.000000in}{0.000000in}}{%
\pgfpathmoveto{\pgfqpoint{-0.000000in}{0.000000in}}%
\pgfpathlineto{\pgfqpoint{-0.027778in}{0.000000in}}%
\pgfusepath{stroke,fill}%
}%
\begin{pgfscope}%
\pgfsys@transformshift{0.688192in}{3.115816in}%
\pgfsys@useobject{currentmarker}{}%
\end{pgfscope}%
\end{pgfscope}%
\begin{pgfscope}%
\pgfpathrectangle{\pgfqpoint{0.688192in}{0.613042in}}{\pgfqpoint{11.096108in}{4.223431in}}%
\pgfusepath{clip}%
\pgfsetrectcap%
\pgfsetroundjoin%
\pgfsetlinewidth{0.803000pt}%
\definecolor{currentstroke}{rgb}{0.690196,0.690196,0.690196}%
\pgfsetstrokecolor{currentstroke}%
\pgfsetstrokeopacity{0.050000}%
\pgfsetdash{}{0pt}%
\pgfpathmoveto{\pgfqpoint{0.688192in}{3.272239in}}%
\pgfpathlineto{\pgfqpoint{11.784299in}{3.272239in}}%
\pgfusepath{stroke}%
\end{pgfscope}%
\begin{pgfscope}%
\pgfsetbuttcap%
\pgfsetroundjoin%
\definecolor{currentfill}{rgb}{0.000000,0.000000,0.000000}%
\pgfsetfillcolor{currentfill}%
\pgfsetlinewidth{0.602250pt}%
\definecolor{currentstroke}{rgb}{0.000000,0.000000,0.000000}%
\pgfsetstrokecolor{currentstroke}%
\pgfsetdash{}{0pt}%
\pgfsys@defobject{currentmarker}{\pgfqpoint{-0.027778in}{0.000000in}}{\pgfqpoint{-0.000000in}{0.000000in}}{%
\pgfpathmoveto{\pgfqpoint{-0.000000in}{0.000000in}}%
\pgfpathlineto{\pgfqpoint{-0.027778in}{0.000000in}}%
\pgfusepath{stroke,fill}%
}%
\begin{pgfscope}%
\pgfsys@transformshift{0.688192in}{3.272239in}%
\pgfsys@useobject{currentmarker}{}%
\end{pgfscope}%
\end{pgfscope}%
\begin{pgfscope}%
\pgfpathrectangle{\pgfqpoint{0.688192in}{0.613042in}}{\pgfqpoint{11.096108in}{4.223431in}}%
\pgfusepath{clip}%
\pgfsetrectcap%
\pgfsetroundjoin%
\pgfsetlinewidth{0.803000pt}%
\definecolor{currentstroke}{rgb}{0.690196,0.690196,0.690196}%
\pgfsetstrokecolor{currentstroke}%
\pgfsetstrokeopacity{0.050000}%
\pgfsetdash{}{0pt}%
\pgfpathmoveto{\pgfqpoint{0.688192in}{3.428662in}}%
\pgfpathlineto{\pgfqpoint{11.784299in}{3.428662in}}%
\pgfusepath{stroke}%
\end{pgfscope}%
\begin{pgfscope}%
\pgfsetbuttcap%
\pgfsetroundjoin%
\definecolor{currentfill}{rgb}{0.000000,0.000000,0.000000}%
\pgfsetfillcolor{currentfill}%
\pgfsetlinewidth{0.602250pt}%
\definecolor{currentstroke}{rgb}{0.000000,0.000000,0.000000}%
\pgfsetstrokecolor{currentstroke}%
\pgfsetdash{}{0pt}%
\pgfsys@defobject{currentmarker}{\pgfqpoint{-0.027778in}{0.000000in}}{\pgfqpoint{-0.000000in}{0.000000in}}{%
\pgfpathmoveto{\pgfqpoint{-0.000000in}{0.000000in}}%
\pgfpathlineto{\pgfqpoint{-0.027778in}{0.000000in}}%
\pgfusepath{stroke,fill}%
}%
\begin{pgfscope}%
\pgfsys@transformshift{0.688192in}{3.428662in}%
\pgfsys@useobject{currentmarker}{}%
\end{pgfscope}%
\end{pgfscope}%
\begin{pgfscope}%
\pgfpathrectangle{\pgfqpoint{0.688192in}{0.613042in}}{\pgfqpoint{11.096108in}{4.223431in}}%
\pgfusepath{clip}%
\pgfsetrectcap%
\pgfsetroundjoin%
\pgfsetlinewidth{0.803000pt}%
\definecolor{currentstroke}{rgb}{0.690196,0.690196,0.690196}%
\pgfsetstrokecolor{currentstroke}%
\pgfsetstrokeopacity{0.050000}%
\pgfsetdash{}{0pt}%
\pgfpathmoveto{\pgfqpoint{0.688192in}{3.585086in}}%
\pgfpathlineto{\pgfqpoint{11.784299in}{3.585086in}}%
\pgfusepath{stroke}%
\end{pgfscope}%
\begin{pgfscope}%
\pgfsetbuttcap%
\pgfsetroundjoin%
\definecolor{currentfill}{rgb}{0.000000,0.000000,0.000000}%
\pgfsetfillcolor{currentfill}%
\pgfsetlinewidth{0.602250pt}%
\definecolor{currentstroke}{rgb}{0.000000,0.000000,0.000000}%
\pgfsetstrokecolor{currentstroke}%
\pgfsetdash{}{0pt}%
\pgfsys@defobject{currentmarker}{\pgfqpoint{-0.027778in}{0.000000in}}{\pgfqpoint{-0.000000in}{0.000000in}}{%
\pgfpathmoveto{\pgfqpoint{-0.000000in}{0.000000in}}%
\pgfpathlineto{\pgfqpoint{-0.027778in}{0.000000in}}%
\pgfusepath{stroke,fill}%
}%
\begin{pgfscope}%
\pgfsys@transformshift{0.688192in}{3.585086in}%
\pgfsys@useobject{currentmarker}{}%
\end{pgfscope}%
\end{pgfscope}%
\begin{pgfscope}%
\pgfpathrectangle{\pgfqpoint{0.688192in}{0.613042in}}{\pgfqpoint{11.096108in}{4.223431in}}%
\pgfusepath{clip}%
\pgfsetrectcap%
\pgfsetroundjoin%
\pgfsetlinewidth{0.803000pt}%
\definecolor{currentstroke}{rgb}{0.690196,0.690196,0.690196}%
\pgfsetstrokecolor{currentstroke}%
\pgfsetstrokeopacity{0.050000}%
\pgfsetdash{}{0pt}%
\pgfpathmoveto{\pgfqpoint{0.688192in}{3.897932in}}%
\pgfpathlineto{\pgfqpoint{11.784299in}{3.897932in}}%
\pgfusepath{stroke}%
\end{pgfscope}%
\begin{pgfscope}%
\pgfsetbuttcap%
\pgfsetroundjoin%
\definecolor{currentfill}{rgb}{0.000000,0.000000,0.000000}%
\pgfsetfillcolor{currentfill}%
\pgfsetlinewidth{0.602250pt}%
\definecolor{currentstroke}{rgb}{0.000000,0.000000,0.000000}%
\pgfsetstrokecolor{currentstroke}%
\pgfsetdash{}{0pt}%
\pgfsys@defobject{currentmarker}{\pgfqpoint{-0.027778in}{0.000000in}}{\pgfqpoint{-0.000000in}{0.000000in}}{%
\pgfpathmoveto{\pgfqpoint{-0.000000in}{0.000000in}}%
\pgfpathlineto{\pgfqpoint{-0.027778in}{0.000000in}}%
\pgfusepath{stroke,fill}%
}%
\begin{pgfscope}%
\pgfsys@transformshift{0.688192in}{3.897932in}%
\pgfsys@useobject{currentmarker}{}%
\end{pgfscope}%
\end{pgfscope}%
\begin{pgfscope}%
\pgfpathrectangle{\pgfqpoint{0.688192in}{0.613042in}}{\pgfqpoint{11.096108in}{4.223431in}}%
\pgfusepath{clip}%
\pgfsetrectcap%
\pgfsetroundjoin%
\pgfsetlinewidth{0.803000pt}%
\definecolor{currentstroke}{rgb}{0.690196,0.690196,0.690196}%
\pgfsetstrokecolor{currentstroke}%
\pgfsetstrokeopacity{0.050000}%
\pgfsetdash{}{0pt}%
\pgfpathmoveto{\pgfqpoint{0.688192in}{4.054356in}}%
\pgfpathlineto{\pgfqpoint{11.784299in}{4.054356in}}%
\pgfusepath{stroke}%
\end{pgfscope}%
\begin{pgfscope}%
\pgfsetbuttcap%
\pgfsetroundjoin%
\definecolor{currentfill}{rgb}{0.000000,0.000000,0.000000}%
\pgfsetfillcolor{currentfill}%
\pgfsetlinewidth{0.602250pt}%
\definecolor{currentstroke}{rgb}{0.000000,0.000000,0.000000}%
\pgfsetstrokecolor{currentstroke}%
\pgfsetdash{}{0pt}%
\pgfsys@defobject{currentmarker}{\pgfqpoint{-0.027778in}{0.000000in}}{\pgfqpoint{-0.000000in}{0.000000in}}{%
\pgfpathmoveto{\pgfqpoint{-0.000000in}{0.000000in}}%
\pgfpathlineto{\pgfqpoint{-0.027778in}{0.000000in}}%
\pgfusepath{stroke,fill}%
}%
\begin{pgfscope}%
\pgfsys@transformshift{0.688192in}{4.054356in}%
\pgfsys@useobject{currentmarker}{}%
\end{pgfscope}%
\end{pgfscope}%
\begin{pgfscope}%
\pgfpathrectangle{\pgfqpoint{0.688192in}{0.613042in}}{\pgfqpoint{11.096108in}{4.223431in}}%
\pgfusepath{clip}%
\pgfsetrectcap%
\pgfsetroundjoin%
\pgfsetlinewidth{0.803000pt}%
\definecolor{currentstroke}{rgb}{0.690196,0.690196,0.690196}%
\pgfsetstrokecolor{currentstroke}%
\pgfsetstrokeopacity{0.050000}%
\pgfsetdash{}{0pt}%
\pgfpathmoveto{\pgfqpoint{0.688192in}{4.210779in}}%
\pgfpathlineto{\pgfqpoint{11.784299in}{4.210779in}}%
\pgfusepath{stroke}%
\end{pgfscope}%
\begin{pgfscope}%
\pgfsetbuttcap%
\pgfsetroundjoin%
\definecolor{currentfill}{rgb}{0.000000,0.000000,0.000000}%
\pgfsetfillcolor{currentfill}%
\pgfsetlinewidth{0.602250pt}%
\definecolor{currentstroke}{rgb}{0.000000,0.000000,0.000000}%
\pgfsetstrokecolor{currentstroke}%
\pgfsetdash{}{0pt}%
\pgfsys@defobject{currentmarker}{\pgfqpoint{-0.027778in}{0.000000in}}{\pgfqpoint{-0.000000in}{0.000000in}}{%
\pgfpathmoveto{\pgfqpoint{-0.000000in}{0.000000in}}%
\pgfpathlineto{\pgfqpoint{-0.027778in}{0.000000in}}%
\pgfusepath{stroke,fill}%
}%
\begin{pgfscope}%
\pgfsys@transformshift{0.688192in}{4.210779in}%
\pgfsys@useobject{currentmarker}{}%
\end{pgfscope}%
\end{pgfscope}%
\begin{pgfscope}%
\pgfpathrectangle{\pgfqpoint{0.688192in}{0.613042in}}{\pgfqpoint{11.096108in}{4.223431in}}%
\pgfusepath{clip}%
\pgfsetrectcap%
\pgfsetroundjoin%
\pgfsetlinewidth{0.803000pt}%
\definecolor{currentstroke}{rgb}{0.690196,0.690196,0.690196}%
\pgfsetstrokecolor{currentstroke}%
\pgfsetstrokeopacity{0.050000}%
\pgfsetdash{}{0pt}%
\pgfpathmoveto{\pgfqpoint{0.688192in}{4.367202in}}%
\pgfpathlineto{\pgfqpoint{11.784299in}{4.367202in}}%
\pgfusepath{stroke}%
\end{pgfscope}%
\begin{pgfscope}%
\pgfsetbuttcap%
\pgfsetroundjoin%
\definecolor{currentfill}{rgb}{0.000000,0.000000,0.000000}%
\pgfsetfillcolor{currentfill}%
\pgfsetlinewidth{0.602250pt}%
\definecolor{currentstroke}{rgb}{0.000000,0.000000,0.000000}%
\pgfsetstrokecolor{currentstroke}%
\pgfsetdash{}{0pt}%
\pgfsys@defobject{currentmarker}{\pgfqpoint{-0.027778in}{0.000000in}}{\pgfqpoint{-0.000000in}{0.000000in}}{%
\pgfpathmoveto{\pgfqpoint{-0.000000in}{0.000000in}}%
\pgfpathlineto{\pgfqpoint{-0.027778in}{0.000000in}}%
\pgfusepath{stroke,fill}%
}%
\begin{pgfscope}%
\pgfsys@transformshift{0.688192in}{4.367202in}%
\pgfsys@useobject{currentmarker}{}%
\end{pgfscope}%
\end{pgfscope}%
\begin{pgfscope}%
\pgfpathrectangle{\pgfqpoint{0.688192in}{0.613042in}}{\pgfqpoint{11.096108in}{4.223431in}}%
\pgfusepath{clip}%
\pgfsetrectcap%
\pgfsetroundjoin%
\pgfsetlinewidth{0.803000pt}%
\definecolor{currentstroke}{rgb}{0.690196,0.690196,0.690196}%
\pgfsetstrokecolor{currentstroke}%
\pgfsetstrokeopacity{0.050000}%
\pgfsetdash{}{0pt}%
\pgfpathmoveto{\pgfqpoint{0.688192in}{4.680049in}}%
\pgfpathlineto{\pgfqpoint{11.784299in}{4.680049in}}%
\pgfusepath{stroke}%
\end{pgfscope}%
\begin{pgfscope}%
\pgfsetbuttcap%
\pgfsetroundjoin%
\definecolor{currentfill}{rgb}{0.000000,0.000000,0.000000}%
\pgfsetfillcolor{currentfill}%
\pgfsetlinewidth{0.602250pt}%
\definecolor{currentstroke}{rgb}{0.000000,0.000000,0.000000}%
\pgfsetstrokecolor{currentstroke}%
\pgfsetdash{}{0pt}%
\pgfsys@defobject{currentmarker}{\pgfqpoint{-0.027778in}{0.000000in}}{\pgfqpoint{-0.000000in}{0.000000in}}{%
\pgfpathmoveto{\pgfqpoint{-0.000000in}{0.000000in}}%
\pgfpathlineto{\pgfqpoint{-0.027778in}{0.000000in}}%
\pgfusepath{stroke,fill}%
}%
\begin{pgfscope}%
\pgfsys@transformshift{0.688192in}{4.680049in}%
\pgfsys@useobject{currentmarker}{}%
\end{pgfscope}%
\end{pgfscope}%
\begin{pgfscope}%
\pgfpathrectangle{\pgfqpoint{0.688192in}{0.613042in}}{\pgfqpoint{11.096108in}{4.223431in}}%
\pgfusepath{clip}%
\pgfsetrectcap%
\pgfsetroundjoin%
\pgfsetlinewidth{0.803000pt}%
\definecolor{currentstroke}{rgb}{0.690196,0.690196,0.690196}%
\pgfsetstrokecolor{currentstroke}%
\pgfsetstrokeopacity{0.050000}%
\pgfsetdash{}{0pt}%
\pgfpathmoveto{\pgfqpoint{0.688192in}{4.836473in}}%
\pgfpathlineto{\pgfqpoint{11.784299in}{4.836473in}}%
\pgfusepath{stroke}%
\end{pgfscope}%
\begin{pgfscope}%
\pgfsetbuttcap%
\pgfsetroundjoin%
\definecolor{currentfill}{rgb}{0.000000,0.000000,0.000000}%
\pgfsetfillcolor{currentfill}%
\pgfsetlinewidth{0.602250pt}%
\definecolor{currentstroke}{rgb}{0.000000,0.000000,0.000000}%
\pgfsetstrokecolor{currentstroke}%
\pgfsetdash{}{0pt}%
\pgfsys@defobject{currentmarker}{\pgfqpoint{-0.027778in}{0.000000in}}{\pgfqpoint{-0.000000in}{0.000000in}}{%
\pgfpathmoveto{\pgfqpoint{-0.000000in}{0.000000in}}%
\pgfpathlineto{\pgfqpoint{-0.027778in}{0.000000in}}%
\pgfusepath{stroke,fill}%
}%
\begin{pgfscope}%
\pgfsys@transformshift{0.688192in}{4.836473in}%
\pgfsys@useobject{currentmarker}{}%
\end{pgfscope}%
\end{pgfscope}%
\begin{pgfscope}%
\definecolor{textcolor}{rgb}{0.000000,0.000000,0.000000}%
\pgfsetstrokecolor{textcolor}%
\pgfsetfillcolor{textcolor}%
\pgftext[x=0.339583in,y=2.724757in,,bottom,rotate=90.000000]{\color{textcolor}{\rmfamily\fontsize{18.000000}{21.600000}\selectfont\catcode`\^=\active\def^{\ifmmode\sp\else\^{}\fi}\catcode`\%=\active\def%{\%}Temoa MGA Capacity (GW)}}%
\end{pgfscope}%
\begin{pgfscope}%
\pgfpathrectangle{\pgfqpoint{0.688192in}{0.613042in}}{\pgfqpoint{11.096108in}{4.223431in}}%
\pgfusepath{clip}%
\pgfsetbuttcap%
\pgfsetroundjoin%
\pgfsetlinewidth{0.941016pt}%
\definecolor{currentstroke}{rgb}{0.240000,0.240000,0.240000}%
\pgfsetstrokecolor{currentstroke}%
\pgfsetdash{}{0pt}%
\pgfpathmoveto{\pgfqpoint{0.799153in}{3.072033in}}%
\pgfpathlineto{\pgfqpoint{1.686841in}{3.072033in}}%
\pgfusepath{stroke}%
\end{pgfscope}%
\begin{pgfscope}%
\pgfpathrectangle{\pgfqpoint{0.688192in}{0.613042in}}{\pgfqpoint{11.096108in}{4.223431in}}%
\pgfusepath{clip}%
\pgfsetbuttcap%
\pgfsetroundjoin%
\pgfsetlinewidth{0.941016pt}%
\definecolor{currentstroke}{rgb}{0.240000,0.240000,0.240000}%
\pgfsetstrokecolor{currentstroke}%
\pgfsetdash{}{0pt}%
\pgfpathmoveto{\pgfqpoint{1.908763in}{0.613042in}}%
\pgfpathlineto{\pgfqpoint{2.796452in}{0.613042in}}%
\pgfusepath{stroke}%
\end{pgfscope}%
\begin{pgfscope}%
\pgfpathrectangle{\pgfqpoint{0.688192in}{0.613042in}}{\pgfqpoint{11.096108in}{4.223431in}}%
\pgfusepath{clip}%
\pgfsetbuttcap%
\pgfsetroundjoin%
\pgfsetlinewidth{0.941016pt}%
\definecolor{currentstroke}{rgb}{0.240000,0.240000,0.240000}%
\pgfsetstrokecolor{currentstroke}%
\pgfsetdash{}{0pt}%
\pgfpathmoveto{\pgfqpoint{3.018374in}{1.814832in}}%
\pgfpathlineto{\pgfqpoint{3.906063in}{1.814832in}}%
\pgfusepath{stroke}%
\end{pgfscope}%
\begin{pgfscope}%
\pgfpathrectangle{\pgfqpoint{0.688192in}{0.613042in}}{\pgfqpoint{11.096108in}{4.223431in}}%
\pgfusepath{clip}%
\pgfsetbuttcap%
\pgfsetroundjoin%
\pgfsetlinewidth{0.941016pt}%
\definecolor{currentstroke}{rgb}{0.240000,0.240000,0.240000}%
\pgfsetstrokecolor{currentstroke}%
\pgfsetdash{}{0pt}%
\pgfpathmoveto{\pgfqpoint{4.127985in}{0.795846in}}%
\pgfpathlineto{\pgfqpoint{5.015674in}{0.795846in}}%
\pgfusepath{stroke}%
\end{pgfscope}%
\begin{pgfscope}%
\pgfpathrectangle{\pgfqpoint{0.688192in}{0.613042in}}{\pgfqpoint{11.096108in}{4.223431in}}%
\pgfusepath{clip}%
\pgfsetbuttcap%
\pgfsetroundjoin%
\pgfsetlinewidth{0.941016pt}%
\definecolor{currentstroke}{rgb}{0.240000,0.240000,0.240000}%
\pgfsetstrokecolor{currentstroke}%
\pgfsetdash{}{0pt}%
\pgfpathmoveto{\pgfqpoint{5.237596in}{0.759541in}}%
\pgfpathlineto{\pgfqpoint{6.125284in}{0.759541in}}%
\pgfusepath{stroke}%
\end{pgfscope}%
\begin{pgfscope}%
\pgfpathrectangle{\pgfqpoint{0.688192in}{0.613042in}}{\pgfqpoint{11.096108in}{4.223431in}}%
\pgfusepath{clip}%
\pgfsetbuttcap%
\pgfsetroundjoin%
\pgfsetlinewidth{0.941016pt}%
\definecolor{currentstroke}{rgb}{0.240000,0.240000,0.240000}%
\pgfsetstrokecolor{currentstroke}%
\pgfsetdash{}{0pt}%
\pgfpathmoveto{\pgfqpoint{6.347207in}{0.613042in}}%
\pgfpathlineto{\pgfqpoint{7.234895in}{0.613042in}}%
\pgfusepath{stroke}%
\end{pgfscope}%
\begin{pgfscope}%
\pgfpathrectangle{\pgfqpoint{0.688192in}{0.613042in}}{\pgfqpoint{11.096108in}{4.223431in}}%
\pgfusepath{clip}%
\pgfsetbuttcap%
\pgfsetroundjoin%
\pgfsetlinewidth{0.941016pt}%
\definecolor{currentstroke}{rgb}{0.240000,0.240000,0.240000}%
\pgfsetstrokecolor{currentstroke}%
\pgfsetdash{}{0pt}%
\pgfpathmoveto{\pgfqpoint{7.456817in}{0.613042in}}%
\pgfpathlineto{\pgfqpoint{8.344506in}{0.613042in}}%
\pgfusepath{stroke}%
\end{pgfscope}%
\begin{pgfscope}%
\pgfpathrectangle{\pgfqpoint{0.688192in}{0.613042in}}{\pgfqpoint{11.096108in}{4.223431in}}%
\pgfusepath{clip}%
\pgfsetbuttcap%
\pgfsetroundjoin%
\pgfsetlinewidth{0.941016pt}%
\definecolor{currentstroke}{rgb}{0.240000,0.240000,0.240000}%
\pgfsetstrokecolor{currentstroke}%
\pgfsetdash{}{0pt}%
\pgfpathmoveto{\pgfqpoint{8.566428in}{0.699492in}}%
\pgfpathlineto{\pgfqpoint{9.454117in}{0.699492in}}%
\pgfusepath{stroke}%
\end{pgfscope}%
\begin{pgfscope}%
\pgfpathrectangle{\pgfqpoint{0.688192in}{0.613042in}}{\pgfqpoint{11.096108in}{4.223431in}}%
\pgfusepath{clip}%
\pgfsetbuttcap%
\pgfsetroundjoin%
\pgfsetlinewidth{0.941016pt}%
\definecolor{currentstroke}{rgb}{0.240000,0.240000,0.240000}%
\pgfsetstrokecolor{currentstroke}%
\pgfsetdash{}{0pt}%
\pgfpathmoveto{\pgfqpoint{9.676039in}{1.155254in}}%
\pgfpathlineto{\pgfqpoint{10.563728in}{1.155254in}}%
\pgfusepath{stroke}%
\end{pgfscope}%
\begin{pgfscope}%
\pgfpathrectangle{\pgfqpoint{0.688192in}{0.613042in}}{\pgfqpoint{11.096108in}{4.223431in}}%
\pgfusepath{clip}%
\pgfsetbuttcap%
\pgfsetroundjoin%
\pgfsetlinewidth{0.941016pt}%
\definecolor{currentstroke}{rgb}{0.240000,0.240000,0.240000}%
\pgfsetstrokecolor{currentstroke}%
\pgfsetdash{}{0pt}%
\pgfpathmoveto{\pgfqpoint{10.785650in}{0.638803in}}%
\pgfpathlineto{\pgfqpoint{11.673338in}{0.638803in}}%
\pgfusepath{stroke}%
\end{pgfscope}%
\begin{pgfscope}%
\pgfsetrectcap%
\pgfsetmiterjoin%
\pgfsetlinewidth{0.803000pt}%
\definecolor{currentstroke}{rgb}{0.000000,0.000000,0.000000}%
\pgfsetstrokecolor{currentstroke}%
\pgfsetdash{}{0pt}%
\pgfpathmoveto{\pgfqpoint{0.688192in}{0.613042in}}%
\pgfpathlineto{\pgfqpoint{0.688192in}{4.836473in}}%
\pgfusepath{stroke}%
\end{pgfscope}%
\begin{pgfscope}%
\pgfsetrectcap%
\pgfsetmiterjoin%
\pgfsetlinewidth{0.803000pt}%
\definecolor{currentstroke}{rgb}{0.000000,0.000000,0.000000}%
\pgfsetstrokecolor{currentstroke}%
\pgfsetdash{}{0pt}%
\pgfpathmoveto{\pgfqpoint{11.784299in}{0.613042in}}%
\pgfpathlineto{\pgfqpoint{11.784299in}{4.836473in}}%
\pgfusepath{stroke}%
\end{pgfscope}%
\begin{pgfscope}%
\pgfsetrectcap%
\pgfsetmiterjoin%
\pgfsetlinewidth{0.803000pt}%
\definecolor{currentstroke}{rgb}{0.000000,0.000000,0.000000}%
\pgfsetstrokecolor{currentstroke}%
\pgfsetdash{}{0pt}%
\pgfpathmoveto{\pgfqpoint{0.688192in}{0.613042in}}%
\pgfpathlineto{\pgfqpoint{11.784299in}{0.613042in}}%
\pgfusepath{stroke}%
\end{pgfscope}%
\begin{pgfscope}%
\pgfsetrectcap%
\pgfsetmiterjoin%
\pgfsetlinewidth{0.803000pt}%
\definecolor{currentstroke}{rgb}{0.000000,0.000000,0.000000}%
\pgfsetstrokecolor{currentstroke}%
\pgfsetdash{}{0pt}%
\pgfpathmoveto{\pgfqpoint{0.688192in}{4.836473in}}%
\pgfpathlineto{\pgfqpoint{11.784299in}{4.836473in}}%
\pgfusepath{stroke}%
\end{pgfscope}%
\begin{pgfscope}%
\pgfsetbuttcap%
\pgfsetmiterjoin%
\definecolor{currentfill}{rgb}{1.000000,1.000000,1.000000}%
\pgfsetfillcolor{currentfill}%
\pgfsetlinewidth{1.003750pt}%
\definecolor{currentstroke}{rgb}{0.000000,0.000000,0.000000}%
\pgfsetstrokecolor{currentstroke}%
\pgfsetdash{}{0pt}%
\pgfpathmoveto{\pgfqpoint{0.842629in}{4.409416in}}%
\pgfpathlineto{\pgfqpoint{1.118599in}{4.409416in}}%
\pgfpathlineto{\pgfqpoint{1.118599in}{4.722193in}}%
\pgfpathlineto{\pgfqpoint{0.842629in}{4.722193in}}%
\pgfpathlineto{\pgfqpoint{0.842629in}{4.409416in}}%
\pgfpathclose%
\pgfusepath{stroke,fill}%
\end{pgfscope}%
\begin{pgfscope}%
\definecolor{textcolor}{rgb}{0.000000,0.000000,0.000000}%
\pgfsetstrokecolor{textcolor}%
\pgfsetfillcolor{textcolor}%
\pgftext[x=0.899018in,y=4.515805in,left,base]{\color{textcolor}{\rmfamily\fontsize{14.000000}{16.800000}\selectfont\catcode`\^=\active\def^{\ifmmode\sp\else\^{}\fi}\catcode`\%=\active\def%{\%}c)}}%
\end{pgfscope}%
\begin{pgfscope}%
\pgfsetbuttcap%
\pgfsetmiterjoin%
\definecolor{currentfill}{rgb}{1.000000,1.000000,1.000000}%
\pgfsetfillcolor{currentfill}%
\pgfsetfillopacity{0.800000}%
\pgfsetlinewidth{1.003750pt}%
\definecolor{currentstroke}{rgb}{0.800000,0.800000,0.800000}%
\pgfsetstrokecolor{currentstroke}%
\pgfsetstrokeopacity{0.800000}%
\pgfsetdash{}{0pt}%
\pgfpathmoveto{\pgfqpoint{5.605288in}{3.305919in}}%
\pgfpathlineto{\pgfqpoint{11.648188in}{3.305919in}}%
\pgfpathquadraticcurveto{\pgfqpoint{11.687077in}{3.305919in}}{\pgfqpoint{11.687077in}{3.344808in}}%
\pgfpathlineto{\pgfqpoint{11.687077in}{4.700361in}}%
\pgfpathquadraticcurveto{\pgfqpoint{11.687077in}{4.739250in}}{\pgfqpoint{11.648188in}{4.739250in}}%
\pgfpathlineto{\pgfqpoint{5.605288in}{4.739250in}}%
\pgfpathquadraticcurveto{\pgfqpoint{5.566399in}{4.739250in}}{\pgfqpoint{5.566399in}{4.700361in}}%
\pgfpathlineto{\pgfqpoint{5.566399in}{3.344808in}}%
\pgfpathquadraticcurveto{\pgfqpoint{5.566399in}{3.305919in}}{\pgfqpoint{5.605288in}{3.305919in}}%
\pgfpathlineto{\pgfqpoint{5.605288in}{3.305919in}}%
\pgfpathclose%
\pgfusepath{stroke,fill}%
\end{pgfscope}%
\begin{pgfscope}%
\definecolor{textcolor}{rgb}{0.000000,0.000000,0.000000}%
\pgfsetstrokecolor{textcolor}%
\pgfsetfillcolor{textcolor}%
\pgftext[x=8.087659in,y=4.522584in,left,base]{\color{textcolor}{\rmfamily\fontsize{14.000000}{16.800000}\selectfont\catcode`\^=\active\def^{\ifmmode\sp\else\^{}\fi}\catcode`\%=\active\def%{\%}Technologies}}%
\end{pgfscope}%
\begin{pgfscope}%
\pgfsetbuttcap%
\pgfsetmiterjoin%
\definecolor{currentfill}{rgb}{0.121569,0.466667,0.705882}%
\pgfsetfillcolor{currentfill}%
\pgfsetlinewidth{1.003750pt}%
\definecolor{currentstroke}{rgb}{0.121569,0.466667,0.705882}%
\pgfsetstrokecolor{currentstroke}%
\pgfsetdash{}{0pt}%
\pgfpathmoveto{\pgfqpoint{5.644176in}{4.247584in}}%
\pgfpathlineto{\pgfqpoint{6.033065in}{4.247584in}}%
\pgfpathlineto{\pgfqpoint{6.033065in}{4.383695in}}%
\pgfpathlineto{\pgfqpoint{5.644176in}{4.383695in}}%
\pgfpathlineto{\pgfqpoint{5.644176in}{4.247584in}}%
\pgfpathclose%
\pgfusepath{stroke,fill}%
\end{pgfscope}%
\begin{pgfscope}%
\definecolor{textcolor}{rgb}{0.000000,0.000000,0.000000}%
\pgfsetstrokecolor{textcolor}%
\pgfsetfillcolor{textcolor}%
\pgftext[x=6.188621in,y=4.247584in,left,base]{\color{textcolor}{\rmfamily\fontsize{14.000000}{16.800000}\selectfont\catcode`\^=\active\def^{\ifmmode\sp\else\^{}\fi}\catcode`\%=\active\def%{\%}Nuclear}}%
\end{pgfscope}%
\begin{pgfscope}%
\pgfsetbuttcap%
\pgfsetmiterjoin%
\definecolor{currentfill}{rgb}{1.000000,0.498039,0.054902}%
\pgfsetfillcolor{currentfill}%
\pgfsetlinewidth{1.003750pt}%
\definecolor{currentstroke}{rgb}{1.000000,0.498039,0.054902}%
\pgfsetstrokecolor{currentstroke}%
\pgfsetdash{}{0pt}%
\pgfpathmoveto{\pgfqpoint{5.644176in}{3.972585in}}%
\pgfpathlineto{\pgfqpoint{6.033065in}{3.972585in}}%
\pgfpathlineto{\pgfqpoint{6.033065in}{4.108696in}}%
\pgfpathlineto{\pgfqpoint{5.644176in}{4.108696in}}%
\pgfpathlineto{\pgfqpoint{5.644176in}{3.972585in}}%
\pgfpathclose%
\pgfusepath{stroke,fill}%
\end{pgfscope}%
\begin{pgfscope}%
\definecolor{textcolor}{rgb}{0.000000,0.000000,0.000000}%
\pgfsetstrokecolor{textcolor}%
\pgfsetfillcolor{textcolor}%
\pgftext[x=6.188621in,y=3.972585in,left,base]{\color{textcolor}{\rmfamily\fontsize{14.000000}{16.800000}\selectfont\catcode`\^=\active\def^{\ifmmode\sp\else\^{}\fi}\catcode`\%=\active\def%{\%}Nuclear\_Adv}}%
\end{pgfscope}%
\begin{pgfscope}%
\pgfsetbuttcap%
\pgfsetmiterjoin%
\definecolor{currentfill}{rgb}{0.172549,0.627451,0.172549}%
\pgfsetfillcolor{currentfill}%
\pgfsetlinewidth{1.003750pt}%
\definecolor{currentstroke}{rgb}{0.172549,0.627451,0.172549}%
\pgfsetstrokecolor{currentstroke}%
\pgfsetdash{}{0pt}%
\pgfpathmoveto{\pgfqpoint{5.644176in}{3.697585in}}%
\pgfpathlineto{\pgfqpoint{6.033065in}{3.697585in}}%
\pgfpathlineto{\pgfqpoint{6.033065in}{3.833696in}}%
\pgfpathlineto{\pgfqpoint{5.644176in}{3.833696in}}%
\pgfpathlineto{\pgfqpoint{5.644176in}{3.697585in}}%
\pgfpathclose%
\pgfusepath{stroke,fill}%
\end{pgfscope}%
\begin{pgfscope}%
\definecolor{textcolor}{rgb}{0.000000,0.000000,0.000000}%
\pgfsetstrokecolor{textcolor}%
\pgfsetfillcolor{textcolor}%
\pgftext[x=6.188621in,y=3.697585in,left,base]{\color{textcolor}{\rmfamily\fontsize{14.000000}{16.800000}\selectfont\catcode`\^=\active\def^{\ifmmode\sp\else\^{}\fi}\catcode`\%=\active\def%{\%}NaturalGas\_Conv}}%
\end{pgfscope}%
\begin{pgfscope}%
\pgfsetbuttcap%
\pgfsetmiterjoin%
\definecolor{currentfill}{rgb}{0.839216,0.152941,0.156863}%
\pgfsetfillcolor{currentfill}%
\pgfsetlinewidth{1.003750pt}%
\definecolor{currentstroke}{rgb}{0.839216,0.152941,0.156863}%
\pgfsetstrokecolor{currentstroke}%
\pgfsetdash{}{0pt}%
\pgfpathmoveto{\pgfqpoint{5.644176in}{3.422586in}}%
\pgfpathlineto{\pgfqpoint{6.033065in}{3.422586in}}%
\pgfpathlineto{\pgfqpoint{6.033065in}{3.558697in}}%
\pgfpathlineto{\pgfqpoint{5.644176in}{3.558697in}}%
\pgfpathlineto{\pgfqpoint{5.644176in}{3.422586in}}%
\pgfpathclose%
\pgfusepath{stroke,fill}%
\end{pgfscope}%
\begin{pgfscope}%
\definecolor{textcolor}{rgb}{0.000000,0.000000,0.000000}%
\pgfsetstrokecolor{textcolor}%
\pgfsetfillcolor{textcolor}%
\pgftext[x=6.188621in,y=3.422586in,left,base]{\color{textcolor}{\rmfamily\fontsize{14.000000}{16.800000}\selectfont\catcode`\^=\active\def^{\ifmmode\sp\else\^{}\fi}\catcode`\%=\active\def%{\%}NaturalGas\_Adv}}%
\end{pgfscope}%
\begin{pgfscope}%
\pgfsetbuttcap%
\pgfsetmiterjoin%
\definecolor{currentfill}{rgb}{0.580392,0.403922,0.741176}%
\pgfsetfillcolor{currentfill}%
\pgfsetlinewidth{1.003750pt}%
\definecolor{currentstroke}{rgb}{0.580392,0.403922,0.741176}%
\pgfsetstrokecolor{currentstroke}%
\pgfsetdash{}{0pt}%
\pgfpathmoveto{\pgfqpoint{8.081024in}{4.247584in}}%
\pgfpathlineto{\pgfqpoint{8.469913in}{4.247584in}}%
\pgfpathlineto{\pgfqpoint{8.469913in}{4.383695in}}%
\pgfpathlineto{\pgfqpoint{8.081024in}{4.383695in}}%
\pgfpathlineto{\pgfqpoint{8.081024in}{4.247584in}}%
\pgfpathclose%
\pgfusepath{stroke,fill}%
\end{pgfscope}%
\begin{pgfscope}%
\definecolor{textcolor}{rgb}{0.000000,0.000000,0.000000}%
\pgfsetstrokecolor{textcolor}%
\pgfsetfillcolor{textcolor}%
\pgftext[x=8.625468in,y=4.247584in,left,base]{\color{textcolor}{\rmfamily\fontsize{14.000000}{16.800000}\selectfont\catcode`\^=\active\def^{\ifmmode\sp\else\^{}\fi}\catcode`\%=\active\def%{\%}Coal\_Conv}}%
\end{pgfscope}%
\begin{pgfscope}%
\pgfsetbuttcap%
\pgfsetmiterjoin%
\definecolor{currentfill}{rgb}{0.549020,0.337255,0.294118}%
\pgfsetfillcolor{currentfill}%
\pgfsetlinewidth{1.003750pt}%
\definecolor{currentstroke}{rgb}{0.549020,0.337255,0.294118}%
\pgfsetstrokecolor{currentstroke}%
\pgfsetdash{}{0pt}%
\pgfpathmoveto{\pgfqpoint{8.081024in}{3.972585in}}%
\pgfpathlineto{\pgfqpoint{8.469913in}{3.972585in}}%
\pgfpathlineto{\pgfqpoint{8.469913in}{4.108696in}}%
\pgfpathlineto{\pgfqpoint{8.081024in}{4.108696in}}%
\pgfpathlineto{\pgfqpoint{8.081024in}{3.972585in}}%
\pgfpathclose%
\pgfusepath{stroke,fill}%
\end{pgfscope}%
\begin{pgfscope}%
\definecolor{textcolor}{rgb}{0.000000,0.000000,0.000000}%
\pgfsetstrokecolor{textcolor}%
\pgfsetfillcolor{textcolor}%
\pgftext[x=8.625468in,y=3.972585in,left,base]{\color{textcolor}{\rmfamily\fontsize{14.000000}{16.800000}\selectfont\catcode`\^=\active\def^{\ifmmode\sp\else\^{}\fi}\catcode`\%=\active\def%{\%}Coal\_Adv}}%
\end{pgfscope}%
\begin{pgfscope}%
\pgfsetbuttcap%
\pgfsetmiterjoin%
\definecolor{currentfill}{rgb}{0.890196,0.466667,0.760784}%
\pgfsetfillcolor{currentfill}%
\pgfsetlinewidth{1.003750pt}%
\definecolor{currentstroke}{rgb}{0.890196,0.466667,0.760784}%
\pgfsetstrokecolor{currentstroke}%
\pgfsetdash{}{0pt}%
\pgfpathmoveto{\pgfqpoint{8.081024in}{3.697585in}}%
\pgfpathlineto{\pgfqpoint{8.469913in}{3.697585in}}%
\pgfpathlineto{\pgfqpoint{8.469913in}{3.833696in}}%
\pgfpathlineto{\pgfqpoint{8.081024in}{3.833696in}}%
\pgfpathlineto{\pgfqpoint{8.081024in}{3.697585in}}%
\pgfpathclose%
\pgfusepath{stroke,fill}%
\end{pgfscope}%
\begin{pgfscope}%
\definecolor{textcolor}{rgb}{0.000000,0.000000,0.000000}%
\pgfsetstrokecolor{textcolor}%
\pgfsetfillcolor{textcolor}%
\pgftext[x=8.625468in,y=3.697585in,left,base]{\color{textcolor}{\rmfamily\fontsize{14.000000}{16.800000}\selectfont\catcode`\^=\active\def^{\ifmmode\sp\else\^{}\fi}\catcode`\%=\active\def%{\%}Biomass}}%
\end{pgfscope}%
\begin{pgfscope}%
\pgfsetbuttcap%
\pgfsetmiterjoin%
\definecolor{currentfill}{rgb}{0.498039,0.498039,0.498039}%
\pgfsetfillcolor{currentfill}%
\pgfsetlinewidth{1.003750pt}%
\definecolor{currentstroke}{rgb}{0.498039,0.498039,0.498039}%
\pgfsetstrokecolor{currentstroke}%
\pgfsetdash{}{0pt}%
\pgfpathmoveto{\pgfqpoint{9.922577in}{4.247584in}}%
\pgfpathlineto{\pgfqpoint{10.311466in}{4.247584in}}%
\pgfpathlineto{\pgfqpoint{10.311466in}{4.383695in}}%
\pgfpathlineto{\pgfqpoint{9.922577in}{4.383695in}}%
\pgfpathlineto{\pgfqpoint{9.922577in}{4.247584in}}%
\pgfpathclose%
\pgfusepath{stroke,fill}%
\end{pgfscope}%
\begin{pgfscope}%
\definecolor{textcolor}{rgb}{0.000000,0.000000,0.000000}%
\pgfsetstrokecolor{textcolor}%
\pgfsetfillcolor{textcolor}%
\pgftext[x=10.467022in,y=4.247584in,left,base]{\color{textcolor}{\rmfamily\fontsize{14.000000}{16.800000}\selectfont\catcode`\^=\active\def^{\ifmmode\sp\else\^{}\fi}\catcode`\%=\active\def%{\%}Battery}}%
\end{pgfscope}%
\begin{pgfscope}%
\pgfsetbuttcap%
\pgfsetmiterjoin%
\definecolor{currentfill}{rgb}{0.737255,0.741176,0.133333}%
\pgfsetfillcolor{currentfill}%
\pgfsetlinewidth{1.003750pt}%
\definecolor{currentstroke}{rgb}{0.737255,0.741176,0.133333}%
\pgfsetstrokecolor{currentstroke}%
\pgfsetdash{}{0pt}%
\pgfpathmoveto{\pgfqpoint{9.922577in}{3.972585in}}%
\pgfpathlineto{\pgfqpoint{10.311466in}{3.972585in}}%
\pgfpathlineto{\pgfqpoint{10.311466in}{4.108696in}}%
\pgfpathlineto{\pgfqpoint{9.922577in}{4.108696in}}%
\pgfpathlineto{\pgfqpoint{9.922577in}{3.972585in}}%
\pgfpathclose%
\pgfusepath{stroke,fill}%
\end{pgfscope}%
\begin{pgfscope}%
\definecolor{textcolor}{rgb}{0.000000,0.000000,0.000000}%
\pgfsetstrokecolor{textcolor}%
\pgfsetfillcolor{textcolor}%
\pgftext[x=10.467022in,y=3.972585in,left,base]{\color{textcolor}{\rmfamily\fontsize{14.000000}{16.800000}\selectfont\catcode`\^=\active\def^{\ifmmode\sp\else\^{}\fi}\catcode`\%=\active\def%{\%}SolarPanel}}%
\end{pgfscope}%
\begin{pgfscope}%
\pgfsetbuttcap%
\pgfsetmiterjoin%
\definecolor{currentfill}{rgb}{0.090196,0.745098,0.811765}%
\pgfsetfillcolor{currentfill}%
\pgfsetlinewidth{1.003750pt}%
\definecolor{currentstroke}{rgb}{0.090196,0.745098,0.811765}%
\pgfsetstrokecolor{currentstroke}%
\pgfsetdash{}{0pt}%
\pgfpathmoveto{\pgfqpoint{9.922577in}{3.697585in}}%
\pgfpathlineto{\pgfqpoint{10.311466in}{3.697585in}}%
\pgfpathlineto{\pgfqpoint{10.311466in}{3.833696in}}%
\pgfpathlineto{\pgfqpoint{9.922577in}{3.833696in}}%
\pgfpathlineto{\pgfqpoint{9.922577in}{3.697585in}}%
\pgfpathclose%
\pgfusepath{stroke,fill}%
\end{pgfscope}%
\begin{pgfscope}%
\definecolor{textcolor}{rgb}{0.000000,0.000000,0.000000}%
\pgfsetstrokecolor{textcolor}%
\pgfsetfillcolor{textcolor}%
\pgftext[x=10.467022in,y=3.697585in,left,base]{\color{textcolor}{\rmfamily\fontsize{14.000000}{16.800000}\selectfont\catcode`\^=\active\def^{\ifmmode\sp\else\^{}\fi}\catcode`\%=\active\def%{\%}WindTurbine}}%
\end{pgfscope}%
\end{pgfpicture}%
\makeatother%
\endgroup%
}
  \caption{The design spaces for a) points on the Pareto-front in Figure
  \ref{fig:osier-temoa-benchmark}, b) selected points in \gls{osier}'s sub-optimal
  space, identified in Figure \ref{fig:osier-near-optimal}, and c) points
  generated by \gls{temoa}'s \gls{mga} algorithm shown in Figure
  \ref{fig:osier-temoa-benchmark}.}
  \label{fig:temoa-benchmark-03}
\end{figure}

\subsection{Reanalyzing the \gls{set}}

\begin{figure}[ht!]
    \begin{center}
        \resizebox{\columnwidth}{!}{%% Creator: Matplotlib, PGF backend
%%
%% To include the figure in your LaTeX document, write
%%   \input{<filename>.pgf}
%%
%% Make sure the required packages are loaded in your preamble
%%   \usepackage{pgf}
%%
%% Also ensure that all the required font packages are loaded; for instance,
%% the lmodern package is sometimes necessary when using math font.
%%   \usepackage{lmodern}
%%
%% Figures using additional raster images can only be included by \input if
%% they are in the same directory as the main LaTeX file. For loading figures
%% from other directories you can use the `import` package
%%   \usepackage{import}
%%
%% and then include the figures with
%%   \import{<path to file>}{<filename>.pgf}
%%
%% Matplotlib used the following preamble
%%   \def\mathdefault#1{#1}
%%   \everymath=\expandafter{\the\everymath\displaystyle}
%%   \IfFileExists{scrextend.sty}{
%%     \usepackage[fontsize=10.000000pt]{scrextend}
%%   }{
%%     \renewcommand{\normalsize}{\fontsize{10.000000}{12.000000}\selectfont}
%%     \normalsize
%%   }
%%   
%%   \makeatletter\@ifpackageloaded{underscore}{}{\usepackage[strings]{underscore}}\makeatother
%%
\begingroup%
\makeatletter%
\begin{pgfpicture}%
\pgfpathrectangle{\pgfpointorigin}{\pgfqpoint{12.375956in}{8.028674in}}%
\pgfusepath{use as bounding box, clip}%
\begin{pgfscope}%
\pgfsetbuttcap%
\pgfsetmiterjoin%
\definecolor{currentfill}{rgb}{1.000000,1.000000,1.000000}%
\pgfsetfillcolor{currentfill}%
\pgfsetlinewidth{0.000000pt}%
\definecolor{currentstroke}{rgb}{0.000000,0.000000,0.000000}%
\pgfsetstrokecolor{currentstroke}%
\pgfsetdash{}{0pt}%
\pgfpathmoveto{\pgfqpoint{0.000000in}{0.000000in}}%
\pgfpathlineto{\pgfqpoint{12.375956in}{0.000000in}}%
\pgfpathlineto{\pgfqpoint{12.375956in}{8.028674in}}%
\pgfpathlineto{\pgfqpoint{0.000000in}{8.028674in}}%
\pgfpathlineto{\pgfqpoint{0.000000in}{0.000000in}}%
\pgfpathclose%
\pgfusepath{fill}%
\end{pgfscope}%
\begin{pgfscope}%
\pgfsetbuttcap%
\pgfsetmiterjoin%
\definecolor{currentfill}{rgb}{1.000000,1.000000,1.000000}%
\pgfsetfillcolor{currentfill}%
\pgfsetlinewidth{0.000000pt}%
\definecolor{currentstroke}{rgb}{0.000000,0.000000,0.000000}%
\pgfsetstrokecolor{currentstroke}%
\pgfsetstrokeopacity{0.000000}%
\pgfsetdash{}{0pt}%
\pgfpathmoveto{\pgfqpoint{0.100000in}{2.802285in}}%
\pgfpathlineto{\pgfqpoint{12.275956in}{2.802285in}}%
\pgfpathlineto{\pgfqpoint{12.275956in}{7.311969in}}%
\pgfpathlineto{\pgfqpoint{0.100000in}{7.311969in}}%
\pgfpathlineto{\pgfqpoint{0.100000in}{2.802285in}}%
\pgfpathclose%
\pgfusepath{fill}%
\end{pgfscope}%
\begin{pgfscope}%
\pgfsetbuttcap%
\pgfsetroundjoin%
\definecolor{currentfill}{rgb}{0.000000,0.000000,0.000000}%
\pgfsetfillcolor{currentfill}%
\pgfsetlinewidth{0.803000pt}%
\definecolor{currentstroke}{rgb}{0.000000,0.000000,0.000000}%
\pgfsetstrokecolor{currentstroke}%
\pgfsetdash{}{0pt}%
\pgfsys@defobject{currentmarker}{\pgfqpoint{0.000000in}{-0.048611in}}{\pgfqpoint{0.000000in}{0.000000in}}{%
\pgfpathmoveto{\pgfqpoint{0.000000in}{0.000000in}}%
\pgfpathlineto{\pgfqpoint{0.000000in}{-0.048611in}}%
\pgfusepath{stroke,fill}%
}%
\begin{pgfscope}%
\pgfsys@transformshift{0.674129in}{2.802285in}%
\pgfsys@useobject{currentmarker}{}%
\end{pgfscope}%
\end{pgfscope}%
\begin{pgfscope}%
\definecolor{textcolor}{rgb}{0.000000,0.000000,0.000000}%
\pgfsetstrokecolor{textcolor}%
\pgfsetfillcolor{textcolor}%
\pgftext[x=0.724129in, y=0.562378in, left, base,rotate=90.000000]{\color{textcolor}{\rmfamily\fontsize{14.000000}{16.800000}\selectfont\catcode`\^=\active\def^{\ifmmode\sp\else\^{}\fi}\catcode`\%=\active\def%{\%}mass snf hlw disposal}}%
\end{pgfscope}%
\begin{pgfscope}%
\pgfsetbuttcap%
\pgfsetroundjoin%
\definecolor{currentfill}{rgb}{0.000000,0.000000,0.000000}%
\pgfsetfillcolor{currentfill}%
\pgfsetlinewidth{0.803000pt}%
\definecolor{currentstroke}{rgb}{0.000000,0.000000,0.000000}%
\pgfsetstrokecolor{currentstroke}%
\pgfsetdash{}{0pt}%
\pgfsys@defobject{currentmarker}{\pgfqpoint{0.000000in}{-0.048611in}}{\pgfqpoint{0.000000in}{0.000000in}}{%
\pgfpathmoveto{\pgfqpoint{0.000000in}{0.000000in}}%
\pgfpathlineto{\pgfqpoint{0.000000in}{-0.048611in}}%
\pgfusepath{stroke,fill}%
}%
\begin{pgfscope}%
\pgfsys@transformshift{1.363361in}{2.802285in}%
\pgfsys@useobject{currentmarker}{}%
\end{pgfscope}%
\end{pgfscope}%
\begin{pgfscope}%
\definecolor{textcolor}{rgb}{0.000000,0.000000,0.000000}%
\pgfsetstrokecolor{textcolor}%
\pgfsetfillcolor{textcolor}%
\pgftext[x=1.413361in, y=0.838717in, left, base,rotate=90.000000]{\color{textcolor}{\rmfamily\fontsize{14.000000}{16.800000}\selectfont\catcode`\^=\active\def^{\ifmmode\sp\else\^{}\fi}\catcode`\%=\active\def%{\%}activity at 100 yrs}}%
\end{pgfscope}%
\begin{pgfscope}%
\pgfsetbuttcap%
\pgfsetroundjoin%
\definecolor{currentfill}{rgb}{0.000000,0.000000,0.000000}%
\pgfsetfillcolor{currentfill}%
\pgfsetlinewidth{0.803000pt}%
\definecolor{currentstroke}{rgb}{0.000000,0.000000,0.000000}%
\pgfsetstrokecolor{currentstroke}%
\pgfsetdash{}{0pt}%
\pgfsys@defobject{currentmarker}{\pgfqpoint{0.000000in}{-0.048611in}}{\pgfqpoint{0.000000in}{0.000000in}}{%
\pgfpathmoveto{\pgfqpoint{0.000000in}{0.000000in}}%
\pgfpathlineto{\pgfqpoint{0.000000in}{-0.048611in}}%
\pgfusepath{stroke,fill}%
}%
\begin{pgfscope}%
\pgfsys@transformshift{2.052592in}{2.802285in}%
\pgfsys@useobject{currentmarker}{}%
\end{pgfscope}%
\end{pgfscope}%
\begin{pgfscope}%
\definecolor{textcolor}{rgb}{0.000000,0.000000,0.000000}%
\pgfsetstrokecolor{textcolor}%
\pgfsetfillcolor{textcolor}%
\pgftext[x=2.102592in, y=0.555851in, left, base,rotate=90.000000]{\color{textcolor}{\rmfamily\fontsize{14.000000}{16.800000}\selectfont\catcode`\^=\active\def^{\ifmmode\sp\else\^{}\fi}\catcode`\%=\active\def%{\%}activity at 100k years}}%
\end{pgfscope}%
\begin{pgfscope}%
\pgfsetbuttcap%
\pgfsetroundjoin%
\definecolor{currentfill}{rgb}{0.000000,0.000000,0.000000}%
\pgfsetfillcolor{currentfill}%
\pgfsetlinewidth{0.803000pt}%
\definecolor{currentstroke}{rgb}{0.000000,0.000000,0.000000}%
\pgfsetstrokecolor{currentstroke}%
\pgfsetdash{}{0pt}%
\pgfsys@defobject{currentmarker}{\pgfqpoint{0.000000in}{-0.048611in}}{\pgfqpoint{0.000000in}{0.000000in}}{%
\pgfpathmoveto{\pgfqpoint{0.000000in}{0.000000in}}%
\pgfpathlineto{\pgfqpoint{0.000000in}{-0.048611in}}%
\pgfusepath{stroke,fill}%
}%
\begin{pgfscope}%
\pgfsys@transformshift{2.741823in}{2.802285in}%
\pgfsys@useobject{currentmarker}{}%
\end{pgfscope}%
\end{pgfscope}%
\begin{pgfscope}%
\definecolor{textcolor}{rgb}{0.000000,0.000000,0.000000}%
\pgfsetstrokecolor{textcolor}%
\pgfsetfillcolor{textcolor}%
\pgftext[x=2.791823in, y=0.383954in, left, base,rotate=90.000000]{\color{textcolor}{\rmfamily\fontsize{14.000000}{16.800000}\selectfont\catcode`\^=\active\def^{\ifmmode\sp\else\^{}\fi}\catcode`\%=\active\def%{\%}mass du ru rth disposal}}%
\end{pgfscope}%
\begin{pgfscope}%
\pgfsetbuttcap%
\pgfsetroundjoin%
\definecolor{currentfill}{rgb}{0.000000,0.000000,0.000000}%
\pgfsetfillcolor{currentfill}%
\pgfsetlinewidth{0.803000pt}%
\definecolor{currentstroke}{rgb}{0.000000,0.000000,0.000000}%
\pgfsetstrokecolor{currentstroke}%
\pgfsetdash{}{0pt}%
\pgfsys@defobject{currentmarker}{\pgfqpoint{0.000000in}{-0.048611in}}{\pgfqpoint{0.000000in}{0.000000in}}{%
\pgfpathmoveto{\pgfqpoint{0.000000in}{0.000000in}}%
\pgfpathlineto{\pgfqpoint{0.000000in}{-0.048611in}}%
\pgfusepath{stroke,fill}%
}%
\begin{pgfscope}%
\pgfsys@transformshift{3.431054in}{2.802285in}%
\pgfsys@useobject{currentmarker}{}%
\end{pgfscope}%
\end{pgfscope}%
\begin{pgfscope}%
\definecolor{textcolor}{rgb}{0.000000,0.000000,0.000000}%
\pgfsetstrokecolor{textcolor}%
\pgfsetfillcolor{textcolor}%
\pgftext[x=3.481054in, y=0.734274in, left, base,rotate=90.000000]{\color{textcolor}{\rmfamily\fontsize{14.000000}{16.800000}\selectfont\catcode`\^=\active\def^{\ifmmode\sp\else\^{}\fi}\catcode`\%=\active\def%{\%}volume llw disposal}}%
\end{pgfscope}%
\begin{pgfscope}%
\pgfsetbuttcap%
\pgfsetroundjoin%
\definecolor{currentfill}{rgb}{0.000000,0.000000,0.000000}%
\pgfsetfillcolor{currentfill}%
\pgfsetlinewidth{0.803000pt}%
\definecolor{currentstroke}{rgb}{0.000000,0.000000,0.000000}%
\pgfsetstrokecolor{currentstroke}%
\pgfsetdash{}{0pt}%
\pgfsys@defobject{currentmarker}{\pgfqpoint{0.000000in}{-0.048611in}}{\pgfqpoint{0.000000in}{0.000000in}}{%
\pgfpathmoveto{\pgfqpoint{0.000000in}{0.000000in}}%
\pgfpathlineto{\pgfqpoint{0.000000in}{-0.048611in}}%
\pgfusepath{stroke,fill}%
}%
\begin{pgfscope}%
\pgfsys@transformshift{4.120285in}{2.802285in}%
\pgfsys@useobject{currentmarker}{}%
\end{pgfscope}%
\end{pgfscope}%
\begin{pgfscope}%
\definecolor{textcolor}{rgb}{0.000000,0.000000,0.000000}%
\pgfsetstrokecolor{textcolor}%
\pgfsetfillcolor{textcolor}%
\pgftext[x=4.170285in, y=0.989942in, left, base,rotate=90.000000]{\color{textcolor}{\rmfamily\fontsize{14.000000}{16.800000}\selectfont\catcode`\^=\active\def^{\ifmmode\sp\else\^{}\fi}\catcode`\%=\active\def%{\%}safety challenges}}%
\end{pgfscope}%
\begin{pgfscope}%
\pgfsetbuttcap%
\pgfsetroundjoin%
\definecolor{currentfill}{rgb}{0.000000,0.000000,0.000000}%
\pgfsetfillcolor{currentfill}%
\pgfsetlinewidth{0.803000pt}%
\definecolor{currentstroke}{rgb}{0.000000,0.000000,0.000000}%
\pgfsetstrokecolor{currentstroke}%
\pgfsetdash{}{0pt}%
\pgfsys@defobject{currentmarker}{\pgfqpoint{0.000000in}{-0.048611in}}{\pgfqpoint{0.000000in}{0.000000in}}{%
\pgfpathmoveto{\pgfqpoint{0.000000in}{0.000000in}}%
\pgfpathlineto{\pgfqpoint{0.000000in}{-0.048611in}}%
\pgfusepath{stroke,fill}%
}%
\begin{pgfscope}%
\pgfsys@transformshift{4.809516in}{2.802285in}%
\pgfsys@useobject{currentmarker}{}%
\end{pgfscope}%
\end{pgfscope}%
\begin{pgfscope}%
\definecolor{textcolor}{rgb}{0.000000,0.000000,0.000000}%
\pgfsetstrokecolor{textcolor}%
\pgfsetfillcolor{textcolor}%
\pgftext[x=4.859516in, y=1.698197in, left, base,rotate=90.000000]{\color{textcolor}{\rmfamily\fontsize{14.000000}{16.800000}\selectfont\catcode`\^=\active\def^{\ifmmode\sp\else\^{}\fi}\catcode`\%=\active\def%{\%}land use}}%
\end{pgfscope}%
\begin{pgfscope}%
\pgfsetbuttcap%
\pgfsetroundjoin%
\definecolor{currentfill}{rgb}{0.000000,0.000000,0.000000}%
\pgfsetfillcolor{currentfill}%
\pgfsetlinewidth{0.803000pt}%
\definecolor{currentstroke}{rgb}{0.000000,0.000000,0.000000}%
\pgfsetstrokecolor{currentstroke}%
\pgfsetdash{}{0pt}%
\pgfsys@defobject{currentmarker}{\pgfqpoint{0.000000in}{-0.048611in}}{\pgfqpoint{0.000000in}{0.000000in}}{%
\pgfpathmoveto{\pgfqpoint{0.000000in}{0.000000in}}%
\pgfpathlineto{\pgfqpoint{0.000000in}{-0.048611in}}%
\pgfusepath{stroke,fill}%
}%
\begin{pgfscope}%
\pgfsys@transformshift{5.498747in}{2.802285in}%
\pgfsys@useobject{currentmarker}{}%
\end{pgfscope}%
\end{pgfscope}%
\begin{pgfscope}%
\definecolor{textcolor}{rgb}{0.000000,0.000000,0.000000}%
\pgfsetstrokecolor{textcolor}%
\pgfsetfillcolor{textcolor}%
\pgftext[x=5.548747in, y=1.594842in, left, base,rotate=90.000000]{\color{textcolor}{\rmfamily\fontsize{14.000000}{16.800000}\selectfont\catcode`\^=\active\def^{\ifmmode\sp\else\^{}\fi}\catcode`\%=\active\def%{\%}water use}}%
\end{pgfscope}%
\begin{pgfscope}%
\pgfsetbuttcap%
\pgfsetroundjoin%
\definecolor{currentfill}{rgb}{0.000000,0.000000,0.000000}%
\pgfsetfillcolor{currentfill}%
\pgfsetlinewidth{0.803000pt}%
\definecolor{currentstroke}{rgb}{0.000000,0.000000,0.000000}%
\pgfsetstrokecolor{currentstroke}%
\pgfsetdash{}{0pt}%
\pgfsys@defobject{currentmarker}{\pgfqpoint{0.000000in}{-0.048611in}}{\pgfqpoint{0.000000in}{0.000000in}}{%
\pgfpathmoveto{\pgfqpoint{0.000000in}{0.000000in}}%
\pgfpathlineto{\pgfqpoint{0.000000in}{-0.048611in}}%
\pgfusepath{stroke,fill}%
}%
\begin{pgfscope}%
\pgfsys@transformshift{6.187978in}{2.802285in}%
\pgfsys@useobject{currentmarker}{}%
\end{pgfscope}%
\end{pgfscope}%
\begin{pgfscope}%
\definecolor{textcolor}{rgb}{0.000000,0.000000,0.000000}%
\pgfsetstrokecolor{textcolor}%
\pgfsetfillcolor{textcolor}%
\pgftext[x=6.237978in, y=0.287127in, left, base,rotate=90.000000]{\color{textcolor}{\rmfamily\fontsize{14.000000}{16.800000}\selectfont\catcode`\^=\active\def^{\ifmmode\sp\else\^{}\fi}\catcode`\%=\active\def%{\%}carbon dioxide emissions}}%
\end{pgfscope}%
\begin{pgfscope}%
\pgfsetbuttcap%
\pgfsetroundjoin%
\definecolor{currentfill}{rgb}{0.000000,0.000000,0.000000}%
\pgfsetfillcolor{currentfill}%
\pgfsetlinewidth{0.803000pt}%
\definecolor{currentstroke}{rgb}{0.000000,0.000000,0.000000}%
\pgfsetstrokecolor{currentstroke}%
\pgfsetdash{}{0pt}%
\pgfsys@defobject{currentmarker}{\pgfqpoint{0.000000in}{-0.048611in}}{\pgfqpoint{0.000000in}{0.000000in}}{%
\pgfpathmoveto{\pgfqpoint{0.000000in}{0.000000in}}%
\pgfpathlineto{\pgfqpoint{0.000000in}{-0.048611in}}%
\pgfusepath{stroke,fill}%
}%
\begin{pgfscope}%
\pgfsys@transformshift{6.877209in}{2.802285in}%
\pgfsys@useobject{currentmarker}{}%
\end{pgfscope}%
\end{pgfscope}%
\begin{pgfscope}%
\definecolor{textcolor}{rgb}{0.000000,0.000000,0.000000}%
\pgfsetstrokecolor{textcolor}%
\pgfsetfillcolor{textcolor}%
\pgftext[x=6.927209in, y=0.931193in, left, base,rotate=90.000000]{\color{textcolor}{\rmfamily\fontsize{14.000000}{16.800000}\selectfont\catcode`\^=\active\def^{\ifmmode\sp\else\^{}\fi}\catcode`\%=\active\def%{\%}total worker dose}}%
\end{pgfscope}%
\begin{pgfscope}%
\pgfsetbuttcap%
\pgfsetroundjoin%
\definecolor{currentfill}{rgb}{0.000000,0.000000,0.000000}%
\pgfsetfillcolor{currentfill}%
\pgfsetlinewidth{0.803000pt}%
\definecolor{currentstroke}{rgb}{0.000000,0.000000,0.000000}%
\pgfsetstrokecolor{currentstroke}%
\pgfsetdash{}{0pt}%
\pgfsys@defobject{currentmarker}{\pgfqpoint{0.000000in}{-0.048611in}}{\pgfqpoint{0.000000in}{0.000000in}}{%
\pgfpathmoveto{\pgfqpoint{0.000000in}{0.000000in}}%
\pgfpathlineto{\pgfqpoint{0.000000in}{-0.048611in}}%
\pgfusepath{stroke,fill}%
}%
\begin{pgfscope}%
\pgfsys@transformshift{7.566440in}{2.802285in}%
\pgfsys@useobject{currentmarker}{}%
\end{pgfscope}%
\end{pgfscope}%
\begin{pgfscope}%
\definecolor{textcolor}{rgb}{0.000000,0.000000,0.000000}%
\pgfsetstrokecolor{textcolor}%
\pgfsetfillcolor{textcolor}%
\pgftext[x=7.616440in, y=0.235993in, left, base,rotate=90.000000]{\color{textcolor}{\rmfamily\fontsize{14.000000}{16.800000}\selectfont\catcode`\^=\active\def^{\ifmmode\sp\else\^{}\fi}\catcode`\%=\active\def%{\%}natural uranium required}}%
\end{pgfscope}%
\begin{pgfscope}%
\pgfsetbuttcap%
\pgfsetroundjoin%
\definecolor{currentfill}{rgb}{0.000000,0.000000,0.000000}%
\pgfsetfillcolor{currentfill}%
\pgfsetlinewidth{0.803000pt}%
\definecolor{currentstroke}{rgb}{0.000000,0.000000,0.000000}%
\pgfsetstrokecolor{currentstroke}%
\pgfsetdash{}{0pt}%
\pgfsys@defobject{currentmarker}{\pgfqpoint{0.000000in}{-0.048611in}}{\pgfqpoint{0.000000in}{0.000000in}}{%
\pgfpathmoveto{\pgfqpoint{0.000000in}{0.000000in}}%
\pgfpathlineto{\pgfqpoint{0.000000in}{-0.048611in}}%
\pgfusepath{stroke,fill}%
}%
\begin{pgfscope}%
\pgfsys@transformshift{8.255671in}{2.802285in}%
\pgfsys@useobject{currentmarker}{}%
\end{pgfscope}%
\end{pgfscope}%
\begin{pgfscope}%
\definecolor{textcolor}{rgb}{0.000000,0.000000,0.000000}%
\pgfsetstrokecolor{textcolor}%
\pgfsetfillcolor{textcolor}%
\pgftext[x=8.305671in, y=0.100000in, left, base,rotate=90.000000]{\color{textcolor}{\rmfamily\fontsize{14.000000}{16.800000}\selectfont\catcode`\^=\active\def^{\ifmmode\sp\else\^{}\fi}\catcode`\%=\active\def%{\%}natural thorium utilization}}%
\end{pgfscope}%
\begin{pgfscope}%
\pgfsetbuttcap%
\pgfsetroundjoin%
\definecolor{currentfill}{rgb}{0.000000,0.000000,0.000000}%
\pgfsetfillcolor{currentfill}%
\pgfsetlinewidth{0.803000pt}%
\definecolor{currentstroke}{rgb}{0.000000,0.000000,0.000000}%
\pgfsetstrokecolor{currentstroke}%
\pgfsetdash{}{0pt}%
\pgfsys@defobject{currentmarker}{\pgfqpoint{0.000000in}{-0.048611in}}{\pgfqpoint{0.000000in}{0.000000in}}{%
\pgfpathmoveto{\pgfqpoint{0.000000in}{0.000000in}}%
\pgfpathlineto{\pgfqpoint{0.000000in}{-0.048611in}}%
\pgfusepath{stroke,fill}%
}%
\begin{pgfscope}%
\pgfsys@transformshift{8.944902in}{2.802285in}%
\pgfsys@useobject{currentmarker}{}%
\end{pgfscope}%
\end{pgfscope}%
\begin{pgfscope}%
\definecolor{textcolor}{rgb}{0.000000,0.000000,0.000000}%
\pgfsetstrokecolor{textcolor}%
\pgfsetfillcolor{textcolor}%
\pgftext[x=8.994902in, y=0.931193in, left, base,rotate=90.000000]{\color{textcolor}{\rmfamily\fontsize{14.000000}{16.800000}\selectfont\catcode`\^=\active\def^{\ifmmode\sp\else\^{}\fi}\catcode`\%=\active\def%{\%}development cost}}%
\end{pgfscope}%
\begin{pgfscope}%
\pgfsetbuttcap%
\pgfsetroundjoin%
\definecolor{currentfill}{rgb}{0.000000,0.000000,0.000000}%
\pgfsetfillcolor{currentfill}%
\pgfsetlinewidth{0.803000pt}%
\definecolor{currentstroke}{rgb}{0.000000,0.000000,0.000000}%
\pgfsetstrokecolor{currentstroke}%
\pgfsetdash{}{0pt}%
\pgfsys@defobject{currentmarker}{\pgfqpoint{0.000000in}{-0.048611in}}{\pgfqpoint{0.000000in}{0.000000in}}{%
\pgfpathmoveto{\pgfqpoint{0.000000in}{0.000000in}}%
\pgfpathlineto{\pgfqpoint{0.000000in}{-0.048611in}}%
\pgfusepath{stroke,fill}%
}%
\begin{pgfscope}%
\pgfsys@transformshift{9.634134in}{2.802285in}%
\pgfsys@useobject{currentmarker}{}%
\end{pgfscope}%
\end{pgfscope}%
\begin{pgfscope}%
\definecolor{textcolor}{rgb}{0.000000,0.000000,0.000000}%
\pgfsetstrokecolor{textcolor}%
\pgfsetfillcolor{textcolor}%
\pgftext[x=9.684134in, y=0.888763in, left, base,rotate=90.000000]{\color{textcolor}{\rmfamily\fontsize{14.000000}{16.800000}\selectfont\catcode`\^=\active\def^{\ifmmode\sp\else\^{}\fi}\catcode`\%=\active\def%{\%}development time}}%
\end{pgfscope}%
\begin{pgfscope}%
\pgfsetbuttcap%
\pgfsetroundjoin%
\definecolor{currentfill}{rgb}{0.000000,0.000000,0.000000}%
\pgfsetfillcolor{currentfill}%
\pgfsetlinewidth{0.803000pt}%
\definecolor{currentstroke}{rgb}{0.000000,0.000000,0.000000}%
\pgfsetstrokecolor{currentstroke}%
\pgfsetdash{}{0pt}%
\pgfsys@defobject{currentmarker}{\pgfqpoint{0.000000in}{-0.048611in}}{\pgfqpoint{0.000000in}{0.000000in}}{%
\pgfpathmoveto{\pgfqpoint{0.000000in}{0.000000in}}%
\pgfpathlineto{\pgfqpoint{0.000000in}{-0.048611in}}%
\pgfusepath{stroke,fill}%
}%
\begin{pgfscope}%
\pgfsys@transformshift{10.323365in}{2.802285in}%
\pgfsys@useobject{currentmarker}{}%
\end{pgfscope}%
\end{pgfscope}%
\begin{pgfscope}%
\definecolor{textcolor}{rgb}{0.000000,0.000000,0.000000}%
\pgfsetstrokecolor{textcolor}%
\pgfsetfillcolor{textcolor}%
\pgftext[x=10.373365in, y=1.643800in, left, base,rotate=90.000000]{\color{textcolor}{\rmfamily\fontsize{14.000000}{16.800000}\selectfont\catcode`\^=\active\def^{\ifmmode\sp\else\^{}\fi}\catcode`\%=\active\def%{\%}foak cost}}%
\end{pgfscope}%
\begin{pgfscope}%
\pgfsetbuttcap%
\pgfsetroundjoin%
\definecolor{currentfill}{rgb}{0.000000,0.000000,0.000000}%
\pgfsetfillcolor{currentfill}%
\pgfsetlinewidth{0.803000pt}%
\definecolor{currentstroke}{rgb}{0.000000,0.000000,0.000000}%
\pgfsetstrokecolor{currentstroke}%
\pgfsetdash{}{0pt}%
\pgfsys@defobject{currentmarker}{\pgfqpoint{0.000000in}{-0.048611in}}{\pgfqpoint{0.000000in}{0.000000in}}{%
\pgfpathmoveto{\pgfqpoint{0.000000in}{0.000000in}}%
\pgfpathlineto{\pgfqpoint{0.000000in}{-0.048611in}}%
\pgfusepath{stroke,fill}%
}%
\begin{pgfscope}%
\pgfsys@transformshift{11.012596in}{2.802285in}%
\pgfsys@useobject{currentmarker}{}%
\end{pgfscope}%
\end{pgfscope}%
\begin{pgfscope}%
\definecolor{textcolor}{rgb}{0.000000,0.000000,0.000000}%
\pgfsetstrokecolor{textcolor}%
\pgfsetfillcolor{textcolor}%
\pgftext[x=11.062596in, y=1.111792in, left, base,rotate=90.000000]{\color{textcolor}{\rmfamily\fontsize{14.000000}{16.800000}\selectfont\catcode`\^=\active\def^{\ifmmode\sp\else\^{}\fi}\catcode`\%=\active\def%{\%}incompatibility}}%
\end{pgfscope}%
\begin{pgfscope}%
\pgfsetbuttcap%
\pgfsetroundjoin%
\definecolor{currentfill}{rgb}{0.000000,0.000000,0.000000}%
\pgfsetfillcolor{currentfill}%
\pgfsetlinewidth{0.803000pt}%
\definecolor{currentstroke}{rgb}{0.000000,0.000000,0.000000}%
\pgfsetstrokecolor{currentstroke}%
\pgfsetdash{}{0pt}%
\pgfsys@defobject{currentmarker}{\pgfqpoint{0.000000in}{-0.048611in}}{\pgfqpoint{0.000000in}{0.000000in}}{%
\pgfpathmoveto{\pgfqpoint{0.000000in}{0.000000in}}%
\pgfpathlineto{\pgfqpoint{0.000000in}{-0.048611in}}%
\pgfusepath{stroke,fill}%
}%
\begin{pgfscope}%
\pgfsys@transformshift{11.701827in}{2.802285in}%
\pgfsys@useobject{currentmarker}{}%
\end{pgfscope}%
\end{pgfscope}%
\begin{pgfscope}%
\definecolor{textcolor}{rgb}{0.000000,0.000000,0.000000}%
\pgfsetstrokecolor{textcolor}%
\pgfsetfillcolor{textcolor}%
\pgftext[x=11.751827in, y=1.302184in, left, base,rotate=90.000000]{\color{textcolor}{\rmfamily\fontsize{14.000000}{16.800000}\selectfont\catcode`\^=\active\def^{\ifmmode\sp\else\^{}\fi}\catcode`\%=\active\def%{\%}unfamiliarity}}%
\end{pgfscope}%
\begin{pgfscope}%
\pgfpathrectangle{\pgfqpoint{0.100000in}{2.802285in}}{\pgfqpoint{12.175956in}{4.509684in}}%
\pgfusepath{clip}%
\pgfsetrectcap%
\pgfsetroundjoin%
\pgfsetlinewidth{1.505625pt}%
\definecolor{currentstroke}{rgb}{0.839216,0.152941,0.156863}%
\pgfsetstrokecolor{currentstroke}%
\pgfsetdash{}{0pt}%
\pgfpathmoveto{\pgfqpoint{0.674129in}{3.586420in}}%
\pgfpathlineto{\pgfqpoint{1.363361in}{5.280169in}}%
\pgfpathlineto{\pgfqpoint{2.052592in}{3.378777in}}%
\pgfpathlineto{\pgfqpoint{2.741823in}{5.311976in}}%
\pgfpathlineto{\pgfqpoint{3.431054in}{3.200433in}}%
\pgfpathlineto{\pgfqpoint{4.120285in}{3.007271in}}%
\pgfpathlineto{\pgfqpoint{4.809516in}{5.328795in}}%
\pgfpathlineto{\pgfqpoint{5.498747in}{3.083619in}}%
\pgfpathlineto{\pgfqpoint{6.187978in}{4.040642in}}%
\pgfpathlineto{\pgfqpoint{6.877209in}{3.954683in}}%
\pgfpathlineto{\pgfqpoint{7.566440in}{5.536721in}}%
\pgfpathlineto{\pgfqpoint{8.255671in}{3.007271in}}%
\pgfpathlineto{\pgfqpoint{8.944902in}{3.007271in}}%
\pgfpathlineto{\pgfqpoint{9.634134in}{3.007271in}}%
\pgfpathlineto{\pgfqpoint{10.323365in}{3.007271in}}%
\pgfpathlineto{\pgfqpoint{11.012596in}{3.007271in}}%
\pgfpathlineto{\pgfqpoint{11.701827in}{3.007271in}}%
\pgfusepath{stroke}%
\end{pgfscope}%
\begin{pgfscope}%
\pgfpathrectangle{\pgfqpoint{0.100000in}{2.802285in}}{\pgfqpoint{12.175956in}{4.509684in}}%
\pgfusepath{clip}%
\pgfsetrectcap%
\pgfsetroundjoin%
\pgfsetlinewidth{1.505625pt}%
\definecolor{currentstroke}{rgb}{0.839216,0.152941,0.156863}%
\pgfsetstrokecolor{currentstroke}%
\pgfsetdash{}{0pt}%
\pgfpathmoveto{\pgfqpoint{0.674129in}{3.230581in}}%
\pgfpathlineto{\pgfqpoint{1.363361in}{5.662525in}}%
\pgfpathlineto{\pgfqpoint{2.052592in}{3.510168in}}%
\pgfpathlineto{\pgfqpoint{2.741823in}{7.106984in}}%
\pgfpathlineto{\pgfqpoint{3.431054in}{3.225156in}}%
\pgfpathlineto{\pgfqpoint{4.120285in}{3.007271in}}%
\pgfpathlineto{\pgfqpoint{4.809516in}{6.193192in}}%
\pgfpathlineto{\pgfqpoint{5.498747in}{3.126127in}}%
\pgfpathlineto{\pgfqpoint{6.187978in}{4.405361in}}%
\pgfpathlineto{\pgfqpoint{6.877209in}{4.264746in}}%
\pgfpathlineto{\pgfqpoint{7.566440in}{7.106984in}}%
\pgfpathlineto{\pgfqpoint{8.255671in}{3.007271in}}%
\pgfpathlineto{\pgfqpoint{8.944902in}{3.827213in}}%
\pgfpathlineto{\pgfqpoint{9.634134in}{4.373842in}}%
\pgfpathlineto{\pgfqpoint{10.323365in}{4.032199in}}%
\pgfpathlineto{\pgfqpoint{11.012596in}{3.007271in}}%
\pgfpathlineto{\pgfqpoint{11.701827in}{3.007271in}}%
\pgfusepath{stroke}%
\end{pgfscope}%
\begin{pgfscope}%
\pgfpathrectangle{\pgfqpoint{0.100000in}{2.802285in}}{\pgfqpoint{12.175956in}{4.509684in}}%
\pgfusepath{clip}%
\pgfsetrectcap%
\pgfsetroundjoin%
\pgfsetlinewidth{1.505625pt}%
\definecolor{currentstroke}{rgb}{0.839216,0.152941,0.156863}%
\pgfsetstrokecolor{currentstroke}%
\pgfsetdash{}{0pt}%
\pgfpathmoveto{\pgfqpoint{0.674129in}{7.106984in}}%
\pgfpathlineto{\pgfqpoint{1.363361in}{5.280169in}}%
\pgfpathlineto{\pgfqpoint{2.052592in}{3.644843in}}%
\pgfpathlineto{\pgfqpoint{2.741823in}{3.007271in}}%
\pgfpathlineto{\pgfqpoint{3.431054in}{3.158922in}}%
\pgfpathlineto{\pgfqpoint{4.120285in}{3.007271in}}%
\pgfpathlineto{\pgfqpoint{4.809516in}{7.106984in}}%
\pgfpathlineto{\pgfqpoint{5.498747in}{3.097238in}}%
\pgfpathlineto{\pgfqpoint{6.187978in}{5.492764in}}%
\pgfpathlineto{\pgfqpoint{6.877209in}{6.211248in}}%
\pgfpathlineto{\pgfqpoint{7.566440in}{4.990146in}}%
\pgfpathlineto{\pgfqpoint{8.255671in}{3.007271in}}%
\pgfpathlineto{\pgfqpoint{8.944902in}{3.007271in}}%
\pgfpathlineto{\pgfqpoint{9.634134in}{3.007271in}}%
\pgfpathlineto{\pgfqpoint{10.323365in}{3.007271in}}%
\pgfpathlineto{\pgfqpoint{11.012596in}{4.373842in}}%
\pgfpathlineto{\pgfqpoint{11.701827in}{4.373842in}}%
\pgfusepath{stroke}%
\end{pgfscope}%
\begin{pgfscope}%
\pgfpathrectangle{\pgfqpoint{0.100000in}{2.802285in}}{\pgfqpoint{12.175956in}{4.509684in}}%
\pgfusepath{clip}%
\pgfsetrectcap%
\pgfsetroundjoin%
\pgfsetlinewidth{1.505625pt}%
\definecolor{currentstroke}{rgb}{0.839216,0.152941,0.156863}%
\pgfsetstrokecolor{currentstroke}%
\pgfsetdash{}{0pt}%
\pgfpathmoveto{\pgfqpoint{0.674129in}{3.084042in}}%
\pgfpathlineto{\pgfqpoint{1.363361in}{3.007271in}}%
\pgfpathlineto{\pgfqpoint{2.052592in}{3.375493in}}%
\pgfpathlineto{\pgfqpoint{2.741823in}{3.007271in}}%
\pgfpathlineto{\pgfqpoint{3.431054in}{3.007271in}}%
\pgfpathlineto{\pgfqpoint{4.120285in}{3.007271in}}%
\pgfpathlineto{\pgfqpoint{4.809516in}{3.031968in}}%
\pgfpathlineto{\pgfqpoint{5.498747in}{3.007271in}}%
\pgfpathlineto{\pgfqpoint{6.187978in}{3.007271in}}%
\pgfpathlineto{\pgfqpoint{6.877209in}{4.161392in}}%
\pgfpathlineto{\pgfqpoint{7.566440in}{3.060909in}}%
\pgfpathlineto{\pgfqpoint{8.255671in}{3.007271in}}%
\pgfpathlineto{\pgfqpoint{8.944902in}{4.647156in}}%
\pgfpathlineto{\pgfqpoint{9.634134in}{5.740413in}}%
\pgfpathlineto{\pgfqpoint{10.323365in}{5.057127in}}%
\pgfpathlineto{\pgfqpoint{11.012596in}{5.740413in}}%
\pgfpathlineto{\pgfqpoint{11.701827in}{5.740413in}}%
\pgfusepath{stroke}%
\end{pgfscope}%
\begin{pgfscope}%
\pgfpathrectangle{\pgfqpoint{0.100000in}{2.802285in}}{\pgfqpoint{12.175956in}{4.509684in}}%
\pgfusepath{clip}%
\pgfsetrectcap%
\pgfsetroundjoin%
\pgfsetlinewidth{1.505625pt}%
\definecolor{currentstroke}{rgb}{0.839216,0.152941,0.156863}%
\pgfsetstrokecolor{currentstroke}%
\pgfsetdash{}{0pt}%
\pgfpathmoveto{\pgfqpoint{0.674129in}{3.291942in}}%
\pgfpathlineto{\pgfqpoint{1.363361in}{5.492589in}}%
\pgfpathlineto{\pgfqpoint{2.052592in}{5.260947in}}%
\pgfpathlineto{\pgfqpoint{2.741823in}{6.912424in}}%
\pgfpathlineto{\pgfqpoint{3.431054in}{3.222506in}}%
\pgfpathlineto{\pgfqpoint{4.120285in}{3.007271in}}%
\pgfpathlineto{\pgfqpoint{4.809516in}{6.094404in}}%
\pgfpathlineto{\pgfqpoint{5.498747in}{3.120762in}}%
\pgfpathlineto{\pgfqpoint{6.187978in}{4.371590in}}%
\pgfpathlineto{\pgfqpoint{6.877209in}{4.247520in}}%
\pgfpathlineto{\pgfqpoint{7.566440in}{6.885323in}}%
\pgfpathlineto{\pgfqpoint{8.255671in}{4.762673in}}%
\pgfpathlineto{\pgfqpoint{8.944902in}{3.827213in}}%
\pgfpathlineto{\pgfqpoint{9.634134in}{4.373842in}}%
\pgfpathlineto{\pgfqpoint{10.323365in}{4.032199in}}%
\pgfpathlineto{\pgfqpoint{11.012596in}{7.106984in}}%
\pgfpathlineto{\pgfqpoint{11.701827in}{7.106984in}}%
\pgfusepath{stroke}%
\end{pgfscope}%
\begin{pgfscope}%
\pgfpathrectangle{\pgfqpoint{0.100000in}{2.802285in}}{\pgfqpoint{12.175956in}{4.509684in}}%
\pgfusepath{clip}%
\pgfsetrectcap%
\pgfsetroundjoin%
\pgfsetlinewidth{1.505625pt}%
\definecolor{currentstroke}{rgb}{0.839216,0.152941,0.156863}%
\pgfsetstrokecolor{currentstroke}%
\pgfsetdash{}{0pt}%
\pgfpathmoveto{\pgfqpoint{0.674129in}{3.248513in}}%
\pgfpathlineto{\pgfqpoint{1.363361in}{7.106984in}}%
\pgfpathlineto{\pgfqpoint{2.052592in}{7.106984in}}%
\pgfpathlineto{\pgfqpoint{2.741823in}{3.007271in}}%
\pgfpathlineto{\pgfqpoint{3.431054in}{3.901625in}}%
\pgfpathlineto{\pgfqpoint{4.120285in}{7.106984in}}%
\pgfpathlineto{\pgfqpoint{4.809516in}{3.945759in}}%
\pgfpathlineto{\pgfqpoint{5.498747in}{6.144579in}}%
\pgfpathlineto{\pgfqpoint{6.187978in}{4.547196in}}%
\pgfpathlineto{\pgfqpoint{6.877209in}{6.900275in}}%
\pgfpathlineto{\pgfqpoint{7.566440in}{3.007271in}}%
\pgfpathlineto{\pgfqpoint{8.255671in}{6.737028in}}%
\pgfpathlineto{\pgfqpoint{8.944902in}{5.467099in}}%
\pgfpathlineto{\pgfqpoint{9.634134in}{5.740413in}}%
\pgfpathlineto{\pgfqpoint{10.323365in}{6.082055in}}%
\pgfpathlineto{\pgfqpoint{11.012596in}{5.740413in}}%
\pgfpathlineto{\pgfqpoint{11.701827in}{7.106984in}}%
\pgfusepath{stroke}%
\end{pgfscope}%
\begin{pgfscope}%
\pgfpathrectangle{\pgfqpoint{0.100000in}{2.802285in}}{\pgfqpoint{12.175956in}{4.509684in}}%
\pgfusepath{clip}%
\pgfsetrectcap%
\pgfsetroundjoin%
\pgfsetlinewidth{1.505625pt}%
\definecolor{currentstroke}{rgb}{0.839216,0.152941,0.156863}%
\pgfsetstrokecolor{currentstroke}%
\pgfsetdash{}{0pt}%
\pgfpathmoveto{\pgfqpoint{0.674129in}{3.017638in}}%
\pgfpathlineto{\pgfqpoint{1.363361in}{3.449105in}}%
\pgfpathlineto{\pgfqpoint{2.052592in}{3.391917in}}%
\pgfpathlineto{\pgfqpoint{2.741823in}{3.007271in}}%
\pgfpathlineto{\pgfqpoint{3.431054in}{3.887408in}}%
\pgfpathlineto{\pgfqpoint{4.120285in}{7.106984in}}%
\pgfpathlineto{\pgfqpoint{4.809516in}{3.698789in}}%
\pgfpathlineto{\pgfqpoint{5.498747in}{7.106984in}}%
\pgfpathlineto{\pgfqpoint{6.187978in}{3.409137in}}%
\pgfpathlineto{\pgfqpoint{6.877209in}{7.106984in}}%
\pgfpathlineto{\pgfqpoint{7.566440in}{3.007271in}}%
\pgfpathlineto{\pgfqpoint{8.255671in}{3.618830in}}%
\pgfpathlineto{\pgfqpoint{8.944902in}{5.467099in}}%
\pgfpathlineto{\pgfqpoint{9.634134in}{5.740413in}}%
\pgfpathlineto{\pgfqpoint{10.323365in}{6.082055in}}%
\pgfpathlineto{\pgfqpoint{11.012596in}{5.740413in}}%
\pgfpathlineto{\pgfqpoint{11.701827in}{7.106984in}}%
\pgfusepath{stroke}%
\end{pgfscope}%
\begin{pgfscope}%
\pgfpathrectangle{\pgfqpoint{0.100000in}{2.802285in}}{\pgfqpoint{12.175956in}{4.509684in}}%
\pgfusepath{clip}%
\pgfsetrectcap%
\pgfsetroundjoin%
\pgfsetlinewidth{1.505625pt}%
\definecolor{currentstroke}{rgb}{0.172549,0.627451,0.172549}%
\pgfsetstrokecolor{currentstroke}%
\pgfsetdash{}{0pt}%
\pgfpathmoveto{\pgfqpoint{0.674129in}{3.035009in}}%
\pgfpathlineto{\pgfqpoint{1.363361in}{3.334398in}}%
\pgfpathlineto{\pgfqpoint{2.052592in}{3.146216in}}%
\pgfpathlineto{\pgfqpoint{2.741823in}{3.007271in}}%
\pgfpathlineto{\pgfqpoint{3.431054in}{3.136431in}}%
\pgfpathlineto{\pgfqpoint{4.120285in}{3.007271in}}%
\pgfpathlineto{\pgfqpoint{4.809516in}{3.007271in}}%
\pgfpathlineto{\pgfqpoint{5.498747in}{3.008509in}}%
\pgfpathlineto{\pgfqpoint{6.187978in}{3.142352in}}%
\pgfpathlineto{\pgfqpoint{6.877209in}{4.144166in}}%
\pgfpathlineto{\pgfqpoint{7.566440in}{3.037442in}}%
\pgfpathlineto{\pgfqpoint{8.255671in}{3.007271in}}%
\pgfpathlineto{\pgfqpoint{8.944902in}{5.467099in}}%
\pgfpathlineto{\pgfqpoint{9.634134in}{5.740413in}}%
\pgfpathlineto{\pgfqpoint{10.323365in}{7.106984in}}%
\pgfpathlineto{\pgfqpoint{11.012596in}{5.740413in}}%
\pgfpathlineto{\pgfqpoint{11.701827in}{7.106984in}}%
\pgfusepath{stroke}%
\end{pgfscope}%
\begin{pgfscope}%
\pgfpathrectangle{\pgfqpoint{0.100000in}{2.802285in}}{\pgfqpoint{12.175956in}{4.509684in}}%
\pgfusepath{clip}%
\pgfsetrectcap%
\pgfsetroundjoin%
\pgfsetlinewidth{1.505625pt}%
\definecolor{currentstroke}{rgb}{0.172549,0.627451,0.172549}%
\pgfsetstrokecolor{currentstroke}%
\pgfsetdash{}{0pt}%
\pgfpathmoveto{\pgfqpoint{0.674129in}{3.275971in}}%
\pgfpathlineto{\pgfqpoint{1.363361in}{5.620041in}}%
\pgfpathlineto{\pgfqpoint{2.052592in}{6.384336in}}%
\pgfpathlineto{\pgfqpoint{2.741823in}{3.007271in}}%
\pgfpathlineto{\pgfqpoint{3.431054in}{7.052508in}}%
\pgfpathlineto{\pgfqpoint{4.120285in}{3.007271in}}%
\pgfpathlineto{\pgfqpoint{4.809516in}{3.427121in}}%
\pgfpathlineto{\pgfqpoint{5.498747in}{3.032445in}}%
\pgfpathlineto{\pgfqpoint{6.187978in}{5.236110in}}%
\pgfpathlineto{\pgfqpoint{6.877209in}{3.110625in}}%
\pgfpathlineto{\pgfqpoint{7.566440in}{3.007271in}}%
\pgfpathlineto{\pgfqpoint{8.255671in}{7.106984in}}%
\pgfpathlineto{\pgfqpoint{8.944902in}{5.467099in}}%
\pgfpathlineto{\pgfqpoint{9.634134in}{5.740413in}}%
\pgfpathlineto{\pgfqpoint{10.323365in}{7.106984in}}%
\pgfpathlineto{\pgfqpoint{11.012596in}{5.740413in}}%
\pgfpathlineto{\pgfqpoint{11.701827in}{7.106984in}}%
\pgfusepath{stroke}%
\end{pgfscope}%
\begin{pgfscope}%
\pgfpathrectangle{\pgfqpoint{0.100000in}{2.802285in}}{\pgfqpoint{12.175956in}{4.509684in}}%
\pgfusepath{clip}%
\pgfsetrectcap%
\pgfsetroundjoin%
\pgfsetlinewidth{1.505625pt}%
\definecolor{currentstroke}{rgb}{0.172549,0.627451,0.172549}%
\pgfsetstrokecolor{currentstroke}%
\pgfsetdash{}{0pt}%
\pgfpathmoveto{\pgfqpoint{0.674129in}{3.175944in}}%
\pgfpathlineto{\pgfqpoint{1.363361in}{5.535073in}}%
\pgfpathlineto{\pgfqpoint{2.052592in}{3.303228in}}%
\pgfpathlineto{\pgfqpoint{2.741823in}{4.458653in}}%
\pgfpathlineto{\pgfqpoint{3.431054in}{5.204151in}}%
\pgfpathlineto{\pgfqpoint{4.120285in}{3.007271in}}%
\pgfpathlineto{\pgfqpoint{4.809516in}{4.390306in}}%
\pgfpathlineto{\pgfqpoint{5.498747in}{3.092286in}}%
\pgfpathlineto{\pgfqpoint{6.187978in}{6.377545in}}%
\pgfpathlineto{\pgfqpoint{6.877209in}{5.539446in}}%
\pgfpathlineto{\pgfqpoint{7.566440in}{4.515313in}}%
\pgfpathlineto{\pgfqpoint{8.255671in}{3.007271in}}%
\pgfpathlineto{\pgfqpoint{8.944902in}{6.287041in}}%
\pgfpathlineto{\pgfqpoint{9.634134in}{7.106984in}}%
\pgfpathlineto{\pgfqpoint{10.323365in}{6.082055in}}%
\pgfpathlineto{\pgfqpoint{11.012596in}{4.373842in}}%
\pgfpathlineto{\pgfqpoint{11.701827in}{5.740413in}}%
\pgfusepath{stroke}%
\end{pgfscope}%
\begin{pgfscope}%
\pgfpathrectangle{\pgfqpoint{0.100000in}{2.802285in}}{\pgfqpoint{12.175956in}{4.509684in}}%
\pgfusepath{clip}%
\pgfsetrectcap%
\pgfsetroundjoin%
\pgfsetlinewidth{1.505625pt}%
\definecolor{currentstroke}{rgb}{0.172549,0.627451,0.172549}%
\pgfsetstrokecolor{currentstroke}%
\pgfsetdash{}{0pt}%
\pgfpathmoveto{\pgfqpoint{0.674129in}{3.068072in}}%
\pgfpathlineto{\pgfqpoint{1.363361in}{5.237685in}}%
\pgfpathlineto{\pgfqpoint{2.052592in}{3.270380in}}%
\pgfpathlineto{\pgfqpoint{2.741823in}{5.326081in}}%
\pgfpathlineto{\pgfqpoint{3.431054in}{3.558740in}}%
\pgfpathlineto{\pgfqpoint{4.120285in}{3.007271in}}%
\pgfpathlineto{\pgfqpoint{4.809516in}{4.859551in}}%
\pgfpathlineto{\pgfqpoint{5.498747in}{3.086096in}}%
\pgfpathlineto{\pgfqpoint{6.187978in}{4.331066in}}%
\pgfpathlineto{\pgfqpoint{6.877209in}{3.989135in}}%
\pgfpathlineto{\pgfqpoint{7.566440in}{5.302456in}}%
\pgfpathlineto{\pgfqpoint{8.255671in}{3.007271in}}%
\pgfpathlineto{\pgfqpoint{8.944902in}{6.287041in}}%
\pgfpathlineto{\pgfqpoint{9.634134in}{7.106984in}}%
\pgfpathlineto{\pgfqpoint{10.323365in}{6.082055in}}%
\pgfpathlineto{\pgfqpoint{11.012596in}{7.106984in}}%
\pgfpathlineto{\pgfqpoint{11.701827in}{5.740413in}}%
\pgfusepath{stroke}%
\end{pgfscope}%
\begin{pgfscope}%
\pgfpathrectangle{\pgfqpoint{0.100000in}{2.802285in}}{\pgfqpoint{12.175956in}{4.509684in}}%
\pgfusepath{clip}%
\pgfsetrectcap%
\pgfsetroundjoin%
\pgfsetlinewidth{1.505625pt}%
\definecolor{currentstroke}{rgb}{0.172549,0.627451,0.172549}%
\pgfsetstrokecolor{currentstroke}%
\pgfsetdash{}{0pt}%
\pgfpathmoveto{\pgfqpoint{0.674129in}{3.205924in}}%
\pgfpathlineto{\pgfqpoint{1.363361in}{4.430487in}}%
\pgfpathlineto{\pgfqpoint{2.052592in}{3.244102in}}%
\pgfpathlineto{\pgfqpoint{2.741823in}{3.007271in}}%
\pgfpathlineto{\pgfqpoint{3.431054in}{3.554097in}}%
\pgfpathlineto{\pgfqpoint{4.120285in}{3.007271in}}%
\pgfpathlineto{\pgfqpoint{4.809516in}{3.328333in}}%
\pgfpathlineto{\pgfqpoint{5.498747in}{3.016350in}}%
\pgfpathlineto{\pgfqpoint{6.187978in}{3.513825in}}%
\pgfpathlineto{\pgfqpoint{6.877209in}{4.092489in}}%
\pgfpathlineto{\pgfqpoint{7.566440in}{3.119643in}}%
\pgfpathlineto{\pgfqpoint{8.255671in}{3.007271in}}%
\pgfpathlineto{\pgfqpoint{8.944902in}{4.647156in}}%
\pgfpathlineto{\pgfqpoint{9.634134in}{5.740413in}}%
\pgfpathlineto{\pgfqpoint{10.323365in}{7.106984in}}%
\pgfpathlineto{\pgfqpoint{11.012596in}{4.373842in}}%
\pgfpathlineto{\pgfqpoint{11.701827in}{5.740413in}}%
\pgfusepath{stroke}%
\end{pgfscope}%
\begin{pgfscope}%
\pgfpathrectangle{\pgfqpoint{0.100000in}{2.802285in}}{\pgfqpoint{12.175956in}{4.509684in}}%
\pgfusepath{clip}%
\pgfsetrectcap%
\pgfsetroundjoin%
\pgfsetlinewidth{1.505625pt}%
\definecolor{currentstroke}{rgb}{0.172549,0.627451,0.172549}%
\pgfsetstrokecolor{currentstroke}%
\pgfsetdash{}{0pt}%
\pgfpathmoveto{\pgfqpoint{0.674129in}{3.031367in}}%
\pgfpathlineto{\pgfqpoint{1.363361in}{4.855328in}}%
\pgfpathlineto{\pgfqpoint{2.052592in}{3.211255in}}%
\pgfpathlineto{\pgfqpoint{2.741823in}{5.355396in}}%
\pgfpathlineto{\pgfqpoint{3.431054in}{3.542113in}}%
\pgfpathlineto{\pgfqpoint{4.120285in}{3.007271in}}%
\pgfpathlineto{\pgfqpoint{4.809516in}{4.686671in}}%
\pgfpathlineto{\pgfqpoint{5.498747in}{3.079492in}}%
\pgfpathlineto{\pgfqpoint{6.187978in}{4.199362in}}%
\pgfpathlineto{\pgfqpoint{6.877209in}{3.989135in}}%
\pgfpathlineto{\pgfqpoint{7.566440in}{5.313183in}}%
\pgfpathlineto{\pgfqpoint{8.255671in}{3.007271in}}%
\pgfpathlineto{\pgfqpoint{8.944902in}{4.647156in}}%
\pgfpathlineto{\pgfqpoint{9.634134in}{5.740413in}}%
\pgfpathlineto{\pgfqpoint{10.323365in}{6.082055in}}%
\pgfpathlineto{\pgfqpoint{11.012596in}{7.106984in}}%
\pgfpathlineto{\pgfqpoint{11.701827in}{5.740413in}}%
\pgfusepath{stroke}%
\end{pgfscope}%
\begin{pgfscope}%
\pgfpathrectangle{\pgfqpoint{0.100000in}{2.802285in}}{\pgfqpoint{12.175956in}{4.509684in}}%
\pgfusepath{clip}%
\pgfsetrectcap%
\pgfsetroundjoin%
\pgfsetlinewidth{1.505625pt}%
\definecolor{currentstroke}{rgb}{0.172549,0.627451,0.172549}%
\pgfsetstrokecolor{currentstroke}%
\pgfsetdash{}{0pt}%
\pgfpathmoveto{\pgfqpoint{0.674129in}{3.066671in}}%
\pgfpathlineto{\pgfqpoint{1.363361in}{5.237685in}}%
\pgfpathlineto{\pgfqpoint{2.052592in}{3.851783in}}%
\pgfpathlineto{\pgfqpoint{2.741823in}{5.370192in}}%
\pgfpathlineto{\pgfqpoint{3.431054in}{3.562290in}}%
\pgfpathlineto{\pgfqpoint{4.120285in}{3.007271in}}%
\pgfpathlineto{\pgfqpoint{4.809516in}{4.736065in}}%
\pgfpathlineto{\pgfqpoint{5.498747in}{3.080318in}}%
\pgfpathlineto{\pgfqpoint{6.187978in}{4.236509in}}%
\pgfpathlineto{\pgfqpoint{6.877209in}{3.954683in}}%
\pgfpathlineto{\pgfqpoint{7.566440in}{5.319218in}}%
\pgfpathlineto{\pgfqpoint{8.255671in}{3.716982in}}%
\pgfpathlineto{\pgfqpoint{8.944902in}{4.647156in}}%
\pgfpathlineto{\pgfqpoint{9.634134in}{5.740413in}}%
\pgfpathlineto{\pgfqpoint{10.323365in}{7.106984in}}%
\pgfpathlineto{\pgfqpoint{11.012596in}{7.106984in}}%
\pgfpathlineto{\pgfqpoint{11.701827in}{7.106984in}}%
\pgfusepath{stroke}%
\end{pgfscope}%
\begin{pgfscope}%
\pgfpathrectangle{\pgfqpoint{0.100000in}{2.802285in}}{\pgfqpoint{12.175956in}{4.509684in}}%
\pgfusepath{clip}%
\pgfsetrectcap%
\pgfsetroundjoin%
\pgfsetlinewidth{1.505625pt}%
\definecolor{currentstroke}{rgb}{0.172549,0.627451,0.172549}%
\pgfsetstrokecolor{currentstroke}%
\pgfsetdash{}{0pt}%
\pgfpathmoveto{\pgfqpoint{0.674129in}{3.166978in}}%
\pgfpathlineto{\pgfqpoint{1.363361in}{5.747493in}}%
\pgfpathlineto{\pgfqpoint{2.052592in}{5.119702in}}%
\pgfpathlineto{\pgfqpoint{2.741823in}{5.062105in}}%
\pgfpathlineto{\pgfqpoint{3.431054in}{3.471734in}}%
\pgfpathlineto{\pgfqpoint{4.120285in}{3.007271in}}%
\pgfpathlineto{\pgfqpoint{4.809516in}{4.637277in}}%
\pgfpathlineto{\pgfqpoint{5.498747in}{3.071239in}}%
\pgfpathlineto{\pgfqpoint{6.187978in}{4.033888in}}%
\pgfpathlineto{\pgfqpoint{6.877209in}{3.851329in}}%
\pgfpathlineto{\pgfqpoint{7.566440in}{5.047673in}}%
\pgfpathlineto{\pgfqpoint{8.255671in}{4.298341in}}%
\pgfpathlineto{\pgfqpoint{8.944902in}{5.467099in}}%
\pgfpathlineto{\pgfqpoint{9.634134in}{5.740413in}}%
\pgfpathlineto{\pgfqpoint{10.323365in}{7.106984in}}%
\pgfpathlineto{\pgfqpoint{11.012596in}{7.106984in}}%
\pgfpathlineto{\pgfqpoint{11.701827in}{7.106984in}}%
\pgfusepath{stroke}%
\end{pgfscope}%
\begin{pgfscope}%
\pgfpathrectangle{\pgfqpoint{0.100000in}{2.802285in}}{\pgfqpoint{12.175956in}{4.509684in}}%
\pgfusepath{clip}%
\pgfsetrectcap%
\pgfsetroundjoin%
\pgfsetlinewidth{1.505625pt}%
\definecolor{currentstroke}{rgb}{0.121569,0.466667,0.705882}%
\pgfsetstrokecolor{currentstroke}%
\pgfsetdash{}{0pt}%
\pgfpathmoveto{\pgfqpoint{0.674129in}{3.044816in}}%
\pgfpathlineto{\pgfqpoint{1.363361in}{4.430487in}}%
\pgfpathlineto{\pgfqpoint{2.052592in}{3.088076in}}%
\pgfpathlineto{\pgfqpoint{2.741823in}{3.913555in}}%
\pgfpathlineto{\pgfqpoint{3.431054in}{5.988109in}}%
\pgfpathlineto{\pgfqpoint{4.120285in}{3.007271in}}%
\pgfpathlineto{\pgfqpoint{4.809516in}{3.995153in}}%
\pgfpathlineto{\pgfqpoint{5.498747in}{3.086096in}}%
\pgfpathlineto{\pgfqpoint{6.187978in}{6.870592in}}%
\pgfpathlineto{\pgfqpoint{6.877209in}{5.573898in}}%
\pgfpathlineto{\pgfqpoint{7.566440in}{3.924621in}}%
\pgfpathlineto{\pgfqpoint{8.255671in}{3.007271in}}%
\pgfpathlineto{\pgfqpoint{8.944902in}{6.287041in}}%
\pgfpathlineto{\pgfqpoint{9.634134in}{7.106984in}}%
\pgfpathlineto{\pgfqpoint{10.323365in}{6.082055in}}%
\pgfpathlineto{\pgfqpoint{11.012596in}{4.373842in}}%
\pgfpathlineto{\pgfqpoint{11.701827in}{5.740413in}}%
\pgfusepath{stroke}%
\end{pgfscope}%
\begin{pgfscope}%
\pgfpathrectangle{\pgfqpoint{0.100000in}{2.802285in}}{\pgfqpoint{12.175956in}{4.509684in}}%
\pgfusepath{clip}%
\pgfsetrectcap%
\pgfsetroundjoin%
\pgfsetlinewidth{1.505625pt}%
\definecolor{currentstroke}{rgb}{0.121569,0.466667,0.705882}%
\pgfsetstrokecolor{currentstroke}%
\pgfsetdash{}{0pt}%
\pgfpathmoveto{\pgfqpoint{0.674129in}{3.045376in}}%
\pgfpathlineto{\pgfqpoint{1.363361in}{4.048131in}}%
\pgfpathlineto{\pgfqpoint{2.052592in}{3.019753in}}%
\pgfpathlineto{\pgfqpoint{2.741823in}{3.966378in}}%
\pgfpathlineto{\pgfqpoint{3.431054in}{6.324954in}}%
\pgfpathlineto{\pgfqpoint{4.120285in}{3.007271in}}%
\pgfpathlineto{\pgfqpoint{4.809516in}{4.044548in}}%
\pgfpathlineto{\pgfqpoint{5.498747in}{3.091048in}}%
\pgfpathlineto{\pgfqpoint{6.187978in}{7.106984in}}%
\pgfpathlineto{\pgfqpoint{6.877209in}{5.608349in}}%
\pgfpathlineto{\pgfqpoint{7.566440in}{3.976248in}}%
\pgfpathlineto{\pgfqpoint{8.255671in}{3.007271in}}%
\pgfpathlineto{\pgfqpoint{8.944902in}{5.467099in}}%
\pgfpathlineto{\pgfqpoint{9.634134in}{5.740413in}}%
\pgfpathlineto{\pgfqpoint{10.323365in}{7.106984in}}%
\pgfpathlineto{\pgfqpoint{11.012596in}{4.373842in}}%
\pgfpathlineto{\pgfqpoint{11.701827in}{7.106984in}}%
\pgfusepath{stroke}%
\end{pgfscope}%
\begin{pgfscope}%
\pgfpathrectangle{\pgfqpoint{0.100000in}{2.802285in}}{\pgfqpoint{12.175956in}{4.509684in}}%
\pgfusepath{clip}%
\pgfsetrectcap%
\pgfsetroundjoin%
\pgfsetlinewidth{1.505625pt}%
\definecolor{currentstroke}{rgb}{0.121569,0.466667,0.705882}%
\pgfsetstrokecolor{currentstroke}%
\pgfsetdash{}{0pt}%
\pgfpathmoveto{\pgfqpoint{0.674129in}{3.013155in}}%
\pgfpathlineto{\pgfqpoint{1.363361in}{4.430487in}}%
\pgfpathlineto{\pgfqpoint{2.052592in}{3.099573in}}%
\pgfpathlineto{\pgfqpoint{2.741823in}{5.185312in}}%
\pgfpathlineto{\pgfqpoint{3.431054in}{3.648959in}}%
\pgfpathlineto{\pgfqpoint{4.120285in}{3.007271in}}%
\pgfpathlineto{\pgfqpoint{4.809516in}{4.538489in}}%
\pgfpathlineto{\pgfqpoint{5.498747in}{3.076604in}}%
\pgfpathlineto{\pgfqpoint{6.187978in}{4.277034in}}%
\pgfpathlineto{\pgfqpoint{6.877209in}{3.868555in}}%
\pgfpathlineto{\pgfqpoint{7.566440in}{5.139663in}}%
\pgfpathlineto{\pgfqpoint{8.255671in}{3.007271in}}%
\pgfpathlineto{\pgfqpoint{8.944902in}{6.287041in}}%
\pgfpathlineto{\pgfqpoint{9.634134in}{7.106984in}}%
\pgfpathlineto{\pgfqpoint{10.323365in}{6.082055in}}%
\pgfpathlineto{\pgfqpoint{11.012596in}{7.106984in}}%
\pgfpathlineto{\pgfqpoint{11.701827in}{5.740413in}}%
\pgfusepath{stroke}%
\end{pgfscope}%
\begin{pgfscope}%
\pgfpathrectangle{\pgfqpoint{0.100000in}{2.802285in}}{\pgfqpoint{12.175956in}{4.509684in}}%
\pgfusepath{clip}%
\pgfsetrectcap%
\pgfsetroundjoin%
\pgfsetlinewidth{1.505625pt}%
\definecolor{currentstroke}{rgb}{0.121569,0.466667,0.705882}%
\pgfsetstrokecolor{currentstroke}%
\pgfsetdash{}{0pt}%
\pgfpathmoveto{\pgfqpoint{0.674129in}{3.011193in}}%
\pgfpathlineto{\pgfqpoint{1.363361in}{3.810220in}}%
\pgfpathlineto{\pgfqpoint{2.052592in}{3.027636in}}%
\pgfpathlineto{\pgfqpoint{2.741823in}{5.432971in}}%
\pgfpathlineto{\pgfqpoint{3.431054in}{3.670084in}}%
\pgfpathlineto{\pgfqpoint{4.120285in}{3.007271in}}%
\pgfpathlineto{\pgfqpoint{4.809516in}{4.711368in}}%
\pgfpathlineto{\pgfqpoint{5.498747in}{3.083619in}}%
\pgfpathlineto{\pgfqpoint{6.187978in}{4.351328in}}%
\pgfpathlineto{\pgfqpoint{6.877209in}{3.920232in}}%
\pgfpathlineto{\pgfqpoint{7.566440in}{5.378890in}}%
\pgfpathlineto{\pgfqpoint{8.255671in}{3.007271in}}%
\pgfpathlineto{\pgfqpoint{8.944902in}{5.467099in}}%
\pgfpathlineto{\pgfqpoint{9.634134in}{5.740413in}}%
\pgfpathlineto{\pgfqpoint{10.323365in}{7.106984in}}%
\pgfpathlineto{\pgfqpoint{11.012596in}{7.106984in}}%
\pgfpathlineto{\pgfqpoint{11.701827in}{7.106984in}}%
\pgfusepath{stroke}%
\end{pgfscope}%
\begin{pgfscope}%
\pgfpathrectangle{\pgfqpoint{0.100000in}{2.802285in}}{\pgfqpoint{12.175956in}{4.509684in}}%
\pgfusepath{clip}%
\pgfsetrectcap%
\pgfsetroundjoin%
\pgfsetlinewidth{1.505625pt}%
\definecolor{currentstroke}{rgb}{0.121569,0.466667,0.705882}%
\pgfsetstrokecolor{currentstroke}%
\pgfsetdash{}{0pt}%
\pgfpathmoveto{\pgfqpoint{0.674129in}{3.008952in}}%
\pgfpathlineto{\pgfqpoint{1.363361in}{3.963162in}}%
\pgfpathlineto{\pgfqpoint{2.052592in}{3.075922in}}%
\pgfpathlineto{\pgfqpoint{2.741823in}{3.007271in}}%
\pgfpathlineto{\pgfqpoint{3.431054in}{3.442448in}}%
\pgfpathlineto{\pgfqpoint{4.120285in}{3.007271in}}%
\pgfpathlineto{\pgfqpoint{4.809516in}{3.007271in}}%
\pgfpathlineto{\pgfqpoint{5.498747in}{3.011810in}}%
\pgfpathlineto{\pgfqpoint{6.187978in}{3.365236in}}%
\pgfpathlineto{\pgfqpoint{6.877209in}{4.144166in}}%
\pgfpathlineto{\pgfqpoint{7.566440in}{3.025240in}}%
\pgfpathlineto{\pgfqpoint{8.255671in}{3.007271in}}%
\pgfpathlineto{\pgfqpoint{8.944902in}{4.647156in}}%
\pgfpathlineto{\pgfqpoint{9.634134in}{5.740413in}}%
\pgfpathlineto{\pgfqpoint{10.323365in}{6.082055in}}%
\pgfpathlineto{\pgfqpoint{11.012596in}{5.740413in}}%
\pgfpathlineto{\pgfqpoint{11.701827in}{5.740413in}}%
\pgfusepath{stroke}%
\end{pgfscope}%
\begin{pgfscope}%
\pgfpathrectangle{\pgfqpoint{0.100000in}{2.802285in}}{\pgfqpoint{12.175956in}{4.509684in}}%
\pgfusepath{clip}%
\pgfsetrectcap%
\pgfsetroundjoin%
\pgfsetlinewidth{1.505625pt}%
\definecolor{currentstroke}{rgb}{0.121569,0.466667,0.705882}%
\pgfsetstrokecolor{currentstroke}%
\pgfsetdash{}{0pt}%
\pgfpathmoveto{\pgfqpoint{0.674129in}{3.009792in}}%
\pgfpathlineto{\pgfqpoint{1.363361in}{4.005646in}}%
\pgfpathlineto{\pgfqpoint{2.052592in}{3.035848in}}%
\pgfpathlineto{\pgfqpoint{2.741823in}{3.007271in}}%
\pgfpathlineto{\pgfqpoint{3.431054in}{3.461613in}}%
\pgfpathlineto{\pgfqpoint{4.120285in}{3.007271in}}%
\pgfpathlineto{\pgfqpoint{4.809516in}{3.031968in}}%
\pgfpathlineto{\pgfqpoint{5.498747in}{3.011810in}}%
\pgfpathlineto{\pgfqpoint{6.187978in}{3.365236in}}%
\pgfpathlineto{\pgfqpoint{6.877209in}{4.144166in}}%
\pgfpathlineto{\pgfqpoint{7.566440in}{3.025642in}}%
\pgfpathlineto{\pgfqpoint{8.255671in}{3.007271in}}%
\pgfpathlineto{\pgfqpoint{8.944902in}{5.467099in}}%
\pgfpathlineto{\pgfqpoint{9.634134in}{5.740413in}}%
\pgfpathlineto{\pgfqpoint{10.323365in}{7.106984in}}%
\pgfpathlineto{\pgfqpoint{11.012596in}{5.740413in}}%
\pgfpathlineto{\pgfqpoint{11.701827in}{7.106984in}}%
\pgfusepath{stroke}%
\end{pgfscope}%
\begin{pgfscope}%
\pgfpathrectangle{\pgfqpoint{0.100000in}{2.802285in}}{\pgfqpoint{12.175956in}{4.509684in}}%
\pgfusepath{clip}%
\pgfsetrectcap%
\pgfsetroundjoin%
\pgfsetlinewidth{1.505625pt}%
\definecolor{currentstroke}{rgb}{0.121569,0.466667,0.705882}%
\pgfsetstrokecolor{currentstroke}%
\pgfsetdash{}{0pt}%
\pgfpathmoveto{\pgfqpoint{0.674129in}{3.014556in}}%
\pgfpathlineto{\pgfqpoint{1.363361in}{5.110232in}}%
\pgfpathlineto{\pgfqpoint{2.052592in}{3.120595in}}%
\pgfpathlineto{\pgfqpoint{2.741823in}{4.567203in}}%
\pgfpathlineto{\pgfqpoint{3.431054in}{3.930766in}}%
\pgfpathlineto{\pgfqpoint{4.120285in}{3.007271in}}%
\pgfpathlineto{\pgfqpoint{4.809516in}{4.168033in}}%
\pgfpathlineto{\pgfqpoint{5.498747in}{3.062159in}}%
\pgfpathlineto{\pgfqpoint{6.187978in}{4.290542in}}%
\pgfpathlineto{\pgfqpoint{6.877209in}{3.696298in}}%
\pgfpathlineto{\pgfqpoint{7.566440in}{4.529795in}}%
\pgfpathlineto{\pgfqpoint{8.255671in}{3.328151in}}%
\pgfpathlineto{\pgfqpoint{8.944902in}{5.467099in}}%
\pgfpathlineto{\pgfqpoint{9.634134in}{5.740413in}}%
\pgfpathlineto{\pgfqpoint{10.323365in}{7.106984in}}%
\pgfpathlineto{\pgfqpoint{11.012596in}{5.740413in}}%
\pgfpathlineto{\pgfqpoint{11.701827in}{7.106984in}}%
\pgfusepath{stroke}%
\end{pgfscope}%
\begin{pgfscope}%
\pgfpathrectangle{\pgfqpoint{0.100000in}{2.802285in}}{\pgfqpoint{12.175956in}{4.509684in}}%
\pgfusepath{clip}%
\pgfsetrectcap%
\pgfsetroundjoin%
\pgfsetlinewidth{1.505625pt}%
\definecolor{currentstroke}{rgb}{0.121569,0.466667,0.705882}%
\pgfsetstrokecolor{currentstroke}%
\pgfsetdash{}{0pt}%
\pgfpathmoveto{\pgfqpoint{0.674129in}{3.007271in}}%
\pgfpathlineto{\pgfqpoint{1.363361in}{5.110232in}}%
\pgfpathlineto{\pgfqpoint{2.052592in}{3.286804in}}%
\pgfpathlineto{\pgfqpoint{2.741823in}{3.007271in}}%
\pgfpathlineto{\pgfqpoint{3.431054in}{7.106984in}}%
\pgfpathlineto{\pgfqpoint{4.120285in}{3.007271in}}%
\pgfpathlineto{\pgfqpoint{4.809516in}{3.130756in}}%
\pgfpathlineto{\pgfqpoint{5.498747in}{3.030382in}}%
\pgfpathlineto{\pgfqpoint{6.187978in}{5.161815in}}%
\pgfpathlineto{\pgfqpoint{6.877209in}{3.007271in}}%
\pgfpathlineto{\pgfqpoint{7.566440in}{3.007271in}}%
\pgfpathlineto{\pgfqpoint{8.255671in}{3.479153in}}%
\pgfpathlineto{\pgfqpoint{8.944902in}{5.467099in}}%
\pgfpathlineto{\pgfqpoint{9.634134in}{5.740413in}}%
\pgfpathlineto{\pgfqpoint{10.323365in}{7.106984in}}%
\pgfpathlineto{\pgfqpoint{11.012596in}{5.740413in}}%
\pgfpathlineto{\pgfqpoint{11.701827in}{7.106984in}}%
\pgfusepath{stroke}%
\end{pgfscope}%
\begin{pgfscope}%
\pgfpathrectangle{\pgfqpoint{0.100000in}{2.802285in}}{\pgfqpoint{12.175956in}{4.509684in}}%
\pgfusepath{clip}%
\pgfsetrectcap%
\pgfsetroundjoin%
\pgfsetlinewidth{1.505625pt}%
\definecolor{currentstroke}{rgb}{0.121569,0.466667,0.705882}%
\pgfsetstrokecolor{currentstroke}%
\pgfsetdash{}{0pt}%
\pgfpathmoveto{\pgfqpoint{0.674129in}{3.016517in}}%
\pgfpathlineto{\pgfqpoint{1.363361in}{4.600423in}}%
\pgfpathlineto{\pgfqpoint{2.052592in}{3.825505in}}%
\pgfpathlineto{\pgfqpoint{2.741823in}{3.007271in}}%
\pgfpathlineto{\pgfqpoint{3.431054in}{4.437222in}}%
\pgfpathlineto{\pgfqpoint{4.120285in}{3.007271in}}%
\pgfpathlineto{\pgfqpoint{4.809516in}{3.130756in}}%
\pgfpathlineto{\pgfqpoint{5.498747in}{3.024604in}}%
\pgfpathlineto{\pgfqpoint{6.187978in}{4.243263in}}%
\pgfpathlineto{\pgfqpoint{6.877209in}{4.195843in}}%
\pgfpathlineto{\pgfqpoint{7.566440in}{3.007271in}}%
\pgfpathlineto{\pgfqpoint{8.255671in}{3.641481in}}%
\pgfpathlineto{\pgfqpoint{8.944902in}{5.467099in}}%
\pgfpathlineto{\pgfqpoint{9.634134in}{5.740413in}}%
\pgfpathlineto{\pgfqpoint{10.323365in}{7.106984in}}%
\pgfpathlineto{\pgfqpoint{11.012596in}{5.740413in}}%
\pgfpathlineto{\pgfqpoint{11.701827in}{7.106984in}}%
\pgfusepath{stroke}%
\end{pgfscope}%
\begin{pgfscope}%
\pgfpathrectangle{\pgfqpoint{0.100000in}{2.802285in}}{\pgfqpoint{12.175956in}{4.509684in}}%
\pgfusepath{clip}%
\pgfsetrectcap%
\pgfsetroundjoin%
\pgfsetlinewidth{1.505625pt}%
\definecolor{currentstroke}{rgb}{0.121569,0.466667,0.705882}%
\pgfsetstrokecolor{currentstroke}%
\pgfsetdash{}{0pt}%
\pgfpathmoveto{\pgfqpoint{0.674129in}{3.012875in}}%
\pgfpathlineto{\pgfqpoint{1.363361in}{4.388003in}}%
\pgfpathlineto{\pgfqpoint{2.052592in}{3.138004in}}%
\pgfpathlineto{\pgfqpoint{2.741823in}{3.007271in}}%
\pgfpathlineto{\pgfqpoint{3.431054in}{3.623545in}}%
\pgfpathlineto{\pgfqpoint{4.120285in}{3.007271in}}%
\pgfpathlineto{\pgfqpoint{4.809516in}{3.056665in}}%
\pgfpathlineto{\pgfqpoint{5.498747in}{3.015112in}}%
\pgfpathlineto{\pgfqpoint{6.187978in}{3.581366in}}%
\pgfpathlineto{\pgfqpoint{6.877209in}{3.816878in}}%
\pgfpathlineto{\pgfqpoint{7.566440in}{3.027251in}}%
\pgfpathlineto{\pgfqpoint{8.255671in}{3.007271in}}%
\pgfpathlineto{\pgfqpoint{8.944902in}{4.647156in}}%
\pgfpathlineto{\pgfqpoint{9.634134in}{5.740413in}}%
\pgfpathlineto{\pgfqpoint{10.323365in}{7.106984in}}%
\pgfpathlineto{\pgfqpoint{11.012596in}{4.373842in}}%
\pgfpathlineto{\pgfqpoint{11.701827in}{5.740413in}}%
\pgfusepath{stroke}%
\end{pgfscope}%
\begin{pgfscope}%
\pgfpathrectangle{\pgfqpoint{0.100000in}{2.802285in}}{\pgfqpoint{12.175956in}{4.509684in}}%
\pgfusepath{clip}%
\pgfsetrectcap%
\pgfsetroundjoin%
\pgfsetlinewidth{1.505625pt}%
\definecolor{currentstroke}{rgb}{0.121569,0.466667,0.705882}%
\pgfsetstrokecolor{currentstroke}%
\pgfsetdash{}{0pt}%
\pgfpathmoveto{\pgfqpoint{0.674129in}{3.008672in}}%
\pgfpathlineto{\pgfqpoint{1.363361in}{3.640283in}}%
\pgfpathlineto{\pgfqpoint{2.052592in}{3.024352in}}%
\pgfpathlineto{\pgfqpoint{2.741823in}{3.007271in}}%
\pgfpathlineto{\pgfqpoint{3.431054in}{3.528394in}}%
\pgfpathlineto{\pgfqpoint{4.120285in}{3.007271in}}%
\pgfpathlineto{\pgfqpoint{4.809516in}{3.007271in}}%
\pgfpathlineto{\pgfqpoint{5.498747in}{3.012636in}}%
\pgfpathlineto{\pgfqpoint{6.187978in}{3.432776in}}%
\pgfpathlineto{\pgfqpoint{6.877209in}{4.023586in}}%
\pgfpathlineto{\pgfqpoint{7.566440in}{3.025106in}}%
\pgfpathlineto{\pgfqpoint{8.255671in}{3.007271in}}%
\pgfpathlineto{\pgfqpoint{8.944902in}{5.467099in}}%
\pgfpathlineto{\pgfqpoint{9.634134in}{5.740413in}}%
\pgfpathlineto{\pgfqpoint{10.323365in}{7.106984in}}%
\pgfpathlineto{\pgfqpoint{11.012596in}{4.373842in}}%
\pgfpathlineto{\pgfqpoint{11.701827in}{7.106984in}}%
\pgfusepath{stroke}%
\end{pgfscope}%
\begin{pgfscope}%
\pgfpathrectangle{\pgfqpoint{0.100000in}{2.802285in}}{\pgfqpoint{12.175956in}{4.509684in}}%
\pgfusepath{clip}%
\pgfsetrectcap%
\pgfsetroundjoin%
\pgfsetlinewidth{1.505625pt}%
\definecolor{currentstroke}{rgb}{0.121569,0.466667,0.705882}%
\pgfsetstrokecolor{currentstroke}%
\pgfsetdash{}{0pt}%
\pgfpathmoveto{\pgfqpoint{0.674129in}{3.010633in}}%
\pgfpathlineto{\pgfqpoint{1.363361in}{4.600423in}}%
\pgfpathlineto{\pgfqpoint{2.052592in}{3.065740in}}%
\pgfpathlineto{\pgfqpoint{2.741823in}{4.895478in}}%
\pgfpathlineto{\pgfqpoint{3.431054in}{3.471942in}}%
\pgfpathlineto{\pgfqpoint{4.120285in}{3.007271in}}%
\pgfpathlineto{\pgfqpoint{4.809516in}{4.316215in}}%
\pgfpathlineto{\pgfqpoint{5.498747in}{3.065461in}}%
\pgfpathlineto{\pgfqpoint{6.187978in}{3.993363in}}%
\pgfpathlineto{\pgfqpoint{6.877209in}{4.006360in}}%
\pgfpathlineto{\pgfqpoint{7.566440in}{4.857257in}}%
\pgfpathlineto{\pgfqpoint{8.255671in}{3.007271in}}%
\pgfpathlineto{\pgfqpoint{8.944902in}{4.647156in}}%
\pgfpathlineto{\pgfqpoint{9.634134in}{5.740413in}}%
\pgfpathlineto{\pgfqpoint{10.323365in}{6.082055in}}%
\pgfpathlineto{\pgfqpoint{11.012596in}{4.373842in}}%
\pgfpathlineto{\pgfqpoint{11.701827in}{5.740413in}}%
\pgfusepath{stroke}%
\end{pgfscope}%
\begin{pgfscope}%
\pgfpathrectangle{\pgfqpoint{0.100000in}{2.802285in}}{\pgfqpoint{12.175956in}{4.509684in}}%
\pgfusepath{clip}%
\pgfsetrectcap%
\pgfsetroundjoin%
\pgfsetlinewidth{1.505625pt}%
\definecolor{currentstroke}{rgb}{0.121569,0.466667,0.705882}%
\pgfsetstrokecolor{currentstroke}%
\pgfsetdash{}{0pt}%
\pgfpathmoveto{\pgfqpoint{0.674129in}{3.009232in}}%
\pgfpathlineto{\pgfqpoint{1.363361in}{4.175583in}}%
\pgfpathlineto{\pgfqpoint{2.052592in}{3.007271in}}%
\pgfpathlineto{\pgfqpoint{2.741823in}{4.765496in}}%
\pgfpathlineto{\pgfqpoint{3.431054in}{3.490288in}}%
\pgfpathlineto{\pgfqpoint{4.120285in}{3.007271in}}%
\pgfpathlineto{\pgfqpoint{4.809516in}{4.217427in}}%
\pgfpathlineto{\pgfqpoint{5.498747in}{3.061747in}}%
\pgfpathlineto{\pgfqpoint{6.187978in}{3.956216in}}%
\pgfpathlineto{\pgfqpoint{6.877209in}{4.006360in}}%
\pgfpathlineto{\pgfqpoint{7.566440in}{4.730403in}}%
\pgfpathlineto{\pgfqpoint{8.255671in}{3.007271in}}%
\pgfpathlineto{\pgfqpoint{8.944902in}{5.467099in}}%
\pgfpathlineto{\pgfqpoint{9.634134in}{5.740413in}}%
\pgfpathlineto{\pgfqpoint{10.323365in}{7.106984in}}%
\pgfpathlineto{\pgfqpoint{11.012596in}{4.373842in}}%
\pgfpathlineto{\pgfqpoint{11.701827in}{7.106984in}}%
\pgfusepath{stroke}%
\end{pgfscope}%
\begin{pgfscope}%
\pgfpathrectangle{\pgfqpoint{0.100000in}{2.802285in}}{\pgfqpoint{12.175956in}{4.509684in}}%
\pgfusepath{clip}%
\pgfsetrectcap%
\pgfsetroundjoin%
\pgfsetlinewidth{1.505625pt}%
\definecolor{currentstroke}{rgb}{0.121569,0.466667,0.705882}%
\pgfsetstrokecolor{currentstroke}%
\pgfsetdash{}{0pt}%
\pgfpathmoveto{\pgfqpoint{0.674129in}{3.011193in}}%
\pgfpathlineto{\pgfqpoint{1.363361in}{3.920678in}}%
\pgfpathlineto{\pgfqpoint{2.052592in}{3.012526in}}%
\pgfpathlineto{\pgfqpoint{2.741823in}{5.069019in}}%
\pgfpathlineto{\pgfqpoint{3.431054in}{3.740721in}}%
\pgfpathlineto{\pgfqpoint{4.120285in}{7.106984in}}%
\pgfpathlineto{\pgfqpoint{4.809516in}{4.464398in}}%
\pgfpathlineto{\pgfqpoint{5.498747in}{3.081969in}}%
\pgfpathlineto{\pgfqpoint{6.187978in}{4.229755in}}%
\pgfpathlineto{\pgfqpoint{6.877209in}{4.023586in}}%
\pgfpathlineto{\pgfqpoint{7.566440in}{5.025950in}}%
\pgfpathlineto{\pgfqpoint{8.255671in}{3.007271in}}%
\pgfpathlineto{\pgfqpoint{8.944902in}{7.106984in}}%
\pgfpathlineto{\pgfqpoint{9.634134in}{5.740413in}}%
\pgfpathlineto{\pgfqpoint{10.323365in}{7.106984in}}%
\pgfpathlineto{\pgfqpoint{11.012596in}{7.106984in}}%
\pgfpathlineto{\pgfqpoint{11.701827in}{7.106984in}}%
\pgfusepath{stroke}%
\end{pgfscope}%
\begin{pgfscope}%
\pgfpathrectangle{\pgfqpoint{0.100000in}{2.802285in}}{\pgfqpoint{12.175956in}{4.509684in}}%
\pgfusepath{clip}%
\pgfsetrectcap%
\pgfsetroundjoin%
\pgfsetlinewidth{1.505625pt}%
\definecolor{currentstroke}{rgb}{0.121569,0.466667,0.705882}%
\pgfsetstrokecolor{currentstroke}%
\pgfsetdash{}{0pt}%
\pgfpathmoveto{\pgfqpoint{0.674129in}{3.009512in}}%
\pgfpathlineto{\pgfqpoint{1.363361in}{4.345519in}}%
\pgfpathlineto{\pgfqpoint{2.052592in}{3.087091in}}%
\pgfpathlineto{\pgfqpoint{2.741823in}{3.331122in}}%
\pgfpathlineto{\pgfqpoint{3.431054in}{3.562981in}}%
\pgfpathlineto{\pgfqpoint{4.120285in}{3.007271in}}%
\pgfpathlineto{\pgfqpoint{4.809516in}{3.254241in}}%
\pgfpathlineto{\pgfqpoint{5.498747in}{3.011810in}}%
\pgfpathlineto{\pgfqpoint{6.187978in}{3.628644in}}%
\pgfpathlineto{\pgfqpoint{6.877209in}{3.799652in}}%
\pgfpathlineto{\pgfqpoint{7.566440in}{3.333928in}}%
\pgfpathlineto{\pgfqpoint{8.255671in}{3.169598in}}%
\pgfpathlineto{\pgfqpoint{8.944902in}{5.467099in}}%
\pgfpathlineto{\pgfqpoint{9.634134in}{5.740413in}}%
\pgfpathlineto{\pgfqpoint{10.323365in}{7.106984in}}%
\pgfpathlineto{\pgfqpoint{11.012596in}{4.373842in}}%
\pgfpathlineto{\pgfqpoint{11.701827in}{7.106984in}}%
\pgfusepath{stroke}%
\end{pgfscope}%
\begin{pgfscope}%
\pgfpathrectangle{\pgfqpoint{0.100000in}{2.802285in}}{\pgfqpoint{12.175956in}{4.509684in}}%
\pgfusepath{clip}%
\pgfsetrectcap%
\pgfsetroundjoin%
\pgfsetlinewidth{1.505625pt}%
\definecolor{currentstroke}{rgb}{0.121569,0.466667,0.705882}%
\pgfsetstrokecolor{currentstroke}%
\pgfsetdash{}{0pt}%
\pgfpathmoveto{\pgfqpoint{0.674129in}{3.022401in}}%
\pgfpathlineto{\pgfqpoint{1.363361in}{4.982780in}}%
\pgfpathlineto{\pgfqpoint{2.052592in}{3.878061in}}%
\pgfpathlineto{\pgfqpoint{2.741823in}{3.007271in}}%
\pgfpathlineto{\pgfqpoint{3.431054in}{4.999503in}}%
\pgfpathlineto{\pgfqpoint{4.120285in}{7.106984in}}%
\pgfpathlineto{\pgfqpoint{4.809516in}{3.229544in}}%
\pgfpathlineto{\pgfqpoint{5.498747in}{3.033683in}}%
\pgfpathlineto{\pgfqpoint{6.187978in}{4.813981in}}%
\pgfpathlineto{\pgfqpoint{6.877209in}{4.092489in}}%
\pgfpathlineto{\pgfqpoint{7.566440in}{3.007271in}}%
\pgfpathlineto{\pgfqpoint{8.255671in}{3.735857in}}%
\pgfpathlineto{\pgfqpoint{8.944902in}{5.467099in}}%
\pgfpathlineto{\pgfqpoint{9.634134in}{5.740413in}}%
\pgfpathlineto{\pgfqpoint{10.323365in}{7.106984in}}%
\pgfpathlineto{\pgfqpoint{11.012596in}{4.373842in}}%
\pgfpathlineto{\pgfqpoint{11.701827in}{7.106984in}}%
\pgfusepath{stroke}%
\end{pgfscope}%
\begin{pgfscope}%
\pgfpathrectangle{\pgfqpoint{0.100000in}{2.802285in}}{\pgfqpoint{12.175956in}{4.509684in}}%
\pgfusepath{clip}%
\pgfsetrectcap%
\pgfsetroundjoin%
\pgfsetlinewidth{1.505625pt}%
\definecolor{currentstroke}{rgb}{0.121569,0.466667,0.705882}%
\pgfsetstrokecolor{currentstroke}%
\pgfsetdash{}{0pt}%
\pgfpathmoveto{\pgfqpoint{0.674129in}{3.013435in}}%
\pgfpathlineto{\pgfqpoint{1.363361in}{5.917430in}}%
\pgfpathlineto{\pgfqpoint{2.052592in}{3.257241in}}%
\pgfpathlineto{\pgfqpoint{2.741823in}{3.007271in}}%
\pgfpathlineto{\pgfqpoint{3.431054in}{3.512313in}}%
\pgfpathlineto{\pgfqpoint{4.120285in}{7.106984in}}%
\pgfpathlineto{\pgfqpoint{4.809516in}{3.328333in}}%
\pgfpathlineto{\pgfqpoint{5.498747in}{4.492973in}}%
\pgfpathlineto{\pgfqpoint{6.187978in}{3.544218in}}%
\pgfpathlineto{\pgfqpoint{6.877209in}{4.626485in}}%
\pgfpathlineto{\pgfqpoint{7.566440in}{3.007271in}}%
\pgfpathlineto{\pgfqpoint{8.255671in}{3.577305in}}%
\pgfpathlineto{\pgfqpoint{8.944902in}{7.106984in}}%
\pgfpathlineto{\pgfqpoint{9.634134in}{5.740413in}}%
\pgfpathlineto{\pgfqpoint{10.323365in}{7.106984in}}%
\pgfpathlineto{\pgfqpoint{11.012596in}{7.106984in}}%
\pgfpathlineto{\pgfqpoint{11.701827in}{7.106984in}}%
\pgfusepath{stroke}%
\end{pgfscope}%
\begin{pgfscope}%
\pgfpathrectangle{\pgfqpoint{0.100000in}{2.802285in}}{\pgfqpoint{12.175956in}{4.509684in}}%
\pgfusepath{clip}%
\pgfsetrectcap%
\pgfsetroundjoin%
\pgfsetlinewidth{2.007500pt}%
\definecolor{currentstroke}{rgb}{0.501961,0.501961,0.501961}%
\pgfsetstrokecolor{currentstroke}%
\pgfsetstrokeopacity{0.500000}%
\pgfsetdash{}{0pt}%
\pgfpathmoveto{\pgfqpoint{0.674129in}{2.802285in}}%
\pgfpathlineto{\pgfqpoint{0.674129in}{7.311969in}}%
\pgfusepath{stroke}%
\end{pgfscope}%
\begin{pgfscope}%
\pgfpathrectangle{\pgfqpoint{0.100000in}{2.802285in}}{\pgfqpoint{12.175956in}{4.509684in}}%
\pgfusepath{clip}%
\pgfsetbuttcap%
\pgfsetroundjoin%
\pgfsetlinewidth{2.007500pt}%
\definecolor{currentstroke}{rgb}{0.501961,0.501961,0.501961}%
\pgfsetstrokecolor{currentstroke}%
\pgfsetstrokeopacity{0.500000}%
\pgfsetdash{}{0pt}%
\pgfpathmoveto{\pgfqpoint{0.653453in}{3.007271in}}%
\pgfpathlineto{\pgfqpoint{0.694806in}{3.007271in}}%
\pgfusepath{stroke}%
\end{pgfscope}%
\begin{pgfscope}%
\pgfpathrectangle{\pgfqpoint{0.100000in}{2.802285in}}{\pgfqpoint{12.175956in}{4.509684in}}%
\pgfusepath{clip}%
\pgfsetbuttcap%
\pgfsetroundjoin%
\pgfsetlinewidth{2.007500pt}%
\definecolor{currentstroke}{rgb}{0.501961,0.501961,0.501961}%
\pgfsetstrokecolor{currentstroke}%
\pgfsetstrokeopacity{0.500000}%
\pgfsetdash{}{0pt}%
\pgfpathmoveto{\pgfqpoint{0.653453in}{3.462794in}}%
\pgfpathlineto{\pgfqpoint{0.694806in}{3.462794in}}%
\pgfusepath{stroke}%
\end{pgfscope}%
\begin{pgfscope}%
\pgfpathrectangle{\pgfqpoint{0.100000in}{2.802285in}}{\pgfqpoint{12.175956in}{4.509684in}}%
\pgfusepath{clip}%
\pgfsetbuttcap%
\pgfsetroundjoin%
\pgfsetlinewidth{2.007500pt}%
\definecolor{currentstroke}{rgb}{0.501961,0.501961,0.501961}%
\pgfsetstrokecolor{currentstroke}%
\pgfsetstrokeopacity{0.500000}%
\pgfsetdash{}{0pt}%
\pgfpathmoveto{\pgfqpoint{0.653453in}{3.918318in}}%
\pgfpathlineto{\pgfqpoint{0.694806in}{3.918318in}}%
\pgfusepath{stroke}%
\end{pgfscope}%
\begin{pgfscope}%
\pgfpathrectangle{\pgfqpoint{0.100000in}{2.802285in}}{\pgfqpoint{12.175956in}{4.509684in}}%
\pgfusepath{clip}%
\pgfsetbuttcap%
\pgfsetroundjoin%
\pgfsetlinewidth{2.007500pt}%
\definecolor{currentstroke}{rgb}{0.501961,0.501961,0.501961}%
\pgfsetstrokecolor{currentstroke}%
\pgfsetstrokeopacity{0.500000}%
\pgfsetdash{}{0pt}%
\pgfpathmoveto{\pgfqpoint{0.653453in}{4.373842in}}%
\pgfpathlineto{\pgfqpoint{0.694806in}{4.373842in}}%
\pgfusepath{stroke}%
\end{pgfscope}%
\begin{pgfscope}%
\pgfpathrectangle{\pgfqpoint{0.100000in}{2.802285in}}{\pgfqpoint{12.175956in}{4.509684in}}%
\pgfusepath{clip}%
\pgfsetbuttcap%
\pgfsetroundjoin%
\pgfsetlinewidth{2.007500pt}%
\definecolor{currentstroke}{rgb}{0.501961,0.501961,0.501961}%
\pgfsetstrokecolor{currentstroke}%
\pgfsetstrokeopacity{0.500000}%
\pgfsetdash{}{0pt}%
\pgfpathmoveto{\pgfqpoint{0.653453in}{4.829365in}}%
\pgfpathlineto{\pgfqpoint{0.694806in}{4.829365in}}%
\pgfusepath{stroke}%
\end{pgfscope}%
\begin{pgfscope}%
\pgfpathrectangle{\pgfqpoint{0.100000in}{2.802285in}}{\pgfqpoint{12.175956in}{4.509684in}}%
\pgfusepath{clip}%
\pgfsetbuttcap%
\pgfsetroundjoin%
\pgfsetlinewidth{2.007500pt}%
\definecolor{currentstroke}{rgb}{0.501961,0.501961,0.501961}%
\pgfsetstrokecolor{currentstroke}%
\pgfsetstrokeopacity{0.500000}%
\pgfsetdash{}{0pt}%
\pgfpathmoveto{\pgfqpoint{0.653453in}{5.284889in}}%
\pgfpathlineto{\pgfqpoint{0.694806in}{5.284889in}}%
\pgfusepath{stroke}%
\end{pgfscope}%
\begin{pgfscope}%
\pgfpathrectangle{\pgfqpoint{0.100000in}{2.802285in}}{\pgfqpoint{12.175956in}{4.509684in}}%
\pgfusepath{clip}%
\pgfsetbuttcap%
\pgfsetroundjoin%
\pgfsetlinewidth{2.007500pt}%
\definecolor{currentstroke}{rgb}{0.501961,0.501961,0.501961}%
\pgfsetstrokecolor{currentstroke}%
\pgfsetstrokeopacity{0.500000}%
\pgfsetdash{}{0pt}%
\pgfpathmoveto{\pgfqpoint{0.653453in}{5.740413in}}%
\pgfpathlineto{\pgfqpoint{0.694806in}{5.740413in}}%
\pgfusepath{stroke}%
\end{pgfscope}%
\begin{pgfscope}%
\pgfpathrectangle{\pgfqpoint{0.100000in}{2.802285in}}{\pgfqpoint{12.175956in}{4.509684in}}%
\pgfusepath{clip}%
\pgfsetbuttcap%
\pgfsetroundjoin%
\pgfsetlinewidth{2.007500pt}%
\definecolor{currentstroke}{rgb}{0.501961,0.501961,0.501961}%
\pgfsetstrokecolor{currentstroke}%
\pgfsetstrokeopacity{0.500000}%
\pgfsetdash{}{0pt}%
\pgfpathmoveto{\pgfqpoint{0.653453in}{6.195936in}}%
\pgfpathlineto{\pgfqpoint{0.694806in}{6.195936in}}%
\pgfusepath{stroke}%
\end{pgfscope}%
\begin{pgfscope}%
\pgfpathrectangle{\pgfqpoint{0.100000in}{2.802285in}}{\pgfqpoint{12.175956in}{4.509684in}}%
\pgfusepath{clip}%
\pgfsetbuttcap%
\pgfsetroundjoin%
\pgfsetlinewidth{2.007500pt}%
\definecolor{currentstroke}{rgb}{0.501961,0.501961,0.501961}%
\pgfsetstrokecolor{currentstroke}%
\pgfsetstrokeopacity{0.500000}%
\pgfsetdash{}{0pt}%
\pgfpathmoveto{\pgfqpoint{0.653453in}{6.651460in}}%
\pgfpathlineto{\pgfqpoint{0.694806in}{6.651460in}}%
\pgfusepath{stroke}%
\end{pgfscope}%
\begin{pgfscope}%
\pgfpathrectangle{\pgfqpoint{0.100000in}{2.802285in}}{\pgfqpoint{12.175956in}{4.509684in}}%
\pgfusepath{clip}%
\pgfsetbuttcap%
\pgfsetroundjoin%
\pgfsetlinewidth{2.007500pt}%
\definecolor{currentstroke}{rgb}{0.501961,0.501961,0.501961}%
\pgfsetstrokecolor{currentstroke}%
\pgfsetstrokeopacity{0.500000}%
\pgfsetdash{}{0pt}%
\pgfpathmoveto{\pgfqpoint{0.653453in}{7.106984in}}%
\pgfpathlineto{\pgfqpoint{0.694806in}{7.106984in}}%
\pgfusepath{stroke}%
\end{pgfscope}%
\begin{pgfscope}%
\pgfpathrectangle{\pgfqpoint{0.100000in}{2.802285in}}{\pgfqpoint{12.175956in}{4.509684in}}%
\pgfusepath{clip}%
\pgfsetrectcap%
\pgfsetroundjoin%
\pgfsetlinewidth{2.007500pt}%
\definecolor{currentstroke}{rgb}{0.501961,0.501961,0.501961}%
\pgfsetstrokecolor{currentstroke}%
\pgfsetstrokeopacity{0.500000}%
\pgfsetdash{}{0pt}%
\pgfpathmoveto{\pgfqpoint{1.363361in}{2.802285in}}%
\pgfpathlineto{\pgfqpoint{1.363361in}{7.311969in}}%
\pgfusepath{stroke}%
\end{pgfscope}%
\begin{pgfscope}%
\pgfpathrectangle{\pgfqpoint{0.100000in}{2.802285in}}{\pgfqpoint{12.175956in}{4.509684in}}%
\pgfusepath{clip}%
\pgfsetbuttcap%
\pgfsetroundjoin%
\pgfsetlinewidth{2.007500pt}%
\definecolor{currentstroke}{rgb}{0.501961,0.501961,0.501961}%
\pgfsetstrokecolor{currentstroke}%
\pgfsetstrokeopacity{0.500000}%
\pgfsetdash{}{0pt}%
\pgfpathmoveto{\pgfqpoint{1.342684in}{3.007271in}}%
\pgfpathlineto{\pgfqpoint{1.384038in}{3.007271in}}%
\pgfusepath{stroke}%
\end{pgfscope}%
\begin{pgfscope}%
\pgfpathrectangle{\pgfqpoint{0.100000in}{2.802285in}}{\pgfqpoint{12.175956in}{4.509684in}}%
\pgfusepath{clip}%
\pgfsetbuttcap%
\pgfsetroundjoin%
\pgfsetlinewidth{2.007500pt}%
\definecolor{currentstroke}{rgb}{0.501961,0.501961,0.501961}%
\pgfsetstrokecolor{currentstroke}%
\pgfsetstrokeopacity{0.500000}%
\pgfsetdash{}{0pt}%
\pgfpathmoveto{\pgfqpoint{1.342684in}{3.462794in}}%
\pgfpathlineto{\pgfqpoint{1.384038in}{3.462794in}}%
\pgfusepath{stroke}%
\end{pgfscope}%
\begin{pgfscope}%
\pgfpathrectangle{\pgfqpoint{0.100000in}{2.802285in}}{\pgfqpoint{12.175956in}{4.509684in}}%
\pgfusepath{clip}%
\pgfsetbuttcap%
\pgfsetroundjoin%
\pgfsetlinewidth{2.007500pt}%
\definecolor{currentstroke}{rgb}{0.501961,0.501961,0.501961}%
\pgfsetstrokecolor{currentstroke}%
\pgfsetstrokeopacity{0.500000}%
\pgfsetdash{}{0pt}%
\pgfpathmoveto{\pgfqpoint{1.342684in}{3.918318in}}%
\pgfpathlineto{\pgfqpoint{1.384038in}{3.918318in}}%
\pgfusepath{stroke}%
\end{pgfscope}%
\begin{pgfscope}%
\pgfpathrectangle{\pgfqpoint{0.100000in}{2.802285in}}{\pgfqpoint{12.175956in}{4.509684in}}%
\pgfusepath{clip}%
\pgfsetbuttcap%
\pgfsetroundjoin%
\pgfsetlinewidth{2.007500pt}%
\definecolor{currentstroke}{rgb}{0.501961,0.501961,0.501961}%
\pgfsetstrokecolor{currentstroke}%
\pgfsetstrokeopacity{0.500000}%
\pgfsetdash{}{0pt}%
\pgfpathmoveto{\pgfqpoint{1.342684in}{4.373842in}}%
\pgfpathlineto{\pgfqpoint{1.384038in}{4.373842in}}%
\pgfusepath{stroke}%
\end{pgfscope}%
\begin{pgfscope}%
\pgfpathrectangle{\pgfqpoint{0.100000in}{2.802285in}}{\pgfqpoint{12.175956in}{4.509684in}}%
\pgfusepath{clip}%
\pgfsetbuttcap%
\pgfsetroundjoin%
\pgfsetlinewidth{2.007500pt}%
\definecolor{currentstroke}{rgb}{0.501961,0.501961,0.501961}%
\pgfsetstrokecolor{currentstroke}%
\pgfsetstrokeopacity{0.500000}%
\pgfsetdash{}{0pt}%
\pgfpathmoveto{\pgfqpoint{1.342684in}{4.829365in}}%
\pgfpathlineto{\pgfqpoint{1.384038in}{4.829365in}}%
\pgfusepath{stroke}%
\end{pgfscope}%
\begin{pgfscope}%
\pgfpathrectangle{\pgfqpoint{0.100000in}{2.802285in}}{\pgfqpoint{12.175956in}{4.509684in}}%
\pgfusepath{clip}%
\pgfsetbuttcap%
\pgfsetroundjoin%
\pgfsetlinewidth{2.007500pt}%
\definecolor{currentstroke}{rgb}{0.501961,0.501961,0.501961}%
\pgfsetstrokecolor{currentstroke}%
\pgfsetstrokeopacity{0.500000}%
\pgfsetdash{}{0pt}%
\pgfpathmoveto{\pgfqpoint{1.342684in}{5.284889in}}%
\pgfpathlineto{\pgfqpoint{1.384038in}{5.284889in}}%
\pgfusepath{stroke}%
\end{pgfscope}%
\begin{pgfscope}%
\pgfpathrectangle{\pgfqpoint{0.100000in}{2.802285in}}{\pgfqpoint{12.175956in}{4.509684in}}%
\pgfusepath{clip}%
\pgfsetbuttcap%
\pgfsetroundjoin%
\pgfsetlinewidth{2.007500pt}%
\definecolor{currentstroke}{rgb}{0.501961,0.501961,0.501961}%
\pgfsetstrokecolor{currentstroke}%
\pgfsetstrokeopacity{0.500000}%
\pgfsetdash{}{0pt}%
\pgfpathmoveto{\pgfqpoint{1.342684in}{5.740413in}}%
\pgfpathlineto{\pgfqpoint{1.384038in}{5.740413in}}%
\pgfusepath{stroke}%
\end{pgfscope}%
\begin{pgfscope}%
\pgfpathrectangle{\pgfqpoint{0.100000in}{2.802285in}}{\pgfqpoint{12.175956in}{4.509684in}}%
\pgfusepath{clip}%
\pgfsetbuttcap%
\pgfsetroundjoin%
\pgfsetlinewidth{2.007500pt}%
\definecolor{currentstroke}{rgb}{0.501961,0.501961,0.501961}%
\pgfsetstrokecolor{currentstroke}%
\pgfsetstrokeopacity{0.500000}%
\pgfsetdash{}{0pt}%
\pgfpathmoveto{\pgfqpoint{1.342684in}{6.195936in}}%
\pgfpathlineto{\pgfqpoint{1.384038in}{6.195936in}}%
\pgfusepath{stroke}%
\end{pgfscope}%
\begin{pgfscope}%
\pgfpathrectangle{\pgfqpoint{0.100000in}{2.802285in}}{\pgfqpoint{12.175956in}{4.509684in}}%
\pgfusepath{clip}%
\pgfsetbuttcap%
\pgfsetroundjoin%
\pgfsetlinewidth{2.007500pt}%
\definecolor{currentstroke}{rgb}{0.501961,0.501961,0.501961}%
\pgfsetstrokecolor{currentstroke}%
\pgfsetstrokeopacity{0.500000}%
\pgfsetdash{}{0pt}%
\pgfpathmoveto{\pgfqpoint{1.342684in}{6.651460in}}%
\pgfpathlineto{\pgfqpoint{1.384038in}{6.651460in}}%
\pgfusepath{stroke}%
\end{pgfscope}%
\begin{pgfscope}%
\pgfpathrectangle{\pgfqpoint{0.100000in}{2.802285in}}{\pgfqpoint{12.175956in}{4.509684in}}%
\pgfusepath{clip}%
\pgfsetbuttcap%
\pgfsetroundjoin%
\pgfsetlinewidth{2.007500pt}%
\definecolor{currentstroke}{rgb}{0.501961,0.501961,0.501961}%
\pgfsetstrokecolor{currentstroke}%
\pgfsetstrokeopacity{0.500000}%
\pgfsetdash{}{0pt}%
\pgfpathmoveto{\pgfqpoint{1.342684in}{7.106984in}}%
\pgfpathlineto{\pgfqpoint{1.384038in}{7.106984in}}%
\pgfusepath{stroke}%
\end{pgfscope}%
\begin{pgfscope}%
\pgfpathrectangle{\pgfqpoint{0.100000in}{2.802285in}}{\pgfqpoint{12.175956in}{4.509684in}}%
\pgfusepath{clip}%
\pgfsetrectcap%
\pgfsetroundjoin%
\pgfsetlinewidth{2.007500pt}%
\definecolor{currentstroke}{rgb}{0.501961,0.501961,0.501961}%
\pgfsetstrokecolor{currentstroke}%
\pgfsetstrokeopacity{0.500000}%
\pgfsetdash{}{0pt}%
\pgfpathmoveto{\pgfqpoint{2.052592in}{2.802285in}}%
\pgfpathlineto{\pgfqpoint{2.052592in}{7.311969in}}%
\pgfusepath{stroke}%
\end{pgfscope}%
\begin{pgfscope}%
\pgfpathrectangle{\pgfqpoint{0.100000in}{2.802285in}}{\pgfqpoint{12.175956in}{4.509684in}}%
\pgfusepath{clip}%
\pgfsetbuttcap%
\pgfsetroundjoin%
\pgfsetlinewidth{2.007500pt}%
\definecolor{currentstroke}{rgb}{0.501961,0.501961,0.501961}%
\pgfsetstrokecolor{currentstroke}%
\pgfsetstrokeopacity{0.500000}%
\pgfsetdash{}{0pt}%
\pgfpathmoveto{\pgfqpoint{2.031915in}{3.007271in}}%
\pgfpathlineto{\pgfqpoint{2.073269in}{3.007271in}}%
\pgfusepath{stroke}%
\end{pgfscope}%
\begin{pgfscope}%
\pgfpathrectangle{\pgfqpoint{0.100000in}{2.802285in}}{\pgfqpoint{12.175956in}{4.509684in}}%
\pgfusepath{clip}%
\pgfsetbuttcap%
\pgfsetroundjoin%
\pgfsetlinewidth{2.007500pt}%
\definecolor{currentstroke}{rgb}{0.501961,0.501961,0.501961}%
\pgfsetstrokecolor{currentstroke}%
\pgfsetstrokeopacity{0.500000}%
\pgfsetdash{}{0pt}%
\pgfpathmoveto{\pgfqpoint{2.031915in}{3.462794in}}%
\pgfpathlineto{\pgfqpoint{2.073269in}{3.462794in}}%
\pgfusepath{stroke}%
\end{pgfscope}%
\begin{pgfscope}%
\pgfpathrectangle{\pgfqpoint{0.100000in}{2.802285in}}{\pgfqpoint{12.175956in}{4.509684in}}%
\pgfusepath{clip}%
\pgfsetbuttcap%
\pgfsetroundjoin%
\pgfsetlinewidth{2.007500pt}%
\definecolor{currentstroke}{rgb}{0.501961,0.501961,0.501961}%
\pgfsetstrokecolor{currentstroke}%
\pgfsetstrokeopacity{0.500000}%
\pgfsetdash{}{0pt}%
\pgfpathmoveto{\pgfqpoint{2.031915in}{3.918318in}}%
\pgfpathlineto{\pgfqpoint{2.073269in}{3.918318in}}%
\pgfusepath{stroke}%
\end{pgfscope}%
\begin{pgfscope}%
\pgfpathrectangle{\pgfqpoint{0.100000in}{2.802285in}}{\pgfqpoint{12.175956in}{4.509684in}}%
\pgfusepath{clip}%
\pgfsetbuttcap%
\pgfsetroundjoin%
\pgfsetlinewidth{2.007500pt}%
\definecolor{currentstroke}{rgb}{0.501961,0.501961,0.501961}%
\pgfsetstrokecolor{currentstroke}%
\pgfsetstrokeopacity{0.500000}%
\pgfsetdash{}{0pt}%
\pgfpathmoveto{\pgfqpoint{2.031915in}{4.373842in}}%
\pgfpathlineto{\pgfqpoint{2.073269in}{4.373842in}}%
\pgfusepath{stroke}%
\end{pgfscope}%
\begin{pgfscope}%
\pgfpathrectangle{\pgfqpoint{0.100000in}{2.802285in}}{\pgfqpoint{12.175956in}{4.509684in}}%
\pgfusepath{clip}%
\pgfsetbuttcap%
\pgfsetroundjoin%
\pgfsetlinewidth{2.007500pt}%
\definecolor{currentstroke}{rgb}{0.501961,0.501961,0.501961}%
\pgfsetstrokecolor{currentstroke}%
\pgfsetstrokeopacity{0.500000}%
\pgfsetdash{}{0pt}%
\pgfpathmoveto{\pgfqpoint{2.031915in}{4.829365in}}%
\pgfpathlineto{\pgfqpoint{2.073269in}{4.829365in}}%
\pgfusepath{stroke}%
\end{pgfscope}%
\begin{pgfscope}%
\pgfpathrectangle{\pgfqpoint{0.100000in}{2.802285in}}{\pgfqpoint{12.175956in}{4.509684in}}%
\pgfusepath{clip}%
\pgfsetbuttcap%
\pgfsetroundjoin%
\pgfsetlinewidth{2.007500pt}%
\definecolor{currentstroke}{rgb}{0.501961,0.501961,0.501961}%
\pgfsetstrokecolor{currentstroke}%
\pgfsetstrokeopacity{0.500000}%
\pgfsetdash{}{0pt}%
\pgfpathmoveto{\pgfqpoint{2.031915in}{5.284889in}}%
\pgfpathlineto{\pgfqpoint{2.073269in}{5.284889in}}%
\pgfusepath{stroke}%
\end{pgfscope}%
\begin{pgfscope}%
\pgfpathrectangle{\pgfqpoint{0.100000in}{2.802285in}}{\pgfqpoint{12.175956in}{4.509684in}}%
\pgfusepath{clip}%
\pgfsetbuttcap%
\pgfsetroundjoin%
\pgfsetlinewidth{2.007500pt}%
\definecolor{currentstroke}{rgb}{0.501961,0.501961,0.501961}%
\pgfsetstrokecolor{currentstroke}%
\pgfsetstrokeopacity{0.500000}%
\pgfsetdash{}{0pt}%
\pgfpathmoveto{\pgfqpoint{2.031915in}{5.740413in}}%
\pgfpathlineto{\pgfqpoint{2.073269in}{5.740413in}}%
\pgfusepath{stroke}%
\end{pgfscope}%
\begin{pgfscope}%
\pgfpathrectangle{\pgfqpoint{0.100000in}{2.802285in}}{\pgfqpoint{12.175956in}{4.509684in}}%
\pgfusepath{clip}%
\pgfsetbuttcap%
\pgfsetroundjoin%
\pgfsetlinewidth{2.007500pt}%
\definecolor{currentstroke}{rgb}{0.501961,0.501961,0.501961}%
\pgfsetstrokecolor{currentstroke}%
\pgfsetstrokeopacity{0.500000}%
\pgfsetdash{}{0pt}%
\pgfpathmoveto{\pgfqpoint{2.031915in}{6.195936in}}%
\pgfpathlineto{\pgfqpoint{2.073269in}{6.195936in}}%
\pgfusepath{stroke}%
\end{pgfscope}%
\begin{pgfscope}%
\pgfpathrectangle{\pgfqpoint{0.100000in}{2.802285in}}{\pgfqpoint{12.175956in}{4.509684in}}%
\pgfusepath{clip}%
\pgfsetbuttcap%
\pgfsetroundjoin%
\pgfsetlinewidth{2.007500pt}%
\definecolor{currentstroke}{rgb}{0.501961,0.501961,0.501961}%
\pgfsetstrokecolor{currentstroke}%
\pgfsetstrokeopacity{0.500000}%
\pgfsetdash{}{0pt}%
\pgfpathmoveto{\pgfqpoint{2.031915in}{6.651460in}}%
\pgfpathlineto{\pgfqpoint{2.073269in}{6.651460in}}%
\pgfusepath{stroke}%
\end{pgfscope}%
\begin{pgfscope}%
\pgfpathrectangle{\pgfqpoint{0.100000in}{2.802285in}}{\pgfqpoint{12.175956in}{4.509684in}}%
\pgfusepath{clip}%
\pgfsetbuttcap%
\pgfsetroundjoin%
\pgfsetlinewidth{2.007500pt}%
\definecolor{currentstroke}{rgb}{0.501961,0.501961,0.501961}%
\pgfsetstrokecolor{currentstroke}%
\pgfsetstrokeopacity{0.500000}%
\pgfsetdash{}{0pt}%
\pgfpathmoveto{\pgfqpoint{2.031915in}{7.106984in}}%
\pgfpathlineto{\pgfqpoint{2.073269in}{7.106984in}}%
\pgfusepath{stroke}%
\end{pgfscope}%
\begin{pgfscope}%
\pgfpathrectangle{\pgfqpoint{0.100000in}{2.802285in}}{\pgfqpoint{12.175956in}{4.509684in}}%
\pgfusepath{clip}%
\pgfsetrectcap%
\pgfsetroundjoin%
\pgfsetlinewidth{2.007500pt}%
\definecolor{currentstroke}{rgb}{0.501961,0.501961,0.501961}%
\pgfsetstrokecolor{currentstroke}%
\pgfsetstrokeopacity{0.500000}%
\pgfsetdash{}{0pt}%
\pgfpathmoveto{\pgfqpoint{2.741823in}{2.802285in}}%
\pgfpathlineto{\pgfqpoint{2.741823in}{7.311969in}}%
\pgfusepath{stroke}%
\end{pgfscope}%
\begin{pgfscope}%
\pgfpathrectangle{\pgfqpoint{0.100000in}{2.802285in}}{\pgfqpoint{12.175956in}{4.509684in}}%
\pgfusepath{clip}%
\pgfsetbuttcap%
\pgfsetroundjoin%
\pgfsetlinewidth{2.007500pt}%
\definecolor{currentstroke}{rgb}{0.501961,0.501961,0.501961}%
\pgfsetstrokecolor{currentstroke}%
\pgfsetstrokeopacity{0.500000}%
\pgfsetdash{}{0pt}%
\pgfpathmoveto{\pgfqpoint{2.721146in}{3.007271in}}%
\pgfpathlineto{\pgfqpoint{2.762500in}{3.007271in}}%
\pgfusepath{stroke}%
\end{pgfscope}%
\begin{pgfscope}%
\pgfpathrectangle{\pgfqpoint{0.100000in}{2.802285in}}{\pgfqpoint{12.175956in}{4.509684in}}%
\pgfusepath{clip}%
\pgfsetbuttcap%
\pgfsetroundjoin%
\pgfsetlinewidth{2.007500pt}%
\definecolor{currentstroke}{rgb}{0.501961,0.501961,0.501961}%
\pgfsetstrokecolor{currentstroke}%
\pgfsetstrokeopacity{0.500000}%
\pgfsetdash{}{0pt}%
\pgfpathmoveto{\pgfqpoint{2.721146in}{3.462794in}}%
\pgfpathlineto{\pgfqpoint{2.762500in}{3.462794in}}%
\pgfusepath{stroke}%
\end{pgfscope}%
\begin{pgfscope}%
\pgfpathrectangle{\pgfqpoint{0.100000in}{2.802285in}}{\pgfqpoint{12.175956in}{4.509684in}}%
\pgfusepath{clip}%
\pgfsetbuttcap%
\pgfsetroundjoin%
\pgfsetlinewidth{2.007500pt}%
\definecolor{currentstroke}{rgb}{0.501961,0.501961,0.501961}%
\pgfsetstrokecolor{currentstroke}%
\pgfsetstrokeopacity{0.500000}%
\pgfsetdash{}{0pt}%
\pgfpathmoveto{\pgfqpoint{2.721146in}{3.918318in}}%
\pgfpathlineto{\pgfqpoint{2.762500in}{3.918318in}}%
\pgfusepath{stroke}%
\end{pgfscope}%
\begin{pgfscope}%
\pgfpathrectangle{\pgfqpoint{0.100000in}{2.802285in}}{\pgfqpoint{12.175956in}{4.509684in}}%
\pgfusepath{clip}%
\pgfsetbuttcap%
\pgfsetroundjoin%
\pgfsetlinewidth{2.007500pt}%
\definecolor{currentstroke}{rgb}{0.501961,0.501961,0.501961}%
\pgfsetstrokecolor{currentstroke}%
\pgfsetstrokeopacity{0.500000}%
\pgfsetdash{}{0pt}%
\pgfpathmoveto{\pgfqpoint{2.721146in}{4.373842in}}%
\pgfpathlineto{\pgfqpoint{2.762500in}{4.373842in}}%
\pgfusepath{stroke}%
\end{pgfscope}%
\begin{pgfscope}%
\pgfpathrectangle{\pgfqpoint{0.100000in}{2.802285in}}{\pgfqpoint{12.175956in}{4.509684in}}%
\pgfusepath{clip}%
\pgfsetbuttcap%
\pgfsetroundjoin%
\pgfsetlinewidth{2.007500pt}%
\definecolor{currentstroke}{rgb}{0.501961,0.501961,0.501961}%
\pgfsetstrokecolor{currentstroke}%
\pgfsetstrokeopacity{0.500000}%
\pgfsetdash{}{0pt}%
\pgfpathmoveto{\pgfqpoint{2.721146in}{4.829365in}}%
\pgfpathlineto{\pgfqpoint{2.762500in}{4.829365in}}%
\pgfusepath{stroke}%
\end{pgfscope}%
\begin{pgfscope}%
\pgfpathrectangle{\pgfqpoint{0.100000in}{2.802285in}}{\pgfqpoint{12.175956in}{4.509684in}}%
\pgfusepath{clip}%
\pgfsetbuttcap%
\pgfsetroundjoin%
\pgfsetlinewidth{2.007500pt}%
\definecolor{currentstroke}{rgb}{0.501961,0.501961,0.501961}%
\pgfsetstrokecolor{currentstroke}%
\pgfsetstrokeopacity{0.500000}%
\pgfsetdash{}{0pt}%
\pgfpathmoveto{\pgfqpoint{2.721146in}{5.284889in}}%
\pgfpathlineto{\pgfqpoint{2.762500in}{5.284889in}}%
\pgfusepath{stroke}%
\end{pgfscope}%
\begin{pgfscope}%
\pgfpathrectangle{\pgfqpoint{0.100000in}{2.802285in}}{\pgfqpoint{12.175956in}{4.509684in}}%
\pgfusepath{clip}%
\pgfsetbuttcap%
\pgfsetroundjoin%
\pgfsetlinewidth{2.007500pt}%
\definecolor{currentstroke}{rgb}{0.501961,0.501961,0.501961}%
\pgfsetstrokecolor{currentstroke}%
\pgfsetstrokeopacity{0.500000}%
\pgfsetdash{}{0pt}%
\pgfpathmoveto{\pgfqpoint{2.721146in}{5.740413in}}%
\pgfpathlineto{\pgfqpoint{2.762500in}{5.740413in}}%
\pgfusepath{stroke}%
\end{pgfscope}%
\begin{pgfscope}%
\pgfpathrectangle{\pgfqpoint{0.100000in}{2.802285in}}{\pgfqpoint{12.175956in}{4.509684in}}%
\pgfusepath{clip}%
\pgfsetbuttcap%
\pgfsetroundjoin%
\pgfsetlinewidth{2.007500pt}%
\definecolor{currentstroke}{rgb}{0.501961,0.501961,0.501961}%
\pgfsetstrokecolor{currentstroke}%
\pgfsetstrokeopacity{0.500000}%
\pgfsetdash{}{0pt}%
\pgfpathmoveto{\pgfqpoint{2.721146in}{6.195936in}}%
\pgfpathlineto{\pgfqpoint{2.762500in}{6.195936in}}%
\pgfusepath{stroke}%
\end{pgfscope}%
\begin{pgfscope}%
\pgfpathrectangle{\pgfqpoint{0.100000in}{2.802285in}}{\pgfqpoint{12.175956in}{4.509684in}}%
\pgfusepath{clip}%
\pgfsetbuttcap%
\pgfsetroundjoin%
\pgfsetlinewidth{2.007500pt}%
\definecolor{currentstroke}{rgb}{0.501961,0.501961,0.501961}%
\pgfsetstrokecolor{currentstroke}%
\pgfsetstrokeopacity{0.500000}%
\pgfsetdash{}{0pt}%
\pgfpathmoveto{\pgfqpoint{2.721146in}{6.651460in}}%
\pgfpathlineto{\pgfqpoint{2.762500in}{6.651460in}}%
\pgfusepath{stroke}%
\end{pgfscope}%
\begin{pgfscope}%
\pgfpathrectangle{\pgfqpoint{0.100000in}{2.802285in}}{\pgfqpoint{12.175956in}{4.509684in}}%
\pgfusepath{clip}%
\pgfsetbuttcap%
\pgfsetroundjoin%
\pgfsetlinewidth{2.007500pt}%
\definecolor{currentstroke}{rgb}{0.501961,0.501961,0.501961}%
\pgfsetstrokecolor{currentstroke}%
\pgfsetstrokeopacity{0.500000}%
\pgfsetdash{}{0pt}%
\pgfpathmoveto{\pgfqpoint{2.721146in}{7.106984in}}%
\pgfpathlineto{\pgfqpoint{2.762500in}{7.106984in}}%
\pgfusepath{stroke}%
\end{pgfscope}%
\begin{pgfscope}%
\pgfpathrectangle{\pgfqpoint{0.100000in}{2.802285in}}{\pgfqpoint{12.175956in}{4.509684in}}%
\pgfusepath{clip}%
\pgfsetrectcap%
\pgfsetroundjoin%
\pgfsetlinewidth{2.007500pt}%
\definecolor{currentstroke}{rgb}{0.501961,0.501961,0.501961}%
\pgfsetstrokecolor{currentstroke}%
\pgfsetstrokeopacity{0.500000}%
\pgfsetdash{}{0pt}%
\pgfpathmoveto{\pgfqpoint{3.431054in}{2.802285in}}%
\pgfpathlineto{\pgfqpoint{3.431054in}{7.311969in}}%
\pgfusepath{stroke}%
\end{pgfscope}%
\begin{pgfscope}%
\pgfpathrectangle{\pgfqpoint{0.100000in}{2.802285in}}{\pgfqpoint{12.175956in}{4.509684in}}%
\pgfusepath{clip}%
\pgfsetbuttcap%
\pgfsetroundjoin%
\pgfsetlinewidth{2.007500pt}%
\definecolor{currentstroke}{rgb}{0.501961,0.501961,0.501961}%
\pgfsetstrokecolor{currentstroke}%
\pgfsetstrokeopacity{0.500000}%
\pgfsetdash{}{0pt}%
\pgfpathmoveto{\pgfqpoint{3.410377in}{3.007271in}}%
\pgfpathlineto{\pgfqpoint{3.451731in}{3.007271in}}%
\pgfusepath{stroke}%
\end{pgfscope}%
\begin{pgfscope}%
\pgfpathrectangle{\pgfqpoint{0.100000in}{2.802285in}}{\pgfqpoint{12.175956in}{4.509684in}}%
\pgfusepath{clip}%
\pgfsetbuttcap%
\pgfsetroundjoin%
\pgfsetlinewidth{2.007500pt}%
\definecolor{currentstroke}{rgb}{0.501961,0.501961,0.501961}%
\pgfsetstrokecolor{currentstroke}%
\pgfsetstrokeopacity{0.500000}%
\pgfsetdash{}{0pt}%
\pgfpathmoveto{\pgfqpoint{3.410377in}{3.462794in}}%
\pgfpathlineto{\pgfqpoint{3.451731in}{3.462794in}}%
\pgfusepath{stroke}%
\end{pgfscope}%
\begin{pgfscope}%
\pgfpathrectangle{\pgfqpoint{0.100000in}{2.802285in}}{\pgfqpoint{12.175956in}{4.509684in}}%
\pgfusepath{clip}%
\pgfsetbuttcap%
\pgfsetroundjoin%
\pgfsetlinewidth{2.007500pt}%
\definecolor{currentstroke}{rgb}{0.501961,0.501961,0.501961}%
\pgfsetstrokecolor{currentstroke}%
\pgfsetstrokeopacity{0.500000}%
\pgfsetdash{}{0pt}%
\pgfpathmoveto{\pgfqpoint{3.410377in}{3.918318in}}%
\pgfpathlineto{\pgfqpoint{3.451731in}{3.918318in}}%
\pgfusepath{stroke}%
\end{pgfscope}%
\begin{pgfscope}%
\pgfpathrectangle{\pgfqpoint{0.100000in}{2.802285in}}{\pgfqpoint{12.175956in}{4.509684in}}%
\pgfusepath{clip}%
\pgfsetbuttcap%
\pgfsetroundjoin%
\pgfsetlinewidth{2.007500pt}%
\definecolor{currentstroke}{rgb}{0.501961,0.501961,0.501961}%
\pgfsetstrokecolor{currentstroke}%
\pgfsetstrokeopacity{0.500000}%
\pgfsetdash{}{0pt}%
\pgfpathmoveto{\pgfqpoint{3.410377in}{4.373842in}}%
\pgfpathlineto{\pgfqpoint{3.451731in}{4.373842in}}%
\pgfusepath{stroke}%
\end{pgfscope}%
\begin{pgfscope}%
\pgfpathrectangle{\pgfqpoint{0.100000in}{2.802285in}}{\pgfqpoint{12.175956in}{4.509684in}}%
\pgfusepath{clip}%
\pgfsetbuttcap%
\pgfsetroundjoin%
\pgfsetlinewidth{2.007500pt}%
\definecolor{currentstroke}{rgb}{0.501961,0.501961,0.501961}%
\pgfsetstrokecolor{currentstroke}%
\pgfsetstrokeopacity{0.500000}%
\pgfsetdash{}{0pt}%
\pgfpathmoveto{\pgfqpoint{3.410377in}{4.829365in}}%
\pgfpathlineto{\pgfqpoint{3.451731in}{4.829365in}}%
\pgfusepath{stroke}%
\end{pgfscope}%
\begin{pgfscope}%
\pgfpathrectangle{\pgfqpoint{0.100000in}{2.802285in}}{\pgfqpoint{12.175956in}{4.509684in}}%
\pgfusepath{clip}%
\pgfsetbuttcap%
\pgfsetroundjoin%
\pgfsetlinewidth{2.007500pt}%
\definecolor{currentstroke}{rgb}{0.501961,0.501961,0.501961}%
\pgfsetstrokecolor{currentstroke}%
\pgfsetstrokeopacity{0.500000}%
\pgfsetdash{}{0pt}%
\pgfpathmoveto{\pgfqpoint{3.410377in}{5.284889in}}%
\pgfpathlineto{\pgfqpoint{3.451731in}{5.284889in}}%
\pgfusepath{stroke}%
\end{pgfscope}%
\begin{pgfscope}%
\pgfpathrectangle{\pgfqpoint{0.100000in}{2.802285in}}{\pgfqpoint{12.175956in}{4.509684in}}%
\pgfusepath{clip}%
\pgfsetbuttcap%
\pgfsetroundjoin%
\pgfsetlinewidth{2.007500pt}%
\definecolor{currentstroke}{rgb}{0.501961,0.501961,0.501961}%
\pgfsetstrokecolor{currentstroke}%
\pgfsetstrokeopacity{0.500000}%
\pgfsetdash{}{0pt}%
\pgfpathmoveto{\pgfqpoint{3.410377in}{5.740413in}}%
\pgfpathlineto{\pgfqpoint{3.451731in}{5.740413in}}%
\pgfusepath{stroke}%
\end{pgfscope}%
\begin{pgfscope}%
\pgfpathrectangle{\pgfqpoint{0.100000in}{2.802285in}}{\pgfqpoint{12.175956in}{4.509684in}}%
\pgfusepath{clip}%
\pgfsetbuttcap%
\pgfsetroundjoin%
\pgfsetlinewidth{2.007500pt}%
\definecolor{currentstroke}{rgb}{0.501961,0.501961,0.501961}%
\pgfsetstrokecolor{currentstroke}%
\pgfsetstrokeopacity{0.500000}%
\pgfsetdash{}{0pt}%
\pgfpathmoveto{\pgfqpoint{3.410377in}{6.195936in}}%
\pgfpathlineto{\pgfqpoint{3.451731in}{6.195936in}}%
\pgfusepath{stroke}%
\end{pgfscope}%
\begin{pgfscope}%
\pgfpathrectangle{\pgfqpoint{0.100000in}{2.802285in}}{\pgfqpoint{12.175956in}{4.509684in}}%
\pgfusepath{clip}%
\pgfsetbuttcap%
\pgfsetroundjoin%
\pgfsetlinewidth{2.007500pt}%
\definecolor{currentstroke}{rgb}{0.501961,0.501961,0.501961}%
\pgfsetstrokecolor{currentstroke}%
\pgfsetstrokeopacity{0.500000}%
\pgfsetdash{}{0pt}%
\pgfpathmoveto{\pgfqpoint{3.410377in}{6.651460in}}%
\pgfpathlineto{\pgfqpoint{3.451731in}{6.651460in}}%
\pgfusepath{stroke}%
\end{pgfscope}%
\begin{pgfscope}%
\pgfpathrectangle{\pgfqpoint{0.100000in}{2.802285in}}{\pgfqpoint{12.175956in}{4.509684in}}%
\pgfusepath{clip}%
\pgfsetbuttcap%
\pgfsetroundjoin%
\pgfsetlinewidth{2.007500pt}%
\definecolor{currentstroke}{rgb}{0.501961,0.501961,0.501961}%
\pgfsetstrokecolor{currentstroke}%
\pgfsetstrokeopacity{0.500000}%
\pgfsetdash{}{0pt}%
\pgfpathmoveto{\pgfqpoint{3.410377in}{7.106984in}}%
\pgfpathlineto{\pgfqpoint{3.451731in}{7.106984in}}%
\pgfusepath{stroke}%
\end{pgfscope}%
\begin{pgfscope}%
\pgfpathrectangle{\pgfqpoint{0.100000in}{2.802285in}}{\pgfqpoint{12.175956in}{4.509684in}}%
\pgfusepath{clip}%
\pgfsetrectcap%
\pgfsetroundjoin%
\pgfsetlinewidth{2.007500pt}%
\definecolor{currentstroke}{rgb}{0.501961,0.501961,0.501961}%
\pgfsetstrokecolor{currentstroke}%
\pgfsetstrokeopacity{0.500000}%
\pgfsetdash{}{0pt}%
\pgfpathmoveto{\pgfqpoint{4.120285in}{2.802285in}}%
\pgfpathlineto{\pgfqpoint{4.120285in}{7.311969in}}%
\pgfusepath{stroke}%
\end{pgfscope}%
\begin{pgfscope}%
\pgfpathrectangle{\pgfqpoint{0.100000in}{2.802285in}}{\pgfqpoint{12.175956in}{4.509684in}}%
\pgfusepath{clip}%
\pgfsetbuttcap%
\pgfsetroundjoin%
\pgfsetlinewidth{2.007500pt}%
\definecolor{currentstroke}{rgb}{0.501961,0.501961,0.501961}%
\pgfsetstrokecolor{currentstroke}%
\pgfsetstrokeopacity{0.500000}%
\pgfsetdash{}{0pt}%
\pgfpathmoveto{\pgfqpoint{4.099608in}{3.007271in}}%
\pgfpathlineto{\pgfqpoint{4.140962in}{3.007271in}}%
\pgfusepath{stroke}%
\end{pgfscope}%
\begin{pgfscope}%
\pgfpathrectangle{\pgfqpoint{0.100000in}{2.802285in}}{\pgfqpoint{12.175956in}{4.509684in}}%
\pgfusepath{clip}%
\pgfsetbuttcap%
\pgfsetroundjoin%
\pgfsetlinewidth{2.007500pt}%
\definecolor{currentstroke}{rgb}{0.501961,0.501961,0.501961}%
\pgfsetstrokecolor{currentstroke}%
\pgfsetstrokeopacity{0.500000}%
\pgfsetdash{}{0pt}%
\pgfpathmoveto{\pgfqpoint{4.099608in}{3.462794in}}%
\pgfpathlineto{\pgfqpoint{4.140962in}{3.462794in}}%
\pgfusepath{stroke}%
\end{pgfscope}%
\begin{pgfscope}%
\pgfpathrectangle{\pgfqpoint{0.100000in}{2.802285in}}{\pgfqpoint{12.175956in}{4.509684in}}%
\pgfusepath{clip}%
\pgfsetbuttcap%
\pgfsetroundjoin%
\pgfsetlinewidth{2.007500pt}%
\definecolor{currentstroke}{rgb}{0.501961,0.501961,0.501961}%
\pgfsetstrokecolor{currentstroke}%
\pgfsetstrokeopacity{0.500000}%
\pgfsetdash{}{0pt}%
\pgfpathmoveto{\pgfqpoint{4.099608in}{3.918318in}}%
\pgfpathlineto{\pgfqpoint{4.140962in}{3.918318in}}%
\pgfusepath{stroke}%
\end{pgfscope}%
\begin{pgfscope}%
\pgfpathrectangle{\pgfqpoint{0.100000in}{2.802285in}}{\pgfqpoint{12.175956in}{4.509684in}}%
\pgfusepath{clip}%
\pgfsetbuttcap%
\pgfsetroundjoin%
\pgfsetlinewidth{2.007500pt}%
\definecolor{currentstroke}{rgb}{0.501961,0.501961,0.501961}%
\pgfsetstrokecolor{currentstroke}%
\pgfsetstrokeopacity{0.500000}%
\pgfsetdash{}{0pt}%
\pgfpathmoveto{\pgfqpoint{4.099608in}{4.373842in}}%
\pgfpathlineto{\pgfqpoint{4.140962in}{4.373842in}}%
\pgfusepath{stroke}%
\end{pgfscope}%
\begin{pgfscope}%
\pgfpathrectangle{\pgfqpoint{0.100000in}{2.802285in}}{\pgfqpoint{12.175956in}{4.509684in}}%
\pgfusepath{clip}%
\pgfsetbuttcap%
\pgfsetroundjoin%
\pgfsetlinewidth{2.007500pt}%
\definecolor{currentstroke}{rgb}{0.501961,0.501961,0.501961}%
\pgfsetstrokecolor{currentstroke}%
\pgfsetstrokeopacity{0.500000}%
\pgfsetdash{}{0pt}%
\pgfpathmoveto{\pgfqpoint{4.099608in}{4.829365in}}%
\pgfpathlineto{\pgfqpoint{4.140962in}{4.829365in}}%
\pgfusepath{stroke}%
\end{pgfscope}%
\begin{pgfscope}%
\pgfpathrectangle{\pgfqpoint{0.100000in}{2.802285in}}{\pgfqpoint{12.175956in}{4.509684in}}%
\pgfusepath{clip}%
\pgfsetbuttcap%
\pgfsetroundjoin%
\pgfsetlinewidth{2.007500pt}%
\definecolor{currentstroke}{rgb}{0.501961,0.501961,0.501961}%
\pgfsetstrokecolor{currentstroke}%
\pgfsetstrokeopacity{0.500000}%
\pgfsetdash{}{0pt}%
\pgfpathmoveto{\pgfqpoint{4.099608in}{5.284889in}}%
\pgfpathlineto{\pgfqpoint{4.140962in}{5.284889in}}%
\pgfusepath{stroke}%
\end{pgfscope}%
\begin{pgfscope}%
\pgfpathrectangle{\pgfqpoint{0.100000in}{2.802285in}}{\pgfqpoint{12.175956in}{4.509684in}}%
\pgfusepath{clip}%
\pgfsetbuttcap%
\pgfsetroundjoin%
\pgfsetlinewidth{2.007500pt}%
\definecolor{currentstroke}{rgb}{0.501961,0.501961,0.501961}%
\pgfsetstrokecolor{currentstroke}%
\pgfsetstrokeopacity{0.500000}%
\pgfsetdash{}{0pt}%
\pgfpathmoveto{\pgfqpoint{4.099608in}{5.740413in}}%
\pgfpathlineto{\pgfqpoint{4.140962in}{5.740413in}}%
\pgfusepath{stroke}%
\end{pgfscope}%
\begin{pgfscope}%
\pgfpathrectangle{\pgfqpoint{0.100000in}{2.802285in}}{\pgfqpoint{12.175956in}{4.509684in}}%
\pgfusepath{clip}%
\pgfsetbuttcap%
\pgfsetroundjoin%
\pgfsetlinewidth{2.007500pt}%
\definecolor{currentstroke}{rgb}{0.501961,0.501961,0.501961}%
\pgfsetstrokecolor{currentstroke}%
\pgfsetstrokeopacity{0.500000}%
\pgfsetdash{}{0pt}%
\pgfpathmoveto{\pgfqpoint{4.099608in}{6.195936in}}%
\pgfpathlineto{\pgfqpoint{4.140962in}{6.195936in}}%
\pgfusepath{stroke}%
\end{pgfscope}%
\begin{pgfscope}%
\pgfpathrectangle{\pgfqpoint{0.100000in}{2.802285in}}{\pgfqpoint{12.175956in}{4.509684in}}%
\pgfusepath{clip}%
\pgfsetbuttcap%
\pgfsetroundjoin%
\pgfsetlinewidth{2.007500pt}%
\definecolor{currentstroke}{rgb}{0.501961,0.501961,0.501961}%
\pgfsetstrokecolor{currentstroke}%
\pgfsetstrokeopacity{0.500000}%
\pgfsetdash{}{0pt}%
\pgfpathmoveto{\pgfqpoint{4.099608in}{6.651460in}}%
\pgfpathlineto{\pgfqpoint{4.140962in}{6.651460in}}%
\pgfusepath{stroke}%
\end{pgfscope}%
\begin{pgfscope}%
\pgfpathrectangle{\pgfqpoint{0.100000in}{2.802285in}}{\pgfqpoint{12.175956in}{4.509684in}}%
\pgfusepath{clip}%
\pgfsetbuttcap%
\pgfsetroundjoin%
\pgfsetlinewidth{2.007500pt}%
\definecolor{currentstroke}{rgb}{0.501961,0.501961,0.501961}%
\pgfsetstrokecolor{currentstroke}%
\pgfsetstrokeopacity{0.500000}%
\pgfsetdash{}{0pt}%
\pgfpathmoveto{\pgfqpoint{4.099608in}{7.106984in}}%
\pgfpathlineto{\pgfqpoint{4.140962in}{7.106984in}}%
\pgfusepath{stroke}%
\end{pgfscope}%
\begin{pgfscope}%
\pgfpathrectangle{\pgfqpoint{0.100000in}{2.802285in}}{\pgfqpoint{12.175956in}{4.509684in}}%
\pgfusepath{clip}%
\pgfsetrectcap%
\pgfsetroundjoin%
\pgfsetlinewidth{2.007500pt}%
\definecolor{currentstroke}{rgb}{0.501961,0.501961,0.501961}%
\pgfsetstrokecolor{currentstroke}%
\pgfsetstrokeopacity{0.500000}%
\pgfsetdash{}{0pt}%
\pgfpathmoveto{\pgfqpoint{4.809516in}{2.802285in}}%
\pgfpathlineto{\pgfqpoint{4.809516in}{7.311969in}}%
\pgfusepath{stroke}%
\end{pgfscope}%
\begin{pgfscope}%
\pgfpathrectangle{\pgfqpoint{0.100000in}{2.802285in}}{\pgfqpoint{12.175956in}{4.509684in}}%
\pgfusepath{clip}%
\pgfsetbuttcap%
\pgfsetroundjoin%
\pgfsetlinewidth{2.007500pt}%
\definecolor{currentstroke}{rgb}{0.501961,0.501961,0.501961}%
\pgfsetstrokecolor{currentstroke}%
\pgfsetstrokeopacity{0.500000}%
\pgfsetdash{}{0pt}%
\pgfpathmoveto{\pgfqpoint{4.788839in}{3.007271in}}%
\pgfpathlineto{\pgfqpoint{4.830193in}{3.007271in}}%
\pgfusepath{stroke}%
\end{pgfscope}%
\begin{pgfscope}%
\pgfpathrectangle{\pgfqpoint{0.100000in}{2.802285in}}{\pgfqpoint{12.175956in}{4.509684in}}%
\pgfusepath{clip}%
\pgfsetbuttcap%
\pgfsetroundjoin%
\pgfsetlinewidth{2.007500pt}%
\definecolor{currentstroke}{rgb}{0.501961,0.501961,0.501961}%
\pgfsetstrokecolor{currentstroke}%
\pgfsetstrokeopacity{0.500000}%
\pgfsetdash{}{0pt}%
\pgfpathmoveto{\pgfqpoint{4.788839in}{3.462794in}}%
\pgfpathlineto{\pgfqpoint{4.830193in}{3.462794in}}%
\pgfusepath{stroke}%
\end{pgfscope}%
\begin{pgfscope}%
\pgfpathrectangle{\pgfqpoint{0.100000in}{2.802285in}}{\pgfqpoint{12.175956in}{4.509684in}}%
\pgfusepath{clip}%
\pgfsetbuttcap%
\pgfsetroundjoin%
\pgfsetlinewidth{2.007500pt}%
\definecolor{currentstroke}{rgb}{0.501961,0.501961,0.501961}%
\pgfsetstrokecolor{currentstroke}%
\pgfsetstrokeopacity{0.500000}%
\pgfsetdash{}{0pt}%
\pgfpathmoveto{\pgfqpoint{4.788839in}{3.918318in}}%
\pgfpathlineto{\pgfqpoint{4.830193in}{3.918318in}}%
\pgfusepath{stroke}%
\end{pgfscope}%
\begin{pgfscope}%
\pgfpathrectangle{\pgfqpoint{0.100000in}{2.802285in}}{\pgfqpoint{12.175956in}{4.509684in}}%
\pgfusepath{clip}%
\pgfsetbuttcap%
\pgfsetroundjoin%
\pgfsetlinewidth{2.007500pt}%
\definecolor{currentstroke}{rgb}{0.501961,0.501961,0.501961}%
\pgfsetstrokecolor{currentstroke}%
\pgfsetstrokeopacity{0.500000}%
\pgfsetdash{}{0pt}%
\pgfpathmoveto{\pgfqpoint{4.788839in}{4.373842in}}%
\pgfpathlineto{\pgfqpoint{4.830193in}{4.373842in}}%
\pgfusepath{stroke}%
\end{pgfscope}%
\begin{pgfscope}%
\pgfpathrectangle{\pgfqpoint{0.100000in}{2.802285in}}{\pgfqpoint{12.175956in}{4.509684in}}%
\pgfusepath{clip}%
\pgfsetbuttcap%
\pgfsetroundjoin%
\pgfsetlinewidth{2.007500pt}%
\definecolor{currentstroke}{rgb}{0.501961,0.501961,0.501961}%
\pgfsetstrokecolor{currentstroke}%
\pgfsetstrokeopacity{0.500000}%
\pgfsetdash{}{0pt}%
\pgfpathmoveto{\pgfqpoint{4.788839in}{4.829365in}}%
\pgfpathlineto{\pgfqpoint{4.830193in}{4.829365in}}%
\pgfusepath{stroke}%
\end{pgfscope}%
\begin{pgfscope}%
\pgfpathrectangle{\pgfqpoint{0.100000in}{2.802285in}}{\pgfqpoint{12.175956in}{4.509684in}}%
\pgfusepath{clip}%
\pgfsetbuttcap%
\pgfsetroundjoin%
\pgfsetlinewidth{2.007500pt}%
\definecolor{currentstroke}{rgb}{0.501961,0.501961,0.501961}%
\pgfsetstrokecolor{currentstroke}%
\pgfsetstrokeopacity{0.500000}%
\pgfsetdash{}{0pt}%
\pgfpathmoveto{\pgfqpoint{4.788839in}{5.284889in}}%
\pgfpathlineto{\pgfqpoint{4.830193in}{5.284889in}}%
\pgfusepath{stroke}%
\end{pgfscope}%
\begin{pgfscope}%
\pgfpathrectangle{\pgfqpoint{0.100000in}{2.802285in}}{\pgfqpoint{12.175956in}{4.509684in}}%
\pgfusepath{clip}%
\pgfsetbuttcap%
\pgfsetroundjoin%
\pgfsetlinewidth{2.007500pt}%
\definecolor{currentstroke}{rgb}{0.501961,0.501961,0.501961}%
\pgfsetstrokecolor{currentstroke}%
\pgfsetstrokeopacity{0.500000}%
\pgfsetdash{}{0pt}%
\pgfpathmoveto{\pgfqpoint{4.788839in}{5.740413in}}%
\pgfpathlineto{\pgfqpoint{4.830193in}{5.740413in}}%
\pgfusepath{stroke}%
\end{pgfscope}%
\begin{pgfscope}%
\pgfpathrectangle{\pgfqpoint{0.100000in}{2.802285in}}{\pgfqpoint{12.175956in}{4.509684in}}%
\pgfusepath{clip}%
\pgfsetbuttcap%
\pgfsetroundjoin%
\pgfsetlinewidth{2.007500pt}%
\definecolor{currentstroke}{rgb}{0.501961,0.501961,0.501961}%
\pgfsetstrokecolor{currentstroke}%
\pgfsetstrokeopacity{0.500000}%
\pgfsetdash{}{0pt}%
\pgfpathmoveto{\pgfqpoint{4.788839in}{6.195936in}}%
\pgfpathlineto{\pgfqpoint{4.830193in}{6.195936in}}%
\pgfusepath{stroke}%
\end{pgfscope}%
\begin{pgfscope}%
\pgfpathrectangle{\pgfqpoint{0.100000in}{2.802285in}}{\pgfqpoint{12.175956in}{4.509684in}}%
\pgfusepath{clip}%
\pgfsetbuttcap%
\pgfsetroundjoin%
\pgfsetlinewidth{2.007500pt}%
\definecolor{currentstroke}{rgb}{0.501961,0.501961,0.501961}%
\pgfsetstrokecolor{currentstroke}%
\pgfsetstrokeopacity{0.500000}%
\pgfsetdash{}{0pt}%
\pgfpathmoveto{\pgfqpoint{4.788839in}{6.651460in}}%
\pgfpathlineto{\pgfqpoint{4.830193in}{6.651460in}}%
\pgfusepath{stroke}%
\end{pgfscope}%
\begin{pgfscope}%
\pgfpathrectangle{\pgfqpoint{0.100000in}{2.802285in}}{\pgfqpoint{12.175956in}{4.509684in}}%
\pgfusepath{clip}%
\pgfsetbuttcap%
\pgfsetroundjoin%
\pgfsetlinewidth{2.007500pt}%
\definecolor{currentstroke}{rgb}{0.501961,0.501961,0.501961}%
\pgfsetstrokecolor{currentstroke}%
\pgfsetstrokeopacity{0.500000}%
\pgfsetdash{}{0pt}%
\pgfpathmoveto{\pgfqpoint{4.788839in}{7.106984in}}%
\pgfpathlineto{\pgfqpoint{4.830193in}{7.106984in}}%
\pgfusepath{stroke}%
\end{pgfscope}%
\begin{pgfscope}%
\pgfpathrectangle{\pgfqpoint{0.100000in}{2.802285in}}{\pgfqpoint{12.175956in}{4.509684in}}%
\pgfusepath{clip}%
\pgfsetrectcap%
\pgfsetroundjoin%
\pgfsetlinewidth{2.007500pt}%
\definecolor{currentstroke}{rgb}{0.501961,0.501961,0.501961}%
\pgfsetstrokecolor{currentstroke}%
\pgfsetstrokeopacity{0.500000}%
\pgfsetdash{}{0pt}%
\pgfpathmoveto{\pgfqpoint{5.498747in}{2.802285in}}%
\pgfpathlineto{\pgfqpoint{5.498747in}{7.311969in}}%
\pgfusepath{stroke}%
\end{pgfscope}%
\begin{pgfscope}%
\pgfpathrectangle{\pgfqpoint{0.100000in}{2.802285in}}{\pgfqpoint{12.175956in}{4.509684in}}%
\pgfusepath{clip}%
\pgfsetbuttcap%
\pgfsetroundjoin%
\pgfsetlinewidth{2.007500pt}%
\definecolor{currentstroke}{rgb}{0.501961,0.501961,0.501961}%
\pgfsetstrokecolor{currentstroke}%
\pgfsetstrokeopacity{0.500000}%
\pgfsetdash{}{0pt}%
\pgfpathmoveto{\pgfqpoint{5.478070in}{3.007271in}}%
\pgfpathlineto{\pgfqpoint{5.519424in}{3.007271in}}%
\pgfusepath{stroke}%
\end{pgfscope}%
\begin{pgfscope}%
\pgfpathrectangle{\pgfqpoint{0.100000in}{2.802285in}}{\pgfqpoint{12.175956in}{4.509684in}}%
\pgfusepath{clip}%
\pgfsetbuttcap%
\pgfsetroundjoin%
\pgfsetlinewidth{2.007500pt}%
\definecolor{currentstroke}{rgb}{0.501961,0.501961,0.501961}%
\pgfsetstrokecolor{currentstroke}%
\pgfsetstrokeopacity{0.500000}%
\pgfsetdash{}{0pt}%
\pgfpathmoveto{\pgfqpoint{5.478070in}{3.462794in}}%
\pgfpathlineto{\pgfqpoint{5.519424in}{3.462794in}}%
\pgfusepath{stroke}%
\end{pgfscope}%
\begin{pgfscope}%
\pgfpathrectangle{\pgfqpoint{0.100000in}{2.802285in}}{\pgfqpoint{12.175956in}{4.509684in}}%
\pgfusepath{clip}%
\pgfsetbuttcap%
\pgfsetroundjoin%
\pgfsetlinewidth{2.007500pt}%
\definecolor{currentstroke}{rgb}{0.501961,0.501961,0.501961}%
\pgfsetstrokecolor{currentstroke}%
\pgfsetstrokeopacity{0.500000}%
\pgfsetdash{}{0pt}%
\pgfpathmoveto{\pgfqpoint{5.478070in}{3.918318in}}%
\pgfpathlineto{\pgfqpoint{5.519424in}{3.918318in}}%
\pgfusepath{stroke}%
\end{pgfscope}%
\begin{pgfscope}%
\pgfpathrectangle{\pgfqpoint{0.100000in}{2.802285in}}{\pgfqpoint{12.175956in}{4.509684in}}%
\pgfusepath{clip}%
\pgfsetbuttcap%
\pgfsetroundjoin%
\pgfsetlinewidth{2.007500pt}%
\definecolor{currentstroke}{rgb}{0.501961,0.501961,0.501961}%
\pgfsetstrokecolor{currentstroke}%
\pgfsetstrokeopacity{0.500000}%
\pgfsetdash{}{0pt}%
\pgfpathmoveto{\pgfqpoint{5.478070in}{4.373842in}}%
\pgfpathlineto{\pgfqpoint{5.519424in}{4.373842in}}%
\pgfusepath{stroke}%
\end{pgfscope}%
\begin{pgfscope}%
\pgfpathrectangle{\pgfqpoint{0.100000in}{2.802285in}}{\pgfqpoint{12.175956in}{4.509684in}}%
\pgfusepath{clip}%
\pgfsetbuttcap%
\pgfsetroundjoin%
\pgfsetlinewidth{2.007500pt}%
\definecolor{currentstroke}{rgb}{0.501961,0.501961,0.501961}%
\pgfsetstrokecolor{currentstroke}%
\pgfsetstrokeopacity{0.500000}%
\pgfsetdash{}{0pt}%
\pgfpathmoveto{\pgfqpoint{5.478070in}{4.829365in}}%
\pgfpathlineto{\pgfqpoint{5.519424in}{4.829365in}}%
\pgfusepath{stroke}%
\end{pgfscope}%
\begin{pgfscope}%
\pgfpathrectangle{\pgfqpoint{0.100000in}{2.802285in}}{\pgfqpoint{12.175956in}{4.509684in}}%
\pgfusepath{clip}%
\pgfsetbuttcap%
\pgfsetroundjoin%
\pgfsetlinewidth{2.007500pt}%
\definecolor{currentstroke}{rgb}{0.501961,0.501961,0.501961}%
\pgfsetstrokecolor{currentstroke}%
\pgfsetstrokeopacity{0.500000}%
\pgfsetdash{}{0pt}%
\pgfpathmoveto{\pgfqpoint{5.478070in}{5.284889in}}%
\pgfpathlineto{\pgfqpoint{5.519424in}{5.284889in}}%
\pgfusepath{stroke}%
\end{pgfscope}%
\begin{pgfscope}%
\pgfpathrectangle{\pgfqpoint{0.100000in}{2.802285in}}{\pgfqpoint{12.175956in}{4.509684in}}%
\pgfusepath{clip}%
\pgfsetbuttcap%
\pgfsetroundjoin%
\pgfsetlinewidth{2.007500pt}%
\definecolor{currentstroke}{rgb}{0.501961,0.501961,0.501961}%
\pgfsetstrokecolor{currentstroke}%
\pgfsetstrokeopacity{0.500000}%
\pgfsetdash{}{0pt}%
\pgfpathmoveto{\pgfqpoint{5.478070in}{5.740413in}}%
\pgfpathlineto{\pgfqpoint{5.519424in}{5.740413in}}%
\pgfusepath{stroke}%
\end{pgfscope}%
\begin{pgfscope}%
\pgfpathrectangle{\pgfqpoint{0.100000in}{2.802285in}}{\pgfqpoint{12.175956in}{4.509684in}}%
\pgfusepath{clip}%
\pgfsetbuttcap%
\pgfsetroundjoin%
\pgfsetlinewidth{2.007500pt}%
\definecolor{currentstroke}{rgb}{0.501961,0.501961,0.501961}%
\pgfsetstrokecolor{currentstroke}%
\pgfsetstrokeopacity{0.500000}%
\pgfsetdash{}{0pt}%
\pgfpathmoveto{\pgfqpoint{5.478070in}{6.195936in}}%
\pgfpathlineto{\pgfqpoint{5.519424in}{6.195936in}}%
\pgfusepath{stroke}%
\end{pgfscope}%
\begin{pgfscope}%
\pgfpathrectangle{\pgfqpoint{0.100000in}{2.802285in}}{\pgfqpoint{12.175956in}{4.509684in}}%
\pgfusepath{clip}%
\pgfsetbuttcap%
\pgfsetroundjoin%
\pgfsetlinewidth{2.007500pt}%
\definecolor{currentstroke}{rgb}{0.501961,0.501961,0.501961}%
\pgfsetstrokecolor{currentstroke}%
\pgfsetstrokeopacity{0.500000}%
\pgfsetdash{}{0pt}%
\pgfpathmoveto{\pgfqpoint{5.478070in}{6.651460in}}%
\pgfpathlineto{\pgfqpoint{5.519424in}{6.651460in}}%
\pgfusepath{stroke}%
\end{pgfscope}%
\begin{pgfscope}%
\pgfpathrectangle{\pgfqpoint{0.100000in}{2.802285in}}{\pgfqpoint{12.175956in}{4.509684in}}%
\pgfusepath{clip}%
\pgfsetbuttcap%
\pgfsetroundjoin%
\pgfsetlinewidth{2.007500pt}%
\definecolor{currentstroke}{rgb}{0.501961,0.501961,0.501961}%
\pgfsetstrokecolor{currentstroke}%
\pgfsetstrokeopacity{0.500000}%
\pgfsetdash{}{0pt}%
\pgfpathmoveto{\pgfqpoint{5.478070in}{7.106984in}}%
\pgfpathlineto{\pgfqpoint{5.519424in}{7.106984in}}%
\pgfusepath{stroke}%
\end{pgfscope}%
\begin{pgfscope}%
\pgfpathrectangle{\pgfqpoint{0.100000in}{2.802285in}}{\pgfqpoint{12.175956in}{4.509684in}}%
\pgfusepath{clip}%
\pgfsetrectcap%
\pgfsetroundjoin%
\pgfsetlinewidth{2.007500pt}%
\definecolor{currentstroke}{rgb}{0.501961,0.501961,0.501961}%
\pgfsetstrokecolor{currentstroke}%
\pgfsetstrokeopacity{0.500000}%
\pgfsetdash{}{0pt}%
\pgfpathmoveto{\pgfqpoint{6.187978in}{2.802285in}}%
\pgfpathlineto{\pgfqpoint{6.187978in}{7.311969in}}%
\pgfusepath{stroke}%
\end{pgfscope}%
\begin{pgfscope}%
\pgfpathrectangle{\pgfqpoint{0.100000in}{2.802285in}}{\pgfqpoint{12.175956in}{4.509684in}}%
\pgfusepath{clip}%
\pgfsetbuttcap%
\pgfsetroundjoin%
\pgfsetlinewidth{2.007500pt}%
\definecolor{currentstroke}{rgb}{0.501961,0.501961,0.501961}%
\pgfsetstrokecolor{currentstroke}%
\pgfsetstrokeopacity{0.500000}%
\pgfsetdash{}{0pt}%
\pgfpathmoveto{\pgfqpoint{6.167301in}{3.007271in}}%
\pgfpathlineto{\pgfqpoint{6.208655in}{3.007271in}}%
\pgfusepath{stroke}%
\end{pgfscope}%
\begin{pgfscope}%
\pgfpathrectangle{\pgfqpoint{0.100000in}{2.802285in}}{\pgfqpoint{12.175956in}{4.509684in}}%
\pgfusepath{clip}%
\pgfsetbuttcap%
\pgfsetroundjoin%
\pgfsetlinewidth{2.007500pt}%
\definecolor{currentstroke}{rgb}{0.501961,0.501961,0.501961}%
\pgfsetstrokecolor{currentstroke}%
\pgfsetstrokeopacity{0.500000}%
\pgfsetdash{}{0pt}%
\pgfpathmoveto{\pgfqpoint{6.167301in}{3.462794in}}%
\pgfpathlineto{\pgfqpoint{6.208655in}{3.462794in}}%
\pgfusepath{stroke}%
\end{pgfscope}%
\begin{pgfscope}%
\pgfpathrectangle{\pgfqpoint{0.100000in}{2.802285in}}{\pgfqpoint{12.175956in}{4.509684in}}%
\pgfusepath{clip}%
\pgfsetbuttcap%
\pgfsetroundjoin%
\pgfsetlinewidth{2.007500pt}%
\definecolor{currentstroke}{rgb}{0.501961,0.501961,0.501961}%
\pgfsetstrokecolor{currentstroke}%
\pgfsetstrokeopacity{0.500000}%
\pgfsetdash{}{0pt}%
\pgfpathmoveto{\pgfqpoint{6.167301in}{3.918318in}}%
\pgfpathlineto{\pgfqpoint{6.208655in}{3.918318in}}%
\pgfusepath{stroke}%
\end{pgfscope}%
\begin{pgfscope}%
\pgfpathrectangle{\pgfqpoint{0.100000in}{2.802285in}}{\pgfqpoint{12.175956in}{4.509684in}}%
\pgfusepath{clip}%
\pgfsetbuttcap%
\pgfsetroundjoin%
\pgfsetlinewidth{2.007500pt}%
\definecolor{currentstroke}{rgb}{0.501961,0.501961,0.501961}%
\pgfsetstrokecolor{currentstroke}%
\pgfsetstrokeopacity{0.500000}%
\pgfsetdash{}{0pt}%
\pgfpathmoveto{\pgfqpoint{6.167301in}{4.373842in}}%
\pgfpathlineto{\pgfqpoint{6.208655in}{4.373842in}}%
\pgfusepath{stroke}%
\end{pgfscope}%
\begin{pgfscope}%
\pgfpathrectangle{\pgfqpoint{0.100000in}{2.802285in}}{\pgfqpoint{12.175956in}{4.509684in}}%
\pgfusepath{clip}%
\pgfsetbuttcap%
\pgfsetroundjoin%
\pgfsetlinewidth{2.007500pt}%
\definecolor{currentstroke}{rgb}{0.501961,0.501961,0.501961}%
\pgfsetstrokecolor{currentstroke}%
\pgfsetstrokeopacity{0.500000}%
\pgfsetdash{}{0pt}%
\pgfpathmoveto{\pgfqpoint{6.167301in}{4.829365in}}%
\pgfpathlineto{\pgfqpoint{6.208655in}{4.829365in}}%
\pgfusepath{stroke}%
\end{pgfscope}%
\begin{pgfscope}%
\pgfpathrectangle{\pgfqpoint{0.100000in}{2.802285in}}{\pgfqpoint{12.175956in}{4.509684in}}%
\pgfusepath{clip}%
\pgfsetbuttcap%
\pgfsetroundjoin%
\pgfsetlinewidth{2.007500pt}%
\definecolor{currentstroke}{rgb}{0.501961,0.501961,0.501961}%
\pgfsetstrokecolor{currentstroke}%
\pgfsetstrokeopacity{0.500000}%
\pgfsetdash{}{0pt}%
\pgfpathmoveto{\pgfqpoint{6.167301in}{5.284889in}}%
\pgfpathlineto{\pgfqpoint{6.208655in}{5.284889in}}%
\pgfusepath{stroke}%
\end{pgfscope}%
\begin{pgfscope}%
\pgfpathrectangle{\pgfqpoint{0.100000in}{2.802285in}}{\pgfqpoint{12.175956in}{4.509684in}}%
\pgfusepath{clip}%
\pgfsetbuttcap%
\pgfsetroundjoin%
\pgfsetlinewidth{2.007500pt}%
\definecolor{currentstroke}{rgb}{0.501961,0.501961,0.501961}%
\pgfsetstrokecolor{currentstroke}%
\pgfsetstrokeopacity{0.500000}%
\pgfsetdash{}{0pt}%
\pgfpathmoveto{\pgfqpoint{6.167301in}{5.740413in}}%
\pgfpathlineto{\pgfqpoint{6.208655in}{5.740413in}}%
\pgfusepath{stroke}%
\end{pgfscope}%
\begin{pgfscope}%
\pgfpathrectangle{\pgfqpoint{0.100000in}{2.802285in}}{\pgfqpoint{12.175956in}{4.509684in}}%
\pgfusepath{clip}%
\pgfsetbuttcap%
\pgfsetroundjoin%
\pgfsetlinewidth{2.007500pt}%
\definecolor{currentstroke}{rgb}{0.501961,0.501961,0.501961}%
\pgfsetstrokecolor{currentstroke}%
\pgfsetstrokeopacity{0.500000}%
\pgfsetdash{}{0pt}%
\pgfpathmoveto{\pgfqpoint{6.167301in}{6.195936in}}%
\pgfpathlineto{\pgfqpoint{6.208655in}{6.195936in}}%
\pgfusepath{stroke}%
\end{pgfscope}%
\begin{pgfscope}%
\pgfpathrectangle{\pgfqpoint{0.100000in}{2.802285in}}{\pgfqpoint{12.175956in}{4.509684in}}%
\pgfusepath{clip}%
\pgfsetbuttcap%
\pgfsetroundjoin%
\pgfsetlinewidth{2.007500pt}%
\definecolor{currentstroke}{rgb}{0.501961,0.501961,0.501961}%
\pgfsetstrokecolor{currentstroke}%
\pgfsetstrokeopacity{0.500000}%
\pgfsetdash{}{0pt}%
\pgfpathmoveto{\pgfqpoint{6.167301in}{6.651460in}}%
\pgfpathlineto{\pgfqpoint{6.208655in}{6.651460in}}%
\pgfusepath{stroke}%
\end{pgfscope}%
\begin{pgfscope}%
\pgfpathrectangle{\pgfqpoint{0.100000in}{2.802285in}}{\pgfqpoint{12.175956in}{4.509684in}}%
\pgfusepath{clip}%
\pgfsetbuttcap%
\pgfsetroundjoin%
\pgfsetlinewidth{2.007500pt}%
\definecolor{currentstroke}{rgb}{0.501961,0.501961,0.501961}%
\pgfsetstrokecolor{currentstroke}%
\pgfsetstrokeopacity{0.500000}%
\pgfsetdash{}{0pt}%
\pgfpathmoveto{\pgfqpoint{6.167301in}{7.106984in}}%
\pgfpathlineto{\pgfqpoint{6.208655in}{7.106984in}}%
\pgfusepath{stroke}%
\end{pgfscope}%
\begin{pgfscope}%
\pgfpathrectangle{\pgfqpoint{0.100000in}{2.802285in}}{\pgfqpoint{12.175956in}{4.509684in}}%
\pgfusepath{clip}%
\pgfsetrectcap%
\pgfsetroundjoin%
\pgfsetlinewidth{2.007500pt}%
\definecolor{currentstroke}{rgb}{0.501961,0.501961,0.501961}%
\pgfsetstrokecolor{currentstroke}%
\pgfsetstrokeopacity{0.500000}%
\pgfsetdash{}{0pt}%
\pgfpathmoveto{\pgfqpoint{6.877209in}{2.802285in}}%
\pgfpathlineto{\pgfqpoint{6.877209in}{7.311969in}}%
\pgfusepath{stroke}%
\end{pgfscope}%
\begin{pgfscope}%
\pgfpathrectangle{\pgfqpoint{0.100000in}{2.802285in}}{\pgfqpoint{12.175956in}{4.509684in}}%
\pgfusepath{clip}%
\pgfsetbuttcap%
\pgfsetroundjoin%
\pgfsetlinewidth{2.007500pt}%
\definecolor{currentstroke}{rgb}{0.501961,0.501961,0.501961}%
\pgfsetstrokecolor{currentstroke}%
\pgfsetstrokeopacity{0.500000}%
\pgfsetdash{}{0pt}%
\pgfpathmoveto{\pgfqpoint{6.856532in}{3.007271in}}%
\pgfpathlineto{\pgfqpoint{6.897886in}{3.007271in}}%
\pgfusepath{stroke}%
\end{pgfscope}%
\begin{pgfscope}%
\pgfpathrectangle{\pgfqpoint{0.100000in}{2.802285in}}{\pgfqpoint{12.175956in}{4.509684in}}%
\pgfusepath{clip}%
\pgfsetbuttcap%
\pgfsetroundjoin%
\pgfsetlinewidth{2.007500pt}%
\definecolor{currentstroke}{rgb}{0.501961,0.501961,0.501961}%
\pgfsetstrokecolor{currentstroke}%
\pgfsetstrokeopacity{0.500000}%
\pgfsetdash{}{0pt}%
\pgfpathmoveto{\pgfqpoint{6.856532in}{3.462794in}}%
\pgfpathlineto{\pgfqpoint{6.897886in}{3.462794in}}%
\pgfusepath{stroke}%
\end{pgfscope}%
\begin{pgfscope}%
\pgfpathrectangle{\pgfqpoint{0.100000in}{2.802285in}}{\pgfqpoint{12.175956in}{4.509684in}}%
\pgfusepath{clip}%
\pgfsetbuttcap%
\pgfsetroundjoin%
\pgfsetlinewidth{2.007500pt}%
\definecolor{currentstroke}{rgb}{0.501961,0.501961,0.501961}%
\pgfsetstrokecolor{currentstroke}%
\pgfsetstrokeopacity{0.500000}%
\pgfsetdash{}{0pt}%
\pgfpathmoveto{\pgfqpoint{6.856532in}{3.918318in}}%
\pgfpathlineto{\pgfqpoint{6.897886in}{3.918318in}}%
\pgfusepath{stroke}%
\end{pgfscope}%
\begin{pgfscope}%
\pgfpathrectangle{\pgfqpoint{0.100000in}{2.802285in}}{\pgfqpoint{12.175956in}{4.509684in}}%
\pgfusepath{clip}%
\pgfsetbuttcap%
\pgfsetroundjoin%
\pgfsetlinewidth{2.007500pt}%
\definecolor{currentstroke}{rgb}{0.501961,0.501961,0.501961}%
\pgfsetstrokecolor{currentstroke}%
\pgfsetstrokeopacity{0.500000}%
\pgfsetdash{}{0pt}%
\pgfpathmoveto{\pgfqpoint{6.856532in}{4.373842in}}%
\pgfpathlineto{\pgfqpoint{6.897886in}{4.373842in}}%
\pgfusepath{stroke}%
\end{pgfscope}%
\begin{pgfscope}%
\pgfpathrectangle{\pgfqpoint{0.100000in}{2.802285in}}{\pgfqpoint{12.175956in}{4.509684in}}%
\pgfusepath{clip}%
\pgfsetbuttcap%
\pgfsetroundjoin%
\pgfsetlinewidth{2.007500pt}%
\definecolor{currentstroke}{rgb}{0.501961,0.501961,0.501961}%
\pgfsetstrokecolor{currentstroke}%
\pgfsetstrokeopacity{0.500000}%
\pgfsetdash{}{0pt}%
\pgfpathmoveto{\pgfqpoint{6.856532in}{4.829365in}}%
\pgfpathlineto{\pgfqpoint{6.897886in}{4.829365in}}%
\pgfusepath{stroke}%
\end{pgfscope}%
\begin{pgfscope}%
\pgfpathrectangle{\pgfqpoint{0.100000in}{2.802285in}}{\pgfqpoint{12.175956in}{4.509684in}}%
\pgfusepath{clip}%
\pgfsetbuttcap%
\pgfsetroundjoin%
\pgfsetlinewidth{2.007500pt}%
\definecolor{currentstroke}{rgb}{0.501961,0.501961,0.501961}%
\pgfsetstrokecolor{currentstroke}%
\pgfsetstrokeopacity{0.500000}%
\pgfsetdash{}{0pt}%
\pgfpathmoveto{\pgfqpoint{6.856532in}{5.284889in}}%
\pgfpathlineto{\pgfqpoint{6.897886in}{5.284889in}}%
\pgfusepath{stroke}%
\end{pgfscope}%
\begin{pgfscope}%
\pgfpathrectangle{\pgfqpoint{0.100000in}{2.802285in}}{\pgfqpoint{12.175956in}{4.509684in}}%
\pgfusepath{clip}%
\pgfsetbuttcap%
\pgfsetroundjoin%
\pgfsetlinewidth{2.007500pt}%
\definecolor{currentstroke}{rgb}{0.501961,0.501961,0.501961}%
\pgfsetstrokecolor{currentstroke}%
\pgfsetstrokeopacity{0.500000}%
\pgfsetdash{}{0pt}%
\pgfpathmoveto{\pgfqpoint{6.856532in}{5.740413in}}%
\pgfpathlineto{\pgfqpoint{6.897886in}{5.740413in}}%
\pgfusepath{stroke}%
\end{pgfscope}%
\begin{pgfscope}%
\pgfpathrectangle{\pgfqpoint{0.100000in}{2.802285in}}{\pgfqpoint{12.175956in}{4.509684in}}%
\pgfusepath{clip}%
\pgfsetbuttcap%
\pgfsetroundjoin%
\pgfsetlinewidth{2.007500pt}%
\definecolor{currentstroke}{rgb}{0.501961,0.501961,0.501961}%
\pgfsetstrokecolor{currentstroke}%
\pgfsetstrokeopacity{0.500000}%
\pgfsetdash{}{0pt}%
\pgfpathmoveto{\pgfqpoint{6.856532in}{6.195936in}}%
\pgfpathlineto{\pgfqpoint{6.897886in}{6.195936in}}%
\pgfusepath{stroke}%
\end{pgfscope}%
\begin{pgfscope}%
\pgfpathrectangle{\pgfqpoint{0.100000in}{2.802285in}}{\pgfqpoint{12.175956in}{4.509684in}}%
\pgfusepath{clip}%
\pgfsetbuttcap%
\pgfsetroundjoin%
\pgfsetlinewidth{2.007500pt}%
\definecolor{currentstroke}{rgb}{0.501961,0.501961,0.501961}%
\pgfsetstrokecolor{currentstroke}%
\pgfsetstrokeopacity{0.500000}%
\pgfsetdash{}{0pt}%
\pgfpathmoveto{\pgfqpoint{6.856532in}{6.651460in}}%
\pgfpathlineto{\pgfqpoint{6.897886in}{6.651460in}}%
\pgfusepath{stroke}%
\end{pgfscope}%
\begin{pgfscope}%
\pgfpathrectangle{\pgfqpoint{0.100000in}{2.802285in}}{\pgfqpoint{12.175956in}{4.509684in}}%
\pgfusepath{clip}%
\pgfsetbuttcap%
\pgfsetroundjoin%
\pgfsetlinewidth{2.007500pt}%
\definecolor{currentstroke}{rgb}{0.501961,0.501961,0.501961}%
\pgfsetstrokecolor{currentstroke}%
\pgfsetstrokeopacity{0.500000}%
\pgfsetdash{}{0pt}%
\pgfpathmoveto{\pgfqpoint{6.856532in}{7.106984in}}%
\pgfpathlineto{\pgfqpoint{6.897886in}{7.106984in}}%
\pgfusepath{stroke}%
\end{pgfscope}%
\begin{pgfscope}%
\pgfpathrectangle{\pgfqpoint{0.100000in}{2.802285in}}{\pgfqpoint{12.175956in}{4.509684in}}%
\pgfusepath{clip}%
\pgfsetrectcap%
\pgfsetroundjoin%
\pgfsetlinewidth{2.007500pt}%
\definecolor{currentstroke}{rgb}{0.501961,0.501961,0.501961}%
\pgfsetstrokecolor{currentstroke}%
\pgfsetstrokeopacity{0.500000}%
\pgfsetdash{}{0pt}%
\pgfpathmoveto{\pgfqpoint{7.566440in}{2.802285in}}%
\pgfpathlineto{\pgfqpoint{7.566440in}{7.311969in}}%
\pgfusepath{stroke}%
\end{pgfscope}%
\begin{pgfscope}%
\pgfpathrectangle{\pgfqpoint{0.100000in}{2.802285in}}{\pgfqpoint{12.175956in}{4.509684in}}%
\pgfusepath{clip}%
\pgfsetbuttcap%
\pgfsetroundjoin%
\pgfsetlinewidth{2.007500pt}%
\definecolor{currentstroke}{rgb}{0.501961,0.501961,0.501961}%
\pgfsetstrokecolor{currentstroke}%
\pgfsetstrokeopacity{0.500000}%
\pgfsetdash{}{0pt}%
\pgfpathmoveto{\pgfqpoint{7.545763in}{3.007271in}}%
\pgfpathlineto{\pgfqpoint{7.587117in}{3.007271in}}%
\pgfusepath{stroke}%
\end{pgfscope}%
\begin{pgfscope}%
\pgfpathrectangle{\pgfqpoint{0.100000in}{2.802285in}}{\pgfqpoint{12.175956in}{4.509684in}}%
\pgfusepath{clip}%
\pgfsetbuttcap%
\pgfsetroundjoin%
\pgfsetlinewidth{2.007500pt}%
\definecolor{currentstroke}{rgb}{0.501961,0.501961,0.501961}%
\pgfsetstrokecolor{currentstroke}%
\pgfsetstrokeopacity{0.500000}%
\pgfsetdash{}{0pt}%
\pgfpathmoveto{\pgfqpoint{7.545763in}{3.462794in}}%
\pgfpathlineto{\pgfqpoint{7.587117in}{3.462794in}}%
\pgfusepath{stroke}%
\end{pgfscope}%
\begin{pgfscope}%
\pgfpathrectangle{\pgfqpoint{0.100000in}{2.802285in}}{\pgfqpoint{12.175956in}{4.509684in}}%
\pgfusepath{clip}%
\pgfsetbuttcap%
\pgfsetroundjoin%
\pgfsetlinewidth{2.007500pt}%
\definecolor{currentstroke}{rgb}{0.501961,0.501961,0.501961}%
\pgfsetstrokecolor{currentstroke}%
\pgfsetstrokeopacity{0.500000}%
\pgfsetdash{}{0pt}%
\pgfpathmoveto{\pgfqpoint{7.545763in}{3.918318in}}%
\pgfpathlineto{\pgfqpoint{7.587117in}{3.918318in}}%
\pgfusepath{stroke}%
\end{pgfscope}%
\begin{pgfscope}%
\pgfpathrectangle{\pgfqpoint{0.100000in}{2.802285in}}{\pgfqpoint{12.175956in}{4.509684in}}%
\pgfusepath{clip}%
\pgfsetbuttcap%
\pgfsetroundjoin%
\pgfsetlinewidth{2.007500pt}%
\definecolor{currentstroke}{rgb}{0.501961,0.501961,0.501961}%
\pgfsetstrokecolor{currentstroke}%
\pgfsetstrokeopacity{0.500000}%
\pgfsetdash{}{0pt}%
\pgfpathmoveto{\pgfqpoint{7.545763in}{4.373842in}}%
\pgfpathlineto{\pgfqpoint{7.587117in}{4.373842in}}%
\pgfusepath{stroke}%
\end{pgfscope}%
\begin{pgfscope}%
\pgfpathrectangle{\pgfqpoint{0.100000in}{2.802285in}}{\pgfqpoint{12.175956in}{4.509684in}}%
\pgfusepath{clip}%
\pgfsetbuttcap%
\pgfsetroundjoin%
\pgfsetlinewidth{2.007500pt}%
\definecolor{currentstroke}{rgb}{0.501961,0.501961,0.501961}%
\pgfsetstrokecolor{currentstroke}%
\pgfsetstrokeopacity{0.500000}%
\pgfsetdash{}{0pt}%
\pgfpathmoveto{\pgfqpoint{7.545763in}{4.829365in}}%
\pgfpathlineto{\pgfqpoint{7.587117in}{4.829365in}}%
\pgfusepath{stroke}%
\end{pgfscope}%
\begin{pgfscope}%
\pgfpathrectangle{\pgfqpoint{0.100000in}{2.802285in}}{\pgfqpoint{12.175956in}{4.509684in}}%
\pgfusepath{clip}%
\pgfsetbuttcap%
\pgfsetroundjoin%
\pgfsetlinewidth{2.007500pt}%
\definecolor{currentstroke}{rgb}{0.501961,0.501961,0.501961}%
\pgfsetstrokecolor{currentstroke}%
\pgfsetstrokeopacity{0.500000}%
\pgfsetdash{}{0pt}%
\pgfpathmoveto{\pgfqpoint{7.545763in}{5.284889in}}%
\pgfpathlineto{\pgfqpoint{7.587117in}{5.284889in}}%
\pgfusepath{stroke}%
\end{pgfscope}%
\begin{pgfscope}%
\pgfpathrectangle{\pgfqpoint{0.100000in}{2.802285in}}{\pgfqpoint{12.175956in}{4.509684in}}%
\pgfusepath{clip}%
\pgfsetbuttcap%
\pgfsetroundjoin%
\pgfsetlinewidth{2.007500pt}%
\definecolor{currentstroke}{rgb}{0.501961,0.501961,0.501961}%
\pgfsetstrokecolor{currentstroke}%
\pgfsetstrokeopacity{0.500000}%
\pgfsetdash{}{0pt}%
\pgfpathmoveto{\pgfqpoint{7.545763in}{5.740413in}}%
\pgfpathlineto{\pgfqpoint{7.587117in}{5.740413in}}%
\pgfusepath{stroke}%
\end{pgfscope}%
\begin{pgfscope}%
\pgfpathrectangle{\pgfqpoint{0.100000in}{2.802285in}}{\pgfqpoint{12.175956in}{4.509684in}}%
\pgfusepath{clip}%
\pgfsetbuttcap%
\pgfsetroundjoin%
\pgfsetlinewidth{2.007500pt}%
\definecolor{currentstroke}{rgb}{0.501961,0.501961,0.501961}%
\pgfsetstrokecolor{currentstroke}%
\pgfsetstrokeopacity{0.500000}%
\pgfsetdash{}{0pt}%
\pgfpathmoveto{\pgfqpoint{7.545763in}{6.195936in}}%
\pgfpathlineto{\pgfqpoint{7.587117in}{6.195936in}}%
\pgfusepath{stroke}%
\end{pgfscope}%
\begin{pgfscope}%
\pgfpathrectangle{\pgfqpoint{0.100000in}{2.802285in}}{\pgfqpoint{12.175956in}{4.509684in}}%
\pgfusepath{clip}%
\pgfsetbuttcap%
\pgfsetroundjoin%
\pgfsetlinewidth{2.007500pt}%
\definecolor{currentstroke}{rgb}{0.501961,0.501961,0.501961}%
\pgfsetstrokecolor{currentstroke}%
\pgfsetstrokeopacity{0.500000}%
\pgfsetdash{}{0pt}%
\pgfpathmoveto{\pgfqpoint{7.545763in}{6.651460in}}%
\pgfpathlineto{\pgfqpoint{7.587117in}{6.651460in}}%
\pgfusepath{stroke}%
\end{pgfscope}%
\begin{pgfscope}%
\pgfpathrectangle{\pgfqpoint{0.100000in}{2.802285in}}{\pgfqpoint{12.175956in}{4.509684in}}%
\pgfusepath{clip}%
\pgfsetbuttcap%
\pgfsetroundjoin%
\pgfsetlinewidth{2.007500pt}%
\definecolor{currentstroke}{rgb}{0.501961,0.501961,0.501961}%
\pgfsetstrokecolor{currentstroke}%
\pgfsetstrokeopacity{0.500000}%
\pgfsetdash{}{0pt}%
\pgfpathmoveto{\pgfqpoint{7.545763in}{7.106984in}}%
\pgfpathlineto{\pgfqpoint{7.587117in}{7.106984in}}%
\pgfusepath{stroke}%
\end{pgfscope}%
\begin{pgfscope}%
\pgfpathrectangle{\pgfqpoint{0.100000in}{2.802285in}}{\pgfqpoint{12.175956in}{4.509684in}}%
\pgfusepath{clip}%
\pgfsetrectcap%
\pgfsetroundjoin%
\pgfsetlinewidth{2.007500pt}%
\definecolor{currentstroke}{rgb}{0.501961,0.501961,0.501961}%
\pgfsetstrokecolor{currentstroke}%
\pgfsetstrokeopacity{0.500000}%
\pgfsetdash{}{0pt}%
\pgfpathmoveto{\pgfqpoint{8.255671in}{2.802285in}}%
\pgfpathlineto{\pgfqpoint{8.255671in}{7.311969in}}%
\pgfusepath{stroke}%
\end{pgfscope}%
\begin{pgfscope}%
\pgfpathrectangle{\pgfqpoint{0.100000in}{2.802285in}}{\pgfqpoint{12.175956in}{4.509684in}}%
\pgfusepath{clip}%
\pgfsetbuttcap%
\pgfsetroundjoin%
\pgfsetlinewidth{2.007500pt}%
\definecolor{currentstroke}{rgb}{0.501961,0.501961,0.501961}%
\pgfsetstrokecolor{currentstroke}%
\pgfsetstrokeopacity{0.500000}%
\pgfsetdash{}{0pt}%
\pgfpathmoveto{\pgfqpoint{8.234994in}{3.007271in}}%
\pgfpathlineto{\pgfqpoint{8.276348in}{3.007271in}}%
\pgfusepath{stroke}%
\end{pgfscope}%
\begin{pgfscope}%
\pgfpathrectangle{\pgfqpoint{0.100000in}{2.802285in}}{\pgfqpoint{12.175956in}{4.509684in}}%
\pgfusepath{clip}%
\pgfsetbuttcap%
\pgfsetroundjoin%
\pgfsetlinewidth{2.007500pt}%
\definecolor{currentstroke}{rgb}{0.501961,0.501961,0.501961}%
\pgfsetstrokecolor{currentstroke}%
\pgfsetstrokeopacity{0.500000}%
\pgfsetdash{}{0pt}%
\pgfpathmoveto{\pgfqpoint{8.234994in}{3.462794in}}%
\pgfpathlineto{\pgfqpoint{8.276348in}{3.462794in}}%
\pgfusepath{stroke}%
\end{pgfscope}%
\begin{pgfscope}%
\pgfpathrectangle{\pgfqpoint{0.100000in}{2.802285in}}{\pgfqpoint{12.175956in}{4.509684in}}%
\pgfusepath{clip}%
\pgfsetbuttcap%
\pgfsetroundjoin%
\pgfsetlinewidth{2.007500pt}%
\definecolor{currentstroke}{rgb}{0.501961,0.501961,0.501961}%
\pgfsetstrokecolor{currentstroke}%
\pgfsetstrokeopacity{0.500000}%
\pgfsetdash{}{0pt}%
\pgfpathmoveto{\pgfqpoint{8.234994in}{3.918318in}}%
\pgfpathlineto{\pgfqpoint{8.276348in}{3.918318in}}%
\pgfusepath{stroke}%
\end{pgfscope}%
\begin{pgfscope}%
\pgfpathrectangle{\pgfqpoint{0.100000in}{2.802285in}}{\pgfqpoint{12.175956in}{4.509684in}}%
\pgfusepath{clip}%
\pgfsetbuttcap%
\pgfsetroundjoin%
\pgfsetlinewidth{2.007500pt}%
\definecolor{currentstroke}{rgb}{0.501961,0.501961,0.501961}%
\pgfsetstrokecolor{currentstroke}%
\pgfsetstrokeopacity{0.500000}%
\pgfsetdash{}{0pt}%
\pgfpathmoveto{\pgfqpoint{8.234994in}{4.373842in}}%
\pgfpathlineto{\pgfqpoint{8.276348in}{4.373842in}}%
\pgfusepath{stroke}%
\end{pgfscope}%
\begin{pgfscope}%
\pgfpathrectangle{\pgfqpoint{0.100000in}{2.802285in}}{\pgfqpoint{12.175956in}{4.509684in}}%
\pgfusepath{clip}%
\pgfsetbuttcap%
\pgfsetroundjoin%
\pgfsetlinewidth{2.007500pt}%
\definecolor{currentstroke}{rgb}{0.501961,0.501961,0.501961}%
\pgfsetstrokecolor{currentstroke}%
\pgfsetstrokeopacity{0.500000}%
\pgfsetdash{}{0pt}%
\pgfpathmoveto{\pgfqpoint{8.234994in}{4.829365in}}%
\pgfpathlineto{\pgfqpoint{8.276348in}{4.829365in}}%
\pgfusepath{stroke}%
\end{pgfscope}%
\begin{pgfscope}%
\pgfpathrectangle{\pgfqpoint{0.100000in}{2.802285in}}{\pgfqpoint{12.175956in}{4.509684in}}%
\pgfusepath{clip}%
\pgfsetbuttcap%
\pgfsetroundjoin%
\pgfsetlinewidth{2.007500pt}%
\definecolor{currentstroke}{rgb}{0.501961,0.501961,0.501961}%
\pgfsetstrokecolor{currentstroke}%
\pgfsetstrokeopacity{0.500000}%
\pgfsetdash{}{0pt}%
\pgfpathmoveto{\pgfqpoint{8.234994in}{5.284889in}}%
\pgfpathlineto{\pgfqpoint{8.276348in}{5.284889in}}%
\pgfusepath{stroke}%
\end{pgfscope}%
\begin{pgfscope}%
\pgfpathrectangle{\pgfqpoint{0.100000in}{2.802285in}}{\pgfqpoint{12.175956in}{4.509684in}}%
\pgfusepath{clip}%
\pgfsetbuttcap%
\pgfsetroundjoin%
\pgfsetlinewidth{2.007500pt}%
\definecolor{currentstroke}{rgb}{0.501961,0.501961,0.501961}%
\pgfsetstrokecolor{currentstroke}%
\pgfsetstrokeopacity{0.500000}%
\pgfsetdash{}{0pt}%
\pgfpathmoveto{\pgfqpoint{8.234994in}{5.740413in}}%
\pgfpathlineto{\pgfqpoint{8.276348in}{5.740413in}}%
\pgfusepath{stroke}%
\end{pgfscope}%
\begin{pgfscope}%
\pgfpathrectangle{\pgfqpoint{0.100000in}{2.802285in}}{\pgfqpoint{12.175956in}{4.509684in}}%
\pgfusepath{clip}%
\pgfsetbuttcap%
\pgfsetroundjoin%
\pgfsetlinewidth{2.007500pt}%
\definecolor{currentstroke}{rgb}{0.501961,0.501961,0.501961}%
\pgfsetstrokecolor{currentstroke}%
\pgfsetstrokeopacity{0.500000}%
\pgfsetdash{}{0pt}%
\pgfpathmoveto{\pgfqpoint{8.234994in}{6.195936in}}%
\pgfpathlineto{\pgfqpoint{8.276348in}{6.195936in}}%
\pgfusepath{stroke}%
\end{pgfscope}%
\begin{pgfscope}%
\pgfpathrectangle{\pgfqpoint{0.100000in}{2.802285in}}{\pgfqpoint{12.175956in}{4.509684in}}%
\pgfusepath{clip}%
\pgfsetbuttcap%
\pgfsetroundjoin%
\pgfsetlinewidth{2.007500pt}%
\definecolor{currentstroke}{rgb}{0.501961,0.501961,0.501961}%
\pgfsetstrokecolor{currentstroke}%
\pgfsetstrokeopacity{0.500000}%
\pgfsetdash{}{0pt}%
\pgfpathmoveto{\pgfqpoint{8.234994in}{6.651460in}}%
\pgfpathlineto{\pgfqpoint{8.276348in}{6.651460in}}%
\pgfusepath{stroke}%
\end{pgfscope}%
\begin{pgfscope}%
\pgfpathrectangle{\pgfqpoint{0.100000in}{2.802285in}}{\pgfqpoint{12.175956in}{4.509684in}}%
\pgfusepath{clip}%
\pgfsetbuttcap%
\pgfsetroundjoin%
\pgfsetlinewidth{2.007500pt}%
\definecolor{currentstroke}{rgb}{0.501961,0.501961,0.501961}%
\pgfsetstrokecolor{currentstroke}%
\pgfsetstrokeopacity{0.500000}%
\pgfsetdash{}{0pt}%
\pgfpathmoveto{\pgfqpoint{8.234994in}{7.106984in}}%
\pgfpathlineto{\pgfqpoint{8.276348in}{7.106984in}}%
\pgfusepath{stroke}%
\end{pgfscope}%
\begin{pgfscope}%
\pgfpathrectangle{\pgfqpoint{0.100000in}{2.802285in}}{\pgfqpoint{12.175956in}{4.509684in}}%
\pgfusepath{clip}%
\pgfsetrectcap%
\pgfsetroundjoin%
\pgfsetlinewidth{2.007500pt}%
\definecolor{currentstroke}{rgb}{0.501961,0.501961,0.501961}%
\pgfsetstrokecolor{currentstroke}%
\pgfsetstrokeopacity{0.500000}%
\pgfsetdash{}{0pt}%
\pgfpathmoveto{\pgfqpoint{8.944902in}{2.802285in}}%
\pgfpathlineto{\pgfqpoint{8.944902in}{7.311969in}}%
\pgfusepath{stroke}%
\end{pgfscope}%
\begin{pgfscope}%
\pgfpathrectangle{\pgfqpoint{0.100000in}{2.802285in}}{\pgfqpoint{12.175956in}{4.509684in}}%
\pgfusepath{clip}%
\pgfsetbuttcap%
\pgfsetroundjoin%
\pgfsetlinewidth{2.007500pt}%
\definecolor{currentstroke}{rgb}{0.501961,0.501961,0.501961}%
\pgfsetstrokecolor{currentstroke}%
\pgfsetstrokeopacity{0.500000}%
\pgfsetdash{}{0pt}%
\pgfpathmoveto{\pgfqpoint{8.924226in}{3.007271in}}%
\pgfpathlineto{\pgfqpoint{8.965579in}{3.007271in}}%
\pgfusepath{stroke}%
\end{pgfscope}%
\begin{pgfscope}%
\pgfpathrectangle{\pgfqpoint{0.100000in}{2.802285in}}{\pgfqpoint{12.175956in}{4.509684in}}%
\pgfusepath{clip}%
\pgfsetbuttcap%
\pgfsetroundjoin%
\pgfsetlinewidth{2.007500pt}%
\definecolor{currentstroke}{rgb}{0.501961,0.501961,0.501961}%
\pgfsetstrokecolor{currentstroke}%
\pgfsetstrokeopacity{0.500000}%
\pgfsetdash{}{0pt}%
\pgfpathmoveto{\pgfqpoint{8.924226in}{3.462794in}}%
\pgfpathlineto{\pgfqpoint{8.965579in}{3.462794in}}%
\pgfusepath{stroke}%
\end{pgfscope}%
\begin{pgfscope}%
\pgfpathrectangle{\pgfqpoint{0.100000in}{2.802285in}}{\pgfqpoint{12.175956in}{4.509684in}}%
\pgfusepath{clip}%
\pgfsetbuttcap%
\pgfsetroundjoin%
\pgfsetlinewidth{2.007500pt}%
\definecolor{currentstroke}{rgb}{0.501961,0.501961,0.501961}%
\pgfsetstrokecolor{currentstroke}%
\pgfsetstrokeopacity{0.500000}%
\pgfsetdash{}{0pt}%
\pgfpathmoveto{\pgfqpoint{8.924226in}{3.918318in}}%
\pgfpathlineto{\pgfqpoint{8.965579in}{3.918318in}}%
\pgfusepath{stroke}%
\end{pgfscope}%
\begin{pgfscope}%
\pgfpathrectangle{\pgfqpoint{0.100000in}{2.802285in}}{\pgfqpoint{12.175956in}{4.509684in}}%
\pgfusepath{clip}%
\pgfsetbuttcap%
\pgfsetroundjoin%
\pgfsetlinewidth{2.007500pt}%
\definecolor{currentstroke}{rgb}{0.501961,0.501961,0.501961}%
\pgfsetstrokecolor{currentstroke}%
\pgfsetstrokeopacity{0.500000}%
\pgfsetdash{}{0pt}%
\pgfpathmoveto{\pgfqpoint{8.924226in}{4.373842in}}%
\pgfpathlineto{\pgfqpoint{8.965579in}{4.373842in}}%
\pgfusepath{stroke}%
\end{pgfscope}%
\begin{pgfscope}%
\pgfpathrectangle{\pgfqpoint{0.100000in}{2.802285in}}{\pgfqpoint{12.175956in}{4.509684in}}%
\pgfusepath{clip}%
\pgfsetbuttcap%
\pgfsetroundjoin%
\pgfsetlinewidth{2.007500pt}%
\definecolor{currentstroke}{rgb}{0.501961,0.501961,0.501961}%
\pgfsetstrokecolor{currentstroke}%
\pgfsetstrokeopacity{0.500000}%
\pgfsetdash{}{0pt}%
\pgfpathmoveto{\pgfqpoint{8.924226in}{4.829365in}}%
\pgfpathlineto{\pgfqpoint{8.965579in}{4.829365in}}%
\pgfusepath{stroke}%
\end{pgfscope}%
\begin{pgfscope}%
\pgfpathrectangle{\pgfqpoint{0.100000in}{2.802285in}}{\pgfqpoint{12.175956in}{4.509684in}}%
\pgfusepath{clip}%
\pgfsetbuttcap%
\pgfsetroundjoin%
\pgfsetlinewidth{2.007500pt}%
\definecolor{currentstroke}{rgb}{0.501961,0.501961,0.501961}%
\pgfsetstrokecolor{currentstroke}%
\pgfsetstrokeopacity{0.500000}%
\pgfsetdash{}{0pt}%
\pgfpathmoveto{\pgfqpoint{8.924226in}{5.284889in}}%
\pgfpathlineto{\pgfqpoint{8.965579in}{5.284889in}}%
\pgfusepath{stroke}%
\end{pgfscope}%
\begin{pgfscope}%
\pgfpathrectangle{\pgfqpoint{0.100000in}{2.802285in}}{\pgfqpoint{12.175956in}{4.509684in}}%
\pgfusepath{clip}%
\pgfsetbuttcap%
\pgfsetroundjoin%
\pgfsetlinewidth{2.007500pt}%
\definecolor{currentstroke}{rgb}{0.501961,0.501961,0.501961}%
\pgfsetstrokecolor{currentstroke}%
\pgfsetstrokeopacity{0.500000}%
\pgfsetdash{}{0pt}%
\pgfpathmoveto{\pgfqpoint{8.924226in}{5.740413in}}%
\pgfpathlineto{\pgfqpoint{8.965579in}{5.740413in}}%
\pgfusepath{stroke}%
\end{pgfscope}%
\begin{pgfscope}%
\pgfpathrectangle{\pgfqpoint{0.100000in}{2.802285in}}{\pgfqpoint{12.175956in}{4.509684in}}%
\pgfusepath{clip}%
\pgfsetbuttcap%
\pgfsetroundjoin%
\pgfsetlinewidth{2.007500pt}%
\definecolor{currentstroke}{rgb}{0.501961,0.501961,0.501961}%
\pgfsetstrokecolor{currentstroke}%
\pgfsetstrokeopacity{0.500000}%
\pgfsetdash{}{0pt}%
\pgfpathmoveto{\pgfqpoint{8.924226in}{6.195936in}}%
\pgfpathlineto{\pgfqpoint{8.965579in}{6.195936in}}%
\pgfusepath{stroke}%
\end{pgfscope}%
\begin{pgfscope}%
\pgfpathrectangle{\pgfqpoint{0.100000in}{2.802285in}}{\pgfqpoint{12.175956in}{4.509684in}}%
\pgfusepath{clip}%
\pgfsetbuttcap%
\pgfsetroundjoin%
\pgfsetlinewidth{2.007500pt}%
\definecolor{currentstroke}{rgb}{0.501961,0.501961,0.501961}%
\pgfsetstrokecolor{currentstroke}%
\pgfsetstrokeopacity{0.500000}%
\pgfsetdash{}{0pt}%
\pgfpathmoveto{\pgfqpoint{8.924226in}{6.651460in}}%
\pgfpathlineto{\pgfqpoint{8.965579in}{6.651460in}}%
\pgfusepath{stroke}%
\end{pgfscope}%
\begin{pgfscope}%
\pgfpathrectangle{\pgfqpoint{0.100000in}{2.802285in}}{\pgfqpoint{12.175956in}{4.509684in}}%
\pgfusepath{clip}%
\pgfsetbuttcap%
\pgfsetroundjoin%
\pgfsetlinewidth{2.007500pt}%
\definecolor{currentstroke}{rgb}{0.501961,0.501961,0.501961}%
\pgfsetstrokecolor{currentstroke}%
\pgfsetstrokeopacity{0.500000}%
\pgfsetdash{}{0pt}%
\pgfpathmoveto{\pgfqpoint{8.924226in}{7.106984in}}%
\pgfpathlineto{\pgfqpoint{8.965579in}{7.106984in}}%
\pgfusepath{stroke}%
\end{pgfscope}%
\begin{pgfscope}%
\pgfpathrectangle{\pgfqpoint{0.100000in}{2.802285in}}{\pgfqpoint{12.175956in}{4.509684in}}%
\pgfusepath{clip}%
\pgfsetrectcap%
\pgfsetroundjoin%
\pgfsetlinewidth{2.007500pt}%
\definecolor{currentstroke}{rgb}{0.501961,0.501961,0.501961}%
\pgfsetstrokecolor{currentstroke}%
\pgfsetstrokeopacity{0.500000}%
\pgfsetdash{}{0pt}%
\pgfpathmoveto{\pgfqpoint{9.634134in}{2.802285in}}%
\pgfpathlineto{\pgfqpoint{9.634134in}{7.311969in}}%
\pgfusepath{stroke}%
\end{pgfscope}%
\begin{pgfscope}%
\pgfpathrectangle{\pgfqpoint{0.100000in}{2.802285in}}{\pgfqpoint{12.175956in}{4.509684in}}%
\pgfusepath{clip}%
\pgfsetbuttcap%
\pgfsetroundjoin%
\pgfsetlinewidth{2.007500pt}%
\definecolor{currentstroke}{rgb}{0.501961,0.501961,0.501961}%
\pgfsetstrokecolor{currentstroke}%
\pgfsetstrokeopacity{0.500000}%
\pgfsetdash{}{0pt}%
\pgfpathmoveto{\pgfqpoint{9.613457in}{3.007271in}}%
\pgfpathlineto{\pgfqpoint{9.654811in}{3.007271in}}%
\pgfusepath{stroke}%
\end{pgfscope}%
\begin{pgfscope}%
\pgfpathrectangle{\pgfqpoint{0.100000in}{2.802285in}}{\pgfqpoint{12.175956in}{4.509684in}}%
\pgfusepath{clip}%
\pgfsetbuttcap%
\pgfsetroundjoin%
\pgfsetlinewidth{2.007500pt}%
\definecolor{currentstroke}{rgb}{0.501961,0.501961,0.501961}%
\pgfsetstrokecolor{currentstroke}%
\pgfsetstrokeopacity{0.500000}%
\pgfsetdash{}{0pt}%
\pgfpathmoveto{\pgfqpoint{9.613457in}{3.462794in}}%
\pgfpathlineto{\pgfqpoint{9.654811in}{3.462794in}}%
\pgfusepath{stroke}%
\end{pgfscope}%
\begin{pgfscope}%
\pgfpathrectangle{\pgfqpoint{0.100000in}{2.802285in}}{\pgfqpoint{12.175956in}{4.509684in}}%
\pgfusepath{clip}%
\pgfsetbuttcap%
\pgfsetroundjoin%
\pgfsetlinewidth{2.007500pt}%
\definecolor{currentstroke}{rgb}{0.501961,0.501961,0.501961}%
\pgfsetstrokecolor{currentstroke}%
\pgfsetstrokeopacity{0.500000}%
\pgfsetdash{}{0pt}%
\pgfpathmoveto{\pgfqpoint{9.613457in}{3.918318in}}%
\pgfpathlineto{\pgfqpoint{9.654811in}{3.918318in}}%
\pgfusepath{stroke}%
\end{pgfscope}%
\begin{pgfscope}%
\pgfpathrectangle{\pgfqpoint{0.100000in}{2.802285in}}{\pgfqpoint{12.175956in}{4.509684in}}%
\pgfusepath{clip}%
\pgfsetbuttcap%
\pgfsetroundjoin%
\pgfsetlinewidth{2.007500pt}%
\definecolor{currentstroke}{rgb}{0.501961,0.501961,0.501961}%
\pgfsetstrokecolor{currentstroke}%
\pgfsetstrokeopacity{0.500000}%
\pgfsetdash{}{0pt}%
\pgfpathmoveto{\pgfqpoint{9.613457in}{4.373842in}}%
\pgfpathlineto{\pgfqpoint{9.654811in}{4.373842in}}%
\pgfusepath{stroke}%
\end{pgfscope}%
\begin{pgfscope}%
\pgfpathrectangle{\pgfqpoint{0.100000in}{2.802285in}}{\pgfqpoint{12.175956in}{4.509684in}}%
\pgfusepath{clip}%
\pgfsetbuttcap%
\pgfsetroundjoin%
\pgfsetlinewidth{2.007500pt}%
\definecolor{currentstroke}{rgb}{0.501961,0.501961,0.501961}%
\pgfsetstrokecolor{currentstroke}%
\pgfsetstrokeopacity{0.500000}%
\pgfsetdash{}{0pt}%
\pgfpathmoveto{\pgfqpoint{9.613457in}{4.829365in}}%
\pgfpathlineto{\pgfqpoint{9.654811in}{4.829365in}}%
\pgfusepath{stroke}%
\end{pgfscope}%
\begin{pgfscope}%
\pgfpathrectangle{\pgfqpoint{0.100000in}{2.802285in}}{\pgfqpoint{12.175956in}{4.509684in}}%
\pgfusepath{clip}%
\pgfsetbuttcap%
\pgfsetroundjoin%
\pgfsetlinewidth{2.007500pt}%
\definecolor{currentstroke}{rgb}{0.501961,0.501961,0.501961}%
\pgfsetstrokecolor{currentstroke}%
\pgfsetstrokeopacity{0.500000}%
\pgfsetdash{}{0pt}%
\pgfpathmoveto{\pgfqpoint{9.613457in}{5.284889in}}%
\pgfpathlineto{\pgfqpoint{9.654811in}{5.284889in}}%
\pgfusepath{stroke}%
\end{pgfscope}%
\begin{pgfscope}%
\pgfpathrectangle{\pgfqpoint{0.100000in}{2.802285in}}{\pgfqpoint{12.175956in}{4.509684in}}%
\pgfusepath{clip}%
\pgfsetbuttcap%
\pgfsetroundjoin%
\pgfsetlinewidth{2.007500pt}%
\definecolor{currentstroke}{rgb}{0.501961,0.501961,0.501961}%
\pgfsetstrokecolor{currentstroke}%
\pgfsetstrokeopacity{0.500000}%
\pgfsetdash{}{0pt}%
\pgfpathmoveto{\pgfqpoint{9.613457in}{5.740413in}}%
\pgfpathlineto{\pgfqpoint{9.654811in}{5.740413in}}%
\pgfusepath{stroke}%
\end{pgfscope}%
\begin{pgfscope}%
\pgfpathrectangle{\pgfqpoint{0.100000in}{2.802285in}}{\pgfqpoint{12.175956in}{4.509684in}}%
\pgfusepath{clip}%
\pgfsetbuttcap%
\pgfsetroundjoin%
\pgfsetlinewidth{2.007500pt}%
\definecolor{currentstroke}{rgb}{0.501961,0.501961,0.501961}%
\pgfsetstrokecolor{currentstroke}%
\pgfsetstrokeopacity{0.500000}%
\pgfsetdash{}{0pt}%
\pgfpathmoveto{\pgfqpoint{9.613457in}{6.195936in}}%
\pgfpathlineto{\pgfqpoint{9.654811in}{6.195936in}}%
\pgfusepath{stroke}%
\end{pgfscope}%
\begin{pgfscope}%
\pgfpathrectangle{\pgfqpoint{0.100000in}{2.802285in}}{\pgfqpoint{12.175956in}{4.509684in}}%
\pgfusepath{clip}%
\pgfsetbuttcap%
\pgfsetroundjoin%
\pgfsetlinewidth{2.007500pt}%
\definecolor{currentstroke}{rgb}{0.501961,0.501961,0.501961}%
\pgfsetstrokecolor{currentstroke}%
\pgfsetstrokeopacity{0.500000}%
\pgfsetdash{}{0pt}%
\pgfpathmoveto{\pgfqpoint{9.613457in}{6.651460in}}%
\pgfpathlineto{\pgfqpoint{9.654811in}{6.651460in}}%
\pgfusepath{stroke}%
\end{pgfscope}%
\begin{pgfscope}%
\pgfpathrectangle{\pgfqpoint{0.100000in}{2.802285in}}{\pgfqpoint{12.175956in}{4.509684in}}%
\pgfusepath{clip}%
\pgfsetbuttcap%
\pgfsetroundjoin%
\pgfsetlinewidth{2.007500pt}%
\definecolor{currentstroke}{rgb}{0.501961,0.501961,0.501961}%
\pgfsetstrokecolor{currentstroke}%
\pgfsetstrokeopacity{0.500000}%
\pgfsetdash{}{0pt}%
\pgfpathmoveto{\pgfqpoint{9.613457in}{7.106984in}}%
\pgfpathlineto{\pgfqpoint{9.654811in}{7.106984in}}%
\pgfusepath{stroke}%
\end{pgfscope}%
\begin{pgfscope}%
\pgfpathrectangle{\pgfqpoint{0.100000in}{2.802285in}}{\pgfqpoint{12.175956in}{4.509684in}}%
\pgfusepath{clip}%
\pgfsetrectcap%
\pgfsetroundjoin%
\pgfsetlinewidth{2.007500pt}%
\definecolor{currentstroke}{rgb}{0.501961,0.501961,0.501961}%
\pgfsetstrokecolor{currentstroke}%
\pgfsetstrokeopacity{0.500000}%
\pgfsetdash{}{0pt}%
\pgfpathmoveto{\pgfqpoint{10.323365in}{2.802285in}}%
\pgfpathlineto{\pgfqpoint{10.323365in}{7.311969in}}%
\pgfusepath{stroke}%
\end{pgfscope}%
\begin{pgfscope}%
\pgfpathrectangle{\pgfqpoint{0.100000in}{2.802285in}}{\pgfqpoint{12.175956in}{4.509684in}}%
\pgfusepath{clip}%
\pgfsetbuttcap%
\pgfsetroundjoin%
\pgfsetlinewidth{2.007500pt}%
\definecolor{currentstroke}{rgb}{0.501961,0.501961,0.501961}%
\pgfsetstrokecolor{currentstroke}%
\pgfsetstrokeopacity{0.500000}%
\pgfsetdash{}{0pt}%
\pgfpathmoveto{\pgfqpoint{10.302688in}{3.007271in}}%
\pgfpathlineto{\pgfqpoint{10.344042in}{3.007271in}}%
\pgfusepath{stroke}%
\end{pgfscope}%
\begin{pgfscope}%
\pgfpathrectangle{\pgfqpoint{0.100000in}{2.802285in}}{\pgfqpoint{12.175956in}{4.509684in}}%
\pgfusepath{clip}%
\pgfsetbuttcap%
\pgfsetroundjoin%
\pgfsetlinewidth{2.007500pt}%
\definecolor{currentstroke}{rgb}{0.501961,0.501961,0.501961}%
\pgfsetstrokecolor{currentstroke}%
\pgfsetstrokeopacity{0.500000}%
\pgfsetdash{}{0pt}%
\pgfpathmoveto{\pgfqpoint{10.302688in}{3.462794in}}%
\pgfpathlineto{\pgfqpoint{10.344042in}{3.462794in}}%
\pgfusepath{stroke}%
\end{pgfscope}%
\begin{pgfscope}%
\pgfpathrectangle{\pgfqpoint{0.100000in}{2.802285in}}{\pgfqpoint{12.175956in}{4.509684in}}%
\pgfusepath{clip}%
\pgfsetbuttcap%
\pgfsetroundjoin%
\pgfsetlinewidth{2.007500pt}%
\definecolor{currentstroke}{rgb}{0.501961,0.501961,0.501961}%
\pgfsetstrokecolor{currentstroke}%
\pgfsetstrokeopacity{0.500000}%
\pgfsetdash{}{0pt}%
\pgfpathmoveto{\pgfqpoint{10.302688in}{3.918318in}}%
\pgfpathlineto{\pgfqpoint{10.344042in}{3.918318in}}%
\pgfusepath{stroke}%
\end{pgfscope}%
\begin{pgfscope}%
\pgfpathrectangle{\pgfqpoint{0.100000in}{2.802285in}}{\pgfqpoint{12.175956in}{4.509684in}}%
\pgfusepath{clip}%
\pgfsetbuttcap%
\pgfsetroundjoin%
\pgfsetlinewidth{2.007500pt}%
\definecolor{currentstroke}{rgb}{0.501961,0.501961,0.501961}%
\pgfsetstrokecolor{currentstroke}%
\pgfsetstrokeopacity{0.500000}%
\pgfsetdash{}{0pt}%
\pgfpathmoveto{\pgfqpoint{10.302688in}{4.373842in}}%
\pgfpathlineto{\pgfqpoint{10.344042in}{4.373842in}}%
\pgfusepath{stroke}%
\end{pgfscope}%
\begin{pgfscope}%
\pgfpathrectangle{\pgfqpoint{0.100000in}{2.802285in}}{\pgfqpoint{12.175956in}{4.509684in}}%
\pgfusepath{clip}%
\pgfsetbuttcap%
\pgfsetroundjoin%
\pgfsetlinewidth{2.007500pt}%
\definecolor{currentstroke}{rgb}{0.501961,0.501961,0.501961}%
\pgfsetstrokecolor{currentstroke}%
\pgfsetstrokeopacity{0.500000}%
\pgfsetdash{}{0pt}%
\pgfpathmoveto{\pgfqpoint{10.302688in}{4.829365in}}%
\pgfpathlineto{\pgfqpoint{10.344042in}{4.829365in}}%
\pgfusepath{stroke}%
\end{pgfscope}%
\begin{pgfscope}%
\pgfpathrectangle{\pgfqpoint{0.100000in}{2.802285in}}{\pgfqpoint{12.175956in}{4.509684in}}%
\pgfusepath{clip}%
\pgfsetbuttcap%
\pgfsetroundjoin%
\pgfsetlinewidth{2.007500pt}%
\definecolor{currentstroke}{rgb}{0.501961,0.501961,0.501961}%
\pgfsetstrokecolor{currentstroke}%
\pgfsetstrokeopacity{0.500000}%
\pgfsetdash{}{0pt}%
\pgfpathmoveto{\pgfqpoint{10.302688in}{5.284889in}}%
\pgfpathlineto{\pgfqpoint{10.344042in}{5.284889in}}%
\pgfusepath{stroke}%
\end{pgfscope}%
\begin{pgfscope}%
\pgfpathrectangle{\pgfqpoint{0.100000in}{2.802285in}}{\pgfqpoint{12.175956in}{4.509684in}}%
\pgfusepath{clip}%
\pgfsetbuttcap%
\pgfsetroundjoin%
\pgfsetlinewidth{2.007500pt}%
\definecolor{currentstroke}{rgb}{0.501961,0.501961,0.501961}%
\pgfsetstrokecolor{currentstroke}%
\pgfsetstrokeopacity{0.500000}%
\pgfsetdash{}{0pt}%
\pgfpathmoveto{\pgfqpoint{10.302688in}{5.740413in}}%
\pgfpathlineto{\pgfqpoint{10.344042in}{5.740413in}}%
\pgfusepath{stroke}%
\end{pgfscope}%
\begin{pgfscope}%
\pgfpathrectangle{\pgfqpoint{0.100000in}{2.802285in}}{\pgfqpoint{12.175956in}{4.509684in}}%
\pgfusepath{clip}%
\pgfsetbuttcap%
\pgfsetroundjoin%
\pgfsetlinewidth{2.007500pt}%
\definecolor{currentstroke}{rgb}{0.501961,0.501961,0.501961}%
\pgfsetstrokecolor{currentstroke}%
\pgfsetstrokeopacity{0.500000}%
\pgfsetdash{}{0pt}%
\pgfpathmoveto{\pgfqpoint{10.302688in}{6.195936in}}%
\pgfpathlineto{\pgfqpoint{10.344042in}{6.195936in}}%
\pgfusepath{stroke}%
\end{pgfscope}%
\begin{pgfscope}%
\pgfpathrectangle{\pgfqpoint{0.100000in}{2.802285in}}{\pgfqpoint{12.175956in}{4.509684in}}%
\pgfusepath{clip}%
\pgfsetbuttcap%
\pgfsetroundjoin%
\pgfsetlinewidth{2.007500pt}%
\definecolor{currentstroke}{rgb}{0.501961,0.501961,0.501961}%
\pgfsetstrokecolor{currentstroke}%
\pgfsetstrokeopacity{0.500000}%
\pgfsetdash{}{0pt}%
\pgfpathmoveto{\pgfqpoint{10.302688in}{6.651460in}}%
\pgfpathlineto{\pgfqpoint{10.344042in}{6.651460in}}%
\pgfusepath{stroke}%
\end{pgfscope}%
\begin{pgfscope}%
\pgfpathrectangle{\pgfqpoint{0.100000in}{2.802285in}}{\pgfqpoint{12.175956in}{4.509684in}}%
\pgfusepath{clip}%
\pgfsetbuttcap%
\pgfsetroundjoin%
\pgfsetlinewidth{2.007500pt}%
\definecolor{currentstroke}{rgb}{0.501961,0.501961,0.501961}%
\pgfsetstrokecolor{currentstroke}%
\pgfsetstrokeopacity{0.500000}%
\pgfsetdash{}{0pt}%
\pgfpathmoveto{\pgfqpoint{10.302688in}{7.106984in}}%
\pgfpathlineto{\pgfqpoint{10.344042in}{7.106984in}}%
\pgfusepath{stroke}%
\end{pgfscope}%
\begin{pgfscope}%
\pgfpathrectangle{\pgfqpoint{0.100000in}{2.802285in}}{\pgfqpoint{12.175956in}{4.509684in}}%
\pgfusepath{clip}%
\pgfsetrectcap%
\pgfsetroundjoin%
\pgfsetlinewidth{2.007500pt}%
\definecolor{currentstroke}{rgb}{0.501961,0.501961,0.501961}%
\pgfsetstrokecolor{currentstroke}%
\pgfsetstrokeopacity{0.500000}%
\pgfsetdash{}{0pt}%
\pgfpathmoveto{\pgfqpoint{11.012596in}{2.802285in}}%
\pgfpathlineto{\pgfqpoint{11.012596in}{7.311969in}}%
\pgfusepath{stroke}%
\end{pgfscope}%
\begin{pgfscope}%
\pgfpathrectangle{\pgfqpoint{0.100000in}{2.802285in}}{\pgfqpoint{12.175956in}{4.509684in}}%
\pgfusepath{clip}%
\pgfsetbuttcap%
\pgfsetroundjoin%
\pgfsetlinewidth{2.007500pt}%
\definecolor{currentstroke}{rgb}{0.501961,0.501961,0.501961}%
\pgfsetstrokecolor{currentstroke}%
\pgfsetstrokeopacity{0.500000}%
\pgfsetdash{}{0pt}%
\pgfpathmoveto{\pgfqpoint{10.991919in}{3.007271in}}%
\pgfpathlineto{\pgfqpoint{11.033273in}{3.007271in}}%
\pgfusepath{stroke}%
\end{pgfscope}%
\begin{pgfscope}%
\pgfpathrectangle{\pgfqpoint{0.100000in}{2.802285in}}{\pgfqpoint{12.175956in}{4.509684in}}%
\pgfusepath{clip}%
\pgfsetbuttcap%
\pgfsetroundjoin%
\pgfsetlinewidth{2.007500pt}%
\definecolor{currentstroke}{rgb}{0.501961,0.501961,0.501961}%
\pgfsetstrokecolor{currentstroke}%
\pgfsetstrokeopacity{0.500000}%
\pgfsetdash{}{0pt}%
\pgfpathmoveto{\pgfqpoint{10.991919in}{3.462794in}}%
\pgfpathlineto{\pgfqpoint{11.033273in}{3.462794in}}%
\pgfusepath{stroke}%
\end{pgfscope}%
\begin{pgfscope}%
\pgfpathrectangle{\pgfqpoint{0.100000in}{2.802285in}}{\pgfqpoint{12.175956in}{4.509684in}}%
\pgfusepath{clip}%
\pgfsetbuttcap%
\pgfsetroundjoin%
\pgfsetlinewidth{2.007500pt}%
\definecolor{currentstroke}{rgb}{0.501961,0.501961,0.501961}%
\pgfsetstrokecolor{currentstroke}%
\pgfsetstrokeopacity{0.500000}%
\pgfsetdash{}{0pt}%
\pgfpathmoveto{\pgfqpoint{10.991919in}{3.918318in}}%
\pgfpathlineto{\pgfqpoint{11.033273in}{3.918318in}}%
\pgfusepath{stroke}%
\end{pgfscope}%
\begin{pgfscope}%
\pgfpathrectangle{\pgfqpoint{0.100000in}{2.802285in}}{\pgfqpoint{12.175956in}{4.509684in}}%
\pgfusepath{clip}%
\pgfsetbuttcap%
\pgfsetroundjoin%
\pgfsetlinewidth{2.007500pt}%
\definecolor{currentstroke}{rgb}{0.501961,0.501961,0.501961}%
\pgfsetstrokecolor{currentstroke}%
\pgfsetstrokeopacity{0.500000}%
\pgfsetdash{}{0pt}%
\pgfpathmoveto{\pgfqpoint{10.991919in}{4.373842in}}%
\pgfpathlineto{\pgfqpoint{11.033273in}{4.373842in}}%
\pgfusepath{stroke}%
\end{pgfscope}%
\begin{pgfscope}%
\pgfpathrectangle{\pgfqpoint{0.100000in}{2.802285in}}{\pgfqpoint{12.175956in}{4.509684in}}%
\pgfusepath{clip}%
\pgfsetbuttcap%
\pgfsetroundjoin%
\pgfsetlinewidth{2.007500pt}%
\definecolor{currentstroke}{rgb}{0.501961,0.501961,0.501961}%
\pgfsetstrokecolor{currentstroke}%
\pgfsetstrokeopacity{0.500000}%
\pgfsetdash{}{0pt}%
\pgfpathmoveto{\pgfqpoint{10.991919in}{4.829365in}}%
\pgfpathlineto{\pgfqpoint{11.033273in}{4.829365in}}%
\pgfusepath{stroke}%
\end{pgfscope}%
\begin{pgfscope}%
\pgfpathrectangle{\pgfqpoint{0.100000in}{2.802285in}}{\pgfqpoint{12.175956in}{4.509684in}}%
\pgfusepath{clip}%
\pgfsetbuttcap%
\pgfsetroundjoin%
\pgfsetlinewidth{2.007500pt}%
\definecolor{currentstroke}{rgb}{0.501961,0.501961,0.501961}%
\pgfsetstrokecolor{currentstroke}%
\pgfsetstrokeopacity{0.500000}%
\pgfsetdash{}{0pt}%
\pgfpathmoveto{\pgfqpoint{10.991919in}{5.284889in}}%
\pgfpathlineto{\pgfqpoint{11.033273in}{5.284889in}}%
\pgfusepath{stroke}%
\end{pgfscope}%
\begin{pgfscope}%
\pgfpathrectangle{\pgfqpoint{0.100000in}{2.802285in}}{\pgfqpoint{12.175956in}{4.509684in}}%
\pgfusepath{clip}%
\pgfsetbuttcap%
\pgfsetroundjoin%
\pgfsetlinewidth{2.007500pt}%
\definecolor{currentstroke}{rgb}{0.501961,0.501961,0.501961}%
\pgfsetstrokecolor{currentstroke}%
\pgfsetstrokeopacity{0.500000}%
\pgfsetdash{}{0pt}%
\pgfpathmoveto{\pgfqpoint{10.991919in}{5.740413in}}%
\pgfpathlineto{\pgfqpoint{11.033273in}{5.740413in}}%
\pgfusepath{stroke}%
\end{pgfscope}%
\begin{pgfscope}%
\pgfpathrectangle{\pgfqpoint{0.100000in}{2.802285in}}{\pgfqpoint{12.175956in}{4.509684in}}%
\pgfusepath{clip}%
\pgfsetbuttcap%
\pgfsetroundjoin%
\pgfsetlinewidth{2.007500pt}%
\definecolor{currentstroke}{rgb}{0.501961,0.501961,0.501961}%
\pgfsetstrokecolor{currentstroke}%
\pgfsetstrokeopacity{0.500000}%
\pgfsetdash{}{0pt}%
\pgfpathmoveto{\pgfqpoint{10.991919in}{6.195936in}}%
\pgfpathlineto{\pgfqpoint{11.033273in}{6.195936in}}%
\pgfusepath{stroke}%
\end{pgfscope}%
\begin{pgfscope}%
\pgfpathrectangle{\pgfqpoint{0.100000in}{2.802285in}}{\pgfqpoint{12.175956in}{4.509684in}}%
\pgfusepath{clip}%
\pgfsetbuttcap%
\pgfsetroundjoin%
\pgfsetlinewidth{2.007500pt}%
\definecolor{currentstroke}{rgb}{0.501961,0.501961,0.501961}%
\pgfsetstrokecolor{currentstroke}%
\pgfsetstrokeopacity{0.500000}%
\pgfsetdash{}{0pt}%
\pgfpathmoveto{\pgfqpoint{10.991919in}{6.651460in}}%
\pgfpathlineto{\pgfqpoint{11.033273in}{6.651460in}}%
\pgfusepath{stroke}%
\end{pgfscope}%
\begin{pgfscope}%
\pgfpathrectangle{\pgfqpoint{0.100000in}{2.802285in}}{\pgfqpoint{12.175956in}{4.509684in}}%
\pgfusepath{clip}%
\pgfsetbuttcap%
\pgfsetroundjoin%
\pgfsetlinewidth{2.007500pt}%
\definecolor{currentstroke}{rgb}{0.501961,0.501961,0.501961}%
\pgfsetstrokecolor{currentstroke}%
\pgfsetstrokeopacity{0.500000}%
\pgfsetdash{}{0pt}%
\pgfpathmoveto{\pgfqpoint{10.991919in}{7.106984in}}%
\pgfpathlineto{\pgfqpoint{11.033273in}{7.106984in}}%
\pgfusepath{stroke}%
\end{pgfscope}%
\begin{pgfscope}%
\pgfpathrectangle{\pgfqpoint{0.100000in}{2.802285in}}{\pgfqpoint{12.175956in}{4.509684in}}%
\pgfusepath{clip}%
\pgfsetrectcap%
\pgfsetroundjoin%
\pgfsetlinewidth{2.007500pt}%
\definecolor{currentstroke}{rgb}{0.501961,0.501961,0.501961}%
\pgfsetstrokecolor{currentstroke}%
\pgfsetstrokeopacity{0.500000}%
\pgfsetdash{}{0pt}%
\pgfpathmoveto{\pgfqpoint{11.701827in}{2.802285in}}%
\pgfpathlineto{\pgfqpoint{11.701827in}{7.311969in}}%
\pgfusepath{stroke}%
\end{pgfscope}%
\begin{pgfscope}%
\pgfpathrectangle{\pgfqpoint{0.100000in}{2.802285in}}{\pgfqpoint{12.175956in}{4.509684in}}%
\pgfusepath{clip}%
\pgfsetbuttcap%
\pgfsetroundjoin%
\pgfsetlinewidth{2.007500pt}%
\definecolor{currentstroke}{rgb}{0.501961,0.501961,0.501961}%
\pgfsetstrokecolor{currentstroke}%
\pgfsetstrokeopacity{0.500000}%
\pgfsetdash{}{0pt}%
\pgfpathmoveto{\pgfqpoint{11.681150in}{3.007271in}}%
\pgfpathlineto{\pgfqpoint{11.722504in}{3.007271in}}%
\pgfusepath{stroke}%
\end{pgfscope}%
\begin{pgfscope}%
\pgfpathrectangle{\pgfqpoint{0.100000in}{2.802285in}}{\pgfqpoint{12.175956in}{4.509684in}}%
\pgfusepath{clip}%
\pgfsetbuttcap%
\pgfsetroundjoin%
\pgfsetlinewidth{2.007500pt}%
\definecolor{currentstroke}{rgb}{0.501961,0.501961,0.501961}%
\pgfsetstrokecolor{currentstroke}%
\pgfsetstrokeopacity{0.500000}%
\pgfsetdash{}{0pt}%
\pgfpathmoveto{\pgfqpoint{11.681150in}{3.462794in}}%
\pgfpathlineto{\pgfqpoint{11.722504in}{3.462794in}}%
\pgfusepath{stroke}%
\end{pgfscope}%
\begin{pgfscope}%
\pgfpathrectangle{\pgfqpoint{0.100000in}{2.802285in}}{\pgfqpoint{12.175956in}{4.509684in}}%
\pgfusepath{clip}%
\pgfsetbuttcap%
\pgfsetroundjoin%
\pgfsetlinewidth{2.007500pt}%
\definecolor{currentstroke}{rgb}{0.501961,0.501961,0.501961}%
\pgfsetstrokecolor{currentstroke}%
\pgfsetstrokeopacity{0.500000}%
\pgfsetdash{}{0pt}%
\pgfpathmoveto{\pgfqpoint{11.681150in}{3.918318in}}%
\pgfpathlineto{\pgfqpoint{11.722504in}{3.918318in}}%
\pgfusepath{stroke}%
\end{pgfscope}%
\begin{pgfscope}%
\pgfpathrectangle{\pgfqpoint{0.100000in}{2.802285in}}{\pgfqpoint{12.175956in}{4.509684in}}%
\pgfusepath{clip}%
\pgfsetbuttcap%
\pgfsetroundjoin%
\pgfsetlinewidth{2.007500pt}%
\definecolor{currentstroke}{rgb}{0.501961,0.501961,0.501961}%
\pgfsetstrokecolor{currentstroke}%
\pgfsetstrokeopacity{0.500000}%
\pgfsetdash{}{0pt}%
\pgfpathmoveto{\pgfqpoint{11.681150in}{4.373842in}}%
\pgfpathlineto{\pgfqpoint{11.722504in}{4.373842in}}%
\pgfusepath{stroke}%
\end{pgfscope}%
\begin{pgfscope}%
\pgfpathrectangle{\pgfqpoint{0.100000in}{2.802285in}}{\pgfqpoint{12.175956in}{4.509684in}}%
\pgfusepath{clip}%
\pgfsetbuttcap%
\pgfsetroundjoin%
\pgfsetlinewidth{2.007500pt}%
\definecolor{currentstroke}{rgb}{0.501961,0.501961,0.501961}%
\pgfsetstrokecolor{currentstroke}%
\pgfsetstrokeopacity{0.500000}%
\pgfsetdash{}{0pt}%
\pgfpathmoveto{\pgfqpoint{11.681150in}{4.829365in}}%
\pgfpathlineto{\pgfqpoint{11.722504in}{4.829365in}}%
\pgfusepath{stroke}%
\end{pgfscope}%
\begin{pgfscope}%
\pgfpathrectangle{\pgfqpoint{0.100000in}{2.802285in}}{\pgfqpoint{12.175956in}{4.509684in}}%
\pgfusepath{clip}%
\pgfsetbuttcap%
\pgfsetroundjoin%
\pgfsetlinewidth{2.007500pt}%
\definecolor{currentstroke}{rgb}{0.501961,0.501961,0.501961}%
\pgfsetstrokecolor{currentstroke}%
\pgfsetstrokeopacity{0.500000}%
\pgfsetdash{}{0pt}%
\pgfpathmoveto{\pgfqpoint{11.681150in}{5.284889in}}%
\pgfpathlineto{\pgfqpoint{11.722504in}{5.284889in}}%
\pgfusepath{stroke}%
\end{pgfscope}%
\begin{pgfscope}%
\pgfpathrectangle{\pgfqpoint{0.100000in}{2.802285in}}{\pgfqpoint{12.175956in}{4.509684in}}%
\pgfusepath{clip}%
\pgfsetbuttcap%
\pgfsetroundjoin%
\pgfsetlinewidth{2.007500pt}%
\definecolor{currentstroke}{rgb}{0.501961,0.501961,0.501961}%
\pgfsetstrokecolor{currentstroke}%
\pgfsetstrokeopacity{0.500000}%
\pgfsetdash{}{0pt}%
\pgfpathmoveto{\pgfqpoint{11.681150in}{5.740413in}}%
\pgfpathlineto{\pgfqpoint{11.722504in}{5.740413in}}%
\pgfusepath{stroke}%
\end{pgfscope}%
\begin{pgfscope}%
\pgfpathrectangle{\pgfqpoint{0.100000in}{2.802285in}}{\pgfqpoint{12.175956in}{4.509684in}}%
\pgfusepath{clip}%
\pgfsetbuttcap%
\pgfsetroundjoin%
\pgfsetlinewidth{2.007500pt}%
\definecolor{currentstroke}{rgb}{0.501961,0.501961,0.501961}%
\pgfsetstrokecolor{currentstroke}%
\pgfsetstrokeopacity{0.500000}%
\pgfsetdash{}{0pt}%
\pgfpathmoveto{\pgfqpoint{11.681150in}{6.195936in}}%
\pgfpathlineto{\pgfqpoint{11.722504in}{6.195936in}}%
\pgfusepath{stroke}%
\end{pgfscope}%
\begin{pgfscope}%
\pgfpathrectangle{\pgfqpoint{0.100000in}{2.802285in}}{\pgfqpoint{12.175956in}{4.509684in}}%
\pgfusepath{clip}%
\pgfsetbuttcap%
\pgfsetroundjoin%
\pgfsetlinewidth{2.007500pt}%
\definecolor{currentstroke}{rgb}{0.501961,0.501961,0.501961}%
\pgfsetstrokecolor{currentstroke}%
\pgfsetstrokeopacity{0.500000}%
\pgfsetdash{}{0pt}%
\pgfpathmoveto{\pgfqpoint{11.681150in}{6.651460in}}%
\pgfpathlineto{\pgfqpoint{11.722504in}{6.651460in}}%
\pgfusepath{stroke}%
\end{pgfscope}%
\begin{pgfscope}%
\pgfpathrectangle{\pgfqpoint{0.100000in}{2.802285in}}{\pgfqpoint{12.175956in}{4.509684in}}%
\pgfusepath{clip}%
\pgfsetbuttcap%
\pgfsetroundjoin%
\pgfsetlinewidth{2.007500pt}%
\definecolor{currentstroke}{rgb}{0.501961,0.501961,0.501961}%
\pgfsetstrokecolor{currentstroke}%
\pgfsetstrokeopacity{0.500000}%
\pgfsetdash{}{0pt}%
\pgfpathmoveto{\pgfqpoint{11.681150in}{7.106984in}}%
\pgfpathlineto{\pgfqpoint{11.722504in}{7.106984in}}%
\pgfusepath{stroke}%
\end{pgfscope}%
\begin{pgfscope}%
\pgfsetrectcap%
\pgfsetmiterjoin%
\pgfsetlinewidth{0.803000pt}%
\definecolor{currentstroke}{rgb}{0.000000,0.000000,0.000000}%
\pgfsetstrokecolor{currentstroke}%
\pgfsetdash{}{0pt}%
\pgfpathmoveto{\pgfqpoint{0.100000in}{2.802285in}}%
\pgfpathlineto{\pgfqpoint{12.275956in}{2.802285in}}%
\pgfusepath{stroke}%
\end{pgfscope}%
\begin{pgfscope}%
\pgfsetrectcap%
\pgfsetmiterjoin%
\pgfsetlinewidth{0.803000pt}%
\definecolor{currentstroke}{rgb}{0.000000,0.000000,0.000000}%
\pgfsetstrokecolor{currentstroke}%
\pgfsetdash{}{0pt}%
\pgfpathmoveto{\pgfqpoint{0.100000in}{7.311969in}}%
\pgfpathlineto{\pgfqpoint{12.275956in}{7.311969in}}%
\pgfusepath{stroke}%
\end{pgfscope}%
\begin{pgfscope}%
\definecolor{textcolor}{rgb}{0.000000,0.000000,0.000000}%
\pgfsetstrokecolor{textcolor}%
\pgfsetfillcolor{textcolor}%
\pgftext[x=0.618991in,y=2.597299in,left,base]{\color{textcolor}{\rmfamily\fontsize{10.000000}{12.000000}\selectfont\catcode`\^=\active\def^{\ifmmode\sp\else\^{}\fi}\catcode`\%=\active\def%{\%}1.25}}%
\end{pgfscope}%
\begin{pgfscope}%
\definecolor{textcolor}{rgb}{0.000000,0.000000,0.000000}%
\pgfsetstrokecolor{textcolor}%
\pgfsetfillcolor{textcolor}%
\pgftext[x=0.618991in,y=7.414462in,left,base]{\color{textcolor}{\rmfamily\fontsize{10.000000}{12.000000}\selectfont\catcode`\^=\active\def^{\ifmmode\sp\else\^{}\fi}\catcode`\%=\active\def%{\%}147.57}}%
\end{pgfscope}%
\begin{pgfscope}%
\definecolor{textcolor}{rgb}{0.000000,0.000000,0.000000}%
\pgfsetstrokecolor{textcolor}%
\pgfsetfillcolor{textcolor}%
\pgftext[x=1.308222in,y=2.597299in,left,base]{\color{textcolor}{\rmfamily\fontsize{10.000000}{12.000000}\selectfont\catcode`\^=\active\def^{\ifmmode\sp\else\^{}\fi}\catcode`\%=\active\def%{\%}8.05e+05}}%
\end{pgfscope}%
\begin{pgfscope}%
\definecolor{textcolor}{rgb}{0.000000,0.000000,0.000000}%
\pgfsetstrokecolor{textcolor}%
\pgfsetfillcolor{textcolor}%
\pgftext[x=1.308222in,y=7.414462in,left,base]{\color{textcolor}{\rmfamily\fontsize{10.000000}{12.000000}\selectfont\catcode`\^=\active\def^{\ifmmode\sp\else\^{}\fi}\catcode`\%=\active\def%{\%}1.77e+06}}%
\end{pgfscope}%
\begin{pgfscope}%
\definecolor{textcolor}{rgb}{0.000000,0.000000,0.000000}%
\pgfsetstrokecolor{textcolor}%
\pgfsetfillcolor{textcolor}%
\pgftext[x=1.997453in,y=2.597299in,left,base]{\color{textcolor}{\rmfamily\fontsize{10.000000}{12.000000}\selectfont\catcode`\^=\active\def^{\ifmmode\sp\else\^{}\fi}\catcode`\%=\active\def%{\%}519.00}}%
\end{pgfscope}%
\begin{pgfscope}%
\definecolor{textcolor}{rgb}{0.000000,0.000000,0.000000}%
\pgfsetstrokecolor{textcolor}%
\pgfsetfillcolor{textcolor}%
\pgftext[x=1.997453in,y=7.414462in,left,base]{\color{textcolor}{\rmfamily\fontsize{10.000000}{12.000000}\selectfont\catcode`\^=\active\def^{\ifmmode\sp\else\^{}\fi}\catcode`\%=\active\def%{\%}1.30e+04}}%
\end{pgfscope}%
\begin{pgfscope}%
\definecolor{textcolor}{rgb}{0.000000,0.000000,0.000000}%
\pgfsetstrokecolor{textcolor}%
\pgfsetfillcolor{textcolor}%
\pgftext[x=2.686684in,y=2.597299in,left,base]{\color{textcolor}{\rmfamily\fontsize{10.000000}{12.000000}\selectfont\catcode`\^=\active\def^{\ifmmode\sp\else\^{}\fi}\catcode`\%=\active\def%{\%}0.00}}%
\end{pgfscope}%
\begin{pgfscope}%
\definecolor{textcolor}{rgb}{0.000000,0.000000,0.000000}%
\pgfsetstrokecolor{textcolor}%
\pgfsetfillcolor{textcolor}%
\pgftext[x=2.686684in,y=7.414462in,left,base]{\color{textcolor}{\rmfamily\fontsize{10.000000}{12.000000}\selectfont\catcode`\^=\active\def^{\ifmmode\sp\else\^{}\fi}\catcode`\%=\active\def%{\%}296.48}}%
\end{pgfscope}%
\begin{pgfscope}%
\definecolor{textcolor}{rgb}{0.000000,0.000000,0.000000}%
\pgfsetstrokecolor{textcolor}%
\pgfsetfillcolor{textcolor}%
\pgftext[x=3.375915in,y=2.597299in,left,base]{\color{textcolor}{\rmfamily\fontsize{10.000000}{12.000000}\selectfont\catcode`\^=\active\def^{\ifmmode\sp\else\^{}\fi}\catcode`\%=\active\def%{\%}278.60}}%
\end{pgfscope}%
\begin{pgfscope}%
\definecolor{textcolor}{rgb}{0.000000,0.000000,0.000000}%
\pgfsetstrokecolor{textcolor}%
\pgfsetfillcolor{textcolor}%
\pgftext[x=3.375915in,y=7.414462in,left,base]{\color{textcolor}{\rmfamily\fontsize{10.000000}{12.000000}\selectfont\catcode`\^=\active\def^{\ifmmode\sp\else\^{}\fi}\catcode`\%=\active\def%{\%}2.83e+03}}%
\end{pgfscope}%
\begin{pgfscope}%
\definecolor{textcolor}{rgb}{0.000000,0.000000,0.000000}%
\pgfsetstrokecolor{textcolor}%
\pgfsetfillcolor{textcolor}%
\pgftext[x=4.065146in,y=2.597299in,left,base]{\color{textcolor}{\rmfamily\fontsize{10.000000}{12.000000}\selectfont\catcode`\^=\active\def^{\ifmmode\sp\else\^{}\fi}\catcode`\%=\active\def%{\%}0.00}}%
\end{pgfscope}%
\begin{pgfscope}%
\definecolor{textcolor}{rgb}{0.000000,0.000000,0.000000}%
\pgfsetstrokecolor{textcolor}%
\pgfsetfillcolor{textcolor}%
\pgftext[x=4.065146in,y=7.414462in,left,base]{\color{textcolor}{\rmfamily\fontsize{10.000000}{12.000000}\selectfont\catcode`\^=\active\def^{\ifmmode\sp\else\^{}\fi}\catcode`\%=\active\def%{\%}1.00}}%
\end{pgfscope}%
\begin{pgfscope}%
\definecolor{textcolor}{rgb}{0.000000,0.000000,0.000000}%
\pgfsetstrokecolor{textcolor}%
\pgfsetfillcolor{textcolor}%
\pgftext[x=4.754378in,y=2.597299in,left,base]{\color{textcolor}{\rmfamily\fontsize{10.000000}{12.000000}\selectfont\catcode`\^=\active\def^{\ifmmode\sp\else\^{}\fi}\catcode`\%=\active\def%{\%}0.08}}%
\end{pgfscope}%
\begin{pgfscope}%
\definecolor{textcolor}{rgb}{0.000000,0.000000,0.000000}%
\pgfsetstrokecolor{textcolor}%
\pgfsetfillcolor{textcolor}%
\pgftext[x=4.754378in,y=7.414462in,left,base]{\color{textcolor}{\rmfamily\fontsize{10.000000}{12.000000}\selectfont\catcode`\^=\active\def^{\ifmmode\sp\else\^{}\fi}\catcode`\%=\active\def%{\%}0.25}}%
\end{pgfscope}%
\begin{pgfscope}%
\definecolor{textcolor}{rgb}{0.000000,0.000000,0.000000}%
\pgfsetstrokecolor{textcolor}%
\pgfsetfillcolor{textcolor}%
\pgftext[x=5.443609in,y=2.597299in,left,base]{\color{textcolor}{\rmfamily\fontsize{10.000000}{12.000000}\selectfont\catcode`\^=\active\def^{\ifmmode\sp\else\^{}\fi}\catcode`\%=\active\def%{\%}2.37e+04}}%
\end{pgfscope}%
\begin{pgfscope}%
\definecolor{textcolor}{rgb}{0.000000,0.000000,0.000000}%
\pgfsetstrokecolor{textcolor}%
\pgfsetfillcolor{textcolor}%
\pgftext[x=5.443609in,y=7.414462in,left,base]{\color{textcolor}{\rmfamily\fontsize{10.000000}{12.000000}\selectfont\catcode`\^=\active\def^{\ifmmode\sp\else\^{}\fi}\catcode`\%=\active\def%{\%}3.36e+04}}%
\end{pgfscope}%
\begin{pgfscope}%
\definecolor{textcolor}{rgb}{0.000000,0.000000,0.000000}%
\pgfsetstrokecolor{textcolor}%
\pgfsetfillcolor{textcolor}%
\pgftext[x=6.132840in,y=2.597299in,left,base]{\color{textcolor}{\rmfamily\fontsize{10.000000}{12.000000}\selectfont\catcode`\^=\active\def^{\ifmmode\sp\else\^{}\fi}\catcode`\%=\active\def%{\%}13.50}}%
\end{pgfscope}%
\begin{pgfscope}%
\definecolor{textcolor}{rgb}{0.000000,0.000000,0.000000}%
\pgfsetstrokecolor{textcolor}%
\pgfsetfillcolor{textcolor}%
\pgftext[x=6.132840in,y=7.414462in,left,base]{\color{textcolor}{\rmfamily\fontsize{10.000000}{12.000000}\selectfont\catcode`\^=\active\def^{\ifmmode\sp\else\^{}\fi}\catcode`\%=\active\def%{\%}134.90}}%
\end{pgfscope}%
\begin{pgfscope}%
\definecolor{textcolor}{rgb}{0.000000,0.000000,0.000000}%
\pgfsetstrokecolor{textcolor}%
\pgfsetfillcolor{textcolor}%
\pgftext[x=6.822071in,y=2.597299in,left,base]{\color{textcolor}{\rmfamily\fontsize{10.000000}{12.000000}\selectfont\catcode`\^=\active\def^{\ifmmode\sp\else\^{}\fi}\catcode`\%=\active\def%{\%}0.55}}%
\end{pgfscope}%
\begin{pgfscope}%
\definecolor{textcolor}{rgb}{0.000000,0.000000,0.000000}%
\pgfsetstrokecolor{textcolor}%
\pgfsetfillcolor{textcolor}%
\pgftext[x=6.822071in,y=7.414462in,left,base]{\color{textcolor}{\rmfamily\fontsize{10.000000}{12.000000}\selectfont\catcode`\^=\active\def^{\ifmmode\sp\else\^{}\fi}\catcode`\%=\active\def%{\%}2.93}}%
\end{pgfscope}%
\begin{pgfscope}%
\definecolor{textcolor}{rgb}{0.000000,0.000000,0.000000}%
\pgfsetstrokecolor{textcolor}%
\pgfsetfillcolor{textcolor}%
\pgftext[x=7.511302in,y=2.597299in,left,base]{\color{textcolor}{\rmfamily\fontsize{10.000000}{12.000000}\selectfont\catcode`\^=\active\def^{\ifmmode\sp\else\^{}\fi}\catcode`\%=\active\def%{\%}0.00}}%
\end{pgfscope}%
\begin{pgfscope}%
\definecolor{textcolor}{rgb}{0.000000,0.000000,0.000000}%
\pgfsetstrokecolor{textcolor}%
\pgfsetfillcolor{textcolor}%
\pgftext[x=7.511302in,y=7.414462in,left,base]{\color{textcolor}{\rmfamily\fontsize{10.000000}{12.000000}\selectfont\catcode`\^=\active\def^{\ifmmode\sp\else\^{}\fi}\catcode`\%=\active\def%{\%}305.73}}%
\end{pgfscope}%
\begin{pgfscope}%
\definecolor{textcolor}{rgb}{0.000000,0.000000,0.000000}%
\pgfsetstrokecolor{textcolor}%
\pgfsetfillcolor{textcolor}%
\pgftext[x=8.200533in,y=2.597299in,left,base]{\color{textcolor}{\rmfamily\fontsize{10.000000}{12.000000}\selectfont\catcode`\^=\active\def^{\ifmmode\sp\else\^{}\fi}\catcode`\%=\active\def%{\%}0.00}}%
\end{pgfscope}%
\begin{pgfscope}%
\definecolor{textcolor}{rgb}{0.000000,0.000000,0.000000}%
\pgfsetstrokecolor{textcolor}%
\pgfsetfillcolor{textcolor}%
\pgftext[x=8.200533in,y=7.414462in,left,base]{\color{textcolor}{\rmfamily\fontsize{10.000000}{12.000000}\selectfont\catcode`\^=\active\def^{\ifmmode\sp\else\^{}\fi}\catcode`\%=\active\def%{\%}10.86}}%
\end{pgfscope}%
\begin{pgfscope}%
\definecolor{textcolor}{rgb}{0.000000,0.000000,0.000000}%
\pgfsetstrokecolor{textcolor}%
\pgfsetfillcolor{textcolor}%
\pgftext[x=8.889764in,y=2.597299in,left,base]{\color{textcolor}{\rmfamily\fontsize{10.000000}{12.000000}\selectfont\catcode`\^=\active\def^{\ifmmode\sp\else\^{}\fi}\catcode`\%=\active\def%{\%}0.00}}%
\end{pgfscope}%
\begin{pgfscope}%
\definecolor{textcolor}{rgb}{0.000000,0.000000,0.000000}%
\pgfsetstrokecolor{textcolor}%
\pgfsetfillcolor{textcolor}%
\pgftext[x=8.889764in,y=7.414462in,left,base]{\color{textcolor}{\rmfamily\fontsize{10.000000}{12.000000}\selectfont\catcode`\^=\active\def^{\ifmmode\sp\else\^{}\fi}\catcode`\%=\active\def%{\%}5.00}}%
\end{pgfscope}%
\begin{pgfscope}%
\definecolor{textcolor}{rgb}{0.000000,0.000000,0.000000}%
\pgfsetstrokecolor{textcolor}%
\pgfsetfillcolor{textcolor}%
\pgftext[x=9.578995in,y=2.597299in,left,base]{\color{textcolor}{\rmfamily\fontsize{10.000000}{12.000000}\selectfont\catcode`\^=\active\def^{\ifmmode\sp\else\^{}\fi}\catcode`\%=\active\def%{\%}0.00}}%
\end{pgfscope}%
\begin{pgfscope}%
\definecolor{textcolor}{rgb}{0.000000,0.000000,0.000000}%
\pgfsetstrokecolor{textcolor}%
\pgfsetfillcolor{textcolor}%
\pgftext[x=9.578995in,y=7.414462in,left,base]{\color{textcolor}{\rmfamily\fontsize{10.000000}{12.000000}\selectfont\catcode`\^=\active\def^{\ifmmode\sp\else\^{}\fi}\catcode`\%=\active\def%{\%}3.00}}%
\end{pgfscope}%
\begin{pgfscope}%
\definecolor{textcolor}{rgb}{0.000000,0.000000,0.000000}%
\pgfsetstrokecolor{textcolor}%
\pgfsetfillcolor{textcolor}%
\pgftext[x=10.268226in,y=2.597299in,left,base]{\color{textcolor}{\rmfamily\fontsize{10.000000}{12.000000}\selectfont\catcode`\^=\active\def^{\ifmmode\sp\else\^{}\fi}\catcode`\%=\active\def%{\%}0.00}}%
\end{pgfscope}%
\begin{pgfscope}%
\definecolor{textcolor}{rgb}{0.000000,0.000000,0.000000}%
\pgfsetstrokecolor{textcolor}%
\pgfsetfillcolor{textcolor}%
\pgftext[x=10.268226in,y=7.414462in,left,base]{\color{textcolor}{\rmfamily\fontsize{10.000000}{12.000000}\selectfont\catcode`\^=\active\def^{\ifmmode\sp\else\^{}\fi}\catcode`\%=\active\def%{\%}4.00}}%
\end{pgfscope}%
\begin{pgfscope}%
\definecolor{textcolor}{rgb}{0.000000,0.000000,0.000000}%
\pgfsetstrokecolor{textcolor}%
\pgfsetfillcolor{textcolor}%
\pgftext[x=10.957457in,y=2.597299in,left,base]{\color{textcolor}{\rmfamily\fontsize{10.000000}{12.000000}\selectfont\catcode`\^=\active\def^{\ifmmode\sp\else\^{}\fi}\catcode`\%=\active\def%{\%}0.00}}%
\end{pgfscope}%
\begin{pgfscope}%
\definecolor{textcolor}{rgb}{0.000000,0.000000,0.000000}%
\pgfsetstrokecolor{textcolor}%
\pgfsetfillcolor{textcolor}%
\pgftext[x=10.957457in,y=7.414462in,left,base]{\color{textcolor}{\rmfamily\fontsize{10.000000}{12.000000}\selectfont\catcode`\^=\active\def^{\ifmmode\sp\else\^{}\fi}\catcode`\%=\active\def%{\%}3.00}}%
\end{pgfscope}%
\begin{pgfscope}%
\definecolor{textcolor}{rgb}{0.000000,0.000000,0.000000}%
\pgfsetstrokecolor{textcolor}%
\pgfsetfillcolor{textcolor}%
\pgftext[x=11.646688in,y=2.597299in,left,base]{\color{textcolor}{\rmfamily\fontsize{10.000000}{12.000000}\selectfont\catcode`\^=\active\def^{\ifmmode\sp\else\^{}\fi}\catcode`\%=\active\def%{\%}0.00}}%
\end{pgfscope}%
\begin{pgfscope}%
\definecolor{textcolor}{rgb}{0.000000,0.000000,0.000000}%
\pgfsetstrokecolor{textcolor}%
\pgfsetfillcolor{textcolor}%
\pgftext[x=11.646688in,y=7.414462in,left,base]{\color{textcolor}{\rmfamily\fontsize{10.000000}{12.000000}\selectfont\catcode`\^=\active\def^{\ifmmode\sp\else\^{}\fi}\catcode`\%=\active\def%{\%}3.00}}%
\end{pgfscope}%
\begin{pgfscope}%
\definecolor{textcolor}{rgb}{0.000000,0.000000,0.000000}%
\pgfsetstrokecolor{textcolor}%
\pgfsetfillcolor{textcolor}%
\pgftext[x=6.187978in,y=7.728636in,,base]{\color{textcolor}{\rmfamily\fontsize{20.000000}{24.000000}\selectfont\catcode`\^=\active\def^{\ifmmode\sp\else\^{}\fi}\catcode`\%=\active\def%{\%}Objective Space}}%
\end{pgfscope}%
\begin{pgfscope}%
\pgfsetbuttcap%
\pgfsetmiterjoin%
\definecolor{currentfill}{rgb}{1.000000,1.000000,1.000000}%
\pgfsetfillcolor{currentfill}%
\pgfsetfillopacity{0.800000}%
\pgfsetlinewidth{1.003750pt}%
\definecolor{currentstroke}{rgb}{0.800000,0.800000,0.800000}%
\pgfsetstrokecolor{currentstroke}%
\pgfsetstrokeopacity{0.800000}%
\pgfsetdash{}{0pt}%
\pgfpathmoveto{\pgfqpoint{0.255556in}{6.160812in}}%
\pgfpathlineto{\pgfqpoint{2.717188in}{6.160812in}}%
\pgfpathquadraticcurveto{\pgfqpoint{2.761633in}{6.160812in}}{\pgfqpoint{2.761633in}{6.205257in}}%
\pgfpathlineto{\pgfqpoint{2.761633in}{7.156414in}}%
\pgfpathquadraticcurveto{\pgfqpoint{2.761633in}{7.200858in}}{\pgfqpoint{2.717188in}{7.200858in}}%
\pgfpathlineto{\pgfqpoint{0.255556in}{7.200858in}}%
\pgfpathquadraticcurveto{\pgfqpoint{0.211111in}{7.200858in}}{\pgfqpoint{0.211111in}{7.156414in}}%
\pgfpathlineto{\pgfqpoint{0.211111in}{6.205257in}}%
\pgfpathquadraticcurveto{\pgfqpoint{0.211111in}{6.160812in}}{\pgfqpoint{0.255556in}{6.160812in}}%
\pgfpathlineto{\pgfqpoint{0.255556in}{6.160812in}}%
\pgfpathclose%
\pgfusepath{stroke,fill}%
\end{pgfscope}%
\begin{pgfscope}%
\pgfsetrectcap%
\pgfsetroundjoin%
\pgfsetlinewidth{1.505625pt}%
\definecolor{currentstroke}{rgb}{0.839216,0.152941,0.156863}%
\pgfsetstrokecolor{currentstroke}%
\pgfsetdash{}{0pt}%
\pgfpathmoveto{\pgfqpoint{0.300000in}{7.023080in}}%
\pgfpathlineto{\pgfqpoint{0.522222in}{7.023080in}}%
\pgfpathlineto{\pgfqpoint{0.744444in}{7.023080in}}%
\pgfusepath{stroke}%
\end{pgfscope}%
\begin{pgfscope}%
\definecolor{textcolor}{rgb}{0.000000,0.000000,0.000000}%
\pgfsetstrokecolor{textcolor}%
\pgfsetfillcolor{textcolor}%
\pgftext[x=0.922222in,y=6.945303in,left,base]{\color{textcolor}{\rmfamily\fontsize{16.000000}{19.200000}\selectfont\catcode`\^=\active\def^{\ifmmode\sp\else\^{}\fi}\catcode`\%=\active\def%{\%}once-through}}%
\end{pgfscope}%
\begin{pgfscope}%
\pgfsetrectcap%
\pgfsetroundjoin%
\pgfsetlinewidth{1.505625pt}%
\definecolor{currentstroke}{rgb}{0.172549,0.627451,0.172549}%
\pgfsetstrokecolor{currentstroke}%
\pgfsetdash{}{0pt}%
\pgfpathmoveto{\pgfqpoint{0.300000in}{6.698621in}}%
\pgfpathlineto{\pgfqpoint{0.522222in}{6.698621in}}%
\pgfpathlineto{\pgfqpoint{0.744444in}{6.698621in}}%
\pgfusepath{stroke}%
\end{pgfscope}%
\begin{pgfscope}%
\definecolor{textcolor}{rgb}{0.000000,0.000000,0.000000}%
\pgfsetstrokecolor{textcolor}%
\pgfsetfillcolor{textcolor}%
\pgftext[x=0.922222in,y=6.620843in,left,base]{\color{textcolor}{\rmfamily\fontsize{16.000000}{19.200000}\selectfont\catcode`\^=\active\def^{\ifmmode\sp\else\^{}\fi}\catcode`\%=\active\def%{\%}limited-recycle}}%
\end{pgfscope}%
\begin{pgfscope}%
\pgfsetrectcap%
\pgfsetroundjoin%
\pgfsetlinewidth{1.505625pt}%
\definecolor{currentstroke}{rgb}{0.121569,0.466667,0.705882}%
\pgfsetstrokecolor{currentstroke}%
\pgfsetdash{}{0pt}%
\pgfpathmoveto{\pgfqpoint{0.300000in}{6.374161in}}%
\pgfpathlineto{\pgfqpoint{0.522222in}{6.374161in}}%
\pgfpathlineto{\pgfqpoint{0.744444in}{6.374161in}}%
\pgfusepath{stroke}%
\end{pgfscope}%
\begin{pgfscope}%
\definecolor{textcolor}{rgb}{0.000000,0.000000,0.000000}%
\pgfsetstrokecolor{textcolor}%
\pgfsetfillcolor{textcolor}%
\pgftext[x=0.922222in,y=6.296383in,left,base]{\color{textcolor}{\rmfamily\fontsize{16.000000}{19.200000}\selectfont\catcode`\^=\active\def^{\ifmmode\sp\else\^{}\fi}\catcode`\%=\active\def%{\%}continuous-recycle}}%
\end{pgfscope}%
\end{pgfpicture}%
\makeatother%
\endgroup%
}
        \caption{The full set of Pareto optimal \glspl{eg} from 
        the \gls{set}.}
        \label{fig:full-set-plot}
    \end{center}
\end{figure}

\begin{figure}[ht!]
    \begin{center}
        \resizebox{\columnwidth}{!}{%% Creator: Matplotlib, PGF backend
%%
%% To include the figure in your LaTeX document, write
%%   \input{<filename>.pgf}
%%
%% Make sure the required packages are loaded in your preamble
%%   \usepackage{pgf}
%%
%% Also ensure that all the required font packages are loaded; for instance,
%% the lmodern package is sometimes necessary when using math font.
%%   \usepackage{lmodern}
%%
%% Figures using additional raster images can only be included by \input if
%% they are in the same directory as the main LaTeX file. For loading figures
%% from other directories you can use the `import` package
%%   \usepackage{import}
%%
%% and then include the figures with
%%   \import{<path to file>}{<filename>.pgf}
%%
%% Matplotlib used the following preamble
%%   \def\mathdefault#1{#1}
%%   \everymath=\expandafter{\the\everymath\displaystyle}
%%   \IfFileExists{scrextend.sty}{
%%     \usepackage[fontsize=10.000000pt]{scrextend}
%%   }{
%%     \renewcommand{\normalsize}{\fontsize{10.000000}{12.000000}\selectfont}
%%     \normalsize
%%   }
%%   
%%   \makeatletter\@ifpackageloaded{underscore}{}{\usepackage[strings]{underscore}}\makeatother
%%
\begingroup%
\makeatletter%
\begin{pgfpicture}%
\pgfpathrectangle{\pgfpointorigin}{\pgfqpoint{12.375956in}{8.028674in}}%
\pgfusepath{use as bounding box, clip}%
\begin{pgfscope}%
\pgfsetbuttcap%
\pgfsetmiterjoin%
\definecolor{currentfill}{rgb}{1.000000,1.000000,1.000000}%
\pgfsetfillcolor{currentfill}%
\pgfsetlinewidth{0.000000pt}%
\definecolor{currentstroke}{rgb}{0.000000,0.000000,0.000000}%
\pgfsetstrokecolor{currentstroke}%
\pgfsetdash{}{0pt}%
\pgfpathmoveto{\pgfqpoint{0.000000in}{0.000000in}}%
\pgfpathlineto{\pgfqpoint{12.375956in}{0.000000in}}%
\pgfpathlineto{\pgfqpoint{12.375956in}{8.028674in}}%
\pgfpathlineto{\pgfqpoint{0.000000in}{8.028674in}}%
\pgfpathlineto{\pgfqpoint{0.000000in}{0.000000in}}%
\pgfpathclose%
\pgfusepath{fill}%
\end{pgfscope}%
\begin{pgfscope}%
\pgfsetbuttcap%
\pgfsetmiterjoin%
\definecolor{currentfill}{rgb}{1.000000,1.000000,1.000000}%
\pgfsetfillcolor{currentfill}%
\pgfsetlinewidth{0.000000pt}%
\definecolor{currentstroke}{rgb}{0.000000,0.000000,0.000000}%
\pgfsetstrokecolor{currentstroke}%
\pgfsetstrokeopacity{0.000000}%
\pgfsetdash{}{0pt}%
\pgfpathmoveto{\pgfqpoint{0.100000in}{2.802285in}}%
\pgfpathlineto{\pgfqpoint{12.275956in}{2.802285in}}%
\pgfpathlineto{\pgfqpoint{12.275956in}{7.015894in}}%
\pgfpathlineto{\pgfqpoint{0.100000in}{7.015894in}}%
\pgfpathlineto{\pgfqpoint{0.100000in}{2.802285in}}%
\pgfpathclose%
\pgfusepath{fill}%
\end{pgfscope}%
\begin{pgfscope}%
\pgfsetbuttcap%
\pgfsetroundjoin%
\definecolor{currentfill}{rgb}{0.000000,0.000000,0.000000}%
\pgfsetfillcolor{currentfill}%
\pgfsetlinewidth{0.803000pt}%
\definecolor{currentstroke}{rgb}{0.000000,0.000000,0.000000}%
\pgfsetstrokecolor{currentstroke}%
\pgfsetdash{}{0pt}%
\pgfsys@defobject{currentmarker}{\pgfqpoint{0.000000in}{-0.048611in}}{\pgfqpoint{0.000000in}{0.000000in}}{%
\pgfpathmoveto{\pgfqpoint{0.000000in}{0.000000in}}%
\pgfpathlineto{\pgfqpoint{0.000000in}{-0.048611in}}%
\pgfusepath{stroke,fill}%
}%
\begin{pgfscope}%
\pgfsys@transformshift{0.674129in}{2.802285in}%
\pgfsys@useobject{currentmarker}{}%
\end{pgfscope}%
\end{pgfscope}%
\begin{pgfscope}%
\definecolor{textcolor}{rgb}{0.000000,0.000000,0.000000}%
\pgfsetstrokecolor{textcolor}%
\pgfsetfillcolor{textcolor}%
\pgftext[x=0.724129in, y=0.562378in, left, base,rotate=90.000000]{\color{textcolor}{\rmfamily\fontsize{14.000000}{16.800000}\selectfont\catcode`\^=\active\def^{\ifmmode\sp\else\^{}\fi}\catcode`\%=\active\def%{\%}mass snf hlw disposal}}%
\end{pgfscope}%
\begin{pgfscope}%
\pgfsetbuttcap%
\pgfsetroundjoin%
\definecolor{currentfill}{rgb}{0.000000,0.000000,0.000000}%
\pgfsetfillcolor{currentfill}%
\pgfsetlinewidth{0.803000pt}%
\definecolor{currentstroke}{rgb}{0.000000,0.000000,0.000000}%
\pgfsetstrokecolor{currentstroke}%
\pgfsetdash{}{0pt}%
\pgfsys@defobject{currentmarker}{\pgfqpoint{0.000000in}{-0.048611in}}{\pgfqpoint{0.000000in}{0.000000in}}{%
\pgfpathmoveto{\pgfqpoint{0.000000in}{0.000000in}}%
\pgfpathlineto{\pgfqpoint{0.000000in}{-0.048611in}}%
\pgfusepath{stroke,fill}%
}%
\begin{pgfscope}%
\pgfsys@transformshift{1.363361in}{2.802285in}%
\pgfsys@useobject{currentmarker}{}%
\end{pgfscope}%
\end{pgfscope}%
\begin{pgfscope}%
\definecolor{textcolor}{rgb}{0.000000,0.000000,0.000000}%
\pgfsetstrokecolor{textcolor}%
\pgfsetfillcolor{textcolor}%
\pgftext[x=1.413361in, y=0.838717in, left, base,rotate=90.000000]{\color{textcolor}{\rmfamily\fontsize{14.000000}{16.800000}\selectfont\catcode`\^=\active\def^{\ifmmode\sp\else\^{}\fi}\catcode`\%=\active\def%{\%}activity at 100 yrs}}%
\end{pgfscope}%
\begin{pgfscope}%
\pgfsetbuttcap%
\pgfsetroundjoin%
\definecolor{currentfill}{rgb}{0.000000,0.000000,0.000000}%
\pgfsetfillcolor{currentfill}%
\pgfsetlinewidth{0.803000pt}%
\definecolor{currentstroke}{rgb}{0.000000,0.000000,0.000000}%
\pgfsetstrokecolor{currentstroke}%
\pgfsetdash{}{0pt}%
\pgfsys@defobject{currentmarker}{\pgfqpoint{0.000000in}{-0.048611in}}{\pgfqpoint{0.000000in}{0.000000in}}{%
\pgfpathmoveto{\pgfqpoint{0.000000in}{0.000000in}}%
\pgfpathlineto{\pgfqpoint{0.000000in}{-0.048611in}}%
\pgfusepath{stroke,fill}%
}%
\begin{pgfscope}%
\pgfsys@transformshift{2.052592in}{2.802285in}%
\pgfsys@useobject{currentmarker}{}%
\end{pgfscope}%
\end{pgfscope}%
\begin{pgfscope}%
\definecolor{textcolor}{rgb}{0.000000,0.000000,0.000000}%
\pgfsetstrokecolor{textcolor}%
\pgfsetfillcolor{textcolor}%
\pgftext[x=2.102592in, y=0.555851in, left, base,rotate=90.000000]{\color{textcolor}{\rmfamily\fontsize{14.000000}{16.800000}\selectfont\catcode`\^=\active\def^{\ifmmode\sp\else\^{}\fi}\catcode`\%=\active\def%{\%}activity at 100k years}}%
\end{pgfscope}%
\begin{pgfscope}%
\pgfsetbuttcap%
\pgfsetroundjoin%
\definecolor{currentfill}{rgb}{0.000000,0.000000,0.000000}%
\pgfsetfillcolor{currentfill}%
\pgfsetlinewidth{0.803000pt}%
\definecolor{currentstroke}{rgb}{0.000000,0.000000,0.000000}%
\pgfsetstrokecolor{currentstroke}%
\pgfsetdash{}{0pt}%
\pgfsys@defobject{currentmarker}{\pgfqpoint{0.000000in}{-0.048611in}}{\pgfqpoint{0.000000in}{0.000000in}}{%
\pgfpathmoveto{\pgfqpoint{0.000000in}{0.000000in}}%
\pgfpathlineto{\pgfqpoint{0.000000in}{-0.048611in}}%
\pgfusepath{stroke,fill}%
}%
\begin{pgfscope}%
\pgfsys@transformshift{2.741823in}{2.802285in}%
\pgfsys@useobject{currentmarker}{}%
\end{pgfscope}%
\end{pgfscope}%
\begin{pgfscope}%
\definecolor{textcolor}{rgb}{0.000000,0.000000,0.000000}%
\pgfsetstrokecolor{textcolor}%
\pgfsetfillcolor{textcolor}%
\pgftext[x=2.791823in, y=0.383954in, left, base,rotate=90.000000]{\color{textcolor}{\rmfamily\fontsize{14.000000}{16.800000}\selectfont\catcode`\^=\active\def^{\ifmmode\sp\else\^{}\fi}\catcode`\%=\active\def%{\%}mass du ru rth disposal}}%
\end{pgfscope}%
\begin{pgfscope}%
\pgfsetbuttcap%
\pgfsetroundjoin%
\definecolor{currentfill}{rgb}{0.000000,0.000000,0.000000}%
\pgfsetfillcolor{currentfill}%
\pgfsetlinewidth{0.803000pt}%
\definecolor{currentstroke}{rgb}{0.000000,0.000000,0.000000}%
\pgfsetstrokecolor{currentstroke}%
\pgfsetdash{}{0pt}%
\pgfsys@defobject{currentmarker}{\pgfqpoint{0.000000in}{-0.048611in}}{\pgfqpoint{0.000000in}{0.000000in}}{%
\pgfpathmoveto{\pgfqpoint{0.000000in}{0.000000in}}%
\pgfpathlineto{\pgfqpoint{0.000000in}{-0.048611in}}%
\pgfusepath{stroke,fill}%
}%
\begin{pgfscope}%
\pgfsys@transformshift{3.431054in}{2.802285in}%
\pgfsys@useobject{currentmarker}{}%
\end{pgfscope}%
\end{pgfscope}%
\begin{pgfscope}%
\definecolor{textcolor}{rgb}{0.000000,0.000000,0.000000}%
\pgfsetstrokecolor{textcolor}%
\pgfsetfillcolor{textcolor}%
\pgftext[x=3.481054in, y=0.734274in, left, base,rotate=90.000000]{\color{textcolor}{\rmfamily\fontsize{14.000000}{16.800000}\selectfont\catcode`\^=\active\def^{\ifmmode\sp\else\^{}\fi}\catcode`\%=\active\def%{\%}volume llw disposal}}%
\end{pgfscope}%
\begin{pgfscope}%
\pgfsetbuttcap%
\pgfsetroundjoin%
\definecolor{currentfill}{rgb}{0.000000,0.000000,0.000000}%
\pgfsetfillcolor{currentfill}%
\pgfsetlinewidth{0.803000pt}%
\definecolor{currentstroke}{rgb}{0.000000,0.000000,0.000000}%
\pgfsetstrokecolor{currentstroke}%
\pgfsetdash{}{0pt}%
\pgfsys@defobject{currentmarker}{\pgfqpoint{0.000000in}{-0.048611in}}{\pgfqpoint{0.000000in}{0.000000in}}{%
\pgfpathmoveto{\pgfqpoint{0.000000in}{0.000000in}}%
\pgfpathlineto{\pgfqpoint{0.000000in}{-0.048611in}}%
\pgfusepath{stroke,fill}%
}%
\begin{pgfscope}%
\pgfsys@transformshift{4.120285in}{2.802285in}%
\pgfsys@useobject{currentmarker}{}%
\end{pgfscope}%
\end{pgfscope}%
\begin{pgfscope}%
\definecolor{textcolor}{rgb}{0.000000,0.000000,0.000000}%
\pgfsetstrokecolor{textcolor}%
\pgfsetfillcolor{textcolor}%
\pgftext[x=4.170285in, y=0.989942in, left, base,rotate=90.000000]{\color{textcolor}{\rmfamily\fontsize{14.000000}{16.800000}\selectfont\catcode`\^=\active\def^{\ifmmode\sp\else\^{}\fi}\catcode`\%=\active\def%{\%}safety challenges}}%
\end{pgfscope}%
\begin{pgfscope}%
\pgfsetbuttcap%
\pgfsetroundjoin%
\definecolor{currentfill}{rgb}{0.000000,0.000000,0.000000}%
\pgfsetfillcolor{currentfill}%
\pgfsetlinewidth{0.803000pt}%
\definecolor{currentstroke}{rgb}{0.000000,0.000000,0.000000}%
\pgfsetstrokecolor{currentstroke}%
\pgfsetdash{}{0pt}%
\pgfsys@defobject{currentmarker}{\pgfqpoint{0.000000in}{-0.048611in}}{\pgfqpoint{0.000000in}{0.000000in}}{%
\pgfpathmoveto{\pgfqpoint{0.000000in}{0.000000in}}%
\pgfpathlineto{\pgfqpoint{0.000000in}{-0.048611in}}%
\pgfusepath{stroke,fill}%
}%
\begin{pgfscope}%
\pgfsys@transformshift{4.809516in}{2.802285in}%
\pgfsys@useobject{currentmarker}{}%
\end{pgfscope}%
\end{pgfscope}%
\begin{pgfscope}%
\definecolor{textcolor}{rgb}{0.000000,0.000000,0.000000}%
\pgfsetstrokecolor{textcolor}%
\pgfsetfillcolor{textcolor}%
\pgftext[x=4.859516in, y=1.698197in, left, base,rotate=90.000000]{\color{textcolor}{\rmfamily\fontsize{14.000000}{16.800000}\selectfont\catcode`\^=\active\def^{\ifmmode\sp\else\^{}\fi}\catcode`\%=\active\def%{\%}land use}}%
\end{pgfscope}%
\begin{pgfscope}%
\pgfsetbuttcap%
\pgfsetroundjoin%
\definecolor{currentfill}{rgb}{0.000000,0.000000,0.000000}%
\pgfsetfillcolor{currentfill}%
\pgfsetlinewidth{0.803000pt}%
\definecolor{currentstroke}{rgb}{0.000000,0.000000,0.000000}%
\pgfsetstrokecolor{currentstroke}%
\pgfsetdash{}{0pt}%
\pgfsys@defobject{currentmarker}{\pgfqpoint{0.000000in}{-0.048611in}}{\pgfqpoint{0.000000in}{0.000000in}}{%
\pgfpathmoveto{\pgfqpoint{0.000000in}{0.000000in}}%
\pgfpathlineto{\pgfqpoint{0.000000in}{-0.048611in}}%
\pgfusepath{stroke,fill}%
}%
\begin{pgfscope}%
\pgfsys@transformshift{5.498747in}{2.802285in}%
\pgfsys@useobject{currentmarker}{}%
\end{pgfscope}%
\end{pgfscope}%
\begin{pgfscope}%
\definecolor{textcolor}{rgb}{0.000000,0.000000,0.000000}%
\pgfsetstrokecolor{textcolor}%
\pgfsetfillcolor{textcolor}%
\pgftext[x=5.548747in, y=1.594842in, left, base,rotate=90.000000]{\color{textcolor}{\rmfamily\fontsize{14.000000}{16.800000}\selectfont\catcode`\^=\active\def^{\ifmmode\sp\else\^{}\fi}\catcode`\%=\active\def%{\%}water use}}%
\end{pgfscope}%
\begin{pgfscope}%
\pgfsetbuttcap%
\pgfsetroundjoin%
\definecolor{currentfill}{rgb}{0.000000,0.000000,0.000000}%
\pgfsetfillcolor{currentfill}%
\pgfsetlinewidth{0.803000pt}%
\definecolor{currentstroke}{rgb}{0.000000,0.000000,0.000000}%
\pgfsetstrokecolor{currentstroke}%
\pgfsetdash{}{0pt}%
\pgfsys@defobject{currentmarker}{\pgfqpoint{0.000000in}{-0.048611in}}{\pgfqpoint{0.000000in}{0.000000in}}{%
\pgfpathmoveto{\pgfqpoint{0.000000in}{0.000000in}}%
\pgfpathlineto{\pgfqpoint{0.000000in}{-0.048611in}}%
\pgfusepath{stroke,fill}%
}%
\begin{pgfscope}%
\pgfsys@transformshift{6.187978in}{2.802285in}%
\pgfsys@useobject{currentmarker}{}%
\end{pgfscope}%
\end{pgfscope}%
\begin{pgfscope}%
\definecolor{textcolor}{rgb}{0.000000,0.000000,0.000000}%
\pgfsetstrokecolor{textcolor}%
\pgfsetfillcolor{textcolor}%
\pgftext[x=6.237978in, y=0.287127in, left, base,rotate=90.000000]{\color{textcolor}{\rmfamily\fontsize{14.000000}{16.800000}\selectfont\catcode`\^=\active\def^{\ifmmode\sp\else\^{}\fi}\catcode`\%=\active\def%{\%}carbon dioxide emissions}}%
\end{pgfscope}%
\begin{pgfscope}%
\pgfsetbuttcap%
\pgfsetroundjoin%
\definecolor{currentfill}{rgb}{0.000000,0.000000,0.000000}%
\pgfsetfillcolor{currentfill}%
\pgfsetlinewidth{0.803000pt}%
\definecolor{currentstroke}{rgb}{0.000000,0.000000,0.000000}%
\pgfsetstrokecolor{currentstroke}%
\pgfsetdash{}{0pt}%
\pgfsys@defobject{currentmarker}{\pgfqpoint{0.000000in}{-0.048611in}}{\pgfqpoint{0.000000in}{0.000000in}}{%
\pgfpathmoveto{\pgfqpoint{0.000000in}{0.000000in}}%
\pgfpathlineto{\pgfqpoint{0.000000in}{-0.048611in}}%
\pgfusepath{stroke,fill}%
}%
\begin{pgfscope}%
\pgfsys@transformshift{6.877209in}{2.802285in}%
\pgfsys@useobject{currentmarker}{}%
\end{pgfscope}%
\end{pgfscope}%
\begin{pgfscope}%
\definecolor{textcolor}{rgb}{0.000000,0.000000,0.000000}%
\pgfsetstrokecolor{textcolor}%
\pgfsetfillcolor{textcolor}%
\pgftext[x=6.927209in, y=0.931193in, left, base,rotate=90.000000]{\color{textcolor}{\rmfamily\fontsize{14.000000}{16.800000}\selectfont\catcode`\^=\active\def^{\ifmmode\sp\else\^{}\fi}\catcode`\%=\active\def%{\%}total worker dose}}%
\end{pgfscope}%
\begin{pgfscope}%
\pgfsetbuttcap%
\pgfsetroundjoin%
\definecolor{currentfill}{rgb}{0.000000,0.000000,0.000000}%
\pgfsetfillcolor{currentfill}%
\pgfsetlinewidth{0.803000pt}%
\definecolor{currentstroke}{rgb}{0.000000,0.000000,0.000000}%
\pgfsetstrokecolor{currentstroke}%
\pgfsetdash{}{0pt}%
\pgfsys@defobject{currentmarker}{\pgfqpoint{0.000000in}{-0.048611in}}{\pgfqpoint{0.000000in}{0.000000in}}{%
\pgfpathmoveto{\pgfqpoint{0.000000in}{0.000000in}}%
\pgfpathlineto{\pgfqpoint{0.000000in}{-0.048611in}}%
\pgfusepath{stroke,fill}%
}%
\begin{pgfscope}%
\pgfsys@transformshift{7.566440in}{2.802285in}%
\pgfsys@useobject{currentmarker}{}%
\end{pgfscope}%
\end{pgfscope}%
\begin{pgfscope}%
\definecolor{textcolor}{rgb}{0.000000,0.000000,0.000000}%
\pgfsetstrokecolor{textcolor}%
\pgfsetfillcolor{textcolor}%
\pgftext[x=7.616440in, y=0.235993in, left, base,rotate=90.000000]{\color{textcolor}{\rmfamily\fontsize{14.000000}{16.800000}\selectfont\catcode`\^=\active\def^{\ifmmode\sp\else\^{}\fi}\catcode`\%=\active\def%{\%}natural uranium required}}%
\end{pgfscope}%
\begin{pgfscope}%
\pgfsetbuttcap%
\pgfsetroundjoin%
\definecolor{currentfill}{rgb}{0.000000,0.000000,0.000000}%
\pgfsetfillcolor{currentfill}%
\pgfsetlinewidth{0.803000pt}%
\definecolor{currentstroke}{rgb}{0.000000,0.000000,0.000000}%
\pgfsetstrokecolor{currentstroke}%
\pgfsetdash{}{0pt}%
\pgfsys@defobject{currentmarker}{\pgfqpoint{0.000000in}{-0.048611in}}{\pgfqpoint{0.000000in}{0.000000in}}{%
\pgfpathmoveto{\pgfqpoint{0.000000in}{0.000000in}}%
\pgfpathlineto{\pgfqpoint{0.000000in}{-0.048611in}}%
\pgfusepath{stroke,fill}%
}%
\begin{pgfscope}%
\pgfsys@transformshift{8.255671in}{2.802285in}%
\pgfsys@useobject{currentmarker}{}%
\end{pgfscope}%
\end{pgfscope}%
\begin{pgfscope}%
\definecolor{textcolor}{rgb}{0.000000,0.000000,0.000000}%
\pgfsetstrokecolor{textcolor}%
\pgfsetfillcolor{textcolor}%
\pgftext[x=8.305671in, y=0.100000in, left, base,rotate=90.000000]{\color{textcolor}{\rmfamily\fontsize{14.000000}{16.800000}\selectfont\catcode`\^=\active\def^{\ifmmode\sp\else\^{}\fi}\catcode`\%=\active\def%{\%}natural thorium utilization}}%
\end{pgfscope}%
\begin{pgfscope}%
\pgfsetbuttcap%
\pgfsetroundjoin%
\definecolor{currentfill}{rgb}{0.000000,0.000000,0.000000}%
\pgfsetfillcolor{currentfill}%
\pgfsetlinewidth{0.803000pt}%
\definecolor{currentstroke}{rgb}{0.000000,0.000000,0.000000}%
\pgfsetstrokecolor{currentstroke}%
\pgfsetdash{}{0pt}%
\pgfsys@defobject{currentmarker}{\pgfqpoint{0.000000in}{-0.048611in}}{\pgfqpoint{0.000000in}{0.000000in}}{%
\pgfpathmoveto{\pgfqpoint{0.000000in}{0.000000in}}%
\pgfpathlineto{\pgfqpoint{0.000000in}{-0.048611in}}%
\pgfusepath{stroke,fill}%
}%
\begin{pgfscope}%
\pgfsys@transformshift{8.944902in}{2.802285in}%
\pgfsys@useobject{currentmarker}{}%
\end{pgfscope}%
\end{pgfscope}%
\begin{pgfscope}%
\definecolor{textcolor}{rgb}{0.000000,0.000000,0.000000}%
\pgfsetstrokecolor{textcolor}%
\pgfsetfillcolor{textcolor}%
\pgftext[x=8.994902in, y=0.931193in, left, base,rotate=90.000000]{\color{textcolor}{\rmfamily\fontsize{14.000000}{16.800000}\selectfont\catcode`\^=\active\def^{\ifmmode\sp\else\^{}\fi}\catcode`\%=\active\def%{\%}development cost}}%
\end{pgfscope}%
\begin{pgfscope}%
\pgfsetbuttcap%
\pgfsetroundjoin%
\definecolor{currentfill}{rgb}{0.000000,0.000000,0.000000}%
\pgfsetfillcolor{currentfill}%
\pgfsetlinewidth{0.803000pt}%
\definecolor{currentstroke}{rgb}{0.000000,0.000000,0.000000}%
\pgfsetstrokecolor{currentstroke}%
\pgfsetdash{}{0pt}%
\pgfsys@defobject{currentmarker}{\pgfqpoint{0.000000in}{-0.048611in}}{\pgfqpoint{0.000000in}{0.000000in}}{%
\pgfpathmoveto{\pgfqpoint{0.000000in}{0.000000in}}%
\pgfpathlineto{\pgfqpoint{0.000000in}{-0.048611in}}%
\pgfusepath{stroke,fill}%
}%
\begin{pgfscope}%
\pgfsys@transformshift{9.634134in}{2.802285in}%
\pgfsys@useobject{currentmarker}{}%
\end{pgfscope}%
\end{pgfscope}%
\begin{pgfscope}%
\definecolor{textcolor}{rgb}{0.000000,0.000000,0.000000}%
\pgfsetstrokecolor{textcolor}%
\pgfsetfillcolor{textcolor}%
\pgftext[x=9.684134in, y=0.888763in, left, base,rotate=90.000000]{\color{textcolor}{\rmfamily\fontsize{14.000000}{16.800000}\selectfont\catcode`\^=\active\def^{\ifmmode\sp\else\^{}\fi}\catcode`\%=\active\def%{\%}development time}}%
\end{pgfscope}%
\begin{pgfscope}%
\pgfsetbuttcap%
\pgfsetroundjoin%
\definecolor{currentfill}{rgb}{0.000000,0.000000,0.000000}%
\pgfsetfillcolor{currentfill}%
\pgfsetlinewidth{0.803000pt}%
\definecolor{currentstroke}{rgb}{0.000000,0.000000,0.000000}%
\pgfsetstrokecolor{currentstroke}%
\pgfsetdash{}{0pt}%
\pgfsys@defobject{currentmarker}{\pgfqpoint{0.000000in}{-0.048611in}}{\pgfqpoint{0.000000in}{0.000000in}}{%
\pgfpathmoveto{\pgfqpoint{0.000000in}{0.000000in}}%
\pgfpathlineto{\pgfqpoint{0.000000in}{-0.048611in}}%
\pgfusepath{stroke,fill}%
}%
\begin{pgfscope}%
\pgfsys@transformshift{10.323365in}{2.802285in}%
\pgfsys@useobject{currentmarker}{}%
\end{pgfscope}%
\end{pgfscope}%
\begin{pgfscope}%
\definecolor{textcolor}{rgb}{0.000000,0.000000,0.000000}%
\pgfsetstrokecolor{textcolor}%
\pgfsetfillcolor{textcolor}%
\pgftext[x=10.373365in, y=1.643800in, left, base,rotate=90.000000]{\color{textcolor}{\rmfamily\fontsize{14.000000}{16.800000}\selectfont\catcode`\^=\active\def^{\ifmmode\sp\else\^{}\fi}\catcode`\%=\active\def%{\%}foak cost}}%
\end{pgfscope}%
\begin{pgfscope}%
\pgfsetbuttcap%
\pgfsetroundjoin%
\definecolor{currentfill}{rgb}{0.000000,0.000000,0.000000}%
\pgfsetfillcolor{currentfill}%
\pgfsetlinewidth{0.803000pt}%
\definecolor{currentstroke}{rgb}{0.000000,0.000000,0.000000}%
\pgfsetstrokecolor{currentstroke}%
\pgfsetdash{}{0pt}%
\pgfsys@defobject{currentmarker}{\pgfqpoint{0.000000in}{-0.048611in}}{\pgfqpoint{0.000000in}{0.000000in}}{%
\pgfpathmoveto{\pgfqpoint{0.000000in}{0.000000in}}%
\pgfpathlineto{\pgfqpoint{0.000000in}{-0.048611in}}%
\pgfusepath{stroke,fill}%
}%
\begin{pgfscope}%
\pgfsys@transformshift{11.012596in}{2.802285in}%
\pgfsys@useobject{currentmarker}{}%
\end{pgfscope}%
\end{pgfscope}%
\begin{pgfscope}%
\definecolor{textcolor}{rgb}{0.000000,0.000000,0.000000}%
\pgfsetstrokecolor{textcolor}%
\pgfsetfillcolor{textcolor}%
\pgftext[x=11.062596in, y=1.111792in, left, base,rotate=90.000000]{\color{textcolor}{\rmfamily\fontsize{14.000000}{16.800000}\selectfont\catcode`\^=\active\def^{\ifmmode\sp\else\^{}\fi}\catcode`\%=\active\def%{\%}incompatibility}}%
\end{pgfscope}%
\begin{pgfscope}%
\pgfsetbuttcap%
\pgfsetroundjoin%
\definecolor{currentfill}{rgb}{0.000000,0.000000,0.000000}%
\pgfsetfillcolor{currentfill}%
\pgfsetlinewidth{0.803000pt}%
\definecolor{currentstroke}{rgb}{0.000000,0.000000,0.000000}%
\pgfsetstrokecolor{currentstroke}%
\pgfsetdash{}{0pt}%
\pgfsys@defobject{currentmarker}{\pgfqpoint{0.000000in}{-0.048611in}}{\pgfqpoint{0.000000in}{0.000000in}}{%
\pgfpathmoveto{\pgfqpoint{0.000000in}{0.000000in}}%
\pgfpathlineto{\pgfqpoint{0.000000in}{-0.048611in}}%
\pgfusepath{stroke,fill}%
}%
\begin{pgfscope}%
\pgfsys@transformshift{11.701827in}{2.802285in}%
\pgfsys@useobject{currentmarker}{}%
\end{pgfscope}%
\end{pgfscope}%
\begin{pgfscope}%
\definecolor{textcolor}{rgb}{0.000000,0.000000,0.000000}%
\pgfsetstrokecolor{textcolor}%
\pgfsetfillcolor{textcolor}%
\pgftext[x=11.751827in, y=1.302184in, left, base,rotate=90.000000]{\color{textcolor}{\rmfamily\fontsize{14.000000}{16.800000}\selectfont\catcode`\^=\active\def^{\ifmmode\sp\else\^{}\fi}\catcode`\%=\active\def%{\%}unfamiliarity}}%
\end{pgfscope}%
\begin{pgfscope}%
\pgfpathrectangle{\pgfqpoint{0.100000in}{2.802285in}}{\pgfqpoint{12.175956in}{4.213609in}}%
\pgfusepath{clip}%
\pgfsetrectcap%
\pgfsetroundjoin%
\pgfsetlinewidth{1.505625pt}%
\definecolor{currentstroke}{rgb}{0.839216,0.152941,0.156863}%
\pgfsetstrokecolor{currentstroke}%
\pgfsetdash{}{0pt}%
\pgfpathmoveto{\pgfqpoint{0.674129in}{6.824366in}}%
\pgfpathlineto{\pgfqpoint{1.363361in}{6.824366in}}%
\pgfpathlineto{\pgfqpoint{2.052592in}{6.824366in}}%
\pgfpathlineto{\pgfqpoint{2.741823in}{2.993813in}}%
\pgfpathlineto{\pgfqpoint{3.431054in}{2.993813in}}%
\pgfpathlineto{\pgfqpoint{4.120285in}{6.824366in}}%
\pgfpathlineto{\pgfqpoint{4.809516in}{6.824366in}}%
\pgfpathlineto{\pgfqpoint{5.498747in}{6.824366in}}%
\pgfpathlineto{\pgfqpoint{6.187978in}{4.493663in}}%
\pgfpathlineto{\pgfqpoint{6.877209in}{6.824366in}}%
\pgfpathlineto{\pgfqpoint{7.566440in}{3.624684in}}%
\pgfpathlineto{\pgfqpoint{8.255671in}{2.993813in}}%
\pgfpathlineto{\pgfqpoint{8.944902in}{2.993813in}}%
\pgfpathlineto{\pgfqpoint{9.634134in}{2.993813in}}%
\pgfpathlineto{\pgfqpoint{10.323365in}{2.993813in}}%
\pgfpathlineto{\pgfqpoint{11.012596in}{4.909090in}}%
\pgfpathlineto{\pgfqpoint{11.701827in}{2.993813in}}%
\pgfusepath{stroke}%
\end{pgfscope}%
\begin{pgfscope}%
\pgfpathrectangle{\pgfqpoint{0.100000in}{2.802285in}}{\pgfqpoint{12.175956in}{4.213609in}}%
\pgfusepath{clip}%
\pgfsetrectcap%
\pgfsetroundjoin%
\pgfsetlinewidth{1.505625pt}%
\definecolor{currentstroke}{rgb}{0.172549,0.627451,0.172549}%
\pgfsetstrokecolor{currentstroke}%
\pgfsetdash{}{0pt}%
\pgfpathmoveto{\pgfqpoint{0.674129in}{3.387268in}}%
\pgfpathlineto{\pgfqpoint{1.363361in}{2.993813in}}%
\pgfpathlineto{\pgfqpoint{2.052592in}{6.676552in}}%
\pgfpathlineto{\pgfqpoint{2.741823in}{5.156249in}}%
\pgfpathlineto{\pgfqpoint{3.431054in}{3.269032in}}%
\pgfpathlineto{\pgfqpoint{4.120285in}{2.993813in}}%
\pgfpathlineto{\pgfqpoint{4.809516in}{4.702033in}}%
\pgfpathlineto{\pgfqpoint{5.498747in}{2.993813in}}%
\pgfpathlineto{\pgfqpoint{6.187978in}{2.993813in}}%
\pgfpathlineto{\pgfqpoint{6.877209in}{3.286737in}}%
\pgfpathlineto{\pgfqpoint{7.566440in}{5.172263in}}%
\pgfpathlineto{\pgfqpoint{8.255671in}{6.824366in}}%
\pgfpathlineto{\pgfqpoint{8.944902in}{2.993813in}}%
\pgfpathlineto{\pgfqpoint{9.634134in}{2.993813in}}%
\pgfpathlineto{\pgfqpoint{10.323365in}{6.824366in}}%
\pgfpathlineto{\pgfqpoint{11.012596in}{4.909090in}}%
\pgfpathlineto{\pgfqpoint{11.701827in}{2.993813in}}%
\pgfusepath{stroke}%
\end{pgfscope}%
\begin{pgfscope}%
\pgfpathrectangle{\pgfqpoint{0.100000in}{2.802285in}}{\pgfqpoint{12.175956in}{4.213609in}}%
\pgfusepath{clip}%
\pgfsetrectcap%
\pgfsetroundjoin%
\pgfsetlinewidth{1.505625pt}%
\definecolor{currentstroke}{rgb}{0.172549,0.627451,0.172549}%
\pgfsetstrokecolor{currentstroke}%
\pgfsetdash{}{0pt}%
\pgfpathmoveto{\pgfqpoint{0.674129in}{3.008849in}}%
\pgfpathlineto{\pgfqpoint{1.363361in}{4.203461in}}%
\pgfpathlineto{\pgfqpoint{2.052592in}{3.150756in}}%
\pgfpathlineto{\pgfqpoint{2.741823in}{6.642936in}}%
\pgfpathlineto{\pgfqpoint{3.431054in}{4.344215in}}%
\pgfpathlineto{\pgfqpoint{4.120285in}{6.824366in}}%
\pgfpathlineto{\pgfqpoint{4.809516in}{6.047903in}}%
\pgfpathlineto{\pgfqpoint{5.498747in}{3.180956in}}%
\pgfpathlineto{\pgfqpoint{6.187978in}{5.054758in}}%
\pgfpathlineto{\pgfqpoint{6.877209in}{3.129009in}}%
\pgfpathlineto{\pgfqpoint{7.566440in}{6.636844in}}%
\pgfpathlineto{\pgfqpoint{8.255671in}{2.993813in}}%
\pgfpathlineto{\pgfqpoint{8.944902in}{2.993813in}}%
\pgfpathlineto{\pgfqpoint{9.634134in}{2.993813in}}%
\pgfpathlineto{\pgfqpoint{10.323365in}{6.824366in}}%
\pgfpathlineto{\pgfqpoint{11.012596in}{6.824366in}}%
\pgfpathlineto{\pgfqpoint{11.701827in}{2.993813in}}%
\pgfusepath{stroke}%
\end{pgfscope}%
\begin{pgfscope}%
\pgfpathrectangle{\pgfqpoint{0.100000in}{2.802285in}}{\pgfqpoint{12.175956in}{4.213609in}}%
\pgfusepath{clip}%
\pgfsetrectcap%
\pgfsetroundjoin%
\pgfsetlinewidth{1.505625pt}%
\definecolor{currentstroke}{rgb}{0.121569,0.466667,0.705882}%
\pgfsetstrokecolor{currentstroke}%
\pgfsetdash{}{0pt}%
\pgfpathmoveto{\pgfqpoint{0.674129in}{3.100321in}}%
\pgfpathlineto{\pgfqpoint{1.363361in}{4.405069in}}%
\pgfpathlineto{\pgfqpoint{2.052592in}{4.276750in}}%
\pgfpathlineto{\pgfqpoint{2.741823in}{6.824366in}}%
\pgfpathlineto{\pgfqpoint{3.431054in}{6.824366in}}%
\pgfpathlineto{\pgfqpoint{4.120285in}{2.993813in}}%
\pgfpathlineto{\pgfqpoint{4.809516in}{6.462017in}}%
\pgfpathlineto{\pgfqpoint{5.498747in}{3.020860in}}%
\pgfpathlineto{\pgfqpoint{6.187978in}{6.824366in}}%
\pgfpathlineto{\pgfqpoint{6.877209in}{3.444466in}}%
\pgfpathlineto{\pgfqpoint{7.566440in}{6.824366in}}%
\pgfpathlineto{\pgfqpoint{8.255671in}{3.834666in}}%
\pgfpathlineto{\pgfqpoint{8.944902in}{2.993813in}}%
\pgfpathlineto{\pgfqpoint{9.634134in}{2.993813in}}%
\pgfpathlineto{\pgfqpoint{10.323365in}{6.824366in}}%
\pgfpathlineto{\pgfqpoint{11.012596in}{4.909090in}}%
\pgfpathlineto{\pgfqpoint{11.701827in}{2.993813in}}%
\pgfusepath{stroke}%
\end{pgfscope}%
\begin{pgfscope}%
\pgfpathrectangle{\pgfqpoint{0.100000in}{2.802285in}}{\pgfqpoint{12.175956in}{4.213609in}}%
\pgfusepath{clip}%
\pgfsetrectcap%
\pgfsetroundjoin%
\pgfsetlinewidth{1.505625pt}%
\definecolor{currentstroke}{rgb}{0.121569,0.466667,0.705882}%
\pgfsetstrokecolor{currentstroke}%
\pgfsetdash{}{0pt}%
\pgfpathmoveto{\pgfqpoint{0.674129in}{3.017621in}}%
\pgfpathlineto{\pgfqpoint{1.363361in}{4.001853in}}%
\pgfpathlineto{\pgfqpoint{2.052592in}{3.085110in}}%
\pgfpathlineto{\pgfqpoint{2.741823in}{2.993813in}}%
\pgfpathlineto{\pgfqpoint{3.431054in}{4.918660in}}%
\pgfpathlineto{\pgfqpoint{4.120285in}{6.824366in}}%
\pgfpathlineto{\pgfqpoint{4.809516in}{3.097341in}}%
\pgfpathlineto{\pgfqpoint{5.498747in}{4.007662in}}%
\pgfpathlineto{\pgfqpoint{6.187978in}{3.781504in}}%
\pgfpathlineto{\pgfqpoint{6.877209in}{4.909090in}}%
\pgfpathlineto{\pgfqpoint{7.566440in}{2.995676in}}%
\pgfpathlineto{\pgfqpoint{8.255671in}{2.993813in}}%
\pgfpathlineto{\pgfqpoint{8.944902in}{2.993813in}}%
\pgfpathlineto{\pgfqpoint{9.634134in}{2.993813in}}%
\pgfpathlineto{\pgfqpoint{10.323365in}{6.824366in}}%
\pgfpathlineto{\pgfqpoint{11.012596in}{2.993813in}}%
\pgfpathlineto{\pgfqpoint{11.701827in}{2.993813in}}%
\pgfusepath{stroke}%
\end{pgfscope}%
\begin{pgfscope}%
\pgfpathrectangle{\pgfqpoint{0.100000in}{2.802285in}}{\pgfqpoint{12.175956in}{4.213609in}}%
\pgfusepath{clip}%
\pgfsetrectcap%
\pgfsetroundjoin%
\pgfsetlinewidth{1.505625pt}%
\definecolor{currentstroke}{rgb}{0.121569,0.466667,0.705882}%
\pgfsetstrokecolor{currentstroke}%
\pgfsetdash{}{0pt}%
\pgfpathmoveto{\pgfqpoint{0.674129in}{3.006343in}}%
\pgfpathlineto{\pgfqpoint{1.363361in}{3.245823in}}%
\pgfpathlineto{\pgfqpoint{2.052592in}{2.993813in}}%
\pgfpathlineto{\pgfqpoint{2.741823in}{2.993813in}}%
\pgfpathlineto{\pgfqpoint{3.431054in}{4.651490in}}%
\pgfpathlineto{\pgfqpoint{4.120285in}{6.824366in}}%
\pgfpathlineto{\pgfqpoint{4.809516in}{2.993813in}}%
\pgfpathlineto{\pgfqpoint{5.498747in}{3.893737in}}%
\pgfpathlineto{\pgfqpoint{6.187978in}{3.684391in}}%
\pgfpathlineto{\pgfqpoint{6.877209in}{4.751361in}}%
\pgfpathlineto{\pgfqpoint{7.566440in}{2.993813in}}%
\pgfpathlineto{\pgfqpoint{8.255671in}{2.993813in}}%
\pgfpathlineto{\pgfqpoint{8.944902in}{6.824366in}}%
\pgfpathlineto{\pgfqpoint{9.634134in}{2.993813in}}%
\pgfpathlineto{\pgfqpoint{10.323365in}{6.824366in}}%
\pgfpathlineto{\pgfqpoint{11.012596in}{2.993813in}}%
\pgfpathlineto{\pgfqpoint{11.701827in}{2.993813in}}%
\pgfusepath{stroke}%
\end{pgfscope}%
\begin{pgfscope}%
\pgfpathrectangle{\pgfqpoint{0.100000in}{2.802285in}}{\pgfqpoint{12.175956in}{4.213609in}}%
\pgfusepath{clip}%
\pgfsetrectcap%
\pgfsetroundjoin%
\pgfsetlinewidth{1.505625pt}%
\definecolor{currentstroke}{rgb}{0.121569,0.466667,0.705882}%
\pgfsetstrokecolor{currentstroke}%
\pgfsetdash{}{0pt}%
\pgfpathmoveto{\pgfqpoint{0.674129in}{2.996319in}}%
\pgfpathlineto{\pgfqpoint{1.363361in}{3.901049in}}%
\pgfpathlineto{\pgfqpoint{2.052592in}{3.025115in}}%
\pgfpathlineto{\pgfqpoint{2.741823in}{6.392667in}}%
\pgfpathlineto{\pgfqpoint{3.431054in}{4.300646in}}%
\pgfpathlineto{\pgfqpoint{4.120285in}{6.824366in}}%
\pgfpathlineto{\pgfqpoint{4.809516in}{5.892610in}}%
\pgfpathlineto{\pgfqpoint{5.498747in}{3.307175in}}%
\pgfpathlineto{\pgfqpoint{6.187978in}{4.903694in}}%
\pgfpathlineto{\pgfqpoint{6.877209in}{3.320536in}}%
\pgfpathlineto{\pgfqpoint{7.566440in}{6.384537in}}%
\pgfpathlineto{\pgfqpoint{8.255671in}{2.993813in}}%
\pgfpathlineto{\pgfqpoint{8.944902in}{2.993813in}}%
\pgfpathlineto{\pgfqpoint{9.634134in}{2.993813in}}%
\pgfpathlineto{\pgfqpoint{10.323365in}{6.824366in}}%
\pgfpathlineto{\pgfqpoint{11.012596in}{6.824366in}}%
\pgfpathlineto{\pgfqpoint{11.701827in}{2.993813in}}%
\pgfusepath{stroke}%
\end{pgfscope}%
\begin{pgfscope}%
\pgfpathrectangle{\pgfqpoint{0.100000in}{2.802285in}}{\pgfqpoint{12.175956in}{4.213609in}}%
\pgfusepath{clip}%
\pgfsetrectcap%
\pgfsetroundjoin%
\pgfsetlinewidth{1.505625pt}%
\definecolor{currentstroke}{rgb}{0.121569,0.466667,0.705882}%
\pgfsetstrokecolor{currentstroke}%
\pgfsetdash{}{0pt}%
\pgfpathmoveto{\pgfqpoint{0.674129in}{2.993813in}}%
\pgfpathlineto{\pgfqpoint{1.363361in}{4.405069in}}%
\pgfpathlineto{\pgfqpoint{2.052592in}{3.034679in}}%
\pgfpathlineto{\pgfqpoint{2.741823in}{5.361109in}}%
\pgfpathlineto{\pgfqpoint{3.431054in}{4.562902in}}%
\pgfpathlineto{\pgfqpoint{4.120285in}{6.824366in}}%
\pgfpathlineto{\pgfqpoint{4.809516in}{4.909090in}}%
\pgfpathlineto{\pgfqpoint{5.498747in}{3.215925in}}%
\pgfpathlineto{\pgfqpoint{6.187978in}{4.062052in}}%
\pgfpathlineto{\pgfqpoint{6.877209in}{2.993813in}}%
\pgfpathlineto{\pgfqpoint{7.566440in}{5.338881in}}%
\pgfpathlineto{\pgfqpoint{8.255671in}{4.395235in}}%
\pgfpathlineto{\pgfqpoint{8.944902in}{6.824366in}}%
\pgfpathlineto{\pgfqpoint{9.634134in}{2.993813in}}%
\pgfpathlineto{\pgfqpoint{10.323365in}{6.824366in}}%
\pgfpathlineto{\pgfqpoint{11.012596in}{6.824366in}}%
\pgfpathlineto{\pgfqpoint{11.701827in}{2.993813in}}%
\pgfusepath{stroke}%
\end{pgfscope}%
\begin{pgfscope}%
\pgfpathrectangle{\pgfqpoint{0.100000in}{2.802285in}}{\pgfqpoint{12.175956in}{4.213609in}}%
\pgfusepath{clip}%
\pgfsetrectcap%
\pgfsetroundjoin%
\pgfsetlinewidth{2.007500pt}%
\definecolor{currentstroke}{rgb}{0.501961,0.501961,0.501961}%
\pgfsetstrokecolor{currentstroke}%
\pgfsetstrokeopacity{0.500000}%
\pgfsetdash{}{0pt}%
\pgfpathmoveto{\pgfqpoint{0.674129in}{2.802285in}}%
\pgfpathlineto{\pgfqpoint{0.674129in}{7.015894in}}%
\pgfusepath{stroke}%
\end{pgfscope}%
\begin{pgfscope}%
\pgfpathrectangle{\pgfqpoint{0.100000in}{2.802285in}}{\pgfqpoint{12.175956in}{4.213609in}}%
\pgfusepath{clip}%
\pgfsetbuttcap%
\pgfsetroundjoin%
\pgfsetlinewidth{2.007500pt}%
\definecolor{currentstroke}{rgb}{0.501961,0.501961,0.501961}%
\pgfsetstrokecolor{currentstroke}%
\pgfsetstrokeopacity{0.500000}%
\pgfsetdash{}{0pt}%
\pgfpathmoveto{\pgfqpoint{0.653453in}{2.993813in}}%
\pgfpathlineto{\pgfqpoint{0.694806in}{2.993813in}}%
\pgfusepath{stroke}%
\end{pgfscope}%
\begin{pgfscope}%
\pgfpathrectangle{\pgfqpoint{0.100000in}{2.802285in}}{\pgfqpoint{12.175956in}{4.213609in}}%
\pgfusepath{clip}%
\pgfsetbuttcap%
\pgfsetroundjoin%
\pgfsetlinewidth{2.007500pt}%
\definecolor{currentstroke}{rgb}{0.501961,0.501961,0.501961}%
\pgfsetstrokecolor{currentstroke}%
\pgfsetstrokeopacity{0.500000}%
\pgfsetdash{}{0pt}%
\pgfpathmoveto{\pgfqpoint{0.653453in}{3.419430in}}%
\pgfpathlineto{\pgfqpoint{0.694806in}{3.419430in}}%
\pgfusepath{stroke}%
\end{pgfscope}%
\begin{pgfscope}%
\pgfpathrectangle{\pgfqpoint{0.100000in}{2.802285in}}{\pgfqpoint{12.175956in}{4.213609in}}%
\pgfusepath{clip}%
\pgfsetbuttcap%
\pgfsetroundjoin%
\pgfsetlinewidth{2.007500pt}%
\definecolor{currentstroke}{rgb}{0.501961,0.501961,0.501961}%
\pgfsetstrokecolor{currentstroke}%
\pgfsetstrokeopacity{0.500000}%
\pgfsetdash{}{0pt}%
\pgfpathmoveto{\pgfqpoint{0.653453in}{3.845047in}}%
\pgfpathlineto{\pgfqpoint{0.694806in}{3.845047in}}%
\pgfusepath{stroke}%
\end{pgfscope}%
\begin{pgfscope}%
\pgfpathrectangle{\pgfqpoint{0.100000in}{2.802285in}}{\pgfqpoint{12.175956in}{4.213609in}}%
\pgfusepath{clip}%
\pgfsetbuttcap%
\pgfsetroundjoin%
\pgfsetlinewidth{2.007500pt}%
\definecolor{currentstroke}{rgb}{0.501961,0.501961,0.501961}%
\pgfsetstrokecolor{currentstroke}%
\pgfsetstrokeopacity{0.500000}%
\pgfsetdash{}{0pt}%
\pgfpathmoveto{\pgfqpoint{0.653453in}{4.270664in}}%
\pgfpathlineto{\pgfqpoint{0.694806in}{4.270664in}}%
\pgfusepath{stroke}%
\end{pgfscope}%
\begin{pgfscope}%
\pgfpathrectangle{\pgfqpoint{0.100000in}{2.802285in}}{\pgfqpoint{12.175956in}{4.213609in}}%
\pgfusepath{clip}%
\pgfsetbuttcap%
\pgfsetroundjoin%
\pgfsetlinewidth{2.007500pt}%
\definecolor{currentstroke}{rgb}{0.501961,0.501961,0.501961}%
\pgfsetstrokecolor{currentstroke}%
\pgfsetstrokeopacity{0.500000}%
\pgfsetdash{}{0pt}%
\pgfpathmoveto{\pgfqpoint{0.653453in}{4.696281in}}%
\pgfpathlineto{\pgfqpoint{0.694806in}{4.696281in}}%
\pgfusepath{stroke}%
\end{pgfscope}%
\begin{pgfscope}%
\pgfpathrectangle{\pgfqpoint{0.100000in}{2.802285in}}{\pgfqpoint{12.175956in}{4.213609in}}%
\pgfusepath{clip}%
\pgfsetbuttcap%
\pgfsetroundjoin%
\pgfsetlinewidth{2.007500pt}%
\definecolor{currentstroke}{rgb}{0.501961,0.501961,0.501961}%
\pgfsetstrokecolor{currentstroke}%
\pgfsetstrokeopacity{0.500000}%
\pgfsetdash{}{0pt}%
\pgfpathmoveto{\pgfqpoint{0.653453in}{5.121898in}}%
\pgfpathlineto{\pgfqpoint{0.694806in}{5.121898in}}%
\pgfusepath{stroke}%
\end{pgfscope}%
\begin{pgfscope}%
\pgfpathrectangle{\pgfqpoint{0.100000in}{2.802285in}}{\pgfqpoint{12.175956in}{4.213609in}}%
\pgfusepath{clip}%
\pgfsetbuttcap%
\pgfsetroundjoin%
\pgfsetlinewidth{2.007500pt}%
\definecolor{currentstroke}{rgb}{0.501961,0.501961,0.501961}%
\pgfsetstrokecolor{currentstroke}%
\pgfsetstrokeopacity{0.500000}%
\pgfsetdash{}{0pt}%
\pgfpathmoveto{\pgfqpoint{0.653453in}{5.547515in}}%
\pgfpathlineto{\pgfqpoint{0.694806in}{5.547515in}}%
\pgfusepath{stroke}%
\end{pgfscope}%
\begin{pgfscope}%
\pgfpathrectangle{\pgfqpoint{0.100000in}{2.802285in}}{\pgfqpoint{12.175956in}{4.213609in}}%
\pgfusepath{clip}%
\pgfsetbuttcap%
\pgfsetroundjoin%
\pgfsetlinewidth{2.007500pt}%
\definecolor{currentstroke}{rgb}{0.501961,0.501961,0.501961}%
\pgfsetstrokecolor{currentstroke}%
\pgfsetstrokeopacity{0.500000}%
\pgfsetdash{}{0pt}%
\pgfpathmoveto{\pgfqpoint{0.653453in}{5.973132in}}%
\pgfpathlineto{\pgfqpoint{0.694806in}{5.973132in}}%
\pgfusepath{stroke}%
\end{pgfscope}%
\begin{pgfscope}%
\pgfpathrectangle{\pgfqpoint{0.100000in}{2.802285in}}{\pgfqpoint{12.175956in}{4.213609in}}%
\pgfusepath{clip}%
\pgfsetbuttcap%
\pgfsetroundjoin%
\pgfsetlinewidth{2.007500pt}%
\definecolor{currentstroke}{rgb}{0.501961,0.501961,0.501961}%
\pgfsetstrokecolor{currentstroke}%
\pgfsetstrokeopacity{0.500000}%
\pgfsetdash{}{0pt}%
\pgfpathmoveto{\pgfqpoint{0.653453in}{6.398749in}}%
\pgfpathlineto{\pgfqpoint{0.694806in}{6.398749in}}%
\pgfusepath{stroke}%
\end{pgfscope}%
\begin{pgfscope}%
\pgfpathrectangle{\pgfqpoint{0.100000in}{2.802285in}}{\pgfqpoint{12.175956in}{4.213609in}}%
\pgfusepath{clip}%
\pgfsetbuttcap%
\pgfsetroundjoin%
\pgfsetlinewidth{2.007500pt}%
\definecolor{currentstroke}{rgb}{0.501961,0.501961,0.501961}%
\pgfsetstrokecolor{currentstroke}%
\pgfsetstrokeopacity{0.500000}%
\pgfsetdash{}{0pt}%
\pgfpathmoveto{\pgfqpoint{0.653453in}{6.824366in}}%
\pgfpathlineto{\pgfqpoint{0.694806in}{6.824366in}}%
\pgfusepath{stroke}%
\end{pgfscope}%
\begin{pgfscope}%
\pgfpathrectangle{\pgfqpoint{0.100000in}{2.802285in}}{\pgfqpoint{12.175956in}{4.213609in}}%
\pgfusepath{clip}%
\pgfsetrectcap%
\pgfsetroundjoin%
\pgfsetlinewidth{2.007500pt}%
\definecolor{currentstroke}{rgb}{0.501961,0.501961,0.501961}%
\pgfsetstrokecolor{currentstroke}%
\pgfsetstrokeopacity{0.500000}%
\pgfsetdash{}{0pt}%
\pgfpathmoveto{\pgfqpoint{1.363361in}{2.802285in}}%
\pgfpathlineto{\pgfqpoint{1.363361in}{7.015894in}}%
\pgfusepath{stroke}%
\end{pgfscope}%
\begin{pgfscope}%
\pgfpathrectangle{\pgfqpoint{0.100000in}{2.802285in}}{\pgfqpoint{12.175956in}{4.213609in}}%
\pgfusepath{clip}%
\pgfsetbuttcap%
\pgfsetroundjoin%
\pgfsetlinewidth{2.007500pt}%
\definecolor{currentstroke}{rgb}{0.501961,0.501961,0.501961}%
\pgfsetstrokecolor{currentstroke}%
\pgfsetstrokeopacity{0.500000}%
\pgfsetdash{}{0pt}%
\pgfpathmoveto{\pgfqpoint{1.342684in}{2.993813in}}%
\pgfpathlineto{\pgfqpoint{1.384038in}{2.993813in}}%
\pgfusepath{stroke}%
\end{pgfscope}%
\begin{pgfscope}%
\pgfpathrectangle{\pgfqpoint{0.100000in}{2.802285in}}{\pgfqpoint{12.175956in}{4.213609in}}%
\pgfusepath{clip}%
\pgfsetbuttcap%
\pgfsetroundjoin%
\pgfsetlinewidth{2.007500pt}%
\definecolor{currentstroke}{rgb}{0.501961,0.501961,0.501961}%
\pgfsetstrokecolor{currentstroke}%
\pgfsetstrokeopacity{0.500000}%
\pgfsetdash{}{0pt}%
\pgfpathmoveto{\pgfqpoint{1.342684in}{3.419430in}}%
\pgfpathlineto{\pgfqpoint{1.384038in}{3.419430in}}%
\pgfusepath{stroke}%
\end{pgfscope}%
\begin{pgfscope}%
\pgfpathrectangle{\pgfqpoint{0.100000in}{2.802285in}}{\pgfqpoint{12.175956in}{4.213609in}}%
\pgfusepath{clip}%
\pgfsetbuttcap%
\pgfsetroundjoin%
\pgfsetlinewidth{2.007500pt}%
\definecolor{currentstroke}{rgb}{0.501961,0.501961,0.501961}%
\pgfsetstrokecolor{currentstroke}%
\pgfsetstrokeopacity{0.500000}%
\pgfsetdash{}{0pt}%
\pgfpathmoveto{\pgfqpoint{1.342684in}{3.845047in}}%
\pgfpathlineto{\pgfqpoint{1.384038in}{3.845047in}}%
\pgfusepath{stroke}%
\end{pgfscope}%
\begin{pgfscope}%
\pgfpathrectangle{\pgfqpoint{0.100000in}{2.802285in}}{\pgfqpoint{12.175956in}{4.213609in}}%
\pgfusepath{clip}%
\pgfsetbuttcap%
\pgfsetroundjoin%
\pgfsetlinewidth{2.007500pt}%
\definecolor{currentstroke}{rgb}{0.501961,0.501961,0.501961}%
\pgfsetstrokecolor{currentstroke}%
\pgfsetstrokeopacity{0.500000}%
\pgfsetdash{}{0pt}%
\pgfpathmoveto{\pgfqpoint{1.342684in}{4.270664in}}%
\pgfpathlineto{\pgfqpoint{1.384038in}{4.270664in}}%
\pgfusepath{stroke}%
\end{pgfscope}%
\begin{pgfscope}%
\pgfpathrectangle{\pgfqpoint{0.100000in}{2.802285in}}{\pgfqpoint{12.175956in}{4.213609in}}%
\pgfusepath{clip}%
\pgfsetbuttcap%
\pgfsetroundjoin%
\pgfsetlinewidth{2.007500pt}%
\definecolor{currentstroke}{rgb}{0.501961,0.501961,0.501961}%
\pgfsetstrokecolor{currentstroke}%
\pgfsetstrokeopacity{0.500000}%
\pgfsetdash{}{0pt}%
\pgfpathmoveto{\pgfqpoint{1.342684in}{4.696281in}}%
\pgfpathlineto{\pgfqpoint{1.384038in}{4.696281in}}%
\pgfusepath{stroke}%
\end{pgfscope}%
\begin{pgfscope}%
\pgfpathrectangle{\pgfqpoint{0.100000in}{2.802285in}}{\pgfqpoint{12.175956in}{4.213609in}}%
\pgfusepath{clip}%
\pgfsetbuttcap%
\pgfsetroundjoin%
\pgfsetlinewidth{2.007500pt}%
\definecolor{currentstroke}{rgb}{0.501961,0.501961,0.501961}%
\pgfsetstrokecolor{currentstroke}%
\pgfsetstrokeopacity{0.500000}%
\pgfsetdash{}{0pt}%
\pgfpathmoveto{\pgfqpoint{1.342684in}{5.121898in}}%
\pgfpathlineto{\pgfqpoint{1.384038in}{5.121898in}}%
\pgfusepath{stroke}%
\end{pgfscope}%
\begin{pgfscope}%
\pgfpathrectangle{\pgfqpoint{0.100000in}{2.802285in}}{\pgfqpoint{12.175956in}{4.213609in}}%
\pgfusepath{clip}%
\pgfsetbuttcap%
\pgfsetroundjoin%
\pgfsetlinewidth{2.007500pt}%
\definecolor{currentstroke}{rgb}{0.501961,0.501961,0.501961}%
\pgfsetstrokecolor{currentstroke}%
\pgfsetstrokeopacity{0.500000}%
\pgfsetdash{}{0pt}%
\pgfpathmoveto{\pgfqpoint{1.342684in}{5.547515in}}%
\pgfpathlineto{\pgfqpoint{1.384038in}{5.547515in}}%
\pgfusepath{stroke}%
\end{pgfscope}%
\begin{pgfscope}%
\pgfpathrectangle{\pgfqpoint{0.100000in}{2.802285in}}{\pgfqpoint{12.175956in}{4.213609in}}%
\pgfusepath{clip}%
\pgfsetbuttcap%
\pgfsetroundjoin%
\pgfsetlinewidth{2.007500pt}%
\definecolor{currentstroke}{rgb}{0.501961,0.501961,0.501961}%
\pgfsetstrokecolor{currentstroke}%
\pgfsetstrokeopacity{0.500000}%
\pgfsetdash{}{0pt}%
\pgfpathmoveto{\pgfqpoint{1.342684in}{5.973132in}}%
\pgfpathlineto{\pgfqpoint{1.384038in}{5.973132in}}%
\pgfusepath{stroke}%
\end{pgfscope}%
\begin{pgfscope}%
\pgfpathrectangle{\pgfqpoint{0.100000in}{2.802285in}}{\pgfqpoint{12.175956in}{4.213609in}}%
\pgfusepath{clip}%
\pgfsetbuttcap%
\pgfsetroundjoin%
\pgfsetlinewidth{2.007500pt}%
\definecolor{currentstroke}{rgb}{0.501961,0.501961,0.501961}%
\pgfsetstrokecolor{currentstroke}%
\pgfsetstrokeopacity{0.500000}%
\pgfsetdash{}{0pt}%
\pgfpathmoveto{\pgfqpoint{1.342684in}{6.398749in}}%
\pgfpathlineto{\pgfqpoint{1.384038in}{6.398749in}}%
\pgfusepath{stroke}%
\end{pgfscope}%
\begin{pgfscope}%
\pgfpathrectangle{\pgfqpoint{0.100000in}{2.802285in}}{\pgfqpoint{12.175956in}{4.213609in}}%
\pgfusepath{clip}%
\pgfsetbuttcap%
\pgfsetroundjoin%
\pgfsetlinewidth{2.007500pt}%
\definecolor{currentstroke}{rgb}{0.501961,0.501961,0.501961}%
\pgfsetstrokecolor{currentstroke}%
\pgfsetstrokeopacity{0.500000}%
\pgfsetdash{}{0pt}%
\pgfpathmoveto{\pgfqpoint{1.342684in}{6.824366in}}%
\pgfpathlineto{\pgfqpoint{1.384038in}{6.824366in}}%
\pgfusepath{stroke}%
\end{pgfscope}%
\begin{pgfscope}%
\pgfpathrectangle{\pgfqpoint{0.100000in}{2.802285in}}{\pgfqpoint{12.175956in}{4.213609in}}%
\pgfusepath{clip}%
\pgfsetrectcap%
\pgfsetroundjoin%
\pgfsetlinewidth{2.007500pt}%
\definecolor{currentstroke}{rgb}{0.501961,0.501961,0.501961}%
\pgfsetstrokecolor{currentstroke}%
\pgfsetstrokeopacity{0.500000}%
\pgfsetdash{}{0pt}%
\pgfpathmoveto{\pgfqpoint{2.052592in}{2.802285in}}%
\pgfpathlineto{\pgfqpoint{2.052592in}{7.015894in}}%
\pgfusepath{stroke}%
\end{pgfscope}%
\begin{pgfscope}%
\pgfpathrectangle{\pgfqpoint{0.100000in}{2.802285in}}{\pgfqpoint{12.175956in}{4.213609in}}%
\pgfusepath{clip}%
\pgfsetbuttcap%
\pgfsetroundjoin%
\pgfsetlinewidth{2.007500pt}%
\definecolor{currentstroke}{rgb}{0.501961,0.501961,0.501961}%
\pgfsetstrokecolor{currentstroke}%
\pgfsetstrokeopacity{0.500000}%
\pgfsetdash{}{0pt}%
\pgfpathmoveto{\pgfqpoint{2.031915in}{2.993813in}}%
\pgfpathlineto{\pgfqpoint{2.073269in}{2.993813in}}%
\pgfusepath{stroke}%
\end{pgfscope}%
\begin{pgfscope}%
\pgfpathrectangle{\pgfqpoint{0.100000in}{2.802285in}}{\pgfqpoint{12.175956in}{4.213609in}}%
\pgfusepath{clip}%
\pgfsetbuttcap%
\pgfsetroundjoin%
\pgfsetlinewidth{2.007500pt}%
\definecolor{currentstroke}{rgb}{0.501961,0.501961,0.501961}%
\pgfsetstrokecolor{currentstroke}%
\pgfsetstrokeopacity{0.500000}%
\pgfsetdash{}{0pt}%
\pgfpathmoveto{\pgfqpoint{2.031915in}{3.419430in}}%
\pgfpathlineto{\pgfqpoint{2.073269in}{3.419430in}}%
\pgfusepath{stroke}%
\end{pgfscope}%
\begin{pgfscope}%
\pgfpathrectangle{\pgfqpoint{0.100000in}{2.802285in}}{\pgfqpoint{12.175956in}{4.213609in}}%
\pgfusepath{clip}%
\pgfsetbuttcap%
\pgfsetroundjoin%
\pgfsetlinewidth{2.007500pt}%
\definecolor{currentstroke}{rgb}{0.501961,0.501961,0.501961}%
\pgfsetstrokecolor{currentstroke}%
\pgfsetstrokeopacity{0.500000}%
\pgfsetdash{}{0pt}%
\pgfpathmoveto{\pgfqpoint{2.031915in}{3.845047in}}%
\pgfpathlineto{\pgfqpoint{2.073269in}{3.845047in}}%
\pgfusepath{stroke}%
\end{pgfscope}%
\begin{pgfscope}%
\pgfpathrectangle{\pgfqpoint{0.100000in}{2.802285in}}{\pgfqpoint{12.175956in}{4.213609in}}%
\pgfusepath{clip}%
\pgfsetbuttcap%
\pgfsetroundjoin%
\pgfsetlinewidth{2.007500pt}%
\definecolor{currentstroke}{rgb}{0.501961,0.501961,0.501961}%
\pgfsetstrokecolor{currentstroke}%
\pgfsetstrokeopacity{0.500000}%
\pgfsetdash{}{0pt}%
\pgfpathmoveto{\pgfqpoint{2.031915in}{4.270664in}}%
\pgfpathlineto{\pgfqpoint{2.073269in}{4.270664in}}%
\pgfusepath{stroke}%
\end{pgfscope}%
\begin{pgfscope}%
\pgfpathrectangle{\pgfqpoint{0.100000in}{2.802285in}}{\pgfqpoint{12.175956in}{4.213609in}}%
\pgfusepath{clip}%
\pgfsetbuttcap%
\pgfsetroundjoin%
\pgfsetlinewidth{2.007500pt}%
\definecolor{currentstroke}{rgb}{0.501961,0.501961,0.501961}%
\pgfsetstrokecolor{currentstroke}%
\pgfsetstrokeopacity{0.500000}%
\pgfsetdash{}{0pt}%
\pgfpathmoveto{\pgfqpoint{2.031915in}{4.696281in}}%
\pgfpathlineto{\pgfqpoint{2.073269in}{4.696281in}}%
\pgfusepath{stroke}%
\end{pgfscope}%
\begin{pgfscope}%
\pgfpathrectangle{\pgfqpoint{0.100000in}{2.802285in}}{\pgfqpoint{12.175956in}{4.213609in}}%
\pgfusepath{clip}%
\pgfsetbuttcap%
\pgfsetroundjoin%
\pgfsetlinewidth{2.007500pt}%
\definecolor{currentstroke}{rgb}{0.501961,0.501961,0.501961}%
\pgfsetstrokecolor{currentstroke}%
\pgfsetstrokeopacity{0.500000}%
\pgfsetdash{}{0pt}%
\pgfpathmoveto{\pgfqpoint{2.031915in}{5.121898in}}%
\pgfpathlineto{\pgfqpoint{2.073269in}{5.121898in}}%
\pgfusepath{stroke}%
\end{pgfscope}%
\begin{pgfscope}%
\pgfpathrectangle{\pgfqpoint{0.100000in}{2.802285in}}{\pgfqpoint{12.175956in}{4.213609in}}%
\pgfusepath{clip}%
\pgfsetbuttcap%
\pgfsetroundjoin%
\pgfsetlinewidth{2.007500pt}%
\definecolor{currentstroke}{rgb}{0.501961,0.501961,0.501961}%
\pgfsetstrokecolor{currentstroke}%
\pgfsetstrokeopacity{0.500000}%
\pgfsetdash{}{0pt}%
\pgfpathmoveto{\pgfqpoint{2.031915in}{5.547515in}}%
\pgfpathlineto{\pgfqpoint{2.073269in}{5.547515in}}%
\pgfusepath{stroke}%
\end{pgfscope}%
\begin{pgfscope}%
\pgfpathrectangle{\pgfqpoint{0.100000in}{2.802285in}}{\pgfqpoint{12.175956in}{4.213609in}}%
\pgfusepath{clip}%
\pgfsetbuttcap%
\pgfsetroundjoin%
\pgfsetlinewidth{2.007500pt}%
\definecolor{currentstroke}{rgb}{0.501961,0.501961,0.501961}%
\pgfsetstrokecolor{currentstroke}%
\pgfsetstrokeopacity{0.500000}%
\pgfsetdash{}{0pt}%
\pgfpathmoveto{\pgfqpoint{2.031915in}{5.973132in}}%
\pgfpathlineto{\pgfqpoint{2.073269in}{5.973132in}}%
\pgfusepath{stroke}%
\end{pgfscope}%
\begin{pgfscope}%
\pgfpathrectangle{\pgfqpoint{0.100000in}{2.802285in}}{\pgfqpoint{12.175956in}{4.213609in}}%
\pgfusepath{clip}%
\pgfsetbuttcap%
\pgfsetroundjoin%
\pgfsetlinewidth{2.007500pt}%
\definecolor{currentstroke}{rgb}{0.501961,0.501961,0.501961}%
\pgfsetstrokecolor{currentstroke}%
\pgfsetstrokeopacity{0.500000}%
\pgfsetdash{}{0pt}%
\pgfpathmoveto{\pgfqpoint{2.031915in}{6.398749in}}%
\pgfpathlineto{\pgfqpoint{2.073269in}{6.398749in}}%
\pgfusepath{stroke}%
\end{pgfscope}%
\begin{pgfscope}%
\pgfpathrectangle{\pgfqpoint{0.100000in}{2.802285in}}{\pgfqpoint{12.175956in}{4.213609in}}%
\pgfusepath{clip}%
\pgfsetbuttcap%
\pgfsetroundjoin%
\pgfsetlinewidth{2.007500pt}%
\definecolor{currentstroke}{rgb}{0.501961,0.501961,0.501961}%
\pgfsetstrokecolor{currentstroke}%
\pgfsetstrokeopacity{0.500000}%
\pgfsetdash{}{0pt}%
\pgfpathmoveto{\pgfqpoint{2.031915in}{6.824366in}}%
\pgfpathlineto{\pgfqpoint{2.073269in}{6.824366in}}%
\pgfusepath{stroke}%
\end{pgfscope}%
\begin{pgfscope}%
\pgfpathrectangle{\pgfqpoint{0.100000in}{2.802285in}}{\pgfqpoint{12.175956in}{4.213609in}}%
\pgfusepath{clip}%
\pgfsetrectcap%
\pgfsetroundjoin%
\pgfsetlinewidth{2.007500pt}%
\definecolor{currentstroke}{rgb}{0.501961,0.501961,0.501961}%
\pgfsetstrokecolor{currentstroke}%
\pgfsetstrokeopacity{0.500000}%
\pgfsetdash{}{0pt}%
\pgfpathmoveto{\pgfqpoint{2.741823in}{2.802285in}}%
\pgfpathlineto{\pgfqpoint{2.741823in}{7.015894in}}%
\pgfusepath{stroke}%
\end{pgfscope}%
\begin{pgfscope}%
\pgfpathrectangle{\pgfqpoint{0.100000in}{2.802285in}}{\pgfqpoint{12.175956in}{4.213609in}}%
\pgfusepath{clip}%
\pgfsetbuttcap%
\pgfsetroundjoin%
\pgfsetlinewidth{2.007500pt}%
\definecolor{currentstroke}{rgb}{0.501961,0.501961,0.501961}%
\pgfsetstrokecolor{currentstroke}%
\pgfsetstrokeopacity{0.500000}%
\pgfsetdash{}{0pt}%
\pgfpathmoveto{\pgfqpoint{2.721146in}{2.993813in}}%
\pgfpathlineto{\pgfqpoint{2.762500in}{2.993813in}}%
\pgfusepath{stroke}%
\end{pgfscope}%
\begin{pgfscope}%
\pgfpathrectangle{\pgfqpoint{0.100000in}{2.802285in}}{\pgfqpoint{12.175956in}{4.213609in}}%
\pgfusepath{clip}%
\pgfsetbuttcap%
\pgfsetroundjoin%
\pgfsetlinewidth{2.007500pt}%
\definecolor{currentstroke}{rgb}{0.501961,0.501961,0.501961}%
\pgfsetstrokecolor{currentstroke}%
\pgfsetstrokeopacity{0.500000}%
\pgfsetdash{}{0pt}%
\pgfpathmoveto{\pgfqpoint{2.721146in}{3.419430in}}%
\pgfpathlineto{\pgfqpoint{2.762500in}{3.419430in}}%
\pgfusepath{stroke}%
\end{pgfscope}%
\begin{pgfscope}%
\pgfpathrectangle{\pgfqpoint{0.100000in}{2.802285in}}{\pgfqpoint{12.175956in}{4.213609in}}%
\pgfusepath{clip}%
\pgfsetbuttcap%
\pgfsetroundjoin%
\pgfsetlinewidth{2.007500pt}%
\definecolor{currentstroke}{rgb}{0.501961,0.501961,0.501961}%
\pgfsetstrokecolor{currentstroke}%
\pgfsetstrokeopacity{0.500000}%
\pgfsetdash{}{0pt}%
\pgfpathmoveto{\pgfqpoint{2.721146in}{3.845047in}}%
\pgfpathlineto{\pgfqpoint{2.762500in}{3.845047in}}%
\pgfusepath{stroke}%
\end{pgfscope}%
\begin{pgfscope}%
\pgfpathrectangle{\pgfqpoint{0.100000in}{2.802285in}}{\pgfqpoint{12.175956in}{4.213609in}}%
\pgfusepath{clip}%
\pgfsetbuttcap%
\pgfsetroundjoin%
\pgfsetlinewidth{2.007500pt}%
\definecolor{currentstroke}{rgb}{0.501961,0.501961,0.501961}%
\pgfsetstrokecolor{currentstroke}%
\pgfsetstrokeopacity{0.500000}%
\pgfsetdash{}{0pt}%
\pgfpathmoveto{\pgfqpoint{2.721146in}{4.270664in}}%
\pgfpathlineto{\pgfqpoint{2.762500in}{4.270664in}}%
\pgfusepath{stroke}%
\end{pgfscope}%
\begin{pgfscope}%
\pgfpathrectangle{\pgfqpoint{0.100000in}{2.802285in}}{\pgfqpoint{12.175956in}{4.213609in}}%
\pgfusepath{clip}%
\pgfsetbuttcap%
\pgfsetroundjoin%
\pgfsetlinewidth{2.007500pt}%
\definecolor{currentstroke}{rgb}{0.501961,0.501961,0.501961}%
\pgfsetstrokecolor{currentstroke}%
\pgfsetstrokeopacity{0.500000}%
\pgfsetdash{}{0pt}%
\pgfpathmoveto{\pgfqpoint{2.721146in}{4.696281in}}%
\pgfpathlineto{\pgfqpoint{2.762500in}{4.696281in}}%
\pgfusepath{stroke}%
\end{pgfscope}%
\begin{pgfscope}%
\pgfpathrectangle{\pgfqpoint{0.100000in}{2.802285in}}{\pgfqpoint{12.175956in}{4.213609in}}%
\pgfusepath{clip}%
\pgfsetbuttcap%
\pgfsetroundjoin%
\pgfsetlinewidth{2.007500pt}%
\definecolor{currentstroke}{rgb}{0.501961,0.501961,0.501961}%
\pgfsetstrokecolor{currentstroke}%
\pgfsetstrokeopacity{0.500000}%
\pgfsetdash{}{0pt}%
\pgfpathmoveto{\pgfqpoint{2.721146in}{5.121898in}}%
\pgfpathlineto{\pgfqpoint{2.762500in}{5.121898in}}%
\pgfusepath{stroke}%
\end{pgfscope}%
\begin{pgfscope}%
\pgfpathrectangle{\pgfqpoint{0.100000in}{2.802285in}}{\pgfqpoint{12.175956in}{4.213609in}}%
\pgfusepath{clip}%
\pgfsetbuttcap%
\pgfsetroundjoin%
\pgfsetlinewidth{2.007500pt}%
\definecolor{currentstroke}{rgb}{0.501961,0.501961,0.501961}%
\pgfsetstrokecolor{currentstroke}%
\pgfsetstrokeopacity{0.500000}%
\pgfsetdash{}{0pt}%
\pgfpathmoveto{\pgfqpoint{2.721146in}{5.547515in}}%
\pgfpathlineto{\pgfqpoint{2.762500in}{5.547515in}}%
\pgfusepath{stroke}%
\end{pgfscope}%
\begin{pgfscope}%
\pgfpathrectangle{\pgfqpoint{0.100000in}{2.802285in}}{\pgfqpoint{12.175956in}{4.213609in}}%
\pgfusepath{clip}%
\pgfsetbuttcap%
\pgfsetroundjoin%
\pgfsetlinewidth{2.007500pt}%
\definecolor{currentstroke}{rgb}{0.501961,0.501961,0.501961}%
\pgfsetstrokecolor{currentstroke}%
\pgfsetstrokeopacity{0.500000}%
\pgfsetdash{}{0pt}%
\pgfpathmoveto{\pgfqpoint{2.721146in}{5.973132in}}%
\pgfpathlineto{\pgfqpoint{2.762500in}{5.973132in}}%
\pgfusepath{stroke}%
\end{pgfscope}%
\begin{pgfscope}%
\pgfpathrectangle{\pgfqpoint{0.100000in}{2.802285in}}{\pgfqpoint{12.175956in}{4.213609in}}%
\pgfusepath{clip}%
\pgfsetbuttcap%
\pgfsetroundjoin%
\pgfsetlinewidth{2.007500pt}%
\definecolor{currentstroke}{rgb}{0.501961,0.501961,0.501961}%
\pgfsetstrokecolor{currentstroke}%
\pgfsetstrokeopacity{0.500000}%
\pgfsetdash{}{0pt}%
\pgfpathmoveto{\pgfqpoint{2.721146in}{6.398749in}}%
\pgfpathlineto{\pgfqpoint{2.762500in}{6.398749in}}%
\pgfusepath{stroke}%
\end{pgfscope}%
\begin{pgfscope}%
\pgfpathrectangle{\pgfqpoint{0.100000in}{2.802285in}}{\pgfqpoint{12.175956in}{4.213609in}}%
\pgfusepath{clip}%
\pgfsetbuttcap%
\pgfsetroundjoin%
\pgfsetlinewidth{2.007500pt}%
\definecolor{currentstroke}{rgb}{0.501961,0.501961,0.501961}%
\pgfsetstrokecolor{currentstroke}%
\pgfsetstrokeopacity{0.500000}%
\pgfsetdash{}{0pt}%
\pgfpathmoveto{\pgfqpoint{2.721146in}{6.824366in}}%
\pgfpathlineto{\pgfqpoint{2.762500in}{6.824366in}}%
\pgfusepath{stroke}%
\end{pgfscope}%
\begin{pgfscope}%
\pgfpathrectangle{\pgfqpoint{0.100000in}{2.802285in}}{\pgfqpoint{12.175956in}{4.213609in}}%
\pgfusepath{clip}%
\pgfsetrectcap%
\pgfsetroundjoin%
\pgfsetlinewidth{2.007500pt}%
\definecolor{currentstroke}{rgb}{0.501961,0.501961,0.501961}%
\pgfsetstrokecolor{currentstroke}%
\pgfsetstrokeopacity{0.500000}%
\pgfsetdash{}{0pt}%
\pgfpathmoveto{\pgfqpoint{3.431054in}{2.802285in}}%
\pgfpathlineto{\pgfqpoint{3.431054in}{7.015894in}}%
\pgfusepath{stroke}%
\end{pgfscope}%
\begin{pgfscope}%
\pgfpathrectangle{\pgfqpoint{0.100000in}{2.802285in}}{\pgfqpoint{12.175956in}{4.213609in}}%
\pgfusepath{clip}%
\pgfsetbuttcap%
\pgfsetroundjoin%
\pgfsetlinewidth{2.007500pt}%
\definecolor{currentstroke}{rgb}{0.501961,0.501961,0.501961}%
\pgfsetstrokecolor{currentstroke}%
\pgfsetstrokeopacity{0.500000}%
\pgfsetdash{}{0pt}%
\pgfpathmoveto{\pgfqpoint{3.410377in}{2.993813in}}%
\pgfpathlineto{\pgfqpoint{3.451731in}{2.993813in}}%
\pgfusepath{stroke}%
\end{pgfscope}%
\begin{pgfscope}%
\pgfpathrectangle{\pgfqpoint{0.100000in}{2.802285in}}{\pgfqpoint{12.175956in}{4.213609in}}%
\pgfusepath{clip}%
\pgfsetbuttcap%
\pgfsetroundjoin%
\pgfsetlinewidth{2.007500pt}%
\definecolor{currentstroke}{rgb}{0.501961,0.501961,0.501961}%
\pgfsetstrokecolor{currentstroke}%
\pgfsetstrokeopacity{0.500000}%
\pgfsetdash{}{0pt}%
\pgfpathmoveto{\pgfqpoint{3.410377in}{3.419430in}}%
\pgfpathlineto{\pgfqpoint{3.451731in}{3.419430in}}%
\pgfusepath{stroke}%
\end{pgfscope}%
\begin{pgfscope}%
\pgfpathrectangle{\pgfqpoint{0.100000in}{2.802285in}}{\pgfqpoint{12.175956in}{4.213609in}}%
\pgfusepath{clip}%
\pgfsetbuttcap%
\pgfsetroundjoin%
\pgfsetlinewidth{2.007500pt}%
\definecolor{currentstroke}{rgb}{0.501961,0.501961,0.501961}%
\pgfsetstrokecolor{currentstroke}%
\pgfsetstrokeopacity{0.500000}%
\pgfsetdash{}{0pt}%
\pgfpathmoveto{\pgfqpoint{3.410377in}{3.845047in}}%
\pgfpathlineto{\pgfqpoint{3.451731in}{3.845047in}}%
\pgfusepath{stroke}%
\end{pgfscope}%
\begin{pgfscope}%
\pgfpathrectangle{\pgfqpoint{0.100000in}{2.802285in}}{\pgfqpoint{12.175956in}{4.213609in}}%
\pgfusepath{clip}%
\pgfsetbuttcap%
\pgfsetroundjoin%
\pgfsetlinewidth{2.007500pt}%
\definecolor{currentstroke}{rgb}{0.501961,0.501961,0.501961}%
\pgfsetstrokecolor{currentstroke}%
\pgfsetstrokeopacity{0.500000}%
\pgfsetdash{}{0pt}%
\pgfpathmoveto{\pgfqpoint{3.410377in}{4.270664in}}%
\pgfpathlineto{\pgfqpoint{3.451731in}{4.270664in}}%
\pgfusepath{stroke}%
\end{pgfscope}%
\begin{pgfscope}%
\pgfpathrectangle{\pgfqpoint{0.100000in}{2.802285in}}{\pgfqpoint{12.175956in}{4.213609in}}%
\pgfusepath{clip}%
\pgfsetbuttcap%
\pgfsetroundjoin%
\pgfsetlinewidth{2.007500pt}%
\definecolor{currentstroke}{rgb}{0.501961,0.501961,0.501961}%
\pgfsetstrokecolor{currentstroke}%
\pgfsetstrokeopacity{0.500000}%
\pgfsetdash{}{0pt}%
\pgfpathmoveto{\pgfqpoint{3.410377in}{4.696281in}}%
\pgfpathlineto{\pgfqpoint{3.451731in}{4.696281in}}%
\pgfusepath{stroke}%
\end{pgfscope}%
\begin{pgfscope}%
\pgfpathrectangle{\pgfqpoint{0.100000in}{2.802285in}}{\pgfqpoint{12.175956in}{4.213609in}}%
\pgfusepath{clip}%
\pgfsetbuttcap%
\pgfsetroundjoin%
\pgfsetlinewidth{2.007500pt}%
\definecolor{currentstroke}{rgb}{0.501961,0.501961,0.501961}%
\pgfsetstrokecolor{currentstroke}%
\pgfsetstrokeopacity{0.500000}%
\pgfsetdash{}{0pt}%
\pgfpathmoveto{\pgfqpoint{3.410377in}{5.121898in}}%
\pgfpathlineto{\pgfqpoint{3.451731in}{5.121898in}}%
\pgfusepath{stroke}%
\end{pgfscope}%
\begin{pgfscope}%
\pgfpathrectangle{\pgfqpoint{0.100000in}{2.802285in}}{\pgfqpoint{12.175956in}{4.213609in}}%
\pgfusepath{clip}%
\pgfsetbuttcap%
\pgfsetroundjoin%
\pgfsetlinewidth{2.007500pt}%
\definecolor{currentstroke}{rgb}{0.501961,0.501961,0.501961}%
\pgfsetstrokecolor{currentstroke}%
\pgfsetstrokeopacity{0.500000}%
\pgfsetdash{}{0pt}%
\pgfpathmoveto{\pgfqpoint{3.410377in}{5.547515in}}%
\pgfpathlineto{\pgfqpoint{3.451731in}{5.547515in}}%
\pgfusepath{stroke}%
\end{pgfscope}%
\begin{pgfscope}%
\pgfpathrectangle{\pgfqpoint{0.100000in}{2.802285in}}{\pgfqpoint{12.175956in}{4.213609in}}%
\pgfusepath{clip}%
\pgfsetbuttcap%
\pgfsetroundjoin%
\pgfsetlinewidth{2.007500pt}%
\definecolor{currentstroke}{rgb}{0.501961,0.501961,0.501961}%
\pgfsetstrokecolor{currentstroke}%
\pgfsetstrokeopacity{0.500000}%
\pgfsetdash{}{0pt}%
\pgfpathmoveto{\pgfqpoint{3.410377in}{5.973132in}}%
\pgfpathlineto{\pgfqpoint{3.451731in}{5.973132in}}%
\pgfusepath{stroke}%
\end{pgfscope}%
\begin{pgfscope}%
\pgfpathrectangle{\pgfqpoint{0.100000in}{2.802285in}}{\pgfqpoint{12.175956in}{4.213609in}}%
\pgfusepath{clip}%
\pgfsetbuttcap%
\pgfsetroundjoin%
\pgfsetlinewidth{2.007500pt}%
\definecolor{currentstroke}{rgb}{0.501961,0.501961,0.501961}%
\pgfsetstrokecolor{currentstroke}%
\pgfsetstrokeopacity{0.500000}%
\pgfsetdash{}{0pt}%
\pgfpathmoveto{\pgfqpoint{3.410377in}{6.398749in}}%
\pgfpathlineto{\pgfqpoint{3.451731in}{6.398749in}}%
\pgfusepath{stroke}%
\end{pgfscope}%
\begin{pgfscope}%
\pgfpathrectangle{\pgfqpoint{0.100000in}{2.802285in}}{\pgfqpoint{12.175956in}{4.213609in}}%
\pgfusepath{clip}%
\pgfsetbuttcap%
\pgfsetroundjoin%
\pgfsetlinewidth{2.007500pt}%
\definecolor{currentstroke}{rgb}{0.501961,0.501961,0.501961}%
\pgfsetstrokecolor{currentstroke}%
\pgfsetstrokeopacity{0.500000}%
\pgfsetdash{}{0pt}%
\pgfpathmoveto{\pgfqpoint{3.410377in}{6.824366in}}%
\pgfpathlineto{\pgfqpoint{3.451731in}{6.824366in}}%
\pgfusepath{stroke}%
\end{pgfscope}%
\begin{pgfscope}%
\pgfpathrectangle{\pgfqpoint{0.100000in}{2.802285in}}{\pgfqpoint{12.175956in}{4.213609in}}%
\pgfusepath{clip}%
\pgfsetrectcap%
\pgfsetroundjoin%
\pgfsetlinewidth{2.007500pt}%
\definecolor{currentstroke}{rgb}{0.501961,0.501961,0.501961}%
\pgfsetstrokecolor{currentstroke}%
\pgfsetstrokeopacity{0.500000}%
\pgfsetdash{}{0pt}%
\pgfpathmoveto{\pgfqpoint{4.120285in}{2.802285in}}%
\pgfpathlineto{\pgfqpoint{4.120285in}{7.015894in}}%
\pgfusepath{stroke}%
\end{pgfscope}%
\begin{pgfscope}%
\pgfpathrectangle{\pgfqpoint{0.100000in}{2.802285in}}{\pgfqpoint{12.175956in}{4.213609in}}%
\pgfusepath{clip}%
\pgfsetbuttcap%
\pgfsetroundjoin%
\pgfsetlinewidth{2.007500pt}%
\definecolor{currentstroke}{rgb}{0.501961,0.501961,0.501961}%
\pgfsetstrokecolor{currentstroke}%
\pgfsetstrokeopacity{0.500000}%
\pgfsetdash{}{0pt}%
\pgfpathmoveto{\pgfqpoint{4.099608in}{2.993813in}}%
\pgfpathlineto{\pgfqpoint{4.140962in}{2.993813in}}%
\pgfusepath{stroke}%
\end{pgfscope}%
\begin{pgfscope}%
\pgfpathrectangle{\pgfqpoint{0.100000in}{2.802285in}}{\pgfqpoint{12.175956in}{4.213609in}}%
\pgfusepath{clip}%
\pgfsetbuttcap%
\pgfsetroundjoin%
\pgfsetlinewidth{2.007500pt}%
\definecolor{currentstroke}{rgb}{0.501961,0.501961,0.501961}%
\pgfsetstrokecolor{currentstroke}%
\pgfsetstrokeopacity{0.500000}%
\pgfsetdash{}{0pt}%
\pgfpathmoveto{\pgfqpoint{4.099608in}{3.419430in}}%
\pgfpathlineto{\pgfqpoint{4.140962in}{3.419430in}}%
\pgfusepath{stroke}%
\end{pgfscope}%
\begin{pgfscope}%
\pgfpathrectangle{\pgfqpoint{0.100000in}{2.802285in}}{\pgfqpoint{12.175956in}{4.213609in}}%
\pgfusepath{clip}%
\pgfsetbuttcap%
\pgfsetroundjoin%
\pgfsetlinewidth{2.007500pt}%
\definecolor{currentstroke}{rgb}{0.501961,0.501961,0.501961}%
\pgfsetstrokecolor{currentstroke}%
\pgfsetstrokeopacity{0.500000}%
\pgfsetdash{}{0pt}%
\pgfpathmoveto{\pgfqpoint{4.099608in}{3.845047in}}%
\pgfpathlineto{\pgfqpoint{4.140962in}{3.845047in}}%
\pgfusepath{stroke}%
\end{pgfscope}%
\begin{pgfscope}%
\pgfpathrectangle{\pgfqpoint{0.100000in}{2.802285in}}{\pgfqpoint{12.175956in}{4.213609in}}%
\pgfusepath{clip}%
\pgfsetbuttcap%
\pgfsetroundjoin%
\pgfsetlinewidth{2.007500pt}%
\definecolor{currentstroke}{rgb}{0.501961,0.501961,0.501961}%
\pgfsetstrokecolor{currentstroke}%
\pgfsetstrokeopacity{0.500000}%
\pgfsetdash{}{0pt}%
\pgfpathmoveto{\pgfqpoint{4.099608in}{4.270664in}}%
\pgfpathlineto{\pgfqpoint{4.140962in}{4.270664in}}%
\pgfusepath{stroke}%
\end{pgfscope}%
\begin{pgfscope}%
\pgfpathrectangle{\pgfqpoint{0.100000in}{2.802285in}}{\pgfqpoint{12.175956in}{4.213609in}}%
\pgfusepath{clip}%
\pgfsetbuttcap%
\pgfsetroundjoin%
\pgfsetlinewidth{2.007500pt}%
\definecolor{currentstroke}{rgb}{0.501961,0.501961,0.501961}%
\pgfsetstrokecolor{currentstroke}%
\pgfsetstrokeopacity{0.500000}%
\pgfsetdash{}{0pt}%
\pgfpathmoveto{\pgfqpoint{4.099608in}{4.696281in}}%
\pgfpathlineto{\pgfqpoint{4.140962in}{4.696281in}}%
\pgfusepath{stroke}%
\end{pgfscope}%
\begin{pgfscope}%
\pgfpathrectangle{\pgfqpoint{0.100000in}{2.802285in}}{\pgfqpoint{12.175956in}{4.213609in}}%
\pgfusepath{clip}%
\pgfsetbuttcap%
\pgfsetroundjoin%
\pgfsetlinewidth{2.007500pt}%
\definecolor{currentstroke}{rgb}{0.501961,0.501961,0.501961}%
\pgfsetstrokecolor{currentstroke}%
\pgfsetstrokeopacity{0.500000}%
\pgfsetdash{}{0pt}%
\pgfpathmoveto{\pgfqpoint{4.099608in}{5.121898in}}%
\pgfpathlineto{\pgfqpoint{4.140962in}{5.121898in}}%
\pgfusepath{stroke}%
\end{pgfscope}%
\begin{pgfscope}%
\pgfpathrectangle{\pgfqpoint{0.100000in}{2.802285in}}{\pgfqpoint{12.175956in}{4.213609in}}%
\pgfusepath{clip}%
\pgfsetbuttcap%
\pgfsetroundjoin%
\pgfsetlinewidth{2.007500pt}%
\definecolor{currentstroke}{rgb}{0.501961,0.501961,0.501961}%
\pgfsetstrokecolor{currentstroke}%
\pgfsetstrokeopacity{0.500000}%
\pgfsetdash{}{0pt}%
\pgfpathmoveto{\pgfqpoint{4.099608in}{5.547515in}}%
\pgfpathlineto{\pgfqpoint{4.140962in}{5.547515in}}%
\pgfusepath{stroke}%
\end{pgfscope}%
\begin{pgfscope}%
\pgfpathrectangle{\pgfqpoint{0.100000in}{2.802285in}}{\pgfqpoint{12.175956in}{4.213609in}}%
\pgfusepath{clip}%
\pgfsetbuttcap%
\pgfsetroundjoin%
\pgfsetlinewidth{2.007500pt}%
\definecolor{currentstroke}{rgb}{0.501961,0.501961,0.501961}%
\pgfsetstrokecolor{currentstroke}%
\pgfsetstrokeopacity{0.500000}%
\pgfsetdash{}{0pt}%
\pgfpathmoveto{\pgfqpoint{4.099608in}{5.973132in}}%
\pgfpathlineto{\pgfqpoint{4.140962in}{5.973132in}}%
\pgfusepath{stroke}%
\end{pgfscope}%
\begin{pgfscope}%
\pgfpathrectangle{\pgfqpoint{0.100000in}{2.802285in}}{\pgfqpoint{12.175956in}{4.213609in}}%
\pgfusepath{clip}%
\pgfsetbuttcap%
\pgfsetroundjoin%
\pgfsetlinewidth{2.007500pt}%
\definecolor{currentstroke}{rgb}{0.501961,0.501961,0.501961}%
\pgfsetstrokecolor{currentstroke}%
\pgfsetstrokeopacity{0.500000}%
\pgfsetdash{}{0pt}%
\pgfpathmoveto{\pgfqpoint{4.099608in}{6.398749in}}%
\pgfpathlineto{\pgfqpoint{4.140962in}{6.398749in}}%
\pgfusepath{stroke}%
\end{pgfscope}%
\begin{pgfscope}%
\pgfpathrectangle{\pgfqpoint{0.100000in}{2.802285in}}{\pgfqpoint{12.175956in}{4.213609in}}%
\pgfusepath{clip}%
\pgfsetbuttcap%
\pgfsetroundjoin%
\pgfsetlinewidth{2.007500pt}%
\definecolor{currentstroke}{rgb}{0.501961,0.501961,0.501961}%
\pgfsetstrokecolor{currentstroke}%
\pgfsetstrokeopacity{0.500000}%
\pgfsetdash{}{0pt}%
\pgfpathmoveto{\pgfqpoint{4.099608in}{6.824366in}}%
\pgfpathlineto{\pgfqpoint{4.140962in}{6.824366in}}%
\pgfusepath{stroke}%
\end{pgfscope}%
\begin{pgfscope}%
\pgfpathrectangle{\pgfqpoint{0.100000in}{2.802285in}}{\pgfqpoint{12.175956in}{4.213609in}}%
\pgfusepath{clip}%
\pgfsetrectcap%
\pgfsetroundjoin%
\pgfsetlinewidth{2.007500pt}%
\definecolor{currentstroke}{rgb}{0.501961,0.501961,0.501961}%
\pgfsetstrokecolor{currentstroke}%
\pgfsetstrokeopacity{0.500000}%
\pgfsetdash{}{0pt}%
\pgfpathmoveto{\pgfqpoint{4.809516in}{2.802285in}}%
\pgfpathlineto{\pgfqpoint{4.809516in}{7.015894in}}%
\pgfusepath{stroke}%
\end{pgfscope}%
\begin{pgfscope}%
\pgfpathrectangle{\pgfqpoint{0.100000in}{2.802285in}}{\pgfqpoint{12.175956in}{4.213609in}}%
\pgfusepath{clip}%
\pgfsetbuttcap%
\pgfsetroundjoin%
\pgfsetlinewidth{2.007500pt}%
\definecolor{currentstroke}{rgb}{0.501961,0.501961,0.501961}%
\pgfsetstrokecolor{currentstroke}%
\pgfsetstrokeopacity{0.500000}%
\pgfsetdash{}{0pt}%
\pgfpathmoveto{\pgfqpoint{4.788839in}{2.993813in}}%
\pgfpathlineto{\pgfqpoint{4.830193in}{2.993813in}}%
\pgfusepath{stroke}%
\end{pgfscope}%
\begin{pgfscope}%
\pgfpathrectangle{\pgfqpoint{0.100000in}{2.802285in}}{\pgfqpoint{12.175956in}{4.213609in}}%
\pgfusepath{clip}%
\pgfsetbuttcap%
\pgfsetroundjoin%
\pgfsetlinewidth{2.007500pt}%
\definecolor{currentstroke}{rgb}{0.501961,0.501961,0.501961}%
\pgfsetstrokecolor{currentstroke}%
\pgfsetstrokeopacity{0.500000}%
\pgfsetdash{}{0pt}%
\pgfpathmoveto{\pgfqpoint{4.788839in}{3.419430in}}%
\pgfpathlineto{\pgfqpoint{4.830193in}{3.419430in}}%
\pgfusepath{stroke}%
\end{pgfscope}%
\begin{pgfscope}%
\pgfpathrectangle{\pgfqpoint{0.100000in}{2.802285in}}{\pgfqpoint{12.175956in}{4.213609in}}%
\pgfusepath{clip}%
\pgfsetbuttcap%
\pgfsetroundjoin%
\pgfsetlinewidth{2.007500pt}%
\definecolor{currentstroke}{rgb}{0.501961,0.501961,0.501961}%
\pgfsetstrokecolor{currentstroke}%
\pgfsetstrokeopacity{0.500000}%
\pgfsetdash{}{0pt}%
\pgfpathmoveto{\pgfqpoint{4.788839in}{3.845047in}}%
\pgfpathlineto{\pgfqpoint{4.830193in}{3.845047in}}%
\pgfusepath{stroke}%
\end{pgfscope}%
\begin{pgfscope}%
\pgfpathrectangle{\pgfqpoint{0.100000in}{2.802285in}}{\pgfqpoint{12.175956in}{4.213609in}}%
\pgfusepath{clip}%
\pgfsetbuttcap%
\pgfsetroundjoin%
\pgfsetlinewidth{2.007500pt}%
\definecolor{currentstroke}{rgb}{0.501961,0.501961,0.501961}%
\pgfsetstrokecolor{currentstroke}%
\pgfsetstrokeopacity{0.500000}%
\pgfsetdash{}{0pt}%
\pgfpathmoveto{\pgfqpoint{4.788839in}{4.270664in}}%
\pgfpathlineto{\pgfqpoint{4.830193in}{4.270664in}}%
\pgfusepath{stroke}%
\end{pgfscope}%
\begin{pgfscope}%
\pgfpathrectangle{\pgfqpoint{0.100000in}{2.802285in}}{\pgfqpoint{12.175956in}{4.213609in}}%
\pgfusepath{clip}%
\pgfsetbuttcap%
\pgfsetroundjoin%
\pgfsetlinewidth{2.007500pt}%
\definecolor{currentstroke}{rgb}{0.501961,0.501961,0.501961}%
\pgfsetstrokecolor{currentstroke}%
\pgfsetstrokeopacity{0.500000}%
\pgfsetdash{}{0pt}%
\pgfpathmoveto{\pgfqpoint{4.788839in}{4.696281in}}%
\pgfpathlineto{\pgfqpoint{4.830193in}{4.696281in}}%
\pgfusepath{stroke}%
\end{pgfscope}%
\begin{pgfscope}%
\pgfpathrectangle{\pgfqpoint{0.100000in}{2.802285in}}{\pgfqpoint{12.175956in}{4.213609in}}%
\pgfusepath{clip}%
\pgfsetbuttcap%
\pgfsetroundjoin%
\pgfsetlinewidth{2.007500pt}%
\definecolor{currentstroke}{rgb}{0.501961,0.501961,0.501961}%
\pgfsetstrokecolor{currentstroke}%
\pgfsetstrokeopacity{0.500000}%
\pgfsetdash{}{0pt}%
\pgfpathmoveto{\pgfqpoint{4.788839in}{5.121898in}}%
\pgfpathlineto{\pgfqpoint{4.830193in}{5.121898in}}%
\pgfusepath{stroke}%
\end{pgfscope}%
\begin{pgfscope}%
\pgfpathrectangle{\pgfqpoint{0.100000in}{2.802285in}}{\pgfqpoint{12.175956in}{4.213609in}}%
\pgfusepath{clip}%
\pgfsetbuttcap%
\pgfsetroundjoin%
\pgfsetlinewidth{2.007500pt}%
\definecolor{currentstroke}{rgb}{0.501961,0.501961,0.501961}%
\pgfsetstrokecolor{currentstroke}%
\pgfsetstrokeopacity{0.500000}%
\pgfsetdash{}{0pt}%
\pgfpathmoveto{\pgfqpoint{4.788839in}{5.547515in}}%
\pgfpathlineto{\pgfqpoint{4.830193in}{5.547515in}}%
\pgfusepath{stroke}%
\end{pgfscope}%
\begin{pgfscope}%
\pgfpathrectangle{\pgfqpoint{0.100000in}{2.802285in}}{\pgfqpoint{12.175956in}{4.213609in}}%
\pgfusepath{clip}%
\pgfsetbuttcap%
\pgfsetroundjoin%
\pgfsetlinewidth{2.007500pt}%
\definecolor{currentstroke}{rgb}{0.501961,0.501961,0.501961}%
\pgfsetstrokecolor{currentstroke}%
\pgfsetstrokeopacity{0.500000}%
\pgfsetdash{}{0pt}%
\pgfpathmoveto{\pgfqpoint{4.788839in}{5.973132in}}%
\pgfpathlineto{\pgfqpoint{4.830193in}{5.973132in}}%
\pgfusepath{stroke}%
\end{pgfscope}%
\begin{pgfscope}%
\pgfpathrectangle{\pgfqpoint{0.100000in}{2.802285in}}{\pgfqpoint{12.175956in}{4.213609in}}%
\pgfusepath{clip}%
\pgfsetbuttcap%
\pgfsetroundjoin%
\pgfsetlinewidth{2.007500pt}%
\definecolor{currentstroke}{rgb}{0.501961,0.501961,0.501961}%
\pgfsetstrokecolor{currentstroke}%
\pgfsetstrokeopacity{0.500000}%
\pgfsetdash{}{0pt}%
\pgfpathmoveto{\pgfqpoint{4.788839in}{6.398749in}}%
\pgfpathlineto{\pgfqpoint{4.830193in}{6.398749in}}%
\pgfusepath{stroke}%
\end{pgfscope}%
\begin{pgfscope}%
\pgfpathrectangle{\pgfqpoint{0.100000in}{2.802285in}}{\pgfqpoint{12.175956in}{4.213609in}}%
\pgfusepath{clip}%
\pgfsetbuttcap%
\pgfsetroundjoin%
\pgfsetlinewidth{2.007500pt}%
\definecolor{currentstroke}{rgb}{0.501961,0.501961,0.501961}%
\pgfsetstrokecolor{currentstroke}%
\pgfsetstrokeopacity{0.500000}%
\pgfsetdash{}{0pt}%
\pgfpathmoveto{\pgfqpoint{4.788839in}{6.824366in}}%
\pgfpathlineto{\pgfqpoint{4.830193in}{6.824366in}}%
\pgfusepath{stroke}%
\end{pgfscope}%
\begin{pgfscope}%
\pgfpathrectangle{\pgfqpoint{0.100000in}{2.802285in}}{\pgfqpoint{12.175956in}{4.213609in}}%
\pgfusepath{clip}%
\pgfsetrectcap%
\pgfsetroundjoin%
\pgfsetlinewidth{2.007500pt}%
\definecolor{currentstroke}{rgb}{0.501961,0.501961,0.501961}%
\pgfsetstrokecolor{currentstroke}%
\pgfsetstrokeopacity{0.500000}%
\pgfsetdash{}{0pt}%
\pgfpathmoveto{\pgfqpoint{5.498747in}{2.802285in}}%
\pgfpathlineto{\pgfqpoint{5.498747in}{7.015894in}}%
\pgfusepath{stroke}%
\end{pgfscope}%
\begin{pgfscope}%
\pgfpathrectangle{\pgfqpoint{0.100000in}{2.802285in}}{\pgfqpoint{12.175956in}{4.213609in}}%
\pgfusepath{clip}%
\pgfsetbuttcap%
\pgfsetroundjoin%
\pgfsetlinewidth{2.007500pt}%
\definecolor{currentstroke}{rgb}{0.501961,0.501961,0.501961}%
\pgfsetstrokecolor{currentstroke}%
\pgfsetstrokeopacity{0.500000}%
\pgfsetdash{}{0pt}%
\pgfpathmoveto{\pgfqpoint{5.478070in}{2.993813in}}%
\pgfpathlineto{\pgfqpoint{5.519424in}{2.993813in}}%
\pgfusepath{stroke}%
\end{pgfscope}%
\begin{pgfscope}%
\pgfpathrectangle{\pgfqpoint{0.100000in}{2.802285in}}{\pgfqpoint{12.175956in}{4.213609in}}%
\pgfusepath{clip}%
\pgfsetbuttcap%
\pgfsetroundjoin%
\pgfsetlinewidth{2.007500pt}%
\definecolor{currentstroke}{rgb}{0.501961,0.501961,0.501961}%
\pgfsetstrokecolor{currentstroke}%
\pgfsetstrokeopacity{0.500000}%
\pgfsetdash{}{0pt}%
\pgfpathmoveto{\pgfqpoint{5.478070in}{3.419430in}}%
\pgfpathlineto{\pgfqpoint{5.519424in}{3.419430in}}%
\pgfusepath{stroke}%
\end{pgfscope}%
\begin{pgfscope}%
\pgfpathrectangle{\pgfqpoint{0.100000in}{2.802285in}}{\pgfqpoint{12.175956in}{4.213609in}}%
\pgfusepath{clip}%
\pgfsetbuttcap%
\pgfsetroundjoin%
\pgfsetlinewidth{2.007500pt}%
\definecolor{currentstroke}{rgb}{0.501961,0.501961,0.501961}%
\pgfsetstrokecolor{currentstroke}%
\pgfsetstrokeopacity{0.500000}%
\pgfsetdash{}{0pt}%
\pgfpathmoveto{\pgfqpoint{5.478070in}{3.845047in}}%
\pgfpathlineto{\pgfqpoint{5.519424in}{3.845047in}}%
\pgfusepath{stroke}%
\end{pgfscope}%
\begin{pgfscope}%
\pgfpathrectangle{\pgfqpoint{0.100000in}{2.802285in}}{\pgfqpoint{12.175956in}{4.213609in}}%
\pgfusepath{clip}%
\pgfsetbuttcap%
\pgfsetroundjoin%
\pgfsetlinewidth{2.007500pt}%
\definecolor{currentstroke}{rgb}{0.501961,0.501961,0.501961}%
\pgfsetstrokecolor{currentstroke}%
\pgfsetstrokeopacity{0.500000}%
\pgfsetdash{}{0pt}%
\pgfpathmoveto{\pgfqpoint{5.478070in}{4.270664in}}%
\pgfpathlineto{\pgfqpoint{5.519424in}{4.270664in}}%
\pgfusepath{stroke}%
\end{pgfscope}%
\begin{pgfscope}%
\pgfpathrectangle{\pgfqpoint{0.100000in}{2.802285in}}{\pgfqpoint{12.175956in}{4.213609in}}%
\pgfusepath{clip}%
\pgfsetbuttcap%
\pgfsetroundjoin%
\pgfsetlinewidth{2.007500pt}%
\definecolor{currentstroke}{rgb}{0.501961,0.501961,0.501961}%
\pgfsetstrokecolor{currentstroke}%
\pgfsetstrokeopacity{0.500000}%
\pgfsetdash{}{0pt}%
\pgfpathmoveto{\pgfqpoint{5.478070in}{4.696281in}}%
\pgfpathlineto{\pgfqpoint{5.519424in}{4.696281in}}%
\pgfusepath{stroke}%
\end{pgfscope}%
\begin{pgfscope}%
\pgfpathrectangle{\pgfqpoint{0.100000in}{2.802285in}}{\pgfqpoint{12.175956in}{4.213609in}}%
\pgfusepath{clip}%
\pgfsetbuttcap%
\pgfsetroundjoin%
\pgfsetlinewidth{2.007500pt}%
\definecolor{currentstroke}{rgb}{0.501961,0.501961,0.501961}%
\pgfsetstrokecolor{currentstroke}%
\pgfsetstrokeopacity{0.500000}%
\pgfsetdash{}{0pt}%
\pgfpathmoveto{\pgfqpoint{5.478070in}{5.121898in}}%
\pgfpathlineto{\pgfqpoint{5.519424in}{5.121898in}}%
\pgfusepath{stroke}%
\end{pgfscope}%
\begin{pgfscope}%
\pgfpathrectangle{\pgfqpoint{0.100000in}{2.802285in}}{\pgfqpoint{12.175956in}{4.213609in}}%
\pgfusepath{clip}%
\pgfsetbuttcap%
\pgfsetroundjoin%
\pgfsetlinewidth{2.007500pt}%
\definecolor{currentstroke}{rgb}{0.501961,0.501961,0.501961}%
\pgfsetstrokecolor{currentstroke}%
\pgfsetstrokeopacity{0.500000}%
\pgfsetdash{}{0pt}%
\pgfpathmoveto{\pgfqpoint{5.478070in}{5.547515in}}%
\pgfpathlineto{\pgfqpoint{5.519424in}{5.547515in}}%
\pgfusepath{stroke}%
\end{pgfscope}%
\begin{pgfscope}%
\pgfpathrectangle{\pgfqpoint{0.100000in}{2.802285in}}{\pgfqpoint{12.175956in}{4.213609in}}%
\pgfusepath{clip}%
\pgfsetbuttcap%
\pgfsetroundjoin%
\pgfsetlinewidth{2.007500pt}%
\definecolor{currentstroke}{rgb}{0.501961,0.501961,0.501961}%
\pgfsetstrokecolor{currentstroke}%
\pgfsetstrokeopacity{0.500000}%
\pgfsetdash{}{0pt}%
\pgfpathmoveto{\pgfqpoint{5.478070in}{5.973132in}}%
\pgfpathlineto{\pgfqpoint{5.519424in}{5.973132in}}%
\pgfusepath{stroke}%
\end{pgfscope}%
\begin{pgfscope}%
\pgfpathrectangle{\pgfqpoint{0.100000in}{2.802285in}}{\pgfqpoint{12.175956in}{4.213609in}}%
\pgfusepath{clip}%
\pgfsetbuttcap%
\pgfsetroundjoin%
\pgfsetlinewidth{2.007500pt}%
\definecolor{currentstroke}{rgb}{0.501961,0.501961,0.501961}%
\pgfsetstrokecolor{currentstroke}%
\pgfsetstrokeopacity{0.500000}%
\pgfsetdash{}{0pt}%
\pgfpathmoveto{\pgfqpoint{5.478070in}{6.398749in}}%
\pgfpathlineto{\pgfqpoint{5.519424in}{6.398749in}}%
\pgfusepath{stroke}%
\end{pgfscope}%
\begin{pgfscope}%
\pgfpathrectangle{\pgfqpoint{0.100000in}{2.802285in}}{\pgfqpoint{12.175956in}{4.213609in}}%
\pgfusepath{clip}%
\pgfsetbuttcap%
\pgfsetroundjoin%
\pgfsetlinewidth{2.007500pt}%
\definecolor{currentstroke}{rgb}{0.501961,0.501961,0.501961}%
\pgfsetstrokecolor{currentstroke}%
\pgfsetstrokeopacity{0.500000}%
\pgfsetdash{}{0pt}%
\pgfpathmoveto{\pgfqpoint{5.478070in}{6.824366in}}%
\pgfpathlineto{\pgfqpoint{5.519424in}{6.824366in}}%
\pgfusepath{stroke}%
\end{pgfscope}%
\begin{pgfscope}%
\pgfpathrectangle{\pgfqpoint{0.100000in}{2.802285in}}{\pgfqpoint{12.175956in}{4.213609in}}%
\pgfusepath{clip}%
\pgfsetrectcap%
\pgfsetroundjoin%
\pgfsetlinewidth{2.007500pt}%
\definecolor{currentstroke}{rgb}{0.501961,0.501961,0.501961}%
\pgfsetstrokecolor{currentstroke}%
\pgfsetstrokeopacity{0.500000}%
\pgfsetdash{}{0pt}%
\pgfpathmoveto{\pgfqpoint{6.187978in}{2.802285in}}%
\pgfpathlineto{\pgfqpoint{6.187978in}{7.015894in}}%
\pgfusepath{stroke}%
\end{pgfscope}%
\begin{pgfscope}%
\pgfpathrectangle{\pgfqpoint{0.100000in}{2.802285in}}{\pgfqpoint{12.175956in}{4.213609in}}%
\pgfusepath{clip}%
\pgfsetbuttcap%
\pgfsetroundjoin%
\pgfsetlinewidth{2.007500pt}%
\definecolor{currentstroke}{rgb}{0.501961,0.501961,0.501961}%
\pgfsetstrokecolor{currentstroke}%
\pgfsetstrokeopacity{0.500000}%
\pgfsetdash{}{0pt}%
\pgfpathmoveto{\pgfqpoint{6.167301in}{2.993813in}}%
\pgfpathlineto{\pgfqpoint{6.208655in}{2.993813in}}%
\pgfusepath{stroke}%
\end{pgfscope}%
\begin{pgfscope}%
\pgfpathrectangle{\pgfqpoint{0.100000in}{2.802285in}}{\pgfqpoint{12.175956in}{4.213609in}}%
\pgfusepath{clip}%
\pgfsetbuttcap%
\pgfsetroundjoin%
\pgfsetlinewidth{2.007500pt}%
\definecolor{currentstroke}{rgb}{0.501961,0.501961,0.501961}%
\pgfsetstrokecolor{currentstroke}%
\pgfsetstrokeopacity{0.500000}%
\pgfsetdash{}{0pt}%
\pgfpathmoveto{\pgfqpoint{6.167301in}{3.419430in}}%
\pgfpathlineto{\pgfqpoint{6.208655in}{3.419430in}}%
\pgfusepath{stroke}%
\end{pgfscope}%
\begin{pgfscope}%
\pgfpathrectangle{\pgfqpoint{0.100000in}{2.802285in}}{\pgfqpoint{12.175956in}{4.213609in}}%
\pgfusepath{clip}%
\pgfsetbuttcap%
\pgfsetroundjoin%
\pgfsetlinewidth{2.007500pt}%
\definecolor{currentstroke}{rgb}{0.501961,0.501961,0.501961}%
\pgfsetstrokecolor{currentstroke}%
\pgfsetstrokeopacity{0.500000}%
\pgfsetdash{}{0pt}%
\pgfpathmoveto{\pgfqpoint{6.167301in}{3.845047in}}%
\pgfpathlineto{\pgfqpoint{6.208655in}{3.845047in}}%
\pgfusepath{stroke}%
\end{pgfscope}%
\begin{pgfscope}%
\pgfpathrectangle{\pgfqpoint{0.100000in}{2.802285in}}{\pgfqpoint{12.175956in}{4.213609in}}%
\pgfusepath{clip}%
\pgfsetbuttcap%
\pgfsetroundjoin%
\pgfsetlinewidth{2.007500pt}%
\definecolor{currentstroke}{rgb}{0.501961,0.501961,0.501961}%
\pgfsetstrokecolor{currentstroke}%
\pgfsetstrokeopacity{0.500000}%
\pgfsetdash{}{0pt}%
\pgfpathmoveto{\pgfqpoint{6.167301in}{4.270664in}}%
\pgfpathlineto{\pgfqpoint{6.208655in}{4.270664in}}%
\pgfusepath{stroke}%
\end{pgfscope}%
\begin{pgfscope}%
\pgfpathrectangle{\pgfqpoint{0.100000in}{2.802285in}}{\pgfqpoint{12.175956in}{4.213609in}}%
\pgfusepath{clip}%
\pgfsetbuttcap%
\pgfsetroundjoin%
\pgfsetlinewidth{2.007500pt}%
\definecolor{currentstroke}{rgb}{0.501961,0.501961,0.501961}%
\pgfsetstrokecolor{currentstroke}%
\pgfsetstrokeopacity{0.500000}%
\pgfsetdash{}{0pt}%
\pgfpathmoveto{\pgfqpoint{6.167301in}{4.696281in}}%
\pgfpathlineto{\pgfqpoint{6.208655in}{4.696281in}}%
\pgfusepath{stroke}%
\end{pgfscope}%
\begin{pgfscope}%
\pgfpathrectangle{\pgfqpoint{0.100000in}{2.802285in}}{\pgfqpoint{12.175956in}{4.213609in}}%
\pgfusepath{clip}%
\pgfsetbuttcap%
\pgfsetroundjoin%
\pgfsetlinewidth{2.007500pt}%
\definecolor{currentstroke}{rgb}{0.501961,0.501961,0.501961}%
\pgfsetstrokecolor{currentstroke}%
\pgfsetstrokeopacity{0.500000}%
\pgfsetdash{}{0pt}%
\pgfpathmoveto{\pgfqpoint{6.167301in}{5.121898in}}%
\pgfpathlineto{\pgfqpoint{6.208655in}{5.121898in}}%
\pgfusepath{stroke}%
\end{pgfscope}%
\begin{pgfscope}%
\pgfpathrectangle{\pgfqpoint{0.100000in}{2.802285in}}{\pgfqpoint{12.175956in}{4.213609in}}%
\pgfusepath{clip}%
\pgfsetbuttcap%
\pgfsetroundjoin%
\pgfsetlinewidth{2.007500pt}%
\definecolor{currentstroke}{rgb}{0.501961,0.501961,0.501961}%
\pgfsetstrokecolor{currentstroke}%
\pgfsetstrokeopacity{0.500000}%
\pgfsetdash{}{0pt}%
\pgfpathmoveto{\pgfqpoint{6.167301in}{5.547515in}}%
\pgfpathlineto{\pgfqpoint{6.208655in}{5.547515in}}%
\pgfusepath{stroke}%
\end{pgfscope}%
\begin{pgfscope}%
\pgfpathrectangle{\pgfqpoint{0.100000in}{2.802285in}}{\pgfqpoint{12.175956in}{4.213609in}}%
\pgfusepath{clip}%
\pgfsetbuttcap%
\pgfsetroundjoin%
\pgfsetlinewidth{2.007500pt}%
\definecolor{currentstroke}{rgb}{0.501961,0.501961,0.501961}%
\pgfsetstrokecolor{currentstroke}%
\pgfsetstrokeopacity{0.500000}%
\pgfsetdash{}{0pt}%
\pgfpathmoveto{\pgfqpoint{6.167301in}{5.973132in}}%
\pgfpathlineto{\pgfqpoint{6.208655in}{5.973132in}}%
\pgfusepath{stroke}%
\end{pgfscope}%
\begin{pgfscope}%
\pgfpathrectangle{\pgfqpoint{0.100000in}{2.802285in}}{\pgfqpoint{12.175956in}{4.213609in}}%
\pgfusepath{clip}%
\pgfsetbuttcap%
\pgfsetroundjoin%
\pgfsetlinewidth{2.007500pt}%
\definecolor{currentstroke}{rgb}{0.501961,0.501961,0.501961}%
\pgfsetstrokecolor{currentstroke}%
\pgfsetstrokeopacity{0.500000}%
\pgfsetdash{}{0pt}%
\pgfpathmoveto{\pgfqpoint{6.167301in}{6.398749in}}%
\pgfpathlineto{\pgfqpoint{6.208655in}{6.398749in}}%
\pgfusepath{stroke}%
\end{pgfscope}%
\begin{pgfscope}%
\pgfpathrectangle{\pgfqpoint{0.100000in}{2.802285in}}{\pgfqpoint{12.175956in}{4.213609in}}%
\pgfusepath{clip}%
\pgfsetbuttcap%
\pgfsetroundjoin%
\pgfsetlinewidth{2.007500pt}%
\definecolor{currentstroke}{rgb}{0.501961,0.501961,0.501961}%
\pgfsetstrokecolor{currentstroke}%
\pgfsetstrokeopacity{0.500000}%
\pgfsetdash{}{0pt}%
\pgfpathmoveto{\pgfqpoint{6.167301in}{6.824366in}}%
\pgfpathlineto{\pgfqpoint{6.208655in}{6.824366in}}%
\pgfusepath{stroke}%
\end{pgfscope}%
\begin{pgfscope}%
\pgfpathrectangle{\pgfqpoint{0.100000in}{2.802285in}}{\pgfqpoint{12.175956in}{4.213609in}}%
\pgfusepath{clip}%
\pgfsetrectcap%
\pgfsetroundjoin%
\pgfsetlinewidth{2.007500pt}%
\definecolor{currentstroke}{rgb}{0.501961,0.501961,0.501961}%
\pgfsetstrokecolor{currentstroke}%
\pgfsetstrokeopacity{0.500000}%
\pgfsetdash{}{0pt}%
\pgfpathmoveto{\pgfqpoint{6.877209in}{2.802285in}}%
\pgfpathlineto{\pgfqpoint{6.877209in}{7.015894in}}%
\pgfusepath{stroke}%
\end{pgfscope}%
\begin{pgfscope}%
\pgfpathrectangle{\pgfqpoint{0.100000in}{2.802285in}}{\pgfqpoint{12.175956in}{4.213609in}}%
\pgfusepath{clip}%
\pgfsetbuttcap%
\pgfsetroundjoin%
\pgfsetlinewidth{2.007500pt}%
\definecolor{currentstroke}{rgb}{0.501961,0.501961,0.501961}%
\pgfsetstrokecolor{currentstroke}%
\pgfsetstrokeopacity{0.500000}%
\pgfsetdash{}{0pt}%
\pgfpathmoveto{\pgfqpoint{6.856532in}{2.993813in}}%
\pgfpathlineto{\pgfqpoint{6.897886in}{2.993813in}}%
\pgfusepath{stroke}%
\end{pgfscope}%
\begin{pgfscope}%
\pgfpathrectangle{\pgfqpoint{0.100000in}{2.802285in}}{\pgfqpoint{12.175956in}{4.213609in}}%
\pgfusepath{clip}%
\pgfsetbuttcap%
\pgfsetroundjoin%
\pgfsetlinewidth{2.007500pt}%
\definecolor{currentstroke}{rgb}{0.501961,0.501961,0.501961}%
\pgfsetstrokecolor{currentstroke}%
\pgfsetstrokeopacity{0.500000}%
\pgfsetdash{}{0pt}%
\pgfpathmoveto{\pgfqpoint{6.856532in}{3.419430in}}%
\pgfpathlineto{\pgfqpoint{6.897886in}{3.419430in}}%
\pgfusepath{stroke}%
\end{pgfscope}%
\begin{pgfscope}%
\pgfpathrectangle{\pgfqpoint{0.100000in}{2.802285in}}{\pgfqpoint{12.175956in}{4.213609in}}%
\pgfusepath{clip}%
\pgfsetbuttcap%
\pgfsetroundjoin%
\pgfsetlinewidth{2.007500pt}%
\definecolor{currentstroke}{rgb}{0.501961,0.501961,0.501961}%
\pgfsetstrokecolor{currentstroke}%
\pgfsetstrokeopacity{0.500000}%
\pgfsetdash{}{0pt}%
\pgfpathmoveto{\pgfqpoint{6.856532in}{3.845047in}}%
\pgfpathlineto{\pgfqpoint{6.897886in}{3.845047in}}%
\pgfusepath{stroke}%
\end{pgfscope}%
\begin{pgfscope}%
\pgfpathrectangle{\pgfqpoint{0.100000in}{2.802285in}}{\pgfqpoint{12.175956in}{4.213609in}}%
\pgfusepath{clip}%
\pgfsetbuttcap%
\pgfsetroundjoin%
\pgfsetlinewidth{2.007500pt}%
\definecolor{currentstroke}{rgb}{0.501961,0.501961,0.501961}%
\pgfsetstrokecolor{currentstroke}%
\pgfsetstrokeopacity{0.500000}%
\pgfsetdash{}{0pt}%
\pgfpathmoveto{\pgfqpoint{6.856532in}{4.270664in}}%
\pgfpathlineto{\pgfqpoint{6.897886in}{4.270664in}}%
\pgfusepath{stroke}%
\end{pgfscope}%
\begin{pgfscope}%
\pgfpathrectangle{\pgfqpoint{0.100000in}{2.802285in}}{\pgfqpoint{12.175956in}{4.213609in}}%
\pgfusepath{clip}%
\pgfsetbuttcap%
\pgfsetroundjoin%
\pgfsetlinewidth{2.007500pt}%
\definecolor{currentstroke}{rgb}{0.501961,0.501961,0.501961}%
\pgfsetstrokecolor{currentstroke}%
\pgfsetstrokeopacity{0.500000}%
\pgfsetdash{}{0pt}%
\pgfpathmoveto{\pgfqpoint{6.856532in}{4.696281in}}%
\pgfpathlineto{\pgfqpoint{6.897886in}{4.696281in}}%
\pgfusepath{stroke}%
\end{pgfscope}%
\begin{pgfscope}%
\pgfpathrectangle{\pgfqpoint{0.100000in}{2.802285in}}{\pgfqpoint{12.175956in}{4.213609in}}%
\pgfusepath{clip}%
\pgfsetbuttcap%
\pgfsetroundjoin%
\pgfsetlinewidth{2.007500pt}%
\definecolor{currentstroke}{rgb}{0.501961,0.501961,0.501961}%
\pgfsetstrokecolor{currentstroke}%
\pgfsetstrokeopacity{0.500000}%
\pgfsetdash{}{0pt}%
\pgfpathmoveto{\pgfqpoint{6.856532in}{5.121898in}}%
\pgfpathlineto{\pgfqpoint{6.897886in}{5.121898in}}%
\pgfusepath{stroke}%
\end{pgfscope}%
\begin{pgfscope}%
\pgfpathrectangle{\pgfqpoint{0.100000in}{2.802285in}}{\pgfqpoint{12.175956in}{4.213609in}}%
\pgfusepath{clip}%
\pgfsetbuttcap%
\pgfsetroundjoin%
\pgfsetlinewidth{2.007500pt}%
\definecolor{currentstroke}{rgb}{0.501961,0.501961,0.501961}%
\pgfsetstrokecolor{currentstroke}%
\pgfsetstrokeopacity{0.500000}%
\pgfsetdash{}{0pt}%
\pgfpathmoveto{\pgfqpoint{6.856532in}{5.547515in}}%
\pgfpathlineto{\pgfqpoint{6.897886in}{5.547515in}}%
\pgfusepath{stroke}%
\end{pgfscope}%
\begin{pgfscope}%
\pgfpathrectangle{\pgfqpoint{0.100000in}{2.802285in}}{\pgfqpoint{12.175956in}{4.213609in}}%
\pgfusepath{clip}%
\pgfsetbuttcap%
\pgfsetroundjoin%
\pgfsetlinewidth{2.007500pt}%
\definecolor{currentstroke}{rgb}{0.501961,0.501961,0.501961}%
\pgfsetstrokecolor{currentstroke}%
\pgfsetstrokeopacity{0.500000}%
\pgfsetdash{}{0pt}%
\pgfpathmoveto{\pgfqpoint{6.856532in}{5.973132in}}%
\pgfpathlineto{\pgfqpoint{6.897886in}{5.973132in}}%
\pgfusepath{stroke}%
\end{pgfscope}%
\begin{pgfscope}%
\pgfpathrectangle{\pgfqpoint{0.100000in}{2.802285in}}{\pgfqpoint{12.175956in}{4.213609in}}%
\pgfusepath{clip}%
\pgfsetbuttcap%
\pgfsetroundjoin%
\pgfsetlinewidth{2.007500pt}%
\definecolor{currentstroke}{rgb}{0.501961,0.501961,0.501961}%
\pgfsetstrokecolor{currentstroke}%
\pgfsetstrokeopacity{0.500000}%
\pgfsetdash{}{0pt}%
\pgfpathmoveto{\pgfqpoint{6.856532in}{6.398749in}}%
\pgfpathlineto{\pgfqpoint{6.897886in}{6.398749in}}%
\pgfusepath{stroke}%
\end{pgfscope}%
\begin{pgfscope}%
\pgfpathrectangle{\pgfqpoint{0.100000in}{2.802285in}}{\pgfqpoint{12.175956in}{4.213609in}}%
\pgfusepath{clip}%
\pgfsetbuttcap%
\pgfsetroundjoin%
\pgfsetlinewidth{2.007500pt}%
\definecolor{currentstroke}{rgb}{0.501961,0.501961,0.501961}%
\pgfsetstrokecolor{currentstroke}%
\pgfsetstrokeopacity{0.500000}%
\pgfsetdash{}{0pt}%
\pgfpathmoveto{\pgfqpoint{6.856532in}{6.824366in}}%
\pgfpathlineto{\pgfqpoint{6.897886in}{6.824366in}}%
\pgfusepath{stroke}%
\end{pgfscope}%
\begin{pgfscope}%
\pgfpathrectangle{\pgfqpoint{0.100000in}{2.802285in}}{\pgfqpoint{12.175956in}{4.213609in}}%
\pgfusepath{clip}%
\pgfsetrectcap%
\pgfsetroundjoin%
\pgfsetlinewidth{2.007500pt}%
\definecolor{currentstroke}{rgb}{0.501961,0.501961,0.501961}%
\pgfsetstrokecolor{currentstroke}%
\pgfsetstrokeopacity{0.500000}%
\pgfsetdash{}{0pt}%
\pgfpathmoveto{\pgfqpoint{7.566440in}{2.802285in}}%
\pgfpathlineto{\pgfqpoint{7.566440in}{7.015894in}}%
\pgfusepath{stroke}%
\end{pgfscope}%
\begin{pgfscope}%
\pgfpathrectangle{\pgfqpoint{0.100000in}{2.802285in}}{\pgfqpoint{12.175956in}{4.213609in}}%
\pgfusepath{clip}%
\pgfsetbuttcap%
\pgfsetroundjoin%
\pgfsetlinewidth{2.007500pt}%
\definecolor{currentstroke}{rgb}{0.501961,0.501961,0.501961}%
\pgfsetstrokecolor{currentstroke}%
\pgfsetstrokeopacity{0.500000}%
\pgfsetdash{}{0pt}%
\pgfpathmoveto{\pgfqpoint{7.545763in}{2.993813in}}%
\pgfpathlineto{\pgfqpoint{7.587117in}{2.993813in}}%
\pgfusepath{stroke}%
\end{pgfscope}%
\begin{pgfscope}%
\pgfpathrectangle{\pgfqpoint{0.100000in}{2.802285in}}{\pgfqpoint{12.175956in}{4.213609in}}%
\pgfusepath{clip}%
\pgfsetbuttcap%
\pgfsetroundjoin%
\pgfsetlinewidth{2.007500pt}%
\definecolor{currentstroke}{rgb}{0.501961,0.501961,0.501961}%
\pgfsetstrokecolor{currentstroke}%
\pgfsetstrokeopacity{0.500000}%
\pgfsetdash{}{0pt}%
\pgfpathmoveto{\pgfqpoint{7.545763in}{3.419430in}}%
\pgfpathlineto{\pgfqpoint{7.587117in}{3.419430in}}%
\pgfusepath{stroke}%
\end{pgfscope}%
\begin{pgfscope}%
\pgfpathrectangle{\pgfqpoint{0.100000in}{2.802285in}}{\pgfqpoint{12.175956in}{4.213609in}}%
\pgfusepath{clip}%
\pgfsetbuttcap%
\pgfsetroundjoin%
\pgfsetlinewidth{2.007500pt}%
\definecolor{currentstroke}{rgb}{0.501961,0.501961,0.501961}%
\pgfsetstrokecolor{currentstroke}%
\pgfsetstrokeopacity{0.500000}%
\pgfsetdash{}{0pt}%
\pgfpathmoveto{\pgfqpoint{7.545763in}{3.845047in}}%
\pgfpathlineto{\pgfqpoint{7.587117in}{3.845047in}}%
\pgfusepath{stroke}%
\end{pgfscope}%
\begin{pgfscope}%
\pgfpathrectangle{\pgfqpoint{0.100000in}{2.802285in}}{\pgfqpoint{12.175956in}{4.213609in}}%
\pgfusepath{clip}%
\pgfsetbuttcap%
\pgfsetroundjoin%
\pgfsetlinewidth{2.007500pt}%
\definecolor{currentstroke}{rgb}{0.501961,0.501961,0.501961}%
\pgfsetstrokecolor{currentstroke}%
\pgfsetstrokeopacity{0.500000}%
\pgfsetdash{}{0pt}%
\pgfpathmoveto{\pgfqpoint{7.545763in}{4.270664in}}%
\pgfpathlineto{\pgfqpoint{7.587117in}{4.270664in}}%
\pgfusepath{stroke}%
\end{pgfscope}%
\begin{pgfscope}%
\pgfpathrectangle{\pgfqpoint{0.100000in}{2.802285in}}{\pgfqpoint{12.175956in}{4.213609in}}%
\pgfusepath{clip}%
\pgfsetbuttcap%
\pgfsetroundjoin%
\pgfsetlinewidth{2.007500pt}%
\definecolor{currentstroke}{rgb}{0.501961,0.501961,0.501961}%
\pgfsetstrokecolor{currentstroke}%
\pgfsetstrokeopacity{0.500000}%
\pgfsetdash{}{0pt}%
\pgfpathmoveto{\pgfqpoint{7.545763in}{4.696281in}}%
\pgfpathlineto{\pgfqpoint{7.587117in}{4.696281in}}%
\pgfusepath{stroke}%
\end{pgfscope}%
\begin{pgfscope}%
\pgfpathrectangle{\pgfqpoint{0.100000in}{2.802285in}}{\pgfqpoint{12.175956in}{4.213609in}}%
\pgfusepath{clip}%
\pgfsetbuttcap%
\pgfsetroundjoin%
\pgfsetlinewidth{2.007500pt}%
\definecolor{currentstroke}{rgb}{0.501961,0.501961,0.501961}%
\pgfsetstrokecolor{currentstroke}%
\pgfsetstrokeopacity{0.500000}%
\pgfsetdash{}{0pt}%
\pgfpathmoveto{\pgfqpoint{7.545763in}{5.121898in}}%
\pgfpathlineto{\pgfqpoint{7.587117in}{5.121898in}}%
\pgfusepath{stroke}%
\end{pgfscope}%
\begin{pgfscope}%
\pgfpathrectangle{\pgfqpoint{0.100000in}{2.802285in}}{\pgfqpoint{12.175956in}{4.213609in}}%
\pgfusepath{clip}%
\pgfsetbuttcap%
\pgfsetroundjoin%
\pgfsetlinewidth{2.007500pt}%
\definecolor{currentstroke}{rgb}{0.501961,0.501961,0.501961}%
\pgfsetstrokecolor{currentstroke}%
\pgfsetstrokeopacity{0.500000}%
\pgfsetdash{}{0pt}%
\pgfpathmoveto{\pgfqpoint{7.545763in}{5.547515in}}%
\pgfpathlineto{\pgfqpoint{7.587117in}{5.547515in}}%
\pgfusepath{stroke}%
\end{pgfscope}%
\begin{pgfscope}%
\pgfpathrectangle{\pgfqpoint{0.100000in}{2.802285in}}{\pgfqpoint{12.175956in}{4.213609in}}%
\pgfusepath{clip}%
\pgfsetbuttcap%
\pgfsetroundjoin%
\pgfsetlinewidth{2.007500pt}%
\definecolor{currentstroke}{rgb}{0.501961,0.501961,0.501961}%
\pgfsetstrokecolor{currentstroke}%
\pgfsetstrokeopacity{0.500000}%
\pgfsetdash{}{0pt}%
\pgfpathmoveto{\pgfqpoint{7.545763in}{5.973132in}}%
\pgfpathlineto{\pgfqpoint{7.587117in}{5.973132in}}%
\pgfusepath{stroke}%
\end{pgfscope}%
\begin{pgfscope}%
\pgfpathrectangle{\pgfqpoint{0.100000in}{2.802285in}}{\pgfqpoint{12.175956in}{4.213609in}}%
\pgfusepath{clip}%
\pgfsetbuttcap%
\pgfsetroundjoin%
\pgfsetlinewidth{2.007500pt}%
\definecolor{currentstroke}{rgb}{0.501961,0.501961,0.501961}%
\pgfsetstrokecolor{currentstroke}%
\pgfsetstrokeopacity{0.500000}%
\pgfsetdash{}{0pt}%
\pgfpathmoveto{\pgfqpoint{7.545763in}{6.398749in}}%
\pgfpathlineto{\pgfqpoint{7.587117in}{6.398749in}}%
\pgfusepath{stroke}%
\end{pgfscope}%
\begin{pgfscope}%
\pgfpathrectangle{\pgfqpoint{0.100000in}{2.802285in}}{\pgfqpoint{12.175956in}{4.213609in}}%
\pgfusepath{clip}%
\pgfsetbuttcap%
\pgfsetroundjoin%
\pgfsetlinewidth{2.007500pt}%
\definecolor{currentstroke}{rgb}{0.501961,0.501961,0.501961}%
\pgfsetstrokecolor{currentstroke}%
\pgfsetstrokeopacity{0.500000}%
\pgfsetdash{}{0pt}%
\pgfpathmoveto{\pgfqpoint{7.545763in}{6.824366in}}%
\pgfpathlineto{\pgfqpoint{7.587117in}{6.824366in}}%
\pgfusepath{stroke}%
\end{pgfscope}%
\begin{pgfscope}%
\pgfpathrectangle{\pgfqpoint{0.100000in}{2.802285in}}{\pgfqpoint{12.175956in}{4.213609in}}%
\pgfusepath{clip}%
\pgfsetrectcap%
\pgfsetroundjoin%
\pgfsetlinewidth{2.007500pt}%
\definecolor{currentstroke}{rgb}{0.501961,0.501961,0.501961}%
\pgfsetstrokecolor{currentstroke}%
\pgfsetstrokeopacity{0.500000}%
\pgfsetdash{}{0pt}%
\pgfpathmoveto{\pgfqpoint{8.255671in}{2.802285in}}%
\pgfpathlineto{\pgfqpoint{8.255671in}{7.015894in}}%
\pgfusepath{stroke}%
\end{pgfscope}%
\begin{pgfscope}%
\pgfpathrectangle{\pgfqpoint{0.100000in}{2.802285in}}{\pgfqpoint{12.175956in}{4.213609in}}%
\pgfusepath{clip}%
\pgfsetbuttcap%
\pgfsetroundjoin%
\pgfsetlinewidth{2.007500pt}%
\definecolor{currentstroke}{rgb}{0.501961,0.501961,0.501961}%
\pgfsetstrokecolor{currentstroke}%
\pgfsetstrokeopacity{0.500000}%
\pgfsetdash{}{0pt}%
\pgfpathmoveto{\pgfqpoint{8.234994in}{2.993813in}}%
\pgfpathlineto{\pgfqpoint{8.276348in}{2.993813in}}%
\pgfusepath{stroke}%
\end{pgfscope}%
\begin{pgfscope}%
\pgfpathrectangle{\pgfqpoint{0.100000in}{2.802285in}}{\pgfqpoint{12.175956in}{4.213609in}}%
\pgfusepath{clip}%
\pgfsetbuttcap%
\pgfsetroundjoin%
\pgfsetlinewidth{2.007500pt}%
\definecolor{currentstroke}{rgb}{0.501961,0.501961,0.501961}%
\pgfsetstrokecolor{currentstroke}%
\pgfsetstrokeopacity{0.500000}%
\pgfsetdash{}{0pt}%
\pgfpathmoveto{\pgfqpoint{8.234994in}{3.419430in}}%
\pgfpathlineto{\pgfqpoint{8.276348in}{3.419430in}}%
\pgfusepath{stroke}%
\end{pgfscope}%
\begin{pgfscope}%
\pgfpathrectangle{\pgfqpoint{0.100000in}{2.802285in}}{\pgfqpoint{12.175956in}{4.213609in}}%
\pgfusepath{clip}%
\pgfsetbuttcap%
\pgfsetroundjoin%
\pgfsetlinewidth{2.007500pt}%
\definecolor{currentstroke}{rgb}{0.501961,0.501961,0.501961}%
\pgfsetstrokecolor{currentstroke}%
\pgfsetstrokeopacity{0.500000}%
\pgfsetdash{}{0pt}%
\pgfpathmoveto{\pgfqpoint{8.234994in}{3.845047in}}%
\pgfpathlineto{\pgfqpoint{8.276348in}{3.845047in}}%
\pgfusepath{stroke}%
\end{pgfscope}%
\begin{pgfscope}%
\pgfpathrectangle{\pgfqpoint{0.100000in}{2.802285in}}{\pgfqpoint{12.175956in}{4.213609in}}%
\pgfusepath{clip}%
\pgfsetbuttcap%
\pgfsetroundjoin%
\pgfsetlinewidth{2.007500pt}%
\definecolor{currentstroke}{rgb}{0.501961,0.501961,0.501961}%
\pgfsetstrokecolor{currentstroke}%
\pgfsetstrokeopacity{0.500000}%
\pgfsetdash{}{0pt}%
\pgfpathmoveto{\pgfqpoint{8.234994in}{4.270664in}}%
\pgfpathlineto{\pgfqpoint{8.276348in}{4.270664in}}%
\pgfusepath{stroke}%
\end{pgfscope}%
\begin{pgfscope}%
\pgfpathrectangle{\pgfqpoint{0.100000in}{2.802285in}}{\pgfqpoint{12.175956in}{4.213609in}}%
\pgfusepath{clip}%
\pgfsetbuttcap%
\pgfsetroundjoin%
\pgfsetlinewidth{2.007500pt}%
\definecolor{currentstroke}{rgb}{0.501961,0.501961,0.501961}%
\pgfsetstrokecolor{currentstroke}%
\pgfsetstrokeopacity{0.500000}%
\pgfsetdash{}{0pt}%
\pgfpathmoveto{\pgfqpoint{8.234994in}{4.696281in}}%
\pgfpathlineto{\pgfqpoint{8.276348in}{4.696281in}}%
\pgfusepath{stroke}%
\end{pgfscope}%
\begin{pgfscope}%
\pgfpathrectangle{\pgfqpoint{0.100000in}{2.802285in}}{\pgfqpoint{12.175956in}{4.213609in}}%
\pgfusepath{clip}%
\pgfsetbuttcap%
\pgfsetroundjoin%
\pgfsetlinewidth{2.007500pt}%
\definecolor{currentstroke}{rgb}{0.501961,0.501961,0.501961}%
\pgfsetstrokecolor{currentstroke}%
\pgfsetstrokeopacity{0.500000}%
\pgfsetdash{}{0pt}%
\pgfpathmoveto{\pgfqpoint{8.234994in}{5.121898in}}%
\pgfpathlineto{\pgfqpoint{8.276348in}{5.121898in}}%
\pgfusepath{stroke}%
\end{pgfscope}%
\begin{pgfscope}%
\pgfpathrectangle{\pgfqpoint{0.100000in}{2.802285in}}{\pgfqpoint{12.175956in}{4.213609in}}%
\pgfusepath{clip}%
\pgfsetbuttcap%
\pgfsetroundjoin%
\pgfsetlinewidth{2.007500pt}%
\definecolor{currentstroke}{rgb}{0.501961,0.501961,0.501961}%
\pgfsetstrokecolor{currentstroke}%
\pgfsetstrokeopacity{0.500000}%
\pgfsetdash{}{0pt}%
\pgfpathmoveto{\pgfqpoint{8.234994in}{5.547515in}}%
\pgfpathlineto{\pgfqpoint{8.276348in}{5.547515in}}%
\pgfusepath{stroke}%
\end{pgfscope}%
\begin{pgfscope}%
\pgfpathrectangle{\pgfqpoint{0.100000in}{2.802285in}}{\pgfqpoint{12.175956in}{4.213609in}}%
\pgfusepath{clip}%
\pgfsetbuttcap%
\pgfsetroundjoin%
\pgfsetlinewidth{2.007500pt}%
\definecolor{currentstroke}{rgb}{0.501961,0.501961,0.501961}%
\pgfsetstrokecolor{currentstroke}%
\pgfsetstrokeopacity{0.500000}%
\pgfsetdash{}{0pt}%
\pgfpathmoveto{\pgfqpoint{8.234994in}{5.973132in}}%
\pgfpathlineto{\pgfqpoint{8.276348in}{5.973132in}}%
\pgfusepath{stroke}%
\end{pgfscope}%
\begin{pgfscope}%
\pgfpathrectangle{\pgfqpoint{0.100000in}{2.802285in}}{\pgfqpoint{12.175956in}{4.213609in}}%
\pgfusepath{clip}%
\pgfsetbuttcap%
\pgfsetroundjoin%
\pgfsetlinewidth{2.007500pt}%
\definecolor{currentstroke}{rgb}{0.501961,0.501961,0.501961}%
\pgfsetstrokecolor{currentstroke}%
\pgfsetstrokeopacity{0.500000}%
\pgfsetdash{}{0pt}%
\pgfpathmoveto{\pgfqpoint{8.234994in}{6.398749in}}%
\pgfpathlineto{\pgfqpoint{8.276348in}{6.398749in}}%
\pgfusepath{stroke}%
\end{pgfscope}%
\begin{pgfscope}%
\pgfpathrectangle{\pgfqpoint{0.100000in}{2.802285in}}{\pgfqpoint{12.175956in}{4.213609in}}%
\pgfusepath{clip}%
\pgfsetbuttcap%
\pgfsetroundjoin%
\pgfsetlinewidth{2.007500pt}%
\definecolor{currentstroke}{rgb}{0.501961,0.501961,0.501961}%
\pgfsetstrokecolor{currentstroke}%
\pgfsetstrokeopacity{0.500000}%
\pgfsetdash{}{0pt}%
\pgfpathmoveto{\pgfqpoint{8.234994in}{6.824366in}}%
\pgfpathlineto{\pgfqpoint{8.276348in}{6.824366in}}%
\pgfusepath{stroke}%
\end{pgfscope}%
\begin{pgfscope}%
\pgfpathrectangle{\pgfqpoint{0.100000in}{2.802285in}}{\pgfqpoint{12.175956in}{4.213609in}}%
\pgfusepath{clip}%
\pgfsetrectcap%
\pgfsetroundjoin%
\pgfsetlinewidth{2.007500pt}%
\definecolor{currentstroke}{rgb}{0.501961,0.501961,0.501961}%
\pgfsetstrokecolor{currentstroke}%
\pgfsetstrokeopacity{0.500000}%
\pgfsetdash{}{0pt}%
\pgfpathmoveto{\pgfqpoint{8.944902in}{2.802285in}}%
\pgfpathlineto{\pgfqpoint{8.944902in}{7.015894in}}%
\pgfusepath{stroke}%
\end{pgfscope}%
\begin{pgfscope}%
\pgfpathrectangle{\pgfqpoint{0.100000in}{2.802285in}}{\pgfqpoint{12.175956in}{4.213609in}}%
\pgfusepath{clip}%
\pgfsetbuttcap%
\pgfsetroundjoin%
\pgfsetlinewidth{2.007500pt}%
\definecolor{currentstroke}{rgb}{0.501961,0.501961,0.501961}%
\pgfsetstrokecolor{currentstroke}%
\pgfsetstrokeopacity{0.500000}%
\pgfsetdash{}{0pt}%
\pgfpathmoveto{\pgfqpoint{8.924226in}{2.993813in}}%
\pgfpathlineto{\pgfqpoint{8.965579in}{2.993813in}}%
\pgfusepath{stroke}%
\end{pgfscope}%
\begin{pgfscope}%
\pgfpathrectangle{\pgfqpoint{0.100000in}{2.802285in}}{\pgfqpoint{12.175956in}{4.213609in}}%
\pgfusepath{clip}%
\pgfsetbuttcap%
\pgfsetroundjoin%
\pgfsetlinewidth{2.007500pt}%
\definecolor{currentstroke}{rgb}{0.501961,0.501961,0.501961}%
\pgfsetstrokecolor{currentstroke}%
\pgfsetstrokeopacity{0.500000}%
\pgfsetdash{}{0pt}%
\pgfpathmoveto{\pgfqpoint{8.924226in}{3.419430in}}%
\pgfpathlineto{\pgfqpoint{8.965579in}{3.419430in}}%
\pgfusepath{stroke}%
\end{pgfscope}%
\begin{pgfscope}%
\pgfpathrectangle{\pgfqpoint{0.100000in}{2.802285in}}{\pgfqpoint{12.175956in}{4.213609in}}%
\pgfusepath{clip}%
\pgfsetbuttcap%
\pgfsetroundjoin%
\pgfsetlinewidth{2.007500pt}%
\definecolor{currentstroke}{rgb}{0.501961,0.501961,0.501961}%
\pgfsetstrokecolor{currentstroke}%
\pgfsetstrokeopacity{0.500000}%
\pgfsetdash{}{0pt}%
\pgfpathmoveto{\pgfqpoint{8.924226in}{3.845047in}}%
\pgfpathlineto{\pgfqpoint{8.965579in}{3.845047in}}%
\pgfusepath{stroke}%
\end{pgfscope}%
\begin{pgfscope}%
\pgfpathrectangle{\pgfqpoint{0.100000in}{2.802285in}}{\pgfqpoint{12.175956in}{4.213609in}}%
\pgfusepath{clip}%
\pgfsetbuttcap%
\pgfsetroundjoin%
\pgfsetlinewidth{2.007500pt}%
\definecolor{currentstroke}{rgb}{0.501961,0.501961,0.501961}%
\pgfsetstrokecolor{currentstroke}%
\pgfsetstrokeopacity{0.500000}%
\pgfsetdash{}{0pt}%
\pgfpathmoveto{\pgfqpoint{8.924226in}{4.270664in}}%
\pgfpathlineto{\pgfqpoint{8.965579in}{4.270664in}}%
\pgfusepath{stroke}%
\end{pgfscope}%
\begin{pgfscope}%
\pgfpathrectangle{\pgfqpoint{0.100000in}{2.802285in}}{\pgfqpoint{12.175956in}{4.213609in}}%
\pgfusepath{clip}%
\pgfsetbuttcap%
\pgfsetroundjoin%
\pgfsetlinewidth{2.007500pt}%
\definecolor{currentstroke}{rgb}{0.501961,0.501961,0.501961}%
\pgfsetstrokecolor{currentstroke}%
\pgfsetstrokeopacity{0.500000}%
\pgfsetdash{}{0pt}%
\pgfpathmoveto{\pgfqpoint{8.924226in}{4.696281in}}%
\pgfpathlineto{\pgfqpoint{8.965579in}{4.696281in}}%
\pgfusepath{stroke}%
\end{pgfscope}%
\begin{pgfscope}%
\pgfpathrectangle{\pgfqpoint{0.100000in}{2.802285in}}{\pgfqpoint{12.175956in}{4.213609in}}%
\pgfusepath{clip}%
\pgfsetbuttcap%
\pgfsetroundjoin%
\pgfsetlinewidth{2.007500pt}%
\definecolor{currentstroke}{rgb}{0.501961,0.501961,0.501961}%
\pgfsetstrokecolor{currentstroke}%
\pgfsetstrokeopacity{0.500000}%
\pgfsetdash{}{0pt}%
\pgfpathmoveto{\pgfqpoint{8.924226in}{5.121898in}}%
\pgfpathlineto{\pgfqpoint{8.965579in}{5.121898in}}%
\pgfusepath{stroke}%
\end{pgfscope}%
\begin{pgfscope}%
\pgfpathrectangle{\pgfqpoint{0.100000in}{2.802285in}}{\pgfqpoint{12.175956in}{4.213609in}}%
\pgfusepath{clip}%
\pgfsetbuttcap%
\pgfsetroundjoin%
\pgfsetlinewidth{2.007500pt}%
\definecolor{currentstroke}{rgb}{0.501961,0.501961,0.501961}%
\pgfsetstrokecolor{currentstroke}%
\pgfsetstrokeopacity{0.500000}%
\pgfsetdash{}{0pt}%
\pgfpathmoveto{\pgfqpoint{8.924226in}{5.547515in}}%
\pgfpathlineto{\pgfqpoint{8.965579in}{5.547515in}}%
\pgfusepath{stroke}%
\end{pgfscope}%
\begin{pgfscope}%
\pgfpathrectangle{\pgfqpoint{0.100000in}{2.802285in}}{\pgfqpoint{12.175956in}{4.213609in}}%
\pgfusepath{clip}%
\pgfsetbuttcap%
\pgfsetroundjoin%
\pgfsetlinewidth{2.007500pt}%
\definecolor{currentstroke}{rgb}{0.501961,0.501961,0.501961}%
\pgfsetstrokecolor{currentstroke}%
\pgfsetstrokeopacity{0.500000}%
\pgfsetdash{}{0pt}%
\pgfpathmoveto{\pgfqpoint{8.924226in}{5.973132in}}%
\pgfpathlineto{\pgfqpoint{8.965579in}{5.973132in}}%
\pgfusepath{stroke}%
\end{pgfscope}%
\begin{pgfscope}%
\pgfpathrectangle{\pgfqpoint{0.100000in}{2.802285in}}{\pgfqpoint{12.175956in}{4.213609in}}%
\pgfusepath{clip}%
\pgfsetbuttcap%
\pgfsetroundjoin%
\pgfsetlinewidth{2.007500pt}%
\definecolor{currentstroke}{rgb}{0.501961,0.501961,0.501961}%
\pgfsetstrokecolor{currentstroke}%
\pgfsetstrokeopacity{0.500000}%
\pgfsetdash{}{0pt}%
\pgfpathmoveto{\pgfqpoint{8.924226in}{6.398749in}}%
\pgfpathlineto{\pgfqpoint{8.965579in}{6.398749in}}%
\pgfusepath{stroke}%
\end{pgfscope}%
\begin{pgfscope}%
\pgfpathrectangle{\pgfqpoint{0.100000in}{2.802285in}}{\pgfqpoint{12.175956in}{4.213609in}}%
\pgfusepath{clip}%
\pgfsetbuttcap%
\pgfsetroundjoin%
\pgfsetlinewidth{2.007500pt}%
\definecolor{currentstroke}{rgb}{0.501961,0.501961,0.501961}%
\pgfsetstrokecolor{currentstroke}%
\pgfsetstrokeopacity{0.500000}%
\pgfsetdash{}{0pt}%
\pgfpathmoveto{\pgfqpoint{8.924226in}{6.824366in}}%
\pgfpathlineto{\pgfqpoint{8.965579in}{6.824366in}}%
\pgfusepath{stroke}%
\end{pgfscope}%
\begin{pgfscope}%
\pgfpathrectangle{\pgfqpoint{0.100000in}{2.802285in}}{\pgfqpoint{12.175956in}{4.213609in}}%
\pgfusepath{clip}%
\pgfsetrectcap%
\pgfsetroundjoin%
\pgfsetlinewidth{2.007500pt}%
\definecolor{currentstroke}{rgb}{0.501961,0.501961,0.501961}%
\pgfsetstrokecolor{currentstroke}%
\pgfsetstrokeopacity{0.500000}%
\pgfsetdash{}{0pt}%
\pgfpathmoveto{\pgfqpoint{9.634134in}{2.802285in}}%
\pgfpathlineto{\pgfqpoint{9.634134in}{7.015894in}}%
\pgfusepath{stroke}%
\end{pgfscope}%
\begin{pgfscope}%
\pgfpathrectangle{\pgfqpoint{0.100000in}{2.802285in}}{\pgfqpoint{12.175956in}{4.213609in}}%
\pgfusepath{clip}%
\pgfsetbuttcap%
\pgfsetroundjoin%
\pgfsetlinewidth{2.007500pt}%
\definecolor{currentstroke}{rgb}{0.501961,0.501961,0.501961}%
\pgfsetstrokecolor{currentstroke}%
\pgfsetstrokeopacity{0.500000}%
\pgfsetdash{}{0pt}%
\pgfpathmoveto{\pgfqpoint{9.613457in}{2.993813in}}%
\pgfpathlineto{\pgfqpoint{9.654811in}{2.993813in}}%
\pgfusepath{stroke}%
\end{pgfscope}%
\begin{pgfscope}%
\pgfpathrectangle{\pgfqpoint{0.100000in}{2.802285in}}{\pgfqpoint{12.175956in}{4.213609in}}%
\pgfusepath{clip}%
\pgfsetbuttcap%
\pgfsetroundjoin%
\pgfsetlinewidth{2.007500pt}%
\definecolor{currentstroke}{rgb}{0.501961,0.501961,0.501961}%
\pgfsetstrokecolor{currentstroke}%
\pgfsetstrokeopacity{0.500000}%
\pgfsetdash{}{0pt}%
\pgfpathmoveto{\pgfqpoint{9.613457in}{3.419430in}}%
\pgfpathlineto{\pgfqpoint{9.654811in}{3.419430in}}%
\pgfusepath{stroke}%
\end{pgfscope}%
\begin{pgfscope}%
\pgfpathrectangle{\pgfqpoint{0.100000in}{2.802285in}}{\pgfqpoint{12.175956in}{4.213609in}}%
\pgfusepath{clip}%
\pgfsetbuttcap%
\pgfsetroundjoin%
\pgfsetlinewidth{2.007500pt}%
\definecolor{currentstroke}{rgb}{0.501961,0.501961,0.501961}%
\pgfsetstrokecolor{currentstroke}%
\pgfsetstrokeopacity{0.500000}%
\pgfsetdash{}{0pt}%
\pgfpathmoveto{\pgfqpoint{9.613457in}{3.845047in}}%
\pgfpathlineto{\pgfqpoint{9.654811in}{3.845047in}}%
\pgfusepath{stroke}%
\end{pgfscope}%
\begin{pgfscope}%
\pgfpathrectangle{\pgfqpoint{0.100000in}{2.802285in}}{\pgfqpoint{12.175956in}{4.213609in}}%
\pgfusepath{clip}%
\pgfsetbuttcap%
\pgfsetroundjoin%
\pgfsetlinewidth{2.007500pt}%
\definecolor{currentstroke}{rgb}{0.501961,0.501961,0.501961}%
\pgfsetstrokecolor{currentstroke}%
\pgfsetstrokeopacity{0.500000}%
\pgfsetdash{}{0pt}%
\pgfpathmoveto{\pgfqpoint{9.613457in}{4.270664in}}%
\pgfpathlineto{\pgfqpoint{9.654811in}{4.270664in}}%
\pgfusepath{stroke}%
\end{pgfscope}%
\begin{pgfscope}%
\pgfpathrectangle{\pgfqpoint{0.100000in}{2.802285in}}{\pgfqpoint{12.175956in}{4.213609in}}%
\pgfusepath{clip}%
\pgfsetbuttcap%
\pgfsetroundjoin%
\pgfsetlinewidth{2.007500pt}%
\definecolor{currentstroke}{rgb}{0.501961,0.501961,0.501961}%
\pgfsetstrokecolor{currentstroke}%
\pgfsetstrokeopacity{0.500000}%
\pgfsetdash{}{0pt}%
\pgfpathmoveto{\pgfqpoint{9.613457in}{4.696281in}}%
\pgfpathlineto{\pgfqpoint{9.654811in}{4.696281in}}%
\pgfusepath{stroke}%
\end{pgfscope}%
\begin{pgfscope}%
\pgfpathrectangle{\pgfqpoint{0.100000in}{2.802285in}}{\pgfqpoint{12.175956in}{4.213609in}}%
\pgfusepath{clip}%
\pgfsetbuttcap%
\pgfsetroundjoin%
\pgfsetlinewidth{2.007500pt}%
\definecolor{currentstroke}{rgb}{0.501961,0.501961,0.501961}%
\pgfsetstrokecolor{currentstroke}%
\pgfsetstrokeopacity{0.500000}%
\pgfsetdash{}{0pt}%
\pgfpathmoveto{\pgfqpoint{9.613457in}{5.121898in}}%
\pgfpathlineto{\pgfqpoint{9.654811in}{5.121898in}}%
\pgfusepath{stroke}%
\end{pgfscope}%
\begin{pgfscope}%
\pgfpathrectangle{\pgfqpoint{0.100000in}{2.802285in}}{\pgfqpoint{12.175956in}{4.213609in}}%
\pgfusepath{clip}%
\pgfsetbuttcap%
\pgfsetroundjoin%
\pgfsetlinewidth{2.007500pt}%
\definecolor{currentstroke}{rgb}{0.501961,0.501961,0.501961}%
\pgfsetstrokecolor{currentstroke}%
\pgfsetstrokeopacity{0.500000}%
\pgfsetdash{}{0pt}%
\pgfpathmoveto{\pgfqpoint{9.613457in}{5.547515in}}%
\pgfpathlineto{\pgfqpoint{9.654811in}{5.547515in}}%
\pgfusepath{stroke}%
\end{pgfscope}%
\begin{pgfscope}%
\pgfpathrectangle{\pgfqpoint{0.100000in}{2.802285in}}{\pgfqpoint{12.175956in}{4.213609in}}%
\pgfusepath{clip}%
\pgfsetbuttcap%
\pgfsetroundjoin%
\pgfsetlinewidth{2.007500pt}%
\definecolor{currentstroke}{rgb}{0.501961,0.501961,0.501961}%
\pgfsetstrokecolor{currentstroke}%
\pgfsetstrokeopacity{0.500000}%
\pgfsetdash{}{0pt}%
\pgfpathmoveto{\pgfqpoint{9.613457in}{5.973132in}}%
\pgfpathlineto{\pgfqpoint{9.654811in}{5.973132in}}%
\pgfusepath{stroke}%
\end{pgfscope}%
\begin{pgfscope}%
\pgfpathrectangle{\pgfqpoint{0.100000in}{2.802285in}}{\pgfqpoint{12.175956in}{4.213609in}}%
\pgfusepath{clip}%
\pgfsetbuttcap%
\pgfsetroundjoin%
\pgfsetlinewidth{2.007500pt}%
\definecolor{currentstroke}{rgb}{0.501961,0.501961,0.501961}%
\pgfsetstrokecolor{currentstroke}%
\pgfsetstrokeopacity{0.500000}%
\pgfsetdash{}{0pt}%
\pgfpathmoveto{\pgfqpoint{9.613457in}{6.398749in}}%
\pgfpathlineto{\pgfqpoint{9.654811in}{6.398749in}}%
\pgfusepath{stroke}%
\end{pgfscope}%
\begin{pgfscope}%
\pgfpathrectangle{\pgfqpoint{0.100000in}{2.802285in}}{\pgfqpoint{12.175956in}{4.213609in}}%
\pgfusepath{clip}%
\pgfsetbuttcap%
\pgfsetroundjoin%
\pgfsetlinewidth{2.007500pt}%
\definecolor{currentstroke}{rgb}{0.501961,0.501961,0.501961}%
\pgfsetstrokecolor{currentstroke}%
\pgfsetstrokeopacity{0.500000}%
\pgfsetdash{}{0pt}%
\pgfpathmoveto{\pgfqpoint{9.613457in}{6.824366in}}%
\pgfpathlineto{\pgfqpoint{9.654811in}{6.824366in}}%
\pgfusepath{stroke}%
\end{pgfscope}%
\begin{pgfscope}%
\pgfpathrectangle{\pgfqpoint{0.100000in}{2.802285in}}{\pgfqpoint{12.175956in}{4.213609in}}%
\pgfusepath{clip}%
\pgfsetrectcap%
\pgfsetroundjoin%
\pgfsetlinewidth{2.007500pt}%
\definecolor{currentstroke}{rgb}{0.501961,0.501961,0.501961}%
\pgfsetstrokecolor{currentstroke}%
\pgfsetstrokeopacity{0.500000}%
\pgfsetdash{}{0pt}%
\pgfpathmoveto{\pgfqpoint{10.323365in}{2.802285in}}%
\pgfpathlineto{\pgfqpoint{10.323365in}{7.015894in}}%
\pgfusepath{stroke}%
\end{pgfscope}%
\begin{pgfscope}%
\pgfpathrectangle{\pgfqpoint{0.100000in}{2.802285in}}{\pgfqpoint{12.175956in}{4.213609in}}%
\pgfusepath{clip}%
\pgfsetbuttcap%
\pgfsetroundjoin%
\pgfsetlinewidth{2.007500pt}%
\definecolor{currentstroke}{rgb}{0.501961,0.501961,0.501961}%
\pgfsetstrokecolor{currentstroke}%
\pgfsetstrokeopacity{0.500000}%
\pgfsetdash{}{0pt}%
\pgfpathmoveto{\pgfqpoint{10.302688in}{2.993813in}}%
\pgfpathlineto{\pgfqpoint{10.344042in}{2.993813in}}%
\pgfusepath{stroke}%
\end{pgfscope}%
\begin{pgfscope}%
\pgfpathrectangle{\pgfqpoint{0.100000in}{2.802285in}}{\pgfqpoint{12.175956in}{4.213609in}}%
\pgfusepath{clip}%
\pgfsetbuttcap%
\pgfsetroundjoin%
\pgfsetlinewidth{2.007500pt}%
\definecolor{currentstroke}{rgb}{0.501961,0.501961,0.501961}%
\pgfsetstrokecolor{currentstroke}%
\pgfsetstrokeopacity{0.500000}%
\pgfsetdash{}{0pt}%
\pgfpathmoveto{\pgfqpoint{10.302688in}{3.419430in}}%
\pgfpathlineto{\pgfqpoint{10.344042in}{3.419430in}}%
\pgfusepath{stroke}%
\end{pgfscope}%
\begin{pgfscope}%
\pgfpathrectangle{\pgfqpoint{0.100000in}{2.802285in}}{\pgfqpoint{12.175956in}{4.213609in}}%
\pgfusepath{clip}%
\pgfsetbuttcap%
\pgfsetroundjoin%
\pgfsetlinewidth{2.007500pt}%
\definecolor{currentstroke}{rgb}{0.501961,0.501961,0.501961}%
\pgfsetstrokecolor{currentstroke}%
\pgfsetstrokeopacity{0.500000}%
\pgfsetdash{}{0pt}%
\pgfpathmoveto{\pgfqpoint{10.302688in}{3.845047in}}%
\pgfpathlineto{\pgfqpoint{10.344042in}{3.845047in}}%
\pgfusepath{stroke}%
\end{pgfscope}%
\begin{pgfscope}%
\pgfpathrectangle{\pgfqpoint{0.100000in}{2.802285in}}{\pgfqpoint{12.175956in}{4.213609in}}%
\pgfusepath{clip}%
\pgfsetbuttcap%
\pgfsetroundjoin%
\pgfsetlinewidth{2.007500pt}%
\definecolor{currentstroke}{rgb}{0.501961,0.501961,0.501961}%
\pgfsetstrokecolor{currentstroke}%
\pgfsetstrokeopacity{0.500000}%
\pgfsetdash{}{0pt}%
\pgfpathmoveto{\pgfqpoint{10.302688in}{4.270664in}}%
\pgfpathlineto{\pgfqpoint{10.344042in}{4.270664in}}%
\pgfusepath{stroke}%
\end{pgfscope}%
\begin{pgfscope}%
\pgfpathrectangle{\pgfqpoint{0.100000in}{2.802285in}}{\pgfqpoint{12.175956in}{4.213609in}}%
\pgfusepath{clip}%
\pgfsetbuttcap%
\pgfsetroundjoin%
\pgfsetlinewidth{2.007500pt}%
\definecolor{currentstroke}{rgb}{0.501961,0.501961,0.501961}%
\pgfsetstrokecolor{currentstroke}%
\pgfsetstrokeopacity{0.500000}%
\pgfsetdash{}{0pt}%
\pgfpathmoveto{\pgfqpoint{10.302688in}{4.696281in}}%
\pgfpathlineto{\pgfqpoint{10.344042in}{4.696281in}}%
\pgfusepath{stroke}%
\end{pgfscope}%
\begin{pgfscope}%
\pgfpathrectangle{\pgfqpoint{0.100000in}{2.802285in}}{\pgfqpoint{12.175956in}{4.213609in}}%
\pgfusepath{clip}%
\pgfsetbuttcap%
\pgfsetroundjoin%
\pgfsetlinewidth{2.007500pt}%
\definecolor{currentstroke}{rgb}{0.501961,0.501961,0.501961}%
\pgfsetstrokecolor{currentstroke}%
\pgfsetstrokeopacity{0.500000}%
\pgfsetdash{}{0pt}%
\pgfpathmoveto{\pgfqpoint{10.302688in}{5.121898in}}%
\pgfpathlineto{\pgfqpoint{10.344042in}{5.121898in}}%
\pgfusepath{stroke}%
\end{pgfscope}%
\begin{pgfscope}%
\pgfpathrectangle{\pgfqpoint{0.100000in}{2.802285in}}{\pgfqpoint{12.175956in}{4.213609in}}%
\pgfusepath{clip}%
\pgfsetbuttcap%
\pgfsetroundjoin%
\pgfsetlinewidth{2.007500pt}%
\definecolor{currentstroke}{rgb}{0.501961,0.501961,0.501961}%
\pgfsetstrokecolor{currentstroke}%
\pgfsetstrokeopacity{0.500000}%
\pgfsetdash{}{0pt}%
\pgfpathmoveto{\pgfqpoint{10.302688in}{5.547515in}}%
\pgfpathlineto{\pgfqpoint{10.344042in}{5.547515in}}%
\pgfusepath{stroke}%
\end{pgfscope}%
\begin{pgfscope}%
\pgfpathrectangle{\pgfqpoint{0.100000in}{2.802285in}}{\pgfqpoint{12.175956in}{4.213609in}}%
\pgfusepath{clip}%
\pgfsetbuttcap%
\pgfsetroundjoin%
\pgfsetlinewidth{2.007500pt}%
\definecolor{currentstroke}{rgb}{0.501961,0.501961,0.501961}%
\pgfsetstrokecolor{currentstroke}%
\pgfsetstrokeopacity{0.500000}%
\pgfsetdash{}{0pt}%
\pgfpathmoveto{\pgfqpoint{10.302688in}{5.973132in}}%
\pgfpathlineto{\pgfqpoint{10.344042in}{5.973132in}}%
\pgfusepath{stroke}%
\end{pgfscope}%
\begin{pgfscope}%
\pgfpathrectangle{\pgfqpoint{0.100000in}{2.802285in}}{\pgfqpoint{12.175956in}{4.213609in}}%
\pgfusepath{clip}%
\pgfsetbuttcap%
\pgfsetroundjoin%
\pgfsetlinewidth{2.007500pt}%
\definecolor{currentstroke}{rgb}{0.501961,0.501961,0.501961}%
\pgfsetstrokecolor{currentstroke}%
\pgfsetstrokeopacity{0.500000}%
\pgfsetdash{}{0pt}%
\pgfpathmoveto{\pgfqpoint{10.302688in}{6.398749in}}%
\pgfpathlineto{\pgfqpoint{10.344042in}{6.398749in}}%
\pgfusepath{stroke}%
\end{pgfscope}%
\begin{pgfscope}%
\pgfpathrectangle{\pgfqpoint{0.100000in}{2.802285in}}{\pgfqpoint{12.175956in}{4.213609in}}%
\pgfusepath{clip}%
\pgfsetbuttcap%
\pgfsetroundjoin%
\pgfsetlinewidth{2.007500pt}%
\definecolor{currentstroke}{rgb}{0.501961,0.501961,0.501961}%
\pgfsetstrokecolor{currentstroke}%
\pgfsetstrokeopacity{0.500000}%
\pgfsetdash{}{0pt}%
\pgfpathmoveto{\pgfqpoint{10.302688in}{6.824366in}}%
\pgfpathlineto{\pgfqpoint{10.344042in}{6.824366in}}%
\pgfusepath{stroke}%
\end{pgfscope}%
\begin{pgfscope}%
\pgfpathrectangle{\pgfqpoint{0.100000in}{2.802285in}}{\pgfqpoint{12.175956in}{4.213609in}}%
\pgfusepath{clip}%
\pgfsetrectcap%
\pgfsetroundjoin%
\pgfsetlinewidth{2.007500pt}%
\definecolor{currentstroke}{rgb}{0.501961,0.501961,0.501961}%
\pgfsetstrokecolor{currentstroke}%
\pgfsetstrokeopacity{0.500000}%
\pgfsetdash{}{0pt}%
\pgfpathmoveto{\pgfqpoint{11.012596in}{2.802285in}}%
\pgfpathlineto{\pgfqpoint{11.012596in}{7.015894in}}%
\pgfusepath{stroke}%
\end{pgfscope}%
\begin{pgfscope}%
\pgfpathrectangle{\pgfqpoint{0.100000in}{2.802285in}}{\pgfqpoint{12.175956in}{4.213609in}}%
\pgfusepath{clip}%
\pgfsetbuttcap%
\pgfsetroundjoin%
\pgfsetlinewidth{2.007500pt}%
\definecolor{currentstroke}{rgb}{0.501961,0.501961,0.501961}%
\pgfsetstrokecolor{currentstroke}%
\pgfsetstrokeopacity{0.500000}%
\pgfsetdash{}{0pt}%
\pgfpathmoveto{\pgfqpoint{10.991919in}{2.993813in}}%
\pgfpathlineto{\pgfqpoint{11.033273in}{2.993813in}}%
\pgfusepath{stroke}%
\end{pgfscope}%
\begin{pgfscope}%
\pgfpathrectangle{\pgfqpoint{0.100000in}{2.802285in}}{\pgfqpoint{12.175956in}{4.213609in}}%
\pgfusepath{clip}%
\pgfsetbuttcap%
\pgfsetroundjoin%
\pgfsetlinewidth{2.007500pt}%
\definecolor{currentstroke}{rgb}{0.501961,0.501961,0.501961}%
\pgfsetstrokecolor{currentstroke}%
\pgfsetstrokeopacity{0.500000}%
\pgfsetdash{}{0pt}%
\pgfpathmoveto{\pgfqpoint{10.991919in}{3.419430in}}%
\pgfpathlineto{\pgfqpoint{11.033273in}{3.419430in}}%
\pgfusepath{stroke}%
\end{pgfscope}%
\begin{pgfscope}%
\pgfpathrectangle{\pgfqpoint{0.100000in}{2.802285in}}{\pgfqpoint{12.175956in}{4.213609in}}%
\pgfusepath{clip}%
\pgfsetbuttcap%
\pgfsetroundjoin%
\pgfsetlinewidth{2.007500pt}%
\definecolor{currentstroke}{rgb}{0.501961,0.501961,0.501961}%
\pgfsetstrokecolor{currentstroke}%
\pgfsetstrokeopacity{0.500000}%
\pgfsetdash{}{0pt}%
\pgfpathmoveto{\pgfqpoint{10.991919in}{3.845047in}}%
\pgfpathlineto{\pgfqpoint{11.033273in}{3.845047in}}%
\pgfusepath{stroke}%
\end{pgfscope}%
\begin{pgfscope}%
\pgfpathrectangle{\pgfqpoint{0.100000in}{2.802285in}}{\pgfqpoint{12.175956in}{4.213609in}}%
\pgfusepath{clip}%
\pgfsetbuttcap%
\pgfsetroundjoin%
\pgfsetlinewidth{2.007500pt}%
\definecolor{currentstroke}{rgb}{0.501961,0.501961,0.501961}%
\pgfsetstrokecolor{currentstroke}%
\pgfsetstrokeopacity{0.500000}%
\pgfsetdash{}{0pt}%
\pgfpathmoveto{\pgfqpoint{10.991919in}{4.270664in}}%
\pgfpathlineto{\pgfqpoint{11.033273in}{4.270664in}}%
\pgfusepath{stroke}%
\end{pgfscope}%
\begin{pgfscope}%
\pgfpathrectangle{\pgfqpoint{0.100000in}{2.802285in}}{\pgfqpoint{12.175956in}{4.213609in}}%
\pgfusepath{clip}%
\pgfsetbuttcap%
\pgfsetroundjoin%
\pgfsetlinewidth{2.007500pt}%
\definecolor{currentstroke}{rgb}{0.501961,0.501961,0.501961}%
\pgfsetstrokecolor{currentstroke}%
\pgfsetstrokeopacity{0.500000}%
\pgfsetdash{}{0pt}%
\pgfpathmoveto{\pgfqpoint{10.991919in}{4.696281in}}%
\pgfpathlineto{\pgfqpoint{11.033273in}{4.696281in}}%
\pgfusepath{stroke}%
\end{pgfscope}%
\begin{pgfscope}%
\pgfpathrectangle{\pgfqpoint{0.100000in}{2.802285in}}{\pgfqpoint{12.175956in}{4.213609in}}%
\pgfusepath{clip}%
\pgfsetbuttcap%
\pgfsetroundjoin%
\pgfsetlinewidth{2.007500pt}%
\definecolor{currentstroke}{rgb}{0.501961,0.501961,0.501961}%
\pgfsetstrokecolor{currentstroke}%
\pgfsetstrokeopacity{0.500000}%
\pgfsetdash{}{0pt}%
\pgfpathmoveto{\pgfqpoint{10.991919in}{5.121898in}}%
\pgfpathlineto{\pgfqpoint{11.033273in}{5.121898in}}%
\pgfusepath{stroke}%
\end{pgfscope}%
\begin{pgfscope}%
\pgfpathrectangle{\pgfqpoint{0.100000in}{2.802285in}}{\pgfqpoint{12.175956in}{4.213609in}}%
\pgfusepath{clip}%
\pgfsetbuttcap%
\pgfsetroundjoin%
\pgfsetlinewidth{2.007500pt}%
\definecolor{currentstroke}{rgb}{0.501961,0.501961,0.501961}%
\pgfsetstrokecolor{currentstroke}%
\pgfsetstrokeopacity{0.500000}%
\pgfsetdash{}{0pt}%
\pgfpathmoveto{\pgfqpoint{10.991919in}{5.547515in}}%
\pgfpathlineto{\pgfqpoint{11.033273in}{5.547515in}}%
\pgfusepath{stroke}%
\end{pgfscope}%
\begin{pgfscope}%
\pgfpathrectangle{\pgfqpoint{0.100000in}{2.802285in}}{\pgfqpoint{12.175956in}{4.213609in}}%
\pgfusepath{clip}%
\pgfsetbuttcap%
\pgfsetroundjoin%
\pgfsetlinewidth{2.007500pt}%
\definecolor{currentstroke}{rgb}{0.501961,0.501961,0.501961}%
\pgfsetstrokecolor{currentstroke}%
\pgfsetstrokeopacity{0.500000}%
\pgfsetdash{}{0pt}%
\pgfpathmoveto{\pgfqpoint{10.991919in}{5.973132in}}%
\pgfpathlineto{\pgfqpoint{11.033273in}{5.973132in}}%
\pgfusepath{stroke}%
\end{pgfscope}%
\begin{pgfscope}%
\pgfpathrectangle{\pgfqpoint{0.100000in}{2.802285in}}{\pgfqpoint{12.175956in}{4.213609in}}%
\pgfusepath{clip}%
\pgfsetbuttcap%
\pgfsetroundjoin%
\pgfsetlinewidth{2.007500pt}%
\definecolor{currentstroke}{rgb}{0.501961,0.501961,0.501961}%
\pgfsetstrokecolor{currentstroke}%
\pgfsetstrokeopacity{0.500000}%
\pgfsetdash{}{0pt}%
\pgfpathmoveto{\pgfqpoint{10.991919in}{6.398749in}}%
\pgfpathlineto{\pgfqpoint{11.033273in}{6.398749in}}%
\pgfusepath{stroke}%
\end{pgfscope}%
\begin{pgfscope}%
\pgfpathrectangle{\pgfqpoint{0.100000in}{2.802285in}}{\pgfqpoint{12.175956in}{4.213609in}}%
\pgfusepath{clip}%
\pgfsetbuttcap%
\pgfsetroundjoin%
\pgfsetlinewidth{2.007500pt}%
\definecolor{currentstroke}{rgb}{0.501961,0.501961,0.501961}%
\pgfsetstrokecolor{currentstroke}%
\pgfsetstrokeopacity{0.500000}%
\pgfsetdash{}{0pt}%
\pgfpathmoveto{\pgfqpoint{10.991919in}{6.824366in}}%
\pgfpathlineto{\pgfqpoint{11.033273in}{6.824366in}}%
\pgfusepath{stroke}%
\end{pgfscope}%
\begin{pgfscope}%
\pgfpathrectangle{\pgfqpoint{0.100000in}{2.802285in}}{\pgfqpoint{12.175956in}{4.213609in}}%
\pgfusepath{clip}%
\pgfsetrectcap%
\pgfsetroundjoin%
\pgfsetlinewidth{2.007500pt}%
\definecolor{currentstroke}{rgb}{0.501961,0.501961,0.501961}%
\pgfsetstrokecolor{currentstroke}%
\pgfsetstrokeopacity{0.500000}%
\pgfsetdash{}{0pt}%
\pgfpathmoveto{\pgfqpoint{11.701827in}{2.802285in}}%
\pgfpathlineto{\pgfqpoint{11.701827in}{7.015894in}}%
\pgfusepath{stroke}%
\end{pgfscope}%
\begin{pgfscope}%
\pgfpathrectangle{\pgfqpoint{0.100000in}{2.802285in}}{\pgfqpoint{12.175956in}{4.213609in}}%
\pgfusepath{clip}%
\pgfsetbuttcap%
\pgfsetroundjoin%
\pgfsetlinewidth{2.007500pt}%
\definecolor{currentstroke}{rgb}{0.501961,0.501961,0.501961}%
\pgfsetstrokecolor{currentstroke}%
\pgfsetstrokeopacity{0.500000}%
\pgfsetdash{}{0pt}%
\pgfpathmoveto{\pgfqpoint{11.681150in}{2.993813in}}%
\pgfpathlineto{\pgfqpoint{11.722504in}{2.993813in}}%
\pgfusepath{stroke}%
\end{pgfscope}%
\begin{pgfscope}%
\pgfpathrectangle{\pgfqpoint{0.100000in}{2.802285in}}{\pgfqpoint{12.175956in}{4.213609in}}%
\pgfusepath{clip}%
\pgfsetbuttcap%
\pgfsetroundjoin%
\pgfsetlinewidth{2.007500pt}%
\definecolor{currentstroke}{rgb}{0.501961,0.501961,0.501961}%
\pgfsetstrokecolor{currentstroke}%
\pgfsetstrokeopacity{0.500000}%
\pgfsetdash{}{0pt}%
\pgfpathmoveto{\pgfqpoint{11.681150in}{3.419430in}}%
\pgfpathlineto{\pgfqpoint{11.722504in}{3.419430in}}%
\pgfusepath{stroke}%
\end{pgfscope}%
\begin{pgfscope}%
\pgfpathrectangle{\pgfqpoint{0.100000in}{2.802285in}}{\pgfqpoint{12.175956in}{4.213609in}}%
\pgfusepath{clip}%
\pgfsetbuttcap%
\pgfsetroundjoin%
\pgfsetlinewidth{2.007500pt}%
\definecolor{currentstroke}{rgb}{0.501961,0.501961,0.501961}%
\pgfsetstrokecolor{currentstroke}%
\pgfsetstrokeopacity{0.500000}%
\pgfsetdash{}{0pt}%
\pgfpathmoveto{\pgfqpoint{11.681150in}{3.845047in}}%
\pgfpathlineto{\pgfqpoint{11.722504in}{3.845047in}}%
\pgfusepath{stroke}%
\end{pgfscope}%
\begin{pgfscope}%
\pgfpathrectangle{\pgfqpoint{0.100000in}{2.802285in}}{\pgfqpoint{12.175956in}{4.213609in}}%
\pgfusepath{clip}%
\pgfsetbuttcap%
\pgfsetroundjoin%
\pgfsetlinewidth{2.007500pt}%
\definecolor{currentstroke}{rgb}{0.501961,0.501961,0.501961}%
\pgfsetstrokecolor{currentstroke}%
\pgfsetstrokeopacity{0.500000}%
\pgfsetdash{}{0pt}%
\pgfpathmoveto{\pgfqpoint{11.681150in}{4.270664in}}%
\pgfpathlineto{\pgfqpoint{11.722504in}{4.270664in}}%
\pgfusepath{stroke}%
\end{pgfscope}%
\begin{pgfscope}%
\pgfpathrectangle{\pgfqpoint{0.100000in}{2.802285in}}{\pgfqpoint{12.175956in}{4.213609in}}%
\pgfusepath{clip}%
\pgfsetbuttcap%
\pgfsetroundjoin%
\pgfsetlinewidth{2.007500pt}%
\definecolor{currentstroke}{rgb}{0.501961,0.501961,0.501961}%
\pgfsetstrokecolor{currentstroke}%
\pgfsetstrokeopacity{0.500000}%
\pgfsetdash{}{0pt}%
\pgfpathmoveto{\pgfqpoint{11.681150in}{4.696281in}}%
\pgfpathlineto{\pgfqpoint{11.722504in}{4.696281in}}%
\pgfusepath{stroke}%
\end{pgfscope}%
\begin{pgfscope}%
\pgfpathrectangle{\pgfqpoint{0.100000in}{2.802285in}}{\pgfqpoint{12.175956in}{4.213609in}}%
\pgfusepath{clip}%
\pgfsetbuttcap%
\pgfsetroundjoin%
\pgfsetlinewidth{2.007500pt}%
\definecolor{currentstroke}{rgb}{0.501961,0.501961,0.501961}%
\pgfsetstrokecolor{currentstroke}%
\pgfsetstrokeopacity{0.500000}%
\pgfsetdash{}{0pt}%
\pgfpathmoveto{\pgfqpoint{11.681150in}{5.121898in}}%
\pgfpathlineto{\pgfqpoint{11.722504in}{5.121898in}}%
\pgfusepath{stroke}%
\end{pgfscope}%
\begin{pgfscope}%
\pgfpathrectangle{\pgfqpoint{0.100000in}{2.802285in}}{\pgfqpoint{12.175956in}{4.213609in}}%
\pgfusepath{clip}%
\pgfsetbuttcap%
\pgfsetroundjoin%
\pgfsetlinewidth{2.007500pt}%
\definecolor{currentstroke}{rgb}{0.501961,0.501961,0.501961}%
\pgfsetstrokecolor{currentstroke}%
\pgfsetstrokeopacity{0.500000}%
\pgfsetdash{}{0pt}%
\pgfpathmoveto{\pgfqpoint{11.681150in}{5.547515in}}%
\pgfpathlineto{\pgfqpoint{11.722504in}{5.547515in}}%
\pgfusepath{stroke}%
\end{pgfscope}%
\begin{pgfscope}%
\pgfpathrectangle{\pgfqpoint{0.100000in}{2.802285in}}{\pgfqpoint{12.175956in}{4.213609in}}%
\pgfusepath{clip}%
\pgfsetbuttcap%
\pgfsetroundjoin%
\pgfsetlinewidth{2.007500pt}%
\definecolor{currentstroke}{rgb}{0.501961,0.501961,0.501961}%
\pgfsetstrokecolor{currentstroke}%
\pgfsetstrokeopacity{0.500000}%
\pgfsetdash{}{0pt}%
\pgfpathmoveto{\pgfqpoint{11.681150in}{5.973132in}}%
\pgfpathlineto{\pgfqpoint{11.722504in}{5.973132in}}%
\pgfusepath{stroke}%
\end{pgfscope}%
\begin{pgfscope}%
\pgfpathrectangle{\pgfqpoint{0.100000in}{2.802285in}}{\pgfqpoint{12.175956in}{4.213609in}}%
\pgfusepath{clip}%
\pgfsetbuttcap%
\pgfsetroundjoin%
\pgfsetlinewidth{2.007500pt}%
\definecolor{currentstroke}{rgb}{0.501961,0.501961,0.501961}%
\pgfsetstrokecolor{currentstroke}%
\pgfsetstrokeopacity{0.500000}%
\pgfsetdash{}{0pt}%
\pgfpathmoveto{\pgfqpoint{11.681150in}{6.398749in}}%
\pgfpathlineto{\pgfqpoint{11.722504in}{6.398749in}}%
\pgfusepath{stroke}%
\end{pgfscope}%
\begin{pgfscope}%
\pgfpathrectangle{\pgfqpoint{0.100000in}{2.802285in}}{\pgfqpoint{12.175956in}{4.213609in}}%
\pgfusepath{clip}%
\pgfsetbuttcap%
\pgfsetroundjoin%
\pgfsetlinewidth{2.007500pt}%
\definecolor{currentstroke}{rgb}{0.501961,0.501961,0.501961}%
\pgfsetstrokecolor{currentstroke}%
\pgfsetstrokeopacity{0.500000}%
\pgfsetdash{}{0pt}%
\pgfpathmoveto{\pgfqpoint{11.681150in}{6.824366in}}%
\pgfpathlineto{\pgfqpoint{11.722504in}{6.824366in}}%
\pgfusepath{stroke}%
\end{pgfscope}%
\begin{pgfscope}%
\pgfsetrectcap%
\pgfsetmiterjoin%
\pgfsetlinewidth{0.803000pt}%
\definecolor{currentstroke}{rgb}{0.000000,0.000000,0.000000}%
\pgfsetstrokecolor{currentstroke}%
\pgfsetdash{}{0pt}%
\pgfpathmoveto{\pgfqpoint{0.100000in}{2.802285in}}%
\pgfpathlineto{\pgfqpoint{12.275956in}{2.802285in}}%
\pgfusepath{stroke}%
\end{pgfscope}%
\begin{pgfscope}%
\pgfsetrectcap%
\pgfsetmiterjoin%
\pgfsetlinewidth{0.803000pt}%
\definecolor{currentstroke}{rgb}{0.000000,0.000000,0.000000}%
\pgfsetstrokecolor{currentstroke}%
\pgfsetdash{}{0pt}%
\pgfpathmoveto{\pgfqpoint{0.100000in}{7.015894in}}%
\pgfpathlineto{\pgfqpoint{12.275956in}{7.015894in}}%
\pgfusepath{stroke}%
\end{pgfscope}%
\begin{pgfscope}%
\definecolor{textcolor}{rgb}{0.000000,0.000000,0.000000}%
\pgfsetstrokecolor{textcolor}%
\pgfsetfillcolor{textcolor}%
\pgftext[x=0.618991in,y=2.610757in,left,base]{\color{textcolor}{\rmfamily\fontsize{10.000000}{12.000000}\selectfont\catcode`\^=\active\def^{\ifmmode\sp\else\^{}\fi}\catcode`\%=\active\def%{\%}1.40}}%
\end{pgfscope}%
\begin{pgfscope}%
\definecolor{textcolor}{rgb}{0.000000,0.000000,0.000000}%
\pgfsetstrokecolor{textcolor}%
\pgfsetfillcolor{textcolor}%
\pgftext[x=0.618991in,y=7.111658in,left,base]{\color{textcolor}{\rmfamily\fontsize{10.000000}{12.000000}\selectfont\catcode`\^=\active\def^{\ifmmode\sp\else\^{}\fi}\catcode`\%=\active\def%{\%}31.97}}%
\end{pgfscope}%
\begin{pgfscope}%
\definecolor{textcolor}{rgb}{0.000000,0.000000,0.000000}%
\pgfsetstrokecolor{textcolor}%
\pgfsetfillcolor{textcolor}%
\pgftext[x=1.308222in,y=2.610757in,left,base]{\color{textcolor}{\rmfamily\fontsize{10.000000}{12.000000}\selectfont\catcode`\^=\active\def^{\ifmmode\sp\else\^{}\fi}\catcode`\%=\active\def%{\%}1.02e+06}}%
\end{pgfscope}%
\begin{pgfscope}%
\definecolor{textcolor}{rgb}{0.000000,0.000000,0.000000}%
\pgfsetstrokecolor{textcolor}%
\pgfsetfillcolor{textcolor}%
\pgftext[x=1.308222in,y=7.111658in,left,base]{\color{textcolor}{\rmfamily\fontsize{10.000000}{12.000000}\selectfont\catcode`\^=\active\def^{\ifmmode\sp\else\^{}\fi}\catcode`\%=\active\def%{\%}1.78e+06}}%
\end{pgfscope}%
\begin{pgfscope}%
\definecolor{textcolor}{rgb}{0.000000,0.000000,0.000000}%
\pgfsetstrokecolor{textcolor}%
\pgfsetfillcolor{textcolor}%
\pgftext[x=1.997453in,y=2.610757in,left,base]{\color{textcolor}{\rmfamily\fontsize{10.000000}{12.000000}\selectfont\catcode`\^=\active\def^{\ifmmode\sp\else\^{}\fi}\catcode`\%=\active\def%{\%}669.00}}%
\end{pgfscope}%
\begin{pgfscope}%
\definecolor{textcolor}{rgb}{0.000000,0.000000,0.000000}%
\pgfsetstrokecolor{textcolor}%
\pgfsetfillcolor{textcolor}%
\pgftext[x=1.997453in,y=7.111658in,left,base]{\color{textcolor}{\rmfamily\fontsize{10.000000}{12.000000}\selectfont\catcode`\^=\active\def^{\ifmmode\sp\else\^{}\fi}\catcode`\%=\active\def%{\%}9.48e+03}}%
\end{pgfscope}%
\begin{pgfscope}%
\definecolor{textcolor}{rgb}{0.000000,0.000000,0.000000}%
\pgfsetstrokecolor{textcolor}%
\pgfsetfillcolor{textcolor}%
\pgftext[x=2.686684in,y=2.610757in,left,base]{\color{textcolor}{\rmfamily\fontsize{10.000000}{12.000000}\selectfont\catcode`\^=\active\def^{\ifmmode\sp\else\^{}\fi}\catcode`\%=\active\def%{\%}0.00}}%
\end{pgfscope}%
\begin{pgfscope}%
\definecolor{textcolor}{rgb}{0.000000,0.000000,0.000000}%
\pgfsetstrokecolor{textcolor}%
\pgfsetfillcolor{textcolor}%
\pgftext[x=2.686684in,y=7.111658in,left,base]{\color{textcolor}{\rmfamily\fontsize{10.000000}{12.000000}\selectfont\catcode`\^=\active\def^{\ifmmode\sp\else\^{}\fi}\catcode`\%=\active\def%{\%}184.74}}%
\end{pgfscope}%
\begin{pgfscope}%
\definecolor{textcolor}{rgb}{0.000000,0.000000,0.000000}%
\pgfsetstrokecolor{textcolor}%
\pgfsetfillcolor{textcolor}%
\pgftext[x=3.375915in,y=2.610757in,left,base]{\color{textcolor}{\rmfamily\fontsize{10.000000}{12.000000}\selectfont\catcode`\^=\active\def^{\ifmmode\sp\else\^{}\fi}\catcode`\%=\active\def%{\%}342.20}}%
\end{pgfscope}%
\begin{pgfscope}%
\definecolor{textcolor}{rgb}{0.000000,0.000000,0.000000}%
\pgfsetstrokecolor{textcolor}%
\pgfsetfillcolor{textcolor}%
\pgftext[x=3.375915in,y=7.111658in,left,base]{\color{textcolor}{\rmfamily\fontsize{10.000000}{12.000000}\selectfont\catcode`\^=\active\def^{\ifmmode\sp\else\^{}\fi}\catcode`\%=\active\def%{\%}1.16e+03}}%
\end{pgfscope}%
\begin{pgfscope}%
\definecolor{textcolor}{rgb}{0.000000,0.000000,0.000000}%
\pgfsetstrokecolor{textcolor}%
\pgfsetfillcolor{textcolor}%
\pgftext[x=4.065146in,y=2.610757in,left,base]{\color{textcolor}{\rmfamily\fontsize{10.000000}{12.000000}\selectfont\catcode`\^=\active\def^{\ifmmode\sp\else\^{}\fi}\catcode`\%=\active\def%{\%}0.00}}%
\end{pgfscope}%
\begin{pgfscope}%
\definecolor{textcolor}{rgb}{0.000000,0.000000,0.000000}%
\pgfsetstrokecolor{textcolor}%
\pgfsetfillcolor{textcolor}%
\pgftext[x=4.065146in,y=7.111658in,left,base]{\color{textcolor}{\rmfamily\fontsize{10.000000}{12.000000}\selectfont\catcode`\^=\active\def^{\ifmmode\sp\else\^{}\fi}\catcode`\%=\active\def%{\%}1.00}}%
\end{pgfscope}%
\begin{pgfscope}%
\definecolor{textcolor}{rgb}{0.000000,0.000000,0.000000}%
\pgfsetstrokecolor{textcolor}%
\pgfsetfillcolor{textcolor}%
\pgftext[x=4.754378in,y=2.610757in,left,base]{\color{textcolor}{\rmfamily\fontsize{10.000000}{12.000000}\selectfont\catcode`\^=\active\def^{\ifmmode\sp\else\^{}\fi}\catcode`\%=\active\def%{\%}0.09}}%
\end{pgfscope}%
\begin{pgfscope}%
\definecolor{textcolor}{rgb}{0.000000,0.000000,0.000000}%
\pgfsetstrokecolor{textcolor}%
\pgfsetfillcolor{textcolor}%
\pgftext[x=4.754378in,y=7.111658in,left,base]{\color{textcolor}{\rmfamily\fontsize{10.000000}{12.000000}\selectfont\catcode`\^=\active\def^{\ifmmode\sp\else\^{}\fi}\catcode`\%=\active\def%{\%}0.17}}%
\end{pgfscope}%
\begin{pgfscope}%
\definecolor{textcolor}{rgb}{0.000000,0.000000,0.000000}%
\pgfsetstrokecolor{textcolor}%
\pgfsetfillcolor{textcolor}%
\pgftext[x=5.443609in,y=2.610757in,left,base]{\color{textcolor}{\rmfamily\fontsize{10.000000}{12.000000}\selectfont\catcode`\^=\active\def^{\ifmmode\sp\else\^{}\fi}\catcode`\%=\active\def%{\%}2.38e+04}}%
\end{pgfscope}%
\begin{pgfscope}%
\definecolor{textcolor}{rgb}{0.000000,0.000000,0.000000}%
\pgfsetstrokecolor{textcolor}%
\pgfsetfillcolor{textcolor}%
\pgftext[x=5.443609in,y=7.111658in,left,base]{\color{textcolor}{\rmfamily\fontsize{10.000000}{12.000000}\selectfont\catcode`\^=\active\def^{\ifmmode\sp\else\^{}\fi}\catcode`\%=\active\def%{\%}3.78e+04}}%
\end{pgfscope}%
\begin{pgfscope}%
\definecolor{textcolor}{rgb}{0.000000,0.000000,0.000000}%
\pgfsetstrokecolor{textcolor}%
\pgfsetfillcolor{textcolor}%
\pgftext[x=6.132840in,y=2.610757in,left,base]{\color{textcolor}{\rmfamily\fontsize{10.000000}{12.000000}\selectfont\catcode`\^=\active\def^{\ifmmode\sp\else\^{}\fi}\catcode`\%=\active\def%{\%}31.80}}%
\end{pgfscope}%
\begin{pgfscope}%
\definecolor{textcolor}{rgb}{0.000000,0.000000,0.000000}%
\pgfsetstrokecolor{textcolor}%
\pgfsetfillcolor{textcolor}%
\pgftext[x=6.132840in,y=7.111658in,left,base]{\color{textcolor}{\rmfamily\fontsize{10.000000}{12.000000}\selectfont\catcode`\^=\active\def^{\ifmmode\sp\else\^{}\fi}\catcode`\%=\active\def%{\%}67.30}}%
\end{pgfscope}%
\begin{pgfscope}%
\definecolor{textcolor}{rgb}{0.000000,0.000000,0.000000}%
\pgfsetstrokecolor{textcolor}%
\pgfsetfillcolor{textcolor}%
\pgftext[x=6.822071in,y=2.610757in,left,base]{\color{textcolor}{\rmfamily\fontsize{10.000000}{12.000000}\selectfont\catcode`\^=\active\def^{\ifmmode\sp\else\^{}\fi}\catcode`\%=\active\def%{\%}1.14}}%
\end{pgfscope}%
\begin{pgfscope}%
\definecolor{textcolor}{rgb}{0.000000,0.000000,0.000000}%
\pgfsetstrokecolor{textcolor}%
\pgfsetfillcolor{textcolor}%
\pgftext[x=6.822071in,y=7.111658in,left,base]{\color{textcolor}{\rmfamily\fontsize{10.000000}{12.000000}\selectfont\catcode`\^=\active\def^{\ifmmode\sp\else\^{}\fi}\catcode`\%=\active\def%{\%}4.54}}%
\end{pgfscope}%
\begin{pgfscope}%
\definecolor{textcolor}{rgb}{0.000000,0.000000,0.000000}%
\pgfsetstrokecolor{textcolor}%
\pgfsetfillcolor{textcolor}%
\pgftext[x=7.511302in,y=2.610757in,left,base]{\color{textcolor}{\rmfamily\fontsize{10.000000}{12.000000}\selectfont\catcode`\^=\active\def^{\ifmmode\sp\else\^{}\fi}\catcode`\%=\active\def%{\%}1.55}}%
\end{pgfscope}%
\begin{pgfscope}%
\definecolor{textcolor}{rgb}{0.000000,0.000000,0.000000}%
\pgfsetstrokecolor{textcolor}%
\pgfsetfillcolor{textcolor}%
\pgftext[x=7.511302in,y=7.111658in,left,base]{\color{textcolor}{\rmfamily\fontsize{10.000000}{12.000000}\selectfont\catcode`\^=\active\def^{\ifmmode\sp\else\^{}\fi}\catcode`\%=\active\def%{\%}186.62}}%
\end{pgfscope}%
\begin{pgfscope}%
\definecolor{textcolor}{rgb}{0.000000,0.000000,0.000000}%
\pgfsetstrokecolor{textcolor}%
\pgfsetfillcolor{textcolor}%
\pgftext[x=8.200533in,y=2.610757in,left,base]{\color{textcolor}{\rmfamily\fontsize{10.000000}{12.000000}\selectfont\catcode`\^=\active\def^{\ifmmode\sp\else\^{}\fi}\catcode`\%=\active\def%{\%}0.00}}%
\end{pgfscope}%
\begin{pgfscope}%
\definecolor{textcolor}{rgb}{0.000000,0.000000,0.000000}%
\pgfsetstrokecolor{textcolor}%
\pgfsetfillcolor{textcolor}%
\pgftext[x=8.200533in,y=7.111658in,left,base]{\color{textcolor}{\rmfamily\fontsize{10.000000}{12.000000}\selectfont\catcode`\^=\active\def^{\ifmmode\sp\else\^{}\fi}\catcode`\%=\active\def%{\%}2.05}}%
\end{pgfscope}%
\begin{pgfscope}%
\definecolor{textcolor}{rgb}{0.000000,0.000000,0.000000}%
\pgfsetstrokecolor{textcolor}%
\pgfsetfillcolor{textcolor}%
\pgftext[x=8.889764in,y=2.610757in,left,base]{\color{textcolor}{\rmfamily\fontsize{10.000000}{12.000000}\selectfont\catcode`\^=\active\def^{\ifmmode\sp\else\^{}\fi}\catcode`\%=\active\def%{\%}3.00}}%
\end{pgfscope}%
\begin{pgfscope}%
\definecolor{textcolor}{rgb}{0.000000,0.000000,0.000000}%
\pgfsetstrokecolor{textcolor}%
\pgfsetfillcolor{textcolor}%
\pgftext[x=8.889764in,y=7.111658in,left,base]{\color{textcolor}{\rmfamily\fontsize{10.000000}{12.000000}\selectfont\catcode`\^=\active\def^{\ifmmode\sp\else\^{}\fi}\catcode`\%=\active\def%{\%}5.00}}%
\end{pgfscope}%
\begin{pgfscope}%
\definecolor{textcolor}{rgb}{0.000000,0.000000,0.000000}%
\pgfsetstrokecolor{textcolor}%
\pgfsetfillcolor{textcolor}%
\pgftext[x=9.578995in,y=2.610757in,left,base]{\color{textcolor}{\rmfamily\fontsize{10.000000}{12.000000}\selectfont\catcode`\^=\active\def^{\ifmmode\sp\else\^{}\fi}\catcode`\%=\active\def%{\%}2.00}}%
\end{pgfscope}%
\begin{pgfscope}%
\definecolor{textcolor}{rgb}{0.000000,0.000000,0.000000}%
\pgfsetstrokecolor{textcolor}%
\pgfsetfillcolor{textcolor}%
\pgftext[x=9.578995in,y=7.111658in,left,base]{\color{textcolor}{\rmfamily\fontsize{10.000000}{12.000000}\selectfont\catcode`\^=\active\def^{\ifmmode\sp\else\^{}\fi}\catcode`\%=\active\def%{\%}2.00}}%
\end{pgfscope}%
\begin{pgfscope}%
\definecolor{textcolor}{rgb}{0.000000,0.000000,0.000000}%
\pgfsetstrokecolor{textcolor}%
\pgfsetfillcolor{textcolor}%
\pgftext[x=10.268226in,y=2.610757in,left,base]{\color{textcolor}{\rmfamily\fontsize{10.000000}{12.000000}\selectfont\catcode`\^=\active\def^{\ifmmode\sp\else\^{}\fi}\catcode`\%=\active\def%{\%}3.00}}%
\end{pgfscope}%
\begin{pgfscope}%
\definecolor{textcolor}{rgb}{0.000000,0.000000,0.000000}%
\pgfsetstrokecolor{textcolor}%
\pgfsetfillcolor{textcolor}%
\pgftext[x=10.268226in,y=7.111658in,left,base]{\color{textcolor}{\rmfamily\fontsize{10.000000}{12.000000}\selectfont\catcode`\^=\active\def^{\ifmmode\sp\else\^{}\fi}\catcode`\%=\active\def%{\%}4.00}}%
\end{pgfscope}%
\begin{pgfscope}%
\definecolor{textcolor}{rgb}{0.000000,0.000000,0.000000}%
\pgfsetstrokecolor{textcolor}%
\pgfsetfillcolor{textcolor}%
\pgftext[x=10.957457in,y=2.610757in,left,base]{\color{textcolor}{\rmfamily\fontsize{10.000000}{12.000000}\selectfont\catcode`\^=\active\def^{\ifmmode\sp\else\^{}\fi}\catcode`\%=\active\def%{\%}1.00}}%
\end{pgfscope}%
\begin{pgfscope}%
\definecolor{textcolor}{rgb}{0.000000,0.000000,0.000000}%
\pgfsetstrokecolor{textcolor}%
\pgfsetfillcolor{textcolor}%
\pgftext[x=10.957457in,y=7.111658in,left,base]{\color{textcolor}{\rmfamily\fontsize{10.000000}{12.000000}\selectfont\catcode`\^=\active\def^{\ifmmode\sp\else\^{}\fi}\catcode`\%=\active\def%{\%}3.00}}%
\end{pgfscope}%
\begin{pgfscope}%
\definecolor{textcolor}{rgb}{0.000000,0.000000,0.000000}%
\pgfsetstrokecolor{textcolor}%
\pgfsetfillcolor{textcolor}%
\pgftext[x=11.646688in,y=2.610757in,left,base]{\color{textcolor}{\rmfamily\fontsize{10.000000}{12.000000}\selectfont\catcode`\^=\active\def^{\ifmmode\sp\else\^{}\fi}\catcode`\%=\active\def%{\%}3.00}}%
\end{pgfscope}%
\begin{pgfscope}%
\definecolor{textcolor}{rgb}{0.000000,0.000000,0.000000}%
\pgfsetstrokecolor{textcolor}%
\pgfsetfillcolor{textcolor}%
\pgftext[x=11.646688in,y=7.111658in,left,base]{\color{textcolor}{\rmfamily\fontsize{10.000000}{12.000000}\selectfont\catcode`\^=\active\def^{\ifmmode\sp\else\^{}\fi}\catcode`\%=\active\def%{\%}3.00}}%
\end{pgfscope}%
\begin{pgfscope}%
\definecolor{textcolor}{rgb}{0.000000,0.000000,0.000000}%
\pgfsetstrokecolor{textcolor}%
\pgfsetfillcolor{textcolor}%
\pgftext[x=5.261131in, y=7.728636in, left, base]{\color{textcolor}{\rmfamily\fontsize{20.000000}{24.000000}\selectfont\catcode`\^=\active\def^{\ifmmode\sp\else\^{}\fi}\catcode`\%=\active\def%{\%}Objective Space}}%
\end{pgfscope}%
\begin{pgfscope}%
\definecolor{textcolor}{rgb}{0.000000,0.000000,0.000000}%
\pgfsetstrokecolor{textcolor}%
\pgfsetfillcolor{textcolor}%
\pgftext[x=4.961261in, y=7.432561in, left, base]{\color{textcolor}{\rmfamily\fontsize{20.000000}{24.000000}\selectfont\catcode`\^=\active\def^{\ifmmode\sp\else\^{}\fi}\catcode`\%=\active\def%{\%} Non-Pareto Optimal}}%
\end{pgfscope}%
\begin{pgfscope}%
\pgfsetbuttcap%
\pgfsetmiterjoin%
\definecolor{currentfill}{rgb}{1.000000,1.000000,1.000000}%
\pgfsetfillcolor{currentfill}%
\pgfsetfillopacity{0.800000}%
\pgfsetlinewidth{1.003750pt}%
\definecolor{currentstroke}{rgb}{0.800000,0.800000,0.800000}%
\pgfsetstrokecolor{currentstroke}%
\pgfsetstrokeopacity{0.800000}%
\pgfsetdash{}{0pt}%
\pgfpathmoveto{\pgfqpoint{0.255556in}{4.242438in}}%
\pgfpathlineto{\pgfqpoint{1.511877in}{4.242438in}}%
\pgfpathquadraticcurveto{\pgfqpoint{1.556322in}{4.242438in}}{\pgfqpoint{1.556322in}{4.286883in}}%
\pgfpathlineto{\pgfqpoint{1.556322in}{6.860338in}}%
\pgfpathquadraticcurveto{\pgfqpoint{1.556322in}{6.904783in}}{\pgfqpoint{1.511877in}{6.904783in}}%
\pgfpathlineto{\pgfqpoint{0.255556in}{6.904783in}}%
\pgfpathquadraticcurveto{\pgfqpoint{0.211111in}{6.904783in}}{\pgfqpoint{0.211111in}{6.860338in}}%
\pgfpathlineto{\pgfqpoint{0.211111in}{4.286883in}}%
\pgfpathquadraticcurveto{\pgfqpoint{0.211111in}{4.242438in}}{\pgfqpoint{0.255556in}{4.242438in}}%
\pgfpathlineto{\pgfqpoint{0.255556in}{4.242438in}}%
\pgfpathclose%
\pgfusepath{stroke,fill}%
\end{pgfscope}%
\begin{pgfscope}%
\pgfsetrectcap%
\pgfsetroundjoin%
\pgfsetlinewidth{1.505625pt}%
\definecolor{currentstroke}{rgb}{0.839216,0.152941,0.156863}%
\pgfsetstrokecolor{currentstroke}%
\pgfsetdash{}{0pt}%
\pgfpathmoveto{\pgfqpoint{0.300000in}{6.727005in}}%
\pgfpathlineto{\pgfqpoint{0.522222in}{6.727005in}}%
\pgfpathlineto{\pgfqpoint{0.744444in}{6.727005in}}%
\pgfusepath{stroke}%
\end{pgfscope}%
\begin{pgfscope}%
\definecolor{textcolor}{rgb}{0.000000,0.000000,0.000000}%
\pgfsetstrokecolor{textcolor}%
\pgfsetfillcolor{textcolor}%
\pgftext[x=0.922222in,y=6.649227in,left,base]{\color{textcolor}{\rmfamily\fontsize{16.000000}{19.200000}\selectfont\catcode`\^=\active\def^{\ifmmode\sp\else\^{}\fi}\catcode`\%=\active\def%{\%}EG07}}%
\end{pgfscope}%
\begin{pgfscope}%
\pgfsetrectcap%
\pgfsetroundjoin%
\pgfsetlinewidth{1.505625pt}%
\definecolor{currentstroke}{rgb}{0.172549,0.627451,0.172549}%
\pgfsetstrokecolor{currentstroke}%
\pgfsetdash{}{0pt}%
\pgfpathmoveto{\pgfqpoint{0.300000in}{6.402545in}}%
\pgfpathlineto{\pgfqpoint{0.522222in}{6.402545in}}%
\pgfpathlineto{\pgfqpoint{0.744444in}{6.402545in}}%
\pgfusepath{stroke}%
\end{pgfscope}%
\begin{pgfscope}%
\definecolor{textcolor}{rgb}{0.000000,0.000000,0.000000}%
\pgfsetstrokecolor{textcolor}%
\pgfsetfillcolor{textcolor}%
\pgftext[x=0.922222in,y=6.324768in,left,base]{\color{textcolor}{\rmfamily\fontsize{16.000000}{19.200000}\selectfont\catcode`\^=\active\def^{\ifmmode\sp\else\^{}\fi}\catcode`\%=\active\def%{\%}EG11}}%
\end{pgfscope}%
\begin{pgfscope}%
\pgfsetrectcap%
\pgfsetroundjoin%
\pgfsetlinewidth{1.505625pt}%
\definecolor{currentstroke}{rgb}{0.172549,0.627451,0.172549}%
\pgfsetstrokecolor{currentstroke}%
\pgfsetdash{}{0pt}%
\pgfpathmoveto{\pgfqpoint{0.300000in}{6.078086in}}%
\pgfpathlineto{\pgfqpoint{0.522222in}{6.078086in}}%
\pgfpathlineto{\pgfqpoint{0.744444in}{6.078086in}}%
\pgfusepath{stroke}%
\end{pgfscope}%
\begin{pgfscope}%
\definecolor{textcolor}{rgb}{0.000000,0.000000,0.000000}%
\pgfsetstrokecolor{textcolor}%
\pgfsetfillcolor{textcolor}%
\pgftext[x=0.922222in,y=6.000308in,left,base]{\color{textcolor}{\rmfamily\fontsize{16.000000}{19.200000}\selectfont\catcode`\^=\active\def^{\ifmmode\sp\else\^{}\fi}\catcode`\%=\active\def%{\%}EG16}}%
\end{pgfscope}%
\begin{pgfscope}%
\pgfsetrectcap%
\pgfsetroundjoin%
\pgfsetlinewidth{1.505625pt}%
\definecolor{currentstroke}{rgb}{0.121569,0.466667,0.705882}%
\pgfsetstrokecolor{currentstroke}%
\pgfsetdash{}{0pt}%
\pgfpathmoveto{\pgfqpoint{0.300000in}{5.753626in}}%
\pgfpathlineto{\pgfqpoint{0.522222in}{5.753626in}}%
\pgfpathlineto{\pgfqpoint{0.744444in}{5.753626in}}%
\pgfusepath{stroke}%
\end{pgfscope}%
\begin{pgfscope}%
\definecolor{textcolor}{rgb}{0.000000,0.000000,0.000000}%
\pgfsetstrokecolor{textcolor}%
\pgfsetfillcolor{textcolor}%
\pgftext[x=0.922222in,y=5.675848in,left,base]{\color{textcolor}{\rmfamily\fontsize{16.000000}{19.200000}\selectfont\catcode`\^=\active\def^{\ifmmode\sp\else\^{}\fi}\catcode`\%=\active\def%{\%}EG27}}%
\end{pgfscope}%
\begin{pgfscope}%
\pgfsetrectcap%
\pgfsetroundjoin%
\pgfsetlinewidth{1.505625pt}%
\definecolor{currentstroke}{rgb}{0.121569,0.466667,0.705882}%
\pgfsetstrokecolor{currentstroke}%
\pgfsetdash{}{0pt}%
\pgfpathmoveto{\pgfqpoint{0.300000in}{5.429166in}}%
\pgfpathlineto{\pgfqpoint{0.522222in}{5.429166in}}%
\pgfpathlineto{\pgfqpoint{0.744444in}{5.429166in}}%
\pgfusepath{stroke}%
\end{pgfscope}%
\begin{pgfscope}%
\definecolor{textcolor}{rgb}{0.000000,0.000000,0.000000}%
\pgfsetstrokecolor{textcolor}%
\pgfsetfillcolor{textcolor}%
\pgftext[x=0.922222in,y=5.351388in,left,base]{\color{textcolor}{\rmfamily\fontsize{16.000000}{19.200000}\selectfont\catcode`\^=\active\def^{\ifmmode\sp\else\^{}\fi}\catcode`\%=\active\def%{\%}EG33}}%
\end{pgfscope}%
\begin{pgfscope}%
\pgfsetrectcap%
\pgfsetroundjoin%
\pgfsetlinewidth{1.505625pt}%
\definecolor{currentstroke}{rgb}{0.121569,0.466667,0.705882}%
\pgfsetstrokecolor{currentstroke}%
\pgfsetdash{}{0pt}%
\pgfpathmoveto{\pgfqpoint{0.300000in}{5.104706in}}%
\pgfpathlineto{\pgfqpoint{0.522222in}{5.104706in}}%
\pgfpathlineto{\pgfqpoint{0.744444in}{5.104706in}}%
\pgfusepath{stroke}%
\end{pgfscope}%
\begin{pgfscope}%
\definecolor{textcolor}{rgb}{0.000000,0.000000,0.000000}%
\pgfsetstrokecolor{textcolor}%
\pgfsetfillcolor{textcolor}%
\pgftext[x=0.922222in,y=5.026929in,left,base]{\color{textcolor}{\rmfamily\fontsize{16.000000}{19.200000}\selectfont\catcode`\^=\active\def^{\ifmmode\sp\else\^{}\fi}\catcode`\%=\active\def%{\%}EG34}}%
\end{pgfscope}%
\begin{pgfscope}%
\pgfsetrectcap%
\pgfsetroundjoin%
\pgfsetlinewidth{1.505625pt}%
\definecolor{currentstroke}{rgb}{0.121569,0.466667,0.705882}%
\pgfsetstrokecolor{currentstroke}%
\pgfsetdash{}{0pt}%
\pgfpathmoveto{\pgfqpoint{0.300000in}{4.780247in}}%
\pgfpathlineto{\pgfqpoint{0.522222in}{4.780247in}}%
\pgfpathlineto{\pgfqpoint{0.744444in}{4.780247in}}%
\pgfusepath{stroke}%
\end{pgfscope}%
\begin{pgfscope}%
\definecolor{textcolor}{rgb}{0.000000,0.000000,0.000000}%
\pgfsetstrokecolor{textcolor}%
\pgfsetfillcolor{textcolor}%
\pgftext[x=0.922222in,y=4.702469in,left,base]{\color{textcolor}{\rmfamily\fontsize{16.000000}{19.200000}\selectfont\catcode`\^=\active\def^{\ifmmode\sp\else\^{}\fi}\catcode`\%=\active\def%{\%}EG35}}%
\end{pgfscope}%
\begin{pgfscope}%
\pgfsetrectcap%
\pgfsetroundjoin%
\pgfsetlinewidth{1.505625pt}%
\definecolor{currentstroke}{rgb}{0.121569,0.466667,0.705882}%
\pgfsetstrokecolor{currentstroke}%
\pgfsetdash{}{0pt}%
\pgfpathmoveto{\pgfqpoint{0.300000in}{4.455787in}}%
\pgfpathlineto{\pgfqpoint{0.522222in}{4.455787in}}%
\pgfpathlineto{\pgfqpoint{0.744444in}{4.455787in}}%
\pgfusepath{stroke}%
\end{pgfscope}%
\begin{pgfscope}%
\definecolor{textcolor}{rgb}{0.000000,0.000000,0.000000}%
\pgfsetstrokecolor{textcolor}%
\pgfsetfillcolor{textcolor}%
\pgftext[x=0.922222in,y=4.378009in,left,base]{\color{textcolor}{\rmfamily\fontsize{16.000000}{19.200000}\selectfont\catcode`\^=\active\def^{\ifmmode\sp\else\^{}\fi}\catcode`\%=\active\def%{\%}EG39}}%
\end{pgfscope}%
\end{pgfpicture}%
\makeatother%
\endgroup%
}
        \caption{}
        \label{fig:non-optimal}
    \end{center}
\end{figure}

% \begin{figure}[ht!]
%     \begin{center}
%         \resizebox{\columnwidth}{!}{\input{images/}}
%         \caption{}
%         \label{fig:}
%     \end{center}
% \end{figure}

% Repeat a combination of figures and paragraphs like this until the key results
% are communicated. If any discussion is needed to illuminate the interpretation
% of the work, you can do so here or, you can include this information in the
% conclusions you discuss in the ``Discussion'' section. 
