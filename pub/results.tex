\section{Results}
Start with the critical figures from Chapter 5 of your thesis and build up just 
enough detail around them to support the conclusions that you draw. While you 
want to keep this level of detail fairly minimal, it must inlude at least a 
paragraph for each experiment of the following form (the below is just an 
example):

\emph{Figure X shows that Osier achieved solutions  within 0.5\%  of those achieved 
by Temoa. This level of agreement is well within the uncertainty expected in an 
energy system model and Osier can therefore be considered in agreement with 
Temo. This clearly indicates that Osier's multi-objective 
optimization can be reduced to single-objective optimization at the level of 
detail common in existing \glspl{esom}. }

Repeat a combination of figures and paragraphs like this until the key results 
are communicated. If any discussion is needed to illuminate the interpretation 
of the work, you can do so here or, you can include this information in the 
conclusions you discuss in the ``Discussion'' section. 
