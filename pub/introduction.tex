\section{Introduction}

%% General Background
% The general background should demarcate the overall scientific setting of 
% your work. Start with a general topic that everyone in your audience cares 
% about. Note that the general background should give your audience a sense of 
% what to expect from your paper, not an overview of the history of a field. 
% Introduce only necessary background that is related to your work, and make 
% sure it can narrow down to your thesis.

%% Specific background: Work done so far
% Give your reader a sense of previous accomplishments, current contradictions, 
% and competing theories in the field. Cite previous work that illustrates your 
% narrative and gives a balanced description of the scientific landscape on 
% this research topic.  

%%  Knowledge gap: Motivation for your work Give 
% evidence of the incompleteness of the current understanding and of the value 
% of investigating the field further. What is the gap that needs to be filled? 
% Demonstrate the importance of this unsolved problem as the motivation for 
% your work.  2

%% Aim of your paper Finally, clearly state the aim and scope 
% of this article (not the project) and what exact question is answered. You 
% may also briefly explain how the study was conducted, and share a preview of 
% your findings.

Preventing the worst impacts of climate change will require our globalized society to transition from
fossil fuel infrastructure to clean energy infrastructure. This transition must
also be done equitably and justly to prevent entrenching further injustices to
marginalized and vulnerable communities. An economy-wide transition presents a
challenge unto itself due to the spatial, temporal, and topological complexities
of energy systems. \glspl{esom} are a class of
tools designed to optimize this transition used by energy planners and
decision-makers to generate insights that inform policy. However, \glspl{esom} 
are challenged by real-world 
optimization challenges, which require simultaneous consideration of multiple 
competing objectives.

Current \glspl{esom} exclusively optimize on cost, yet real world decisions are 
also informed by non-financial priorities, such as sustainability and social 
benefits. Since these objectives cannot be neatly captured by a cost metric, 
ESOMs fail to optimize for these goals.  Rather than returning a single optimal 
solution, multi-objective optimization generates a set of co-optimal solutions 
called a \textit{Pareto front}, where no objective can be improved without 
making another objective worse. The existence of a Pareto front evinces 
tradeoffs which can only be resolved through dialogue in a participatory 
process. 

A new first multi-objective energy system optimization framework, \texttt{Osier}
allows users to 
optimize arbitrarily many objectives and define new objectives to create a 
bespoke, contextualized, model. Since some structural 
uncertainty will persist regardless of the number of objectives, 
\texttt{Osier} leverages genetic algorithms for their ability to sample complex 
Pareto fronts and because their search methods automatically sample sub-optimal 
space. Further, \texttt{Osier} leverages a novel algorithm to calculate a subset of 
maximally different solutions within the sub-optimal space to address this 
uncertainty related to unmodeled objectives. By producing multiple solutions, 
\texttt{Osier} gives modelers and decision-makers the tools to meaningfully 
engage with public stakeholders and learn their preferences, thereby attending 
to issues of procedural and recognition justice.

This work verifies \texttt{Osier}'s suitability for energy modeling problems with two \textit{in silico} experiments. The first set of experiments
compare \texttt{Osier} to a more mature energy system optimization model, \texttt{Temoa}, to verify
that \texttt{Osier} produces results consistent with known methods. The results
for a least-cost optimization with \texttt{Osier} and \texttt{Temoa} show
strong agreement, with 0.5\% of each other. In the second set, \texttt{Osier}
reanalyzes a set of nuclear fuel cycle options through the lens of Pareto optimality.
