
\section{Introduction}
We used \Cyclus \cite{huff_fundamental_2016} to analyze
rt the \gls{EU} and to model the French transition. \Cyclus is an agent-based extensible
framework for modeling the flow of material through future nuclear cycles.
We calculate the used fuel
inventory in \gls{EU} member states, and propose a potential collaborative strategy of used fuel
management.
A major focus of this paper is to determine the potential for France to utilize
\gls{UNF} from other \gls{EU} nations to produce \gls{MOX} for the \glspl{SFR}.
The \gls{MOX} created will fuel French transition to a \gls{SFR} fleet
and allow France to avoid building additional \glspl{LWR}.

Past research focuses solely on France, and typically assumes that additional \glspl{LWR},
namely \glspl{EPR} supply the \gls{UNF} required to produce \gls{MOX} \cite{carre_overview_2009, martin_symbiotic_2017, freynet_multiobjective_2016}.
Other recent works implement partitioning and transmutation
in a regional (European) context, with \glspl{ADS} and Gen-IV reactors \cite{fazio_study_2013},
to reduce radiotoxicity for disposal.
Little recent work considers synergistic international spent fuel arrangements.
The present work finds that this collaborative strategy can reduce the
need to construct additional \glspl{LWR} in France.
