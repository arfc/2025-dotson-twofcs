\section{Introduction}
This paper uses CYCLUS, the agent-based simulator \cite{huff_fundamental_2016}, to analyze
the future nuclear inventory in the European Union. This paper focuses on different scenarios of France,
and how potential French legislation and policy can change the nuclear landscape of the European
Union. This paper begins with analysis of a once through fuel cycle, where all the spent nuclear fuel,
goes to a central repository after discharge from reactors. Next, a reprocessing fuel cycle is evaluated.
In this fuel cycle only French \gls{PWR}s accept \gls{MOX} fuel, where the spent \gls{UOX} from
all the member states is recycled in France. Finally, a EG01-EG23 transition scenario \cite{wigeland_nuclear_2014}
is modeled,
where a once-through-cycle transitions to a closed cycle with continuous reprocessing and \gls{SFR}s. 

\section{Methodology}
This mainly utilizes CYCLUS, an agent-based simulator, to simulate the nuclear fuel cycle
and track material flows in EU nations. An open-source database from \gls{IAEA} called
\gls{PRIS} is imported as a csv file, listing the country, reactor unit, type, net capacity (MWe), status,
operator, construction date, first criticality date, first grid date, commercial date, shutdown
date (if applicable), and unit capacity factor for 2013. Then only the EU countries are extracted
from the csv file. A python script is written up to generate a CYCLUS input file from the csv file,
which lists the individual reactor units as agents. Various plots are 
generated with matplotlib to describe the output data which includes mass of fuel used, final used nuclear fuel
inventory, and major isotope mass inventory over time.


\section{Time Scope and Future Projections}
Because of the eg01-eg23 transition scenario, the simulation should run to a distant
future to demonstrate substantial change. In this case, the year is set from
1960 to 2100. From 1960 to current, the existing data is used for the simulation.
For the near future (~2035), different projections and plans
made by each governments are taken
into account to estimate the direction of nuclear fleets in the EU (explained
further in later sections). For the far future (2035~2100), an equilibrium capacity
is set and decommissioned reactors are replaced by the exact
same reactors with the same
capacity. Of course in the case of the transition scenario the replaced French reactors
would be SFRs, not LWRs.

\subsection{Depletion Calculations}
Depletion calculations of the nuclear fuel are done
naively, with a model burnup depletion recipe (mass fraction) used
for each reactor type, and the mass of the fuel 
adjusted linearly with capacity. For example, a PWR of
1,000 MWe capacity has 193 assemblies of 3.2\% enriched
uranium fuel, with each assembly with a mass of 523.4 kg.
The core has a 18 month cycle, where one-third of the 
core (64 assemblies) are discharged per refueling. The refueling
is assumed to take 2 months to complete, during which the reactor
is shut down. The discharged
assemblies has a radioisotope inventory of a reference
depletion calculation from ORIGEN (see appendix).


\subsection{Scenario Descriptions}
The simulation follows the model fuel cycle, where a `source'
provides natural uranium, which is enriched by an 'enrichment'
facility to \gls{UOX}, while disposing enrichment waste (tailings)
to the 'sink' facility. The enriched \gls{UOX} is used
in the \gls{LWR}s and \gls{UOX} waste is produced. Then the waste
would either go directly to the sink (scenario 1), or is reprocessed
for further use (scenario 2 and 3). In scenario 2, the waste is 
separated in the separations facility (reprocessing plant), into
two streams, Plutonium and Uranium. The plutonium is mixed with depleted uranium from enrichment
to create \gls{MOX}, and the reprocessed uranium is stockpiled in the fabrication facility.
The produced \gls{MOX} is then used in French \gls{PWR}s.
In scenario 3, the waste goes through the same process but is mixed
to create \gls{SFR} fuel.

\subsection{Reprocessed Uranium}
Reprocessed uranium contains a range of uranium isotopes, from $^{232}U$ to $^{238}U$.
This brings complications in reusing reprocessed uranium as a fuel source \cite{IAEA_management_2007}.
The presence of neutron-absorbing isotopes $^{234}U$ and $^{236}U$ requires reprocessed uranium
to require a higher enrichment of $^{235}U$. There is trace amounts (2 ppb) of fissile istope $^{233}U$,
which provides little benefit.  
Also, $^{232}U$ has a decay chain of short-lived
daughter products that undergo intense beta and gamma radiation.
The French nuclear program utilizes a fraction (1/3) of reprocessed uranium as fuel \cite{IAEA_management_200&}.
However for this simulation the reprocessed uranium is simply stockpiled.


\section{Scenario Definition}
The following scenarios are considered: 
\begin{itemize}
	\item Halt reprocessing at 2020.
	\item \gls{MOX} reprocessing for French \gls{PWR}s.
	\item Continuous recycle with \gls{SFR}s.
\end{itemize}
