\section{Introduction}
This paper uses CYCLUS, the agent-based simulator \cite{huff_fundamental_2016} to analyze
the future nuclear inventory in the European Union. This paper focuses on the spent fuel
inventory in the EU nations in 2060, and analyzes various potential strategies of spent fuel
management from regional
repositories to France taking all the spent fuel.
A major focus of this paper is to see the implications of trade
relationships and repository partnerships between EU nations.
This paper looks into the  A CYCLUS simulation is run to see
how much spent fuel is created from different nations from 1970 to 2060, and then the various
options mentioned above are explored.  The paper assumes a once-through cycle for all the 
EU nations with the exception of France. France can reprocess spent \gls{UOX} and \gls{MOX} to
make \gls{MOX} from reprocessed plutonium and depleted uranium (tails). It is assumed that
\gls{MOX} is recycled infinitely. 


\section{Methodology}
This mainly utilizes CYCLUS, an agent-based simulator, to simulate the nuclear fuel cycle
and track material flows in EU nations. An open-source database from \gls{IAEA} called
\gls{PRIS} is imported as a csv file, listing the country, reactor unit, type, net capacity (MWe), status,
operator, construction date, first criticality date, first grid date, commercial date, shutdown
date (if applicable), and unit capacity factor for 2013. Then only the EU countries are extracted
from the csv file. A python script is written up to generate a CYCLUS input file from the csv file,
which lists the individual reactor units as agents. Various plots are 
generated with matplotlib to describe the output data which includes mass of fuel used, final used nuclear fuel
inventory, and major isotope mass inventory over time.


\section{Time Scope and Future Projections}
From the \gls{PRIS} data and other sources \cite{world_nuclear_2017} \cite{joskow_future_2012} \cite{hatch_politics_2013},
the future projections of each EU nation's plan for their nuclear program has been assessed.
The projections go up to 2050 at the latest. This justifies the simulation to take place from
1970 to 2060, the latest foreseeable future. The specific plans for each EU nation is explained
in detail in later sections.


\subsection{Depletion Calculations}
Depletion calculations of the nuclear fuel are done
naively, with a model burnup depletion recipe (mass fraction) used
for each reactor type, and the mass of the fuel 
adjusted linearly with capacity. For example, a PWR of
1,000 MWe capacity has 193 assemblies of 3.2\% enriched
uranium fuel, with each assembly with a mass of 523.4 kg.
The core has a 18 month cycle, where one-third of the 
core (64 assemblies) are discharged per refueling. The refueling
is assumed to take 2 months to complete, during which the reactor
is shut down. 

For the compositions of the fuel, a reference depletion calculation
from ORIGEN is used (see appendix). The recipe has also been used for
[[[[[[PAPER?????]]]]]]

\subsection{Scenario Descriptions}
The simulation follows the model fuel cycle, where a `source'
provides natural uranium, which is enriched by an 'enrichment'
facility to \gls{UOX}, while disposing enrichment waste (tailings)
to the 'sink' facility. The enriched \gls{UOX} is used
in the \gls{LWR}s and \gls{UOX} waste is produced. Then the spent fuel
goes directly to the sink. However, in the case of France, the spent fuel
is reprocessed to be fabricated into \gls{MOX}.


This paper explores two scenarios.
First is the case of having regional repositories. Taking into consideration 
the spent fuel inventory, repository research progress and geographical location,
regional repositories of reasonable capacity ($~70,000$ MTHM) are considered.
Second is to consider French-centralized repository, where France accepts all
spent fuel from EU nations. It is also analyzed how much \gls{MOX} can be produced
from the spent fuel received. In essence, the paper explores how taking fuel
from EU nations can have an economic incentive to France.



\subsection{Reprocessed Uranium}
Reprocessed uranium contains a range of uranium isotopes, from $^{232}U$ to $^{238}U$.
This brings complications in reusing reprocessed uranium as a fuel source \cite{IAEA_management_2007}.
The presence of neutron-absorbing isotopes $^{234}U$ and $^{236}U$ requires reprocessed uranium
to require a higher enrichment of $^{235}U$. There is trace amounts (2 ppb) of fissile istope $^{233}U$,
which provides little benefit.  
Also, $^{232}U$ has a decay chain of short-lived
daughter products that undergo intense beta and gamma radiation.
The French nuclear program utilizes a fraction (1/3) of reprocessed uranium as fuel \cite{IAEA_management_200&}.
However for this simulation the reprocessed uranium is simply stockpiled.


\section{Scenario Definition}
The following scenarios are considered: 
\begin{itemize}
	\item Halt reprocessing at 2020.
	\item \gls{MOX} reprocessing for French \gls{PWR}s.
	\item Continuous recycle with \gls{SFR}s.
\end{itemize}
